En este trabajo hemos desarrollado un modelo de optimización energética para
una vivienda unifamiliar en Madrid, alcanzando con éxito los objetivos
propuestos.

Implementamos el método de gradientes adjuntos discretos para obtener las
derivadas de manera eficiente en un sistema de ecuaciones diferenciales
algebraicas (DAE) que representa el comportamiento dinámico de la vivienda y
sus sistemas energéticos. Este enfoque ha permitido optimizar tanto el control
del sistema como el dimensionamiento de los equipos (bomba de calor, paneles
fotovoltaicos, baterías y tanque de agua) a lo largo de un año completo,
reduciendo significativamente el coste total de operación.

Además, utilizamos la técnica SAND (Simultaneous ANalysis and Design) para
resolver el problema, comparando su rendimiento con el enfoque basado en
gradientes adjuntos. Los resultados han demostrado que SAND es notablemente más
eficiente, reduciendo el tiempo computacional en aproximadamente un orden de
magnitud. Esta mejora en eficiencia es crucial dado el gran número de variables
de diseño involucradas, que en nuestro caso ascienden a cientos de miles.

Los escenarios de operación evaluados (autoconsumo con compensación
simplificada, conexión a la red con venta de excedentes, y operación off-grid)
nos han permitido analizar el comportamiento del sistema bajo diferentes
condiciones realistas. Los resultados obtenidos son aplicables no solo a
viviendas unifamiliares, sino también a otros casos como negocios o bloques de
viviendas.

Si bien no se entró en el modelado en detalle de la bomba de calor ni se
consideraron paneles solares térmicos, nuestro enfoque tiene el potencial de
ser ampliado en futuros trabajos para incluir estos elementos. Dichas
extensiones podrían proporcionar una visión más integral y detallada del
sistema energético, acercándonos aún más a una optimización completa y realista
para diferentes tipos de edificaciones.

En resumen, este trabajo muestra el poder de los métodos numéricos como los
gradientes adjuntos y SAND en la resolución de problemas de gran escala y
dimensionalidad.
