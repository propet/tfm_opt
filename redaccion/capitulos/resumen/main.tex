Se optimiza conjuntamente el control y dimensionamiento de un sistema
energético para una vivienda equipada con bomba de calor, paneles
fotovoltaicos, y almacenamiento en forma de baterías y tanque de agua.

La optimización basada en gradientes explora el uso de derivadas adjuntas para
el sistema de ecuaciones diferenciales algebraicas que describen el sistema
térmico. Además, se implementa la simulación y optimización conjunta mediante
el método SAND (Simultaneous ANalysis and Design), también conocido como "todo
a la vez" (all-at-once).

La demanda eléctrica se basa en un perfil de consumo real extraído de una
vivienda existente, mientras que el consumo de calor se estima para una
vivienda localizada en Madrid con calefacción de suelo radiante.

Para un año completo, se obtienen las señales de control de los equipos y los
tamaños óptimos de los componentes (potencia de la bomba de calor, potencia
instalada de paneles solares, potencia máxima contratada con la
comercializadora, capacidad de la batería y volumen del tanque de agua) que
minimizan el coste total, manteniendo simultáneamente la temperatura de la
vivienda estable y cercana a los 20°C.

Se analizan tres escenarios:

\begin{itemize}
	\item Vivienda bajo la modalidad de autoconsumo con compensación
	      simplificada, donde el término variable en la factura de la electricidad no
	      puede ser negativo, limitando así la compensación económica por el excedente de
	      electricidad volcada a la red.
	\item Vivienda conectada a la red, vendiendo excedentes de energía al precio de
	      mercado.
	\item Vivienda off-grid (desconectada de la red eléctrica).
\end{itemize}

\begin{flushleft} \textbf{Palabras clave:} optimización, derivadas adjuntas,
	diferenciación automática, control óptimo, eficiencia energética, pyoptsparse.
\end{flushleft}
