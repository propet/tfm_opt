Como vimos en la sección \ref{sec:optimization}, solvers de ecuaciones en
derivadas parciales normalmente no exponen el sistema que resuelven, se limitan
a devolver la solución y con suerte el cálculo de las derivadas adjuntas, por
algún método visto en la sección \ref{sec:adjoints}.

Si tuviéramos un modelo en forma residual, con variables de diseño $x$, y
estados $y$:

\begin{equation}
	r(x, y) = 0
\end{equation}

El problema de optimización típicamente planteado tiene la forma:

\begin{align}
	\min_{\mathbf{x}} \quad & f(\mathbf{x}, \text{devuelve\_y}(x)) \nonumber     \\
	\text{sujeto a} \quad   & h(\mathbf{x}, \text{devuelve\_y}(x)) = 0 \nonumber \\
	                        & g(\mathbf{x}, \text{devuelve\_y}(x)) < 0 \nonumber \\
\end{align}

$h$ son condiciones de igualdad, $g$ de desigualdad y la función
$\text{devuelve\_y}$ resuelve el sistema $r$ para encontrar los estados $y$.

Pero en otras circunstancias, como cuando resolvemos nosotros mismos las
ecuaciones, o realizamos el modelado de un equipo, podemos optar por aplicar
las ecuaciones del sistema como restricciones de igualdad en el problema de
optimización.

Entonces podemos plantear un problema como:

\begin{align}
	\min_{\mathbf{x}, \mathbf{y}} \quad & f(\mathbf{x}, \mathbf{y}) \nonumber     \\
	\text{sujeto a} \quad               & h(\mathbf{x}, \mathbf{y}) = 0 \nonumber \\
	                                    & g(\mathbf{x}, \mathbf{y}) < 0 \nonumber \\
	                                    & r(\mathbf{x}, \mathbf{y}) = 0 \nonumber \\
\end{align}

En comparación con cuando cuando usamos derivadas adjuntas, la cantidad de
variables de diseño y restricciones es mayor, pero a cambio el análisis del
sistema y su optimización se realizan de forma simultánea (SAND: Simultaneous
Analysis and Design).

En algunos casos como el que se explorará más adelante, se encuentra que la
ejecución del problema en esta forma es hasta un orden de magnitud más rápida
que con el uso de derivadas adjuntas.
