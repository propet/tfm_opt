\section{Optimización basada en gradientes con restricciones}

\subsection{Gradientes}

La optimización paramétrica sin el uso de gradientes no es escalable para
problemas con un gran número de variables de diseño, como ocurre en problemas
de control óptimo, optimización topológica, o aprendizaje automático.

Los algoritmos libres de gradiente buscan el óptimo poblando el espacio de
diseño y evaluando la función objetivo en estos puntos. Sin embargo, al
introducir más variables de diseño, el espacio crece exponencialmente, así como
el número de puntos a evaluar.

Asumiendo que la evaluación de cada punto tiene un coste computacional no
despreciable, el uso de estos algoritmos se vuelve inviable para problemas con
más de unas pocas decenas de variables de diseño. El volumen de un hipercubo
n-dimensional crece como:

\begin{equation}
	V = l^n
\end{equation}

donde $l$ es la longitud del lado y $n$ es la dimensión.

\begin{figure}[h] \centering
	% Insertar aquí la imagen del cubo en 1, 2 y 3 dimensiones
	\caption{Representación de un hipercubo en 1, 2 y 3 dimensiones}
	\label{fig:hypercube}
\end{figure}

Mientras que el espacio de diseño crece exponencialmente en volumen con el
incremento de las variables de diseño, la distancia entre dos puntos dentro
de este volumen no lo hace de manera tan drástica. Para un hipercubo
n-dimensional, su diagonal viene dada por:

\begin{equation}
	d = l\sqrt{n}
\end{equation}

Los algoritmos basados en gradientes, en lugar de poblar el espacio de
diseño, lo navegan avanzando hacia el óptimo por un camino determinado por
el gradiente de la función objetivo respecto a las variables de diseño.

\begin{figure}[h] \centering
	% Insertar aquí la imagen del camino dentro de un cubo 3-dimensional
	\caption{Camino de optimización en un espacio tridimensional}
	\label{fig:optimization_path}
\end{figure}

Esta estrategia permite la optimización de problemas con un gran número de
incógnitas. Por ejemplo, el modelo de lenguaje Llama 3.1-405 de Meta
utiliza 405 mil millones de variables \cite{dubey2024llama}.


% Reescribir
Y por qué funciona a pesar de que en principio introducir más variables, podría llevar
a la producción de muchos más mínimos locales?

En espacios con muchas dimensiones, se dan pocos mínimos locales porque la mayoría
de puntos son de inflexión, puntos de silla.
Para que se diese un mínimo local, la función debe de ser convexa en todas las variables
para ese punto, lo cual es cada vez menos probable con más dimensiones(variables).
% Reescribir

\begin{figure}[h] \centering
	% Insertar aquí la imagen de un punto de silla
	\caption{Punto de silla en 3 dimensiones}
	\label{fig:inflexion_point}
\end{figure}


\subsection{Restricciones}

La gran mayoría de los problemas en ingeniería requieren el uso de
restricciones en su formulación. Estas restricciones pueden ser de igualdad o
desigualdad y representan limitaciones físicas, económicas o de diseño que
deben satisfacerse durante el proceso de optimización.
