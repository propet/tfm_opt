Ahora queremos optimizar las dimensiones de los elementos que componen nuestra
instalación solar además de los controles para su funcionamiento. Esto es, qué
potencia de paneles solares debemos de instalar, qué capacidad para la batería,
potencia a contratar con la comercializadora, potencia máxima de la bomba de
calor y cómo de voluminoso ha de ser el depósito de agua caliente.

La formulación es similar a \eqref{eq:sand_control_optimization}, pero entonces
las dimensiones las tratábamos como parámetros fijos, y ahora serán variables
de diseño.

\begin{align}
	\min_{\mathbf{x}} \quad & \text{Coste\_Autoconsumo}(\mathbf{x}) \label{eq:regulated_sizing_optimization}                                                   \\
	\text{sujeto a} \quad   & \nonumber                                                                                                                        \\
	                        & P_{red_k} = -P_{solar_k} + P_{bomba_k} + P_{bat_k} + P_{carga_k} \quad                                           & \forall k > 0 \\
	                        & e_k = e_{k-1} + \eta_{bat} \cdot P_{bat_k} \cdot h \quad                                                         & \forall k > 0 \\
	                        & \text{cop}(T_{cond_k}) \cdot P_{bomba_k} \nonumber                                                                               \\
	                        & \quad - \dot{m}_{cond_k} \cdot cp_{agua} \cdot (T_{cond_k} - T_{tanque_k}) = 0                                   & \forall k > 0 \\
	                        & m_{tanque} \cdot cp_{agua} \cdot ( T_{tanque_k} - T_{tanque_{k-1}}) / h  \nonumber                                               \\
	                        & \quad - \dot{m}_{cond_k} \cdot cp_{agua} \cdot T_{cond_k} \nonumber                                                              \\
	                        & \quad - \dot{m}_{cale_k} \cdot cp_{agua} \cdot T_{cale_k} \nonumber                                                              \\
	                        & \quad + (\dot{m}_{cond_k} + \dot{m}_{cale_k}) \cdot cp_{agua} \cdot T_{tanque_k} \nonumber                                       \\
	                        & \quad + U_{tanque} \cdot A_{tanque} \cdot (T_{tanque_k} - T_{amb_k}) = 0  \label{eq:sizing_dae2}                 & \forall k > 0 \\
	                        & m_{suelo} \cdot cp_{suelo} \cdot ( T_{suelo_k} - T_{suelo_{k-1}}) / h \nonumber                                                  \\
	                        & \quad - \dot{m}_{cale_k} \cdot cp_{agua} \cdot (T_{tanque_k} - T_{cale_k})                             \nonumber                 \\
	                        & \quad + h_{suelo_k} \cdot A_{suelo} \cdot (T_{suelo_k} - T_{habitacion_k})                             \nonumber                 \\
	                        & \quad + \sigma \cdot \epsilon_{hormigon} \cdot A_{suelo} \cdot (T_{suelo_k}^4 - T_{habitacion_k}^4) = 0          & \forall k > 0 \\
	                        & m_{aire} \cdot cp_{aire} \cdot ( T_{habitacion_k} - T_{habitacion_{k-1}}) / h  \nonumber                                         \\
	                        & \quad - h_{suelo_k} \cdot A_{suelo} \cdot (T_{suelo_k} - T_{habitacion_k})  \nonumber                                            \\
	                        & \quad - \sigma \cdot \epsilon_{hormigon} \cdot A_{suelo} \cdot (T_{suelo_k}^4 - T_{habitacion_k}^4)  \nonumber                   \\
	                        & \quad + U_{paredes} \cdot A_{paredes} \cdot (T_{habitacion_k} - T_{amb_k}) \nonumber                                             \\
	                        & \quad + U_{techo} \cdot A_{techo} \cdot (T_{habitacion_k} - T_{amb_k}) \nonumber                                                 \\
	                        & \quad + U_{ventanas} \cdot A_{ventanas} \cdot (T_{habitacion_k} - T_{amb_k}) = 0                                 & \forall k > 0
\end{align}

con las mismas condiciones iniciales

\begin{align}
	 & e_0 = \text{SOC}_{min} \cdot e_{max} \\
	 & T_{cond_0} = 296.56 [K]              \\
	 & T_{tanque_0} = 296.05[K]             \\
	 & T_{suelo_0} = 295.27[K]              \\
	 & T_{habitacion_0} = 293.47[K]         \\
\end{align}

y límites

\begin{align}
	 & T_{objetivo} \leq T_{habitacion_k} \leq 500[K]                                    & \forall k \\
	 & 273[K] \leq T_{cond_k}, T_{tanque_k}, T_{suelo_k} \leq 500[K]                     & \forall k \\
	 & 0 \leq P_{bomba_k} \leq P_{bomba_{max}} \quad                                     & \forall k \\
	 & \text{SOC}_{min} \cdot e_{max} \leq e_k \leq \text{SOC}_{max} \cdot e_{max} \quad & \forall k \\
	 & -P_{bat_{max}} \leq P_{bat_k} \leq P_{bat_{max}} \quad                            & \forall k \\
	 & -P_{red_{max}} \leq P_{red_k} \leq P_{red_{max}} \quad                            & \forall k \\
	 & 0 \leq \dot{m}_{cond_k} \leq \dot{m}_{cond_{max}} \quad                           & \forall k \\
	 & 0 \leq \dot{m}_{cale_k} \leq \dot{m}_{cale_{max}} \quad                           & \forall k
\end{align}

donde $P_{bomba_{max}}$, $e_{max}$, $P_{red_{max}}$ son variables de diseño
pertenecientes al vector $\mathbf{x}$, y el volumen del tanque de agua no
aparece explícitamente, pero en la condición \eqref{eq:sizing_dae2} se emplea
para determinar la masa y área del depósito, como se hizo en la relación
\eqref{eq:tank_mass_and_area}.

Y parámetros que serán comunes a todos los casos se muestran en la tabla
\ref{tab:sizing_optimization_data}.

\begin{table}[ht]
	\centering
	\caption{Parámetros para problemas de dimensionamiento.}
	\label{tab:sizing_optimization_data}
	\begin{tabular}{@{}lll@{}}
		\toprule
		\textbf{Parámetro}     & \textbf{Valor}         & \textbf{Descripción}                       \\
		\midrule
		$h$                    & $1000[s]$              & Tamaño de paso                             \\
		$T_{objetivo}$         & $20 ^\circ C$          & Temperatura objetivo para la habitación    \\
		$\text{SOC}_{min}$     & $0.3$                  & min State Of Charge                        \\
		$\text{SOC}_{max}$     & $0.9$                  & max State Of Charge                        \\
		$\eta_{bat}$           & $0.95$                 & Eficiencia de carga/descarga de la batería \\
		$\dot{m}_{cond_{max}}$ & $0.5[kg \cdot s^{-1}]$ & Máximo caudal a bomba de calor             \\
		$\dot{m}_{cale_{max}}$ & $0.5[kg \cdot s^{-1}]$ & Máximo caudal a suelo radiante             \\
		\bottomrule
	\end{tabular}
\end{table}


\subsection{Autoconsumo con compensación}

Exactamente la formulación que acabamos de ver en
\eqref{eq:regulated_sizing_optimization}.

En caso de que la comercializadora eléctrica nos ofrezca el servicio de batería
virtual, podremos mantener un saldo positivo mensual con la compañía. Pero a
final de año deseamos tener esta cifra a 0, ya que no es posible recuperarlo en
efectivo. Asumimos que el rendimiento de la instalación será constante de un
año a otro; por lo tanto, si este año hemos tenido un saldo positivo, también
lo tendremos el próximo año.

Por tanto, la optimización para el año completo (2022), se muestran en la
figura \ref{fig:regulated_year} y cuadro \ref{tab:regulated_year}.

\begin{figure}[h] \centering
	\centering
	\includesvg[width=0.92\textwidth]{./capitulos/resultados_discusion/images/control_sand_30_days}
	\caption{Control óptimo para los 30 primeros días del año 2022, método 'all-at-once'.}
	\label{fig:regulated_year}
\end{figure}

\begin{table}[ht]
	\centering
	\caption{Datos para ejecución mostrada en la figura \ref{fig:regulated_year}}
	\label{tab:regulated_year}
	\begin{tabular}{@{}lcc@{}}
		\toprule
		Parámetro                                                 & Valor \\
		\midrule
		Número de iteraciones realizadas                          & 300   \\
		Tiempo total (segundos)                                   & 396   \\
		Número de pasos                                           & 2592  \\
		Número de variables de diseño                             & 23328 \\
		Número de elementos en matrices dispersas (Igualdades)    & 54416 \\
		Número de elementos en matrices dispersas (Desigualdades) & 7774  \\
		\midrule
		Valor de la función objetivo                              & 85.92 \\
		\midrule
		Coste fijo de energía                                     & 16.06 \\
		Coste variable de energía                                 & 43.17 \\
		Coste de depreciación de batería                          & 2.48  \\
		Coste de depreciación de paneles solares                  & 14.31 \\
		Coste de depreciación de bomba de calor                   & 5.08  \\
		Coste de depreciación de depósito de agua                 & 4.87  \\
		\bottomrule
	\end{tabular}
\end{table}


\clearpage
\subsection{Autoconsumo con compensación ilimitada}

\clearpage
\subsection{Sistema off-grid}
