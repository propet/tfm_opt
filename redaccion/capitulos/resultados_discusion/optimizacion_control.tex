Aquí integramos los grupos eléctrico y térmico para minimizar el coste de
operación de la instalación bajo una tarifa regulada de compensación de
excedentes con energía solar, con función objetivo \eqref{eq:cost_regulated}.

Y planteamos el problema de dos formas diferentes. Primero obteniendo los
gradientes del sistema térmico con el método de gradientes adjuntos discretos,
y después con un método 'all-at-once', donde introducimos las ecuaciones
térmicas como restricciones al problema de optimización.

Hemos empleado el framework de optimización pyOptSparse \cite{Wu2020}, que
permite el uso de matrices dispersas y es mayormente agnóstico frente al uso de
un optimizador u otro, aunque se ha empleado principalmente el optimizador
Ipopt \cite{wachter2006implementation}, algoritmo open-source de punto
interior adecuado para problemas no-lineales de gran escala.

\subsection{Derivadas adjuntas}

El problema de optimización a resolver es:

\begin{align}
	\min_{\mathbf{x}} \quad & \text{Coste\_Total}(\mathbf{x})                                                                                                            \\
	\text{sujeto a} \quad   &                                                                                                                                            \\
	                        & P_{red_k} = -P_{solar_k} + P_{bomba_k} + P_{bat_k} + P_{carga_k} \quad                                                     & \forall k     \\
	                        & e_k = e_{k-1} + \eta_{bat} \cdot P_{bat_k} \cdot h \quad                                                                   & \forall k > 0 \\
	                        & min(T_{habitacion}) \geq T_{objetivo}                                               \label{eq:min_t_habitacion_constraint}                 \\
	\text{condiciones iniciales} \quad                                                                                                                                   \\
	                        & P_{bat_0} = 0                                                                                                                              \\
	                        & e_0 = \text{SOC}_{min} \cdot e_{max}                                                                                                       \\
	\text{límites} \quad    &                                                                                                                                            \\
	                        & 0 \leq P_{bomba_k} \leq P_{bomba_{max}} \quad                                                                              & \forall k     \\
	                        & \text{SOC}_{min} \cdot e_{max} \leq e_k \leq \text{SOC}_{max} \cdot e_{max}, \quad                                         & \forall k     \\
	                        & -P_{bat_{max}} \leq P_{bat_k} \leq P_{bat_{max}} \quad                                                                     & \forall k     \\
	                        & -P_{red_{max}} \leq P_{red_k} \leq P_{red_{max}} \quad                                                                     & \forall k     \\
	                        & 0 \leq \dot{m}_{cond_k} \leq \dot{m}_{cond_{max}} \quad                                                                    & \forall k     \\
	                        & 0 \leq \dot{m}_{cale_k} \leq \dot{m}_{cale_{max}} \quad                                                                    & \forall k
\end{align}


donde $\mathbf{x}$ es el vector de variables de diseño: $P_{bomba}$, $P_{bat}$,
$\dot{m}_{cond}$ y $\dot{m}_{cale}$. Cada una de ellas con tantos elementos
como pasos en la simulación, siendo k el índice, o número de paso.

Las dos primeras restricciones representan el balance de potencias y dinámica
de la batería a cada paso, respectivamente, mientras que la tercera
\eqref{eq:min_t_habitacion_constraint} es una condición sobre el sistema
térmico.

Para obtener la temperatura mínima de la habitación a lo largo de la evolución
de este sistema, tenemos que resolver implícitamente el sistema de ecuaciones
algebraicas diferenciales de la sección \ref{subsec:reduced_system}.

$min(T_{habitacion})$ es realmente un valor que disponemos a través de la
solución del sistema térmico, y una función que agrega los estados y controles
en un único resultado, en este caso es el valor mínimo de todos los
$T_{habitacion_k}$.

\begin{minted}{python}
def j_t_room_min(y, u, p, h):
    """
    y: State variables (t_cond, t_tank, t_floor, t_room)
    u: Control variables (m_dot_cond, m_dot_heating, p_compressor, t_amb)
    p: Parameters
    h: Step size

    Returns:
    float: The minimum room temperature.
    """
    t_room = y[3]
    return jnp.min(t_room)


y = solve(y0, u, dae_p, h, n_steps)  # solve DAE system
t_room_min = j_t_room_min(y, u, p, h)  # aggregation function
\end{minted}


y ya que la función 'j\_t\_room\_min' que nos devuelve el valor de la
$min(T_{habitacion})$ dependiendo de las variables de diseño $P_{bomba}$,
$\dot{m}_{cond}$ y $\dot{m}_{cale}$, es clave que podamos calcular los
gradientes de este valor respecto de las variables de diseño. Necesitamos las
derivadas:

\begin{equation*}
	\frac{d \min(T_{habitacion})}{d P_{bomba_k}}, \quad \frac{d \min(T_{habitacion})}{d \dot{m}_{cond_k}}, \quad \frac{d \min(T_{habitacion})}{d \dot{m}_{cale_k}}
\end{equation*}

y de forma eficiente podemos averiguarlas usando el método de los gradientes
adjuntos discretos, explicado en la sección \ref{sec:adjoints}.

Tenemos una primera evaluación de esta función en la que aprovechamos a guardar
los estados $y$ solución

\begin{minted}{python}
def t_room_min_fun(
    y0_arr,
    m_dot_cond,
    m_dot_heating,
    p_compressor,
    t_amb,
    dae_p,
    h,
    n_steps,
):
    # t_target < t_room_min
    # where t_room_min is result of solving the thermal system DAE
    u = np.zeros((4, n_steps))
    u[0, :] = m_dot_cond
    u[1, :] = m_dot_heating
    u[2, :] = p_compressor
    u[3, :] = t_amb
    y = dae_forward(y0_arr, u, dae_p, h, n_steps)
    parameters["dae_last_forward_solution"] = y  # save solution for the backward pass
    t_room_min = jnp.min(y[3])
    return np.array(t_room_min)
\end{minted}

y a la hora de calcular las derivadas, rescatamos este vector $y$ y lo
proporcionamos para de forma análoga a la diferenciación automática en modo
inverso, propagar los gradientes desde la salida hasta el inicio.

\begin{minted}{python}
y = parameters["dae_last_forward_solution"]  # get solution from last forward pass
dj_dy0, dj_dp, dj_du = dae_adjoints(
    y,
    u,
    dae_p,
    h,
    n_steps,
    dae_system,
    j_t_room_min,
    j_extra_args=(),
)
\end{minted}

De forma general obtenemos los gradientes respecto de todas la variables de
control $u$, los parametros $p$, y las condiciones iniciales $y0$, aunque aquí
solo estamos interesados en 'dj\_du', los gradientes de la función agregación
que nos devuelve la temperatura mínima de la habitación a lo largo de la
trayectoria respecto de las variables de control, que se corresponden con
nuestras variables de diseño.


En la figura \ref{fig:control_adjoints_7_days} vemos la optimización realizada
para un periodo de 7 días

\begin{figure}[h] \centering
	\centering
	\includesvg[width=1\textwidth]{./capitulos/resultados_discusion/images/control_adjoints_7_days}
	\caption{Sistema rígido.}
	\label{fig:control_adjoints_7_days}
\end{figure}



\clearpage
\subsection{SAND}

El problema de optimización a resolver es:

\begin{align}
	\min_{\mathbf{x}} \quad & \text{Coste\_Total}(\mathbf{x})                                                                                                                \\
	\text{sujeto a} \quad   &                                                                                                                                                \\
	                        & P_{red_k} = -P_{solar_k} + P_{bomba_k} + P_{bat_k} + P_{carga_k} \quad                                                         & \forall k     \\
	                        & e_k = e_{k-1} + \eta_{bat} \cdot P_{bat_k} \cdot h \quad                                                                       & \forall k > 0 \\
	                        & \text{cop}(T_{cond_k}) \cdot P_{bomba_k} \nonumber                                                                                             \\
	                        & \quad - \dot{m}_{cond_k} \cdot cp_{agua} \cdot (T_{cond_k} - T_{tanque_k}) = 0 \label{eq:sys_1_sand}                           & \forall k > 0 \\
	                        & m_{tanque} \cdot cp_{agua} \cdot ( T_{tanque_k} - T_{tanque_{k-1}}) / h  \nonumber                                                             \\
	                        & \quad - \dot{m}_{cond_k} \cdot cp_{agua} \cdot T_{cond_k} \nonumber                                                                            \\
	                        & \quad - \dot{m}_{cale_k} \cdot cp_{agua} \cdot T_{cale_k} \nonumber                                                                            \\
	                        & \quad + (\dot{m}_{cond_k} + \dot{m}_{cale_k}) \cdot cp_{agua} \cdot T_{tanque_k} \nonumber                                                     \\
	                        & \quad + U_{tanque} \cdot A_{tanque} \cdot (T_{tanque_k} - T_{amb_k}) = 0 \label{eq:sys_2_sand}                                 & \forall k > 0 \\
	                        & m_{suelo} \cdot cp_{suelo} \cdot ( T_{suelo_k} - T_{suelo_{k-1}}) / h \nonumber                                                                \\
	                        & \quad - \dot{m}_{cale_k} \cdot cp_{agua} \cdot (T_{tanque_k} - T_{cale_k})                             \nonumber                               \\
	                        & \quad + h_{suelo_k} \cdot A_{suelo} \cdot (T_{suelo_k} - T_{habitacion_k})                             \nonumber                               \\
	                        & \quad + \sigma \cdot \epsilon_{hormigon} \cdot A_{suelo} \cdot (T_{suelo_k}^4 - T_{habitacion_k}^4) = 0  \label{eq:sys_3_sand} & \forall k > 0 \\
	                        & m_{aire} \cdot cp_{aire} \cdot ( T_{habitacion_k} - T_{habitacion_{k-1}}) / h  \nonumber                                                       \\
	                        & \quad - h_{suelo_k} \cdot A_{suelo} \cdot (T_{suelo_k} - T_{habitacion_k})  \nonumber                                                          \\
	                        & \quad - \sigma \cdot \epsilon_{hormigon} \cdot A_{suelo} \cdot (T_{suelo_k}^4 - T_{habitacion_k}^4)  \nonumber                                 \\
	                        & \quad + U_{paredes} \cdot A_{paredes} \cdot (T_{habitacion_k} - T_{amb_k}) \nonumber                                                           \\
	                        & \quad + U_{techo} \cdot A_{techo} \cdot (T_{habitacion_k} - T_{amb_k}) \nonumber                                                               \\
	                        & \quad + U_{ventanas} \cdot A_{ventanas} \cdot (T_{habitacion_k} - T_{amb_k}) = 0  \label{eq:sys_4_sand}                        & \forall k > 0
\end{align}

con los límites:

\begin{align}
	 & T_{objetivo} \leq T_{habitacion_k} \leq 500[K]  \label{eq:min_t_habitacion_constraint_sand} & \forall k \\
	 & 273[K] \leq T_{cond_k}, T_{tanque_k}, T_{suelo_k} \leq 500[K]                               & \forall k \\
	 & 0 \leq P_{bomba_k} \leq P_{bomba_{max}} \quad                                               & \forall k \\
	 & \text{SOC}_{min} \cdot e_{max} \leq e_k \leq \text{SOC}_{max} \cdot e_{max}, \quad          & \forall k \\
	 & -P_{bat_{max}} \leq P_{bat_k} \leq P_{bat_{max}} \quad                                      & \forall k \\
	 & -P_{red_{max}} \leq P_{red_k} \leq P_{red_{max}} \quad                                      & \forall k \\
	 & 0 \leq \dot{m}_{cond_k} \leq \dot{m}_{cond_{max}} \quad                                     & \forall k \\
	 & 0 \leq \dot{m}_{cale_k} \leq \dot{m}_{cale_{max}} \quad                                     & \forall k
\end{align}

y teniendo como condiciones iniciales para las variables de estado:

\begin{align}
	 & e_0 = \text{SOC}_{min} \cdot e_{max} \\
	 & T_{cond_0} = 296.56 [K]              \\
	 & T_{tanque_0} = 296.05[K]             \\
	 & T_{suelo_0} = 295.27[K]              \\
	 & T_{habitacion_0} = 293.47[K]         \\
\end{align}
