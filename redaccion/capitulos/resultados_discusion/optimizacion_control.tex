Aquí integramos los grupos eléctrico y térmico para minimizar el coste de
operación de la instalación bajo una tarifa regulada de compensación de
excedentes con energía solar, con función objetivo \eqref{eq:cost_regulated}.

Y planteamos el problema de dos formas diferentes. Primero obteniendo los
gradientes del sistema térmico con el método de gradientes adjuntos discretos,
y después con un método 'all-at-once', donde introducimos las ecuaciones
térmicas como restricciones al problema de optimización.

Hemos empleado el framework de optimización pyOptSparse \cite{Wu2020}, que
permite el uso de matrices dispersas y es mayormente agnóstico frente al uso de
un optimizador u otro, aunque se ha empleado principalmente Ipopt
\cite{wachter2006implementation}, algoritmo open-source de punto interior
adecuado para problemas no-lineales de gran escala.

De momento optimizamos únicamente el control, los valores para las dimensiones
de los equipos por tanto son fijos, y en la tabla \ref{tab:control_data}
mostramos algunos datos de interés común para todas las ejecuciones de esta
sección, donde la capacidad, eficiencia y potencia máxima nominal de la batería
se ha copiado de un modelo de tesla powerwall
\footnote{\url{https://www.tesla.com/sites/default/files/pdfs/powerwall/Powerwall\%202\_AC\_Datasheet\_en\_northamerica.pdf}}
.

\begin{table}[ht]
	\centering
	\label{tab:control_data}
	\caption{Parámetros para problemas de control.}
	\begin{tabular}{@{}lll@{}}
		\toprule
		\textbf{Parámetro}     & \textbf{Valor}         & \textbf{Descripción}                       \\
		\midrule
		$h$                    & $1000[s]$              & Tamaño de paso                             \\
		$T_{objetivo}$         & $20 ^\circ C$          & Temperatura objetivo para la habitación    \\
		$e_{max}$              & $13[kWh]$              & Capacidad de la batería                    \\
		$P_{bat_{max}}$        & $5[kW]$                & Máxima potencia nominal de la batería      \\
		$\text{SOC}_{min}$     & $0.3$                  & min State Of Charge                        \\
		$\text{SOC}_{max}$     & $0.9$                  & max State Of Charge                        \\
		$P_{red_{max}}$        & $5[kW]$                & Potencia máxima contratada                 \\
		$\eta_{bat}$           & $0.95$                 & Eficiencia de carga/descarga de la batería \\
		$\dot{m}_{cond_{max}}$ & $0.5[kg \cdot s^{-1}]$ & Máximo caudal a bomba de calor             \\
		$\dot{m}_{cale_{max}}$ & $0.5[kg \cdot s^{-1}]$ & Máximo caudal a suelo radiante             \\
		\bottomrule
	\end{tabular}
\end{table}


\subsection{Derivadas adjuntas}

El problema de optimización a resolver es:

\begin{align}
	\min_{\mathbf{x}} \quad & \text{Coste\_Total}(\mathbf{x})                                                                                                            \\
	\text{sujeto a} \quad   & \nonumber                                                                                                                                  \\
	                        & P_{red_k} = -P_{solar_k} + P_{bomba_k} + P_{bat_k} + P_{carga_k} \quad                                                     & \forall k > 0 \\
	                        & e_k = e_{k-1} + \eta_{bat} \cdot P_{bat_k} \cdot h \quad                                                                   & \forall k > 0 \\
	                        & min(T_{habitacion}) \geq T_{objetivo}                                               \label{eq:min_t_habitacion_constraint}                 \\
	\text{condiciones iniciales} \quad \nonumber                                                                                                                         \\
	                        & e_0 = \text{SOC}_{min} \cdot e_{max}                                                                                                       \\
	\text{límites} \quad    & \nonumber                                                                                                                                  \\
	                        & 0 \leq P_{bomba_k} \leq P_{bomba_{max}} \quad                                                                              & \forall k     \\
	                        & \text{SOC}_{min} \cdot e_{max} \leq e_k \leq \text{SOC}_{max} \cdot e_{max} \quad                                          & \forall k     \\
	                        & -P_{bat_{max}} \leq P_{bat_k} \leq P_{bat_{max}} \quad                                                                     & \forall k     \\
	                        & -P_{red_{max}} \leq P_{red_k} \leq P_{red_{max}} \quad                                                                     & \forall k     \\
	                        & 0 \leq \dot{m}_{cond_k} \leq \dot{m}_{cond_{max}} \quad                                                                    & \forall k     \\
	                        & 0 \leq \dot{m}_{cale_k} \leq \dot{m}_{cale_{max}} \quad                                                                    & \forall k
\end{align}


donde $\mathbf{x}$ es el vector de variables de diseño: $P_{bomba}$, $P_{bat}$,
$\dot{m}_{cond}$ y $\dot{m}_{cale}$. Cada una de ellas con tantos elementos
como pasos en la simulación, siendo k el índice, o número de paso.

Las dos primeras restricciones representan el balance de potencias y dinámica
de la batería a cada paso, respectivamente, mientras que la tercera
\eqref{eq:min_t_habitacion_constraint} es una condición agregada sobre el
sistema térmico. Agregada porque podríamos haber creado una restricción para
cada $T_{habitacion_k}$, pero como por cada condición debemos de calcular sus
gradientes, y esto implica resolver el sistema de ecuaciones algebraicas
diferenciales de la sección \ref{subsec:reduced_system}, preferimos englobar
nuestro requisito

\begin{equation*}
	T_{habitacion_k} > T_{objetivo}  \quad \forall k
\end{equation*}

en una sola inigualdad

\begin{equation*}
	min(T_{habitacion}) > T_{objetivo}
\end{equation*}


$min(T_{habitacion})$ es un escalar que disponemos a través de la siguiente
función de agregación

\begin{minted}{python}
import jax.numpy as jnp

def j_t_room_min(y, u, p, h):
    """
    y: State variables (t_cond, t_tank, t_floor, t_room)
    u: Control variables (m_dot_cond, m_dot_heating, p_compressor, t_amb)
    p: Parameters
    h: Step size

    Returns:
    float: The minimum room temperature.
    """
    t_room = y[3]
    return jnp.min(t_room)


y = solve(y0, u, dae_p, h, n_steps)  # solve DAE system
t_room_min = j_t_room_min(y, u, p, h)  # aggregation function
\end{minted}


y ya que 'j\_t\_room\_min' nos devuelve el valor de la $min(T_{habitacion})$
dependiendo de las variables de diseño $P_{bomba}$, $\dot{m}_{cond}$ y
$\dot{m}_{cale}$, necesitamos los gradientes de este valor respecto de las
variables de diseño:

\begin{equation*}
	\frac{d \min(T_{habitacion})}{d P_{bomba_k}}, \quad \frac{d \min(T_{habitacion})}{d \dot{m}_{cond_k}}, \quad \frac{d \min(T_{habitacion})}{d \dot{m}_{cale_k}}
\end{equation*}

de forma eficiente podemos averiguarlas usando el método de los gradientes
adjuntos discretos, explicado en la sección \ref{sec:adjoints}.

Tenemos una primera evaluación de esta función en la que aprovechamos a guardar
los estados solución $y$

\begin{minted}{python}
def t_room_min_fun(
    y0_arr,
    m_dot_cond,
    m_dot_heating,
    p_compressor,
    t_amb,
    dae_p,
    h,
    n_steps,
):
    # t_target < t_room_min
    # where t_room_min is result of solving the thermal system DAE
    u = np.zeros((4, n_steps))
    u[0, :] = m_dot_cond
    u[1, :] = m_dot_heating
    u[2, :] = p_compressor
    u[3, :] = t_amb
    y = dae_forward(y0_arr, u, dae_p, h, n_steps)
    parameters["dae_last_forward_solution"] = y  # save solution for the backward pass
    t_room_min = jnp.min(y[3])
    return np.array(t_room_min)
\end{minted}

y a la hora de calcular las derivadas, rescatamos este vector $y$ y lo
proporcionamos para de forma análoga a la diferenciación automática en modo
inverso, propagar los gradientes desde la salida hasta el inicio.

\begin{minted}{python}
y = parameters["dae_last_forward_solution"]  # get solution from last forward pass
dj_dy0, dj_dp, dj_du = dae_adjoints(
    y,
    u,
    dae_p,
    h,
    n_steps,
    dae_system,
    j_t_room_min,
    j_extra_args=(),
)
\end{minted}

De forma general obtenemos los gradientes respecto de todas la variables de
control $u$, los parámetros $p$, y las condiciones iniciales $y0$, aunque aquí
solo estamos interesados en 'dj\_du', los gradientes de la función agregación
que nos devuelve la temperatura mínima de la habitación a lo largo de la
trayectoria respecto de las variables de control, que se corresponden con
nuestras variables de diseño.


Por otra parte, también notamos que las restricciones se aplican solo para
$k > 0$, mientras que tenemos variables de diseño $\forall k$. Esto es debido
a que al discretizar con euler implícito, las variables de control en el
punto inicial no afectan las dinámicas, quedan sueltas.

Véase en un problema de valor inicial con euler-implícito, que $u_0$ no aparece
en ese sistema:

\begin{align*}
	y_0 & = 0                                 \\
	y_k & = y_{k-1} + h f(t_k, u_k, y_k, p_k)
\end{align*}

pero en nuestra formulación sí tenemos presente $u_0$, que serían las variables
de diseño o controles en el paso $k=0$.

Lo que hacemos es obviar las restricciones para el paso inicial, y eliminar la
aportación de los controles $u_0$ en la función objetivo.

Otra posibilidad sería reducir el tamaño de los vectores con las variables de
control en un elemento: $u_{1 \ldots N}$ en vez de $u_{0 \ldots N}$. Pero ello
complicaría la formulación mezclando índices distintos para variables de estado
y variables de control en las ecuaciones.


\subsubsection{Soluciones}

En la figura \ref{fig:control_adjoints_7_days} vemos la optimización realizada
para un periodo de 7 días, y en el cuadro \ref{tab:control_adjoints_7_days}
datos de la ejecución.

\begin{figure}[h] \centering
	\centering
	\includesvg[width=0.92\textwidth]{./capitulos/resultados_discusion/images/control_adjoints_7_days}
	\caption{Control óptimo para los 7 primeros días del año 2022, gradientes adjuntos.}
	\label{fig:control_adjoints_7_days}
\end{figure}

\begin{table}[ht]
	\centering
	\caption{Datos para ejecución mostrada en la figura \ref{fig:control_adjoints_7_days}}
	\label{tab:control_adjoints_7_days}
	\begin{tabular}{@{}lcc@{}}
		\toprule
		Parámetro                                 & Valor \\
		\midrule
		Número de iteraciones realizadas          & 300   \\
		Tiempo total (segundos)                   & 424   \\
		Número de pasos                           & 605   \\
		Número de variables de diseño             & 3025  \\
		\midrule
		Valor de la función objetivo (€)          & 17.67 \\
		\midrule
		Coste fijo de energía                     & 3.75  \\
		Coste variable de energía                 & 7.52  \\
		Coste de depreciación de batería          & 0.94  \\
		Coste de depreciación de paneles solares  & 3.34  \\
		Coste de depreciación de bomba de calor   & 1.02  \\
		Coste de depreciación de depósito de agua & 1.13  \\
		\bottomrule
	\end{tabular}
\end{table}

Y usando un periodo de 30 días en la figura \ref{fig:control_adjoints_30_days}
vemos una ejecución no exitosa, que no ha conseguido satisfacer las
resticciones con 300 iteraciones, más visiblemente la de temperatura mínima
para la vivienda. La información para esta operación se muestra en el cuadro
\ref{tab:control_adjoints_30_days}.

\begin{figure}[h] \centering
	\centering
	\includesvg[width=0.92\textwidth]{./capitulos/resultados_discusion/images/control_adjoints_30_days}
	\caption{Intento de control óptimo para 30 días con gradientes adjuntos.}
	\label{fig:control_adjoints_30_days}
\end{figure}

\begin{table}[ht]
	\centering
	\caption{Datos para ejecución mostrada en la figura \ref{fig:control_adjoints_30_days}}
	\label{tab:control_adjoints_30_days}
	\begin{tabular}{@{}lcc@{}}
		\toprule
		Parámetro                                 & Valor \\
		\midrule
		Número de iteraciones realizadas          & 300   \\
		Tiempo total (segundos)                   & 1684  \\
		Número de pasos                           & 2592  \\
		Número de variables de diseño             & 12960 \\
		\midrule
		Valor de la función objetivo (€)          & 90.72 \\
		\midrule
		Coste fijo de energía                     & 16.06 \\
		Coste variable de energía                 & 45.24 \\
		Coste de depreciación de batería          & 5.76  \\
		Coste de depreciación de paneles solares  & 14.31 \\
		Coste de depreciación de bomba de calor   & 4.56  \\
		Coste de depreciación de depósito de agua & 4.87  \\
		\bottomrule
	\end{tabular}
\end{table}





\clearpage
\subsection{SAND}

El problema de optimización a resolver es:

\begin{align}
	\min_{\mathbf{x}} \quad & \text{Coste\_Total}(\mathbf{x}) \label{eq:sand_control_optimization}                                                                           \\
	\text{sujeto a} \quad   & \nonumber                                                                                                                                      \\
	                        & P_{red_k} = -P_{solar_k} + P_{bomba_k} + P_{bat_k} + P_{carga_k} \quad                                                         & \forall k > 0 \\
	                        & e_k = e_{k-1} + \eta_{bat} \cdot P_{bat_k} \cdot h \quad                                                                       & \forall k > 0 \\
	                        & \text{cop}(T_{cond_k}) \cdot P_{bomba_k} \nonumber                                                                                             \\
	                        & \quad - \dot{m}_{cond_k} \cdot cp_{agua} \cdot (T_{cond_k} - T_{tanque_k}) = 0 \label{eq:sys_1_sand}                           & \forall k > 0 \\
	                        & m_{tanque} \cdot cp_{agua} \cdot ( T_{tanque_k} - T_{tanque_{k-1}}) / h  \nonumber                                                             \\
	                        & \quad - \dot{m}_{cond_k} \cdot cp_{agua} \cdot T_{cond_k} \nonumber                                                                            \\
	                        & \quad - \dot{m}_{cale_k} \cdot cp_{agua} \cdot T_{cale_k} \nonumber                                                                            \\
	                        & \quad + (\dot{m}_{cond_k} + \dot{m}_{cale_k}) \cdot cp_{agua} \cdot T_{tanque_k} \nonumber                                                     \\
	                        & \quad + U_{tanque} \cdot A_{tanque} \cdot (T_{tanque_k} - T_{amb_k}) = 0 \label{eq:sys_2_sand}                                 & \forall k > 0 \\
	                        & m_{suelo} \cdot cp_{suelo} \cdot ( T_{suelo_k} - T_{suelo_{k-1}}) / h \nonumber                                                                \\
	                        & \quad - \dot{m}_{cale_k} \cdot cp_{agua} \cdot (T_{tanque_k} - T_{cale_k})                             \nonumber                               \\
	                        & \quad + h_{suelo_k} \cdot A_{suelo} \cdot (T_{suelo_k} - T_{habitacion_k})                             \nonumber                               \\
	                        & \quad + \sigma \cdot \epsilon_{hormigon} \cdot A_{suelo} \cdot (T_{suelo_k}^4 - T_{habitacion_k}^4) = 0  \label{eq:sys_3_sand} & \forall k > 0 \\
	                        & m_{aire} \cdot cp_{aire} \cdot ( T_{habitacion_k} - T_{habitacion_{k-1}}) / h  \nonumber                                                       \\
	                        & \quad - h_{suelo_k} \cdot A_{suelo} \cdot (T_{suelo_k} - T_{habitacion_k})  \nonumber                                                          \\
	                        & \quad - \sigma \cdot \epsilon_{hormigon} \cdot A_{suelo} \cdot (T_{suelo_k}^4 - T_{habitacion_k}^4)  \nonumber                                 \\
	                        & \quad + U_{paredes} \cdot A_{paredes} \cdot (T_{habitacion_k} - T_{amb_k}) \nonumber                                                           \\
	                        & \quad + U_{techo} \cdot A_{techo} \cdot (T_{habitacion_k} - T_{amb_k}) \nonumber                                                               \\
	                        & \quad + U_{ventanas} \cdot A_{ventanas} \cdot (T_{habitacion_k} - T_{amb_k}) = 0  \label{eq:sys_4_sand}                        & \forall k > 0
\end{align}

con los límites:

\begin{align}
	 & T_{objetivo} \leq T_{habitacion_k} \leq 500[K]  \label{eq:min_t_habitacion_constraint_sand} & \forall k \\
	 & 273[K] \leq T_{cond_k}, T_{tanque_k}, T_{suelo_k} \leq 500[K]                               & \forall k \\
	 & 0 \leq P_{bomba_k} \leq P_{bomba_{max}} \quad                                               & \forall k \\
	 & \text{SOC}_{min} \cdot e_{max} \leq e_k \leq \text{SOC}_{max} \cdot e_{max} \quad           & \forall k \\
	 & -P_{bat_{max}} \leq P_{bat_k} \leq P_{bat_{max}} \quad                                      & \forall k \\
	 & -P_{red_{max}} \leq P_{red_k} \leq P_{red_{max}} \quad                                      & \forall k \\
	 & 0 \leq \dot{m}_{cond_k} \leq \dot{m}_{cond_{max}} \quad                                     & \forall k \\
	 & 0 \leq \dot{m}_{cale_k} \leq \dot{m}_{cale_{max}} \quad                                     & \forall k
\end{align}

y teniendo como condiciones iniciales para las variables de estado:

\begin{align}
	 & e_0 = \text{SOC}_{min} \cdot e_{max} \\
	 & T_{cond_0} = 296.56 [K]              \\
	 & T_{tanque_0} = 296.05[K]             \\
	 & T_{suelo_0} = 295.27[K]              \\
	 & T_{habitacion_0} = 293.47[K]         \\
\end{align}


Mientras que en la formulación anterior con gradientes adjuntos resolvíamos el
sistema térmico de forma implícita cada vez que evaluábamos la temperatura
mínima de la habitación $min(T_{habitacion})$ como función de todas las
variables de control tomadas, aquí resolvemos el conjunto de ecuaciones de
forma explícita a la vez que optimizamos. De ahí el nombre SAND: Simultaneous
ANalysis and Design.

Ahora las variables de estado $T_{cond}$, $T_{tanque}$, $T_{suelo}$ y
$T_{habitacion}$, son más variables de diseño que el optimizador ha de
manipular para satisfacer las igualdades del sistema (ahora restricciones).

Por tanto lleva la desventaja de que ahora la solución para el modelo térmico
puede no ser válido si el diseño aún no ha llegado a un punto de convergencia.

Adicionalmente nos encontramos con un número notablemente superior de variables
de diseño y restricciones, que añaden complejidad, pero a pesar de ello la
optimización resulta ser de alrededor un orden de magnitud más rápida, y
aplicar condiciones sobre las variables de estado es trivial, por lo que ahora
nos permitimos aplicar los límites

\begin{equation*}
	T_{objetivo} < T_{habitacion_k} \quad \forall k
\end{equation*}

que podría argumentarse proporciona unos gradientes más suaves que la función
mínimo en $min(T_{habitacion})$, y por ende el problema resulta más sencillo de
resolver.

Sin embargo, con este tipo de formulación se hace especialmente crítico el uso
de matrices dispersas.

Si de forma ingenua fuéramos a almacenar una
matriz jacobiana para las derivadas de una igualdad que se repite a cada paso
$h$ respecto de una variable de diseño como $P_{bomba}$.

\begin{equation} \label{eq:jacobian_matrix_sand}
	\begin{bmatrix}
		\frac{\partial h_1}{\partial P_{bomba_1}} & \ldots & \frac{\partial h_1}{\partial P_{bomba_k}} & \ldots & \frac{\partial h_1}{\partial P_{bomba_N}} \\
		\vdots                                    & \ddots & \vdots                                    & \ddots & \vdots                                    \\
		\frac{\partial h_k}{\partial P_{bomba_1}} & \ldots & \frac{\partial h_k}{\partial P_{bomba_k}} & \ldots & \frac{\partial h_k}{\partial P_{bomba_N}} \\
		\vdots                                    & \ddots & \vdots                                    & \ddots & \vdots                                    \\
		\frac{\partial h_N}{\partial P_{bomba_1}} & \ldots & \frac{\partial h_N}{\partial P_{bomba_k}} & \ldots & \frac{\partial h_N}{\partial P_{bomba_N}} \\
	\end{bmatrix}
\end{equation}

si tuviéramos 1000 pasos ($N=1000$), la matrix \eqref{eq:jacobian_matrix_sand}
tendría 1 millón de elementos, con 10000, 100 millones. El coste de memoria
crece de forma cuadrática con el número de pasos, y el coste computacional
también, ya que la complejidad temporal de un algoritmo de resolución de
sistemas lineales como el método del gradiente conjugado es de $O(m\sqrt{k})$
donde $m$ es el número de elementos no nulos de la matriz, y $k$ su número de
condición (Shewchuk, capítulo 10 \cite{shewchuk1994introduction}).

Pero en realidad la mayoría de las entradas serían nulas, porque en la
discretización de la dinámica de un sistema, a cada paso la igualdad solo
depende de unos cuantos estados vecinos. Por ejemplo, usando euler implícito
solo aparecen términos del estado actual y anterior.

\begin{equation} \label{eq:implicit_euler_equality}
	h_k: \quad y_k - y_{k-1} - h f(t_k, u_k, y_k, p_k) = 0
\end{equation}

siendo la jacobiana de $h$ respecto de $u$, una matriz diagonal:

\begin{equation} \label{eq:diagonal_jacobian}
	J_h = \frac{\partial h}{\partial u} =
	\begin{bmatrix}
		-h \frac{\partial f(t_1, u_1, y_1, p_1)}{\partial u_1} & 0                                                      & \cdots & 0                                                      \\
		0                                                      & -h \frac{\partial f(t_2, u_2, y_2, p_2)}{\partial u_2} & \cdots & 0                                                      \\
		\vdots                                                 & \vdots                                                 & \ddots & \vdots                                                 \\
		0                                                      & 0                                                      & \cdots & -h \frac{\partial f(t_N, u_N, y_N, p_N)}{\partial u_N}
	\end{bmatrix}
\end{equation}

Igualmente, en nuestro problema de optimización en el que tenemos dinámicas
como restricciones, sus jacobianas serán dispersas, y ganaremos mucha eficiencia
si optamos por el uso de un framework que nos permita usar matrices dispersas y
posteriormente se las proporcione a un optimizador que sea compatible con ellas,
como lo es Ipopt.

Así se explica que anteriormente hayamos elegido las energías de la batería como variables
de diseño.

Habría sido posible limitar el SOC usando meramente las potencias destinadas a
la batería, como:

\begin{equation}
	\text{SOC}_{min} \cdot e_{max} \leq e_0 + \sum P_{bat} \cdot h \leq \text{SOC}_{max} \cdot e_{max}, \quad \forall k
\end{equation}

pero entonces la jacobiana de la inigualdad respecto de la potencia a batería,
habría sido tringular inferior, densa:

\begin{equation}
	\mathbf{J} =
	\begin{pmatrix}
		h      & 0      & 0      & \cdots & 0      \\
		h      & h      & 0      & \cdots & 0      \\
		h      & h      & h      & \cdots & 0      \\
		\vdots & \vdots & \vdots & \ddots & \vdots \\
		h      & h      & h      & \cdots & h      \\
	\end{pmatrix}
\end{equation}

introduciendo la energía almacenada como una nueva variable de diseño, incorporamos
además una nueva dinámica:

\begin{equation}
	e_k = e_{k-1} + \eta_{bat} \cdot P_{bat_k} \cdot h \quad \forall k > 0
\end{equation}

pero tanto la jacobiana de esta, como de la nueva limitación para el SOC

\begin{equation}
	\text{SOC}_{min} \cdot e_{max} \leq e_k \leq \text{SOC}_{max} \cdot e_{max} \quad \forall k
\end{equation}

son ahora diagonales, y por tanto el problema resultante tiene un menor coste
computacional.

\subsubsection{Soluciones}

En la figura \ref{fig:control_sand_7_days} vemos la optimización realizada para
un periodo de 7 días, y en el cuadro \ref{tab:control_sand_7_days} los
datos de la ejecución.

\begin{figure}[h] \centering
	\centering
	\includesvg[width=0.92\textwidth]{./capitulos/resultados_discusion/images/control_sand_7_days}
	\caption{Control óptimo para los 7 primeros días del año 2022, método 'all-at-once'.}
	\label{fig:control_sand_7_days}
\end{figure}

\begin{table}[ht]
	\centering
	\caption{Datos para ejecución mostrada en la figura \ref{fig:control_sand_7_days}}
	\label{tab:control_sand_7_days}
	\begin{tabular}{@{}lcc@{}}
		\toprule
		Parámetro                                 & Valor \\
		\midrule
		Número de iteraciones realizadas          & 300   \\
		Tiempo total (segundos)                   & 90    \\
		Número de pasos                           & 605   \\
		Número de variables de diseño             & 5445  \\
		\midrule
		Valor de la función objetivo (€)          & 17.43 \\
		\midrule
		Coste fijo de energía                     & 3.75  \\
		Coste variable de energía                 & 7.39  \\
		Coste de depreciación de batería          & 0.79  \\
		Coste de depreciación de paneles solares  & 3.34  \\
		Coste de depreciación de bomba de calor   & 1.07  \\
		Coste de depreciación de depósito de agua & 1.13  \\
		\bottomrule
	\end{tabular}
\end{table}

Y usando un periodo de 30 días en la figura \ref{fig:control_sand_30_days}, y
cuadro \ref{tab:control_sand_30_days}.

Llegamos a la conclusión de que el análisis y optimización del problema
simultáneo rinde mejor a pesar de añadir complejidad a la formulación, y por
ello solo vamos a tratar en lo siguiente con el método SAND.

\begin{figure}[h] \centering
	\centering
	\includesvg[width=0.92\textwidth]{./capitulos/resultados_discusion/images/control_sand_30_days}
	\caption{Control óptimo para los 30 primeros días del año 2022, método 'all-at-once'.}
	\label{fig:control_sand_30_days}
\end{figure}

\begin{table}[ht]
	\centering
	\caption{Datos para ejecución mostrada en la figura \ref{fig:control_sand_30_days}}
	\label{tab:control_sand_30_days}
	\begin{tabular}{@{}lcc@{}}
		\toprule
		Parámetro                                 & Valor \\
		\midrule
		Número de iteraciones realizadas          & 300   \\
		Tiempo total (segundos)                   & 396   \\
		Número de pasos                           & 2592  \\
		Número de variables de diseño             & 23328 \\
		\midrule
		Valor de la función objetivo (€)          & 85.92 \\
		\midrule
		Coste fijo de energía                     & 16.06 \\
		Coste variable de energía                 & 43.17 \\
		Coste de depreciación de batería          & 2.48  \\
		Coste de depreciación de paneles solares  & 14.31 \\
		Coste de depreciación de bomba de calor   & 5.08  \\
		Coste de depreciación de depósito de agua & 4.87  \\
		\bottomrule
	\end{tabular}
\end{table}
