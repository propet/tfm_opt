Aquí integramos los grupos eléctrico y térmico para minimizar el coste 
de operación de la instalación bajo una tarifa regulada de compensación
de excedentes con energía solar.

Tenemos como variables de diseño las señales de control: potencia a baterías,
potencia a bomba de calor, caudales de agua por el suelo radiante y bomba de calor.

Y planteamos el problema de dos formas diferentes. Primero obteniendo
los gradientes del sistema térmico con el método de gradientes adjuntos discretos,
y después con un método 'all-at-once', donde introducimos las ecuaciones
térmicas como restricciones al problema de optimización.


\subsection{Coste en modalidad de autoconsumo con compensación}
\subsection{Derivadas adjuntas}
\subsection{SAND}

aqui vemos que SAND es superior a MDF, en caso de ser viable su uso,
y por tanto a partir de ahora estudiamos solo el uso de SAND


