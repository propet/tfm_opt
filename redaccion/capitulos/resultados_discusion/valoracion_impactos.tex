La optimización energética de una vivienda, como la realizada en este estudio,
ofrece un enfoque integral para hacer un uso más eficiente y práctico de los
recursos disponibles.

Un impacto directo de este tipo de estudios es la ampliación del uso de
energías renovables, no solo como un suplemento, sino como un pilar fundamental
en la alimentación energética de edificaciones. Nuestra simulación cuantifica
ese aporte al determinar las dimensiones óptimas y el control de equipos como
paneles fotovoltaicos, baterías y sistemas de calentamiento basados en bomba de
calor, permitiendo a las viviendas reducir su dependencia de fuentes no
renovables, y por tanto su contribución a la emisión de gases de efecto
invernadero.

Sin embargo, este trabajo también ofrece una visión del mercado energético
español. Nuestros resultados indican que bajo la tarifa regulada con
compensación por excedentes, la inversión en baterías no resulta económicamente
viable, exceptuando escenarios donde la vivienda esté desconectada de la red
(off-grid) o en condiciones donde se permita la venta ilimitada de energía al
precio de mercado. En tal caso los paneles solares fotovoltaicos se convierten
en una opción muy rentable al permitir que la vivienda no sólo cubra su demanda
energética, sino que también participe activamente en el mercado eléctrico,
generando ingresos adicionales.

En el aspecto térmico se aprecia que en viviendas con suelo radiante, el propio
suelo otorga suficiente inercia térmica, lo que hace innecesario el uso de un
tanque de agua como almacenamiento térmico. No obstante, es importante
mencionar que esta conclusión está sujeta a las condiciones del estudio, en el
que hemos considerado únicamente la calefacción y no el consumo de agua
caliente sanitaria (ACS), ni el uso de paneles solares térmicos.

En resumen, la aplicación de la optimización energética nos permite maximizar
el impacto de las energías renovables, y obtener una visión más informada de
cómo la regulación y precios de mercado afectan al dimensionamiento de los
equipos de una vivienda.
