\subsection{Sistema de ecuaciones reducido}

Queremos reorganizar las ecuaciones obtenidas previamente para obtener un
sistema más reducido, con menos variables de estados y control.

Para ello, primero podemos introducir \eqref{eq:q_cond_1} y \eqref{eq:q_cond_2}
en \eqref{eq:cop_t_cond}, para tener:

\begin{equation} \label{eq:sys_1}
	\text{cop}(T_{cond}) \cdot P_{bomba} - \dot{m}_{cond} \cdot cp_{agua} \cdot \left(T_{cond} - T_{tanque}\right) = 0
\end{equation}

Por otro lado, sustituyendo \eqref{eq:m_dot_tanque} y \eqref{eq:q_perdida} en
la ecuación del balance de energía para el depósito
\eqref{eq:t_tanque_balance}, queda:

\begin{align} \label{eq:sys_2}
	m_{tanque} \cdot cp_{agua} \cdot \left( \frac{dT_{tanque}}{dt} \right) & - \dot{m}_{cond} \cdot cp_{agua} \cdot T_{cond} \nonumber                      \\
	                                                                       & - \dot{m}_{cale} \cdot cp_{agua} \cdot T_{cale} \nonumber                      \\
	                                                                       & + (\dot{m}_{cond} + \dot{m}_{cale}) \cdot cp_{agua} \cdot T_{tanque} \nonumber \\
	                                                                       & + U_{tanque} \cdot A_{tanque} \cdot (T_{tanque} - T_{amb}) \nonumber           \\
	                                                                       & = 0
\end{align}


E insertando las expresiones para los tasas de calor por conducción, radiación
y convección del suelo radiante \eqref{eq:q_conduccion_1},
\eqref{eq:q_radiacion} y \eqref{eq:q_conveccion} respectivamente, en las
fórmulas \eqref{eq:q_conduccion_2}, \eqref{eq:floor_energy_conservation}, y
\eqref{eq:room_energy_conservation}, resultan:


\begin{align}
	\dot{m}_{cale} \cdot cp_{agua} \cdot \left(T_{tanque} - T_{cale}\right)  & - U_{tubos} \cdot A_{tubos} \cdot \Delta T_{tubos} \nonumber                                         \\
	                                                                         & = 0 \label{eq:sys_3}                                                                                 \\
	m_{suelo} \cdot cp_{suelo} \cdot \left( \frac{dT_{suelo}}{dt} \right)    & - \dot{m}_{cale} \cdot cp_{agua} \cdot (T_{tanque} - T_{cale})                             \nonumber \\
	                                                                         & + h_{suelo} \cdot A_{suelo} \cdot (T_{suelo} - T_{habitacion})                             \nonumber \\
	                                                                         & + \sigma \cdot \epsilon_{hormigon} \cdot A_{suelo} \cdot (T_{suelo}^4 - T_{habitacion}^4)  \nonumber \\
	                                                                         & = 0  \label{eq:sys_4}                                                                                \\
	m_{aire} \cdot cp_{aire} \cdot \left( \frac{dT_{habitacion}}{dt} \right) & - h_{suelo} \cdot A_{suelo} \cdot (T_{suelo} - T_{habitacion})  \nonumber                            \\
	                                                                         & - \sigma \cdot \epsilon_{hormigon} \cdot A_{suelo} \cdot (T_{suelo}^4 - T_{habitacion}^4)  \nonumber \\
	                                                                         & + U_{paredes} \cdot A_{paredes} \cdot (T_{habitacion} - T_{amb}) \nonumber                           \\
	                                                                         & + U_{techo} \cdot A_{techo} \cdot (T_{habitacion} - T_{amb}) \nonumber                               \\
	                                                                         & + U_{ventanas} \cdot A_{ventanas} \cdot (T_{habitacion} - T_{amb}) \nonumber                         \\
	                                                                         & = 0  \label{eq:sys_5}
\end{align}

También podríamos haber despejado $\dot{m}_{cale}$ de \eqref{eq:sys_3} y
reemplazado en \eqref{eq:sys_4} y \eqref{eq:sys_5}, para deshacernos de otra
expresión, a costa de oscurecer un poco más las identidades \eqref{eq:sys_4} y
\eqref{eq:sys_5}.


\subsection{Resolución con euler implícito y Newton-Raphson}

Finalmente, con las igualdades \eqref{eq:sys_1}, \eqref{eq:sys_2},
\eqref{eq:sys_3}, \eqref{eq:sys_4} y \eqref{eq:sys_5}, tenemos un sistema de 5
ecuaciones

para 5 incógnitas: $T_{cond}$, $T_{tanque}$, $T_{cale}$, $T_{suelo}$ y
$T_{habitacion}$. Teniendo 4 variables de control: $\dot{m}_{cond}$,
$\dot{m}_{cale}$, $P_{bomba}$ y $T_{amb}$, aunque realmente la temperatura
ambiente no sea una variable que podamos controlar.

Se trata de un sistema de ecuaciones diferenciales algebraicas (DAE,
Differential Algebraic Equation), donde \eqref{eq:sys_1} y \eqref{eq:sys_3} son
algebraicas, siendo el resto diferenciales.

De índice diferencial 1, porque podemos diferenciar una vez tanto
\eqref{eq:sys_1} como \eqref{eq:sys_3} para obtener $\frac{dT_{cond}}{dt}$ y
$\frac{dT_{cale}}{dt}$ respectivamente, convirtiendo el conjunto en un sistema
de ecuaciones diferenciales ordinarias.

Y no lineal, al tener presente la fórmula $\text{cop}(T_{cond})$ en
\eqref{eq:sys_1}, que a pesar de ser lineal a tramos, es no-lineal en su
conjunto.

La estrategia seguida para la resolución ha sido discretizar con un método
implícito, en este caso hemos optado por el método de Euler hacia atrás
(backward Euler), y en cada paso resolvemos el sistema de ecuaciones
no-lineales algebraicas resultantes con Newton-Raphson.

Recordando que la discretización con Euler implícito de un modelo ejemplo como

\begin{equation}
	\frac{dy}{dt} = f(y, t)
\end{equation}

corresponde a

\begin{equation}
	\frac{y_k - y_{k-1}}{h} = f(y_k, t_k)
\end{equation}

donde $h$ es el paso empleado.

Nuestro sistema reducido se presenta de la siguiente forma:

\begin{align}
	\text{cop}(T_{cond,k}) \cdot P_{bomba,k}                                                    & - \dot{m}_{cond,k} \cdot cp_{agua} \cdot \left(T_{cond,k} - T_{tanque,k}\right) = 0                     \\
	m_{tanque} \cdot cp_{agua} \cdot \frac{\left(T_{tanque,k} - T_{tanque,k-1}\right)}{h}       & - \dot{m}_{cond,k} \cdot cp_{agua} \cdot T_{cond,k} \nonumber                                           \\
	                                                                                            & - \dot{m}_{cale,k} \cdot cp_{agua} \cdot T_{cale,k} \nonumber                                           \\
	                                                                                            & + (\dot{m}_{cond,k} + \dot{m}_{cale,k}) \cdot cp_{agua} \cdot T_{tanque,k} \nonumber                    \\
	                                                                                            & + U_{tanque} \cdot A_{tanque} \cdot (T_{tanque,k} - T_{amb,k}) \nonumber                                \\
	                                                                                            & = 0                                                                                                     \\
	\dot{m}_{cale,k} \cdot cp_{agua} \cdot \left(T_{tanque,k} - T_{cale,k}\right)               & - U_{tubos} \cdot A_{tubos} \cdot \Delta T_{tubos,k} \nonumber                                          \\
	                                                                                            & = 0                                                                                                     \\
	m_{suelo} \cdot cp_{suelo} \cdot \frac{\left(T_{suelo,k} - T_{suelo,k-1}\right)}{h}         & - \dot{m}_{cale,k} \cdot cp_{agua} \cdot (T_{tanque,k} - T_{cale,k}) \nonumber                          \\
	                                                                                            & + h_{suelo} \cdot A_{suelo} \cdot (T_{suelo,k} - T_{habitacion,k}) \nonumber                            \\
	                                                                                            & + \sigma \cdot \epsilon_{hormigon} \cdot A_{suelo} \cdot (T_{suelo,k}^4 - T_{habitacion,k}^4) \nonumber \\
	                                                                                            & = 0                                                                                                     \\
	m_{aire} \cdot cp_{aire} \cdot \frac{\left(T_{habitacion,k} - T_{habitacion,k-1}\right)}{h} & - h_{suelo} \cdot A_{suelo} \cdot (T_{suelo,k} - T_{habitacion,k}) \nonumber                            \\
	                                                                                            & - \sigma \cdot \epsilon_{hormigon} \cdot A_{suelo} \cdot (T_{suelo,k}^4 - T_{habitacion,k}^4) \nonumber \\
	                                                                                            & + U_{paredes} \cdot A_{paredes} \cdot (T_{habitacion,k} - T_{amb,k}) \nonumber                          \\
	                                                                                            & + U_{techo} \cdot A_{techo} \cdot (T_{habitacion,k} - T_{amb,k}) \nonumber                              \\
	                                                                                            & + U_{ventanas} \cdot A_{ventanas} \cdot (T_{habitacion,k} - T_{amb,k}) \nonumber                        \\
	                                                                                            & = 0
\end{align}


\subsection{Simulación}



\begin{minted}{python}
def fibonacci(n):
    if n <= 0:
        return 0
    elif n == 1:
        return 1
    else:
        return fibonacci(n-1) + fibonacci(n-2)
\end{minted}
