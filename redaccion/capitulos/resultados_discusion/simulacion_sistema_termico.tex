\subsection{Sistema de ecuaciones reducido}
\label{subsec:reduced_system}

Queremos reorganizar las ecuaciones obtenidas previamente para obtener un
sistema más reducido, con menos variables de estado y control.

Para ello, primero podemos introducir \eqref{eq:q_cond_1} y \eqref{eq:q_cond_2}
en \eqref{eq:cop_t_cond}, para tener:

\begin{equation}
	\text{cop}(T_{cond}) \cdot P_{bomba} - \dot{m}_{cond} \cdot cp_{agua} \cdot \left(T_{cond} - T_{tanque}\right) = 0
\end{equation}

Por otro lado, sustituyendo \eqref{eq:m_dot_tanque} y \eqref{eq:q_perdida} en
la ecuación del balance de energía para el depósito
\eqref{eq:t_tanque_balance}, queda:

\begin{align}
	m_{tanque} \cdot cp_{agua} \cdot \left( \frac{dT_{tanque}}{dt} \right) & - \dot{m}_{cond} \cdot cp_{agua} \cdot T_{cond} \nonumber                      \\
	                                                                       & - \dot{m}_{cale} \cdot cp_{agua} \cdot T_{cale} \nonumber                      \\
	                                                                       & + (\dot{m}_{cond} + \dot{m}_{cale}) \cdot cp_{agua} \cdot T_{tanque} \nonumber \\
	                                                                       & + U_{tanque} \cdot A_{tanque} \cdot (T_{tanque} - T_{amb}) \nonumber           \\
	                                                                       & = 0
\end{align}


E insertando las expresiones para las tasas de calor por conducción, radiación
y convección del suelo radiante \eqref{eq:q_conduccion_1},
\eqref{eq:q_radiacion} y \eqref{eq:q_conveccion} respectivamente, en las
fórmulas \eqref{eq:q_conduccion_2}, \eqref{eq:floor_energy_conservation}, y
\eqref{eq:room_energy_conservation}, resultan:


\begin{align}
	\dot{m}_{cale} \cdot cp_{agua} \cdot \left(T_{tanque} - T_{cale}\right)  & - U_{tubos} \cdot A_{tubos} \cdot \Delta T_{tubos} \nonumber                                         \\
	                                                                         & = 0 \label{eq:m_cale_delta_t}                                                                        \\
	m_{suelo} \cdot cp_{suelo} \cdot \left( \frac{dT_{suelo}}{dt} \right)    & - \dot{m}_{cale} \cdot cp_{agua} \cdot (T_{tanque} - T_{cale})                             \nonumber \\
	                                                                         & + h_{suelo} \cdot A_{suelo} \cdot (T_{suelo} - T_{habitacion})                             \nonumber \\
	                                                                         & + \sigma \cdot \epsilon_{hormigon} \cdot A_{suelo} \cdot (T_{suelo}^4 - T_{habitacion}^4)  \nonumber \\
	                                                                         & = 0                                                                                                  \\
	m_{aire} \cdot cp_{aire} \cdot \left( \frac{dT_{habitacion}}{dt} \right) & - h_{suelo} \cdot A_{suelo} \cdot (T_{suelo} - T_{habitacion})  \nonumber                            \\
	                                                                         & - \sigma \cdot \epsilon_{hormigon} \cdot A_{suelo} \cdot (T_{suelo}^4 - T_{habitacion}^4)  \nonumber \\
	                                                                         & + U_{paredes} \cdot A_{paredes} \cdot (T_{habitacion} - T_{amb}) \nonumber                           \\
	                                                                         & + U_{techo} \cdot A_{techo} \cdot (T_{habitacion} - T_{amb}) \nonumber                               \\
	                                                                         & + U_{ventanas} \cdot A_{ventanas} \cdot (T_{habitacion} - T_{amb}) \nonumber                         \\
	                                                                         & = 0
\end{align}

También, de \eqref{eq:m_cale_delta_t} y \eqref{eq:mean_delta_T}, podemos despejar $T_{cale}$.

\begin{equation} \label{eq:t_cale}
	T_{cale} = \frac{2 \cdot A_{tubos} \cdot U_{tubos} \cdot T_{suelo} - A_{tubos} \cdot U_{tubos} \cdot T_{tanque} + 2 \cdot \dot{m}_{cale} \cdot cp_{agua} \cdot T_{tanque}}{A_{tubos} \cdot U_{tubos} + 2 \cdot \dot{m}_{cale} \cdot cp_{agua}}
\end{equation}

De forma que nuestro sistema finalmente consiste de:

\begin{align}
	\text{cop}(T_{cond}) \cdot P_{bomba}                                     & - \dot{m}_{cond} \cdot cp_{agua} \cdot \left(T_{cond} - T_{tanque}\right) \nonumber                  \\
	                                                                         & = 0 \label{eq:sys_1}                                                                                 \\
	m_{tanque} \cdot cp_{agua} \cdot \left( \frac{dT_{tanque}}{dt} \right)   & - \dot{m}_{cond} \cdot cp_{agua} \cdot T_{cond} \nonumber                                            \\
	                                                                         & - \dot{m}_{cale} \cdot cp_{agua} \cdot T_{cale} \nonumber                                            \\
	                                                                         & + (\dot{m}_{cond} + \dot{m}_{cale}) \cdot cp_{agua} \cdot T_{tanque} \nonumber                       \\
	                                                                         & + U_{tanque} \cdot A_{tanque} \cdot (T_{tanque} - T_{amb}) \nonumber                                 \\
	                                                                         & = 0 \label{eq:sys_2}                                                                                 \\
	m_{suelo} \cdot cp_{suelo} \cdot \left( \frac{dT_{suelo}}{dt} \right)    & - \dot{m}_{cale} \cdot cp_{agua} \cdot (T_{tanque} - T_{cale})                             \nonumber \\
	                                                                         & + h_{suelo} \cdot A_{suelo} \cdot (T_{suelo} - T_{habitacion})                             \nonumber \\
	                                                                         & + \sigma \cdot \epsilon_{hormigon} \cdot A_{suelo} \cdot (T_{suelo}^4 - T_{habitacion}^4)  \nonumber \\
	                                                                         & = 0  \label{eq:sys_3}                                                                                \\
	m_{aire} \cdot cp_{aire} \cdot \left( \frac{dT_{habitacion}}{dt} \right) & - h_{suelo} \cdot A_{suelo} \cdot (T_{suelo} - T_{habitacion})  \nonumber                            \\
	                                                                         & - \sigma \cdot \epsilon_{hormigon} \cdot A_{suelo} \cdot (T_{suelo}^4 - T_{habitacion}^4)  \nonumber \\
	                                                                         & + U_{paredes} \cdot A_{paredes} \cdot (T_{habitacion} - T_{amb}) \nonumber                           \\
	                                                                         & + U_{techo} \cdot A_{techo} \cdot (T_{habitacion} - T_{amb}) \nonumber                               \\
	                                                                         & + U_{ventanas} \cdot A_{ventanas} \cdot (T_{habitacion} - T_{amb}) \nonumber                         \\
	                                                                         & = 0  \label{eq:sys_4}
\end{align}

donde $T_{cale}$ sale de \eqref{eq:t_cale}, y $h_{suelo}$ de \eqref{eq:h_suelo}.


\subsection{Resolución con euler implícito y Newton-Raphson}

Con las igualdades \eqref{eq:sys_1}, \eqref{eq:sys_2},
\eqref{eq:sys_3}, y \eqref{eq:sys_4}, tenemos un sistema de 4
ecuaciones para 4 incógnitas: $T_{cond}$, $T_{tanque}$, $T_{suelo}$
y $T_{habitacion}$. Teniendo 4 variables de control: $\dot{m}_{cond}$,
$\dot{m}_{cale}$, $P_{bomba}$ y $T_{amb}$, aunque realmente la temperatura
ambiente no sea una variable que podamos controlar.

Se trata de ecuaciones diferenciales algebraicas (DAE,
Differential Algebraic Equation), donde solo \eqref{eq:sys_1} es algebraica,
siendo el resto diferenciales.

De índice diferencial 1, porque podemos diferenciar una vez \eqref{eq:sys_1}
para obtener $\frac{dT_{cond}}{dt}$, convirtiéndose en un conjunto de
ecuaciones diferenciales ordinarias.

Y no lineal, al tener presente la fórmula $\text{cop}(T_{cond})$ en
\eqref{eq:sys_1}, que a pesar de ser lineal a tramos, es no-lineal en su
conjunto.

La estrategia seguida para la resolución ha sido discretizar con un método
implícito, en este caso hemos optado por el método de Euler hacia atrás
(backward Euler), y en cada paso resolvemos el sistema de ecuaciones
no-lineales algebraicas resultantes con Newton-Raphson.

Recordando que la discretización con Euler implícito de un modelo ejemplo como

\begin{equation}
	\frac{dy}{dt} = f(y, t)
\end{equation}

corresponde a

\begin{equation}
	\frac{y_k - y_{k-1}}{h} = f(y_k, t_k)
\end{equation}

donde $h$ es el paso empleado.

Nuestro sistema reducido se presenta de la siguiente forma:

\begin{align}
	\text{cop}(T_{cond,k}) \cdot P_{bomba,k}                                                    & - \dot{m}_{cond,k} \cdot cp_{agua} \cdot \left(T_{cond,k} - T_{tanque,k}\right) = 0                     \\
	m_{tanque} \cdot cp_{agua} \cdot \frac{\left(T_{tanque,k} - T_{tanque,k-1}\right)}{h}       & - \dot{m}_{cond,k} \cdot cp_{agua} \cdot T_{cond,k} \nonumber                                           \\
	                                                                                            & - \dot{m}_{cale,k} \cdot cp_{agua} \cdot T_{cale} \nonumber                                             \\
	                                                                                            & + (\dot{m}_{cond,k} + \dot{m}_{cale,k}) \cdot cp_{agua} \cdot T_{tanque,k} \nonumber                    \\
	                                                                                            & + U_{tanque} \cdot A_{tanque} \cdot (T_{tanque,k} - T_{amb,k}) \nonumber                                \\
	                                                                                            & = 0                                                                                                     \\
	m_{suelo} \cdot cp_{suelo} \cdot \frac{\left(T_{suelo,k} - T_{suelo,k-1}\right)}{h}         & - \dot{m}_{cale,k} \cdot cp_{agua} \cdot (T_{tanque,k} - T_{cale}) \nonumber                            \\
	                                                                                            & + h_{suelo} \cdot A_{suelo} \cdot (T_{suelo,k} - T_{habitacion,k}) \nonumber                            \\
	                                                                                            & + \sigma \cdot \epsilon_{hormigon} \cdot A_{suelo} \cdot (T_{suelo,k}^4 - T_{habitacion,k}^4) \nonumber \\
	                                                                                            & = 0                                                                                                     \\
	m_{aire} \cdot cp_{aire} \cdot \frac{\left(T_{habitacion,k} - T_{habitacion,k-1}\right)}{h} & - h_{suelo} \cdot A_{suelo} \cdot (T_{suelo,k} - T_{habitacion,k}) \nonumber                            \\
	                                                                                            & - \sigma \cdot \epsilon_{hormigon} \cdot A_{suelo} \cdot (T_{suelo,k}^4 - T_{habitacion,k}^4) \nonumber \\
	                                                                                            & + U_{paredes} \cdot A_{paredes} \cdot (T_{habitacion,k} - T_{amb,k}) \nonumber                          \\
	                                                                                            & + U_{techo} \cdot A_{techo} \cdot (T_{habitacion,k} - T_{amb,k}) \nonumber                              \\
	                                                                                            & + U_{ventanas} \cdot A_{ventanas} \cdot (T_{habitacion,k} - T_{amb,k}) \nonumber                        \\
	                                                                                            & = 0
\end{align}


\subsection{Simulación}


En python, empleamos la rutina `fsolve` del paquete `scipy.optimize` para
resolver el sistema discretizado a cada paso con Newton-Raphson.

\begin{minted}{python}
import numpy as np
from scipy.optimize import fsolve

def solve(y_0, u, dae_p, h, n_steps):
    """
    Solve the differential-algebraic equation (DAE) system using implicit time-stepping.

    Parameters:
    y_0: Initial conditions of the states.
    u: Control inputs over time.
    dae_p: Parameters for the DAE system.
    h: Time step size.
    n_steps: Number of time steps to perform.

    Returns:
    y:  State evolution over time.
    """
    # Initialize solution
    y = np.zeros((len(y_0), n_steps))

    # Initial conditions
    y[:, 0] = y_0

    # Time-stepping loop
    for n in range(1, n_steps):
        # Use fsolve to solve the nonlinear system
        y_n = fsolve(dae_system, y[:, n - 1], args=(y[:, n - 1], u[:, n], dae_p, h))
        y[:, n] = y_n

    return y
\end{minted}

y la función \textit{dae\_system} a su vez recibe el estado actual, previo,
variables de control actuales, parámetros y paso, y devuelve el valor de las
ecuaciones, con todos los términos en un mismo miembro.


\begin{minted}{python}
def dae_system(y, y_prev, u, p, h):
		# Here we name all variables from y, y_prev, u and p:
		# ...

    h_floor_air = get_h_floor_air(
        t_floor,
        t_room,
        gravity_acceleration,
        air_volumetric_expansion_coeff,
        floor_width,
        nu_air,
        Pr_air,
        k_air,
        A_roof,
    )

    h_tube_water = get_h_tube_water(
        tube_inner_diameter,
        mu_water_at_320K,
        Pr_water,
        k_water,
        m_dot_heating,
    )

    U_tubes = 1 / ((1 / h_tube_water) + (1 / (k_pex / tube_thickness)))

    t_out_heating = (
        2 * A_tubes * U_tubes * t_floor - A_tubes * U_tubes * t_tank + 2 * cp_water * m_dot_heating * t_tank
    ) / (A_tubes * U_tubes + 2 * cp_water * m_dot_heating)

    return [
        # 1
        cop(t_cond) * p_compressor - m_dot_cond * cp_water * (t_cond - t_tank),
        # 2
        m_tank * cp_water * ((t_tank - t_tank_prev) / h)
        - m_dot_cond * cp_water * t_cond
        - m_dot_heating * cp_water * t_out_heating
        + (m_dot_cond + m_dot_heating) * cp_water * t_tank
        + U_tank * A_tank * (t_tank - t_amb),
        # 3
        floor_mass * cp_concrete * ((t_floor - t_floor_prev) / h)
        - m_dot_heating * cp_water * (t_tank - t_out_heating)
        + h_floor_air * floor_area * (t_floor - t_room)
        + stefan_boltzmann_constant * epsilon_concrete * floor_area * (t_floor**4 - t_room**4),
        # 4
        room_air_mass * cp_air * ((t_room - t_room_prev) / h)
        - h_floor_air * floor_area * (t_floor - t_room)
        - stefan_boltzmann_constant * epsilon_concrete * floor_area * (t_floor**4 - t_room**4)
        + U_walls * A_walls * (t_room - t_amb)
        + U_roof * A_roof * (t_room - t_amb)
        + U_windows * A_windows * (t_room - t_amb),
    ]
\end{minted}

Probamos a simular este modelo aplicando un paso de $100[s]$, condiciones iniciales:

\begin{align}
	T_{cond_0} = 296.56[K]       \\
	T_{tanque_0} = 296.05[K]     \\
	T_{suelo_0} = 295.27[K]      \\
	T_{habitacion_0} = 293.47[K] \\
\end{align}

y los siguientes controles:

\begin{equation}
	\dot{m}_{cond}(t)[kg \cdot s^{-1}] =
	\begin{cases}
		0.3 & \text{si } 0 \leq t < \frac{3T}{4},            \\
		0.1 & \text{si } \frac{3T}{4} \leq t < \frac{5T}{6}, \\
		0.3 & \text{si } \frac{5T}{6} \leq t \leq T.
	\end{cases}
\end{equation}

\begin{equation}
	\dot{m}_{cale}(t)[kg \cdot s^{-1}] =
	\begin{cases}
		0.2 & \text{si } 0 \leq t < \frac{T}{2},            \\
		0.1 & \text{si } \frac{T}{2} \leq t < \frac{4T}{5}, \\
		0.2 & \text{si } \frac{4T}{5} \leq t \leq T.
	\end{cases}
\end{equation}

\begin{equation}
	P_{bomba}(t)[W] =
	\begin{cases}
		300                                                      \\
		1 \cdot 10^{-6} & \text{si } \frac{2T}{3} \leq t \leq T.
	\end{cases}
\end{equation}

siendo $T$ el periodo de simulación, los 7 primeros días del año 2022, y
$T_{amb}$ la temperatura ambiente para estos.

El resultado se aprecia en la figura \ref{fig:simulacion_1_h:100s},
donde el primer gráfico muestra la evolución de las temperaturas,
mantiéndose siempre

\begin{equation*}
	T_{cond} > T_{tanque} > T_{suelo} > T_{habitacion}
\end{equation*}

En la segunda ilustración vemos las tasas de calor por transferencia de calor
por conducción (de los tubos a la losa de suelo), por radiación y convección
natural, del suelo al aire de la habitación.

El suelo principalmente calienta por radiación, y la aportación por
convección es alrededor de un tercio de esta.

Cuando la losa de hormigón recibe más calor del agua de la calefacción
(conducción) que evacúa en la habitación, observamos su temperatura aumentar, y
viceversa. Para más claridad se ha representado la suma de los términos
convectivo y radiativo.

En las dos últimas gráficas aparecen las variables de control empleadas: los
caudales dirigidos a la calefacción del suelo radiante, y de la bomba de calor
al tanque de agua, y la potencia destinada a la bomba de calor (o compresor de
esta).

Para ambos, es visible que hemos aplicado funciones escalón para estimular al
sistema.

\begin{figure}[h] \centering
	\centering
	\includesvg[width=1\textwidth]{./capitulos/resultados_discusion/images/simulacion_1_h:100s}
	\caption{Simulación del sistema térmico con paso $h=100[s]$.}
	\label{fig:simulacion_1_h:100s}
\end{figure}


El paso $h$ empleado en las simulaciones debe ser lo suficientemente pequeño
como para poder representar las dinámicas más rápidas del sistema. Estas
dinámicas las observamos al aplicar una función escalón en los controles del
modelo. Y reducimos progresivamente el tamaño del paso hasta que la solución
converja.

En este caso tenemos una respuesta lenta, debido a las inercias térmicas de
tanto depósito, como suelo radiante y habitación, de modo que hemos empleado en
lo siguiente un paso de 1000 segundos, al no apreciar diferencia en los
resultados que se arrojan (figura \ref{fig:simulacion_2_h:1000s}), y así no
tenemos la necesidad de considerar tantas variables de diseño en nuestros
estudios de control óptimo, donde a cada paso elegimos qué caudales y potencia
destinar a la bomba de calor.

\begin{figure}[h] \centering
	\centering
	\includesvg[width=1\textwidth]{./capitulos/resultados_discusion/images/simulacion_2_h:1000s}
	\caption{Temperaturas del sistema térmico con paso $h=1000[s]$.}
	\label{fig:simulacion_2_h:1000s}
\end{figure}

Tampoco usamos un paso mayor que este porque el sistema eléctrico con el que
vamos a optimizar conjuntamente el sistema, tiene una dinámica horaria
(precios, potencias solares dadas por los paneles y demanda eléctrica del
domicilio se dan horariamente), y es recomendable usar un paso alrededor de un orden
de magnitud menor que la dinámica más rápida.

En el extremo donde se dan respuestas muy rápidas y también muy lentas, se
habla de un sistema rígido, en referencia a un péndulo con un periodo asociado,
que además tiene otra oscilación mucho más rápida en su cuerpo, que puede
modelarse como un resorte rígido (figura \ref{fig:stiff_system}). El problema
que existe en estas circunstancias es que para simular el comportamiento del
péndulo debemos de usar un paso muy pequeño, del orden de la velocidad del
resorte rígido, lo que puede llevar a elevados costes computacionales en la
simulación.

\begin{figure}[h] \centering
	\centering
	\includesvg[width=0.3\textwidth]{./capitulos/resultados_discusion/images/stiff_system}
	\caption{Sistema rígido.}
	\label{fig:stiff_system}
\end{figure}
