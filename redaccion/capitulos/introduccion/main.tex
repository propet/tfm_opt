La eficiencia energética se ha convertido en una prioridad tanto por razones
económicas como ambientales. Con el aumento de los precios de la energía y los
intereses por reducir las emisiones de carbono, optimizar el consumo energético
en nuestros hogares resulta deseado. Siendo Madrid una ciudad con considerables
niveles de radiación solar a lo largo del año, las tecnologías como los paneles
solares y otros sistemas sostenibles presentan una oportunidad para reducir
costos.

Este trabajo se centra en la optimización energética de una vivienda
unifamiliar en Madrid, utilizando un sistema que mezcla varias elementos: una
bomba de calor, paneles fotovoltaicos, baterías para almacenar la energía
eléctrica, y un tanque de agua para almacenamiento térmico. El objetivo es
encontrar la combinación óptima de control y dimensiones de estos equipos para
que el propietario pague lo menos posible en su factura anual de energía y
mantenga la casa a una temperatura confortable en torno a los 20°C.

La bomba de calor se destaca como una de las soluciones más eficientes para la
climatización del hogar, especialmente en comparación con la calefacción
eléctrica o de gas, que suelen ser menos eficientes y más costosas. Este
sistema permite mover el calor desde el exterior hacia el interior de la
vivienda con un menor consumo de electricidad, lo cual es una gran ventaja
cuando se combina con paneles fotovoltaicos.

Y elegimos una calefacción por suelo radiante que se acopla bien con el sistema
de aerotermia, ya que una bomba de calor no es muy eficiente generando agua
caliente sanitaria a muy alta temperatura, pero esto va de la mano con las
necesidades de un suelo radiante, que con una gran superficie de radiación
puede trabajar con bajas temperaturas.

El problema de optimización abordado en este trabajo es de gran escala,
involucrando cientos de miles de variables de diseño que abarcan aspectos como
la potencia de la bomba de calor, la capacidad instalada de paneles solares, la
potencia máxima contratada con la comercializadora, la capacidad de la batería
y el volumen del tanque de agua. Para resolver este problema, se recurre a
técnicas de optimización basadas en gradientes, en concreto a través del método
de gradientes adjuntos discretos para obtener las derivadas de manera
eficiente. La formulación del problema se basa en un conjunto de ecuaciones
diferenciales algebraicas que modelan el comportamiento del sistema durante
todo el año.

Además, en un análisis comparativo, se implementa la técnica SAND (Simultaneous
ANalysis and Design), permitiendo optimizar simultáneamente el diseño y las
decisiones de control de todo el sistema de manera conjunta. Esta técnica,
también conocida como 'all-at-once', se compara con el enfoque de optimización
con gradientes adjuntos para analizar el rendimiento de ambos métodos en la
obtención de soluciones óptimas que minimicen el coste total.

Este trabajo explora las posibilidades de optimización en un escenario de
autoconsumo con compensación simplificada, donde se limita la compensación
económica por los excedentes de electricidad vertidos a la red, también
considera un escenario de conexión a la red con venta de excedentes al precio
de mercado y un escenario de operación en modo 'off-grid', donde la vivienda
permanece desconectada de la red eléctrica.

Se pretende ilustrar el método para poder optimizar el consumo energético en
viviendas o negocios con el uso de energías renovables, no solo para nuestra
localización y casa modelo, ya que los principios son igualmente aplicables a
otro contexto.
