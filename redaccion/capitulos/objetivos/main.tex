El principal objetivo de este trabajo es desarrollar un modelo de optimización
energética para una vivienda que permita minimizar el coste total anual
derivado del uso de una bomba de calor, paneles fotovoltaicos, almacenamiento
de energía (baterías) y un sistema de acumulación térmica (tanque de agua).
Para alcanzar este objetivo, el trabajo abordará los siguientes aspectos:

\begin{itemize}
	\item \textbf{Implentación de gradientes adjuntos discretos para un sistema
		      DAE}: para calcular de manera eficiente los gradientes del sistema.
	      Sección \ref{sec:adjoints}.

	\item \textbf{Técnica SAND (Simultaneous Analysis and Design)}: para resolver
	      el problema de optimización, comparando este enfoque con la metodología con
	      gradientes adjuntos, explorando ventajas y desventajas en términos de
	      rendimiento. Sección \ref{sec:all_at_once}.

	\item \textbf{Adquisición de datos}: precios de la electricidad y aproximación
	      del coste real de los equipos por regresión lineal de datos comerciales.
	      Capítulo \ref{chap:data_acquisition}.

	\item \textbf{Modelar el sistema energético de la vivienda}: se representará
	      el comportamiento dinámico de la vivienda y sus sistemas energéticos mediante
	      un conjunto de ecuaciones diferenciales y algebraicas (DAE's). Secciones
	      \ref{sec:modelo_vivienda}, \ref{sec:sistema_termico} y
	      \ref{sec:sistema_termico}.

	\item \textbf{Evaluar escenarios de operación}: se estudiarán tres escenarios
	      representativos: (1) autoconsumo con compensación simplificada, (2) conexión
	      a la red con venta de excedentes de energía a precio de mercado, y (3)
	      operación off-grid, para minimizar el coste. Sección
	      \ref{sec:funciones_objetivo}.

	\item \textbf{Realizar simulaciones y evaluar resultados}: finalmente, el
	      modelo se validará y se ejecutarán simulaciones que permitan determinar las
	      configuraciones óptimas de los equipos y las señales de control a lo largo de
	      un año completo. Secciones \ref{sec:control} y
	      \ref{sec:control_y_dimensionamiento}.
\end{itemize}
