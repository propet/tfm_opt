\documentclass[a4paper,11pt,twoside]{report}
\usepackage[left=25mm,right=25mm,top=25mm,bottom=25mm,includehead,includefoot,headsep=15mm,footskip=15mm]{geometry}
\usepackage{graphicx}
\usepackage{fancyhdr}
\usepackage{titlesec}
\usepackage[spanish]{babel}
\usepackage[utf8]{inputenc}
\usepackage{amsmath}
\usepackage{setspace}
\usepackage{url}
\usepackage{svg}
\usepackage{booktabs}
\usepackage{hyperref}
\usepackage[backend=biber,style=numeric]{biblatex}
\addbibresource{references.bib}
\hypersetup{
    colorlinks=true,
    linkcolor=blue,      % color of internal links (sections, etc.)
    urlcolor=blue,       % color of external links
    pdftitle={Optimización energética de sistema híbrido con bomba de calor, suelo radiante, fotovoltaica y almacenamiento para vivienda},    % title
    pdfauthor={Luis D. Aranda Sánchez},     % author
    pdfkeywords={palabra1, palabra2, código1, etc.} % list of keywords
}

% Code colorscheme
\usepackage{minted}
\setminted{
    frame=lines,
    linenos,
    fontsize=\small,
    numbersep=5pt,
    tabsize=4,
    breaklines,
    breakanywhere,
    style=friendly  % Cambia este valor por el esquema de colores que prefieras
}

% Font change to Arial
\usepackage{helvet}
\renewcommand{\familydefault}{\sfdefault}

% Chapter titles in uppercase and larger font
\titleformat{\chapter}[hang]{\large\bfseries}{\thechapter.}{1em}{\MakeUppercase}
\titleformat{\section}[hang]{\bfseries}{\thesection.}{1em}{}
\titleformat{\subsection}[hang]{\bfseries}{\thesubsection.}{1em}{}

% Fancyhdr setup
\setlength{\headheight}{14.30174pt} % Adjust to recommended value, slightly larger for safety
\fancyhf{} % Clear all headers and footers
\fancyhead[LE]{\nouppercase{\leftmark}}
\fancyhead[RO]{Optimización energética para vivienda}
\fancyfoot[LE]{\thepage}
\fancyfoot[RE]{Escuela Técnica Superior de Ingenieros Industriales (UPM)}
\fancyfoot[LO]{Luis D. Aranda Sánchez}
\fancyfoot[RO]{\thepage}
\renewcommand{\headrulewidth}{0.4pt}
\renewcommand{\footrulewidth}{0.4pt}

\fancypagestyle{myfancy}{
    \fancyhf{} % Clear all headers and footers
    \fancyhead[LE]{\nouppercase{\leftmark}}
    \fancyhead[RO]{Optimización energética para vivienda}
    \fancyfoot[LE]{\thepage}
    \fancyfoot[RE]{Escuela Técnica Superior de Ingenieros Industriales (UPM)}
    \fancyfoot[LO]{Luis D. Aranda Sánchez}
    \fancyfoot[RO]{\thepage}
    \renewcommand{\headrulewidth}{0.4pt}
    \renewcommand{\footrulewidth}{0.4pt}
}

% Redefine the plain page style (start of chapters)
\fancypagestyle{plain}{
    \fancyhf{}
    \fancyfoot[LE]{\thepage}
    \fancyfoot[RE]{Escuela Técnica Superior de Ingenieros Industriales (UPM)}
    \fancyfoot[LO]{Luis D. Aranda Sánchez}
    \fancyfoot[RO]{\thepage}
    \renewcommand{\headrulewidth}{0pt}
    \renewcommand{\footrulewidth}{0.4pt}
}
\title{A short example}

% Line spacing
\setstretch{1.2}

% Ajusta el espacio superior de las notas al pie
\setlength{\skip\footins}{1.5em}

% Document starts here
\begin{document}

% Portada
\begin{titlepage}
	\centering
	{\scshape\LARGE Universidad Politécnica de Madrid \par}
	\vspace{1cm}
	{\scshape\Large Escuela Técnica Superior de Ingenieros Industriales\par}
	\vspace{1.5cm}
	{\huge\bfseries Optimización energética de sistema híbrido con bomba de calor, suelo radiante, fotovoltaica y almacenamiento para vivienda \par}
	\vspace{1.5cm}
	{\Large\bfseries Trabajo de Fin de Máster\par}
	\vspace{0.5cm}
	{\large Máster Universitario en Ingeniería de la Energía \par}
	\vspace{2cm}
	{\Large Luis D. Aranda Sánchez\par}
	\vfill
	Director: Javier Rodríguez Martín
	\vfill
	{\large Septiembre 6, 2024\par}
\end{titlepage}

% Resumen (máximo de 5 páginas, incluyendo al final Palabras clave)
\cleardoublepage
\chapter*{Resumen}
\addcontentsline{toc}{chapter}{Resumen}
\documentclass[a4paper,11pt,twoside]{report}
\usepackage[left=25mm,right=25mm,top=25mm,bottom=25mm,includehead,includefoot,headsep=15mm,footskip=15mm]{geometry}
\usepackage{graphicx}
\usepackage{fancyhdr}
\usepackage{titlesec}
\usepackage[spanish]{babel}
\usepackage[utf8]{inputenc}
\usepackage{amsmath}
\usepackage{setspace}
\usepackage{svg}
\usepackage{hyperref}
\usepackage[backend=biber,style=numeric]{biblatex}
\addbibresource{references.bib}
\hypersetup{
    colorlinks=true,
    linkcolor=blue,      % color of internal links (sections, etc.)
    urlcolor=blue,       % color of external links
    pdftitle={Optimización energética de sistema híbrido con bomba de calor, suelo radiante, fotovoltaica y almacenamiento para vivienda},    % title
    pdfauthor={Luis D. Aranda Sánchez},     % author
    pdfkeywords={palabra1, palabra2, código1, etc.} % list of keywords
}

% Font change to Arial
\usepackage{helvet}
\renewcommand{\familydefault}{\sfdefault}

% Chapter titles in uppercase and larger font
\titleformat{\chapter}[hang]{\large\bfseries}{\thechapter.}{1em}{\MakeUppercase}
\titleformat{\section}[hang]{\bfseries}{\thesection.}{1em}{}
\titleformat{\subsection}[hang]{\bfseries}{\thesubsection.}{1em}{}

% Fancyhdr setup
\setlength{\headheight}{14.30174pt} % Adjust to recommended value, slightly larger for safety
\fancyhf{} % Clear all headers and footers
\fancyhead[LE]{\nouppercase{\leftmark}}
\fancyhead[RO]{Optimización energética para vivienda}
\fancyfoot[LE]{\thepage}
\fancyfoot[RE]{Escuela Técnica Superior de Ingenieros Industriales (UPM)}
\fancyfoot[LO]{Luis D. Aranda Sánchez}
\fancyfoot[RO]{\thepage}
\renewcommand{\headrulewidth}{0.4pt}
\renewcommand{\footrulewidth}{0.4pt}

\fancypagestyle{myfancy}{
    \fancyhf{} % Clear all headers and footers
    \fancyhead[LE]{\nouppercase{\leftmark}}
    \fancyhead[RO]{Optimización energética para vivienda}
    \fancyfoot[LE]{\thepage}
    \fancyfoot[RE]{Escuela Técnica Superior de Ingenieros Industriales (UPM)}
    \fancyfoot[LO]{Luis D. Aranda Sánchez}
    \fancyfoot[RO]{\thepage}
    \renewcommand{\headrulewidth}{0.4pt}
    \renewcommand{\footrulewidth}{0.4pt}
}

\fancypagestyle{simple}{
    \fancyhf{} % Clear all headers and footers
    \renewcommand{\headrulewidth}{0pt}
    \renewcommand{\footrulewidth}{0pt}
}

% Line spacing
\setstretch{1.2}

% Document starts here
\begin{document}

% Portada
\begin{titlepage}
    \centering
    {\scshape\LARGE Universidad Politécnica de Madrid \par}
    \vspace{1cm}
    {\scshape\Large Escuela Técnica Superior de Ingenieros Industriales\par}
    \vspace{1.5cm}
    {\huge\bfseries Optimización energética de sistema híbrido con bomba de calor, suelo radiante, fotovoltaica y almacenamiento para vivienda \par}
    \vspace{1.5cm}
    {\Large\bfseries Trabajo de Fin de Máster\par}
    \vspace{0.5cm}
    {\large Máster Universitario en Ingeniería de la Energía \par}
    \vspace{2cm}
    {\Large Luis D. Aranda Sánchez\par}
    \vfill
    Director: Javier Rodríguez Martín
    \vfill
    {\large Septiembre 6, 2024\par}
\end{titlepage}

% Resumen (máximo de 5 páginas, incluyendo al final Palabras clave)
\clearpage
\pagestyle{simple}
% \newpage
\chapter*{Resumen}
\addcontentsline{toc}{chapter}{Resumen}
\documentclass[a4paper,11pt,twoside]{report}
\usepackage[left=25mm,right=25mm,top=25mm,bottom=25mm,includehead,includefoot,headsep=15mm,footskip=15mm]{geometry}
\usepackage{graphicx}
\usepackage{fancyhdr}
\usepackage{titlesec}
\usepackage[spanish]{babel}
\usepackage[utf8]{inputenc}
\usepackage{amsmath}
\usepackage{setspace}
\usepackage{svg}
\usepackage{hyperref}
\usepackage[backend=biber,style=numeric]{biblatex}
\addbibresource{references.bib}
\hypersetup{
    colorlinks=true,
    linkcolor=blue,      % color of internal links (sections, etc.)
    urlcolor=blue,       % color of external links
    pdftitle={Optimización energética de sistema híbrido con bomba de calor, suelo radiante, fotovoltaica y almacenamiento para vivienda},    % title
    pdfauthor={Luis D. Aranda Sánchez},     % author
    pdfkeywords={palabra1, palabra2, código1, etc.} % list of keywords
}

% Font change to Arial
\usepackage{helvet}
\renewcommand{\familydefault}{\sfdefault}

% Chapter titles in uppercase and larger font
\titleformat{\chapter}[hang]{\large\bfseries}{\thechapter.}{1em}{\MakeUppercase}
\titleformat{\section}[hang]{\bfseries}{\thesection.}{1em}{}
\titleformat{\subsection}[hang]{\bfseries}{\thesubsection.}{1em}{}

% Fancyhdr setup
\setlength{\headheight}{14.30174pt} % Adjust to recommended value, slightly larger for safety
\fancyhf{} % Clear all headers and footers
\fancyhead[LE]{\nouppercase{\leftmark}}
\fancyhead[RO]{Optimización energética para vivienda}
\fancyfoot[LE]{\thepage}
\fancyfoot[RE]{Escuela Técnica Superior de Ingenieros Industriales (UPM)}
\fancyfoot[LO]{Luis D. Aranda Sánchez}
\fancyfoot[RO]{\thepage}
\renewcommand{\headrulewidth}{0.4pt}
\renewcommand{\footrulewidth}{0.4pt}

\fancypagestyle{myfancy}{
    \fancyhf{} % Clear all headers and footers
    \fancyhead[LE]{\nouppercase{\leftmark}}
    \fancyhead[RO]{Optimización energética para vivienda}
    \fancyfoot[LE]{\thepage}
    \fancyfoot[RE]{Escuela Técnica Superior de Ingenieros Industriales (UPM)}
    \fancyfoot[LO]{Luis D. Aranda Sánchez}
    \fancyfoot[RO]{\thepage}
    \renewcommand{\headrulewidth}{0.4pt}
    \renewcommand{\footrulewidth}{0.4pt}
}

\fancypagestyle{simple}{
    \fancyhf{} % Clear all headers and footers
    \renewcommand{\headrulewidth}{0pt}
    \renewcommand{\footrulewidth}{0pt}
}

% Line spacing
\setstretch{1.2}

% Document starts here
\begin{document}

% Portada
\begin{titlepage}
    \centering
    {\scshape\LARGE Universidad Politécnica de Madrid \par}
    \vspace{1cm}
    {\scshape\Large Escuela Técnica Superior de Ingenieros Industriales\par}
    \vspace{1.5cm}
    {\huge\bfseries Optimización energética de sistema híbrido con bomba de calor, suelo radiante, fotovoltaica y almacenamiento para vivienda \par}
    \vspace{1.5cm}
    {\Large\bfseries Trabajo de Fin de Máster\par}
    \vspace{0.5cm}
    {\large Máster Universitario en Ingeniería de la Energía \par}
    \vspace{2cm}
    {\Large Luis D. Aranda Sánchez\par}
    \vfill
    Director: Javier Rodríguez Martín
    \vfill
    {\large Septiembre 6, 2024\par}
\end{titlepage}

% Resumen (máximo de 5 páginas, incluyendo al final Palabras clave)
\clearpage
\pagestyle{simple}
% \newpage
\chapter*{Resumen}
\addcontentsline{toc}{chapter}{Resumen}
\documentclass[a4paper,11pt,twoside]{report}
\usepackage[left=25mm,right=25mm,top=25mm,bottom=25mm,includehead,includefoot,headsep=15mm,footskip=15mm]{geometry}
\usepackage{graphicx}
\usepackage{fancyhdr}
\usepackage{titlesec}
\usepackage[spanish]{babel}
\usepackage[utf8]{inputenc}
\usepackage{amsmath}
\usepackage{setspace}
\usepackage{svg}
\usepackage{hyperref}
\usepackage[backend=biber,style=numeric]{biblatex}
\addbibresource{references.bib}
\hypersetup{
    colorlinks=true,
    linkcolor=blue,      % color of internal links (sections, etc.)
    urlcolor=blue,       % color of external links
    pdftitle={Optimización energética de sistema híbrido con bomba de calor, suelo radiante, fotovoltaica y almacenamiento para vivienda},    % title
    pdfauthor={Luis D. Aranda Sánchez},     % author
    pdfkeywords={palabra1, palabra2, código1, etc.} % list of keywords
}

% Font change to Arial
\usepackage{helvet}
\renewcommand{\familydefault}{\sfdefault}

% Chapter titles in uppercase and larger font
\titleformat{\chapter}[hang]{\large\bfseries}{\thechapter.}{1em}{\MakeUppercase}
\titleformat{\section}[hang]{\bfseries}{\thesection.}{1em}{}
\titleformat{\subsection}[hang]{\bfseries}{\thesubsection.}{1em}{}

% Fancyhdr setup
\setlength{\headheight}{14.30174pt} % Adjust to recommended value, slightly larger for safety
\fancyhf{} % Clear all headers and footers
\fancyhead[LE]{\nouppercase{\leftmark}}
\fancyhead[RO]{Optimización energética para vivienda}
\fancyfoot[LE]{\thepage}
\fancyfoot[RE]{Escuela Técnica Superior de Ingenieros Industriales (UPM)}
\fancyfoot[LO]{Luis D. Aranda Sánchez}
\fancyfoot[RO]{\thepage}
\renewcommand{\headrulewidth}{0.4pt}
\renewcommand{\footrulewidth}{0.4pt}

\fancypagestyle{myfancy}{
    \fancyhf{} % Clear all headers and footers
    \fancyhead[LE]{\nouppercase{\leftmark}}
    \fancyhead[RO]{Optimización energética para vivienda}
    \fancyfoot[LE]{\thepage}
    \fancyfoot[RE]{Escuela Técnica Superior de Ingenieros Industriales (UPM)}
    \fancyfoot[LO]{Luis D. Aranda Sánchez}
    \fancyfoot[RO]{\thepage}
    \renewcommand{\headrulewidth}{0.4pt}
    \renewcommand{\footrulewidth}{0.4pt}
}

\fancypagestyle{simple}{
    \fancyhf{} % Clear all headers and footers
    \renewcommand{\headrulewidth}{0pt}
    \renewcommand{\footrulewidth}{0pt}
}

% Line spacing
\setstretch{1.2}

% Document starts here
\begin{document}

% Portada
\begin{titlepage}
    \centering
    {\scshape\LARGE Universidad Politécnica de Madrid \par}
    \vspace{1cm}
    {\scshape\Large Escuela Técnica Superior de Ingenieros Industriales\par}
    \vspace{1.5cm}
    {\huge\bfseries Optimización energética de sistema híbrido con bomba de calor, suelo radiante, fotovoltaica y almacenamiento para vivienda \par}
    \vspace{1.5cm}
    {\Large\bfseries Trabajo de Fin de Máster\par}
    \vspace{0.5cm}
    {\large Máster Universitario en Ingeniería de la Energía \par}
    \vspace{2cm}
    {\Large Luis D. Aranda Sánchez\par}
    \vfill
    Director: Javier Rodríguez Martín
    \vfill
    {\large Septiembre 6, 2024\par}
\end{titlepage}

% Resumen (máximo de 5 páginas, incluyendo al final Palabras clave)
\clearpage
\pagestyle{simple}
% \newpage
\chapter*{Resumen}
\addcontentsline{toc}{chapter}{Resumen}
\input{capitulos/resumen/main.tex}

% Índice (paginado)
\clearpage
\pagestyle{simple}
% \newpage
\tableofcontents

% Introducción (donde se incluya los antecedentes y justificación)
\clearpage
\pagestyle{myfancy}
\newpage
\chapter{Introducción}
\input{capitulos/introduccion/main.tex}

% Objetivos
\chapter{Objetivos}
\input{capitulos/objetivos/main.tex}

% Metodología
\chapter{Metodología}
\input{capitulos/metodologia/main.tex}

% Resultados y discusión (incluyendo la valoración de impactos y de aspectos de responsabilidad legal, ética y profesional relacionados con el trabajo)
\chapter{Resultados y Discusión}
\input{capitulos/resultados_discusion/main.tex}

% Conclusiones
\chapter{Conclusiones}
\input{capitulos/conclusiones/main.tex}

% Planificación temporal y presupuesto
\chapter{Planificación Temporal y Presupuesto}
\input{capitulos/planificacion_presupuesto/main.tex}

% Bibliografía
\newpage
\addcontentsline{toc}{chapter}{Bibliografía}
\printbibliography

\end{document}


% Índice (paginado)
\clearpage
\pagestyle{simple}
% \newpage
\tableofcontents

% Introducción (donde se incluya los antecedentes y justificación)
\clearpage
\pagestyle{myfancy}
\newpage
\chapter{Introducción}
\documentclass[a4paper,11pt,twoside]{report}
\usepackage[left=25mm,right=25mm,top=25mm,bottom=25mm,includehead,includefoot,headsep=15mm,footskip=15mm]{geometry}
\usepackage{graphicx}
\usepackage{fancyhdr}
\usepackage{titlesec}
\usepackage[spanish]{babel}
\usepackage[utf8]{inputenc}
\usepackage{amsmath}
\usepackage{setspace}
\usepackage{svg}
\usepackage{hyperref}
\usepackage[backend=biber,style=numeric]{biblatex}
\addbibresource{references.bib}
\hypersetup{
    colorlinks=true,
    linkcolor=blue,      % color of internal links (sections, etc.)
    urlcolor=blue,       % color of external links
    pdftitle={Optimización energética de sistema híbrido con bomba de calor, suelo radiante, fotovoltaica y almacenamiento para vivienda},    % title
    pdfauthor={Luis D. Aranda Sánchez},     % author
    pdfkeywords={palabra1, palabra2, código1, etc.} % list of keywords
}

% Font change to Arial
\usepackage{helvet}
\renewcommand{\familydefault}{\sfdefault}

% Chapter titles in uppercase and larger font
\titleformat{\chapter}[hang]{\large\bfseries}{\thechapter.}{1em}{\MakeUppercase}
\titleformat{\section}[hang]{\bfseries}{\thesection.}{1em}{}
\titleformat{\subsection}[hang]{\bfseries}{\thesubsection.}{1em}{}

% Fancyhdr setup
\setlength{\headheight}{14.30174pt} % Adjust to recommended value, slightly larger for safety
\fancyhf{} % Clear all headers and footers
\fancyhead[LE]{\nouppercase{\leftmark}}
\fancyhead[RO]{Optimización energética para vivienda}
\fancyfoot[LE]{\thepage}
\fancyfoot[RE]{Escuela Técnica Superior de Ingenieros Industriales (UPM)}
\fancyfoot[LO]{Luis D. Aranda Sánchez}
\fancyfoot[RO]{\thepage}
\renewcommand{\headrulewidth}{0.4pt}
\renewcommand{\footrulewidth}{0.4pt}

\fancypagestyle{myfancy}{
    \fancyhf{} % Clear all headers and footers
    \fancyhead[LE]{\nouppercase{\leftmark}}
    \fancyhead[RO]{Optimización energética para vivienda}
    \fancyfoot[LE]{\thepage}
    \fancyfoot[RE]{Escuela Técnica Superior de Ingenieros Industriales (UPM)}
    \fancyfoot[LO]{Luis D. Aranda Sánchez}
    \fancyfoot[RO]{\thepage}
    \renewcommand{\headrulewidth}{0.4pt}
    \renewcommand{\footrulewidth}{0.4pt}
}

\fancypagestyle{simple}{
    \fancyhf{} % Clear all headers and footers
    \renewcommand{\headrulewidth}{0pt}
    \renewcommand{\footrulewidth}{0pt}
}

% Line spacing
\setstretch{1.2}

% Document starts here
\begin{document}

% Portada
\begin{titlepage}
    \centering
    {\scshape\LARGE Universidad Politécnica de Madrid \par}
    \vspace{1cm}
    {\scshape\Large Escuela Técnica Superior de Ingenieros Industriales\par}
    \vspace{1.5cm}
    {\huge\bfseries Optimización energética de sistema híbrido con bomba de calor, suelo radiante, fotovoltaica y almacenamiento para vivienda \par}
    \vspace{1.5cm}
    {\Large\bfseries Trabajo de Fin de Máster\par}
    \vspace{0.5cm}
    {\large Máster Universitario en Ingeniería de la Energía \par}
    \vspace{2cm}
    {\Large Luis D. Aranda Sánchez\par}
    \vfill
    Director: Javier Rodríguez Martín
    \vfill
    {\large Septiembre 6, 2024\par}
\end{titlepage}

% Resumen (máximo de 5 páginas, incluyendo al final Palabras clave)
\clearpage
\pagestyle{simple}
% \newpage
\chapter*{Resumen}
\addcontentsline{toc}{chapter}{Resumen}
\input{capitulos/resumen/main.tex}

% Índice (paginado)
\clearpage
\pagestyle{simple}
% \newpage
\tableofcontents

% Introducción (donde se incluya los antecedentes y justificación)
\clearpage
\pagestyle{myfancy}
\newpage
\chapter{Introducción}
\input{capitulos/introduccion/main.tex}

% Objetivos
\chapter{Objetivos}
\input{capitulos/objetivos/main.tex}

% Metodología
\chapter{Metodología}
\input{capitulos/metodologia/main.tex}

% Resultados y discusión (incluyendo la valoración de impactos y de aspectos de responsabilidad legal, ética y profesional relacionados con el trabajo)
\chapter{Resultados y Discusión}
\input{capitulos/resultados_discusion/main.tex}

% Conclusiones
\chapter{Conclusiones}
\input{capitulos/conclusiones/main.tex}

% Planificación temporal y presupuesto
\chapter{Planificación Temporal y Presupuesto}
\input{capitulos/planificacion_presupuesto/main.tex}

% Bibliografía
\newpage
\addcontentsline{toc}{chapter}{Bibliografía}
\printbibliography

\end{document}


% Objetivos
\chapter{Objetivos}
\documentclass[a4paper,11pt,twoside]{report}
\usepackage[left=25mm,right=25mm,top=25mm,bottom=25mm,includehead,includefoot,headsep=15mm,footskip=15mm]{geometry}
\usepackage{graphicx}
\usepackage{fancyhdr}
\usepackage{titlesec}
\usepackage[spanish]{babel}
\usepackage[utf8]{inputenc}
\usepackage{amsmath}
\usepackage{setspace}
\usepackage{svg}
\usepackage{hyperref}
\usepackage[backend=biber,style=numeric]{biblatex}
\addbibresource{references.bib}
\hypersetup{
    colorlinks=true,
    linkcolor=blue,      % color of internal links (sections, etc.)
    urlcolor=blue,       % color of external links
    pdftitle={Optimización energética de sistema híbrido con bomba de calor, suelo radiante, fotovoltaica y almacenamiento para vivienda},    % title
    pdfauthor={Luis D. Aranda Sánchez},     % author
    pdfkeywords={palabra1, palabra2, código1, etc.} % list of keywords
}

% Font change to Arial
\usepackage{helvet}
\renewcommand{\familydefault}{\sfdefault}

% Chapter titles in uppercase and larger font
\titleformat{\chapter}[hang]{\large\bfseries}{\thechapter.}{1em}{\MakeUppercase}
\titleformat{\section}[hang]{\bfseries}{\thesection.}{1em}{}
\titleformat{\subsection}[hang]{\bfseries}{\thesubsection.}{1em}{}

% Fancyhdr setup
\setlength{\headheight}{14.30174pt} % Adjust to recommended value, slightly larger for safety
\fancyhf{} % Clear all headers and footers
\fancyhead[LE]{\nouppercase{\leftmark}}
\fancyhead[RO]{Optimización energética para vivienda}
\fancyfoot[LE]{\thepage}
\fancyfoot[RE]{Escuela Técnica Superior de Ingenieros Industriales (UPM)}
\fancyfoot[LO]{Luis D. Aranda Sánchez}
\fancyfoot[RO]{\thepage}
\renewcommand{\headrulewidth}{0.4pt}
\renewcommand{\footrulewidth}{0.4pt}

\fancypagestyle{myfancy}{
    \fancyhf{} % Clear all headers and footers
    \fancyhead[LE]{\nouppercase{\leftmark}}
    \fancyhead[RO]{Optimización energética para vivienda}
    \fancyfoot[LE]{\thepage}
    \fancyfoot[RE]{Escuela Técnica Superior de Ingenieros Industriales (UPM)}
    \fancyfoot[LO]{Luis D. Aranda Sánchez}
    \fancyfoot[RO]{\thepage}
    \renewcommand{\headrulewidth}{0.4pt}
    \renewcommand{\footrulewidth}{0.4pt}
}

\fancypagestyle{simple}{
    \fancyhf{} % Clear all headers and footers
    \renewcommand{\headrulewidth}{0pt}
    \renewcommand{\footrulewidth}{0pt}
}

% Line spacing
\setstretch{1.2}

% Document starts here
\begin{document}

% Portada
\begin{titlepage}
    \centering
    {\scshape\LARGE Universidad Politécnica de Madrid \par}
    \vspace{1cm}
    {\scshape\Large Escuela Técnica Superior de Ingenieros Industriales\par}
    \vspace{1.5cm}
    {\huge\bfseries Optimización energética de sistema híbrido con bomba de calor, suelo radiante, fotovoltaica y almacenamiento para vivienda \par}
    \vspace{1.5cm}
    {\Large\bfseries Trabajo de Fin de Máster\par}
    \vspace{0.5cm}
    {\large Máster Universitario en Ingeniería de la Energía \par}
    \vspace{2cm}
    {\Large Luis D. Aranda Sánchez\par}
    \vfill
    Director: Javier Rodríguez Martín
    \vfill
    {\large Septiembre 6, 2024\par}
\end{titlepage}

% Resumen (máximo de 5 páginas, incluyendo al final Palabras clave)
\clearpage
\pagestyle{simple}
% \newpage
\chapter*{Resumen}
\addcontentsline{toc}{chapter}{Resumen}
\input{capitulos/resumen/main.tex}

% Índice (paginado)
\clearpage
\pagestyle{simple}
% \newpage
\tableofcontents

% Introducción (donde se incluya los antecedentes y justificación)
\clearpage
\pagestyle{myfancy}
\newpage
\chapter{Introducción}
\input{capitulos/introduccion/main.tex}

% Objetivos
\chapter{Objetivos}
\input{capitulos/objetivos/main.tex}

% Metodología
\chapter{Metodología}
\input{capitulos/metodologia/main.tex}

% Resultados y discusión (incluyendo la valoración de impactos y de aspectos de responsabilidad legal, ética y profesional relacionados con el trabajo)
\chapter{Resultados y Discusión}
\input{capitulos/resultados_discusion/main.tex}

% Conclusiones
\chapter{Conclusiones}
\input{capitulos/conclusiones/main.tex}

% Planificación temporal y presupuesto
\chapter{Planificación Temporal y Presupuesto}
\input{capitulos/planificacion_presupuesto/main.tex}

% Bibliografía
\newpage
\addcontentsline{toc}{chapter}{Bibliografía}
\printbibliography

\end{document}


% Metodología
\chapter{Metodología}
\documentclass[a4paper,11pt,twoside]{report}
\usepackage[left=25mm,right=25mm,top=25mm,bottom=25mm,includehead,includefoot,headsep=15mm,footskip=15mm]{geometry}
\usepackage{graphicx}
\usepackage{fancyhdr}
\usepackage{titlesec}
\usepackage[spanish]{babel}
\usepackage[utf8]{inputenc}
\usepackage{amsmath}
\usepackage{setspace}
\usepackage{svg}
\usepackage{hyperref}
\usepackage[backend=biber,style=numeric]{biblatex}
\addbibresource{references.bib}
\hypersetup{
    colorlinks=true,
    linkcolor=blue,      % color of internal links (sections, etc.)
    urlcolor=blue,       % color of external links
    pdftitle={Optimización energética de sistema híbrido con bomba de calor, suelo radiante, fotovoltaica y almacenamiento para vivienda},    % title
    pdfauthor={Luis D. Aranda Sánchez},     % author
    pdfkeywords={palabra1, palabra2, código1, etc.} % list of keywords
}

% Font change to Arial
\usepackage{helvet}
\renewcommand{\familydefault}{\sfdefault}

% Chapter titles in uppercase and larger font
\titleformat{\chapter}[hang]{\large\bfseries}{\thechapter.}{1em}{\MakeUppercase}
\titleformat{\section}[hang]{\bfseries}{\thesection.}{1em}{}
\titleformat{\subsection}[hang]{\bfseries}{\thesubsection.}{1em}{}

% Fancyhdr setup
\setlength{\headheight}{14.30174pt} % Adjust to recommended value, slightly larger for safety
\fancyhf{} % Clear all headers and footers
\fancyhead[LE]{\nouppercase{\leftmark}}
\fancyhead[RO]{Optimización energética para vivienda}
\fancyfoot[LE]{\thepage}
\fancyfoot[RE]{Escuela Técnica Superior de Ingenieros Industriales (UPM)}
\fancyfoot[LO]{Luis D. Aranda Sánchez}
\fancyfoot[RO]{\thepage}
\renewcommand{\headrulewidth}{0.4pt}
\renewcommand{\footrulewidth}{0.4pt}

\fancypagestyle{myfancy}{
    \fancyhf{} % Clear all headers and footers
    \fancyhead[LE]{\nouppercase{\leftmark}}
    \fancyhead[RO]{Optimización energética para vivienda}
    \fancyfoot[LE]{\thepage}
    \fancyfoot[RE]{Escuela Técnica Superior de Ingenieros Industriales (UPM)}
    \fancyfoot[LO]{Luis D. Aranda Sánchez}
    \fancyfoot[RO]{\thepage}
    \renewcommand{\headrulewidth}{0.4pt}
    \renewcommand{\footrulewidth}{0.4pt}
}

\fancypagestyle{simple}{
    \fancyhf{} % Clear all headers and footers
    \renewcommand{\headrulewidth}{0pt}
    \renewcommand{\footrulewidth}{0pt}
}

% Line spacing
\setstretch{1.2}

% Document starts here
\begin{document}

% Portada
\begin{titlepage}
    \centering
    {\scshape\LARGE Universidad Politécnica de Madrid \par}
    \vspace{1cm}
    {\scshape\Large Escuela Técnica Superior de Ingenieros Industriales\par}
    \vspace{1.5cm}
    {\huge\bfseries Optimización energética de sistema híbrido con bomba de calor, suelo radiante, fotovoltaica y almacenamiento para vivienda \par}
    \vspace{1.5cm}
    {\Large\bfseries Trabajo de Fin de Máster\par}
    \vspace{0.5cm}
    {\large Máster Universitario en Ingeniería de la Energía \par}
    \vspace{2cm}
    {\Large Luis D. Aranda Sánchez\par}
    \vfill
    Director: Javier Rodríguez Martín
    \vfill
    {\large Septiembre 6, 2024\par}
\end{titlepage}

% Resumen (máximo de 5 páginas, incluyendo al final Palabras clave)
\clearpage
\pagestyle{simple}
% \newpage
\chapter*{Resumen}
\addcontentsline{toc}{chapter}{Resumen}
\input{capitulos/resumen/main.tex}

% Índice (paginado)
\clearpage
\pagestyle{simple}
% \newpage
\tableofcontents

% Introducción (donde se incluya los antecedentes y justificación)
\clearpage
\pagestyle{myfancy}
\newpage
\chapter{Introducción}
\input{capitulos/introduccion/main.tex}

% Objetivos
\chapter{Objetivos}
\input{capitulos/objetivos/main.tex}

% Metodología
\chapter{Metodología}
\input{capitulos/metodologia/main.tex}

% Resultados y discusión (incluyendo la valoración de impactos y de aspectos de responsabilidad legal, ética y profesional relacionados con el trabajo)
\chapter{Resultados y Discusión}
\input{capitulos/resultados_discusion/main.tex}

% Conclusiones
\chapter{Conclusiones}
\input{capitulos/conclusiones/main.tex}

% Planificación temporal y presupuesto
\chapter{Planificación Temporal y Presupuesto}
\input{capitulos/planificacion_presupuesto/main.tex}

% Bibliografía
\newpage
\addcontentsline{toc}{chapter}{Bibliografía}
\printbibliography

\end{document}


% Resultados y discusión (incluyendo la valoración de impactos y de aspectos de responsabilidad legal, ética y profesional relacionados con el trabajo)
\chapter{Resultados y Discusión}
\documentclass[a4paper,11pt,twoside]{report}
\usepackage[left=25mm,right=25mm,top=25mm,bottom=25mm,includehead,includefoot,headsep=15mm,footskip=15mm]{geometry}
\usepackage{graphicx}
\usepackage{fancyhdr}
\usepackage{titlesec}
\usepackage[spanish]{babel}
\usepackage[utf8]{inputenc}
\usepackage{amsmath}
\usepackage{setspace}
\usepackage{svg}
\usepackage{hyperref}
\usepackage[backend=biber,style=numeric]{biblatex}
\addbibresource{references.bib}
\hypersetup{
    colorlinks=true,
    linkcolor=blue,      % color of internal links (sections, etc.)
    urlcolor=blue,       % color of external links
    pdftitle={Optimización energética de sistema híbrido con bomba de calor, suelo radiante, fotovoltaica y almacenamiento para vivienda},    % title
    pdfauthor={Luis D. Aranda Sánchez},     % author
    pdfkeywords={palabra1, palabra2, código1, etc.} % list of keywords
}

% Font change to Arial
\usepackage{helvet}
\renewcommand{\familydefault}{\sfdefault}

% Chapter titles in uppercase and larger font
\titleformat{\chapter}[hang]{\large\bfseries}{\thechapter.}{1em}{\MakeUppercase}
\titleformat{\section}[hang]{\bfseries}{\thesection.}{1em}{}
\titleformat{\subsection}[hang]{\bfseries}{\thesubsection.}{1em}{}

% Fancyhdr setup
\setlength{\headheight}{14.30174pt} % Adjust to recommended value, slightly larger for safety
\fancyhf{} % Clear all headers and footers
\fancyhead[LE]{\nouppercase{\leftmark}}
\fancyhead[RO]{Optimización energética para vivienda}
\fancyfoot[LE]{\thepage}
\fancyfoot[RE]{Escuela Técnica Superior de Ingenieros Industriales (UPM)}
\fancyfoot[LO]{Luis D. Aranda Sánchez}
\fancyfoot[RO]{\thepage}
\renewcommand{\headrulewidth}{0.4pt}
\renewcommand{\footrulewidth}{0.4pt}

\fancypagestyle{myfancy}{
    \fancyhf{} % Clear all headers and footers
    \fancyhead[LE]{\nouppercase{\leftmark}}
    \fancyhead[RO]{Optimización energética para vivienda}
    \fancyfoot[LE]{\thepage}
    \fancyfoot[RE]{Escuela Técnica Superior de Ingenieros Industriales (UPM)}
    \fancyfoot[LO]{Luis D. Aranda Sánchez}
    \fancyfoot[RO]{\thepage}
    \renewcommand{\headrulewidth}{0.4pt}
    \renewcommand{\footrulewidth}{0.4pt}
}

\fancypagestyle{simple}{
    \fancyhf{} % Clear all headers and footers
    \renewcommand{\headrulewidth}{0pt}
    \renewcommand{\footrulewidth}{0pt}
}

% Line spacing
\setstretch{1.2}

% Document starts here
\begin{document}

% Portada
\begin{titlepage}
    \centering
    {\scshape\LARGE Universidad Politécnica de Madrid \par}
    \vspace{1cm}
    {\scshape\Large Escuela Técnica Superior de Ingenieros Industriales\par}
    \vspace{1.5cm}
    {\huge\bfseries Optimización energética de sistema híbrido con bomba de calor, suelo radiante, fotovoltaica y almacenamiento para vivienda \par}
    \vspace{1.5cm}
    {\Large\bfseries Trabajo de Fin de Máster\par}
    \vspace{0.5cm}
    {\large Máster Universitario en Ingeniería de la Energía \par}
    \vspace{2cm}
    {\Large Luis D. Aranda Sánchez\par}
    \vfill
    Director: Javier Rodríguez Martín
    \vfill
    {\large Septiembre 6, 2024\par}
\end{titlepage}

% Resumen (máximo de 5 páginas, incluyendo al final Palabras clave)
\clearpage
\pagestyle{simple}
% \newpage
\chapter*{Resumen}
\addcontentsline{toc}{chapter}{Resumen}
\input{capitulos/resumen/main.tex}

% Índice (paginado)
\clearpage
\pagestyle{simple}
% \newpage
\tableofcontents

% Introducción (donde se incluya los antecedentes y justificación)
\clearpage
\pagestyle{myfancy}
\newpage
\chapter{Introducción}
\input{capitulos/introduccion/main.tex}

% Objetivos
\chapter{Objetivos}
\input{capitulos/objetivos/main.tex}

% Metodología
\chapter{Metodología}
\input{capitulos/metodologia/main.tex}

% Resultados y discusión (incluyendo la valoración de impactos y de aspectos de responsabilidad legal, ética y profesional relacionados con el trabajo)
\chapter{Resultados y Discusión}
\input{capitulos/resultados_discusion/main.tex}

% Conclusiones
\chapter{Conclusiones}
\input{capitulos/conclusiones/main.tex}

% Planificación temporal y presupuesto
\chapter{Planificación Temporal y Presupuesto}
\input{capitulos/planificacion_presupuesto/main.tex}

% Bibliografía
\newpage
\addcontentsline{toc}{chapter}{Bibliografía}
\printbibliography

\end{document}


% Conclusiones
\chapter{Conclusiones}
\documentclass[a4paper,11pt,twoside]{report}
\usepackage[left=25mm,right=25mm,top=25mm,bottom=25mm,includehead,includefoot,headsep=15mm,footskip=15mm]{geometry}
\usepackage{graphicx}
\usepackage{fancyhdr}
\usepackage{titlesec}
\usepackage[spanish]{babel}
\usepackage[utf8]{inputenc}
\usepackage{amsmath}
\usepackage{setspace}
\usepackage{svg}
\usepackage{hyperref}
\usepackage[backend=biber,style=numeric]{biblatex}
\addbibresource{references.bib}
\hypersetup{
    colorlinks=true,
    linkcolor=blue,      % color of internal links (sections, etc.)
    urlcolor=blue,       % color of external links
    pdftitle={Optimización energética de sistema híbrido con bomba de calor, suelo radiante, fotovoltaica y almacenamiento para vivienda},    % title
    pdfauthor={Luis D. Aranda Sánchez},     % author
    pdfkeywords={palabra1, palabra2, código1, etc.} % list of keywords
}

% Font change to Arial
\usepackage{helvet}
\renewcommand{\familydefault}{\sfdefault}

% Chapter titles in uppercase and larger font
\titleformat{\chapter}[hang]{\large\bfseries}{\thechapter.}{1em}{\MakeUppercase}
\titleformat{\section}[hang]{\bfseries}{\thesection.}{1em}{}
\titleformat{\subsection}[hang]{\bfseries}{\thesubsection.}{1em}{}

% Fancyhdr setup
\setlength{\headheight}{14.30174pt} % Adjust to recommended value, slightly larger for safety
\fancyhf{} % Clear all headers and footers
\fancyhead[LE]{\nouppercase{\leftmark}}
\fancyhead[RO]{Optimización energética para vivienda}
\fancyfoot[LE]{\thepage}
\fancyfoot[RE]{Escuela Técnica Superior de Ingenieros Industriales (UPM)}
\fancyfoot[LO]{Luis D. Aranda Sánchez}
\fancyfoot[RO]{\thepage}
\renewcommand{\headrulewidth}{0.4pt}
\renewcommand{\footrulewidth}{0.4pt}

\fancypagestyle{myfancy}{
    \fancyhf{} % Clear all headers and footers
    \fancyhead[LE]{\nouppercase{\leftmark}}
    \fancyhead[RO]{Optimización energética para vivienda}
    \fancyfoot[LE]{\thepage}
    \fancyfoot[RE]{Escuela Técnica Superior de Ingenieros Industriales (UPM)}
    \fancyfoot[LO]{Luis D. Aranda Sánchez}
    \fancyfoot[RO]{\thepage}
    \renewcommand{\headrulewidth}{0.4pt}
    \renewcommand{\footrulewidth}{0.4pt}
}

\fancypagestyle{simple}{
    \fancyhf{} % Clear all headers and footers
    \renewcommand{\headrulewidth}{0pt}
    \renewcommand{\footrulewidth}{0pt}
}

% Line spacing
\setstretch{1.2}

% Document starts here
\begin{document}

% Portada
\begin{titlepage}
    \centering
    {\scshape\LARGE Universidad Politécnica de Madrid \par}
    \vspace{1cm}
    {\scshape\Large Escuela Técnica Superior de Ingenieros Industriales\par}
    \vspace{1.5cm}
    {\huge\bfseries Optimización energética de sistema híbrido con bomba de calor, suelo radiante, fotovoltaica y almacenamiento para vivienda \par}
    \vspace{1.5cm}
    {\Large\bfseries Trabajo de Fin de Máster\par}
    \vspace{0.5cm}
    {\large Máster Universitario en Ingeniería de la Energía \par}
    \vspace{2cm}
    {\Large Luis D. Aranda Sánchez\par}
    \vfill
    Director: Javier Rodríguez Martín
    \vfill
    {\large Septiembre 6, 2024\par}
\end{titlepage}

% Resumen (máximo de 5 páginas, incluyendo al final Palabras clave)
\clearpage
\pagestyle{simple}
% \newpage
\chapter*{Resumen}
\addcontentsline{toc}{chapter}{Resumen}
\input{capitulos/resumen/main.tex}

% Índice (paginado)
\clearpage
\pagestyle{simple}
% \newpage
\tableofcontents

% Introducción (donde se incluya los antecedentes y justificación)
\clearpage
\pagestyle{myfancy}
\newpage
\chapter{Introducción}
\input{capitulos/introduccion/main.tex}

% Objetivos
\chapter{Objetivos}
\input{capitulos/objetivos/main.tex}

% Metodología
\chapter{Metodología}
\input{capitulos/metodologia/main.tex}

% Resultados y discusión (incluyendo la valoración de impactos y de aspectos de responsabilidad legal, ética y profesional relacionados con el trabajo)
\chapter{Resultados y Discusión}
\input{capitulos/resultados_discusion/main.tex}

% Conclusiones
\chapter{Conclusiones}
\input{capitulos/conclusiones/main.tex}

% Planificación temporal y presupuesto
\chapter{Planificación Temporal y Presupuesto}
\input{capitulos/planificacion_presupuesto/main.tex}

% Bibliografía
\newpage
\addcontentsline{toc}{chapter}{Bibliografía}
\printbibliography

\end{document}


% Planificación temporal y presupuesto
\chapter{Planificación Temporal y Presupuesto}
\documentclass[a4paper,11pt,twoside]{report}
\usepackage[left=25mm,right=25mm,top=25mm,bottom=25mm,includehead,includefoot,headsep=15mm,footskip=15mm]{geometry}
\usepackage{graphicx}
\usepackage{fancyhdr}
\usepackage{titlesec}
\usepackage[spanish]{babel}
\usepackage[utf8]{inputenc}
\usepackage{amsmath}
\usepackage{setspace}
\usepackage{svg}
\usepackage{hyperref}
\usepackage[backend=biber,style=numeric]{biblatex}
\addbibresource{references.bib}
\hypersetup{
    colorlinks=true,
    linkcolor=blue,      % color of internal links (sections, etc.)
    urlcolor=blue,       % color of external links
    pdftitle={Optimización energética de sistema híbrido con bomba de calor, suelo radiante, fotovoltaica y almacenamiento para vivienda},    % title
    pdfauthor={Luis D. Aranda Sánchez},     % author
    pdfkeywords={palabra1, palabra2, código1, etc.} % list of keywords
}

% Font change to Arial
\usepackage{helvet}
\renewcommand{\familydefault}{\sfdefault}

% Chapter titles in uppercase and larger font
\titleformat{\chapter}[hang]{\large\bfseries}{\thechapter.}{1em}{\MakeUppercase}
\titleformat{\section}[hang]{\bfseries}{\thesection.}{1em}{}
\titleformat{\subsection}[hang]{\bfseries}{\thesubsection.}{1em}{}

% Fancyhdr setup
\setlength{\headheight}{14.30174pt} % Adjust to recommended value, slightly larger for safety
\fancyhf{} % Clear all headers and footers
\fancyhead[LE]{\nouppercase{\leftmark}}
\fancyhead[RO]{Optimización energética para vivienda}
\fancyfoot[LE]{\thepage}
\fancyfoot[RE]{Escuela Técnica Superior de Ingenieros Industriales (UPM)}
\fancyfoot[LO]{Luis D. Aranda Sánchez}
\fancyfoot[RO]{\thepage}
\renewcommand{\headrulewidth}{0.4pt}
\renewcommand{\footrulewidth}{0.4pt}

\fancypagestyle{myfancy}{
    \fancyhf{} % Clear all headers and footers
    \fancyhead[LE]{\nouppercase{\leftmark}}
    \fancyhead[RO]{Optimización energética para vivienda}
    \fancyfoot[LE]{\thepage}
    \fancyfoot[RE]{Escuela Técnica Superior de Ingenieros Industriales (UPM)}
    \fancyfoot[LO]{Luis D. Aranda Sánchez}
    \fancyfoot[RO]{\thepage}
    \renewcommand{\headrulewidth}{0.4pt}
    \renewcommand{\footrulewidth}{0.4pt}
}

\fancypagestyle{simple}{
    \fancyhf{} % Clear all headers and footers
    \renewcommand{\headrulewidth}{0pt}
    \renewcommand{\footrulewidth}{0pt}
}

% Line spacing
\setstretch{1.2}

% Document starts here
\begin{document}

% Portada
\begin{titlepage}
    \centering
    {\scshape\LARGE Universidad Politécnica de Madrid \par}
    \vspace{1cm}
    {\scshape\Large Escuela Técnica Superior de Ingenieros Industriales\par}
    \vspace{1.5cm}
    {\huge\bfseries Optimización energética de sistema híbrido con bomba de calor, suelo radiante, fotovoltaica y almacenamiento para vivienda \par}
    \vspace{1.5cm}
    {\Large\bfseries Trabajo de Fin de Máster\par}
    \vspace{0.5cm}
    {\large Máster Universitario en Ingeniería de la Energía \par}
    \vspace{2cm}
    {\Large Luis D. Aranda Sánchez\par}
    \vfill
    Director: Javier Rodríguez Martín
    \vfill
    {\large Septiembre 6, 2024\par}
\end{titlepage}

% Resumen (máximo de 5 páginas, incluyendo al final Palabras clave)
\clearpage
\pagestyle{simple}
% \newpage
\chapter*{Resumen}
\addcontentsline{toc}{chapter}{Resumen}
\input{capitulos/resumen/main.tex}

% Índice (paginado)
\clearpage
\pagestyle{simple}
% \newpage
\tableofcontents

% Introducción (donde se incluya los antecedentes y justificación)
\clearpage
\pagestyle{myfancy}
\newpage
\chapter{Introducción}
\input{capitulos/introduccion/main.tex}

% Objetivos
\chapter{Objetivos}
\input{capitulos/objetivos/main.tex}

% Metodología
\chapter{Metodología}
\input{capitulos/metodologia/main.tex}

% Resultados y discusión (incluyendo la valoración de impactos y de aspectos de responsabilidad legal, ética y profesional relacionados con el trabajo)
\chapter{Resultados y Discusión}
\input{capitulos/resultados_discusion/main.tex}

% Conclusiones
\chapter{Conclusiones}
\input{capitulos/conclusiones/main.tex}

% Planificación temporal y presupuesto
\chapter{Planificación Temporal y Presupuesto}
\input{capitulos/planificacion_presupuesto/main.tex}

% Bibliografía
\newpage
\addcontentsline{toc}{chapter}{Bibliografía}
\printbibliography

\end{document}


% Bibliografía
\newpage
\addcontentsline{toc}{chapter}{Bibliografía}
\printbibliography

\end{document}


% Índice (paginado)
\clearpage
\pagestyle{simple}
% \newpage
\tableofcontents

% Introducción (donde se incluya los antecedentes y justificación)
\clearpage
\pagestyle{myfancy}
\newpage
\chapter{Introducción}
\documentclass[a4paper,11pt,twoside]{report}
\usepackage[left=25mm,right=25mm,top=25mm,bottom=25mm,includehead,includefoot,headsep=15mm,footskip=15mm]{geometry}
\usepackage{graphicx}
\usepackage{fancyhdr}
\usepackage{titlesec}
\usepackage[spanish]{babel}
\usepackage[utf8]{inputenc}
\usepackage{amsmath}
\usepackage{setspace}
\usepackage{svg}
\usepackage{hyperref}
\usepackage[backend=biber,style=numeric]{biblatex}
\addbibresource{references.bib}
\hypersetup{
    colorlinks=true,
    linkcolor=blue,      % color of internal links (sections, etc.)
    urlcolor=blue,       % color of external links
    pdftitle={Optimización energética de sistema híbrido con bomba de calor, suelo radiante, fotovoltaica y almacenamiento para vivienda},    % title
    pdfauthor={Luis D. Aranda Sánchez},     % author
    pdfkeywords={palabra1, palabra2, código1, etc.} % list of keywords
}

% Font change to Arial
\usepackage{helvet}
\renewcommand{\familydefault}{\sfdefault}

% Chapter titles in uppercase and larger font
\titleformat{\chapter}[hang]{\large\bfseries}{\thechapter.}{1em}{\MakeUppercase}
\titleformat{\section}[hang]{\bfseries}{\thesection.}{1em}{}
\titleformat{\subsection}[hang]{\bfseries}{\thesubsection.}{1em}{}

% Fancyhdr setup
\setlength{\headheight}{14.30174pt} % Adjust to recommended value, slightly larger for safety
\fancyhf{} % Clear all headers and footers
\fancyhead[LE]{\nouppercase{\leftmark}}
\fancyhead[RO]{Optimización energética para vivienda}
\fancyfoot[LE]{\thepage}
\fancyfoot[RE]{Escuela Técnica Superior de Ingenieros Industriales (UPM)}
\fancyfoot[LO]{Luis D. Aranda Sánchez}
\fancyfoot[RO]{\thepage}
\renewcommand{\headrulewidth}{0.4pt}
\renewcommand{\footrulewidth}{0.4pt}

\fancypagestyle{myfancy}{
    \fancyhf{} % Clear all headers and footers
    \fancyhead[LE]{\nouppercase{\leftmark}}
    \fancyhead[RO]{Optimización energética para vivienda}
    \fancyfoot[LE]{\thepage}
    \fancyfoot[RE]{Escuela Técnica Superior de Ingenieros Industriales (UPM)}
    \fancyfoot[LO]{Luis D. Aranda Sánchez}
    \fancyfoot[RO]{\thepage}
    \renewcommand{\headrulewidth}{0.4pt}
    \renewcommand{\footrulewidth}{0.4pt}
}

\fancypagestyle{simple}{
    \fancyhf{} % Clear all headers and footers
    \renewcommand{\headrulewidth}{0pt}
    \renewcommand{\footrulewidth}{0pt}
}

% Line spacing
\setstretch{1.2}

% Document starts here
\begin{document}

% Portada
\begin{titlepage}
    \centering
    {\scshape\LARGE Universidad Politécnica de Madrid \par}
    \vspace{1cm}
    {\scshape\Large Escuela Técnica Superior de Ingenieros Industriales\par}
    \vspace{1.5cm}
    {\huge\bfseries Optimización energética de sistema híbrido con bomba de calor, suelo radiante, fotovoltaica y almacenamiento para vivienda \par}
    \vspace{1.5cm}
    {\Large\bfseries Trabajo de Fin de Máster\par}
    \vspace{0.5cm}
    {\large Máster Universitario en Ingeniería de la Energía \par}
    \vspace{2cm}
    {\Large Luis D. Aranda Sánchez\par}
    \vfill
    Director: Javier Rodríguez Martín
    \vfill
    {\large Septiembre 6, 2024\par}
\end{titlepage}

% Resumen (máximo de 5 páginas, incluyendo al final Palabras clave)
\clearpage
\pagestyle{simple}
% \newpage
\chapter*{Resumen}
\addcontentsline{toc}{chapter}{Resumen}
\documentclass[a4paper,11pt,twoside]{report}
\usepackage[left=25mm,right=25mm,top=25mm,bottom=25mm,includehead,includefoot,headsep=15mm,footskip=15mm]{geometry}
\usepackage{graphicx}
\usepackage{fancyhdr}
\usepackage{titlesec}
\usepackage[spanish]{babel}
\usepackage[utf8]{inputenc}
\usepackage{amsmath}
\usepackage{setspace}
\usepackage{svg}
\usepackage{hyperref}
\usepackage[backend=biber,style=numeric]{biblatex}
\addbibresource{references.bib}
\hypersetup{
    colorlinks=true,
    linkcolor=blue,      % color of internal links (sections, etc.)
    urlcolor=blue,       % color of external links
    pdftitle={Optimización energética de sistema híbrido con bomba de calor, suelo radiante, fotovoltaica y almacenamiento para vivienda},    % title
    pdfauthor={Luis D. Aranda Sánchez},     % author
    pdfkeywords={palabra1, palabra2, código1, etc.} % list of keywords
}

% Font change to Arial
\usepackage{helvet}
\renewcommand{\familydefault}{\sfdefault}

% Chapter titles in uppercase and larger font
\titleformat{\chapter}[hang]{\large\bfseries}{\thechapter.}{1em}{\MakeUppercase}
\titleformat{\section}[hang]{\bfseries}{\thesection.}{1em}{}
\titleformat{\subsection}[hang]{\bfseries}{\thesubsection.}{1em}{}

% Fancyhdr setup
\setlength{\headheight}{14.30174pt} % Adjust to recommended value, slightly larger for safety
\fancyhf{} % Clear all headers and footers
\fancyhead[LE]{\nouppercase{\leftmark}}
\fancyhead[RO]{Optimización energética para vivienda}
\fancyfoot[LE]{\thepage}
\fancyfoot[RE]{Escuela Técnica Superior de Ingenieros Industriales (UPM)}
\fancyfoot[LO]{Luis D. Aranda Sánchez}
\fancyfoot[RO]{\thepage}
\renewcommand{\headrulewidth}{0.4pt}
\renewcommand{\footrulewidth}{0.4pt}

\fancypagestyle{myfancy}{
    \fancyhf{} % Clear all headers and footers
    \fancyhead[LE]{\nouppercase{\leftmark}}
    \fancyhead[RO]{Optimización energética para vivienda}
    \fancyfoot[LE]{\thepage}
    \fancyfoot[RE]{Escuela Técnica Superior de Ingenieros Industriales (UPM)}
    \fancyfoot[LO]{Luis D. Aranda Sánchez}
    \fancyfoot[RO]{\thepage}
    \renewcommand{\headrulewidth}{0.4pt}
    \renewcommand{\footrulewidth}{0.4pt}
}

\fancypagestyle{simple}{
    \fancyhf{} % Clear all headers and footers
    \renewcommand{\headrulewidth}{0pt}
    \renewcommand{\footrulewidth}{0pt}
}

% Line spacing
\setstretch{1.2}

% Document starts here
\begin{document}

% Portada
\begin{titlepage}
    \centering
    {\scshape\LARGE Universidad Politécnica de Madrid \par}
    \vspace{1cm}
    {\scshape\Large Escuela Técnica Superior de Ingenieros Industriales\par}
    \vspace{1.5cm}
    {\huge\bfseries Optimización energética de sistema híbrido con bomba de calor, suelo radiante, fotovoltaica y almacenamiento para vivienda \par}
    \vspace{1.5cm}
    {\Large\bfseries Trabajo de Fin de Máster\par}
    \vspace{0.5cm}
    {\large Máster Universitario en Ingeniería de la Energía \par}
    \vspace{2cm}
    {\Large Luis D. Aranda Sánchez\par}
    \vfill
    Director: Javier Rodríguez Martín
    \vfill
    {\large Septiembre 6, 2024\par}
\end{titlepage}

% Resumen (máximo de 5 páginas, incluyendo al final Palabras clave)
\clearpage
\pagestyle{simple}
% \newpage
\chapter*{Resumen}
\addcontentsline{toc}{chapter}{Resumen}
\input{capitulos/resumen/main.tex}

% Índice (paginado)
\clearpage
\pagestyle{simple}
% \newpage
\tableofcontents

% Introducción (donde se incluya los antecedentes y justificación)
\clearpage
\pagestyle{myfancy}
\newpage
\chapter{Introducción}
\input{capitulos/introduccion/main.tex}

% Objetivos
\chapter{Objetivos}
\input{capitulos/objetivos/main.tex}

% Metodología
\chapter{Metodología}
\input{capitulos/metodologia/main.tex}

% Resultados y discusión (incluyendo la valoración de impactos y de aspectos de responsabilidad legal, ética y profesional relacionados con el trabajo)
\chapter{Resultados y Discusión}
\input{capitulos/resultados_discusion/main.tex}

% Conclusiones
\chapter{Conclusiones}
\input{capitulos/conclusiones/main.tex}

% Planificación temporal y presupuesto
\chapter{Planificación Temporal y Presupuesto}
\input{capitulos/planificacion_presupuesto/main.tex}

% Bibliografía
\newpage
\addcontentsline{toc}{chapter}{Bibliografía}
\printbibliography

\end{document}


% Índice (paginado)
\clearpage
\pagestyle{simple}
% \newpage
\tableofcontents

% Introducción (donde se incluya los antecedentes y justificación)
\clearpage
\pagestyle{myfancy}
\newpage
\chapter{Introducción}
\documentclass[a4paper,11pt,twoside]{report}
\usepackage[left=25mm,right=25mm,top=25mm,bottom=25mm,includehead,includefoot,headsep=15mm,footskip=15mm]{geometry}
\usepackage{graphicx}
\usepackage{fancyhdr}
\usepackage{titlesec}
\usepackage[spanish]{babel}
\usepackage[utf8]{inputenc}
\usepackage{amsmath}
\usepackage{setspace}
\usepackage{svg}
\usepackage{hyperref}
\usepackage[backend=biber,style=numeric]{biblatex}
\addbibresource{references.bib}
\hypersetup{
    colorlinks=true,
    linkcolor=blue,      % color of internal links (sections, etc.)
    urlcolor=blue,       % color of external links
    pdftitle={Optimización energética de sistema híbrido con bomba de calor, suelo radiante, fotovoltaica y almacenamiento para vivienda},    % title
    pdfauthor={Luis D. Aranda Sánchez},     % author
    pdfkeywords={palabra1, palabra2, código1, etc.} % list of keywords
}

% Font change to Arial
\usepackage{helvet}
\renewcommand{\familydefault}{\sfdefault}

% Chapter titles in uppercase and larger font
\titleformat{\chapter}[hang]{\large\bfseries}{\thechapter.}{1em}{\MakeUppercase}
\titleformat{\section}[hang]{\bfseries}{\thesection.}{1em}{}
\titleformat{\subsection}[hang]{\bfseries}{\thesubsection.}{1em}{}

% Fancyhdr setup
\setlength{\headheight}{14.30174pt} % Adjust to recommended value, slightly larger for safety
\fancyhf{} % Clear all headers and footers
\fancyhead[LE]{\nouppercase{\leftmark}}
\fancyhead[RO]{Optimización energética para vivienda}
\fancyfoot[LE]{\thepage}
\fancyfoot[RE]{Escuela Técnica Superior de Ingenieros Industriales (UPM)}
\fancyfoot[LO]{Luis D. Aranda Sánchez}
\fancyfoot[RO]{\thepage}
\renewcommand{\headrulewidth}{0.4pt}
\renewcommand{\footrulewidth}{0.4pt}

\fancypagestyle{myfancy}{
    \fancyhf{} % Clear all headers and footers
    \fancyhead[LE]{\nouppercase{\leftmark}}
    \fancyhead[RO]{Optimización energética para vivienda}
    \fancyfoot[LE]{\thepage}
    \fancyfoot[RE]{Escuela Técnica Superior de Ingenieros Industriales (UPM)}
    \fancyfoot[LO]{Luis D. Aranda Sánchez}
    \fancyfoot[RO]{\thepage}
    \renewcommand{\headrulewidth}{0.4pt}
    \renewcommand{\footrulewidth}{0.4pt}
}

\fancypagestyle{simple}{
    \fancyhf{} % Clear all headers and footers
    \renewcommand{\headrulewidth}{0pt}
    \renewcommand{\footrulewidth}{0pt}
}

% Line spacing
\setstretch{1.2}

% Document starts here
\begin{document}

% Portada
\begin{titlepage}
    \centering
    {\scshape\LARGE Universidad Politécnica de Madrid \par}
    \vspace{1cm}
    {\scshape\Large Escuela Técnica Superior de Ingenieros Industriales\par}
    \vspace{1.5cm}
    {\huge\bfseries Optimización energética de sistema híbrido con bomba de calor, suelo radiante, fotovoltaica y almacenamiento para vivienda \par}
    \vspace{1.5cm}
    {\Large\bfseries Trabajo de Fin de Máster\par}
    \vspace{0.5cm}
    {\large Máster Universitario en Ingeniería de la Energía \par}
    \vspace{2cm}
    {\Large Luis D. Aranda Sánchez\par}
    \vfill
    Director: Javier Rodríguez Martín
    \vfill
    {\large Septiembre 6, 2024\par}
\end{titlepage}

% Resumen (máximo de 5 páginas, incluyendo al final Palabras clave)
\clearpage
\pagestyle{simple}
% \newpage
\chapter*{Resumen}
\addcontentsline{toc}{chapter}{Resumen}
\input{capitulos/resumen/main.tex}

% Índice (paginado)
\clearpage
\pagestyle{simple}
% \newpage
\tableofcontents

% Introducción (donde se incluya los antecedentes y justificación)
\clearpage
\pagestyle{myfancy}
\newpage
\chapter{Introducción}
\input{capitulos/introduccion/main.tex}

% Objetivos
\chapter{Objetivos}
\input{capitulos/objetivos/main.tex}

% Metodología
\chapter{Metodología}
\input{capitulos/metodologia/main.tex}

% Resultados y discusión (incluyendo la valoración de impactos y de aspectos de responsabilidad legal, ética y profesional relacionados con el trabajo)
\chapter{Resultados y Discusión}
\input{capitulos/resultados_discusion/main.tex}

% Conclusiones
\chapter{Conclusiones}
\input{capitulos/conclusiones/main.tex}

% Planificación temporal y presupuesto
\chapter{Planificación Temporal y Presupuesto}
\input{capitulos/planificacion_presupuesto/main.tex}

% Bibliografía
\newpage
\addcontentsline{toc}{chapter}{Bibliografía}
\printbibliography

\end{document}


% Objetivos
\chapter{Objetivos}
\documentclass[a4paper,11pt,twoside]{report}
\usepackage[left=25mm,right=25mm,top=25mm,bottom=25mm,includehead,includefoot,headsep=15mm,footskip=15mm]{geometry}
\usepackage{graphicx}
\usepackage{fancyhdr}
\usepackage{titlesec}
\usepackage[spanish]{babel}
\usepackage[utf8]{inputenc}
\usepackage{amsmath}
\usepackage{setspace}
\usepackage{svg}
\usepackage{hyperref}
\usepackage[backend=biber,style=numeric]{biblatex}
\addbibresource{references.bib}
\hypersetup{
    colorlinks=true,
    linkcolor=blue,      % color of internal links (sections, etc.)
    urlcolor=blue,       % color of external links
    pdftitle={Optimización energética de sistema híbrido con bomba de calor, suelo radiante, fotovoltaica y almacenamiento para vivienda},    % title
    pdfauthor={Luis D. Aranda Sánchez},     % author
    pdfkeywords={palabra1, palabra2, código1, etc.} % list of keywords
}

% Font change to Arial
\usepackage{helvet}
\renewcommand{\familydefault}{\sfdefault}

% Chapter titles in uppercase and larger font
\titleformat{\chapter}[hang]{\large\bfseries}{\thechapter.}{1em}{\MakeUppercase}
\titleformat{\section}[hang]{\bfseries}{\thesection.}{1em}{}
\titleformat{\subsection}[hang]{\bfseries}{\thesubsection.}{1em}{}

% Fancyhdr setup
\setlength{\headheight}{14.30174pt} % Adjust to recommended value, slightly larger for safety
\fancyhf{} % Clear all headers and footers
\fancyhead[LE]{\nouppercase{\leftmark}}
\fancyhead[RO]{Optimización energética para vivienda}
\fancyfoot[LE]{\thepage}
\fancyfoot[RE]{Escuela Técnica Superior de Ingenieros Industriales (UPM)}
\fancyfoot[LO]{Luis D. Aranda Sánchez}
\fancyfoot[RO]{\thepage}
\renewcommand{\headrulewidth}{0.4pt}
\renewcommand{\footrulewidth}{0.4pt}

\fancypagestyle{myfancy}{
    \fancyhf{} % Clear all headers and footers
    \fancyhead[LE]{\nouppercase{\leftmark}}
    \fancyhead[RO]{Optimización energética para vivienda}
    \fancyfoot[LE]{\thepage}
    \fancyfoot[RE]{Escuela Técnica Superior de Ingenieros Industriales (UPM)}
    \fancyfoot[LO]{Luis D. Aranda Sánchez}
    \fancyfoot[RO]{\thepage}
    \renewcommand{\headrulewidth}{0.4pt}
    \renewcommand{\footrulewidth}{0.4pt}
}

\fancypagestyle{simple}{
    \fancyhf{} % Clear all headers and footers
    \renewcommand{\headrulewidth}{0pt}
    \renewcommand{\footrulewidth}{0pt}
}

% Line spacing
\setstretch{1.2}

% Document starts here
\begin{document}

% Portada
\begin{titlepage}
    \centering
    {\scshape\LARGE Universidad Politécnica de Madrid \par}
    \vspace{1cm}
    {\scshape\Large Escuela Técnica Superior de Ingenieros Industriales\par}
    \vspace{1.5cm}
    {\huge\bfseries Optimización energética de sistema híbrido con bomba de calor, suelo radiante, fotovoltaica y almacenamiento para vivienda \par}
    \vspace{1.5cm}
    {\Large\bfseries Trabajo de Fin de Máster\par}
    \vspace{0.5cm}
    {\large Máster Universitario en Ingeniería de la Energía \par}
    \vspace{2cm}
    {\Large Luis D. Aranda Sánchez\par}
    \vfill
    Director: Javier Rodríguez Martín
    \vfill
    {\large Septiembre 6, 2024\par}
\end{titlepage}

% Resumen (máximo de 5 páginas, incluyendo al final Palabras clave)
\clearpage
\pagestyle{simple}
% \newpage
\chapter*{Resumen}
\addcontentsline{toc}{chapter}{Resumen}
\input{capitulos/resumen/main.tex}

% Índice (paginado)
\clearpage
\pagestyle{simple}
% \newpage
\tableofcontents

% Introducción (donde se incluya los antecedentes y justificación)
\clearpage
\pagestyle{myfancy}
\newpage
\chapter{Introducción}
\input{capitulos/introduccion/main.tex}

% Objetivos
\chapter{Objetivos}
\input{capitulos/objetivos/main.tex}

% Metodología
\chapter{Metodología}
\input{capitulos/metodologia/main.tex}

% Resultados y discusión (incluyendo la valoración de impactos y de aspectos de responsabilidad legal, ética y profesional relacionados con el trabajo)
\chapter{Resultados y Discusión}
\input{capitulos/resultados_discusion/main.tex}

% Conclusiones
\chapter{Conclusiones}
\input{capitulos/conclusiones/main.tex}

% Planificación temporal y presupuesto
\chapter{Planificación Temporal y Presupuesto}
\input{capitulos/planificacion_presupuesto/main.tex}

% Bibliografía
\newpage
\addcontentsline{toc}{chapter}{Bibliografía}
\printbibliography

\end{document}


% Metodología
\chapter{Metodología}
\documentclass[a4paper,11pt,twoside]{report}
\usepackage[left=25mm,right=25mm,top=25mm,bottom=25mm,includehead,includefoot,headsep=15mm,footskip=15mm]{geometry}
\usepackage{graphicx}
\usepackage{fancyhdr}
\usepackage{titlesec}
\usepackage[spanish]{babel}
\usepackage[utf8]{inputenc}
\usepackage{amsmath}
\usepackage{setspace}
\usepackage{svg}
\usepackage{hyperref}
\usepackage[backend=biber,style=numeric]{biblatex}
\addbibresource{references.bib}
\hypersetup{
    colorlinks=true,
    linkcolor=blue,      % color of internal links (sections, etc.)
    urlcolor=blue,       % color of external links
    pdftitle={Optimización energética de sistema híbrido con bomba de calor, suelo radiante, fotovoltaica y almacenamiento para vivienda},    % title
    pdfauthor={Luis D. Aranda Sánchez},     % author
    pdfkeywords={palabra1, palabra2, código1, etc.} % list of keywords
}

% Font change to Arial
\usepackage{helvet}
\renewcommand{\familydefault}{\sfdefault}

% Chapter titles in uppercase and larger font
\titleformat{\chapter}[hang]{\large\bfseries}{\thechapter.}{1em}{\MakeUppercase}
\titleformat{\section}[hang]{\bfseries}{\thesection.}{1em}{}
\titleformat{\subsection}[hang]{\bfseries}{\thesubsection.}{1em}{}

% Fancyhdr setup
\setlength{\headheight}{14.30174pt} % Adjust to recommended value, slightly larger for safety
\fancyhf{} % Clear all headers and footers
\fancyhead[LE]{\nouppercase{\leftmark}}
\fancyhead[RO]{Optimización energética para vivienda}
\fancyfoot[LE]{\thepage}
\fancyfoot[RE]{Escuela Técnica Superior de Ingenieros Industriales (UPM)}
\fancyfoot[LO]{Luis D. Aranda Sánchez}
\fancyfoot[RO]{\thepage}
\renewcommand{\headrulewidth}{0.4pt}
\renewcommand{\footrulewidth}{0.4pt}

\fancypagestyle{myfancy}{
    \fancyhf{} % Clear all headers and footers
    \fancyhead[LE]{\nouppercase{\leftmark}}
    \fancyhead[RO]{Optimización energética para vivienda}
    \fancyfoot[LE]{\thepage}
    \fancyfoot[RE]{Escuela Técnica Superior de Ingenieros Industriales (UPM)}
    \fancyfoot[LO]{Luis D. Aranda Sánchez}
    \fancyfoot[RO]{\thepage}
    \renewcommand{\headrulewidth}{0.4pt}
    \renewcommand{\footrulewidth}{0.4pt}
}

\fancypagestyle{simple}{
    \fancyhf{} % Clear all headers and footers
    \renewcommand{\headrulewidth}{0pt}
    \renewcommand{\footrulewidth}{0pt}
}

% Line spacing
\setstretch{1.2}

% Document starts here
\begin{document}

% Portada
\begin{titlepage}
    \centering
    {\scshape\LARGE Universidad Politécnica de Madrid \par}
    \vspace{1cm}
    {\scshape\Large Escuela Técnica Superior de Ingenieros Industriales\par}
    \vspace{1.5cm}
    {\huge\bfseries Optimización energética de sistema híbrido con bomba de calor, suelo radiante, fotovoltaica y almacenamiento para vivienda \par}
    \vspace{1.5cm}
    {\Large\bfseries Trabajo de Fin de Máster\par}
    \vspace{0.5cm}
    {\large Máster Universitario en Ingeniería de la Energía \par}
    \vspace{2cm}
    {\Large Luis D. Aranda Sánchez\par}
    \vfill
    Director: Javier Rodríguez Martín
    \vfill
    {\large Septiembre 6, 2024\par}
\end{titlepage}

% Resumen (máximo de 5 páginas, incluyendo al final Palabras clave)
\clearpage
\pagestyle{simple}
% \newpage
\chapter*{Resumen}
\addcontentsline{toc}{chapter}{Resumen}
\input{capitulos/resumen/main.tex}

% Índice (paginado)
\clearpage
\pagestyle{simple}
% \newpage
\tableofcontents

% Introducción (donde se incluya los antecedentes y justificación)
\clearpage
\pagestyle{myfancy}
\newpage
\chapter{Introducción}
\input{capitulos/introduccion/main.tex}

% Objetivos
\chapter{Objetivos}
\input{capitulos/objetivos/main.tex}

% Metodología
\chapter{Metodología}
\input{capitulos/metodologia/main.tex}

% Resultados y discusión (incluyendo la valoración de impactos y de aspectos de responsabilidad legal, ética y profesional relacionados con el trabajo)
\chapter{Resultados y Discusión}
\input{capitulos/resultados_discusion/main.tex}

% Conclusiones
\chapter{Conclusiones}
\input{capitulos/conclusiones/main.tex}

% Planificación temporal y presupuesto
\chapter{Planificación Temporal y Presupuesto}
\input{capitulos/planificacion_presupuesto/main.tex}

% Bibliografía
\newpage
\addcontentsline{toc}{chapter}{Bibliografía}
\printbibliography

\end{document}


% Resultados y discusión (incluyendo la valoración de impactos y de aspectos de responsabilidad legal, ética y profesional relacionados con el trabajo)
\chapter{Resultados y Discusión}
\documentclass[a4paper,11pt,twoside]{report}
\usepackage[left=25mm,right=25mm,top=25mm,bottom=25mm,includehead,includefoot,headsep=15mm,footskip=15mm]{geometry}
\usepackage{graphicx}
\usepackage{fancyhdr}
\usepackage{titlesec}
\usepackage[spanish]{babel}
\usepackage[utf8]{inputenc}
\usepackage{amsmath}
\usepackage{setspace}
\usepackage{svg}
\usepackage{hyperref}
\usepackage[backend=biber,style=numeric]{biblatex}
\addbibresource{references.bib}
\hypersetup{
    colorlinks=true,
    linkcolor=blue,      % color of internal links (sections, etc.)
    urlcolor=blue,       % color of external links
    pdftitle={Optimización energética de sistema híbrido con bomba de calor, suelo radiante, fotovoltaica y almacenamiento para vivienda},    % title
    pdfauthor={Luis D. Aranda Sánchez},     % author
    pdfkeywords={palabra1, palabra2, código1, etc.} % list of keywords
}

% Font change to Arial
\usepackage{helvet}
\renewcommand{\familydefault}{\sfdefault}

% Chapter titles in uppercase and larger font
\titleformat{\chapter}[hang]{\large\bfseries}{\thechapter.}{1em}{\MakeUppercase}
\titleformat{\section}[hang]{\bfseries}{\thesection.}{1em}{}
\titleformat{\subsection}[hang]{\bfseries}{\thesubsection.}{1em}{}

% Fancyhdr setup
\setlength{\headheight}{14.30174pt} % Adjust to recommended value, slightly larger for safety
\fancyhf{} % Clear all headers and footers
\fancyhead[LE]{\nouppercase{\leftmark}}
\fancyhead[RO]{Optimización energética para vivienda}
\fancyfoot[LE]{\thepage}
\fancyfoot[RE]{Escuela Técnica Superior de Ingenieros Industriales (UPM)}
\fancyfoot[LO]{Luis D. Aranda Sánchez}
\fancyfoot[RO]{\thepage}
\renewcommand{\headrulewidth}{0.4pt}
\renewcommand{\footrulewidth}{0.4pt}

\fancypagestyle{myfancy}{
    \fancyhf{} % Clear all headers and footers
    \fancyhead[LE]{\nouppercase{\leftmark}}
    \fancyhead[RO]{Optimización energética para vivienda}
    \fancyfoot[LE]{\thepage}
    \fancyfoot[RE]{Escuela Técnica Superior de Ingenieros Industriales (UPM)}
    \fancyfoot[LO]{Luis D. Aranda Sánchez}
    \fancyfoot[RO]{\thepage}
    \renewcommand{\headrulewidth}{0.4pt}
    \renewcommand{\footrulewidth}{0.4pt}
}

\fancypagestyle{simple}{
    \fancyhf{} % Clear all headers and footers
    \renewcommand{\headrulewidth}{0pt}
    \renewcommand{\footrulewidth}{0pt}
}

% Line spacing
\setstretch{1.2}

% Document starts here
\begin{document}

% Portada
\begin{titlepage}
    \centering
    {\scshape\LARGE Universidad Politécnica de Madrid \par}
    \vspace{1cm}
    {\scshape\Large Escuela Técnica Superior de Ingenieros Industriales\par}
    \vspace{1.5cm}
    {\huge\bfseries Optimización energética de sistema híbrido con bomba de calor, suelo radiante, fotovoltaica y almacenamiento para vivienda \par}
    \vspace{1.5cm}
    {\Large\bfseries Trabajo de Fin de Máster\par}
    \vspace{0.5cm}
    {\large Máster Universitario en Ingeniería de la Energía \par}
    \vspace{2cm}
    {\Large Luis D. Aranda Sánchez\par}
    \vfill
    Director: Javier Rodríguez Martín
    \vfill
    {\large Septiembre 6, 2024\par}
\end{titlepage}

% Resumen (máximo de 5 páginas, incluyendo al final Palabras clave)
\clearpage
\pagestyle{simple}
% \newpage
\chapter*{Resumen}
\addcontentsline{toc}{chapter}{Resumen}
\input{capitulos/resumen/main.tex}

% Índice (paginado)
\clearpage
\pagestyle{simple}
% \newpage
\tableofcontents

% Introducción (donde se incluya los antecedentes y justificación)
\clearpage
\pagestyle{myfancy}
\newpage
\chapter{Introducción}
\input{capitulos/introduccion/main.tex}

% Objetivos
\chapter{Objetivos}
\input{capitulos/objetivos/main.tex}

% Metodología
\chapter{Metodología}
\input{capitulos/metodologia/main.tex}

% Resultados y discusión (incluyendo la valoración de impactos y de aspectos de responsabilidad legal, ética y profesional relacionados con el trabajo)
\chapter{Resultados y Discusión}
\input{capitulos/resultados_discusion/main.tex}

% Conclusiones
\chapter{Conclusiones}
\input{capitulos/conclusiones/main.tex}

% Planificación temporal y presupuesto
\chapter{Planificación Temporal y Presupuesto}
\input{capitulos/planificacion_presupuesto/main.tex}

% Bibliografía
\newpage
\addcontentsline{toc}{chapter}{Bibliografía}
\printbibliography

\end{document}


% Conclusiones
\chapter{Conclusiones}
\documentclass[a4paper,11pt,twoside]{report}
\usepackage[left=25mm,right=25mm,top=25mm,bottom=25mm,includehead,includefoot,headsep=15mm,footskip=15mm]{geometry}
\usepackage{graphicx}
\usepackage{fancyhdr}
\usepackage{titlesec}
\usepackage[spanish]{babel}
\usepackage[utf8]{inputenc}
\usepackage{amsmath}
\usepackage{setspace}
\usepackage{svg}
\usepackage{hyperref}
\usepackage[backend=biber,style=numeric]{biblatex}
\addbibresource{references.bib}
\hypersetup{
    colorlinks=true,
    linkcolor=blue,      % color of internal links (sections, etc.)
    urlcolor=blue,       % color of external links
    pdftitle={Optimización energética de sistema híbrido con bomba de calor, suelo radiante, fotovoltaica y almacenamiento para vivienda},    % title
    pdfauthor={Luis D. Aranda Sánchez},     % author
    pdfkeywords={palabra1, palabra2, código1, etc.} % list of keywords
}

% Font change to Arial
\usepackage{helvet}
\renewcommand{\familydefault}{\sfdefault}

% Chapter titles in uppercase and larger font
\titleformat{\chapter}[hang]{\large\bfseries}{\thechapter.}{1em}{\MakeUppercase}
\titleformat{\section}[hang]{\bfseries}{\thesection.}{1em}{}
\titleformat{\subsection}[hang]{\bfseries}{\thesubsection.}{1em}{}

% Fancyhdr setup
\setlength{\headheight}{14.30174pt} % Adjust to recommended value, slightly larger for safety
\fancyhf{} % Clear all headers and footers
\fancyhead[LE]{\nouppercase{\leftmark}}
\fancyhead[RO]{Optimización energética para vivienda}
\fancyfoot[LE]{\thepage}
\fancyfoot[RE]{Escuela Técnica Superior de Ingenieros Industriales (UPM)}
\fancyfoot[LO]{Luis D. Aranda Sánchez}
\fancyfoot[RO]{\thepage}
\renewcommand{\headrulewidth}{0.4pt}
\renewcommand{\footrulewidth}{0.4pt}

\fancypagestyle{myfancy}{
    \fancyhf{} % Clear all headers and footers
    \fancyhead[LE]{\nouppercase{\leftmark}}
    \fancyhead[RO]{Optimización energética para vivienda}
    \fancyfoot[LE]{\thepage}
    \fancyfoot[RE]{Escuela Técnica Superior de Ingenieros Industriales (UPM)}
    \fancyfoot[LO]{Luis D. Aranda Sánchez}
    \fancyfoot[RO]{\thepage}
    \renewcommand{\headrulewidth}{0.4pt}
    \renewcommand{\footrulewidth}{0.4pt}
}

\fancypagestyle{simple}{
    \fancyhf{} % Clear all headers and footers
    \renewcommand{\headrulewidth}{0pt}
    \renewcommand{\footrulewidth}{0pt}
}

% Line spacing
\setstretch{1.2}

% Document starts here
\begin{document}

% Portada
\begin{titlepage}
    \centering
    {\scshape\LARGE Universidad Politécnica de Madrid \par}
    \vspace{1cm}
    {\scshape\Large Escuela Técnica Superior de Ingenieros Industriales\par}
    \vspace{1.5cm}
    {\huge\bfseries Optimización energética de sistema híbrido con bomba de calor, suelo radiante, fotovoltaica y almacenamiento para vivienda \par}
    \vspace{1.5cm}
    {\Large\bfseries Trabajo de Fin de Máster\par}
    \vspace{0.5cm}
    {\large Máster Universitario en Ingeniería de la Energía \par}
    \vspace{2cm}
    {\Large Luis D. Aranda Sánchez\par}
    \vfill
    Director: Javier Rodríguez Martín
    \vfill
    {\large Septiembre 6, 2024\par}
\end{titlepage}

% Resumen (máximo de 5 páginas, incluyendo al final Palabras clave)
\clearpage
\pagestyle{simple}
% \newpage
\chapter*{Resumen}
\addcontentsline{toc}{chapter}{Resumen}
\input{capitulos/resumen/main.tex}

% Índice (paginado)
\clearpage
\pagestyle{simple}
% \newpage
\tableofcontents

% Introducción (donde se incluya los antecedentes y justificación)
\clearpage
\pagestyle{myfancy}
\newpage
\chapter{Introducción}
\input{capitulos/introduccion/main.tex}

% Objetivos
\chapter{Objetivos}
\input{capitulos/objetivos/main.tex}

% Metodología
\chapter{Metodología}
\input{capitulos/metodologia/main.tex}

% Resultados y discusión (incluyendo la valoración de impactos y de aspectos de responsabilidad legal, ética y profesional relacionados con el trabajo)
\chapter{Resultados y Discusión}
\input{capitulos/resultados_discusion/main.tex}

% Conclusiones
\chapter{Conclusiones}
\input{capitulos/conclusiones/main.tex}

% Planificación temporal y presupuesto
\chapter{Planificación Temporal y Presupuesto}
\input{capitulos/planificacion_presupuesto/main.tex}

% Bibliografía
\newpage
\addcontentsline{toc}{chapter}{Bibliografía}
\printbibliography

\end{document}


% Planificación temporal y presupuesto
\chapter{Planificación Temporal y Presupuesto}
\documentclass[a4paper,11pt,twoside]{report}
\usepackage[left=25mm,right=25mm,top=25mm,bottom=25mm,includehead,includefoot,headsep=15mm,footskip=15mm]{geometry}
\usepackage{graphicx}
\usepackage{fancyhdr}
\usepackage{titlesec}
\usepackage[spanish]{babel}
\usepackage[utf8]{inputenc}
\usepackage{amsmath}
\usepackage{setspace}
\usepackage{svg}
\usepackage{hyperref}
\usepackage[backend=biber,style=numeric]{biblatex}
\addbibresource{references.bib}
\hypersetup{
    colorlinks=true,
    linkcolor=blue,      % color of internal links (sections, etc.)
    urlcolor=blue,       % color of external links
    pdftitle={Optimización energética de sistema híbrido con bomba de calor, suelo radiante, fotovoltaica y almacenamiento para vivienda},    % title
    pdfauthor={Luis D. Aranda Sánchez},     % author
    pdfkeywords={palabra1, palabra2, código1, etc.} % list of keywords
}

% Font change to Arial
\usepackage{helvet}
\renewcommand{\familydefault}{\sfdefault}

% Chapter titles in uppercase and larger font
\titleformat{\chapter}[hang]{\large\bfseries}{\thechapter.}{1em}{\MakeUppercase}
\titleformat{\section}[hang]{\bfseries}{\thesection.}{1em}{}
\titleformat{\subsection}[hang]{\bfseries}{\thesubsection.}{1em}{}

% Fancyhdr setup
\setlength{\headheight}{14.30174pt} % Adjust to recommended value, slightly larger for safety
\fancyhf{} % Clear all headers and footers
\fancyhead[LE]{\nouppercase{\leftmark}}
\fancyhead[RO]{Optimización energética para vivienda}
\fancyfoot[LE]{\thepage}
\fancyfoot[RE]{Escuela Técnica Superior de Ingenieros Industriales (UPM)}
\fancyfoot[LO]{Luis D. Aranda Sánchez}
\fancyfoot[RO]{\thepage}
\renewcommand{\headrulewidth}{0.4pt}
\renewcommand{\footrulewidth}{0.4pt}

\fancypagestyle{myfancy}{
    \fancyhf{} % Clear all headers and footers
    \fancyhead[LE]{\nouppercase{\leftmark}}
    \fancyhead[RO]{Optimización energética para vivienda}
    \fancyfoot[LE]{\thepage}
    \fancyfoot[RE]{Escuela Técnica Superior de Ingenieros Industriales (UPM)}
    \fancyfoot[LO]{Luis D. Aranda Sánchez}
    \fancyfoot[RO]{\thepage}
    \renewcommand{\headrulewidth}{0.4pt}
    \renewcommand{\footrulewidth}{0.4pt}
}

\fancypagestyle{simple}{
    \fancyhf{} % Clear all headers and footers
    \renewcommand{\headrulewidth}{0pt}
    \renewcommand{\footrulewidth}{0pt}
}

% Line spacing
\setstretch{1.2}

% Document starts here
\begin{document}

% Portada
\begin{titlepage}
    \centering
    {\scshape\LARGE Universidad Politécnica de Madrid \par}
    \vspace{1cm}
    {\scshape\Large Escuela Técnica Superior de Ingenieros Industriales\par}
    \vspace{1.5cm}
    {\huge\bfseries Optimización energética de sistema híbrido con bomba de calor, suelo radiante, fotovoltaica y almacenamiento para vivienda \par}
    \vspace{1.5cm}
    {\Large\bfseries Trabajo de Fin de Máster\par}
    \vspace{0.5cm}
    {\large Máster Universitario en Ingeniería de la Energía \par}
    \vspace{2cm}
    {\Large Luis D. Aranda Sánchez\par}
    \vfill
    Director: Javier Rodríguez Martín
    \vfill
    {\large Septiembre 6, 2024\par}
\end{titlepage}

% Resumen (máximo de 5 páginas, incluyendo al final Palabras clave)
\clearpage
\pagestyle{simple}
% \newpage
\chapter*{Resumen}
\addcontentsline{toc}{chapter}{Resumen}
\input{capitulos/resumen/main.tex}

% Índice (paginado)
\clearpage
\pagestyle{simple}
% \newpage
\tableofcontents

% Introducción (donde se incluya los antecedentes y justificación)
\clearpage
\pagestyle{myfancy}
\newpage
\chapter{Introducción}
\input{capitulos/introduccion/main.tex}

% Objetivos
\chapter{Objetivos}
\input{capitulos/objetivos/main.tex}

% Metodología
\chapter{Metodología}
\input{capitulos/metodologia/main.tex}

% Resultados y discusión (incluyendo la valoración de impactos y de aspectos de responsabilidad legal, ética y profesional relacionados con el trabajo)
\chapter{Resultados y Discusión}
\input{capitulos/resultados_discusion/main.tex}

% Conclusiones
\chapter{Conclusiones}
\input{capitulos/conclusiones/main.tex}

% Planificación temporal y presupuesto
\chapter{Planificación Temporal y Presupuesto}
\input{capitulos/planificacion_presupuesto/main.tex}

% Bibliografía
\newpage
\addcontentsline{toc}{chapter}{Bibliografía}
\printbibliography

\end{document}


% Bibliografía
\newpage
\addcontentsline{toc}{chapter}{Bibliografía}
\printbibliography

\end{document}


% Objetivos
\chapter{Objetivos}
\documentclass[a4paper,11pt,twoside]{report}
\usepackage[left=25mm,right=25mm,top=25mm,bottom=25mm,includehead,includefoot,headsep=15mm,footskip=15mm]{geometry}
\usepackage{graphicx}
\usepackage{fancyhdr}
\usepackage{titlesec}
\usepackage[spanish]{babel}
\usepackage[utf8]{inputenc}
\usepackage{amsmath}
\usepackage{setspace}
\usepackage{svg}
\usepackage{hyperref}
\usepackage[backend=biber,style=numeric]{biblatex}
\addbibresource{references.bib}
\hypersetup{
    colorlinks=true,
    linkcolor=blue,      % color of internal links (sections, etc.)
    urlcolor=blue,       % color of external links
    pdftitle={Optimización energética de sistema híbrido con bomba de calor, suelo radiante, fotovoltaica y almacenamiento para vivienda},    % title
    pdfauthor={Luis D. Aranda Sánchez},     % author
    pdfkeywords={palabra1, palabra2, código1, etc.} % list of keywords
}

% Font change to Arial
\usepackage{helvet}
\renewcommand{\familydefault}{\sfdefault}

% Chapter titles in uppercase and larger font
\titleformat{\chapter}[hang]{\large\bfseries}{\thechapter.}{1em}{\MakeUppercase}
\titleformat{\section}[hang]{\bfseries}{\thesection.}{1em}{}
\titleformat{\subsection}[hang]{\bfseries}{\thesubsection.}{1em}{}

% Fancyhdr setup
\setlength{\headheight}{14.30174pt} % Adjust to recommended value, slightly larger for safety
\fancyhf{} % Clear all headers and footers
\fancyhead[LE]{\nouppercase{\leftmark}}
\fancyhead[RO]{Optimización energética para vivienda}
\fancyfoot[LE]{\thepage}
\fancyfoot[RE]{Escuela Técnica Superior de Ingenieros Industriales (UPM)}
\fancyfoot[LO]{Luis D. Aranda Sánchez}
\fancyfoot[RO]{\thepage}
\renewcommand{\headrulewidth}{0.4pt}
\renewcommand{\footrulewidth}{0.4pt}

\fancypagestyle{myfancy}{
    \fancyhf{} % Clear all headers and footers
    \fancyhead[LE]{\nouppercase{\leftmark}}
    \fancyhead[RO]{Optimización energética para vivienda}
    \fancyfoot[LE]{\thepage}
    \fancyfoot[RE]{Escuela Técnica Superior de Ingenieros Industriales (UPM)}
    \fancyfoot[LO]{Luis D. Aranda Sánchez}
    \fancyfoot[RO]{\thepage}
    \renewcommand{\headrulewidth}{0.4pt}
    \renewcommand{\footrulewidth}{0.4pt}
}

\fancypagestyle{simple}{
    \fancyhf{} % Clear all headers and footers
    \renewcommand{\headrulewidth}{0pt}
    \renewcommand{\footrulewidth}{0pt}
}

% Line spacing
\setstretch{1.2}

% Document starts here
\begin{document}

% Portada
\begin{titlepage}
    \centering
    {\scshape\LARGE Universidad Politécnica de Madrid \par}
    \vspace{1cm}
    {\scshape\Large Escuela Técnica Superior de Ingenieros Industriales\par}
    \vspace{1.5cm}
    {\huge\bfseries Optimización energética de sistema híbrido con bomba de calor, suelo radiante, fotovoltaica y almacenamiento para vivienda \par}
    \vspace{1.5cm}
    {\Large\bfseries Trabajo de Fin de Máster\par}
    \vspace{0.5cm}
    {\large Máster Universitario en Ingeniería de la Energía \par}
    \vspace{2cm}
    {\Large Luis D. Aranda Sánchez\par}
    \vfill
    Director: Javier Rodríguez Martín
    \vfill
    {\large Septiembre 6, 2024\par}
\end{titlepage}

% Resumen (máximo de 5 páginas, incluyendo al final Palabras clave)
\clearpage
\pagestyle{simple}
% \newpage
\chapter*{Resumen}
\addcontentsline{toc}{chapter}{Resumen}
\documentclass[a4paper,11pt,twoside]{report}
\usepackage[left=25mm,right=25mm,top=25mm,bottom=25mm,includehead,includefoot,headsep=15mm,footskip=15mm]{geometry}
\usepackage{graphicx}
\usepackage{fancyhdr}
\usepackage{titlesec}
\usepackage[spanish]{babel}
\usepackage[utf8]{inputenc}
\usepackage{amsmath}
\usepackage{setspace}
\usepackage{svg}
\usepackage{hyperref}
\usepackage[backend=biber,style=numeric]{biblatex}
\addbibresource{references.bib}
\hypersetup{
    colorlinks=true,
    linkcolor=blue,      % color of internal links (sections, etc.)
    urlcolor=blue,       % color of external links
    pdftitle={Optimización energética de sistema híbrido con bomba de calor, suelo radiante, fotovoltaica y almacenamiento para vivienda},    % title
    pdfauthor={Luis D. Aranda Sánchez},     % author
    pdfkeywords={palabra1, palabra2, código1, etc.} % list of keywords
}

% Font change to Arial
\usepackage{helvet}
\renewcommand{\familydefault}{\sfdefault}

% Chapter titles in uppercase and larger font
\titleformat{\chapter}[hang]{\large\bfseries}{\thechapter.}{1em}{\MakeUppercase}
\titleformat{\section}[hang]{\bfseries}{\thesection.}{1em}{}
\titleformat{\subsection}[hang]{\bfseries}{\thesubsection.}{1em}{}

% Fancyhdr setup
\setlength{\headheight}{14.30174pt} % Adjust to recommended value, slightly larger for safety
\fancyhf{} % Clear all headers and footers
\fancyhead[LE]{\nouppercase{\leftmark}}
\fancyhead[RO]{Optimización energética para vivienda}
\fancyfoot[LE]{\thepage}
\fancyfoot[RE]{Escuela Técnica Superior de Ingenieros Industriales (UPM)}
\fancyfoot[LO]{Luis D. Aranda Sánchez}
\fancyfoot[RO]{\thepage}
\renewcommand{\headrulewidth}{0.4pt}
\renewcommand{\footrulewidth}{0.4pt}

\fancypagestyle{myfancy}{
    \fancyhf{} % Clear all headers and footers
    \fancyhead[LE]{\nouppercase{\leftmark}}
    \fancyhead[RO]{Optimización energética para vivienda}
    \fancyfoot[LE]{\thepage}
    \fancyfoot[RE]{Escuela Técnica Superior de Ingenieros Industriales (UPM)}
    \fancyfoot[LO]{Luis D. Aranda Sánchez}
    \fancyfoot[RO]{\thepage}
    \renewcommand{\headrulewidth}{0.4pt}
    \renewcommand{\footrulewidth}{0.4pt}
}

\fancypagestyle{simple}{
    \fancyhf{} % Clear all headers and footers
    \renewcommand{\headrulewidth}{0pt}
    \renewcommand{\footrulewidth}{0pt}
}

% Line spacing
\setstretch{1.2}

% Document starts here
\begin{document}

% Portada
\begin{titlepage}
    \centering
    {\scshape\LARGE Universidad Politécnica de Madrid \par}
    \vspace{1cm}
    {\scshape\Large Escuela Técnica Superior de Ingenieros Industriales\par}
    \vspace{1.5cm}
    {\huge\bfseries Optimización energética de sistema híbrido con bomba de calor, suelo radiante, fotovoltaica y almacenamiento para vivienda \par}
    \vspace{1.5cm}
    {\Large\bfseries Trabajo de Fin de Máster\par}
    \vspace{0.5cm}
    {\large Máster Universitario en Ingeniería de la Energía \par}
    \vspace{2cm}
    {\Large Luis D. Aranda Sánchez\par}
    \vfill
    Director: Javier Rodríguez Martín
    \vfill
    {\large Septiembre 6, 2024\par}
\end{titlepage}

% Resumen (máximo de 5 páginas, incluyendo al final Palabras clave)
\clearpage
\pagestyle{simple}
% \newpage
\chapter*{Resumen}
\addcontentsline{toc}{chapter}{Resumen}
\input{capitulos/resumen/main.tex}

% Índice (paginado)
\clearpage
\pagestyle{simple}
% \newpage
\tableofcontents

% Introducción (donde se incluya los antecedentes y justificación)
\clearpage
\pagestyle{myfancy}
\newpage
\chapter{Introducción}
\input{capitulos/introduccion/main.tex}

% Objetivos
\chapter{Objetivos}
\input{capitulos/objetivos/main.tex}

% Metodología
\chapter{Metodología}
\input{capitulos/metodologia/main.tex}

% Resultados y discusión (incluyendo la valoración de impactos y de aspectos de responsabilidad legal, ética y profesional relacionados con el trabajo)
\chapter{Resultados y Discusión}
\input{capitulos/resultados_discusion/main.tex}

% Conclusiones
\chapter{Conclusiones}
\input{capitulos/conclusiones/main.tex}

% Planificación temporal y presupuesto
\chapter{Planificación Temporal y Presupuesto}
\input{capitulos/planificacion_presupuesto/main.tex}

% Bibliografía
\newpage
\addcontentsline{toc}{chapter}{Bibliografía}
\printbibliography

\end{document}


% Índice (paginado)
\clearpage
\pagestyle{simple}
% \newpage
\tableofcontents

% Introducción (donde se incluya los antecedentes y justificación)
\clearpage
\pagestyle{myfancy}
\newpage
\chapter{Introducción}
\documentclass[a4paper,11pt,twoside]{report}
\usepackage[left=25mm,right=25mm,top=25mm,bottom=25mm,includehead,includefoot,headsep=15mm,footskip=15mm]{geometry}
\usepackage{graphicx}
\usepackage{fancyhdr}
\usepackage{titlesec}
\usepackage[spanish]{babel}
\usepackage[utf8]{inputenc}
\usepackage{amsmath}
\usepackage{setspace}
\usepackage{svg}
\usepackage{hyperref}
\usepackage[backend=biber,style=numeric]{biblatex}
\addbibresource{references.bib}
\hypersetup{
    colorlinks=true,
    linkcolor=blue,      % color of internal links (sections, etc.)
    urlcolor=blue,       % color of external links
    pdftitle={Optimización energética de sistema híbrido con bomba de calor, suelo radiante, fotovoltaica y almacenamiento para vivienda},    % title
    pdfauthor={Luis D. Aranda Sánchez},     % author
    pdfkeywords={palabra1, palabra2, código1, etc.} % list of keywords
}

% Font change to Arial
\usepackage{helvet}
\renewcommand{\familydefault}{\sfdefault}

% Chapter titles in uppercase and larger font
\titleformat{\chapter}[hang]{\large\bfseries}{\thechapter.}{1em}{\MakeUppercase}
\titleformat{\section}[hang]{\bfseries}{\thesection.}{1em}{}
\titleformat{\subsection}[hang]{\bfseries}{\thesubsection.}{1em}{}

% Fancyhdr setup
\setlength{\headheight}{14.30174pt} % Adjust to recommended value, slightly larger for safety
\fancyhf{} % Clear all headers and footers
\fancyhead[LE]{\nouppercase{\leftmark}}
\fancyhead[RO]{Optimización energética para vivienda}
\fancyfoot[LE]{\thepage}
\fancyfoot[RE]{Escuela Técnica Superior de Ingenieros Industriales (UPM)}
\fancyfoot[LO]{Luis D. Aranda Sánchez}
\fancyfoot[RO]{\thepage}
\renewcommand{\headrulewidth}{0.4pt}
\renewcommand{\footrulewidth}{0.4pt}

\fancypagestyle{myfancy}{
    \fancyhf{} % Clear all headers and footers
    \fancyhead[LE]{\nouppercase{\leftmark}}
    \fancyhead[RO]{Optimización energética para vivienda}
    \fancyfoot[LE]{\thepage}
    \fancyfoot[RE]{Escuela Técnica Superior de Ingenieros Industriales (UPM)}
    \fancyfoot[LO]{Luis D. Aranda Sánchez}
    \fancyfoot[RO]{\thepage}
    \renewcommand{\headrulewidth}{0.4pt}
    \renewcommand{\footrulewidth}{0.4pt}
}

\fancypagestyle{simple}{
    \fancyhf{} % Clear all headers and footers
    \renewcommand{\headrulewidth}{0pt}
    \renewcommand{\footrulewidth}{0pt}
}

% Line spacing
\setstretch{1.2}

% Document starts here
\begin{document}

% Portada
\begin{titlepage}
    \centering
    {\scshape\LARGE Universidad Politécnica de Madrid \par}
    \vspace{1cm}
    {\scshape\Large Escuela Técnica Superior de Ingenieros Industriales\par}
    \vspace{1.5cm}
    {\huge\bfseries Optimización energética de sistema híbrido con bomba de calor, suelo radiante, fotovoltaica y almacenamiento para vivienda \par}
    \vspace{1.5cm}
    {\Large\bfseries Trabajo de Fin de Máster\par}
    \vspace{0.5cm}
    {\large Máster Universitario en Ingeniería de la Energía \par}
    \vspace{2cm}
    {\Large Luis D. Aranda Sánchez\par}
    \vfill
    Director: Javier Rodríguez Martín
    \vfill
    {\large Septiembre 6, 2024\par}
\end{titlepage}

% Resumen (máximo de 5 páginas, incluyendo al final Palabras clave)
\clearpage
\pagestyle{simple}
% \newpage
\chapter*{Resumen}
\addcontentsline{toc}{chapter}{Resumen}
\input{capitulos/resumen/main.tex}

% Índice (paginado)
\clearpage
\pagestyle{simple}
% \newpage
\tableofcontents

% Introducción (donde se incluya los antecedentes y justificación)
\clearpage
\pagestyle{myfancy}
\newpage
\chapter{Introducción}
\input{capitulos/introduccion/main.tex}

% Objetivos
\chapter{Objetivos}
\input{capitulos/objetivos/main.tex}

% Metodología
\chapter{Metodología}
\input{capitulos/metodologia/main.tex}

% Resultados y discusión (incluyendo la valoración de impactos y de aspectos de responsabilidad legal, ética y profesional relacionados con el trabajo)
\chapter{Resultados y Discusión}
\input{capitulos/resultados_discusion/main.tex}

% Conclusiones
\chapter{Conclusiones}
\input{capitulos/conclusiones/main.tex}

% Planificación temporal y presupuesto
\chapter{Planificación Temporal y Presupuesto}
\input{capitulos/planificacion_presupuesto/main.tex}

% Bibliografía
\newpage
\addcontentsline{toc}{chapter}{Bibliografía}
\printbibliography

\end{document}


% Objetivos
\chapter{Objetivos}
\documentclass[a4paper,11pt,twoside]{report}
\usepackage[left=25mm,right=25mm,top=25mm,bottom=25mm,includehead,includefoot,headsep=15mm,footskip=15mm]{geometry}
\usepackage{graphicx}
\usepackage{fancyhdr}
\usepackage{titlesec}
\usepackage[spanish]{babel}
\usepackage[utf8]{inputenc}
\usepackage{amsmath}
\usepackage{setspace}
\usepackage{svg}
\usepackage{hyperref}
\usepackage[backend=biber,style=numeric]{biblatex}
\addbibresource{references.bib}
\hypersetup{
    colorlinks=true,
    linkcolor=blue,      % color of internal links (sections, etc.)
    urlcolor=blue,       % color of external links
    pdftitle={Optimización energética de sistema híbrido con bomba de calor, suelo radiante, fotovoltaica y almacenamiento para vivienda},    % title
    pdfauthor={Luis D. Aranda Sánchez},     % author
    pdfkeywords={palabra1, palabra2, código1, etc.} % list of keywords
}

% Font change to Arial
\usepackage{helvet}
\renewcommand{\familydefault}{\sfdefault}

% Chapter titles in uppercase and larger font
\titleformat{\chapter}[hang]{\large\bfseries}{\thechapter.}{1em}{\MakeUppercase}
\titleformat{\section}[hang]{\bfseries}{\thesection.}{1em}{}
\titleformat{\subsection}[hang]{\bfseries}{\thesubsection.}{1em}{}

% Fancyhdr setup
\setlength{\headheight}{14.30174pt} % Adjust to recommended value, slightly larger for safety
\fancyhf{} % Clear all headers and footers
\fancyhead[LE]{\nouppercase{\leftmark}}
\fancyhead[RO]{Optimización energética para vivienda}
\fancyfoot[LE]{\thepage}
\fancyfoot[RE]{Escuela Técnica Superior de Ingenieros Industriales (UPM)}
\fancyfoot[LO]{Luis D. Aranda Sánchez}
\fancyfoot[RO]{\thepage}
\renewcommand{\headrulewidth}{0.4pt}
\renewcommand{\footrulewidth}{0.4pt}

\fancypagestyle{myfancy}{
    \fancyhf{} % Clear all headers and footers
    \fancyhead[LE]{\nouppercase{\leftmark}}
    \fancyhead[RO]{Optimización energética para vivienda}
    \fancyfoot[LE]{\thepage}
    \fancyfoot[RE]{Escuela Técnica Superior de Ingenieros Industriales (UPM)}
    \fancyfoot[LO]{Luis D. Aranda Sánchez}
    \fancyfoot[RO]{\thepage}
    \renewcommand{\headrulewidth}{0.4pt}
    \renewcommand{\footrulewidth}{0.4pt}
}

\fancypagestyle{simple}{
    \fancyhf{} % Clear all headers and footers
    \renewcommand{\headrulewidth}{0pt}
    \renewcommand{\footrulewidth}{0pt}
}

% Line spacing
\setstretch{1.2}

% Document starts here
\begin{document}

% Portada
\begin{titlepage}
    \centering
    {\scshape\LARGE Universidad Politécnica de Madrid \par}
    \vspace{1cm}
    {\scshape\Large Escuela Técnica Superior de Ingenieros Industriales\par}
    \vspace{1.5cm}
    {\huge\bfseries Optimización energética de sistema híbrido con bomba de calor, suelo radiante, fotovoltaica y almacenamiento para vivienda \par}
    \vspace{1.5cm}
    {\Large\bfseries Trabajo de Fin de Máster\par}
    \vspace{0.5cm}
    {\large Máster Universitario en Ingeniería de la Energía \par}
    \vspace{2cm}
    {\Large Luis D. Aranda Sánchez\par}
    \vfill
    Director: Javier Rodríguez Martín
    \vfill
    {\large Septiembre 6, 2024\par}
\end{titlepage}

% Resumen (máximo de 5 páginas, incluyendo al final Palabras clave)
\clearpage
\pagestyle{simple}
% \newpage
\chapter*{Resumen}
\addcontentsline{toc}{chapter}{Resumen}
\input{capitulos/resumen/main.tex}

% Índice (paginado)
\clearpage
\pagestyle{simple}
% \newpage
\tableofcontents

% Introducción (donde se incluya los antecedentes y justificación)
\clearpage
\pagestyle{myfancy}
\newpage
\chapter{Introducción}
\input{capitulos/introduccion/main.tex}

% Objetivos
\chapter{Objetivos}
\input{capitulos/objetivos/main.tex}

% Metodología
\chapter{Metodología}
\input{capitulos/metodologia/main.tex}

% Resultados y discusión (incluyendo la valoración de impactos y de aspectos de responsabilidad legal, ética y profesional relacionados con el trabajo)
\chapter{Resultados y Discusión}
\input{capitulos/resultados_discusion/main.tex}

% Conclusiones
\chapter{Conclusiones}
\input{capitulos/conclusiones/main.tex}

% Planificación temporal y presupuesto
\chapter{Planificación Temporal y Presupuesto}
\input{capitulos/planificacion_presupuesto/main.tex}

% Bibliografía
\newpage
\addcontentsline{toc}{chapter}{Bibliografía}
\printbibliography

\end{document}


% Metodología
\chapter{Metodología}
\documentclass[a4paper,11pt,twoside]{report}
\usepackage[left=25mm,right=25mm,top=25mm,bottom=25mm,includehead,includefoot,headsep=15mm,footskip=15mm]{geometry}
\usepackage{graphicx}
\usepackage{fancyhdr}
\usepackage{titlesec}
\usepackage[spanish]{babel}
\usepackage[utf8]{inputenc}
\usepackage{amsmath}
\usepackage{setspace}
\usepackage{svg}
\usepackage{hyperref}
\usepackage[backend=biber,style=numeric]{biblatex}
\addbibresource{references.bib}
\hypersetup{
    colorlinks=true,
    linkcolor=blue,      % color of internal links (sections, etc.)
    urlcolor=blue,       % color of external links
    pdftitle={Optimización energética de sistema híbrido con bomba de calor, suelo radiante, fotovoltaica y almacenamiento para vivienda},    % title
    pdfauthor={Luis D. Aranda Sánchez},     % author
    pdfkeywords={palabra1, palabra2, código1, etc.} % list of keywords
}

% Font change to Arial
\usepackage{helvet}
\renewcommand{\familydefault}{\sfdefault}

% Chapter titles in uppercase and larger font
\titleformat{\chapter}[hang]{\large\bfseries}{\thechapter.}{1em}{\MakeUppercase}
\titleformat{\section}[hang]{\bfseries}{\thesection.}{1em}{}
\titleformat{\subsection}[hang]{\bfseries}{\thesubsection.}{1em}{}

% Fancyhdr setup
\setlength{\headheight}{14.30174pt} % Adjust to recommended value, slightly larger for safety
\fancyhf{} % Clear all headers and footers
\fancyhead[LE]{\nouppercase{\leftmark}}
\fancyhead[RO]{Optimización energética para vivienda}
\fancyfoot[LE]{\thepage}
\fancyfoot[RE]{Escuela Técnica Superior de Ingenieros Industriales (UPM)}
\fancyfoot[LO]{Luis D. Aranda Sánchez}
\fancyfoot[RO]{\thepage}
\renewcommand{\headrulewidth}{0.4pt}
\renewcommand{\footrulewidth}{0.4pt}

\fancypagestyle{myfancy}{
    \fancyhf{} % Clear all headers and footers
    \fancyhead[LE]{\nouppercase{\leftmark}}
    \fancyhead[RO]{Optimización energética para vivienda}
    \fancyfoot[LE]{\thepage}
    \fancyfoot[RE]{Escuela Técnica Superior de Ingenieros Industriales (UPM)}
    \fancyfoot[LO]{Luis D. Aranda Sánchez}
    \fancyfoot[RO]{\thepage}
    \renewcommand{\headrulewidth}{0.4pt}
    \renewcommand{\footrulewidth}{0.4pt}
}

\fancypagestyle{simple}{
    \fancyhf{} % Clear all headers and footers
    \renewcommand{\headrulewidth}{0pt}
    \renewcommand{\footrulewidth}{0pt}
}

% Line spacing
\setstretch{1.2}

% Document starts here
\begin{document}

% Portada
\begin{titlepage}
    \centering
    {\scshape\LARGE Universidad Politécnica de Madrid \par}
    \vspace{1cm}
    {\scshape\Large Escuela Técnica Superior de Ingenieros Industriales\par}
    \vspace{1.5cm}
    {\huge\bfseries Optimización energética de sistema híbrido con bomba de calor, suelo radiante, fotovoltaica y almacenamiento para vivienda \par}
    \vspace{1.5cm}
    {\Large\bfseries Trabajo de Fin de Máster\par}
    \vspace{0.5cm}
    {\large Máster Universitario en Ingeniería de la Energía \par}
    \vspace{2cm}
    {\Large Luis D. Aranda Sánchez\par}
    \vfill
    Director: Javier Rodríguez Martín
    \vfill
    {\large Septiembre 6, 2024\par}
\end{titlepage}

% Resumen (máximo de 5 páginas, incluyendo al final Palabras clave)
\clearpage
\pagestyle{simple}
% \newpage
\chapter*{Resumen}
\addcontentsline{toc}{chapter}{Resumen}
\input{capitulos/resumen/main.tex}

% Índice (paginado)
\clearpage
\pagestyle{simple}
% \newpage
\tableofcontents

% Introducción (donde se incluya los antecedentes y justificación)
\clearpage
\pagestyle{myfancy}
\newpage
\chapter{Introducción}
\input{capitulos/introduccion/main.tex}

% Objetivos
\chapter{Objetivos}
\input{capitulos/objetivos/main.tex}

% Metodología
\chapter{Metodología}
\input{capitulos/metodologia/main.tex}

% Resultados y discusión (incluyendo la valoración de impactos y de aspectos de responsabilidad legal, ética y profesional relacionados con el trabajo)
\chapter{Resultados y Discusión}
\input{capitulos/resultados_discusion/main.tex}

% Conclusiones
\chapter{Conclusiones}
\input{capitulos/conclusiones/main.tex}

% Planificación temporal y presupuesto
\chapter{Planificación Temporal y Presupuesto}
\input{capitulos/planificacion_presupuesto/main.tex}

% Bibliografía
\newpage
\addcontentsline{toc}{chapter}{Bibliografía}
\printbibliography

\end{document}


% Resultados y discusión (incluyendo la valoración de impactos y de aspectos de responsabilidad legal, ética y profesional relacionados con el trabajo)
\chapter{Resultados y Discusión}
\documentclass[a4paper,11pt,twoside]{report}
\usepackage[left=25mm,right=25mm,top=25mm,bottom=25mm,includehead,includefoot,headsep=15mm,footskip=15mm]{geometry}
\usepackage{graphicx}
\usepackage{fancyhdr}
\usepackage{titlesec}
\usepackage[spanish]{babel}
\usepackage[utf8]{inputenc}
\usepackage{amsmath}
\usepackage{setspace}
\usepackage{svg}
\usepackage{hyperref}
\usepackage[backend=biber,style=numeric]{biblatex}
\addbibresource{references.bib}
\hypersetup{
    colorlinks=true,
    linkcolor=blue,      % color of internal links (sections, etc.)
    urlcolor=blue,       % color of external links
    pdftitle={Optimización energética de sistema híbrido con bomba de calor, suelo radiante, fotovoltaica y almacenamiento para vivienda},    % title
    pdfauthor={Luis D. Aranda Sánchez},     % author
    pdfkeywords={palabra1, palabra2, código1, etc.} % list of keywords
}

% Font change to Arial
\usepackage{helvet}
\renewcommand{\familydefault}{\sfdefault}

% Chapter titles in uppercase and larger font
\titleformat{\chapter}[hang]{\large\bfseries}{\thechapter.}{1em}{\MakeUppercase}
\titleformat{\section}[hang]{\bfseries}{\thesection.}{1em}{}
\titleformat{\subsection}[hang]{\bfseries}{\thesubsection.}{1em}{}

% Fancyhdr setup
\setlength{\headheight}{14.30174pt} % Adjust to recommended value, slightly larger for safety
\fancyhf{} % Clear all headers and footers
\fancyhead[LE]{\nouppercase{\leftmark}}
\fancyhead[RO]{Optimización energética para vivienda}
\fancyfoot[LE]{\thepage}
\fancyfoot[RE]{Escuela Técnica Superior de Ingenieros Industriales (UPM)}
\fancyfoot[LO]{Luis D. Aranda Sánchez}
\fancyfoot[RO]{\thepage}
\renewcommand{\headrulewidth}{0.4pt}
\renewcommand{\footrulewidth}{0.4pt}

\fancypagestyle{myfancy}{
    \fancyhf{} % Clear all headers and footers
    \fancyhead[LE]{\nouppercase{\leftmark}}
    \fancyhead[RO]{Optimización energética para vivienda}
    \fancyfoot[LE]{\thepage}
    \fancyfoot[RE]{Escuela Técnica Superior de Ingenieros Industriales (UPM)}
    \fancyfoot[LO]{Luis D. Aranda Sánchez}
    \fancyfoot[RO]{\thepage}
    \renewcommand{\headrulewidth}{0.4pt}
    \renewcommand{\footrulewidth}{0.4pt}
}

\fancypagestyle{simple}{
    \fancyhf{} % Clear all headers and footers
    \renewcommand{\headrulewidth}{0pt}
    \renewcommand{\footrulewidth}{0pt}
}

% Line spacing
\setstretch{1.2}

% Document starts here
\begin{document}

% Portada
\begin{titlepage}
    \centering
    {\scshape\LARGE Universidad Politécnica de Madrid \par}
    \vspace{1cm}
    {\scshape\Large Escuela Técnica Superior de Ingenieros Industriales\par}
    \vspace{1.5cm}
    {\huge\bfseries Optimización energética de sistema híbrido con bomba de calor, suelo radiante, fotovoltaica y almacenamiento para vivienda \par}
    \vspace{1.5cm}
    {\Large\bfseries Trabajo de Fin de Máster\par}
    \vspace{0.5cm}
    {\large Máster Universitario en Ingeniería de la Energía \par}
    \vspace{2cm}
    {\Large Luis D. Aranda Sánchez\par}
    \vfill
    Director: Javier Rodríguez Martín
    \vfill
    {\large Septiembre 6, 2024\par}
\end{titlepage}

% Resumen (máximo de 5 páginas, incluyendo al final Palabras clave)
\clearpage
\pagestyle{simple}
% \newpage
\chapter*{Resumen}
\addcontentsline{toc}{chapter}{Resumen}
\input{capitulos/resumen/main.tex}

% Índice (paginado)
\clearpage
\pagestyle{simple}
% \newpage
\tableofcontents

% Introducción (donde se incluya los antecedentes y justificación)
\clearpage
\pagestyle{myfancy}
\newpage
\chapter{Introducción}
\input{capitulos/introduccion/main.tex}

% Objetivos
\chapter{Objetivos}
\input{capitulos/objetivos/main.tex}

% Metodología
\chapter{Metodología}
\input{capitulos/metodologia/main.tex}

% Resultados y discusión (incluyendo la valoración de impactos y de aspectos de responsabilidad legal, ética y profesional relacionados con el trabajo)
\chapter{Resultados y Discusión}
\input{capitulos/resultados_discusion/main.tex}

% Conclusiones
\chapter{Conclusiones}
\input{capitulos/conclusiones/main.tex}

% Planificación temporal y presupuesto
\chapter{Planificación Temporal y Presupuesto}
\input{capitulos/planificacion_presupuesto/main.tex}

% Bibliografía
\newpage
\addcontentsline{toc}{chapter}{Bibliografía}
\printbibliography

\end{document}


% Conclusiones
\chapter{Conclusiones}
\documentclass[a4paper,11pt,twoside]{report}
\usepackage[left=25mm,right=25mm,top=25mm,bottom=25mm,includehead,includefoot,headsep=15mm,footskip=15mm]{geometry}
\usepackage{graphicx}
\usepackage{fancyhdr}
\usepackage{titlesec}
\usepackage[spanish]{babel}
\usepackage[utf8]{inputenc}
\usepackage{amsmath}
\usepackage{setspace}
\usepackage{svg}
\usepackage{hyperref}
\usepackage[backend=biber,style=numeric]{biblatex}
\addbibresource{references.bib}
\hypersetup{
    colorlinks=true,
    linkcolor=blue,      % color of internal links (sections, etc.)
    urlcolor=blue,       % color of external links
    pdftitle={Optimización energética de sistema híbrido con bomba de calor, suelo radiante, fotovoltaica y almacenamiento para vivienda},    % title
    pdfauthor={Luis D. Aranda Sánchez},     % author
    pdfkeywords={palabra1, palabra2, código1, etc.} % list of keywords
}

% Font change to Arial
\usepackage{helvet}
\renewcommand{\familydefault}{\sfdefault}

% Chapter titles in uppercase and larger font
\titleformat{\chapter}[hang]{\large\bfseries}{\thechapter.}{1em}{\MakeUppercase}
\titleformat{\section}[hang]{\bfseries}{\thesection.}{1em}{}
\titleformat{\subsection}[hang]{\bfseries}{\thesubsection.}{1em}{}

% Fancyhdr setup
\setlength{\headheight}{14.30174pt} % Adjust to recommended value, slightly larger for safety
\fancyhf{} % Clear all headers and footers
\fancyhead[LE]{\nouppercase{\leftmark}}
\fancyhead[RO]{Optimización energética para vivienda}
\fancyfoot[LE]{\thepage}
\fancyfoot[RE]{Escuela Técnica Superior de Ingenieros Industriales (UPM)}
\fancyfoot[LO]{Luis D. Aranda Sánchez}
\fancyfoot[RO]{\thepage}
\renewcommand{\headrulewidth}{0.4pt}
\renewcommand{\footrulewidth}{0.4pt}

\fancypagestyle{myfancy}{
    \fancyhf{} % Clear all headers and footers
    \fancyhead[LE]{\nouppercase{\leftmark}}
    \fancyhead[RO]{Optimización energética para vivienda}
    \fancyfoot[LE]{\thepage}
    \fancyfoot[RE]{Escuela Técnica Superior de Ingenieros Industriales (UPM)}
    \fancyfoot[LO]{Luis D. Aranda Sánchez}
    \fancyfoot[RO]{\thepage}
    \renewcommand{\headrulewidth}{0.4pt}
    \renewcommand{\footrulewidth}{0.4pt}
}

\fancypagestyle{simple}{
    \fancyhf{} % Clear all headers and footers
    \renewcommand{\headrulewidth}{0pt}
    \renewcommand{\footrulewidth}{0pt}
}

% Line spacing
\setstretch{1.2}

% Document starts here
\begin{document}

% Portada
\begin{titlepage}
    \centering
    {\scshape\LARGE Universidad Politécnica de Madrid \par}
    \vspace{1cm}
    {\scshape\Large Escuela Técnica Superior de Ingenieros Industriales\par}
    \vspace{1.5cm}
    {\huge\bfseries Optimización energética de sistema híbrido con bomba de calor, suelo radiante, fotovoltaica y almacenamiento para vivienda \par}
    \vspace{1.5cm}
    {\Large\bfseries Trabajo de Fin de Máster\par}
    \vspace{0.5cm}
    {\large Máster Universitario en Ingeniería de la Energía \par}
    \vspace{2cm}
    {\Large Luis D. Aranda Sánchez\par}
    \vfill
    Director: Javier Rodríguez Martín
    \vfill
    {\large Septiembre 6, 2024\par}
\end{titlepage}

% Resumen (máximo de 5 páginas, incluyendo al final Palabras clave)
\clearpage
\pagestyle{simple}
% \newpage
\chapter*{Resumen}
\addcontentsline{toc}{chapter}{Resumen}
\input{capitulos/resumen/main.tex}

% Índice (paginado)
\clearpage
\pagestyle{simple}
% \newpage
\tableofcontents

% Introducción (donde se incluya los antecedentes y justificación)
\clearpage
\pagestyle{myfancy}
\newpage
\chapter{Introducción}
\input{capitulos/introduccion/main.tex}

% Objetivos
\chapter{Objetivos}
\input{capitulos/objetivos/main.tex}

% Metodología
\chapter{Metodología}
\input{capitulos/metodologia/main.tex}

% Resultados y discusión (incluyendo la valoración de impactos y de aspectos de responsabilidad legal, ética y profesional relacionados con el trabajo)
\chapter{Resultados y Discusión}
\input{capitulos/resultados_discusion/main.tex}

% Conclusiones
\chapter{Conclusiones}
\input{capitulos/conclusiones/main.tex}

% Planificación temporal y presupuesto
\chapter{Planificación Temporal y Presupuesto}
\input{capitulos/planificacion_presupuesto/main.tex}

% Bibliografía
\newpage
\addcontentsline{toc}{chapter}{Bibliografía}
\printbibliography

\end{document}


% Planificación temporal y presupuesto
\chapter{Planificación Temporal y Presupuesto}
\documentclass[a4paper,11pt,twoside]{report}
\usepackage[left=25mm,right=25mm,top=25mm,bottom=25mm,includehead,includefoot,headsep=15mm,footskip=15mm]{geometry}
\usepackage{graphicx}
\usepackage{fancyhdr}
\usepackage{titlesec}
\usepackage[spanish]{babel}
\usepackage[utf8]{inputenc}
\usepackage{amsmath}
\usepackage{setspace}
\usepackage{svg}
\usepackage{hyperref}
\usepackage[backend=biber,style=numeric]{biblatex}
\addbibresource{references.bib}
\hypersetup{
    colorlinks=true,
    linkcolor=blue,      % color of internal links (sections, etc.)
    urlcolor=blue,       % color of external links
    pdftitle={Optimización energética de sistema híbrido con bomba de calor, suelo radiante, fotovoltaica y almacenamiento para vivienda},    % title
    pdfauthor={Luis D. Aranda Sánchez},     % author
    pdfkeywords={palabra1, palabra2, código1, etc.} % list of keywords
}

% Font change to Arial
\usepackage{helvet}
\renewcommand{\familydefault}{\sfdefault}

% Chapter titles in uppercase and larger font
\titleformat{\chapter}[hang]{\large\bfseries}{\thechapter.}{1em}{\MakeUppercase}
\titleformat{\section}[hang]{\bfseries}{\thesection.}{1em}{}
\titleformat{\subsection}[hang]{\bfseries}{\thesubsection.}{1em}{}

% Fancyhdr setup
\setlength{\headheight}{14.30174pt} % Adjust to recommended value, slightly larger for safety
\fancyhf{} % Clear all headers and footers
\fancyhead[LE]{\nouppercase{\leftmark}}
\fancyhead[RO]{Optimización energética para vivienda}
\fancyfoot[LE]{\thepage}
\fancyfoot[RE]{Escuela Técnica Superior de Ingenieros Industriales (UPM)}
\fancyfoot[LO]{Luis D. Aranda Sánchez}
\fancyfoot[RO]{\thepage}
\renewcommand{\headrulewidth}{0.4pt}
\renewcommand{\footrulewidth}{0.4pt}

\fancypagestyle{myfancy}{
    \fancyhf{} % Clear all headers and footers
    \fancyhead[LE]{\nouppercase{\leftmark}}
    \fancyhead[RO]{Optimización energética para vivienda}
    \fancyfoot[LE]{\thepage}
    \fancyfoot[RE]{Escuela Técnica Superior de Ingenieros Industriales (UPM)}
    \fancyfoot[LO]{Luis D. Aranda Sánchez}
    \fancyfoot[RO]{\thepage}
    \renewcommand{\headrulewidth}{0.4pt}
    \renewcommand{\footrulewidth}{0.4pt}
}

\fancypagestyle{simple}{
    \fancyhf{} % Clear all headers and footers
    \renewcommand{\headrulewidth}{0pt}
    \renewcommand{\footrulewidth}{0pt}
}

% Line spacing
\setstretch{1.2}

% Document starts here
\begin{document}

% Portada
\begin{titlepage}
    \centering
    {\scshape\LARGE Universidad Politécnica de Madrid \par}
    \vspace{1cm}
    {\scshape\Large Escuela Técnica Superior de Ingenieros Industriales\par}
    \vspace{1.5cm}
    {\huge\bfseries Optimización energética de sistema híbrido con bomba de calor, suelo radiante, fotovoltaica y almacenamiento para vivienda \par}
    \vspace{1.5cm}
    {\Large\bfseries Trabajo de Fin de Máster\par}
    \vspace{0.5cm}
    {\large Máster Universitario en Ingeniería de la Energía \par}
    \vspace{2cm}
    {\Large Luis D. Aranda Sánchez\par}
    \vfill
    Director: Javier Rodríguez Martín
    \vfill
    {\large Septiembre 6, 2024\par}
\end{titlepage}

% Resumen (máximo de 5 páginas, incluyendo al final Palabras clave)
\clearpage
\pagestyle{simple}
% \newpage
\chapter*{Resumen}
\addcontentsline{toc}{chapter}{Resumen}
\input{capitulos/resumen/main.tex}

% Índice (paginado)
\clearpage
\pagestyle{simple}
% \newpage
\tableofcontents

% Introducción (donde se incluya los antecedentes y justificación)
\clearpage
\pagestyle{myfancy}
\newpage
\chapter{Introducción}
\input{capitulos/introduccion/main.tex}

% Objetivos
\chapter{Objetivos}
\input{capitulos/objetivos/main.tex}

% Metodología
\chapter{Metodología}
\input{capitulos/metodologia/main.tex}

% Resultados y discusión (incluyendo la valoración de impactos y de aspectos de responsabilidad legal, ética y profesional relacionados con el trabajo)
\chapter{Resultados y Discusión}
\input{capitulos/resultados_discusion/main.tex}

% Conclusiones
\chapter{Conclusiones}
\input{capitulos/conclusiones/main.tex}

% Planificación temporal y presupuesto
\chapter{Planificación Temporal y Presupuesto}
\input{capitulos/planificacion_presupuesto/main.tex}

% Bibliografía
\newpage
\addcontentsline{toc}{chapter}{Bibliografía}
\printbibliography

\end{document}


% Bibliografía
\newpage
\addcontentsline{toc}{chapter}{Bibliografía}
\printbibliography

\end{document}


% Metodología
\chapter{Metodología}
\documentclass[a4paper,11pt,twoside]{report}
\usepackage[left=25mm,right=25mm,top=25mm,bottom=25mm,includehead,includefoot,headsep=15mm,footskip=15mm]{geometry}
\usepackage{graphicx}
\usepackage{fancyhdr}
\usepackage{titlesec}
\usepackage[spanish]{babel}
\usepackage[utf8]{inputenc}
\usepackage{amsmath}
\usepackage{setspace}
\usepackage{svg}
\usepackage{hyperref}
\usepackage[backend=biber,style=numeric]{biblatex}
\addbibresource{references.bib}
\hypersetup{
    colorlinks=true,
    linkcolor=blue,      % color of internal links (sections, etc.)
    urlcolor=blue,       % color of external links
    pdftitle={Optimización energética de sistema híbrido con bomba de calor, suelo radiante, fotovoltaica y almacenamiento para vivienda},    % title
    pdfauthor={Luis D. Aranda Sánchez},     % author
    pdfkeywords={palabra1, palabra2, código1, etc.} % list of keywords
}

% Font change to Arial
\usepackage{helvet}
\renewcommand{\familydefault}{\sfdefault}

% Chapter titles in uppercase and larger font
\titleformat{\chapter}[hang]{\large\bfseries}{\thechapter.}{1em}{\MakeUppercase}
\titleformat{\section}[hang]{\bfseries}{\thesection.}{1em}{}
\titleformat{\subsection}[hang]{\bfseries}{\thesubsection.}{1em}{}

% Fancyhdr setup
\setlength{\headheight}{14.30174pt} % Adjust to recommended value, slightly larger for safety
\fancyhf{} % Clear all headers and footers
\fancyhead[LE]{\nouppercase{\leftmark}}
\fancyhead[RO]{Optimización energética para vivienda}
\fancyfoot[LE]{\thepage}
\fancyfoot[RE]{Escuela Técnica Superior de Ingenieros Industriales (UPM)}
\fancyfoot[LO]{Luis D. Aranda Sánchez}
\fancyfoot[RO]{\thepage}
\renewcommand{\headrulewidth}{0.4pt}
\renewcommand{\footrulewidth}{0.4pt}

\fancypagestyle{myfancy}{
    \fancyhf{} % Clear all headers and footers
    \fancyhead[LE]{\nouppercase{\leftmark}}
    \fancyhead[RO]{Optimización energética para vivienda}
    \fancyfoot[LE]{\thepage}
    \fancyfoot[RE]{Escuela Técnica Superior de Ingenieros Industriales (UPM)}
    \fancyfoot[LO]{Luis D. Aranda Sánchez}
    \fancyfoot[RO]{\thepage}
    \renewcommand{\headrulewidth}{0.4pt}
    \renewcommand{\footrulewidth}{0.4pt}
}

\fancypagestyle{simple}{
    \fancyhf{} % Clear all headers and footers
    \renewcommand{\headrulewidth}{0pt}
    \renewcommand{\footrulewidth}{0pt}
}

% Line spacing
\setstretch{1.2}

% Document starts here
\begin{document}

% Portada
\begin{titlepage}
    \centering
    {\scshape\LARGE Universidad Politécnica de Madrid \par}
    \vspace{1cm}
    {\scshape\Large Escuela Técnica Superior de Ingenieros Industriales\par}
    \vspace{1.5cm}
    {\huge\bfseries Optimización energética de sistema híbrido con bomba de calor, suelo radiante, fotovoltaica y almacenamiento para vivienda \par}
    \vspace{1.5cm}
    {\Large\bfseries Trabajo de Fin de Máster\par}
    \vspace{0.5cm}
    {\large Máster Universitario en Ingeniería de la Energía \par}
    \vspace{2cm}
    {\Large Luis D. Aranda Sánchez\par}
    \vfill
    Director: Javier Rodríguez Martín
    \vfill
    {\large Septiembre 6, 2024\par}
\end{titlepage}

% Resumen (máximo de 5 páginas, incluyendo al final Palabras clave)
\clearpage
\pagestyle{simple}
% \newpage
\chapter*{Resumen}
\addcontentsline{toc}{chapter}{Resumen}
\documentclass[a4paper,11pt,twoside]{report}
\usepackage[left=25mm,right=25mm,top=25mm,bottom=25mm,includehead,includefoot,headsep=15mm,footskip=15mm]{geometry}
\usepackage{graphicx}
\usepackage{fancyhdr}
\usepackage{titlesec}
\usepackage[spanish]{babel}
\usepackage[utf8]{inputenc}
\usepackage{amsmath}
\usepackage{setspace}
\usepackage{svg}
\usepackage{hyperref}
\usepackage[backend=biber,style=numeric]{biblatex}
\addbibresource{references.bib}
\hypersetup{
    colorlinks=true,
    linkcolor=blue,      % color of internal links (sections, etc.)
    urlcolor=blue,       % color of external links
    pdftitle={Optimización energética de sistema híbrido con bomba de calor, suelo radiante, fotovoltaica y almacenamiento para vivienda},    % title
    pdfauthor={Luis D. Aranda Sánchez},     % author
    pdfkeywords={palabra1, palabra2, código1, etc.} % list of keywords
}

% Font change to Arial
\usepackage{helvet}
\renewcommand{\familydefault}{\sfdefault}

% Chapter titles in uppercase and larger font
\titleformat{\chapter}[hang]{\large\bfseries}{\thechapter.}{1em}{\MakeUppercase}
\titleformat{\section}[hang]{\bfseries}{\thesection.}{1em}{}
\titleformat{\subsection}[hang]{\bfseries}{\thesubsection.}{1em}{}

% Fancyhdr setup
\setlength{\headheight}{14.30174pt} % Adjust to recommended value, slightly larger for safety
\fancyhf{} % Clear all headers and footers
\fancyhead[LE]{\nouppercase{\leftmark}}
\fancyhead[RO]{Optimización energética para vivienda}
\fancyfoot[LE]{\thepage}
\fancyfoot[RE]{Escuela Técnica Superior de Ingenieros Industriales (UPM)}
\fancyfoot[LO]{Luis D. Aranda Sánchez}
\fancyfoot[RO]{\thepage}
\renewcommand{\headrulewidth}{0.4pt}
\renewcommand{\footrulewidth}{0.4pt}

\fancypagestyle{myfancy}{
    \fancyhf{} % Clear all headers and footers
    \fancyhead[LE]{\nouppercase{\leftmark}}
    \fancyhead[RO]{Optimización energética para vivienda}
    \fancyfoot[LE]{\thepage}
    \fancyfoot[RE]{Escuela Técnica Superior de Ingenieros Industriales (UPM)}
    \fancyfoot[LO]{Luis D. Aranda Sánchez}
    \fancyfoot[RO]{\thepage}
    \renewcommand{\headrulewidth}{0.4pt}
    \renewcommand{\footrulewidth}{0.4pt}
}

\fancypagestyle{simple}{
    \fancyhf{} % Clear all headers and footers
    \renewcommand{\headrulewidth}{0pt}
    \renewcommand{\footrulewidth}{0pt}
}

% Line spacing
\setstretch{1.2}

% Document starts here
\begin{document}

% Portada
\begin{titlepage}
    \centering
    {\scshape\LARGE Universidad Politécnica de Madrid \par}
    \vspace{1cm}
    {\scshape\Large Escuela Técnica Superior de Ingenieros Industriales\par}
    \vspace{1.5cm}
    {\huge\bfseries Optimización energética de sistema híbrido con bomba de calor, suelo radiante, fotovoltaica y almacenamiento para vivienda \par}
    \vspace{1.5cm}
    {\Large\bfseries Trabajo de Fin de Máster\par}
    \vspace{0.5cm}
    {\large Máster Universitario en Ingeniería de la Energía \par}
    \vspace{2cm}
    {\Large Luis D. Aranda Sánchez\par}
    \vfill
    Director: Javier Rodríguez Martín
    \vfill
    {\large Septiembre 6, 2024\par}
\end{titlepage}

% Resumen (máximo de 5 páginas, incluyendo al final Palabras clave)
\clearpage
\pagestyle{simple}
% \newpage
\chapter*{Resumen}
\addcontentsline{toc}{chapter}{Resumen}
\input{capitulos/resumen/main.tex}

% Índice (paginado)
\clearpage
\pagestyle{simple}
% \newpage
\tableofcontents

% Introducción (donde se incluya los antecedentes y justificación)
\clearpage
\pagestyle{myfancy}
\newpage
\chapter{Introducción}
\input{capitulos/introduccion/main.tex}

% Objetivos
\chapter{Objetivos}
\input{capitulos/objetivos/main.tex}

% Metodología
\chapter{Metodología}
\input{capitulos/metodologia/main.tex}

% Resultados y discusión (incluyendo la valoración de impactos y de aspectos de responsabilidad legal, ética y profesional relacionados con el trabajo)
\chapter{Resultados y Discusión}
\input{capitulos/resultados_discusion/main.tex}

% Conclusiones
\chapter{Conclusiones}
\input{capitulos/conclusiones/main.tex}

% Planificación temporal y presupuesto
\chapter{Planificación Temporal y Presupuesto}
\input{capitulos/planificacion_presupuesto/main.tex}

% Bibliografía
\newpage
\addcontentsline{toc}{chapter}{Bibliografía}
\printbibliography

\end{document}


% Índice (paginado)
\clearpage
\pagestyle{simple}
% \newpage
\tableofcontents

% Introducción (donde se incluya los antecedentes y justificación)
\clearpage
\pagestyle{myfancy}
\newpage
\chapter{Introducción}
\documentclass[a4paper,11pt,twoside]{report}
\usepackage[left=25mm,right=25mm,top=25mm,bottom=25mm,includehead,includefoot,headsep=15mm,footskip=15mm]{geometry}
\usepackage{graphicx}
\usepackage{fancyhdr}
\usepackage{titlesec}
\usepackage[spanish]{babel}
\usepackage[utf8]{inputenc}
\usepackage{amsmath}
\usepackage{setspace}
\usepackage{svg}
\usepackage{hyperref}
\usepackage[backend=biber,style=numeric]{biblatex}
\addbibresource{references.bib}
\hypersetup{
    colorlinks=true,
    linkcolor=blue,      % color of internal links (sections, etc.)
    urlcolor=blue,       % color of external links
    pdftitle={Optimización energética de sistema híbrido con bomba de calor, suelo radiante, fotovoltaica y almacenamiento para vivienda},    % title
    pdfauthor={Luis D. Aranda Sánchez},     % author
    pdfkeywords={palabra1, palabra2, código1, etc.} % list of keywords
}

% Font change to Arial
\usepackage{helvet}
\renewcommand{\familydefault}{\sfdefault}

% Chapter titles in uppercase and larger font
\titleformat{\chapter}[hang]{\large\bfseries}{\thechapter.}{1em}{\MakeUppercase}
\titleformat{\section}[hang]{\bfseries}{\thesection.}{1em}{}
\titleformat{\subsection}[hang]{\bfseries}{\thesubsection.}{1em}{}

% Fancyhdr setup
\setlength{\headheight}{14.30174pt} % Adjust to recommended value, slightly larger for safety
\fancyhf{} % Clear all headers and footers
\fancyhead[LE]{\nouppercase{\leftmark}}
\fancyhead[RO]{Optimización energética para vivienda}
\fancyfoot[LE]{\thepage}
\fancyfoot[RE]{Escuela Técnica Superior de Ingenieros Industriales (UPM)}
\fancyfoot[LO]{Luis D. Aranda Sánchez}
\fancyfoot[RO]{\thepage}
\renewcommand{\headrulewidth}{0.4pt}
\renewcommand{\footrulewidth}{0.4pt}

\fancypagestyle{myfancy}{
    \fancyhf{} % Clear all headers and footers
    \fancyhead[LE]{\nouppercase{\leftmark}}
    \fancyhead[RO]{Optimización energética para vivienda}
    \fancyfoot[LE]{\thepage}
    \fancyfoot[RE]{Escuela Técnica Superior de Ingenieros Industriales (UPM)}
    \fancyfoot[LO]{Luis D. Aranda Sánchez}
    \fancyfoot[RO]{\thepage}
    \renewcommand{\headrulewidth}{0.4pt}
    \renewcommand{\footrulewidth}{0.4pt}
}

\fancypagestyle{simple}{
    \fancyhf{} % Clear all headers and footers
    \renewcommand{\headrulewidth}{0pt}
    \renewcommand{\footrulewidth}{0pt}
}

% Line spacing
\setstretch{1.2}

% Document starts here
\begin{document}

% Portada
\begin{titlepage}
    \centering
    {\scshape\LARGE Universidad Politécnica de Madrid \par}
    \vspace{1cm}
    {\scshape\Large Escuela Técnica Superior de Ingenieros Industriales\par}
    \vspace{1.5cm}
    {\huge\bfseries Optimización energética de sistema híbrido con bomba de calor, suelo radiante, fotovoltaica y almacenamiento para vivienda \par}
    \vspace{1.5cm}
    {\Large\bfseries Trabajo de Fin de Máster\par}
    \vspace{0.5cm}
    {\large Máster Universitario en Ingeniería de la Energía \par}
    \vspace{2cm}
    {\Large Luis D. Aranda Sánchez\par}
    \vfill
    Director: Javier Rodríguez Martín
    \vfill
    {\large Septiembre 6, 2024\par}
\end{titlepage}

% Resumen (máximo de 5 páginas, incluyendo al final Palabras clave)
\clearpage
\pagestyle{simple}
% \newpage
\chapter*{Resumen}
\addcontentsline{toc}{chapter}{Resumen}
\input{capitulos/resumen/main.tex}

% Índice (paginado)
\clearpage
\pagestyle{simple}
% \newpage
\tableofcontents

% Introducción (donde se incluya los antecedentes y justificación)
\clearpage
\pagestyle{myfancy}
\newpage
\chapter{Introducción}
\input{capitulos/introduccion/main.tex}

% Objetivos
\chapter{Objetivos}
\input{capitulos/objetivos/main.tex}

% Metodología
\chapter{Metodología}
\input{capitulos/metodologia/main.tex}

% Resultados y discusión (incluyendo la valoración de impactos y de aspectos de responsabilidad legal, ética y profesional relacionados con el trabajo)
\chapter{Resultados y Discusión}
\input{capitulos/resultados_discusion/main.tex}

% Conclusiones
\chapter{Conclusiones}
\input{capitulos/conclusiones/main.tex}

% Planificación temporal y presupuesto
\chapter{Planificación Temporal y Presupuesto}
\input{capitulos/planificacion_presupuesto/main.tex}

% Bibliografía
\newpage
\addcontentsline{toc}{chapter}{Bibliografía}
\printbibliography

\end{document}


% Objetivos
\chapter{Objetivos}
\documentclass[a4paper,11pt,twoside]{report}
\usepackage[left=25mm,right=25mm,top=25mm,bottom=25mm,includehead,includefoot,headsep=15mm,footskip=15mm]{geometry}
\usepackage{graphicx}
\usepackage{fancyhdr}
\usepackage{titlesec}
\usepackage[spanish]{babel}
\usepackage[utf8]{inputenc}
\usepackage{amsmath}
\usepackage{setspace}
\usepackage{svg}
\usepackage{hyperref}
\usepackage[backend=biber,style=numeric]{biblatex}
\addbibresource{references.bib}
\hypersetup{
    colorlinks=true,
    linkcolor=blue,      % color of internal links (sections, etc.)
    urlcolor=blue,       % color of external links
    pdftitle={Optimización energética de sistema híbrido con bomba de calor, suelo radiante, fotovoltaica y almacenamiento para vivienda},    % title
    pdfauthor={Luis D. Aranda Sánchez},     % author
    pdfkeywords={palabra1, palabra2, código1, etc.} % list of keywords
}

% Font change to Arial
\usepackage{helvet}
\renewcommand{\familydefault}{\sfdefault}

% Chapter titles in uppercase and larger font
\titleformat{\chapter}[hang]{\large\bfseries}{\thechapter.}{1em}{\MakeUppercase}
\titleformat{\section}[hang]{\bfseries}{\thesection.}{1em}{}
\titleformat{\subsection}[hang]{\bfseries}{\thesubsection.}{1em}{}

% Fancyhdr setup
\setlength{\headheight}{14.30174pt} % Adjust to recommended value, slightly larger for safety
\fancyhf{} % Clear all headers and footers
\fancyhead[LE]{\nouppercase{\leftmark}}
\fancyhead[RO]{Optimización energética para vivienda}
\fancyfoot[LE]{\thepage}
\fancyfoot[RE]{Escuela Técnica Superior de Ingenieros Industriales (UPM)}
\fancyfoot[LO]{Luis D. Aranda Sánchez}
\fancyfoot[RO]{\thepage}
\renewcommand{\headrulewidth}{0.4pt}
\renewcommand{\footrulewidth}{0.4pt}

\fancypagestyle{myfancy}{
    \fancyhf{} % Clear all headers and footers
    \fancyhead[LE]{\nouppercase{\leftmark}}
    \fancyhead[RO]{Optimización energética para vivienda}
    \fancyfoot[LE]{\thepage}
    \fancyfoot[RE]{Escuela Técnica Superior de Ingenieros Industriales (UPM)}
    \fancyfoot[LO]{Luis D. Aranda Sánchez}
    \fancyfoot[RO]{\thepage}
    \renewcommand{\headrulewidth}{0.4pt}
    \renewcommand{\footrulewidth}{0.4pt}
}

\fancypagestyle{simple}{
    \fancyhf{} % Clear all headers and footers
    \renewcommand{\headrulewidth}{0pt}
    \renewcommand{\footrulewidth}{0pt}
}

% Line spacing
\setstretch{1.2}

% Document starts here
\begin{document}

% Portada
\begin{titlepage}
    \centering
    {\scshape\LARGE Universidad Politécnica de Madrid \par}
    \vspace{1cm}
    {\scshape\Large Escuela Técnica Superior de Ingenieros Industriales\par}
    \vspace{1.5cm}
    {\huge\bfseries Optimización energética de sistema híbrido con bomba de calor, suelo radiante, fotovoltaica y almacenamiento para vivienda \par}
    \vspace{1.5cm}
    {\Large\bfseries Trabajo de Fin de Máster\par}
    \vspace{0.5cm}
    {\large Máster Universitario en Ingeniería de la Energía \par}
    \vspace{2cm}
    {\Large Luis D. Aranda Sánchez\par}
    \vfill
    Director: Javier Rodríguez Martín
    \vfill
    {\large Septiembre 6, 2024\par}
\end{titlepage}

% Resumen (máximo de 5 páginas, incluyendo al final Palabras clave)
\clearpage
\pagestyle{simple}
% \newpage
\chapter*{Resumen}
\addcontentsline{toc}{chapter}{Resumen}
\input{capitulos/resumen/main.tex}

% Índice (paginado)
\clearpage
\pagestyle{simple}
% \newpage
\tableofcontents

% Introducción (donde se incluya los antecedentes y justificación)
\clearpage
\pagestyle{myfancy}
\newpage
\chapter{Introducción}
\input{capitulos/introduccion/main.tex}

% Objetivos
\chapter{Objetivos}
\input{capitulos/objetivos/main.tex}

% Metodología
\chapter{Metodología}
\input{capitulos/metodologia/main.tex}

% Resultados y discusión (incluyendo la valoración de impactos y de aspectos de responsabilidad legal, ética y profesional relacionados con el trabajo)
\chapter{Resultados y Discusión}
\input{capitulos/resultados_discusion/main.tex}

% Conclusiones
\chapter{Conclusiones}
\input{capitulos/conclusiones/main.tex}

% Planificación temporal y presupuesto
\chapter{Planificación Temporal y Presupuesto}
\input{capitulos/planificacion_presupuesto/main.tex}

% Bibliografía
\newpage
\addcontentsline{toc}{chapter}{Bibliografía}
\printbibliography

\end{document}


% Metodología
\chapter{Metodología}
\documentclass[a4paper,11pt,twoside]{report}
\usepackage[left=25mm,right=25mm,top=25mm,bottom=25mm,includehead,includefoot,headsep=15mm,footskip=15mm]{geometry}
\usepackage{graphicx}
\usepackage{fancyhdr}
\usepackage{titlesec}
\usepackage[spanish]{babel}
\usepackage[utf8]{inputenc}
\usepackage{amsmath}
\usepackage{setspace}
\usepackage{svg}
\usepackage{hyperref}
\usepackage[backend=biber,style=numeric]{biblatex}
\addbibresource{references.bib}
\hypersetup{
    colorlinks=true,
    linkcolor=blue,      % color of internal links (sections, etc.)
    urlcolor=blue,       % color of external links
    pdftitle={Optimización energética de sistema híbrido con bomba de calor, suelo radiante, fotovoltaica y almacenamiento para vivienda},    % title
    pdfauthor={Luis D. Aranda Sánchez},     % author
    pdfkeywords={palabra1, palabra2, código1, etc.} % list of keywords
}

% Font change to Arial
\usepackage{helvet}
\renewcommand{\familydefault}{\sfdefault}

% Chapter titles in uppercase and larger font
\titleformat{\chapter}[hang]{\large\bfseries}{\thechapter.}{1em}{\MakeUppercase}
\titleformat{\section}[hang]{\bfseries}{\thesection.}{1em}{}
\titleformat{\subsection}[hang]{\bfseries}{\thesubsection.}{1em}{}

% Fancyhdr setup
\setlength{\headheight}{14.30174pt} % Adjust to recommended value, slightly larger for safety
\fancyhf{} % Clear all headers and footers
\fancyhead[LE]{\nouppercase{\leftmark}}
\fancyhead[RO]{Optimización energética para vivienda}
\fancyfoot[LE]{\thepage}
\fancyfoot[RE]{Escuela Técnica Superior de Ingenieros Industriales (UPM)}
\fancyfoot[LO]{Luis D. Aranda Sánchez}
\fancyfoot[RO]{\thepage}
\renewcommand{\headrulewidth}{0.4pt}
\renewcommand{\footrulewidth}{0.4pt}

\fancypagestyle{myfancy}{
    \fancyhf{} % Clear all headers and footers
    \fancyhead[LE]{\nouppercase{\leftmark}}
    \fancyhead[RO]{Optimización energética para vivienda}
    \fancyfoot[LE]{\thepage}
    \fancyfoot[RE]{Escuela Técnica Superior de Ingenieros Industriales (UPM)}
    \fancyfoot[LO]{Luis D. Aranda Sánchez}
    \fancyfoot[RO]{\thepage}
    \renewcommand{\headrulewidth}{0.4pt}
    \renewcommand{\footrulewidth}{0.4pt}
}

\fancypagestyle{simple}{
    \fancyhf{} % Clear all headers and footers
    \renewcommand{\headrulewidth}{0pt}
    \renewcommand{\footrulewidth}{0pt}
}

% Line spacing
\setstretch{1.2}

% Document starts here
\begin{document}

% Portada
\begin{titlepage}
    \centering
    {\scshape\LARGE Universidad Politécnica de Madrid \par}
    \vspace{1cm}
    {\scshape\Large Escuela Técnica Superior de Ingenieros Industriales\par}
    \vspace{1.5cm}
    {\huge\bfseries Optimización energética de sistema híbrido con bomba de calor, suelo radiante, fotovoltaica y almacenamiento para vivienda \par}
    \vspace{1.5cm}
    {\Large\bfseries Trabajo de Fin de Máster\par}
    \vspace{0.5cm}
    {\large Máster Universitario en Ingeniería de la Energía \par}
    \vspace{2cm}
    {\Large Luis D. Aranda Sánchez\par}
    \vfill
    Director: Javier Rodríguez Martín
    \vfill
    {\large Septiembre 6, 2024\par}
\end{titlepage}

% Resumen (máximo de 5 páginas, incluyendo al final Palabras clave)
\clearpage
\pagestyle{simple}
% \newpage
\chapter*{Resumen}
\addcontentsline{toc}{chapter}{Resumen}
\input{capitulos/resumen/main.tex}

% Índice (paginado)
\clearpage
\pagestyle{simple}
% \newpage
\tableofcontents

% Introducción (donde se incluya los antecedentes y justificación)
\clearpage
\pagestyle{myfancy}
\newpage
\chapter{Introducción}
\input{capitulos/introduccion/main.tex}

% Objetivos
\chapter{Objetivos}
\input{capitulos/objetivos/main.tex}

% Metodología
\chapter{Metodología}
\input{capitulos/metodologia/main.tex}

% Resultados y discusión (incluyendo la valoración de impactos y de aspectos de responsabilidad legal, ética y profesional relacionados con el trabajo)
\chapter{Resultados y Discusión}
\input{capitulos/resultados_discusion/main.tex}

% Conclusiones
\chapter{Conclusiones}
\input{capitulos/conclusiones/main.tex}

% Planificación temporal y presupuesto
\chapter{Planificación Temporal y Presupuesto}
\input{capitulos/planificacion_presupuesto/main.tex}

% Bibliografía
\newpage
\addcontentsline{toc}{chapter}{Bibliografía}
\printbibliography

\end{document}


% Resultados y discusión (incluyendo la valoración de impactos y de aspectos de responsabilidad legal, ética y profesional relacionados con el trabajo)
\chapter{Resultados y Discusión}
\documentclass[a4paper,11pt,twoside]{report}
\usepackage[left=25mm,right=25mm,top=25mm,bottom=25mm,includehead,includefoot,headsep=15mm,footskip=15mm]{geometry}
\usepackage{graphicx}
\usepackage{fancyhdr}
\usepackage{titlesec}
\usepackage[spanish]{babel}
\usepackage[utf8]{inputenc}
\usepackage{amsmath}
\usepackage{setspace}
\usepackage{svg}
\usepackage{hyperref}
\usepackage[backend=biber,style=numeric]{biblatex}
\addbibresource{references.bib}
\hypersetup{
    colorlinks=true,
    linkcolor=blue,      % color of internal links (sections, etc.)
    urlcolor=blue,       % color of external links
    pdftitle={Optimización energética de sistema híbrido con bomba de calor, suelo radiante, fotovoltaica y almacenamiento para vivienda},    % title
    pdfauthor={Luis D. Aranda Sánchez},     % author
    pdfkeywords={palabra1, palabra2, código1, etc.} % list of keywords
}

% Font change to Arial
\usepackage{helvet}
\renewcommand{\familydefault}{\sfdefault}

% Chapter titles in uppercase and larger font
\titleformat{\chapter}[hang]{\large\bfseries}{\thechapter.}{1em}{\MakeUppercase}
\titleformat{\section}[hang]{\bfseries}{\thesection.}{1em}{}
\titleformat{\subsection}[hang]{\bfseries}{\thesubsection.}{1em}{}

% Fancyhdr setup
\setlength{\headheight}{14.30174pt} % Adjust to recommended value, slightly larger for safety
\fancyhf{} % Clear all headers and footers
\fancyhead[LE]{\nouppercase{\leftmark}}
\fancyhead[RO]{Optimización energética para vivienda}
\fancyfoot[LE]{\thepage}
\fancyfoot[RE]{Escuela Técnica Superior de Ingenieros Industriales (UPM)}
\fancyfoot[LO]{Luis D. Aranda Sánchez}
\fancyfoot[RO]{\thepage}
\renewcommand{\headrulewidth}{0.4pt}
\renewcommand{\footrulewidth}{0.4pt}

\fancypagestyle{myfancy}{
    \fancyhf{} % Clear all headers and footers
    \fancyhead[LE]{\nouppercase{\leftmark}}
    \fancyhead[RO]{Optimización energética para vivienda}
    \fancyfoot[LE]{\thepage}
    \fancyfoot[RE]{Escuela Técnica Superior de Ingenieros Industriales (UPM)}
    \fancyfoot[LO]{Luis D. Aranda Sánchez}
    \fancyfoot[RO]{\thepage}
    \renewcommand{\headrulewidth}{0.4pt}
    \renewcommand{\footrulewidth}{0.4pt}
}

\fancypagestyle{simple}{
    \fancyhf{} % Clear all headers and footers
    \renewcommand{\headrulewidth}{0pt}
    \renewcommand{\footrulewidth}{0pt}
}

% Line spacing
\setstretch{1.2}

% Document starts here
\begin{document}

% Portada
\begin{titlepage}
    \centering
    {\scshape\LARGE Universidad Politécnica de Madrid \par}
    \vspace{1cm}
    {\scshape\Large Escuela Técnica Superior de Ingenieros Industriales\par}
    \vspace{1.5cm}
    {\huge\bfseries Optimización energética de sistema híbrido con bomba de calor, suelo radiante, fotovoltaica y almacenamiento para vivienda \par}
    \vspace{1.5cm}
    {\Large\bfseries Trabajo de Fin de Máster\par}
    \vspace{0.5cm}
    {\large Máster Universitario en Ingeniería de la Energía \par}
    \vspace{2cm}
    {\Large Luis D. Aranda Sánchez\par}
    \vfill
    Director: Javier Rodríguez Martín
    \vfill
    {\large Septiembre 6, 2024\par}
\end{titlepage}

% Resumen (máximo de 5 páginas, incluyendo al final Palabras clave)
\clearpage
\pagestyle{simple}
% \newpage
\chapter*{Resumen}
\addcontentsline{toc}{chapter}{Resumen}
\input{capitulos/resumen/main.tex}

% Índice (paginado)
\clearpage
\pagestyle{simple}
% \newpage
\tableofcontents

% Introducción (donde se incluya los antecedentes y justificación)
\clearpage
\pagestyle{myfancy}
\newpage
\chapter{Introducción}
\input{capitulos/introduccion/main.tex}

% Objetivos
\chapter{Objetivos}
\input{capitulos/objetivos/main.tex}

% Metodología
\chapter{Metodología}
\input{capitulos/metodologia/main.tex}

% Resultados y discusión (incluyendo la valoración de impactos y de aspectos de responsabilidad legal, ética y profesional relacionados con el trabajo)
\chapter{Resultados y Discusión}
\input{capitulos/resultados_discusion/main.tex}

% Conclusiones
\chapter{Conclusiones}
\input{capitulos/conclusiones/main.tex}

% Planificación temporal y presupuesto
\chapter{Planificación Temporal y Presupuesto}
\input{capitulos/planificacion_presupuesto/main.tex}

% Bibliografía
\newpage
\addcontentsline{toc}{chapter}{Bibliografía}
\printbibliography

\end{document}


% Conclusiones
\chapter{Conclusiones}
\documentclass[a4paper,11pt,twoside]{report}
\usepackage[left=25mm,right=25mm,top=25mm,bottom=25mm,includehead,includefoot,headsep=15mm,footskip=15mm]{geometry}
\usepackage{graphicx}
\usepackage{fancyhdr}
\usepackage{titlesec}
\usepackage[spanish]{babel}
\usepackage[utf8]{inputenc}
\usepackage{amsmath}
\usepackage{setspace}
\usepackage{svg}
\usepackage{hyperref}
\usepackage[backend=biber,style=numeric]{biblatex}
\addbibresource{references.bib}
\hypersetup{
    colorlinks=true,
    linkcolor=blue,      % color of internal links (sections, etc.)
    urlcolor=blue,       % color of external links
    pdftitle={Optimización energética de sistema híbrido con bomba de calor, suelo radiante, fotovoltaica y almacenamiento para vivienda},    % title
    pdfauthor={Luis D. Aranda Sánchez},     % author
    pdfkeywords={palabra1, palabra2, código1, etc.} % list of keywords
}

% Font change to Arial
\usepackage{helvet}
\renewcommand{\familydefault}{\sfdefault}

% Chapter titles in uppercase and larger font
\titleformat{\chapter}[hang]{\large\bfseries}{\thechapter.}{1em}{\MakeUppercase}
\titleformat{\section}[hang]{\bfseries}{\thesection.}{1em}{}
\titleformat{\subsection}[hang]{\bfseries}{\thesubsection.}{1em}{}

% Fancyhdr setup
\setlength{\headheight}{14.30174pt} % Adjust to recommended value, slightly larger for safety
\fancyhf{} % Clear all headers and footers
\fancyhead[LE]{\nouppercase{\leftmark}}
\fancyhead[RO]{Optimización energética para vivienda}
\fancyfoot[LE]{\thepage}
\fancyfoot[RE]{Escuela Técnica Superior de Ingenieros Industriales (UPM)}
\fancyfoot[LO]{Luis D. Aranda Sánchez}
\fancyfoot[RO]{\thepage}
\renewcommand{\headrulewidth}{0.4pt}
\renewcommand{\footrulewidth}{0.4pt}

\fancypagestyle{myfancy}{
    \fancyhf{} % Clear all headers and footers
    \fancyhead[LE]{\nouppercase{\leftmark}}
    \fancyhead[RO]{Optimización energética para vivienda}
    \fancyfoot[LE]{\thepage}
    \fancyfoot[RE]{Escuela Técnica Superior de Ingenieros Industriales (UPM)}
    \fancyfoot[LO]{Luis D. Aranda Sánchez}
    \fancyfoot[RO]{\thepage}
    \renewcommand{\headrulewidth}{0.4pt}
    \renewcommand{\footrulewidth}{0.4pt}
}

\fancypagestyle{simple}{
    \fancyhf{} % Clear all headers and footers
    \renewcommand{\headrulewidth}{0pt}
    \renewcommand{\footrulewidth}{0pt}
}

% Line spacing
\setstretch{1.2}

% Document starts here
\begin{document}

% Portada
\begin{titlepage}
    \centering
    {\scshape\LARGE Universidad Politécnica de Madrid \par}
    \vspace{1cm}
    {\scshape\Large Escuela Técnica Superior de Ingenieros Industriales\par}
    \vspace{1.5cm}
    {\huge\bfseries Optimización energética de sistema híbrido con bomba de calor, suelo radiante, fotovoltaica y almacenamiento para vivienda \par}
    \vspace{1.5cm}
    {\Large\bfseries Trabajo de Fin de Máster\par}
    \vspace{0.5cm}
    {\large Máster Universitario en Ingeniería de la Energía \par}
    \vspace{2cm}
    {\Large Luis D. Aranda Sánchez\par}
    \vfill
    Director: Javier Rodríguez Martín
    \vfill
    {\large Septiembre 6, 2024\par}
\end{titlepage}

% Resumen (máximo de 5 páginas, incluyendo al final Palabras clave)
\clearpage
\pagestyle{simple}
% \newpage
\chapter*{Resumen}
\addcontentsline{toc}{chapter}{Resumen}
\input{capitulos/resumen/main.tex}

% Índice (paginado)
\clearpage
\pagestyle{simple}
% \newpage
\tableofcontents

% Introducción (donde se incluya los antecedentes y justificación)
\clearpage
\pagestyle{myfancy}
\newpage
\chapter{Introducción}
\input{capitulos/introduccion/main.tex}

% Objetivos
\chapter{Objetivos}
\input{capitulos/objetivos/main.tex}

% Metodología
\chapter{Metodología}
\input{capitulos/metodologia/main.tex}

% Resultados y discusión (incluyendo la valoración de impactos y de aspectos de responsabilidad legal, ética y profesional relacionados con el trabajo)
\chapter{Resultados y Discusión}
\input{capitulos/resultados_discusion/main.tex}

% Conclusiones
\chapter{Conclusiones}
\input{capitulos/conclusiones/main.tex}

% Planificación temporal y presupuesto
\chapter{Planificación Temporal y Presupuesto}
\input{capitulos/planificacion_presupuesto/main.tex}

% Bibliografía
\newpage
\addcontentsline{toc}{chapter}{Bibliografía}
\printbibliography

\end{document}


% Planificación temporal y presupuesto
\chapter{Planificación Temporal y Presupuesto}
\documentclass[a4paper,11pt,twoside]{report}
\usepackage[left=25mm,right=25mm,top=25mm,bottom=25mm,includehead,includefoot,headsep=15mm,footskip=15mm]{geometry}
\usepackage{graphicx}
\usepackage{fancyhdr}
\usepackage{titlesec}
\usepackage[spanish]{babel}
\usepackage[utf8]{inputenc}
\usepackage{amsmath}
\usepackage{setspace}
\usepackage{svg}
\usepackage{hyperref}
\usepackage[backend=biber,style=numeric]{biblatex}
\addbibresource{references.bib}
\hypersetup{
    colorlinks=true,
    linkcolor=blue,      % color of internal links (sections, etc.)
    urlcolor=blue,       % color of external links
    pdftitle={Optimización energética de sistema híbrido con bomba de calor, suelo radiante, fotovoltaica y almacenamiento para vivienda},    % title
    pdfauthor={Luis D. Aranda Sánchez},     % author
    pdfkeywords={palabra1, palabra2, código1, etc.} % list of keywords
}

% Font change to Arial
\usepackage{helvet}
\renewcommand{\familydefault}{\sfdefault}

% Chapter titles in uppercase and larger font
\titleformat{\chapter}[hang]{\large\bfseries}{\thechapter.}{1em}{\MakeUppercase}
\titleformat{\section}[hang]{\bfseries}{\thesection.}{1em}{}
\titleformat{\subsection}[hang]{\bfseries}{\thesubsection.}{1em}{}

% Fancyhdr setup
\setlength{\headheight}{14.30174pt} % Adjust to recommended value, slightly larger for safety
\fancyhf{} % Clear all headers and footers
\fancyhead[LE]{\nouppercase{\leftmark}}
\fancyhead[RO]{Optimización energética para vivienda}
\fancyfoot[LE]{\thepage}
\fancyfoot[RE]{Escuela Técnica Superior de Ingenieros Industriales (UPM)}
\fancyfoot[LO]{Luis D. Aranda Sánchez}
\fancyfoot[RO]{\thepage}
\renewcommand{\headrulewidth}{0.4pt}
\renewcommand{\footrulewidth}{0.4pt}

\fancypagestyle{myfancy}{
    \fancyhf{} % Clear all headers and footers
    \fancyhead[LE]{\nouppercase{\leftmark}}
    \fancyhead[RO]{Optimización energética para vivienda}
    \fancyfoot[LE]{\thepage}
    \fancyfoot[RE]{Escuela Técnica Superior de Ingenieros Industriales (UPM)}
    \fancyfoot[LO]{Luis D. Aranda Sánchez}
    \fancyfoot[RO]{\thepage}
    \renewcommand{\headrulewidth}{0.4pt}
    \renewcommand{\footrulewidth}{0.4pt}
}

\fancypagestyle{simple}{
    \fancyhf{} % Clear all headers and footers
    \renewcommand{\headrulewidth}{0pt}
    \renewcommand{\footrulewidth}{0pt}
}

% Line spacing
\setstretch{1.2}

% Document starts here
\begin{document}

% Portada
\begin{titlepage}
    \centering
    {\scshape\LARGE Universidad Politécnica de Madrid \par}
    \vspace{1cm}
    {\scshape\Large Escuela Técnica Superior de Ingenieros Industriales\par}
    \vspace{1.5cm}
    {\huge\bfseries Optimización energética de sistema híbrido con bomba de calor, suelo radiante, fotovoltaica y almacenamiento para vivienda \par}
    \vspace{1.5cm}
    {\Large\bfseries Trabajo de Fin de Máster\par}
    \vspace{0.5cm}
    {\large Máster Universitario en Ingeniería de la Energía \par}
    \vspace{2cm}
    {\Large Luis D. Aranda Sánchez\par}
    \vfill
    Director: Javier Rodríguez Martín
    \vfill
    {\large Septiembre 6, 2024\par}
\end{titlepage}

% Resumen (máximo de 5 páginas, incluyendo al final Palabras clave)
\clearpage
\pagestyle{simple}
% \newpage
\chapter*{Resumen}
\addcontentsline{toc}{chapter}{Resumen}
\input{capitulos/resumen/main.tex}

% Índice (paginado)
\clearpage
\pagestyle{simple}
% \newpage
\tableofcontents

% Introducción (donde se incluya los antecedentes y justificación)
\clearpage
\pagestyle{myfancy}
\newpage
\chapter{Introducción}
\input{capitulos/introduccion/main.tex}

% Objetivos
\chapter{Objetivos}
\input{capitulos/objetivos/main.tex}

% Metodología
\chapter{Metodología}
\input{capitulos/metodologia/main.tex}

% Resultados y discusión (incluyendo la valoración de impactos y de aspectos de responsabilidad legal, ética y profesional relacionados con el trabajo)
\chapter{Resultados y Discusión}
\input{capitulos/resultados_discusion/main.tex}

% Conclusiones
\chapter{Conclusiones}
\input{capitulos/conclusiones/main.tex}

% Planificación temporal y presupuesto
\chapter{Planificación Temporal y Presupuesto}
\input{capitulos/planificacion_presupuesto/main.tex}

% Bibliografía
\newpage
\addcontentsline{toc}{chapter}{Bibliografía}
\printbibliography

\end{document}


% Bibliografía
\newpage
\addcontentsline{toc}{chapter}{Bibliografía}
\printbibliography

\end{document}


% Resultados y discusión (incluyendo la valoración de impactos y de aspectos de responsabilidad legal, ética y profesional relacionados con el trabajo)
\chapter{Resultados y Discusión}
\documentclass[a4paper,11pt,twoside]{report}
\usepackage[left=25mm,right=25mm,top=25mm,bottom=25mm,includehead,includefoot,headsep=15mm,footskip=15mm]{geometry}
\usepackage{graphicx}
\usepackage{fancyhdr}
\usepackage{titlesec}
\usepackage[spanish]{babel}
\usepackage[utf8]{inputenc}
\usepackage{amsmath}
\usepackage{setspace}
\usepackage{svg}
\usepackage{hyperref}
\usepackage[backend=biber,style=numeric]{biblatex}
\addbibresource{references.bib}
\hypersetup{
    colorlinks=true,
    linkcolor=blue,      % color of internal links (sections, etc.)
    urlcolor=blue,       % color of external links
    pdftitle={Optimización energética de sistema híbrido con bomba de calor, suelo radiante, fotovoltaica y almacenamiento para vivienda},    % title
    pdfauthor={Luis D. Aranda Sánchez},     % author
    pdfkeywords={palabra1, palabra2, código1, etc.} % list of keywords
}

% Font change to Arial
\usepackage{helvet}
\renewcommand{\familydefault}{\sfdefault}

% Chapter titles in uppercase and larger font
\titleformat{\chapter}[hang]{\large\bfseries}{\thechapter.}{1em}{\MakeUppercase}
\titleformat{\section}[hang]{\bfseries}{\thesection.}{1em}{}
\titleformat{\subsection}[hang]{\bfseries}{\thesubsection.}{1em}{}

% Fancyhdr setup
\setlength{\headheight}{14.30174pt} % Adjust to recommended value, slightly larger for safety
\fancyhf{} % Clear all headers and footers
\fancyhead[LE]{\nouppercase{\leftmark}}
\fancyhead[RO]{Optimización energética para vivienda}
\fancyfoot[LE]{\thepage}
\fancyfoot[RE]{Escuela Técnica Superior de Ingenieros Industriales (UPM)}
\fancyfoot[LO]{Luis D. Aranda Sánchez}
\fancyfoot[RO]{\thepage}
\renewcommand{\headrulewidth}{0.4pt}
\renewcommand{\footrulewidth}{0.4pt}

\fancypagestyle{myfancy}{
    \fancyhf{} % Clear all headers and footers
    \fancyhead[LE]{\nouppercase{\leftmark}}
    \fancyhead[RO]{Optimización energética para vivienda}
    \fancyfoot[LE]{\thepage}
    \fancyfoot[RE]{Escuela Técnica Superior de Ingenieros Industriales (UPM)}
    \fancyfoot[LO]{Luis D. Aranda Sánchez}
    \fancyfoot[RO]{\thepage}
    \renewcommand{\headrulewidth}{0.4pt}
    \renewcommand{\footrulewidth}{0.4pt}
}

\fancypagestyle{simple}{
    \fancyhf{} % Clear all headers and footers
    \renewcommand{\headrulewidth}{0pt}
    \renewcommand{\footrulewidth}{0pt}
}

% Line spacing
\setstretch{1.2}

% Document starts here
\begin{document}

% Portada
\begin{titlepage}
    \centering
    {\scshape\LARGE Universidad Politécnica de Madrid \par}
    \vspace{1cm}
    {\scshape\Large Escuela Técnica Superior de Ingenieros Industriales\par}
    \vspace{1.5cm}
    {\huge\bfseries Optimización energética de sistema híbrido con bomba de calor, suelo radiante, fotovoltaica y almacenamiento para vivienda \par}
    \vspace{1.5cm}
    {\Large\bfseries Trabajo de Fin de Máster\par}
    \vspace{0.5cm}
    {\large Máster Universitario en Ingeniería de la Energía \par}
    \vspace{2cm}
    {\Large Luis D. Aranda Sánchez\par}
    \vfill
    Director: Javier Rodríguez Martín
    \vfill
    {\large Septiembre 6, 2024\par}
\end{titlepage}

% Resumen (máximo de 5 páginas, incluyendo al final Palabras clave)
\clearpage
\pagestyle{simple}
% \newpage
\chapter*{Resumen}
\addcontentsline{toc}{chapter}{Resumen}
\documentclass[a4paper,11pt,twoside]{report}
\usepackage[left=25mm,right=25mm,top=25mm,bottom=25mm,includehead,includefoot,headsep=15mm,footskip=15mm]{geometry}
\usepackage{graphicx}
\usepackage{fancyhdr}
\usepackage{titlesec}
\usepackage[spanish]{babel}
\usepackage[utf8]{inputenc}
\usepackage{amsmath}
\usepackage{setspace}
\usepackage{svg}
\usepackage{hyperref}
\usepackage[backend=biber,style=numeric]{biblatex}
\addbibresource{references.bib}
\hypersetup{
    colorlinks=true,
    linkcolor=blue,      % color of internal links (sections, etc.)
    urlcolor=blue,       % color of external links
    pdftitle={Optimización energética de sistema híbrido con bomba de calor, suelo radiante, fotovoltaica y almacenamiento para vivienda},    % title
    pdfauthor={Luis D. Aranda Sánchez},     % author
    pdfkeywords={palabra1, palabra2, código1, etc.} % list of keywords
}

% Font change to Arial
\usepackage{helvet}
\renewcommand{\familydefault}{\sfdefault}

% Chapter titles in uppercase and larger font
\titleformat{\chapter}[hang]{\large\bfseries}{\thechapter.}{1em}{\MakeUppercase}
\titleformat{\section}[hang]{\bfseries}{\thesection.}{1em}{}
\titleformat{\subsection}[hang]{\bfseries}{\thesubsection.}{1em}{}

% Fancyhdr setup
\setlength{\headheight}{14.30174pt} % Adjust to recommended value, slightly larger for safety
\fancyhf{} % Clear all headers and footers
\fancyhead[LE]{\nouppercase{\leftmark}}
\fancyhead[RO]{Optimización energética para vivienda}
\fancyfoot[LE]{\thepage}
\fancyfoot[RE]{Escuela Técnica Superior de Ingenieros Industriales (UPM)}
\fancyfoot[LO]{Luis D. Aranda Sánchez}
\fancyfoot[RO]{\thepage}
\renewcommand{\headrulewidth}{0.4pt}
\renewcommand{\footrulewidth}{0.4pt}

\fancypagestyle{myfancy}{
    \fancyhf{} % Clear all headers and footers
    \fancyhead[LE]{\nouppercase{\leftmark}}
    \fancyhead[RO]{Optimización energética para vivienda}
    \fancyfoot[LE]{\thepage}
    \fancyfoot[RE]{Escuela Técnica Superior de Ingenieros Industriales (UPM)}
    \fancyfoot[LO]{Luis D. Aranda Sánchez}
    \fancyfoot[RO]{\thepage}
    \renewcommand{\headrulewidth}{0.4pt}
    \renewcommand{\footrulewidth}{0.4pt}
}

\fancypagestyle{simple}{
    \fancyhf{} % Clear all headers and footers
    \renewcommand{\headrulewidth}{0pt}
    \renewcommand{\footrulewidth}{0pt}
}

% Line spacing
\setstretch{1.2}

% Document starts here
\begin{document}

% Portada
\begin{titlepage}
    \centering
    {\scshape\LARGE Universidad Politécnica de Madrid \par}
    \vspace{1cm}
    {\scshape\Large Escuela Técnica Superior de Ingenieros Industriales\par}
    \vspace{1.5cm}
    {\huge\bfseries Optimización energética de sistema híbrido con bomba de calor, suelo radiante, fotovoltaica y almacenamiento para vivienda \par}
    \vspace{1.5cm}
    {\Large\bfseries Trabajo de Fin de Máster\par}
    \vspace{0.5cm}
    {\large Máster Universitario en Ingeniería de la Energía \par}
    \vspace{2cm}
    {\Large Luis D. Aranda Sánchez\par}
    \vfill
    Director: Javier Rodríguez Martín
    \vfill
    {\large Septiembre 6, 2024\par}
\end{titlepage}

% Resumen (máximo de 5 páginas, incluyendo al final Palabras clave)
\clearpage
\pagestyle{simple}
% \newpage
\chapter*{Resumen}
\addcontentsline{toc}{chapter}{Resumen}
\input{capitulos/resumen/main.tex}

% Índice (paginado)
\clearpage
\pagestyle{simple}
% \newpage
\tableofcontents

% Introducción (donde se incluya los antecedentes y justificación)
\clearpage
\pagestyle{myfancy}
\newpage
\chapter{Introducción}
\input{capitulos/introduccion/main.tex}

% Objetivos
\chapter{Objetivos}
\input{capitulos/objetivos/main.tex}

% Metodología
\chapter{Metodología}
\input{capitulos/metodologia/main.tex}

% Resultados y discusión (incluyendo la valoración de impactos y de aspectos de responsabilidad legal, ética y profesional relacionados con el trabajo)
\chapter{Resultados y Discusión}
\input{capitulos/resultados_discusion/main.tex}

% Conclusiones
\chapter{Conclusiones}
\input{capitulos/conclusiones/main.tex}

% Planificación temporal y presupuesto
\chapter{Planificación Temporal y Presupuesto}
\input{capitulos/planificacion_presupuesto/main.tex}

% Bibliografía
\newpage
\addcontentsline{toc}{chapter}{Bibliografía}
\printbibliography

\end{document}


% Índice (paginado)
\clearpage
\pagestyle{simple}
% \newpage
\tableofcontents

% Introducción (donde se incluya los antecedentes y justificación)
\clearpage
\pagestyle{myfancy}
\newpage
\chapter{Introducción}
\documentclass[a4paper,11pt,twoside]{report}
\usepackage[left=25mm,right=25mm,top=25mm,bottom=25mm,includehead,includefoot,headsep=15mm,footskip=15mm]{geometry}
\usepackage{graphicx}
\usepackage{fancyhdr}
\usepackage{titlesec}
\usepackage[spanish]{babel}
\usepackage[utf8]{inputenc}
\usepackage{amsmath}
\usepackage{setspace}
\usepackage{svg}
\usepackage{hyperref}
\usepackage[backend=biber,style=numeric]{biblatex}
\addbibresource{references.bib}
\hypersetup{
    colorlinks=true,
    linkcolor=blue,      % color of internal links (sections, etc.)
    urlcolor=blue,       % color of external links
    pdftitle={Optimización energética de sistema híbrido con bomba de calor, suelo radiante, fotovoltaica y almacenamiento para vivienda},    % title
    pdfauthor={Luis D. Aranda Sánchez},     % author
    pdfkeywords={palabra1, palabra2, código1, etc.} % list of keywords
}

% Font change to Arial
\usepackage{helvet}
\renewcommand{\familydefault}{\sfdefault}

% Chapter titles in uppercase and larger font
\titleformat{\chapter}[hang]{\large\bfseries}{\thechapter.}{1em}{\MakeUppercase}
\titleformat{\section}[hang]{\bfseries}{\thesection.}{1em}{}
\titleformat{\subsection}[hang]{\bfseries}{\thesubsection.}{1em}{}

% Fancyhdr setup
\setlength{\headheight}{14.30174pt} % Adjust to recommended value, slightly larger for safety
\fancyhf{} % Clear all headers and footers
\fancyhead[LE]{\nouppercase{\leftmark}}
\fancyhead[RO]{Optimización energética para vivienda}
\fancyfoot[LE]{\thepage}
\fancyfoot[RE]{Escuela Técnica Superior de Ingenieros Industriales (UPM)}
\fancyfoot[LO]{Luis D. Aranda Sánchez}
\fancyfoot[RO]{\thepage}
\renewcommand{\headrulewidth}{0.4pt}
\renewcommand{\footrulewidth}{0.4pt}

\fancypagestyle{myfancy}{
    \fancyhf{} % Clear all headers and footers
    \fancyhead[LE]{\nouppercase{\leftmark}}
    \fancyhead[RO]{Optimización energética para vivienda}
    \fancyfoot[LE]{\thepage}
    \fancyfoot[RE]{Escuela Técnica Superior de Ingenieros Industriales (UPM)}
    \fancyfoot[LO]{Luis D. Aranda Sánchez}
    \fancyfoot[RO]{\thepage}
    \renewcommand{\headrulewidth}{0.4pt}
    \renewcommand{\footrulewidth}{0.4pt}
}

\fancypagestyle{simple}{
    \fancyhf{} % Clear all headers and footers
    \renewcommand{\headrulewidth}{0pt}
    \renewcommand{\footrulewidth}{0pt}
}

% Line spacing
\setstretch{1.2}

% Document starts here
\begin{document}

% Portada
\begin{titlepage}
    \centering
    {\scshape\LARGE Universidad Politécnica de Madrid \par}
    \vspace{1cm}
    {\scshape\Large Escuela Técnica Superior de Ingenieros Industriales\par}
    \vspace{1.5cm}
    {\huge\bfseries Optimización energética de sistema híbrido con bomba de calor, suelo radiante, fotovoltaica y almacenamiento para vivienda \par}
    \vspace{1.5cm}
    {\Large\bfseries Trabajo de Fin de Máster\par}
    \vspace{0.5cm}
    {\large Máster Universitario en Ingeniería de la Energía \par}
    \vspace{2cm}
    {\Large Luis D. Aranda Sánchez\par}
    \vfill
    Director: Javier Rodríguez Martín
    \vfill
    {\large Septiembre 6, 2024\par}
\end{titlepage}

% Resumen (máximo de 5 páginas, incluyendo al final Palabras clave)
\clearpage
\pagestyle{simple}
% \newpage
\chapter*{Resumen}
\addcontentsline{toc}{chapter}{Resumen}
\input{capitulos/resumen/main.tex}

% Índice (paginado)
\clearpage
\pagestyle{simple}
% \newpage
\tableofcontents

% Introducción (donde se incluya los antecedentes y justificación)
\clearpage
\pagestyle{myfancy}
\newpage
\chapter{Introducción}
\input{capitulos/introduccion/main.tex}

% Objetivos
\chapter{Objetivos}
\input{capitulos/objetivos/main.tex}

% Metodología
\chapter{Metodología}
\input{capitulos/metodologia/main.tex}

% Resultados y discusión (incluyendo la valoración de impactos y de aspectos de responsabilidad legal, ética y profesional relacionados con el trabajo)
\chapter{Resultados y Discusión}
\input{capitulos/resultados_discusion/main.tex}

% Conclusiones
\chapter{Conclusiones}
\input{capitulos/conclusiones/main.tex}

% Planificación temporal y presupuesto
\chapter{Planificación Temporal y Presupuesto}
\input{capitulos/planificacion_presupuesto/main.tex}

% Bibliografía
\newpage
\addcontentsline{toc}{chapter}{Bibliografía}
\printbibliography

\end{document}


% Objetivos
\chapter{Objetivos}
\documentclass[a4paper,11pt,twoside]{report}
\usepackage[left=25mm,right=25mm,top=25mm,bottom=25mm,includehead,includefoot,headsep=15mm,footskip=15mm]{geometry}
\usepackage{graphicx}
\usepackage{fancyhdr}
\usepackage{titlesec}
\usepackage[spanish]{babel}
\usepackage[utf8]{inputenc}
\usepackage{amsmath}
\usepackage{setspace}
\usepackage{svg}
\usepackage{hyperref}
\usepackage[backend=biber,style=numeric]{biblatex}
\addbibresource{references.bib}
\hypersetup{
    colorlinks=true,
    linkcolor=blue,      % color of internal links (sections, etc.)
    urlcolor=blue,       % color of external links
    pdftitle={Optimización energética de sistema híbrido con bomba de calor, suelo radiante, fotovoltaica y almacenamiento para vivienda},    % title
    pdfauthor={Luis D. Aranda Sánchez},     % author
    pdfkeywords={palabra1, palabra2, código1, etc.} % list of keywords
}

% Font change to Arial
\usepackage{helvet}
\renewcommand{\familydefault}{\sfdefault}

% Chapter titles in uppercase and larger font
\titleformat{\chapter}[hang]{\large\bfseries}{\thechapter.}{1em}{\MakeUppercase}
\titleformat{\section}[hang]{\bfseries}{\thesection.}{1em}{}
\titleformat{\subsection}[hang]{\bfseries}{\thesubsection.}{1em}{}

% Fancyhdr setup
\setlength{\headheight}{14.30174pt} % Adjust to recommended value, slightly larger for safety
\fancyhf{} % Clear all headers and footers
\fancyhead[LE]{\nouppercase{\leftmark}}
\fancyhead[RO]{Optimización energética para vivienda}
\fancyfoot[LE]{\thepage}
\fancyfoot[RE]{Escuela Técnica Superior de Ingenieros Industriales (UPM)}
\fancyfoot[LO]{Luis D. Aranda Sánchez}
\fancyfoot[RO]{\thepage}
\renewcommand{\headrulewidth}{0.4pt}
\renewcommand{\footrulewidth}{0.4pt}

\fancypagestyle{myfancy}{
    \fancyhf{} % Clear all headers and footers
    \fancyhead[LE]{\nouppercase{\leftmark}}
    \fancyhead[RO]{Optimización energética para vivienda}
    \fancyfoot[LE]{\thepage}
    \fancyfoot[RE]{Escuela Técnica Superior de Ingenieros Industriales (UPM)}
    \fancyfoot[LO]{Luis D. Aranda Sánchez}
    \fancyfoot[RO]{\thepage}
    \renewcommand{\headrulewidth}{0.4pt}
    \renewcommand{\footrulewidth}{0.4pt}
}

\fancypagestyle{simple}{
    \fancyhf{} % Clear all headers and footers
    \renewcommand{\headrulewidth}{0pt}
    \renewcommand{\footrulewidth}{0pt}
}

% Line spacing
\setstretch{1.2}

% Document starts here
\begin{document}

% Portada
\begin{titlepage}
    \centering
    {\scshape\LARGE Universidad Politécnica de Madrid \par}
    \vspace{1cm}
    {\scshape\Large Escuela Técnica Superior de Ingenieros Industriales\par}
    \vspace{1.5cm}
    {\huge\bfseries Optimización energética de sistema híbrido con bomba de calor, suelo radiante, fotovoltaica y almacenamiento para vivienda \par}
    \vspace{1.5cm}
    {\Large\bfseries Trabajo de Fin de Máster\par}
    \vspace{0.5cm}
    {\large Máster Universitario en Ingeniería de la Energía \par}
    \vspace{2cm}
    {\Large Luis D. Aranda Sánchez\par}
    \vfill
    Director: Javier Rodríguez Martín
    \vfill
    {\large Septiembre 6, 2024\par}
\end{titlepage}

% Resumen (máximo de 5 páginas, incluyendo al final Palabras clave)
\clearpage
\pagestyle{simple}
% \newpage
\chapter*{Resumen}
\addcontentsline{toc}{chapter}{Resumen}
\input{capitulos/resumen/main.tex}

% Índice (paginado)
\clearpage
\pagestyle{simple}
% \newpage
\tableofcontents

% Introducción (donde se incluya los antecedentes y justificación)
\clearpage
\pagestyle{myfancy}
\newpage
\chapter{Introducción}
\input{capitulos/introduccion/main.tex}

% Objetivos
\chapter{Objetivos}
\input{capitulos/objetivos/main.tex}

% Metodología
\chapter{Metodología}
\input{capitulos/metodologia/main.tex}

% Resultados y discusión (incluyendo la valoración de impactos y de aspectos de responsabilidad legal, ética y profesional relacionados con el trabajo)
\chapter{Resultados y Discusión}
\input{capitulos/resultados_discusion/main.tex}

% Conclusiones
\chapter{Conclusiones}
\input{capitulos/conclusiones/main.tex}

% Planificación temporal y presupuesto
\chapter{Planificación Temporal y Presupuesto}
\input{capitulos/planificacion_presupuesto/main.tex}

% Bibliografía
\newpage
\addcontentsline{toc}{chapter}{Bibliografía}
\printbibliography

\end{document}


% Metodología
\chapter{Metodología}
\documentclass[a4paper,11pt,twoside]{report}
\usepackage[left=25mm,right=25mm,top=25mm,bottom=25mm,includehead,includefoot,headsep=15mm,footskip=15mm]{geometry}
\usepackage{graphicx}
\usepackage{fancyhdr}
\usepackage{titlesec}
\usepackage[spanish]{babel}
\usepackage[utf8]{inputenc}
\usepackage{amsmath}
\usepackage{setspace}
\usepackage{svg}
\usepackage{hyperref}
\usepackage[backend=biber,style=numeric]{biblatex}
\addbibresource{references.bib}
\hypersetup{
    colorlinks=true,
    linkcolor=blue,      % color of internal links (sections, etc.)
    urlcolor=blue,       % color of external links
    pdftitle={Optimización energética de sistema híbrido con bomba de calor, suelo radiante, fotovoltaica y almacenamiento para vivienda},    % title
    pdfauthor={Luis D. Aranda Sánchez},     % author
    pdfkeywords={palabra1, palabra2, código1, etc.} % list of keywords
}

% Font change to Arial
\usepackage{helvet}
\renewcommand{\familydefault}{\sfdefault}

% Chapter titles in uppercase and larger font
\titleformat{\chapter}[hang]{\large\bfseries}{\thechapter.}{1em}{\MakeUppercase}
\titleformat{\section}[hang]{\bfseries}{\thesection.}{1em}{}
\titleformat{\subsection}[hang]{\bfseries}{\thesubsection.}{1em}{}

% Fancyhdr setup
\setlength{\headheight}{14.30174pt} % Adjust to recommended value, slightly larger for safety
\fancyhf{} % Clear all headers and footers
\fancyhead[LE]{\nouppercase{\leftmark}}
\fancyhead[RO]{Optimización energética para vivienda}
\fancyfoot[LE]{\thepage}
\fancyfoot[RE]{Escuela Técnica Superior de Ingenieros Industriales (UPM)}
\fancyfoot[LO]{Luis D. Aranda Sánchez}
\fancyfoot[RO]{\thepage}
\renewcommand{\headrulewidth}{0.4pt}
\renewcommand{\footrulewidth}{0.4pt}

\fancypagestyle{myfancy}{
    \fancyhf{} % Clear all headers and footers
    \fancyhead[LE]{\nouppercase{\leftmark}}
    \fancyhead[RO]{Optimización energética para vivienda}
    \fancyfoot[LE]{\thepage}
    \fancyfoot[RE]{Escuela Técnica Superior de Ingenieros Industriales (UPM)}
    \fancyfoot[LO]{Luis D. Aranda Sánchez}
    \fancyfoot[RO]{\thepage}
    \renewcommand{\headrulewidth}{0.4pt}
    \renewcommand{\footrulewidth}{0.4pt}
}

\fancypagestyle{simple}{
    \fancyhf{} % Clear all headers and footers
    \renewcommand{\headrulewidth}{0pt}
    \renewcommand{\footrulewidth}{0pt}
}

% Line spacing
\setstretch{1.2}

% Document starts here
\begin{document}

% Portada
\begin{titlepage}
    \centering
    {\scshape\LARGE Universidad Politécnica de Madrid \par}
    \vspace{1cm}
    {\scshape\Large Escuela Técnica Superior de Ingenieros Industriales\par}
    \vspace{1.5cm}
    {\huge\bfseries Optimización energética de sistema híbrido con bomba de calor, suelo radiante, fotovoltaica y almacenamiento para vivienda \par}
    \vspace{1.5cm}
    {\Large\bfseries Trabajo de Fin de Máster\par}
    \vspace{0.5cm}
    {\large Máster Universitario en Ingeniería de la Energía \par}
    \vspace{2cm}
    {\Large Luis D. Aranda Sánchez\par}
    \vfill
    Director: Javier Rodríguez Martín
    \vfill
    {\large Septiembre 6, 2024\par}
\end{titlepage}

% Resumen (máximo de 5 páginas, incluyendo al final Palabras clave)
\clearpage
\pagestyle{simple}
% \newpage
\chapter*{Resumen}
\addcontentsline{toc}{chapter}{Resumen}
\input{capitulos/resumen/main.tex}

% Índice (paginado)
\clearpage
\pagestyle{simple}
% \newpage
\tableofcontents

% Introducción (donde se incluya los antecedentes y justificación)
\clearpage
\pagestyle{myfancy}
\newpage
\chapter{Introducción}
\input{capitulos/introduccion/main.tex}

% Objetivos
\chapter{Objetivos}
\input{capitulos/objetivos/main.tex}

% Metodología
\chapter{Metodología}
\input{capitulos/metodologia/main.tex}

% Resultados y discusión (incluyendo la valoración de impactos y de aspectos de responsabilidad legal, ética y profesional relacionados con el trabajo)
\chapter{Resultados y Discusión}
\input{capitulos/resultados_discusion/main.tex}

% Conclusiones
\chapter{Conclusiones}
\input{capitulos/conclusiones/main.tex}

% Planificación temporal y presupuesto
\chapter{Planificación Temporal y Presupuesto}
\input{capitulos/planificacion_presupuesto/main.tex}

% Bibliografía
\newpage
\addcontentsline{toc}{chapter}{Bibliografía}
\printbibliography

\end{document}


% Resultados y discusión (incluyendo la valoración de impactos y de aspectos de responsabilidad legal, ética y profesional relacionados con el trabajo)
\chapter{Resultados y Discusión}
\documentclass[a4paper,11pt,twoside]{report}
\usepackage[left=25mm,right=25mm,top=25mm,bottom=25mm,includehead,includefoot,headsep=15mm,footskip=15mm]{geometry}
\usepackage{graphicx}
\usepackage{fancyhdr}
\usepackage{titlesec}
\usepackage[spanish]{babel}
\usepackage[utf8]{inputenc}
\usepackage{amsmath}
\usepackage{setspace}
\usepackage{svg}
\usepackage{hyperref}
\usepackage[backend=biber,style=numeric]{biblatex}
\addbibresource{references.bib}
\hypersetup{
    colorlinks=true,
    linkcolor=blue,      % color of internal links (sections, etc.)
    urlcolor=blue,       % color of external links
    pdftitle={Optimización energética de sistema híbrido con bomba de calor, suelo radiante, fotovoltaica y almacenamiento para vivienda},    % title
    pdfauthor={Luis D. Aranda Sánchez},     % author
    pdfkeywords={palabra1, palabra2, código1, etc.} % list of keywords
}

% Font change to Arial
\usepackage{helvet}
\renewcommand{\familydefault}{\sfdefault}

% Chapter titles in uppercase and larger font
\titleformat{\chapter}[hang]{\large\bfseries}{\thechapter.}{1em}{\MakeUppercase}
\titleformat{\section}[hang]{\bfseries}{\thesection.}{1em}{}
\titleformat{\subsection}[hang]{\bfseries}{\thesubsection.}{1em}{}

% Fancyhdr setup
\setlength{\headheight}{14.30174pt} % Adjust to recommended value, slightly larger for safety
\fancyhf{} % Clear all headers and footers
\fancyhead[LE]{\nouppercase{\leftmark}}
\fancyhead[RO]{Optimización energética para vivienda}
\fancyfoot[LE]{\thepage}
\fancyfoot[RE]{Escuela Técnica Superior de Ingenieros Industriales (UPM)}
\fancyfoot[LO]{Luis D. Aranda Sánchez}
\fancyfoot[RO]{\thepage}
\renewcommand{\headrulewidth}{0.4pt}
\renewcommand{\footrulewidth}{0.4pt}

\fancypagestyle{myfancy}{
    \fancyhf{} % Clear all headers and footers
    \fancyhead[LE]{\nouppercase{\leftmark}}
    \fancyhead[RO]{Optimización energética para vivienda}
    \fancyfoot[LE]{\thepage}
    \fancyfoot[RE]{Escuela Técnica Superior de Ingenieros Industriales (UPM)}
    \fancyfoot[LO]{Luis D. Aranda Sánchez}
    \fancyfoot[RO]{\thepage}
    \renewcommand{\headrulewidth}{0.4pt}
    \renewcommand{\footrulewidth}{0.4pt}
}

\fancypagestyle{simple}{
    \fancyhf{} % Clear all headers and footers
    \renewcommand{\headrulewidth}{0pt}
    \renewcommand{\footrulewidth}{0pt}
}

% Line spacing
\setstretch{1.2}

% Document starts here
\begin{document}

% Portada
\begin{titlepage}
    \centering
    {\scshape\LARGE Universidad Politécnica de Madrid \par}
    \vspace{1cm}
    {\scshape\Large Escuela Técnica Superior de Ingenieros Industriales\par}
    \vspace{1.5cm}
    {\huge\bfseries Optimización energética de sistema híbrido con bomba de calor, suelo radiante, fotovoltaica y almacenamiento para vivienda \par}
    \vspace{1.5cm}
    {\Large\bfseries Trabajo de Fin de Máster\par}
    \vspace{0.5cm}
    {\large Máster Universitario en Ingeniería de la Energía \par}
    \vspace{2cm}
    {\Large Luis D. Aranda Sánchez\par}
    \vfill
    Director: Javier Rodríguez Martín
    \vfill
    {\large Septiembre 6, 2024\par}
\end{titlepage}

% Resumen (máximo de 5 páginas, incluyendo al final Palabras clave)
\clearpage
\pagestyle{simple}
% \newpage
\chapter*{Resumen}
\addcontentsline{toc}{chapter}{Resumen}
\input{capitulos/resumen/main.tex}

% Índice (paginado)
\clearpage
\pagestyle{simple}
% \newpage
\tableofcontents

% Introducción (donde se incluya los antecedentes y justificación)
\clearpage
\pagestyle{myfancy}
\newpage
\chapter{Introducción}
\input{capitulos/introduccion/main.tex}

% Objetivos
\chapter{Objetivos}
\input{capitulos/objetivos/main.tex}

% Metodología
\chapter{Metodología}
\input{capitulos/metodologia/main.tex}

% Resultados y discusión (incluyendo la valoración de impactos y de aspectos de responsabilidad legal, ética y profesional relacionados con el trabajo)
\chapter{Resultados y Discusión}
\input{capitulos/resultados_discusion/main.tex}

% Conclusiones
\chapter{Conclusiones}
\input{capitulos/conclusiones/main.tex}

% Planificación temporal y presupuesto
\chapter{Planificación Temporal y Presupuesto}
\input{capitulos/planificacion_presupuesto/main.tex}

% Bibliografía
\newpage
\addcontentsline{toc}{chapter}{Bibliografía}
\printbibliography

\end{document}


% Conclusiones
\chapter{Conclusiones}
\documentclass[a4paper,11pt,twoside]{report}
\usepackage[left=25mm,right=25mm,top=25mm,bottom=25mm,includehead,includefoot,headsep=15mm,footskip=15mm]{geometry}
\usepackage{graphicx}
\usepackage{fancyhdr}
\usepackage{titlesec}
\usepackage[spanish]{babel}
\usepackage[utf8]{inputenc}
\usepackage{amsmath}
\usepackage{setspace}
\usepackage{svg}
\usepackage{hyperref}
\usepackage[backend=biber,style=numeric]{biblatex}
\addbibresource{references.bib}
\hypersetup{
    colorlinks=true,
    linkcolor=blue,      % color of internal links (sections, etc.)
    urlcolor=blue,       % color of external links
    pdftitle={Optimización energética de sistema híbrido con bomba de calor, suelo radiante, fotovoltaica y almacenamiento para vivienda},    % title
    pdfauthor={Luis D. Aranda Sánchez},     % author
    pdfkeywords={palabra1, palabra2, código1, etc.} % list of keywords
}

% Font change to Arial
\usepackage{helvet}
\renewcommand{\familydefault}{\sfdefault}

% Chapter titles in uppercase and larger font
\titleformat{\chapter}[hang]{\large\bfseries}{\thechapter.}{1em}{\MakeUppercase}
\titleformat{\section}[hang]{\bfseries}{\thesection.}{1em}{}
\titleformat{\subsection}[hang]{\bfseries}{\thesubsection.}{1em}{}

% Fancyhdr setup
\setlength{\headheight}{14.30174pt} % Adjust to recommended value, slightly larger for safety
\fancyhf{} % Clear all headers and footers
\fancyhead[LE]{\nouppercase{\leftmark}}
\fancyhead[RO]{Optimización energética para vivienda}
\fancyfoot[LE]{\thepage}
\fancyfoot[RE]{Escuela Técnica Superior de Ingenieros Industriales (UPM)}
\fancyfoot[LO]{Luis D. Aranda Sánchez}
\fancyfoot[RO]{\thepage}
\renewcommand{\headrulewidth}{0.4pt}
\renewcommand{\footrulewidth}{0.4pt}

\fancypagestyle{myfancy}{
    \fancyhf{} % Clear all headers and footers
    \fancyhead[LE]{\nouppercase{\leftmark}}
    \fancyhead[RO]{Optimización energética para vivienda}
    \fancyfoot[LE]{\thepage}
    \fancyfoot[RE]{Escuela Técnica Superior de Ingenieros Industriales (UPM)}
    \fancyfoot[LO]{Luis D. Aranda Sánchez}
    \fancyfoot[RO]{\thepage}
    \renewcommand{\headrulewidth}{0.4pt}
    \renewcommand{\footrulewidth}{0.4pt}
}

\fancypagestyle{simple}{
    \fancyhf{} % Clear all headers and footers
    \renewcommand{\headrulewidth}{0pt}
    \renewcommand{\footrulewidth}{0pt}
}

% Line spacing
\setstretch{1.2}

% Document starts here
\begin{document}

% Portada
\begin{titlepage}
    \centering
    {\scshape\LARGE Universidad Politécnica de Madrid \par}
    \vspace{1cm}
    {\scshape\Large Escuela Técnica Superior de Ingenieros Industriales\par}
    \vspace{1.5cm}
    {\huge\bfseries Optimización energética de sistema híbrido con bomba de calor, suelo radiante, fotovoltaica y almacenamiento para vivienda \par}
    \vspace{1.5cm}
    {\Large\bfseries Trabajo de Fin de Máster\par}
    \vspace{0.5cm}
    {\large Máster Universitario en Ingeniería de la Energía \par}
    \vspace{2cm}
    {\Large Luis D. Aranda Sánchez\par}
    \vfill
    Director: Javier Rodríguez Martín
    \vfill
    {\large Septiembre 6, 2024\par}
\end{titlepage}

% Resumen (máximo de 5 páginas, incluyendo al final Palabras clave)
\clearpage
\pagestyle{simple}
% \newpage
\chapter*{Resumen}
\addcontentsline{toc}{chapter}{Resumen}
\input{capitulos/resumen/main.tex}

% Índice (paginado)
\clearpage
\pagestyle{simple}
% \newpage
\tableofcontents

% Introducción (donde se incluya los antecedentes y justificación)
\clearpage
\pagestyle{myfancy}
\newpage
\chapter{Introducción}
\input{capitulos/introduccion/main.tex}

% Objetivos
\chapter{Objetivos}
\input{capitulos/objetivos/main.tex}

% Metodología
\chapter{Metodología}
\input{capitulos/metodologia/main.tex}

% Resultados y discusión (incluyendo la valoración de impactos y de aspectos de responsabilidad legal, ética y profesional relacionados con el trabajo)
\chapter{Resultados y Discusión}
\input{capitulos/resultados_discusion/main.tex}

% Conclusiones
\chapter{Conclusiones}
\input{capitulos/conclusiones/main.tex}

% Planificación temporal y presupuesto
\chapter{Planificación Temporal y Presupuesto}
\input{capitulos/planificacion_presupuesto/main.tex}

% Bibliografía
\newpage
\addcontentsline{toc}{chapter}{Bibliografía}
\printbibliography

\end{document}


% Planificación temporal y presupuesto
\chapter{Planificación Temporal y Presupuesto}
\documentclass[a4paper,11pt,twoside]{report}
\usepackage[left=25mm,right=25mm,top=25mm,bottom=25mm,includehead,includefoot,headsep=15mm,footskip=15mm]{geometry}
\usepackage{graphicx}
\usepackage{fancyhdr}
\usepackage{titlesec}
\usepackage[spanish]{babel}
\usepackage[utf8]{inputenc}
\usepackage{amsmath}
\usepackage{setspace}
\usepackage{svg}
\usepackage{hyperref}
\usepackage[backend=biber,style=numeric]{biblatex}
\addbibresource{references.bib}
\hypersetup{
    colorlinks=true,
    linkcolor=blue,      % color of internal links (sections, etc.)
    urlcolor=blue,       % color of external links
    pdftitle={Optimización energética de sistema híbrido con bomba de calor, suelo radiante, fotovoltaica y almacenamiento para vivienda},    % title
    pdfauthor={Luis D. Aranda Sánchez},     % author
    pdfkeywords={palabra1, palabra2, código1, etc.} % list of keywords
}

% Font change to Arial
\usepackage{helvet}
\renewcommand{\familydefault}{\sfdefault}

% Chapter titles in uppercase and larger font
\titleformat{\chapter}[hang]{\large\bfseries}{\thechapter.}{1em}{\MakeUppercase}
\titleformat{\section}[hang]{\bfseries}{\thesection.}{1em}{}
\titleformat{\subsection}[hang]{\bfseries}{\thesubsection.}{1em}{}

% Fancyhdr setup
\setlength{\headheight}{14.30174pt} % Adjust to recommended value, slightly larger for safety
\fancyhf{} % Clear all headers and footers
\fancyhead[LE]{\nouppercase{\leftmark}}
\fancyhead[RO]{Optimización energética para vivienda}
\fancyfoot[LE]{\thepage}
\fancyfoot[RE]{Escuela Técnica Superior de Ingenieros Industriales (UPM)}
\fancyfoot[LO]{Luis D. Aranda Sánchez}
\fancyfoot[RO]{\thepage}
\renewcommand{\headrulewidth}{0.4pt}
\renewcommand{\footrulewidth}{0.4pt}

\fancypagestyle{myfancy}{
    \fancyhf{} % Clear all headers and footers
    \fancyhead[LE]{\nouppercase{\leftmark}}
    \fancyhead[RO]{Optimización energética para vivienda}
    \fancyfoot[LE]{\thepage}
    \fancyfoot[RE]{Escuela Técnica Superior de Ingenieros Industriales (UPM)}
    \fancyfoot[LO]{Luis D. Aranda Sánchez}
    \fancyfoot[RO]{\thepage}
    \renewcommand{\headrulewidth}{0.4pt}
    \renewcommand{\footrulewidth}{0.4pt}
}

\fancypagestyle{simple}{
    \fancyhf{} % Clear all headers and footers
    \renewcommand{\headrulewidth}{0pt}
    \renewcommand{\footrulewidth}{0pt}
}

% Line spacing
\setstretch{1.2}

% Document starts here
\begin{document}

% Portada
\begin{titlepage}
    \centering
    {\scshape\LARGE Universidad Politécnica de Madrid \par}
    \vspace{1cm}
    {\scshape\Large Escuela Técnica Superior de Ingenieros Industriales\par}
    \vspace{1.5cm}
    {\huge\bfseries Optimización energética de sistema híbrido con bomba de calor, suelo radiante, fotovoltaica y almacenamiento para vivienda \par}
    \vspace{1.5cm}
    {\Large\bfseries Trabajo de Fin de Máster\par}
    \vspace{0.5cm}
    {\large Máster Universitario en Ingeniería de la Energía \par}
    \vspace{2cm}
    {\Large Luis D. Aranda Sánchez\par}
    \vfill
    Director: Javier Rodríguez Martín
    \vfill
    {\large Septiembre 6, 2024\par}
\end{titlepage}

% Resumen (máximo de 5 páginas, incluyendo al final Palabras clave)
\clearpage
\pagestyle{simple}
% \newpage
\chapter*{Resumen}
\addcontentsline{toc}{chapter}{Resumen}
\input{capitulos/resumen/main.tex}

% Índice (paginado)
\clearpage
\pagestyle{simple}
% \newpage
\tableofcontents

% Introducción (donde se incluya los antecedentes y justificación)
\clearpage
\pagestyle{myfancy}
\newpage
\chapter{Introducción}
\input{capitulos/introduccion/main.tex}

% Objetivos
\chapter{Objetivos}
\input{capitulos/objetivos/main.tex}

% Metodología
\chapter{Metodología}
\input{capitulos/metodologia/main.tex}

% Resultados y discusión (incluyendo la valoración de impactos y de aspectos de responsabilidad legal, ética y profesional relacionados con el trabajo)
\chapter{Resultados y Discusión}
\input{capitulos/resultados_discusion/main.tex}

% Conclusiones
\chapter{Conclusiones}
\input{capitulos/conclusiones/main.tex}

% Planificación temporal y presupuesto
\chapter{Planificación Temporal y Presupuesto}
\input{capitulos/planificacion_presupuesto/main.tex}

% Bibliografía
\newpage
\addcontentsline{toc}{chapter}{Bibliografía}
\printbibliography

\end{document}


% Bibliografía
\newpage
\addcontentsline{toc}{chapter}{Bibliografía}
\printbibliography

\end{document}


% Conclusiones
\chapter{Conclusiones}
\documentclass[a4paper,11pt,twoside]{report}
\usepackage[left=25mm,right=25mm,top=25mm,bottom=25mm,includehead,includefoot,headsep=15mm,footskip=15mm]{geometry}
\usepackage{graphicx}
\usepackage{fancyhdr}
\usepackage{titlesec}
\usepackage[spanish]{babel}
\usepackage[utf8]{inputenc}
\usepackage{amsmath}
\usepackage{setspace}
\usepackage{svg}
\usepackage{hyperref}
\usepackage[backend=biber,style=numeric]{biblatex}
\addbibresource{references.bib}
\hypersetup{
    colorlinks=true,
    linkcolor=blue,      % color of internal links (sections, etc.)
    urlcolor=blue,       % color of external links
    pdftitle={Optimización energética de sistema híbrido con bomba de calor, suelo radiante, fotovoltaica y almacenamiento para vivienda},    % title
    pdfauthor={Luis D. Aranda Sánchez},     % author
    pdfkeywords={palabra1, palabra2, código1, etc.} % list of keywords
}

% Font change to Arial
\usepackage{helvet}
\renewcommand{\familydefault}{\sfdefault}

% Chapter titles in uppercase and larger font
\titleformat{\chapter}[hang]{\large\bfseries}{\thechapter.}{1em}{\MakeUppercase}
\titleformat{\section}[hang]{\bfseries}{\thesection.}{1em}{}
\titleformat{\subsection}[hang]{\bfseries}{\thesubsection.}{1em}{}

% Fancyhdr setup
\setlength{\headheight}{14.30174pt} % Adjust to recommended value, slightly larger for safety
\fancyhf{} % Clear all headers and footers
\fancyhead[LE]{\nouppercase{\leftmark}}
\fancyhead[RO]{Optimización energética para vivienda}
\fancyfoot[LE]{\thepage}
\fancyfoot[RE]{Escuela Técnica Superior de Ingenieros Industriales (UPM)}
\fancyfoot[LO]{Luis D. Aranda Sánchez}
\fancyfoot[RO]{\thepage}
\renewcommand{\headrulewidth}{0.4pt}
\renewcommand{\footrulewidth}{0.4pt}

\fancypagestyle{myfancy}{
    \fancyhf{} % Clear all headers and footers
    \fancyhead[LE]{\nouppercase{\leftmark}}
    \fancyhead[RO]{Optimización energética para vivienda}
    \fancyfoot[LE]{\thepage}
    \fancyfoot[RE]{Escuela Técnica Superior de Ingenieros Industriales (UPM)}
    \fancyfoot[LO]{Luis D. Aranda Sánchez}
    \fancyfoot[RO]{\thepage}
    \renewcommand{\headrulewidth}{0.4pt}
    \renewcommand{\footrulewidth}{0.4pt}
}

\fancypagestyle{simple}{
    \fancyhf{} % Clear all headers and footers
    \renewcommand{\headrulewidth}{0pt}
    \renewcommand{\footrulewidth}{0pt}
}

% Line spacing
\setstretch{1.2}

% Document starts here
\begin{document}

% Portada
\begin{titlepage}
    \centering
    {\scshape\LARGE Universidad Politécnica de Madrid \par}
    \vspace{1cm}
    {\scshape\Large Escuela Técnica Superior de Ingenieros Industriales\par}
    \vspace{1.5cm}
    {\huge\bfseries Optimización energética de sistema híbrido con bomba de calor, suelo radiante, fotovoltaica y almacenamiento para vivienda \par}
    \vspace{1.5cm}
    {\Large\bfseries Trabajo de Fin de Máster\par}
    \vspace{0.5cm}
    {\large Máster Universitario en Ingeniería de la Energía \par}
    \vspace{2cm}
    {\Large Luis D. Aranda Sánchez\par}
    \vfill
    Director: Javier Rodríguez Martín
    \vfill
    {\large Septiembre 6, 2024\par}
\end{titlepage}

% Resumen (máximo de 5 páginas, incluyendo al final Palabras clave)
\clearpage
\pagestyle{simple}
% \newpage
\chapter*{Resumen}
\addcontentsline{toc}{chapter}{Resumen}
\documentclass[a4paper,11pt,twoside]{report}
\usepackage[left=25mm,right=25mm,top=25mm,bottom=25mm,includehead,includefoot,headsep=15mm,footskip=15mm]{geometry}
\usepackage{graphicx}
\usepackage{fancyhdr}
\usepackage{titlesec}
\usepackage[spanish]{babel}
\usepackage[utf8]{inputenc}
\usepackage{amsmath}
\usepackage{setspace}
\usepackage{svg}
\usepackage{hyperref}
\usepackage[backend=biber,style=numeric]{biblatex}
\addbibresource{references.bib}
\hypersetup{
    colorlinks=true,
    linkcolor=blue,      % color of internal links (sections, etc.)
    urlcolor=blue,       % color of external links
    pdftitle={Optimización energética de sistema híbrido con bomba de calor, suelo radiante, fotovoltaica y almacenamiento para vivienda},    % title
    pdfauthor={Luis D. Aranda Sánchez},     % author
    pdfkeywords={palabra1, palabra2, código1, etc.} % list of keywords
}

% Font change to Arial
\usepackage{helvet}
\renewcommand{\familydefault}{\sfdefault}

% Chapter titles in uppercase and larger font
\titleformat{\chapter}[hang]{\large\bfseries}{\thechapter.}{1em}{\MakeUppercase}
\titleformat{\section}[hang]{\bfseries}{\thesection.}{1em}{}
\titleformat{\subsection}[hang]{\bfseries}{\thesubsection.}{1em}{}

% Fancyhdr setup
\setlength{\headheight}{14.30174pt} % Adjust to recommended value, slightly larger for safety
\fancyhf{} % Clear all headers and footers
\fancyhead[LE]{\nouppercase{\leftmark}}
\fancyhead[RO]{Optimización energética para vivienda}
\fancyfoot[LE]{\thepage}
\fancyfoot[RE]{Escuela Técnica Superior de Ingenieros Industriales (UPM)}
\fancyfoot[LO]{Luis D. Aranda Sánchez}
\fancyfoot[RO]{\thepage}
\renewcommand{\headrulewidth}{0.4pt}
\renewcommand{\footrulewidth}{0.4pt}

\fancypagestyle{myfancy}{
    \fancyhf{} % Clear all headers and footers
    \fancyhead[LE]{\nouppercase{\leftmark}}
    \fancyhead[RO]{Optimización energética para vivienda}
    \fancyfoot[LE]{\thepage}
    \fancyfoot[RE]{Escuela Técnica Superior de Ingenieros Industriales (UPM)}
    \fancyfoot[LO]{Luis D. Aranda Sánchez}
    \fancyfoot[RO]{\thepage}
    \renewcommand{\headrulewidth}{0.4pt}
    \renewcommand{\footrulewidth}{0.4pt}
}

\fancypagestyle{simple}{
    \fancyhf{} % Clear all headers and footers
    \renewcommand{\headrulewidth}{0pt}
    \renewcommand{\footrulewidth}{0pt}
}

% Line spacing
\setstretch{1.2}

% Document starts here
\begin{document}

% Portada
\begin{titlepage}
    \centering
    {\scshape\LARGE Universidad Politécnica de Madrid \par}
    \vspace{1cm}
    {\scshape\Large Escuela Técnica Superior de Ingenieros Industriales\par}
    \vspace{1.5cm}
    {\huge\bfseries Optimización energética de sistema híbrido con bomba de calor, suelo radiante, fotovoltaica y almacenamiento para vivienda \par}
    \vspace{1.5cm}
    {\Large\bfseries Trabajo de Fin de Máster\par}
    \vspace{0.5cm}
    {\large Máster Universitario en Ingeniería de la Energía \par}
    \vspace{2cm}
    {\Large Luis D. Aranda Sánchez\par}
    \vfill
    Director: Javier Rodríguez Martín
    \vfill
    {\large Septiembre 6, 2024\par}
\end{titlepage}

% Resumen (máximo de 5 páginas, incluyendo al final Palabras clave)
\clearpage
\pagestyle{simple}
% \newpage
\chapter*{Resumen}
\addcontentsline{toc}{chapter}{Resumen}
\input{capitulos/resumen/main.tex}

% Índice (paginado)
\clearpage
\pagestyle{simple}
% \newpage
\tableofcontents

% Introducción (donde se incluya los antecedentes y justificación)
\clearpage
\pagestyle{myfancy}
\newpage
\chapter{Introducción}
\input{capitulos/introduccion/main.tex}

% Objetivos
\chapter{Objetivos}
\input{capitulos/objetivos/main.tex}

% Metodología
\chapter{Metodología}
\input{capitulos/metodologia/main.tex}

% Resultados y discusión (incluyendo la valoración de impactos y de aspectos de responsabilidad legal, ética y profesional relacionados con el trabajo)
\chapter{Resultados y Discusión}
\input{capitulos/resultados_discusion/main.tex}

% Conclusiones
\chapter{Conclusiones}
\input{capitulos/conclusiones/main.tex}

% Planificación temporal y presupuesto
\chapter{Planificación Temporal y Presupuesto}
\input{capitulos/planificacion_presupuesto/main.tex}

% Bibliografía
\newpage
\addcontentsline{toc}{chapter}{Bibliografía}
\printbibliography

\end{document}


% Índice (paginado)
\clearpage
\pagestyle{simple}
% \newpage
\tableofcontents

% Introducción (donde se incluya los antecedentes y justificación)
\clearpage
\pagestyle{myfancy}
\newpage
\chapter{Introducción}
\documentclass[a4paper,11pt,twoside]{report}
\usepackage[left=25mm,right=25mm,top=25mm,bottom=25mm,includehead,includefoot,headsep=15mm,footskip=15mm]{geometry}
\usepackage{graphicx}
\usepackage{fancyhdr}
\usepackage{titlesec}
\usepackage[spanish]{babel}
\usepackage[utf8]{inputenc}
\usepackage{amsmath}
\usepackage{setspace}
\usepackage{svg}
\usepackage{hyperref}
\usepackage[backend=biber,style=numeric]{biblatex}
\addbibresource{references.bib}
\hypersetup{
    colorlinks=true,
    linkcolor=blue,      % color of internal links (sections, etc.)
    urlcolor=blue,       % color of external links
    pdftitle={Optimización energética de sistema híbrido con bomba de calor, suelo radiante, fotovoltaica y almacenamiento para vivienda},    % title
    pdfauthor={Luis D. Aranda Sánchez},     % author
    pdfkeywords={palabra1, palabra2, código1, etc.} % list of keywords
}

% Font change to Arial
\usepackage{helvet}
\renewcommand{\familydefault}{\sfdefault}

% Chapter titles in uppercase and larger font
\titleformat{\chapter}[hang]{\large\bfseries}{\thechapter.}{1em}{\MakeUppercase}
\titleformat{\section}[hang]{\bfseries}{\thesection.}{1em}{}
\titleformat{\subsection}[hang]{\bfseries}{\thesubsection.}{1em}{}

% Fancyhdr setup
\setlength{\headheight}{14.30174pt} % Adjust to recommended value, slightly larger for safety
\fancyhf{} % Clear all headers and footers
\fancyhead[LE]{\nouppercase{\leftmark}}
\fancyhead[RO]{Optimización energética para vivienda}
\fancyfoot[LE]{\thepage}
\fancyfoot[RE]{Escuela Técnica Superior de Ingenieros Industriales (UPM)}
\fancyfoot[LO]{Luis D. Aranda Sánchez}
\fancyfoot[RO]{\thepage}
\renewcommand{\headrulewidth}{0.4pt}
\renewcommand{\footrulewidth}{0.4pt}

\fancypagestyle{myfancy}{
    \fancyhf{} % Clear all headers and footers
    \fancyhead[LE]{\nouppercase{\leftmark}}
    \fancyhead[RO]{Optimización energética para vivienda}
    \fancyfoot[LE]{\thepage}
    \fancyfoot[RE]{Escuela Técnica Superior de Ingenieros Industriales (UPM)}
    \fancyfoot[LO]{Luis D. Aranda Sánchez}
    \fancyfoot[RO]{\thepage}
    \renewcommand{\headrulewidth}{0.4pt}
    \renewcommand{\footrulewidth}{0.4pt}
}

\fancypagestyle{simple}{
    \fancyhf{} % Clear all headers and footers
    \renewcommand{\headrulewidth}{0pt}
    \renewcommand{\footrulewidth}{0pt}
}

% Line spacing
\setstretch{1.2}

% Document starts here
\begin{document}

% Portada
\begin{titlepage}
    \centering
    {\scshape\LARGE Universidad Politécnica de Madrid \par}
    \vspace{1cm}
    {\scshape\Large Escuela Técnica Superior de Ingenieros Industriales\par}
    \vspace{1.5cm}
    {\huge\bfseries Optimización energética de sistema híbrido con bomba de calor, suelo radiante, fotovoltaica y almacenamiento para vivienda \par}
    \vspace{1.5cm}
    {\Large\bfseries Trabajo de Fin de Máster\par}
    \vspace{0.5cm}
    {\large Máster Universitario en Ingeniería de la Energía \par}
    \vspace{2cm}
    {\Large Luis D. Aranda Sánchez\par}
    \vfill
    Director: Javier Rodríguez Martín
    \vfill
    {\large Septiembre 6, 2024\par}
\end{titlepage}

% Resumen (máximo de 5 páginas, incluyendo al final Palabras clave)
\clearpage
\pagestyle{simple}
% \newpage
\chapter*{Resumen}
\addcontentsline{toc}{chapter}{Resumen}
\input{capitulos/resumen/main.tex}

% Índice (paginado)
\clearpage
\pagestyle{simple}
% \newpage
\tableofcontents

% Introducción (donde se incluya los antecedentes y justificación)
\clearpage
\pagestyle{myfancy}
\newpage
\chapter{Introducción}
\input{capitulos/introduccion/main.tex}

% Objetivos
\chapter{Objetivos}
\input{capitulos/objetivos/main.tex}

% Metodología
\chapter{Metodología}
\input{capitulos/metodologia/main.tex}

% Resultados y discusión (incluyendo la valoración de impactos y de aspectos de responsabilidad legal, ética y profesional relacionados con el trabajo)
\chapter{Resultados y Discusión}
\input{capitulos/resultados_discusion/main.tex}

% Conclusiones
\chapter{Conclusiones}
\input{capitulos/conclusiones/main.tex}

% Planificación temporal y presupuesto
\chapter{Planificación Temporal y Presupuesto}
\input{capitulos/planificacion_presupuesto/main.tex}

% Bibliografía
\newpage
\addcontentsline{toc}{chapter}{Bibliografía}
\printbibliography

\end{document}


% Objetivos
\chapter{Objetivos}
\documentclass[a4paper,11pt,twoside]{report}
\usepackage[left=25mm,right=25mm,top=25mm,bottom=25mm,includehead,includefoot,headsep=15mm,footskip=15mm]{geometry}
\usepackage{graphicx}
\usepackage{fancyhdr}
\usepackage{titlesec}
\usepackage[spanish]{babel}
\usepackage[utf8]{inputenc}
\usepackage{amsmath}
\usepackage{setspace}
\usepackage{svg}
\usepackage{hyperref}
\usepackage[backend=biber,style=numeric]{biblatex}
\addbibresource{references.bib}
\hypersetup{
    colorlinks=true,
    linkcolor=blue,      % color of internal links (sections, etc.)
    urlcolor=blue,       % color of external links
    pdftitle={Optimización energética de sistema híbrido con bomba de calor, suelo radiante, fotovoltaica y almacenamiento para vivienda},    % title
    pdfauthor={Luis D. Aranda Sánchez},     % author
    pdfkeywords={palabra1, palabra2, código1, etc.} % list of keywords
}

% Font change to Arial
\usepackage{helvet}
\renewcommand{\familydefault}{\sfdefault}

% Chapter titles in uppercase and larger font
\titleformat{\chapter}[hang]{\large\bfseries}{\thechapter.}{1em}{\MakeUppercase}
\titleformat{\section}[hang]{\bfseries}{\thesection.}{1em}{}
\titleformat{\subsection}[hang]{\bfseries}{\thesubsection.}{1em}{}

% Fancyhdr setup
\setlength{\headheight}{14.30174pt} % Adjust to recommended value, slightly larger for safety
\fancyhf{} % Clear all headers and footers
\fancyhead[LE]{\nouppercase{\leftmark}}
\fancyhead[RO]{Optimización energética para vivienda}
\fancyfoot[LE]{\thepage}
\fancyfoot[RE]{Escuela Técnica Superior de Ingenieros Industriales (UPM)}
\fancyfoot[LO]{Luis D. Aranda Sánchez}
\fancyfoot[RO]{\thepage}
\renewcommand{\headrulewidth}{0.4pt}
\renewcommand{\footrulewidth}{0.4pt}

\fancypagestyle{myfancy}{
    \fancyhf{} % Clear all headers and footers
    \fancyhead[LE]{\nouppercase{\leftmark}}
    \fancyhead[RO]{Optimización energética para vivienda}
    \fancyfoot[LE]{\thepage}
    \fancyfoot[RE]{Escuela Técnica Superior de Ingenieros Industriales (UPM)}
    \fancyfoot[LO]{Luis D. Aranda Sánchez}
    \fancyfoot[RO]{\thepage}
    \renewcommand{\headrulewidth}{0.4pt}
    \renewcommand{\footrulewidth}{0.4pt}
}

\fancypagestyle{simple}{
    \fancyhf{} % Clear all headers and footers
    \renewcommand{\headrulewidth}{0pt}
    \renewcommand{\footrulewidth}{0pt}
}

% Line spacing
\setstretch{1.2}

% Document starts here
\begin{document}

% Portada
\begin{titlepage}
    \centering
    {\scshape\LARGE Universidad Politécnica de Madrid \par}
    \vspace{1cm}
    {\scshape\Large Escuela Técnica Superior de Ingenieros Industriales\par}
    \vspace{1.5cm}
    {\huge\bfseries Optimización energética de sistema híbrido con bomba de calor, suelo radiante, fotovoltaica y almacenamiento para vivienda \par}
    \vspace{1.5cm}
    {\Large\bfseries Trabajo de Fin de Máster\par}
    \vspace{0.5cm}
    {\large Máster Universitario en Ingeniería de la Energía \par}
    \vspace{2cm}
    {\Large Luis D. Aranda Sánchez\par}
    \vfill
    Director: Javier Rodríguez Martín
    \vfill
    {\large Septiembre 6, 2024\par}
\end{titlepage}

% Resumen (máximo de 5 páginas, incluyendo al final Palabras clave)
\clearpage
\pagestyle{simple}
% \newpage
\chapter*{Resumen}
\addcontentsline{toc}{chapter}{Resumen}
\input{capitulos/resumen/main.tex}

% Índice (paginado)
\clearpage
\pagestyle{simple}
% \newpage
\tableofcontents

% Introducción (donde se incluya los antecedentes y justificación)
\clearpage
\pagestyle{myfancy}
\newpage
\chapter{Introducción}
\input{capitulos/introduccion/main.tex}

% Objetivos
\chapter{Objetivos}
\input{capitulos/objetivos/main.tex}

% Metodología
\chapter{Metodología}
\input{capitulos/metodologia/main.tex}

% Resultados y discusión (incluyendo la valoración de impactos y de aspectos de responsabilidad legal, ética y profesional relacionados con el trabajo)
\chapter{Resultados y Discusión}
\input{capitulos/resultados_discusion/main.tex}

% Conclusiones
\chapter{Conclusiones}
\input{capitulos/conclusiones/main.tex}

% Planificación temporal y presupuesto
\chapter{Planificación Temporal y Presupuesto}
\input{capitulos/planificacion_presupuesto/main.tex}

% Bibliografía
\newpage
\addcontentsline{toc}{chapter}{Bibliografía}
\printbibliography

\end{document}


% Metodología
\chapter{Metodología}
\documentclass[a4paper,11pt,twoside]{report}
\usepackage[left=25mm,right=25mm,top=25mm,bottom=25mm,includehead,includefoot,headsep=15mm,footskip=15mm]{geometry}
\usepackage{graphicx}
\usepackage{fancyhdr}
\usepackage{titlesec}
\usepackage[spanish]{babel}
\usepackage[utf8]{inputenc}
\usepackage{amsmath}
\usepackage{setspace}
\usepackage{svg}
\usepackage{hyperref}
\usepackage[backend=biber,style=numeric]{biblatex}
\addbibresource{references.bib}
\hypersetup{
    colorlinks=true,
    linkcolor=blue,      % color of internal links (sections, etc.)
    urlcolor=blue,       % color of external links
    pdftitle={Optimización energética de sistema híbrido con bomba de calor, suelo radiante, fotovoltaica y almacenamiento para vivienda},    % title
    pdfauthor={Luis D. Aranda Sánchez},     % author
    pdfkeywords={palabra1, palabra2, código1, etc.} % list of keywords
}

% Font change to Arial
\usepackage{helvet}
\renewcommand{\familydefault}{\sfdefault}

% Chapter titles in uppercase and larger font
\titleformat{\chapter}[hang]{\large\bfseries}{\thechapter.}{1em}{\MakeUppercase}
\titleformat{\section}[hang]{\bfseries}{\thesection.}{1em}{}
\titleformat{\subsection}[hang]{\bfseries}{\thesubsection.}{1em}{}

% Fancyhdr setup
\setlength{\headheight}{14.30174pt} % Adjust to recommended value, slightly larger for safety
\fancyhf{} % Clear all headers and footers
\fancyhead[LE]{\nouppercase{\leftmark}}
\fancyhead[RO]{Optimización energética para vivienda}
\fancyfoot[LE]{\thepage}
\fancyfoot[RE]{Escuela Técnica Superior de Ingenieros Industriales (UPM)}
\fancyfoot[LO]{Luis D. Aranda Sánchez}
\fancyfoot[RO]{\thepage}
\renewcommand{\headrulewidth}{0.4pt}
\renewcommand{\footrulewidth}{0.4pt}

\fancypagestyle{myfancy}{
    \fancyhf{} % Clear all headers and footers
    \fancyhead[LE]{\nouppercase{\leftmark}}
    \fancyhead[RO]{Optimización energética para vivienda}
    \fancyfoot[LE]{\thepage}
    \fancyfoot[RE]{Escuela Técnica Superior de Ingenieros Industriales (UPM)}
    \fancyfoot[LO]{Luis D. Aranda Sánchez}
    \fancyfoot[RO]{\thepage}
    \renewcommand{\headrulewidth}{0.4pt}
    \renewcommand{\footrulewidth}{0.4pt}
}

\fancypagestyle{simple}{
    \fancyhf{} % Clear all headers and footers
    \renewcommand{\headrulewidth}{0pt}
    \renewcommand{\footrulewidth}{0pt}
}

% Line spacing
\setstretch{1.2}

% Document starts here
\begin{document}

% Portada
\begin{titlepage}
    \centering
    {\scshape\LARGE Universidad Politécnica de Madrid \par}
    \vspace{1cm}
    {\scshape\Large Escuela Técnica Superior de Ingenieros Industriales\par}
    \vspace{1.5cm}
    {\huge\bfseries Optimización energética de sistema híbrido con bomba de calor, suelo radiante, fotovoltaica y almacenamiento para vivienda \par}
    \vspace{1.5cm}
    {\Large\bfseries Trabajo de Fin de Máster\par}
    \vspace{0.5cm}
    {\large Máster Universitario en Ingeniería de la Energía \par}
    \vspace{2cm}
    {\Large Luis D. Aranda Sánchez\par}
    \vfill
    Director: Javier Rodríguez Martín
    \vfill
    {\large Septiembre 6, 2024\par}
\end{titlepage}

% Resumen (máximo de 5 páginas, incluyendo al final Palabras clave)
\clearpage
\pagestyle{simple}
% \newpage
\chapter*{Resumen}
\addcontentsline{toc}{chapter}{Resumen}
\input{capitulos/resumen/main.tex}

% Índice (paginado)
\clearpage
\pagestyle{simple}
% \newpage
\tableofcontents

% Introducción (donde se incluya los antecedentes y justificación)
\clearpage
\pagestyle{myfancy}
\newpage
\chapter{Introducción}
\input{capitulos/introduccion/main.tex}

% Objetivos
\chapter{Objetivos}
\input{capitulos/objetivos/main.tex}

% Metodología
\chapter{Metodología}
\input{capitulos/metodologia/main.tex}

% Resultados y discusión (incluyendo la valoración de impactos y de aspectos de responsabilidad legal, ética y profesional relacionados con el trabajo)
\chapter{Resultados y Discusión}
\input{capitulos/resultados_discusion/main.tex}

% Conclusiones
\chapter{Conclusiones}
\input{capitulos/conclusiones/main.tex}

% Planificación temporal y presupuesto
\chapter{Planificación Temporal y Presupuesto}
\input{capitulos/planificacion_presupuesto/main.tex}

% Bibliografía
\newpage
\addcontentsline{toc}{chapter}{Bibliografía}
\printbibliography

\end{document}


% Resultados y discusión (incluyendo la valoración de impactos y de aspectos de responsabilidad legal, ética y profesional relacionados con el trabajo)
\chapter{Resultados y Discusión}
\documentclass[a4paper,11pt,twoside]{report}
\usepackage[left=25mm,right=25mm,top=25mm,bottom=25mm,includehead,includefoot,headsep=15mm,footskip=15mm]{geometry}
\usepackage{graphicx}
\usepackage{fancyhdr}
\usepackage{titlesec}
\usepackage[spanish]{babel}
\usepackage[utf8]{inputenc}
\usepackage{amsmath}
\usepackage{setspace}
\usepackage{svg}
\usepackage{hyperref}
\usepackage[backend=biber,style=numeric]{biblatex}
\addbibresource{references.bib}
\hypersetup{
    colorlinks=true,
    linkcolor=blue,      % color of internal links (sections, etc.)
    urlcolor=blue,       % color of external links
    pdftitle={Optimización energética de sistema híbrido con bomba de calor, suelo radiante, fotovoltaica y almacenamiento para vivienda},    % title
    pdfauthor={Luis D. Aranda Sánchez},     % author
    pdfkeywords={palabra1, palabra2, código1, etc.} % list of keywords
}

% Font change to Arial
\usepackage{helvet}
\renewcommand{\familydefault}{\sfdefault}

% Chapter titles in uppercase and larger font
\titleformat{\chapter}[hang]{\large\bfseries}{\thechapter.}{1em}{\MakeUppercase}
\titleformat{\section}[hang]{\bfseries}{\thesection.}{1em}{}
\titleformat{\subsection}[hang]{\bfseries}{\thesubsection.}{1em}{}

% Fancyhdr setup
\setlength{\headheight}{14.30174pt} % Adjust to recommended value, slightly larger for safety
\fancyhf{} % Clear all headers and footers
\fancyhead[LE]{\nouppercase{\leftmark}}
\fancyhead[RO]{Optimización energética para vivienda}
\fancyfoot[LE]{\thepage}
\fancyfoot[RE]{Escuela Técnica Superior de Ingenieros Industriales (UPM)}
\fancyfoot[LO]{Luis D. Aranda Sánchez}
\fancyfoot[RO]{\thepage}
\renewcommand{\headrulewidth}{0.4pt}
\renewcommand{\footrulewidth}{0.4pt}

\fancypagestyle{myfancy}{
    \fancyhf{} % Clear all headers and footers
    \fancyhead[LE]{\nouppercase{\leftmark}}
    \fancyhead[RO]{Optimización energética para vivienda}
    \fancyfoot[LE]{\thepage}
    \fancyfoot[RE]{Escuela Técnica Superior de Ingenieros Industriales (UPM)}
    \fancyfoot[LO]{Luis D. Aranda Sánchez}
    \fancyfoot[RO]{\thepage}
    \renewcommand{\headrulewidth}{0.4pt}
    \renewcommand{\footrulewidth}{0.4pt}
}

\fancypagestyle{simple}{
    \fancyhf{} % Clear all headers and footers
    \renewcommand{\headrulewidth}{0pt}
    \renewcommand{\footrulewidth}{0pt}
}

% Line spacing
\setstretch{1.2}

% Document starts here
\begin{document}

% Portada
\begin{titlepage}
    \centering
    {\scshape\LARGE Universidad Politécnica de Madrid \par}
    \vspace{1cm}
    {\scshape\Large Escuela Técnica Superior de Ingenieros Industriales\par}
    \vspace{1.5cm}
    {\huge\bfseries Optimización energética de sistema híbrido con bomba de calor, suelo radiante, fotovoltaica y almacenamiento para vivienda \par}
    \vspace{1.5cm}
    {\Large\bfseries Trabajo de Fin de Máster\par}
    \vspace{0.5cm}
    {\large Máster Universitario en Ingeniería de la Energía \par}
    \vspace{2cm}
    {\Large Luis D. Aranda Sánchez\par}
    \vfill
    Director: Javier Rodríguez Martín
    \vfill
    {\large Septiembre 6, 2024\par}
\end{titlepage}

% Resumen (máximo de 5 páginas, incluyendo al final Palabras clave)
\clearpage
\pagestyle{simple}
% \newpage
\chapter*{Resumen}
\addcontentsline{toc}{chapter}{Resumen}
\input{capitulos/resumen/main.tex}

% Índice (paginado)
\clearpage
\pagestyle{simple}
% \newpage
\tableofcontents

% Introducción (donde se incluya los antecedentes y justificación)
\clearpage
\pagestyle{myfancy}
\newpage
\chapter{Introducción}
\input{capitulos/introduccion/main.tex}

% Objetivos
\chapter{Objetivos}
\input{capitulos/objetivos/main.tex}

% Metodología
\chapter{Metodología}
\input{capitulos/metodologia/main.tex}

% Resultados y discusión (incluyendo la valoración de impactos y de aspectos de responsabilidad legal, ética y profesional relacionados con el trabajo)
\chapter{Resultados y Discusión}
\input{capitulos/resultados_discusion/main.tex}

% Conclusiones
\chapter{Conclusiones}
\input{capitulos/conclusiones/main.tex}

% Planificación temporal y presupuesto
\chapter{Planificación Temporal y Presupuesto}
\input{capitulos/planificacion_presupuesto/main.tex}

% Bibliografía
\newpage
\addcontentsline{toc}{chapter}{Bibliografía}
\printbibliography

\end{document}


% Conclusiones
\chapter{Conclusiones}
\documentclass[a4paper,11pt,twoside]{report}
\usepackage[left=25mm,right=25mm,top=25mm,bottom=25mm,includehead,includefoot,headsep=15mm,footskip=15mm]{geometry}
\usepackage{graphicx}
\usepackage{fancyhdr}
\usepackage{titlesec}
\usepackage[spanish]{babel}
\usepackage[utf8]{inputenc}
\usepackage{amsmath}
\usepackage{setspace}
\usepackage{svg}
\usepackage{hyperref}
\usepackage[backend=biber,style=numeric]{biblatex}
\addbibresource{references.bib}
\hypersetup{
    colorlinks=true,
    linkcolor=blue,      % color of internal links (sections, etc.)
    urlcolor=blue,       % color of external links
    pdftitle={Optimización energética de sistema híbrido con bomba de calor, suelo radiante, fotovoltaica y almacenamiento para vivienda},    % title
    pdfauthor={Luis D. Aranda Sánchez},     % author
    pdfkeywords={palabra1, palabra2, código1, etc.} % list of keywords
}

% Font change to Arial
\usepackage{helvet}
\renewcommand{\familydefault}{\sfdefault}

% Chapter titles in uppercase and larger font
\titleformat{\chapter}[hang]{\large\bfseries}{\thechapter.}{1em}{\MakeUppercase}
\titleformat{\section}[hang]{\bfseries}{\thesection.}{1em}{}
\titleformat{\subsection}[hang]{\bfseries}{\thesubsection.}{1em}{}

% Fancyhdr setup
\setlength{\headheight}{14.30174pt} % Adjust to recommended value, slightly larger for safety
\fancyhf{} % Clear all headers and footers
\fancyhead[LE]{\nouppercase{\leftmark}}
\fancyhead[RO]{Optimización energética para vivienda}
\fancyfoot[LE]{\thepage}
\fancyfoot[RE]{Escuela Técnica Superior de Ingenieros Industriales (UPM)}
\fancyfoot[LO]{Luis D. Aranda Sánchez}
\fancyfoot[RO]{\thepage}
\renewcommand{\headrulewidth}{0.4pt}
\renewcommand{\footrulewidth}{0.4pt}

\fancypagestyle{myfancy}{
    \fancyhf{} % Clear all headers and footers
    \fancyhead[LE]{\nouppercase{\leftmark}}
    \fancyhead[RO]{Optimización energética para vivienda}
    \fancyfoot[LE]{\thepage}
    \fancyfoot[RE]{Escuela Técnica Superior de Ingenieros Industriales (UPM)}
    \fancyfoot[LO]{Luis D. Aranda Sánchez}
    \fancyfoot[RO]{\thepage}
    \renewcommand{\headrulewidth}{0.4pt}
    \renewcommand{\footrulewidth}{0.4pt}
}

\fancypagestyle{simple}{
    \fancyhf{} % Clear all headers and footers
    \renewcommand{\headrulewidth}{0pt}
    \renewcommand{\footrulewidth}{0pt}
}

% Line spacing
\setstretch{1.2}

% Document starts here
\begin{document}

% Portada
\begin{titlepage}
    \centering
    {\scshape\LARGE Universidad Politécnica de Madrid \par}
    \vspace{1cm}
    {\scshape\Large Escuela Técnica Superior de Ingenieros Industriales\par}
    \vspace{1.5cm}
    {\huge\bfseries Optimización energética de sistema híbrido con bomba de calor, suelo radiante, fotovoltaica y almacenamiento para vivienda \par}
    \vspace{1.5cm}
    {\Large\bfseries Trabajo de Fin de Máster\par}
    \vspace{0.5cm}
    {\large Máster Universitario en Ingeniería de la Energía \par}
    \vspace{2cm}
    {\Large Luis D. Aranda Sánchez\par}
    \vfill
    Director: Javier Rodríguez Martín
    \vfill
    {\large Septiembre 6, 2024\par}
\end{titlepage}

% Resumen (máximo de 5 páginas, incluyendo al final Palabras clave)
\clearpage
\pagestyle{simple}
% \newpage
\chapter*{Resumen}
\addcontentsline{toc}{chapter}{Resumen}
\input{capitulos/resumen/main.tex}

% Índice (paginado)
\clearpage
\pagestyle{simple}
% \newpage
\tableofcontents

% Introducción (donde se incluya los antecedentes y justificación)
\clearpage
\pagestyle{myfancy}
\newpage
\chapter{Introducción}
\input{capitulos/introduccion/main.tex}

% Objetivos
\chapter{Objetivos}
\input{capitulos/objetivos/main.tex}

% Metodología
\chapter{Metodología}
\input{capitulos/metodologia/main.tex}

% Resultados y discusión (incluyendo la valoración de impactos y de aspectos de responsabilidad legal, ética y profesional relacionados con el trabajo)
\chapter{Resultados y Discusión}
\input{capitulos/resultados_discusion/main.tex}

% Conclusiones
\chapter{Conclusiones}
\input{capitulos/conclusiones/main.tex}

% Planificación temporal y presupuesto
\chapter{Planificación Temporal y Presupuesto}
\input{capitulos/planificacion_presupuesto/main.tex}

% Bibliografía
\newpage
\addcontentsline{toc}{chapter}{Bibliografía}
\printbibliography

\end{document}


% Planificación temporal y presupuesto
\chapter{Planificación Temporal y Presupuesto}
\documentclass[a4paper,11pt,twoside]{report}
\usepackage[left=25mm,right=25mm,top=25mm,bottom=25mm,includehead,includefoot,headsep=15mm,footskip=15mm]{geometry}
\usepackage{graphicx}
\usepackage{fancyhdr}
\usepackage{titlesec}
\usepackage[spanish]{babel}
\usepackage[utf8]{inputenc}
\usepackage{amsmath}
\usepackage{setspace}
\usepackage{svg}
\usepackage{hyperref}
\usepackage[backend=biber,style=numeric]{biblatex}
\addbibresource{references.bib}
\hypersetup{
    colorlinks=true,
    linkcolor=blue,      % color of internal links (sections, etc.)
    urlcolor=blue,       % color of external links
    pdftitle={Optimización energética de sistema híbrido con bomba de calor, suelo radiante, fotovoltaica y almacenamiento para vivienda},    % title
    pdfauthor={Luis D. Aranda Sánchez},     % author
    pdfkeywords={palabra1, palabra2, código1, etc.} % list of keywords
}

% Font change to Arial
\usepackage{helvet}
\renewcommand{\familydefault}{\sfdefault}

% Chapter titles in uppercase and larger font
\titleformat{\chapter}[hang]{\large\bfseries}{\thechapter.}{1em}{\MakeUppercase}
\titleformat{\section}[hang]{\bfseries}{\thesection.}{1em}{}
\titleformat{\subsection}[hang]{\bfseries}{\thesubsection.}{1em}{}

% Fancyhdr setup
\setlength{\headheight}{14.30174pt} % Adjust to recommended value, slightly larger for safety
\fancyhf{} % Clear all headers and footers
\fancyhead[LE]{\nouppercase{\leftmark}}
\fancyhead[RO]{Optimización energética para vivienda}
\fancyfoot[LE]{\thepage}
\fancyfoot[RE]{Escuela Técnica Superior de Ingenieros Industriales (UPM)}
\fancyfoot[LO]{Luis D. Aranda Sánchez}
\fancyfoot[RO]{\thepage}
\renewcommand{\headrulewidth}{0.4pt}
\renewcommand{\footrulewidth}{0.4pt}

\fancypagestyle{myfancy}{
    \fancyhf{} % Clear all headers and footers
    \fancyhead[LE]{\nouppercase{\leftmark}}
    \fancyhead[RO]{Optimización energética para vivienda}
    \fancyfoot[LE]{\thepage}
    \fancyfoot[RE]{Escuela Técnica Superior de Ingenieros Industriales (UPM)}
    \fancyfoot[LO]{Luis D. Aranda Sánchez}
    \fancyfoot[RO]{\thepage}
    \renewcommand{\headrulewidth}{0.4pt}
    \renewcommand{\footrulewidth}{0.4pt}
}

\fancypagestyle{simple}{
    \fancyhf{} % Clear all headers and footers
    \renewcommand{\headrulewidth}{0pt}
    \renewcommand{\footrulewidth}{0pt}
}

% Line spacing
\setstretch{1.2}

% Document starts here
\begin{document}

% Portada
\begin{titlepage}
    \centering
    {\scshape\LARGE Universidad Politécnica de Madrid \par}
    \vspace{1cm}
    {\scshape\Large Escuela Técnica Superior de Ingenieros Industriales\par}
    \vspace{1.5cm}
    {\huge\bfseries Optimización energética de sistema híbrido con bomba de calor, suelo radiante, fotovoltaica y almacenamiento para vivienda \par}
    \vspace{1.5cm}
    {\Large\bfseries Trabajo de Fin de Máster\par}
    \vspace{0.5cm}
    {\large Máster Universitario en Ingeniería de la Energía \par}
    \vspace{2cm}
    {\Large Luis D. Aranda Sánchez\par}
    \vfill
    Director: Javier Rodríguez Martín
    \vfill
    {\large Septiembre 6, 2024\par}
\end{titlepage}

% Resumen (máximo de 5 páginas, incluyendo al final Palabras clave)
\clearpage
\pagestyle{simple}
% \newpage
\chapter*{Resumen}
\addcontentsline{toc}{chapter}{Resumen}
\input{capitulos/resumen/main.tex}

% Índice (paginado)
\clearpage
\pagestyle{simple}
% \newpage
\tableofcontents

% Introducción (donde se incluya los antecedentes y justificación)
\clearpage
\pagestyle{myfancy}
\newpage
\chapter{Introducción}
\input{capitulos/introduccion/main.tex}

% Objetivos
\chapter{Objetivos}
\input{capitulos/objetivos/main.tex}

% Metodología
\chapter{Metodología}
\input{capitulos/metodologia/main.tex}

% Resultados y discusión (incluyendo la valoración de impactos y de aspectos de responsabilidad legal, ética y profesional relacionados con el trabajo)
\chapter{Resultados y Discusión}
\input{capitulos/resultados_discusion/main.tex}

% Conclusiones
\chapter{Conclusiones}
\input{capitulos/conclusiones/main.tex}

% Planificación temporal y presupuesto
\chapter{Planificación Temporal y Presupuesto}
\input{capitulos/planificacion_presupuesto/main.tex}

% Bibliografía
\newpage
\addcontentsline{toc}{chapter}{Bibliografía}
\printbibliography

\end{document}


% Bibliografía
\newpage
\addcontentsline{toc}{chapter}{Bibliografía}
\printbibliography

\end{document}


% Planificación temporal y presupuesto
\chapter{Planificación Temporal y Presupuesto}
\documentclass[a4paper,11pt,twoside]{report}
\usepackage[left=25mm,right=25mm,top=25mm,bottom=25mm,includehead,includefoot,headsep=15mm,footskip=15mm]{geometry}
\usepackage{graphicx}
\usepackage{fancyhdr}
\usepackage{titlesec}
\usepackage[spanish]{babel}
\usepackage[utf8]{inputenc}
\usepackage{amsmath}
\usepackage{setspace}
\usepackage{svg}
\usepackage{hyperref}
\usepackage[backend=biber,style=numeric]{biblatex}
\addbibresource{references.bib}
\hypersetup{
    colorlinks=true,
    linkcolor=blue,      % color of internal links (sections, etc.)
    urlcolor=blue,       % color of external links
    pdftitle={Optimización energética de sistema híbrido con bomba de calor, suelo radiante, fotovoltaica y almacenamiento para vivienda},    % title
    pdfauthor={Luis D. Aranda Sánchez},     % author
    pdfkeywords={palabra1, palabra2, código1, etc.} % list of keywords
}

% Font change to Arial
\usepackage{helvet}
\renewcommand{\familydefault}{\sfdefault}

% Chapter titles in uppercase and larger font
\titleformat{\chapter}[hang]{\large\bfseries}{\thechapter.}{1em}{\MakeUppercase}
\titleformat{\section}[hang]{\bfseries}{\thesection.}{1em}{}
\titleformat{\subsection}[hang]{\bfseries}{\thesubsection.}{1em}{}

% Fancyhdr setup
\setlength{\headheight}{14.30174pt} % Adjust to recommended value, slightly larger for safety
\fancyhf{} % Clear all headers and footers
\fancyhead[LE]{\nouppercase{\leftmark}}
\fancyhead[RO]{Optimización energética para vivienda}
\fancyfoot[LE]{\thepage}
\fancyfoot[RE]{Escuela Técnica Superior de Ingenieros Industriales (UPM)}
\fancyfoot[LO]{Luis D. Aranda Sánchez}
\fancyfoot[RO]{\thepage}
\renewcommand{\headrulewidth}{0.4pt}
\renewcommand{\footrulewidth}{0.4pt}

\fancypagestyle{myfancy}{
    \fancyhf{} % Clear all headers and footers
    \fancyhead[LE]{\nouppercase{\leftmark}}
    \fancyhead[RO]{Optimización energética para vivienda}
    \fancyfoot[LE]{\thepage}
    \fancyfoot[RE]{Escuela Técnica Superior de Ingenieros Industriales (UPM)}
    \fancyfoot[LO]{Luis D. Aranda Sánchez}
    \fancyfoot[RO]{\thepage}
    \renewcommand{\headrulewidth}{0.4pt}
    \renewcommand{\footrulewidth}{0.4pt}
}

\fancypagestyle{simple}{
    \fancyhf{} % Clear all headers and footers
    \renewcommand{\headrulewidth}{0pt}
    \renewcommand{\footrulewidth}{0pt}
}

% Line spacing
\setstretch{1.2}

% Document starts here
\begin{document}

% Portada
\begin{titlepage}
    \centering
    {\scshape\LARGE Universidad Politécnica de Madrid \par}
    \vspace{1cm}
    {\scshape\Large Escuela Técnica Superior de Ingenieros Industriales\par}
    \vspace{1.5cm}
    {\huge\bfseries Optimización energética de sistema híbrido con bomba de calor, suelo radiante, fotovoltaica y almacenamiento para vivienda \par}
    \vspace{1.5cm}
    {\Large\bfseries Trabajo de Fin de Máster\par}
    \vspace{0.5cm}
    {\large Máster Universitario en Ingeniería de la Energía \par}
    \vspace{2cm}
    {\Large Luis D. Aranda Sánchez\par}
    \vfill
    Director: Javier Rodríguez Martín
    \vfill
    {\large Septiembre 6, 2024\par}
\end{titlepage}

% Resumen (máximo de 5 páginas, incluyendo al final Palabras clave)
\clearpage
\pagestyle{simple}
% \newpage
\chapter*{Resumen}
\addcontentsline{toc}{chapter}{Resumen}
\documentclass[a4paper,11pt,twoside]{report}
\usepackage[left=25mm,right=25mm,top=25mm,bottom=25mm,includehead,includefoot,headsep=15mm,footskip=15mm]{geometry}
\usepackage{graphicx}
\usepackage{fancyhdr}
\usepackage{titlesec}
\usepackage[spanish]{babel}
\usepackage[utf8]{inputenc}
\usepackage{amsmath}
\usepackage{setspace}
\usepackage{svg}
\usepackage{hyperref}
\usepackage[backend=biber,style=numeric]{biblatex}
\addbibresource{references.bib}
\hypersetup{
    colorlinks=true,
    linkcolor=blue,      % color of internal links (sections, etc.)
    urlcolor=blue,       % color of external links
    pdftitle={Optimización energética de sistema híbrido con bomba de calor, suelo radiante, fotovoltaica y almacenamiento para vivienda},    % title
    pdfauthor={Luis D. Aranda Sánchez},     % author
    pdfkeywords={palabra1, palabra2, código1, etc.} % list of keywords
}

% Font change to Arial
\usepackage{helvet}
\renewcommand{\familydefault}{\sfdefault}

% Chapter titles in uppercase and larger font
\titleformat{\chapter}[hang]{\large\bfseries}{\thechapter.}{1em}{\MakeUppercase}
\titleformat{\section}[hang]{\bfseries}{\thesection.}{1em}{}
\titleformat{\subsection}[hang]{\bfseries}{\thesubsection.}{1em}{}

% Fancyhdr setup
\setlength{\headheight}{14.30174pt} % Adjust to recommended value, slightly larger for safety
\fancyhf{} % Clear all headers and footers
\fancyhead[LE]{\nouppercase{\leftmark}}
\fancyhead[RO]{Optimización energética para vivienda}
\fancyfoot[LE]{\thepage}
\fancyfoot[RE]{Escuela Técnica Superior de Ingenieros Industriales (UPM)}
\fancyfoot[LO]{Luis D. Aranda Sánchez}
\fancyfoot[RO]{\thepage}
\renewcommand{\headrulewidth}{0.4pt}
\renewcommand{\footrulewidth}{0.4pt}

\fancypagestyle{myfancy}{
    \fancyhf{} % Clear all headers and footers
    \fancyhead[LE]{\nouppercase{\leftmark}}
    \fancyhead[RO]{Optimización energética para vivienda}
    \fancyfoot[LE]{\thepage}
    \fancyfoot[RE]{Escuela Técnica Superior de Ingenieros Industriales (UPM)}
    \fancyfoot[LO]{Luis D. Aranda Sánchez}
    \fancyfoot[RO]{\thepage}
    \renewcommand{\headrulewidth}{0.4pt}
    \renewcommand{\footrulewidth}{0.4pt}
}

\fancypagestyle{simple}{
    \fancyhf{} % Clear all headers and footers
    \renewcommand{\headrulewidth}{0pt}
    \renewcommand{\footrulewidth}{0pt}
}

% Line spacing
\setstretch{1.2}

% Document starts here
\begin{document}

% Portada
\begin{titlepage}
    \centering
    {\scshape\LARGE Universidad Politécnica de Madrid \par}
    \vspace{1cm}
    {\scshape\Large Escuela Técnica Superior de Ingenieros Industriales\par}
    \vspace{1.5cm}
    {\huge\bfseries Optimización energética de sistema híbrido con bomba de calor, suelo radiante, fotovoltaica y almacenamiento para vivienda \par}
    \vspace{1.5cm}
    {\Large\bfseries Trabajo de Fin de Máster\par}
    \vspace{0.5cm}
    {\large Máster Universitario en Ingeniería de la Energía \par}
    \vspace{2cm}
    {\Large Luis D. Aranda Sánchez\par}
    \vfill
    Director: Javier Rodríguez Martín
    \vfill
    {\large Septiembre 6, 2024\par}
\end{titlepage}

% Resumen (máximo de 5 páginas, incluyendo al final Palabras clave)
\clearpage
\pagestyle{simple}
% \newpage
\chapter*{Resumen}
\addcontentsline{toc}{chapter}{Resumen}
\input{capitulos/resumen/main.tex}

% Índice (paginado)
\clearpage
\pagestyle{simple}
% \newpage
\tableofcontents

% Introducción (donde se incluya los antecedentes y justificación)
\clearpage
\pagestyle{myfancy}
\newpage
\chapter{Introducción}
\input{capitulos/introduccion/main.tex}

% Objetivos
\chapter{Objetivos}
\input{capitulos/objetivos/main.tex}

% Metodología
\chapter{Metodología}
\input{capitulos/metodologia/main.tex}

% Resultados y discusión (incluyendo la valoración de impactos y de aspectos de responsabilidad legal, ética y profesional relacionados con el trabajo)
\chapter{Resultados y Discusión}
\input{capitulos/resultados_discusion/main.tex}

% Conclusiones
\chapter{Conclusiones}
\input{capitulos/conclusiones/main.tex}

% Planificación temporal y presupuesto
\chapter{Planificación Temporal y Presupuesto}
\input{capitulos/planificacion_presupuesto/main.tex}

% Bibliografía
\newpage
\addcontentsline{toc}{chapter}{Bibliografía}
\printbibliography

\end{document}


% Índice (paginado)
\clearpage
\pagestyle{simple}
% \newpage
\tableofcontents

% Introducción (donde se incluya los antecedentes y justificación)
\clearpage
\pagestyle{myfancy}
\newpage
\chapter{Introducción}
\documentclass[a4paper,11pt,twoside]{report}
\usepackage[left=25mm,right=25mm,top=25mm,bottom=25mm,includehead,includefoot,headsep=15mm,footskip=15mm]{geometry}
\usepackage{graphicx}
\usepackage{fancyhdr}
\usepackage{titlesec}
\usepackage[spanish]{babel}
\usepackage[utf8]{inputenc}
\usepackage{amsmath}
\usepackage{setspace}
\usepackage{svg}
\usepackage{hyperref}
\usepackage[backend=biber,style=numeric]{biblatex}
\addbibresource{references.bib}
\hypersetup{
    colorlinks=true,
    linkcolor=blue,      % color of internal links (sections, etc.)
    urlcolor=blue,       % color of external links
    pdftitle={Optimización energética de sistema híbrido con bomba de calor, suelo radiante, fotovoltaica y almacenamiento para vivienda},    % title
    pdfauthor={Luis D. Aranda Sánchez},     % author
    pdfkeywords={palabra1, palabra2, código1, etc.} % list of keywords
}

% Font change to Arial
\usepackage{helvet}
\renewcommand{\familydefault}{\sfdefault}

% Chapter titles in uppercase and larger font
\titleformat{\chapter}[hang]{\large\bfseries}{\thechapter.}{1em}{\MakeUppercase}
\titleformat{\section}[hang]{\bfseries}{\thesection.}{1em}{}
\titleformat{\subsection}[hang]{\bfseries}{\thesubsection.}{1em}{}

% Fancyhdr setup
\setlength{\headheight}{14.30174pt} % Adjust to recommended value, slightly larger for safety
\fancyhf{} % Clear all headers and footers
\fancyhead[LE]{\nouppercase{\leftmark}}
\fancyhead[RO]{Optimización energética para vivienda}
\fancyfoot[LE]{\thepage}
\fancyfoot[RE]{Escuela Técnica Superior de Ingenieros Industriales (UPM)}
\fancyfoot[LO]{Luis D. Aranda Sánchez}
\fancyfoot[RO]{\thepage}
\renewcommand{\headrulewidth}{0.4pt}
\renewcommand{\footrulewidth}{0.4pt}

\fancypagestyle{myfancy}{
    \fancyhf{} % Clear all headers and footers
    \fancyhead[LE]{\nouppercase{\leftmark}}
    \fancyhead[RO]{Optimización energética para vivienda}
    \fancyfoot[LE]{\thepage}
    \fancyfoot[RE]{Escuela Técnica Superior de Ingenieros Industriales (UPM)}
    \fancyfoot[LO]{Luis D. Aranda Sánchez}
    \fancyfoot[RO]{\thepage}
    \renewcommand{\headrulewidth}{0.4pt}
    \renewcommand{\footrulewidth}{0.4pt}
}

\fancypagestyle{simple}{
    \fancyhf{} % Clear all headers and footers
    \renewcommand{\headrulewidth}{0pt}
    \renewcommand{\footrulewidth}{0pt}
}

% Line spacing
\setstretch{1.2}

% Document starts here
\begin{document}

% Portada
\begin{titlepage}
    \centering
    {\scshape\LARGE Universidad Politécnica de Madrid \par}
    \vspace{1cm}
    {\scshape\Large Escuela Técnica Superior de Ingenieros Industriales\par}
    \vspace{1.5cm}
    {\huge\bfseries Optimización energética de sistema híbrido con bomba de calor, suelo radiante, fotovoltaica y almacenamiento para vivienda \par}
    \vspace{1.5cm}
    {\Large\bfseries Trabajo de Fin de Máster\par}
    \vspace{0.5cm}
    {\large Máster Universitario en Ingeniería de la Energía \par}
    \vspace{2cm}
    {\Large Luis D. Aranda Sánchez\par}
    \vfill
    Director: Javier Rodríguez Martín
    \vfill
    {\large Septiembre 6, 2024\par}
\end{titlepage}

% Resumen (máximo de 5 páginas, incluyendo al final Palabras clave)
\clearpage
\pagestyle{simple}
% \newpage
\chapter*{Resumen}
\addcontentsline{toc}{chapter}{Resumen}
\input{capitulos/resumen/main.tex}

% Índice (paginado)
\clearpage
\pagestyle{simple}
% \newpage
\tableofcontents

% Introducción (donde se incluya los antecedentes y justificación)
\clearpage
\pagestyle{myfancy}
\newpage
\chapter{Introducción}
\input{capitulos/introduccion/main.tex}

% Objetivos
\chapter{Objetivos}
\input{capitulos/objetivos/main.tex}

% Metodología
\chapter{Metodología}
\input{capitulos/metodologia/main.tex}

% Resultados y discusión (incluyendo la valoración de impactos y de aspectos de responsabilidad legal, ética y profesional relacionados con el trabajo)
\chapter{Resultados y Discusión}
\input{capitulos/resultados_discusion/main.tex}

% Conclusiones
\chapter{Conclusiones}
\input{capitulos/conclusiones/main.tex}

% Planificación temporal y presupuesto
\chapter{Planificación Temporal y Presupuesto}
\input{capitulos/planificacion_presupuesto/main.tex}

% Bibliografía
\newpage
\addcontentsline{toc}{chapter}{Bibliografía}
\printbibliography

\end{document}


% Objetivos
\chapter{Objetivos}
\documentclass[a4paper,11pt,twoside]{report}
\usepackage[left=25mm,right=25mm,top=25mm,bottom=25mm,includehead,includefoot,headsep=15mm,footskip=15mm]{geometry}
\usepackage{graphicx}
\usepackage{fancyhdr}
\usepackage{titlesec}
\usepackage[spanish]{babel}
\usepackage[utf8]{inputenc}
\usepackage{amsmath}
\usepackage{setspace}
\usepackage{svg}
\usepackage{hyperref}
\usepackage[backend=biber,style=numeric]{biblatex}
\addbibresource{references.bib}
\hypersetup{
    colorlinks=true,
    linkcolor=blue,      % color of internal links (sections, etc.)
    urlcolor=blue,       % color of external links
    pdftitle={Optimización energética de sistema híbrido con bomba de calor, suelo radiante, fotovoltaica y almacenamiento para vivienda},    % title
    pdfauthor={Luis D. Aranda Sánchez},     % author
    pdfkeywords={palabra1, palabra2, código1, etc.} % list of keywords
}

% Font change to Arial
\usepackage{helvet}
\renewcommand{\familydefault}{\sfdefault}

% Chapter titles in uppercase and larger font
\titleformat{\chapter}[hang]{\large\bfseries}{\thechapter.}{1em}{\MakeUppercase}
\titleformat{\section}[hang]{\bfseries}{\thesection.}{1em}{}
\titleformat{\subsection}[hang]{\bfseries}{\thesubsection.}{1em}{}

% Fancyhdr setup
\setlength{\headheight}{14.30174pt} % Adjust to recommended value, slightly larger for safety
\fancyhf{} % Clear all headers and footers
\fancyhead[LE]{\nouppercase{\leftmark}}
\fancyhead[RO]{Optimización energética para vivienda}
\fancyfoot[LE]{\thepage}
\fancyfoot[RE]{Escuela Técnica Superior de Ingenieros Industriales (UPM)}
\fancyfoot[LO]{Luis D. Aranda Sánchez}
\fancyfoot[RO]{\thepage}
\renewcommand{\headrulewidth}{0.4pt}
\renewcommand{\footrulewidth}{0.4pt}

\fancypagestyle{myfancy}{
    \fancyhf{} % Clear all headers and footers
    \fancyhead[LE]{\nouppercase{\leftmark}}
    \fancyhead[RO]{Optimización energética para vivienda}
    \fancyfoot[LE]{\thepage}
    \fancyfoot[RE]{Escuela Técnica Superior de Ingenieros Industriales (UPM)}
    \fancyfoot[LO]{Luis D. Aranda Sánchez}
    \fancyfoot[RO]{\thepage}
    \renewcommand{\headrulewidth}{0.4pt}
    \renewcommand{\footrulewidth}{0.4pt}
}

\fancypagestyle{simple}{
    \fancyhf{} % Clear all headers and footers
    \renewcommand{\headrulewidth}{0pt}
    \renewcommand{\footrulewidth}{0pt}
}

% Line spacing
\setstretch{1.2}

% Document starts here
\begin{document}

% Portada
\begin{titlepage}
    \centering
    {\scshape\LARGE Universidad Politécnica de Madrid \par}
    \vspace{1cm}
    {\scshape\Large Escuela Técnica Superior de Ingenieros Industriales\par}
    \vspace{1.5cm}
    {\huge\bfseries Optimización energética de sistema híbrido con bomba de calor, suelo radiante, fotovoltaica y almacenamiento para vivienda \par}
    \vspace{1.5cm}
    {\Large\bfseries Trabajo de Fin de Máster\par}
    \vspace{0.5cm}
    {\large Máster Universitario en Ingeniería de la Energía \par}
    \vspace{2cm}
    {\Large Luis D. Aranda Sánchez\par}
    \vfill
    Director: Javier Rodríguez Martín
    \vfill
    {\large Septiembre 6, 2024\par}
\end{titlepage}

% Resumen (máximo de 5 páginas, incluyendo al final Palabras clave)
\clearpage
\pagestyle{simple}
% \newpage
\chapter*{Resumen}
\addcontentsline{toc}{chapter}{Resumen}
\input{capitulos/resumen/main.tex}

% Índice (paginado)
\clearpage
\pagestyle{simple}
% \newpage
\tableofcontents

% Introducción (donde se incluya los antecedentes y justificación)
\clearpage
\pagestyle{myfancy}
\newpage
\chapter{Introducción}
\input{capitulos/introduccion/main.tex}

% Objetivos
\chapter{Objetivos}
\input{capitulos/objetivos/main.tex}

% Metodología
\chapter{Metodología}
\input{capitulos/metodologia/main.tex}

% Resultados y discusión (incluyendo la valoración de impactos y de aspectos de responsabilidad legal, ética y profesional relacionados con el trabajo)
\chapter{Resultados y Discusión}
\input{capitulos/resultados_discusion/main.tex}

% Conclusiones
\chapter{Conclusiones}
\input{capitulos/conclusiones/main.tex}

% Planificación temporal y presupuesto
\chapter{Planificación Temporal y Presupuesto}
\input{capitulos/planificacion_presupuesto/main.tex}

% Bibliografía
\newpage
\addcontentsline{toc}{chapter}{Bibliografía}
\printbibliography

\end{document}


% Metodología
\chapter{Metodología}
\documentclass[a4paper,11pt,twoside]{report}
\usepackage[left=25mm,right=25mm,top=25mm,bottom=25mm,includehead,includefoot,headsep=15mm,footskip=15mm]{geometry}
\usepackage{graphicx}
\usepackage{fancyhdr}
\usepackage{titlesec}
\usepackage[spanish]{babel}
\usepackage[utf8]{inputenc}
\usepackage{amsmath}
\usepackage{setspace}
\usepackage{svg}
\usepackage{hyperref}
\usepackage[backend=biber,style=numeric]{biblatex}
\addbibresource{references.bib}
\hypersetup{
    colorlinks=true,
    linkcolor=blue,      % color of internal links (sections, etc.)
    urlcolor=blue,       % color of external links
    pdftitle={Optimización energética de sistema híbrido con bomba de calor, suelo radiante, fotovoltaica y almacenamiento para vivienda},    % title
    pdfauthor={Luis D. Aranda Sánchez},     % author
    pdfkeywords={palabra1, palabra2, código1, etc.} % list of keywords
}

% Font change to Arial
\usepackage{helvet}
\renewcommand{\familydefault}{\sfdefault}

% Chapter titles in uppercase and larger font
\titleformat{\chapter}[hang]{\large\bfseries}{\thechapter.}{1em}{\MakeUppercase}
\titleformat{\section}[hang]{\bfseries}{\thesection.}{1em}{}
\titleformat{\subsection}[hang]{\bfseries}{\thesubsection.}{1em}{}

% Fancyhdr setup
\setlength{\headheight}{14.30174pt} % Adjust to recommended value, slightly larger for safety
\fancyhf{} % Clear all headers and footers
\fancyhead[LE]{\nouppercase{\leftmark}}
\fancyhead[RO]{Optimización energética para vivienda}
\fancyfoot[LE]{\thepage}
\fancyfoot[RE]{Escuela Técnica Superior de Ingenieros Industriales (UPM)}
\fancyfoot[LO]{Luis D. Aranda Sánchez}
\fancyfoot[RO]{\thepage}
\renewcommand{\headrulewidth}{0.4pt}
\renewcommand{\footrulewidth}{0.4pt}

\fancypagestyle{myfancy}{
    \fancyhf{} % Clear all headers and footers
    \fancyhead[LE]{\nouppercase{\leftmark}}
    \fancyhead[RO]{Optimización energética para vivienda}
    \fancyfoot[LE]{\thepage}
    \fancyfoot[RE]{Escuela Técnica Superior de Ingenieros Industriales (UPM)}
    \fancyfoot[LO]{Luis D. Aranda Sánchez}
    \fancyfoot[RO]{\thepage}
    \renewcommand{\headrulewidth}{0.4pt}
    \renewcommand{\footrulewidth}{0.4pt}
}

\fancypagestyle{simple}{
    \fancyhf{} % Clear all headers and footers
    \renewcommand{\headrulewidth}{0pt}
    \renewcommand{\footrulewidth}{0pt}
}

% Line spacing
\setstretch{1.2}

% Document starts here
\begin{document}

% Portada
\begin{titlepage}
    \centering
    {\scshape\LARGE Universidad Politécnica de Madrid \par}
    \vspace{1cm}
    {\scshape\Large Escuela Técnica Superior de Ingenieros Industriales\par}
    \vspace{1.5cm}
    {\huge\bfseries Optimización energética de sistema híbrido con bomba de calor, suelo radiante, fotovoltaica y almacenamiento para vivienda \par}
    \vspace{1.5cm}
    {\Large\bfseries Trabajo de Fin de Máster\par}
    \vspace{0.5cm}
    {\large Máster Universitario en Ingeniería de la Energía \par}
    \vspace{2cm}
    {\Large Luis D. Aranda Sánchez\par}
    \vfill
    Director: Javier Rodríguez Martín
    \vfill
    {\large Septiembre 6, 2024\par}
\end{titlepage}

% Resumen (máximo de 5 páginas, incluyendo al final Palabras clave)
\clearpage
\pagestyle{simple}
% \newpage
\chapter*{Resumen}
\addcontentsline{toc}{chapter}{Resumen}
\input{capitulos/resumen/main.tex}

% Índice (paginado)
\clearpage
\pagestyle{simple}
% \newpage
\tableofcontents

% Introducción (donde se incluya los antecedentes y justificación)
\clearpage
\pagestyle{myfancy}
\newpage
\chapter{Introducción}
\input{capitulos/introduccion/main.tex}

% Objetivos
\chapter{Objetivos}
\input{capitulos/objetivos/main.tex}

% Metodología
\chapter{Metodología}
\input{capitulos/metodologia/main.tex}

% Resultados y discusión (incluyendo la valoración de impactos y de aspectos de responsabilidad legal, ética y profesional relacionados con el trabajo)
\chapter{Resultados y Discusión}
\input{capitulos/resultados_discusion/main.tex}

% Conclusiones
\chapter{Conclusiones}
\input{capitulos/conclusiones/main.tex}

% Planificación temporal y presupuesto
\chapter{Planificación Temporal y Presupuesto}
\input{capitulos/planificacion_presupuesto/main.tex}

% Bibliografía
\newpage
\addcontentsline{toc}{chapter}{Bibliografía}
\printbibliography

\end{document}


% Resultados y discusión (incluyendo la valoración de impactos y de aspectos de responsabilidad legal, ética y profesional relacionados con el trabajo)
\chapter{Resultados y Discusión}
\documentclass[a4paper,11pt,twoside]{report}
\usepackage[left=25mm,right=25mm,top=25mm,bottom=25mm,includehead,includefoot,headsep=15mm,footskip=15mm]{geometry}
\usepackage{graphicx}
\usepackage{fancyhdr}
\usepackage{titlesec}
\usepackage[spanish]{babel}
\usepackage[utf8]{inputenc}
\usepackage{amsmath}
\usepackage{setspace}
\usepackage{svg}
\usepackage{hyperref}
\usepackage[backend=biber,style=numeric]{biblatex}
\addbibresource{references.bib}
\hypersetup{
    colorlinks=true,
    linkcolor=blue,      % color of internal links (sections, etc.)
    urlcolor=blue,       % color of external links
    pdftitle={Optimización energética de sistema híbrido con bomba de calor, suelo radiante, fotovoltaica y almacenamiento para vivienda},    % title
    pdfauthor={Luis D. Aranda Sánchez},     % author
    pdfkeywords={palabra1, palabra2, código1, etc.} % list of keywords
}

% Font change to Arial
\usepackage{helvet}
\renewcommand{\familydefault}{\sfdefault}

% Chapter titles in uppercase and larger font
\titleformat{\chapter}[hang]{\large\bfseries}{\thechapter.}{1em}{\MakeUppercase}
\titleformat{\section}[hang]{\bfseries}{\thesection.}{1em}{}
\titleformat{\subsection}[hang]{\bfseries}{\thesubsection.}{1em}{}

% Fancyhdr setup
\setlength{\headheight}{14.30174pt} % Adjust to recommended value, slightly larger for safety
\fancyhf{} % Clear all headers and footers
\fancyhead[LE]{\nouppercase{\leftmark}}
\fancyhead[RO]{Optimización energética para vivienda}
\fancyfoot[LE]{\thepage}
\fancyfoot[RE]{Escuela Técnica Superior de Ingenieros Industriales (UPM)}
\fancyfoot[LO]{Luis D. Aranda Sánchez}
\fancyfoot[RO]{\thepage}
\renewcommand{\headrulewidth}{0.4pt}
\renewcommand{\footrulewidth}{0.4pt}

\fancypagestyle{myfancy}{
    \fancyhf{} % Clear all headers and footers
    \fancyhead[LE]{\nouppercase{\leftmark}}
    \fancyhead[RO]{Optimización energética para vivienda}
    \fancyfoot[LE]{\thepage}
    \fancyfoot[RE]{Escuela Técnica Superior de Ingenieros Industriales (UPM)}
    \fancyfoot[LO]{Luis D. Aranda Sánchez}
    \fancyfoot[RO]{\thepage}
    \renewcommand{\headrulewidth}{0.4pt}
    \renewcommand{\footrulewidth}{0.4pt}
}

\fancypagestyle{simple}{
    \fancyhf{} % Clear all headers and footers
    \renewcommand{\headrulewidth}{0pt}
    \renewcommand{\footrulewidth}{0pt}
}

% Line spacing
\setstretch{1.2}

% Document starts here
\begin{document}

% Portada
\begin{titlepage}
    \centering
    {\scshape\LARGE Universidad Politécnica de Madrid \par}
    \vspace{1cm}
    {\scshape\Large Escuela Técnica Superior de Ingenieros Industriales\par}
    \vspace{1.5cm}
    {\huge\bfseries Optimización energética de sistema híbrido con bomba de calor, suelo radiante, fotovoltaica y almacenamiento para vivienda \par}
    \vspace{1.5cm}
    {\Large\bfseries Trabajo de Fin de Máster\par}
    \vspace{0.5cm}
    {\large Máster Universitario en Ingeniería de la Energía \par}
    \vspace{2cm}
    {\Large Luis D. Aranda Sánchez\par}
    \vfill
    Director: Javier Rodríguez Martín
    \vfill
    {\large Septiembre 6, 2024\par}
\end{titlepage}

% Resumen (máximo de 5 páginas, incluyendo al final Palabras clave)
\clearpage
\pagestyle{simple}
% \newpage
\chapter*{Resumen}
\addcontentsline{toc}{chapter}{Resumen}
\input{capitulos/resumen/main.tex}

% Índice (paginado)
\clearpage
\pagestyle{simple}
% \newpage
\tableofcontents

% Introducción (donde se incluya los antecedentes y justificación)
\clearpage
\pagestyle{myfancy}
\newpage
\chapter{Introducción}
\input{capitulos/introduccion/main.tex}

% Objetivos
\chapter{Objetivos}
\input{capitulos/objetivos/main.tex}

% Metodología
\chapter{Metodología}
\input{capitulos/metodologia/main.tex}

% Resultados y discusión (incluyendo la valoración de impactos y de aspectos de responsabilidad legal, ética y profesional relacionados con el trabajo)
\chapter{Resultados y Discusión}
\input{capitulos/resultados_discusion/main.tex}

% Conclusiones
\chapter{Conclusiones}
\input{capitulos/conclusiones/main.tex}

% Planificación temporal y presupuesto
\chapter{Planificación Temporal y Presupuesto}
\input{capitulos/planificacion_presupuesto/main.tex}

% Bibliografía
\newpage
\addcontentsline{toc}{chapter}{Bibliografía}
\printbibliography

\end{document}


% Conclusiones
\chapter{Conclusiones}
\documentclass[a4paper,11pt,twoside]{report}
\usepackage[left=25mm,right=25mm,top=25mm,bottom=25mm,includehead,includefoot,headsep=15mm,footskip=15mm]{geometry}
\usepackage{graphicx}
\usepackage{fancyhdr}
\usepackage{titlesec}
\usepackage[spanish]{babel}
\usepackage[utf8]{inputenc}
\usepackage{amsmath}
\usepackage{setspace}
\usepackage{svg}
\usepackage{hyperref}
\usepackage[backend=biber,style=numeric]{biblatex}
\addbibresource{references.bib}
\hypersetup{
    colorlinks=true,
    linkcolor=blue,      % color of internal links (sections, etc.)
    urlcolor=blue,       % color of external links
    pdftitle={Optimización energética de sistema híbrido con bomba de calor, suelo radiante, fotovoltaica y almacenamiento para vivienda},    % title
    pdfauthor={Luis D. Aranda Sánchez},     % author
    pdfkeywords={palabra1, palabra2, código1, etc.} % list of keywords
}

% Font change to Arial
\usepackage{helvet}
\renewcommand{\familydefault}{\sfdefault}

% Chapter titles in uppercase and larger font
\titleformat{\chapter}[hang]{\large\bfseries}{\thechapter.}{1em}{\MakeUppercase}
\titleformat{\section}[hang]{\bfseries}{\thesection.}{1em}{}
\titleformat{\subsection}[hang]{\bfseries}{\thesubsection.}{1em}{}

% Fancyhdr setup
\setlength{\headheight}{14.30174pt} % Adjust to recommended value, slightly larger for safety
\fancyhf{} % Clear all headers and footers
\fancyhead[LE]{\nouppercase{\leftmark}}
\fancyhead[RO]{Optimización energética para vivienda}
\fancyfoot[LE]{\thepage}
\fancyfoot[RE]{Escuela Técnica Superior de Ingenieros Industriales (UPM)}
\fancyfoot[LO]{Luis D. Aranda Sánchez}
\fancyfoot[RO]{\thepage}
\renewcommand{\headrulewidth}{0.4pt}
\renewcommand{\footrulewidth}{0.4pt}

\fancypagestyle{myfancy}{
    \fancyhf{} % Clear all headers and footers
    \fancyhead[LE]{\nouppercase{\leftmark}}
    \fancyhead[RO]{Optimización energética para vivienda}
    \fancyfoot[LE]{\thepage}
    \fancyfoot[RE]{Escuela Técnica Superior de Ingenieros Industriales (UPM)}
    \fancyfoot[LO]{Luis D. Aranda Sánchez}
    \fancyfoot[RO]{\thepage}
    \renewcommand{\headrulewidth}{0.4pt}
    \renewcommand{\footrulewidth}{0.4pt}
}

\fancypagestyle{simple}{
    \fancyhf{} % Clear all headers and footers
    \renewcommand{\headrulewidth}{0pt}
    \renewcommand{\footrulewidth}{0pt}
}

% Line spacing
\setstretch{1.2}

% Document starts here
\begin{document}

% Portada
\begin{titlepage}
    \centering
    {\scshape\LARGE Universidad Politécnica de Madrid \par}
    \vspace{1cm}
    {\scshape\Large Escuela Técnica Superior de Ingenieros Industriales\par}
    \vspace{1.5cm}
    {\huge\bfseries Optimización energética de sistema híbrido con bomba de calor, suelo radiante, fotovoltaica y almacenamiento para vivienda \par}
    \vspace{1.5cm}
    {\Large\bfseries Trabajo de Fin de Máster\par}
    \vspace{0.5cm}
    {\large Máster Universitario en Ingeniería de la Energía \par}
    \vspace{2cm}
    {\Large Luis D. Aranda Sánchez\par}
    \vfill
    Director: Javier Rodríguez Martín
    \vfill
    {\large Septiembre 6, 2024\par}
\end{titlepage}

% Resumen (máximo de 5 páginas, incluyendo al final Palabras clave)
\clearpage
\pagestyle{simple}
% \newpage
\chapter*{Resumen}
\addcontentsline{toc}{chapter}{Resumen}
\input{capitulos/resumen/main.tex}

% Índice (paginado)
\clearpage
\pagestyle{simple}
% \newpage
\tableofcontents

% Introducción (donde se incluya los antecedentes y justificación)
\clearpage
\pagestyle{myfancy}
\newpage
\chapter{Introducción}
\input{capitulos/introduccion/main.tex}

% Objetivos
\chapter{Objetivos}
\input{capitulos/objetivos/main.tex}

% Metodología
\chapter{Metodología}
\input{capitulos/metodologia/main.tex}

% Resultados y discusión (incluyendo la valoración de impactos y de aspectos de responsabilidad legal, ética y profesional relacionados con el trabajo)
\chapter{Resultados y Discusión}
\input{capitulos/resultados_discusion/main.tex}

% Conclusiones
\chapter{Conclusiones}
\input{capitulos/conclusiones/main.tex}

% Planificación temporal y presupuesto
\chapter{Planificación Temporal y Presupuesto}
\input{capitulos/planificacion_presupuesto/main.tex}

% Bibliografía
\newpage
\addcontentsline{toc}{chapter}{Bibliografía}
\printbibliography

\end{document}


% Planificación temporal y presupuesto
\chapter{Planificación Temporal y Presupuesto}
\documentclass[a4paper,11pt,twoside]{report}
\usepackage[left=25mm,right=25mm,top=25mm,bottom=25mm,includehead,includefoot,headsep=15mm,footskip=15mm]{geometry}
\usepackage{graphicx}
\usepackage{fancyhdr}
\usepackage{titlesec}
\usepackage[spanish]{babel}
\usepackage[utf8]{inputenc}
\usepackage{amsmath}
\usepackage{setspace}
\usepackage{svg}
\usepackage{hyperref}
\usepackage[backend=biber,style=numeric]{biblatex}
\addbibresource{references.bib}
\hypersetup{
    colorlinks=true,
    linkcolor=blue,      % color of internal links (sections, etc.)
    urlcolor=blue,       % color of external links
    pdftitle={Optimización energética de sistema híbrido con bomba de calor, suelo radiante, fotovoltaica y almacenamiento para vivienda},    % title
    pdfauthor={Luis D. Aranda Sánchez},     % author
    pdfkeywords={palabra1, palabra2, código1, etc.} % list of keywords
}

% Font change to Arial
\usepackage{helvet}
\renewcommand{\familydefault}{\sfdefault}

% Chapter titles in uppercase and larger font
\titleformat{\chapter}[hang]{\large\bfseries}{\thechapter.}{1em}{\MakeUppercase}
\titleformat{\section}[hang]{\bfseries}{\thesection.}{1em}{}
\titleformat{\subsection}[hang]{\bfseries}{\thesubsection.}{1em}{}

% Fancyhdr setup
\setlength{\headheight}{14.30174pt} % Adjust to recommended value, slightly larger for safety
\fancyhf{} % Clear all headers and footers
\fancyhead[LE]{\nouppercase{\leftmark}}
\fancyhead[RO]{Optimización energética para vivienda}
\fancyfoot[LE]{\thepage}
\fancyfoot[RE]{Escuela Técnica Superior de Ingenieros Industriales (UPM)}
\fancyfoot[LO]{Luis D. Aranda Sánchez}
\fancyfoot[RO]{\thepage}
\renewcommand{\headrulewidth}{0.4pt}
\renewcommand{\footrulewidth}{0.4pt}

\fancypagestyle{myfancy}{
    \fancyhf{} % Clear all headers and footers
    \fancyhead[LE]{\nouppercase{\leftmark}}
    \fancyhead[RO]{Optimización energética para vivienda}
    \fancyfoot[LE]{\thepage}
    \fancyfoot[RE]{Escuela Técnica Superior de Ingenieros Industriales (UPM)}
    \fancyfoot[LO]{Luis D. Aranda Sánchez}
    \fancyfoot[RO]{\thepage}
    \renewcommand{\headrulewidth}{0.4pt}
    \renewcommand{\footrulewidth}{0.4pt}
}

\fancypagestyle{simple}{
    \fancyhf{} % Clear all headers and footers
    \renewcommand{\headrulewidth}{0pt}
    \renewcommand{\footrulewidth}{0pt}
}

% Line spacing
\setstretch{1.2}

% Document starts here
\begin{document}

% Portada
\begin{titlepage}
    \centering
    {\scshape\LARGE Universidad Politécnica de Madrid \par}
    \vspace{1cm}
    {\scshape\Large Escuela Técnica Superior de Ingenieros Industriales\par}
    \vspace{1.5cm}
    {\huge\bfseries Optimización energética de sistema híbrido con bomba de calor, suelo radiante, fotovoltaica y almacenamiento para vivienda \par}
    \vspace{1.5cm}
    {\Large\bfseries Trabajo de Fin de Máster\par}
    \vspace{0.5cm}
    {\large Máster Universitario en Ingeniería de la Energía \par}
    \vspace{2cm}
    {\Large Luis D. Aranda Sánchez\par}
    \vfill
    Director: Javier Rodríguez Martín
    \vfill
    {\large Septiembre 6, 2024\par}
\end{titlepage}

% Resumen (máximo de 5 páginas, incluyendo al final Palabras clave)
\clearpage
\pagestyle{simple}
% \newpage
\chapter*{Resumen}
\addcontentsline{toc}{chapter}{Resumen}
\input{capitulos/resumen/main.tex}

% Índice (paginado)
\clearpage
\pagestyle{simple}
% \newpage
\tableofcontents

% Introducción (donde se incluya los antecedentes y justificación)
\clearpage
\pagestyle{myfancy}
\newpage
\chapter{Introducción}
\input{capitulos/introduccion/main.tex}

% Objetivos
\chapter{Objetivos}
\input{capitulos/objetivos/main.tex}

% Metodología
\chapter{Metodología}
\input{capitulos/metodologia/main.tex}

% Resultados y discusión (incluyendo la valoración de impactos y de aspectos de responsabilidad legal, ética y profesional relacionados con el trabajo)
\chapter{Resultados y Discusión}
\input{capitulos/resultados_discusion/main.tex}

% Conclusiones
\chapter{Conclusiones}
\input{capitulos/conclusiones/main.tex}

% Planificación temporal y presupuesto
\chapter{Planificación Temporal y Presupuesto}
\input{capitulos/planificacion_presupuesto/main.tex}

% Bibliografía
\newpage
\addcontentsline{toc}{chapter}{Bibliografía}
\printbibliography

\end{document}


% Bibliografía
\newpage
\addcontentsline{toc}{chapter}{Bibliografía}
\printbibliography

\end{document}


% Bibliografía
\newpage
\addcontentsline{toc}{chapter}{Bibliografía}
\printbibliography

\end{document}


% Índice (paginado)
\cleardoublepage
\pagestyle{myfancy}
\tableofcontents

% Introducción
\cleardoublepage
\chapter{Introducción}
\documentclass[a4paper,11pt,twoside]{report}
\usepackage[left=25mm,right=25mm,top=25mm,bottom=25mm,includehead,includefoot,headsep=15mm,footskip=15mm]{geometry}
\usepackage{graphicx}
\usepackage{fancyhdr}
\usepackage{titlesec}
\usepackage[spanish]{babel}
\usepackage[utf8]{inputenc}
\usepackage{amsmath}
\usepackage{setspace}
\usepackage{svg}
\usepackage{hyperref}
\usepackage[backend=biber,style=numeric]{biblatex}
\addbibresource{references.bib}
\hypersetup{
    colorlinks=true,
    linkcolor=blue,      % color of internal links (sections, etc.)
    urlcolor=blue,       % color of external links
    pdftitle={Optimización energética de sistema híbrido con bomba de calor, suelo radiante, fotovoltaica y almacenamiento para vivienda},    % title
    pdfauthor={Luis D. Aranda Sánchez},     % author
    pdfkeywords={palabra1, palabra2, código1, etc.} % list of keywords
}

% Font change to Arial
\usepackage{helvet}
\renewcommand{\familydefault}{\sfdefault}

% Chapter titles in uppercase and larger font
\titleformat{\chapter}[hang]{\large\bfseries}{\thechapter.}{1em}{\MakeUppercase}
\titleformat{\section}[hang]{\bfseries}{\thesection.}{1em}{}
\titleformat{\subsection}[hang]{\bfseries}{\thesubsection.}{1em}{}

% Fancyhdr setup
\setlength{\headheight}{14.30174pt} % Adjust to recommended value, slightly larger for safety
\fancyhf{} % Clear all headers and footers
\fancyhead[LE]{\nouppercase{\leftmark}}
\fancyhead[RO]{Optimización energética para vivienda}
\fancyfoot[LE]{\thepage}
\fancyfoot[RE]{Escuela Técnica Superior de Ingenieros Industriales (UPM)}
\fancyfoot[LO]{Luis D. Aranda Sánchez}
\fancyfoot[RO]{\thepage}
\renewcommand{\headrulewidth}{0.4pt}
\renewcommand{\footrulewidth}{0.4pt}

\fancypagestyle{myfancy}{
    \fancyhf{} % Clear all headers and footers
    \fancyhead[LE]{\nouppercase{\leftmark}}
    \fancyhead[RO]{Optimización energética para vivienda}
    \fancyfoot[LE]{\thepage}
    \fancyfoot[RE]{Escuela Técnica Superior de Ingenieros Industriales (UPM)}
    \fancyfoot[LO]{Luis D. Aranda Sánchez}
    \fancyfoot[RO]{\thepage}
    \renewcommand{\headrulewidth}{0.4pt}
    \renewcommand{\footrulewidth}{0.4pt}
}

\fancypagestyle{simple}{
    \fancyhf{} % Clear all headers and footers
    \renewcommand{\headrulewidth}{0pt}
    \renewcommand{\footrulewidth}{0pt}
}

% Line spacing
\setstretch{1.2}

% Document starts here
\begin{document}

% Portada
\begin{titlepage}
    \centering
    {\scshape\LARGE Universidad Politécnica de Madrid \par}
    \vspace{1cm}
    {\scshape\Large Escuela Técnica Superior de Ingenieros Industriales\par}
    \vspace{1.5cm}
    {\huge\bfseries Optimización energética de sistema híbrido con bomba de calor, suelo radiante, fotovoltaica y almacenamiento para vivienda \par}
    \vspace{1.5cm}
    {\Large\bfseries Trabajo de Fin de Máster\par}
    \vspace{0.5cm}
    {\large Máster Universitario en Ingeniería de la Energía \par}
    \vspace{2cm}
    {\Large Luis D. Aranda Sánchez\par}
    \vfill
    Director: Javier Rodríguez Martín
    \vfill
    {\large Septiembre 6, 2024\par}
\end{titlepage}

% Resumen (máximo de 5 páginas, incluyendo al final Palabras clave)
\clearpage
\pagestyle{simple}
% \newpage
\chapter*{Resumen}
\addcontentsline{toc}{chapter}{Resumen}
\documentclass[a4paper,11pt,twoside]{report}
\usepackage[left=25mm,right=25mm,top=25mm,bottom=25mm,includehead,includefoot,headsep=15mm,footskip=15mm]{geometry}
\usepackage{graphicx}
\usepackage{fancyhdr}
\usepackage{titlesec}
\usepackage[spanish]{babel}
\usepackage[utf8]{inputenc}
\usepackage{amsmath}
\usepackage{setspace}
\usepackage{svg}
\usepackage{hyperref}
\usepackage[backend=biber,style=numeric]{biblatex}
\addbibresource{references.bib}
\hypersetup{
    colorlinks=true,
    linkcolor=blue,      % color of internal links (sections, etc.)
    urlcolor=blue,       % color of external links
    pdftitle={Optimización energética de sistema híbrido con bomba de calor, suelo radiante, fotovoltaica y almacenamiento para vivienda},    % title
    pdfauthor={Luis D. Aranda Sánchez},     % author
    pdfkeywords={palabra1, palabra2, código1, etc.} % list of keywords
}

% Font change to Arial
\usepackage{helvet}
\renewcommand{\familydefault}{\sfdefault}

% Chapter titles in uppercase and larger font
\titleformat{\chapter}[hang]{\large\bfseries}{\thechapter.}{1em}{\MakeUppercase}
\titleformat{\section}[hang]{\bfseries}{\thesection.}{1em}{}
\titleformat{\subsection}[hang]{\bfseries}{\thesubsection.}{1em}{}

% Fancyhdr setup
\setlength{\headheight}{14.30174pt} % Adjust to recommended value, slightly larger for safety
\fancyhf{} % Clear all headers and footers
\fancyhead[LE]{\nouppercase{\leftmark}}
\fancyhead[RO]{Optimización energética para vivienda}
\fancyfoot[LE]{\thepage}
\fancyfoot[RE]{Escuela Técnica Superior de Ingenieros Industriales (UPM)}
\fancyfoot[LO]{Luis D. Aranda Sánchez}
\fancyfoot[RO]{\thepage}
\renewcommand{\headrulewidth}{0.4pt}
\renewcommand{\footrulewidth}{0.4pt}

\fancypagestyle{myfancy}{
    \fancyhf{} % Clear all headers and footers
    \fancyhead[LE]{\nouppercase{\leftmark}}
    \fancyhead[RO]{Optimización energética para vivienda}
    \fancyfoot[LE]{\thepage}
    \fancyfoot[RE]{Escuela Técnica Superior de Ingenieros Industriales (UPM)}
    \fancyfoot[LO]{Luis D. Aranda Sánchez}
    \fancyfoot[RO]{\thepage}
    \renewcommand{\headrulewidth}{0.4pt}
    \renewcommand{\footrulewidth}{0.4pt}
}

\fancypagestyle{simple}{
    \fancyhf{} % Clear all headers and footers
    \renewcommand{\headrulewidth}{0pt}
    \renewcommand{\footrulewidth}{0pt}
}

% Line spacing
\setstretch{1.2}

% Document starts here
\begin{document}

% Portada
\begin{titlepage}
    \centering
    {\scshape\LARGE Universidad Politécnica de Madrid \par}
    \vspace{1cm}
    {\scshape\Large Escuela Técnica Superior de Ingenieros Industriales\par}
    \vspace{1.5cm}
    {\huge\bfseries Optimización energética de sistema híbrido con bomba de calor, suelo radiante, fotovoltaica y almacenamiento para vivienda \par}
    \vspace{1.5cm}
    {\Large\bfseries Trabajo de Fin de Máster\par}
    \vspace{0.5cm}
    {\large Máster Universitario en Ingeniería de la Energía \par}
    \vspace{2cm}
    {\Large Luis D. Aranda Sánchez\par}
    \vfill
    Director: Javier Rodríguez Martín
    \vfill
    {\large Septiembre 6, 2024\par}
\end{titlepage}

% Resumen (máximo de 5 páginas, incluyendo al final Palabras clave)
\clearpage
\pagestyle{simple}
% \newpage
\chapter*{Resumen}
\addcontentsline{toc}{chapter}{Resumen}
\documentclass[a4paper,11pt,twoside]{report}
\usepackage[left=25mm,right=25mm,top=25mm,bottom=25mm,includehead,includefoot,headsep=15mm,footskip=15mm]{geometry}
\usepackage{graphicx}
\usepackage{fancyhdr}
\usepackage{titlesec}
\usepackage[spanish]{babel}
\usepackage[utf8]{inputenc}
\usepackage{amsmath}
\usepackage{setspace}
\usepackage{svg}
\usepackage{hyperref}
\usepackage[backend=biber,style=numeric]{biblatex}
\addbibresource{references.bib}
\hypersetup{
    colorlinks=true,
    linkcolor=blue,      % color of internal links (sections, etc.)
    urlcolor=blue,       % color of external links
    pdftitle={Optimización energética de sistema híbrido con bomba de calor, suelo radiante, fotovoltaica y almacenamiento para vivienda},    % title
    pdfauthor={Luis D. Aranda Sánchez},     % author
    pdfkeywords={palabra1, palabra2, código1, etc.} % list of keywords
}

% Font change to Arial
\usepackage{helvet}
\renewcommand{\familydefault}{\sfdefault}

% Chapter titles in uppercase and larger font
\titleformat{\chapter}[hang]{\large\bfseries}{\thechapter.}{1em}{\MakeUppercase}
\titleformat{\section}[hang]{\bfseries}{\thesection.}{1em}{}
\titleformat{\subsection}[hang]{\bfseries}{\thesubsection.}{1em}{}

% Fancyhdr setup
\setlength{\headheight}{14.30174pt} % Adjust to recommended value, slightly larger for safety
\fancyhf{} % Clear all headers and footers
\fancyhead[LE]{\nouppercase{\leftmark}}
\fancyhead[RO]{Optimización energética para vivienda}
\fancyfoot[LE]{\thepage}
\fancyfoot[RE]{Escuela Técnica Superior de Ingenieros Industriales (UPM)}
\fancyfoot[LO]{Luis D. Aranda Sánchez}
\fancyfoot[RO]{\thepage}
\renewcommand{\headrulewidth}{0.4pt}
\renewcommand{\footrulewidth}{0.4pt}

\fancypagestyle{myfancy}{
    \fancyhf{} % Clear all headers and footers
    \fancyhead[LE]{\nouppercase{\leftmark}}
    \fancyhead[RO]{Optimización energética para vivienda}
    \fancyfoot[LE]{\thepage}
    \fancyfoot[RE]{Escuela Técnica Superior de Ingenieros Industriales (UPM)}
    \fancyfoot[LO]{Luis D. Aranda Sánchez}
    \fancyfoot[RO]{\thepage}
    \renewcommand{\headrulewidth}{0.4pt}
    \renewcommand{\footrulewidth}{0.4pt}
}

\fancypagestyle{simple}{
    \fancyhf{} % Clear all headers and footers
    \renewcommand{\headrulewidth}{0pt}
    \renewcommand{\footrulewidth}{0pt}
}

% Line spacing
\setstretch{1.2}

% Document starts here
\begin{document}

% Portada
\begin{titlepage}
    \centering
    {\scshape\LARGE Universidad Politécnica de Madrid \par}
    \vspace{1cm}
    {\scshape\Large Escuela Técnica Superior de Ingenieros Industriales\par}
    \vspace{1.5cm}
    {\huge\bfseries Optimización energética de sistema híbrido con bomba de calor, suelo radiante, fotovoltaica y almacenamiento para vivienda \par}
    \vspace{1.5cm}
    {\Large\bfseries Trabajo de Fin de Máster\par}
    \vspace{0.5cm}
    {\large Máster Universitario en Ingeniería de la Energía \par}
    \vspace{2cm}
    {\Large Luis D. Aranda Sánchez\par}
    \vfill
    Director: Javier Rodríguez Martín
    \vfill
    {\large Septiembre 6, 2024\par}
\end{titlepage}

% Resumen (máximo de 5 páginas, incluyendo al final Palabras clave)
\clearpage
\pagestyle{simple}
% \newpage
\chapter*{Resumen}
\addcontentsline{toc}{chapter}{Resumen}
\input{capitulos/resumen/main.tex}

% Índice (paginado)
\clearpage
\pagestyle{simple}
% \newpage
\tableofcontents

% Introducción (donde se incluya los antecedentes y justificación)
\clearpage
\pagestyle{myfancy}
\newpage
\chapter{Introducción}
\input{capitulos/introduccion/main.tex}

% Objetivos
\chapter{Objetivos}
\input{capitulos/objetivos/main.tex}

% Metodología
\chapter{Metodología}
\input{capitulos/metodologia/main.tex}

% Resultados y discusión (incluyendo la valoración de impactos y de aspectos de responsabilidad legal, ética y profesional relacionados con el trabajo)
\chapter{Resultados y Discusión}
\input{capitulos/resultados_discusion/main.tex}

% Conclusiones
\chapter{Conclusiones}
\input{capitulos/conclusiones/main.tex}

% Planificación temporal y presupuesto
\chapter{Planificación Temporal y Presupuesto}
\input{capitulos/planificacion_presupuesto/main.tex}

% Bibliografía
\newpage
\addcontentsline{toc}{chapter}{Bibliografía}
\printbibliography

\end{document}


% Índice (paginado)
\clearpage
\pagestyle{simple}
% \newpage
\tableofcontents

% Introducción (donde se incluya los antecedentes y justificación)
\clearpage
\pagestyle{myfancy}
\newpage
\chapter{Introducción}
\documentclass[a4paper,11pt,twoside]{report}
\usepackage[left=25mm,right=25mm,top=25mm,bottom=25mm,includehead,includefoot,headsep=15mm,footskip=15mm]{geometry}
\usepackage{graphicx}
\usepackage{fancyhdr}
\usepackage{titlesec}
\usepackage[spanish]{babel}
\usepackage[utf8]{inputenc}
\usepackage{amsmath}
\usepackage{setspace}
\usepackage{svg}
\usepackage{hyperref}
\usepackage[backend=biber,style=numeric]{biblatex}
\addbibresource{references.bib}
\hypersetup{
    colorlinks=true,
    linkcolor=blue,      % color of internal links (sections, etc.)
    urlcolor=blue,       % color of external links
    pdftitle={Optimización energética de sistema híbrido con bomba de calor, suelo radiante, fotovoltaica y almacenamiento para vivienda},    % title
    pdfauthor={Luis D. Aranda Sánchez},     % author
    pdfkeywords={palabra1, palabra2, código1, etc.} % list of keywords
}

% Font change to Arial
\usepackage{helvet}
\renewcommand{\familydefault}{\sfdefault}

% Chapter titles in uppercase and larger font
\titleformat{\chapter}[hang]{\large\bfseries}{\thechapter.}{1em}{\MakeUppercase}
\titleformat{\section}[hang]{\bfseries}{\thesection.}{1em}{}
\titleformat{\subsection}[hang]{\bfseries}{\thesubsection.}{1em}{}

% Fancyhdr setup
\setlength{\headheight}{14.30174pt} % Adjust to recommended value, slightly larger for safety
\fancyhf{} % Clear all headers and footers
\fancyhead[LE]{\nouppercase{\leftmark}}
\fancyhead[RO]{Optimización energética para vivienda}
\fancyfoot[LE]{\thepage}
\fancyfoot[RE]{Escuela Técnica Superior de Ingenieros Industriales (UPM)}
\fancyfoot[LO]{Luis D. Aranda Sánchez}
\fancyfoot[RO]{\thepage}
\renewcommand{\headrulewidth}{0.4pt}
\renewcommand{\footrulewidth}{0.4pt}

\fancypagestyle{myfancy}{
    \fancyhf{} % Clear all headers and footers
    \fancyhead[LE]{\nouppercase{\leftmark}}
    \fancyhead[RO]{Optimización energética para vivienda}
    \fancyfoot[LE]{\thepage}
    \fancyfoot[RE]{Escuela Técnica Superior de Ingenieros Industriales (UPM)}
    \fancyfoot[LO]{Luis D. Aranda Sánchez}
    \fancyfoot[RO]{\thepage}
    \renewcommand{\headrulewidth}{0.4pt}
    \renewcommand{\footrulewidth}{0.4pt}
}

\fancypagestyle{simple}{
    \fancyhf{} % Clear all headers and footers
    \renewcommand{\headrulewidth}{0pt}
    \renewcommand{\footrulewidth}{0pt}
}

% Line spacing
\setstretch{1.2}

% Document starts here
\begin{document}

% Portada
\begin{titlepage}
    \centering
    {\scshape\LARGE Universidad Politécnica de Madrid \par}
    \vspace{1cm}
    {\scshape\Large Escuela Técnica Superior de Ingenieros Industriales\par}
    \vspace{1.5cm}
    {\huge\bfseries Optimización energética de sistema híbrido con bomba de calor, suelo radiante, fotovoltaica y almacenamiento para vivienda \par}
    \vspace{1.5cm}
    {\Large\bfseries Trabajo de Fin de Máster\par}
    \vspace{0.5cm}
    {\large Máster Universitario en Ingeniería de la Energía \par}
    \vspace{2cm}
    {\Large Luis D. Aranda Sánchez\par}
    \vfill
    Director: Javier Rodríguez Martín
    \vfill
    {\large Septiembre 6, 2024\par}
\end{titlepage}

% Resumen (máximo de 5 páginas, incluyendo al final Palabras clave)
\clearpage
\pagestyle{simple}
% \newpage
\chapter*{Resumen}
\addcontentsline{toc}{chapter}{Resumen}
\input{capitulos/resumen/main.tex}

% Índice (paginado)
\clearpage
\pagestyle{simple}
% \newpage
\tableofcontents

% Introducción (donde se incluya los antecedentes y justificación)
\clearpage
\pagestyle{myfancy}
\newpage
\chapter{Introducción}
\input{capitulos/introduccion/main.tex}

% Objetivos
\chapter{Objetivos}
\input{capitulos/objetivos/main.tex}

% Metodología
\chapter{Metodología}
\input{capitulos/metodologia/main.tex}

% Resultados y discusión (incluyendo la valoración de impactos y de aspectos de responsabilidad legal, ética y profesional relacionados con el trabajo)
\chapter{Resultados y Discusión}
\input{capitulos/resultados_discusion/main.tex}

% Conclusiones
\chapter{Conclusiones}
\input{capitulos/conclusiones/main.tex}

% Planificación temporal y presupuesto
\chapter{Planificación Temporal y Presupuesto}
\input{capitulos/planificacion_presupuesto/main.tex}

% Bibliografía
\newpage
\addcontentsline{toc}{chapter}{Bibliografía}
\printbibliography

\end{document}


% Objetivos
\chapter{Objetivos}
\documentclass[a4paper,11pt,twoside]{report}
\usepackage[left=25mm,right=25mm,top=25mm,bottom=25mm,includehead,includefoot,headsep=15mm,footskip=15mm]{geometry}
\usepackage{graphicx}
\usepackage{fancyhdr}
\usepackage{titlesec}
\usepackage[spanish]{babel}
\usepackage[utf8]{inputenc}
\usepackage{amsmath}
\usepackage{setspace}
\usepackage{svg}
\usepackage{hyperref}
\usepackage[backend=biber,style=numeric]{biblatex}
\addbibresource{references.bib}
\hypersetup{
    colorlinks=true,
    linkcolor=blue,      % color of internal links (sections, etc.)
    urlcolor=blue,       % color of external links
    pdftitle={Optimización energética de sistema híbrido con bomba de calor, suelo radiante, fotovoltaica y almacenamiento para vivienda},    % title
    pdfauthor={Luis D. Aranda Sánchez},     % author
    pdfkeywords={palabra1, palabra2, código1, etc.} % list of keywords
}

% Font change to Arial
\usepackage{helvet}
\renewcommand{\familydefault}{\sfdefault}

% Chapter titles in uppercase and larger font
\titleformat{\chapter}[hang]{\large\bfseries}{\thechapter.}{1em}{\MakeUppercase}
\titleformat{\section}[hang]{\bfseries}{\thesection.}{1em}{}
\titleformat{\subsection}[hang]{\bfseries}{\thesubsection.}{1em}{}

% Fancyhdr setup
\setlength{\headheight}{14.30174pt} % Adjust to recommended value, slightly larger for safety
\fancyhf{} % Clear all headers and footers
\fancyhead[LE]{\nouppercase{\leftmark}}
\fancyhead[RO]{Optimización energética para vivienda}
\fancyfoot[LE]{\thepage}
\fancyfoot[RE]{Escuela Técnica Superior de Ingenieros Industriales (UPM)}
\fancyfoot[LO]{Luis D. Aranda Sánchez}
\fancyfoot[RO]{\thepage}
\renewcommand{\headrulewidth}{0.4pt}
\renewcommand{\footrulewidth}{0.4pt}

\fancypagestyle{myfancy}{
    \fancyhf{} % Clear all headers and footers
    \fancyhead[LE]{\nouppercase{\leftmark}}
    \fancyhead[RO]{Optimización energética para vivienda}
    \fancyfoot[LE]{\thepage}
    \fancyfoot[RE]{Escuela Técnica Superior de Ingenieros Industriales (UPM)}
    \fancyfoot[LO]{Luis D. Aranda Sánchez}
    \fancyfoot[RO]{\thepage}
    \renewcommand{\headrulewidth}{0.4pt}
    \renewcommand{\footrulewidth}{0.4pt}
}

\fancypagestyle{simple}{
    \fancyhf{} % Clear all headers and footers
    \renewcommand{\headrulewidth}{0pt}
    \renewcommand{\footrulewidth}{0pt}
}

% Line spacing
\setstretch{1.2}

% Document starts here
\begin{document}

% Portada
\begin{titlepage}
    \centering
    {\scshape\LARGE Universidad Politécnica de Madrid \par}
    \vspace{1cm}
    {\scshape\Large Escuela Técnica Superior de Ingenieros Industriales\par}
    \vspace{1.5cm}
    {\huge\bfseries Optimización energética de sistema híbrido con bomba de calor, suelo radiante, fotovoltaica y almacenamiento para vivienda \par}
    \vspace{1.5cm}
    {\Large\bfseries Trabajo de Fin de Máster\par}
    \vspace{0.5cm}
    {\large Máster Universitario en Ingeniería de la Energía \par}
    \vspace{2cm}
    {\Large Luis D. Aranda Sánchez\par}
    \vfill
    Director: Javier Rodríguez Martín
    \vfill
    {\large Septiembre 6, 2024\par}
\end{titlepage}

% Resumen (máximo de 5 páginas, incluyendo al final Palabras clave)
\clearpage
\pagestyle{simple}
% \newpage
\chapter*{Resumen}
\addcontentsline{toc}{chapter}{Resumen}
\input{capitulos/resumen/main.tex}

% Índice (paginado)
\clearpage
\pagestyle{simple}
% \newpage
\tableofcontents

% Introducción (donde se incluya los antecedentes y justificación)
\clearpage
\pagestyle{myfancy}
\newpage
\chapter{Introducción}
\input{capitulos/introduccion/main.tex}

% Objetivos
\chapter{Objetivos}
\input{capitulos/objetivos/main.tex}

% Metodología
\chapter{Metodología}
\input{capitulos/metodologia/main.tex}

% Resultados y discusión (incluyendo la valoración de impactos y de aspectos de responsabilidad legal, ética y profesional relacionados con el trabajo)
\chapter{Resultados y Discusión}
\input{capitulos/resultados_discusion/main.tex}

% Conclusiones
\chapter{Conclusiones}
\input{capitulos/conclusiones/main.tex}

% Planificación temporal y presupuesto
\chapter{Planificación Temporal y Presupuesto}
\input{capitulos/planificacion_presupuesto/main.tex}

% Bibliografía
\newpage
\addcontentsline{toc}{chapter}{Bibliografía}
\printbibliography

\end{document}


% Metodología
\chapter{Metodología}
\documentclass[a4paper,11pt,twoside]{report}
\usepackage[left=25mm,right=25mm,top=25mm,bottom=25mm,includehead,includefoot,headsep=15mm,footskip=15mm]{geometry}
\usepackage{graphicx}
\usepackage{fancyhdr}
\usepackage{titlesec}
\usepackage[spanish]{babel}
\usepackage[utf8]{inputenc}
\usepackage{amsmath}
\usepackage{setspace}
\usepackage{svg}
\usepackage{hyperref}
\usepackage[backend=biber,style=numeric]{biblatex}
\addbibresource{references.bib}
\hypersetup{
    colorlinks=true,
    linkcolor=blue,      % color of internal links (sections, etc.)
    urlcolor=blue,       % color of external links
    pdftitle={Optimización energética de sistema híbrido con bomba de calor, suelo radiante, fotovoltaica y almacenamiento para vivienda},    % title
    pdfauthor={Luis D. Aranda Sánchez},     % author
    pdfkeywords={palabra1, palabra2, código1, etc.} % list of keywords
}

% Font change to Arial
\usepackage{helvet}
\renewcommand{\familydefault}{\sfdefault}

% Chapter titles in uppercase and larger font
\titleformat{\chapter}[hang]{\large\bfseries}{\thechapter.}{1em}{\MakeUppercase}
\titleformat{\section}[hang]{\bfseries}{\thesection.}{1em}{}
\titleformat{\subsection}[hang]{\bfseries}{\thesubsection.}{1em}{}

% Fancyhdr setup
\setlength{\headheight}{14.30174pt} % Adjust to recommended value, slightly larger for safety
\fancyhf{} % Clear all headers and footers
\fancyhead[LE]{\nouppercase{\leftmark}}
\fancyhead[RO]{Optimización energética para vivienda}
\fancyfoot[LE]{\thepage}
\fancyfoot[RE]{Escuela Técnica Superior de Ingenieros Industriales (UPM)}
\fancyfoot[LO]{Luis D. Aranda Sánchez}
\fancyfoot[RO]{\thepage}
\renewcommand{\headrulewidth}{0.4pt}
\renewcommand{\footrulewidth}{0.4pt}

\fancypagestyle{myfancy}{
    \fancyhf{} % Clear all headers and footers
    \fancyhead[LE]{\nouppercase{\leftmark}}
    \fancyhead[RO]{Optimización energética para vivienda}
    \fancyfoot[LE]{\thepage}
    \fancyfoot[RE]{Escuela Técnica Superior de Ingenieros Industriales (UPM)}
    \fancyfoot[LO]{Luis D. Aranda Sánchez}
    \fancyfoot[RO]{\thepage}
    \renewcommand{\headrulewidth}{0.4pt}
    \renewcommand{\footrulewidth}{0.4pt}
}

\fancypagestyle{simple}{
    \fancyhf{} % Clear all headers and footers
    \renewcommand{\headrulewidth}{0pt}
    \renewcommand{\footrulewidth}{0pt}
}

% Line spacing
\setstretch{1.2}

% Document starts here
\begin{document}

% Portada
\begin{titlepage}
    \centering
    {\scshape\LARGE Universidad Politécnica de Madrid \par}
    \vspace{1cm}
    {\scshape\Large Escuela Técnica Superior de Ingenieros Industriales\par}
    \vspace{1.5cm}
    {\huge\bfseries Optimización energética de sistema híbrido con bomba de calor, suelo radiante, fotovoltaica y almacenamiento para vivienda \par}
    \vspace{1.5cm}
    {\Large\bfseries Trabajo de Fin de Máster\par}
    \vspace{0.5cm}
    {\large Máster Universitario en Ingeniería de la Energía \par}
    \vspace{2cm}
    {\Large Luis D. Aranda Sánchez\par}
    \vfill
    Director: Javier Rodríguez Martín
    \vfill
    {\large Septiembre 6, 2024\par}
\end{titlepage}

% Resumen (máximo de 5 páginas, incluyendo al final Palabras clave)
\clearpage
\pagestyle{simple}
% \newpage
\chapter*{Resumen}
\addcontentsline{toc}{chapter}{Resumen}
\input{capitulos/resumen/main.tex}

% Índice (paginado)
\clearpage
\pagestyle{simple}
% \newpage
\tableofcontents

% Introducción (donde se incluya los antecedentes y justificación)
\clearpage
\pagestyle{myfancy}
\newpage
\chapter{Introducción}
\input{capitulos/introduccion/main.tex}

% Objetivos
\chapter{Objetivos}
\input{capitulos/objetivos/main.tex}

% Metodología
\chapter{Metodología}
\input{capitulos/metodologia/main.tex}

% Resultados y discusión (incluyendo la valoración de impactos y de aspectos de responsabilidad legal, ética y profesional relacionados con el trabajo)
\chapter{Resultados y Discusión}
\input{capitulos/resultados_discusion/main.tex}

% Conclusiones
\chapter{Conclusiones}
\input{capitulos/conclusiones/main.tex}

% Planificación temporal y presupuesto
\chapter{Planificación Temporal y Presupuesto}
\input{capitulos/planificacion_presupuesto/main.tex}

% Bibliografía
\newpage
\addcontentsline{toc}{chapter}{Bibliografía}
\printbibliography

\end{document}


% Resultados y discusión (incluyendo la valoración de impactos y de aspectos de responsabilidad legal, ética y profesional relacionados con el trabajo)
\chapter{Resultados y Discusión}
\documentclass[a4paper,11pt,twoside]{report}
\usepackage[left=25mm,right=25mm,top=25mm,bottom=25mm,includehead,includefoot,headsep=15mm,footskip=15mm]{geometry}
\usepackage{graphicx}
\usepackage{fancyhdr}
\usepackage{titlesec}
\usepackage[spanish]{babel}
\usepackage[utf8]{inputenc}
\usepackage{amsmath}
\usepackage{setspace}
\usepackage{svg}
\usepackage{hyperref}
\usepackage[backend=biber,style=numeric]{biblatex}
\addbibresource{references.bib}
\hypersetup{
    colorlinks=true,
    linkcolor=blue,      % color of internal links (sections, etc.)
    urlcolor=blue,       % color of external links
    pdftitle={Optimización energética de sistema híbrido con bomba de calor, suelo radiante, fotovoltaica y almacenamiento para vivienda},    % title
    pdfauthor={Luis D. Aranda Sánchez},     % author
    pdfkeywords={palabra1, palabra2, código1, etc.} % list of keywords
}

% Font change to Arial
\usepackage{helvet}
\renewcommand{\familydefault}{\sfdefault}

% Chapter titles in uppercase and larger font
\titleformat{\chapter}[hang]{\large\bfseries}{\thechapter.}{1em}{\MakeUppercase}
\titleformat{\section}[hang]{\bfseries}{\thesection.}{1em}{}
\titleformat{\subsection}[hang]{\bfseries}{\thesubsection.}{1em}{}

% Fancyhdr setup
\setlength{\headheight}{14.30174pt} % Adjust to recommended value, slightly larger for safety
\fancyhf{} % Clear all headers and footers
\fancyhead[LE]{\nouppercase{\leftmark}}
\fancyhead[RO]{Optimización energética para vivienda}
\fancyfoot[LE]{\thepage}
\fancyfoot[RE]{Escuela Técnica Superior de Ingenieros Industriales (UPM)}
\fancyfoot[LO]{Luis D. Aranda Sánchez}
\fancyfoot[RO]{\thepage}
\renewcommand{\headrulewidth}{0.4pt}
\renewcommand{\footrulewidth}{0.4pt}

\fancypagestyle{myfancy}{
    \fancyhf{} % Clear all headers and footers
    \fancyhead[LE]{\nouppercase{\leftmark}}
    \fancyhead[RO]{Optimización energética para vivienda}
    \fancyfoot[LE]{\thepage}
    \fancyfoot[RE]{Escuela Técnica Superior de Ingenieros Industriales (UPM)}
    \fancyfoot[LO]{Luis D. Aranda Sánchez}
    \fancyfoot[RO]{\thepage}
    \renewcommand{\headrulewidth}{0.4pt}
    \renewcommand{\footrulewidth}{0.4pt}
}

\fancypagestyle{simple}{
    \fancyhf{} % Clear all headers and footers
    \renewcommand{\headrulewidth}{0pt}
    \renewcommand{\footrulewidth}{0pt}
}

% Line spacing
\setstretch{1.2}

% Document starts here
\begin{document}

% Portada
\begin{titlepage}
    \centering
    {\scshape\LARGE Universidad Politécnica de Madrid \par}
    \vspace{1cm}
    {\scshape\Large Escuela Técnica Superior de Ingenieros Industriales\par}
    \vspace{1.5cm}
    {\huge\bfseries Optimización energética de sistema híbrido con bomba de calor, suelo radiante, fotovoltaica y almacenamiento para vivienda \par}
    \vspace{1.5cm}
    {\Large\bfseries Trabajo de Fin de Máster\par}
    \vspace{0.5cm}
    {\large Máster Universitario en Ingeniería de la Energía \par}
    \vspace{2cm}
    {\Large Luis D. Aranda Sánchez\par}
    \vfill
    Director: Javier Rodríguez Martín
    \vfill
    {\large Septiembre 6, 2024\par}
\end{titlepage}

% Resumen (máximo de 5 páginas, incluyendo al final Palabras clave)
\clearpage
\pagestyle{simple}
% \newpage
\chapter*{Resumen}
\addcontentsline{toc}{chapter}{Resumen}
\input{capitulos/resumen/main.tex}

% Índice (paginado)
\clearpage
\pagestyle{simple}
% \newpage
\tableofcontents

% Introducción (donde se incluya los antecedentes y justificación)
\clearpage
\pagestyle{myfancy}
\newpage
\chapter{Introducción}
\input{capitulos/introduccion/main.tex}

% Objetivos
\chapter{Objetivos}
\input{capitulos/objetivos/main.tex}

% Metodología
\chapter{Metodología}
\input{capitulos/metodologia/main.tex}

% Resultados y discusión (incluyendo la valoración de impactos y de aspectos de responsabilidad legal, ética y profesional relacionados con el trabajo)
\chapter{Resultados y Discusión}
\input{capitulos/resultados_discusion/main.tex}

% Conclusiones
\chapter{Conclusiones}
\input{capitulos/conclusiones/main.tex}

% Planificación temporal y presupuesto
\chapter{Planificación Temporal y Presupuesto}
\input{capitulos/planificacion_presupuesto/main.tex}

% Bibliografía
\newpage
\addcontentsline{toc}{chapter}{Bibliografía}
\printbibliography

\end{document}


% Conclusiones
\chapter{Conclusiones}
\documentclass[a4paper,11pt,twoside]{report}
\usepackage[left=25mm,right=25mm,top=25mm,bottom=25mm,includehead,includefoot,headsep=15mm,footskip=15mm]{geometry}
\usepackage{graphicx}
\usepackage{fancyhdr}
\usepackage{titlesec}
\usepackage[spanish]{babel}
\usepackage[utf8]{inputenc}
\usepackage{amsmath}
\usepackage{setspace}
\usepackage{svg}
\usepackage{hyperref}
\usepackage[backend=biber,style=numeric]{biblatex}
\addbibresource{references.bib}
\hypersetup{
    colorlinks=true,
    linkcolor=blue,      % color of internal links (sections, etc.)
    urlcolor=blue,       % color of external links
    pdftitle={Optimización energética de sistema híbrido con bomba de calor, suelo radiante, fotovoltaica y almacenamiento para vivienda},    % title
    pdfauthor={Luis D. Aranda Sánchez},     % author
    pdfkeywords={palabra1, palabra2, código1, etc.} % list of keywords
}

% Font change to Arial
\usepackage{helvet}
\renewcommand{\familydefault}{\sfdefault}

% Chapter titles in uppercase and larger font
\titleformat{\chapter}[hang]{\large\bfseries}{\thechapter.}{1em}{\MakeUppercase}
\titleformat{\section}[hang]{\bfseries}{\thesection.}{1em}{}
\titleformat{\subsection}[hang]{\bfseries}{\thesubsection.}{1em}{}

% Fancyhdr setup
\setlength{\headheight}{14.30174pt} % Adjust to recommended value, slightly larger for safety
\fancyhf{} % Clear all headers and footers
\fancyhead[LE]{\nouppercase{\leftmark}}
\fancyhead[RO]{Optimización energética para vivienda}
\fancyfoot[LE]{\thepage}
\fancyfoot[RE]{Escuela Técnica Superior de Ingenieros Industriales (UPM)}
\fancyfoot[LO]{Luis D. Aranda Sánchez}
\fancyfoot[RO]{\thepage}
\renewcommand{\headrulewidth}{0.4pt}
\renewcommand{\footrulewidth}{0.4pt}

\fancypagestyle{myfancy}{
    \fancyhf{} % Clear all headers and footers
    \fancyhead[LE]{\nouppercase{\leftmark}}
    \fancyhead[RO]{Optimización energética para vivienda}
    \fancyfoot[LE]{\thepage}
    \fancyfoot[RE]{Escuela Técnica Superior de Ingenieros Industriales (UPM)}
    \fancyfoot[LO]{Luis D. Aranda Sánchez}
    \fancyfoot[RO]{\thepage}
    \renewcommand{\headrulewidth}{0.4pt}
    \renewcommand{\footrulewidth}{0.4pt}
}

\fancypagestyle{simple}{
    \fancyhf{} % Clear all headers and footers
    \renewcommand{\headrulewidth}{0pt}
    \renewcommand{\footrulewidth}{0pt}
}

% Line spacing
\setstretch{1.2}

% Document starts here
\begin{document}

% Portada
\begin{titlepage}
    \centering
    {\scshape\LARGE Universidad Politécnica de Madrid \par}
    \vspace{1cm}
    {\scshape\Large Escuela Técnica Superior de Ingenieros Industriales\par}
    \vspace{1.5cm}
    {\huge\bfseries Optimización energética de sistema híbrido con bomba de calor, suelo radiante, fotovoltaica y almacenamiento para vivienda \par}
    \vspace{1.5cm}
    {\Large\bfseries Trabajo de Fin de Máster\par}
    \vspace{0.5cm}
    {\large Máster Universitario en Ingeniería de la Energía \par}
    \vspace{2cm}
    {\Large Luis D. Aranda Sánchez\par}
    \vfill
    Director: Javier Rodríguez Martín
    \vfill
    {\large Septiembre 6, 2024\par}
\end{titlepage}

% Resumen (máximo de 5 páginas, incluyendo al final Palabras clave)
\clearpage
\pagestyle{simple}
% \newpage
\chapter*{Resumen}
\addcontentsline{toc}{chapter}{Resumen}
\input{capitulos/resumen/main.tex}

% Índice (paginado)
\clearpage
\pagestyle{simple}
% \newpage
\tableofcontents

% Introducción (donde se incluya los antecedentes y justificación)
\clearpage
\pagestyle{myfancy}
\newpage
\chapter{Introducción}
\input{capitulos/introduccion/main.tex}

% Objetivos
\chapter{Objetivos}
\input{capitulos/objetivos/main.tex}

% Metodología
\chapter{Metodología}
\input{capitulos/metodologia/main.tex}

% Resultados y discusión (incluyendo la valoración de impactos y de aspectos de responsabilidad legal, ética y profesional relacionados con el trabajo)
\chapter{Resultados y Discusión}
\input{capitulos/resultados_discusion/main.tex}

% Conclusiones
\chapter{Conclusiones}
\input{capitulos/conclusiones/main.tex}

% Planificación temporal y presupuesto
\chapter{Planificación Temporal y Presupuesto}
\input{capitulos/planificacion_presupuesto/main.tex}

% Bibliografía
\newpage
\addcontentsline{toc}{chapter}{Bibliografía}
\printbibliography

\end{document}


% Planificación temporal y presupuesto
\chapter{Planificación Temporal y Presupuesto}
\documentclass[a4paper,11pt,twoside]{report}
\usepackage[left=25mm,right=25mm,top=25mm,bottom=25mm,includehead,includefoot,headsep=15mm,footskip=15mm]{geometry}
\usepackage{graphicx}
\usepackage{fancyhdr}
\usepackage{titlesec}
\usepackage[spanish]{babel}
\usepackage[utf8]{inputenc}
\usepackage{amsmath}
\usepackage{setspace}
\usepackage{svg}
\usepackage{hyperref}
\usepackage[backend=biber,style=numeric]{biblatex}
\addbibresource{references.bib}
\hypersetup{
    colorlinks=true,
    linkcolor=blue,      % color of internal links (sections, etc.)
    urlcolor=blue,       % color of external links
    pdftitle={Optimización energética de sistema híbrido con bomba de calor, suelo radiante, fotovoltaica y almacenamiento para vivienda},    % title
    pdfauthor={Luis D. Aranda Sánchez},     % author
    pdfkeywords={palabra1, palabra2, código1, etc.} % list of keywords
}

% Font change to Arial
\usepackage{helvet}
\renewcommand{\familydefault}{\sfdefault}

% Chapter titles in uppercase and larger font
\titleformat{\chapter}[hang]{\large\bfseries}{\thechapter.}{1em}{\MakeUppercase}
\titleformat{\section}[hang]{\bfseries}{\thesection.}{1em}{}
\titleformat{\subsection}[hang]{\bfseries}{\thesubsection.}{1em}{}

% Fancyhdr setup
\setlength{\headheight}{14.30174pt} % Adjust to recommended value, slightly larger for safety
\fancyhf{} % Clear all headers and footers
\fancyhead[LE]{\nouppercase{\leftmark}}
\fancyhead[RO]{Optimización energética para vivienda}
\fancyfoot[LE]{\thepage}
\fancyfoot[RE]{Escuela Técnica Superior de Ingenieros Industriales (UPM)}
\fancyfoot[LO]{Luis D. Aranda Sánchez}
\fancyfoot[RO]{\thepage}
\renewcommand{\headrulewidth}{0.4pt}
\renewcommand{\footrulewidth}{0.4pt}

\fancypagestyle{myfancy}{
    \fancyhf{} % Clear all headers and footers
    \fancyhead[LE]{\nouppercase{\leftmark}}
    \fancyhead[RO]{Optimización energética para vivienda}
    \fancyfoot[LE]{\thepage}
    \fancyfoot[RE]{Escuela Técnica Superior de Ingenieros Industriales (UPM)}
    \fancyfoot[LO]{Luis D. Aranda Sánchez}
    \fancyfoot[RO]{\thepage}
    \renewcommand{\headrulewidth}{0.4pt}
    \renewcommand{\footrulewidth}{0.4pt}
}

\fancypagestyle{simple}{
    \fancyhf{} % Clear all headers and footers
    \renewcommand{\headrulewidth}{0pt}
    \renewcommand{\footrulewidth}{0pt}
}

% Line spacing
\setstretch{1.2}

% Document starts here
\begin{document}

% Portada
\begin{titlepage}
    \centering
    {\scshape\LARGE Universidad Politécnica de Madrid \par}
    \vspace{1cm}
    {\scshape\Large Escuela Técnica Superior de Ingenieros Industriales\par}
    \vspace{1.5cm}
    {\huge\bfseries Optimización energética de sistema híbrido con bomba de calor, suelo radiante, fotovoltaica y almacenamiento para vivienda \par}
    \vspace{1.5cm}
    {\Large\bfseries Trabajo de Fin de Máster\par}
    \vspace{0.5cm}
    {\large Máster Universitario en Ingeniería de la Energía \par}
    \vspace{2cm}
    {\Large Luis D. Aranda Sánchez\par}
    \vfill
    Director: Javier Rodríguez Martín
    \vfill
    {\large Septiembre 6, 2024\par}
\end{titlepage}

% Resumen (máximo de 5 páginas, incluyendo al final Palabras clave)
\clearpage
\pagestyle{simple}
% \newpage
\chapter*{Resumen}
\addcontentsline{toc}{chapter}{Resumen}
\input{capitulos/resumen/main.tex}

% Índice (paginado)
\clearpage
\pagestyle{simple}
% \newpage
\tableofcontents

% Introducción (donde se incluya los antecedentes y justificación)
\clearpage
\pagestyle{myfancy}
\newpage
\chapter{Introducción}
\input{capitulos/introduccion/main.tex}

% Objetivos
\chapter{Objetivos}
\input{capitulos/objetivos/main.tex}

% Metodología
\chapter{Metodología}
\input{capitulos/metodologia/main.tex}

% Resultados y discusión (incluyendo la valoración de impactos y de aspectos de responsabilidad legal, ética y profesional relacionados con el trabajo)
\chapter{Resultados y Discusión}
\input{capitulos/resultados_discusion/main.tex}

% Conclusiones
\chapter{Conclusiones}
\input{capitulos/conclusiones/main.tex}

% Planificación temporal y presupuesto
\chapter{Planificación Temporal y Presupuesto}
\input{capitulos/planificacion_presupuesto/main.tex}

% Bibliografía
\newpage
\addcontentsline{toc}{chapter}{Bibliografía}
\printbibliography

\end{document}


% Bibliografía
\newpage
\addcontentsline{toc}{chapter}{Bibliografía}
\printbibliography

\end{document}


% Índice (paginado)
\clearpage
\pagestyle{simple}
% \newpage
\tableofcontents

% Introducción (donde se incluya los antecedentes y justificación)
\clearpage
\pagestyle{myfancy}
\newpage
\chapter{Introducción}
\documentclass[a4paper,11pt,twoside]{report}
\usepackage[left=25mm,right=25mm,top=25mm,bottom=25mm,includehead,includefoot,headsep=15mm,footskip=15mm]{geometry}
\usepackage{graphicx}
\usepackage{fancyhdr}
\usepackage{titlesec}
\usepackage[spanish]{babel}
\usepackage[utf8]{inputenc}
\usepackage{amsmath}
\usepackage{setspace}
\usepackage{svg}
\usepackage{hyperref}
\usepackage[backend=biber,style=numeric]{biblatex}
\addbibresource{references.bib}
\hypersetup{
    colorlinks=true,
    linkcolor=blue,      % color of internal links (sections, etc.)
    urlcolor=blue,       % color of external links
    pdftitle={Optimización energética de sistema híbrido con bomba de calor, suelo radiante, fotovoltaica y almacenamiento para vivienda},    % title
    pdfauthor={Luis D. Aranda Sánchez},     % author
    pdfkeywords={palabra1, palabra2, código1, etc.} % list of keywords
}

% Font change to Arial
\usepackage{helvet}
\renewcommand{\familydefault}{\sfdefault}

% Chapter titles in uppercase and larger font
\titleformat{\chapter}[hang]{\large\bfseries}{\thechapter.}{1em}{\MakeUppercase}
\titleformat{\section}[hang]{\bfseries}{\thesection.}{1em}{}
\titleformat{\subsection}[hang]{\bfseries}{\thesubsection.}{1em}{}

% Fancyhdr setup
\setlength{\headheight}{14.30174pt} % Adjust to recommended value, slightly larger for safety
\fancyhf{} % Clear all headers and footers
\fancyhead[LE]{\nouppercase{\leftmark}}
\fancyhead[RO]{Optimización energética para vivienda}
\fancyfoot[LE]{\thepage}
\fancyfoot[RE]{Escuela Técnica Superior de Ingenieros Industriales (UPM)}
\fancyfoot[LO]{Luis D. Aranda Sánchez}
\fancyfoot[RO]{\thepage}
\renewcommand{\headrulewidth}{0.4pt}
\renewcommand{\footrulewidth}{0.4pt}

\fancypagestyle{myfancy}{
    \fancyhf{} % Clear all headers and footers
    \fancyhead[LE]{\nouppercase{\leftmark}}
    \fancyhead[RO]{Optimización energética para vivienda}
    \fancyfoot[LE]{\thepage}
    \fancyfoot[RE]{Escuela Técnica Superior de Ingenieros Industriales (UPM)}
    \fancyfoot[LO]{Luis D. Aranda Sánchez}
    \fancyfoot[RO]{\thepage}
    \renewcommand{\headrulewidth}{0.4pt}
    \renewcommand{\footrulewidth}{0.4pt}
}

\fancypagestyle{simple}{
    \fancyhf{} % Clear all headers and footers
    \renewcommand{\headrulewidth}{0pt}
    \renewcommand{\footrulewidth}{0pt}
}

% Line spacing
\setstretch{1.2}

% Document starts here
\begin{document}

% Portada
\begin{titlepage}
    \centering
    {\scshape\LARGE Universidad Politécnica de Madrid \par}
    \vspace{1cm}
    {\scshape\Large Escuela Técnica Superior de Ingenieros Industriales\par}
    \vspace{1.5cm}
    {\huge\bfseries Optimización energética de sistema híbrido con bomba de calor, suelo radiante, fotovoltaica y almacenamiento para vivienda \par}
    \vspace{1.5cm}
    {\Large\bfseries Trabajo de Fin de Máster\par}
    \vspace{0.5cm}
    {\large Máster Universitario en Ingeniería de la Energía \par}
    \vspace{2cm}
    {\Large Luis D. Aranda Sánchez\par}
    \vfill
    Director: Javier Rodríguez Martín
    \vfill
    {\large Septiembre 6, 2024\par}
\end{titlepage}

% Resumen (máximo de 5 páginas, incluyendo al final Palabras clave)
\clearpage
\pagestyle{simple}
% \newpage
\chapter*{Resumen}
\addcontentsline{toc}{chapter}{Resumen}
\documentclass[a4paper,11pt,twoside]{report}
\usepackage[left=25mm,right=25mm,top=25mm,bottom=25mm,includehead,includefoot,headsep=15mm,footskip=15mm]{geometry}
\usepackage{graphicx}
\usepackage{fancyhdr}
\usepackage{titlesec}
\usepackage[spanish]{babel}
\usepackage[utf8]{inputenc}
\usepackage{amsmath}
\usepackage{setspace}
\usepackage{svg}
\usepackage{hyperref}
\usepackage[backend=biber,style=numeric]{biblatex}
\addbibresource{references.bib}
\hypersetup{
    colorlinks=true,
    linkcolor=blue,      % color of internal links (sections, etc.)
    urlcolor=blue,       % color of external links
    pdftitle={Optimización energética de sistema híbrido con bomba de calor, suelo radiante, fotovoltaica y almacenamiento para vivienda},    % title
    pdfauthor={Luis D. Aranda Sánchez},     % author
    pdfkeywords={palabra1, palabra2, código1, etc.} % list of keywords
}

% Font change to Arial
\usepackage{helvet}
\renewcommand{\familydefault}{\sfdefault}

% Chapter titles in uppercase and larger font
\titleformat{\chapter}[hang]{\large\bfseries}{\thechapter.}{1em}{\MakeUppercase}
\titleformat{\section}[hang]{\bfseries}{\thesection.}{1em}{}
\titleformat{\subsection}[hang]{\bfseries}{\thesubsection.}{1em}{}

% Fancyhdr setup
\setlength{\headheight}{14.30174pt} % Adjust to recommended value, slightly larger for safety
\fancyhf{} % Clear all headers and footers
\fancyhead[LE]{\nouppercase{\leftmark}}
\fancyhead[RO]{Optimización energética para vivienda}
\fancyfoot[LE]{\thepage}
\fancyfoot[RE]{Escuela Técnica Superior de Ingenieros Industriales (UPM)}
\fancyfoot[LO]{Luis D. Aranda Sánchez}
\fancyfoot[RO]{\thepage}
\renewcommand{\headrulewidth}{0.4pt}
\renewcommand{\footrulewidth}{0.4pt}

\fancypagestyle{myfancy}{
    \fancyhf{} % Clear all headers and footers
    \fancyhead[LE]{\nouppercase{\leftmark}}
    \fancyhead[RO]{Optimización energética para vivienda}
    \fancyfoot[LE]{\thepage}
    \fancyfoot[RE]{Escuela Técnica Superior de Ingenieros Industriales (UPM)}
    \fancyfoot[LO]{Luis D. Aranda Sánchez}
    \fancyfoot[RO]{\thepage}
    \renewcommand{\headrulewidth}{0.4pt}
    \renewcommand{\footrulewidth}{0.4pt}
}

\fancypagestyle{simple}{
    \fancyhf{} % Clear all headers and footers
    \renewcommand{\headrulewidth}{0pt}
    \renewcommand{\footrulewidth}{0pt}
}

% Line spacing
\setstretch{1.2}

% Document starts here
\begin{document}

% Portada
\begin{titlepage}
    \centering
    {\scshape\LARGE Universidad Politécnica de Madrid \par}
    \vspace{1cm}
    {\scshape\Large Escuela Técnica Superior de Ingenieros Industriales\par}
    \vspace{1.5cm}
    {\huge\bfseries Optimización energética de sistema híbrido con bomba de calor, suelo radiante, fotovoltaica y almacenamiento para vivienda \par}
    \vspace{1.5cm}
    {\Large\bfseries Trabajo de Fin de Máster\par}
    \vspace{0.5cm}
    {\large Máster Universitario en Ingeniería de la Energía \par}
    \vspace{2cm}
    {\Large Luis D. Aranda Sánchez\par}
    \vfill
    Director: Javier Rodríguez Martín
    \vfill
    {\large Septiembre 6, 2024\par}
\end{titlepage}

% Resumen (máximo de 5 páginas, incluyendo al final Palabras clave)
\clearpage
\pagestyle{simple}
% \newpage
\chapter*{Resumen}
\addcontentsline{toc}{chapter}{Resumen}
\input{capitulos/resumen/main.tex}

% Índice (paginado)
\clearpage
\pagestyle{simple}
% \newpage
\tableofcontents

% Introducción (donde se incluya los antecedentes y justificación)
\clearpage
\pagestyle{myfancy}
\newpage
\chapter{Introducción}
\input{capitulos/introduccion/main.tex}

% Objetivos
\chapter{Objetivos}
\input{capitulos/objetivos/main.tex}

% Metodología
\chapter{Metodología}
\input{capitulos/metodologia/main.tex}

% Resultados y discusión (incluyendo la valoración de impactos y de aspectos de responsabilidad legal, ética y profesional relacionados con el trabajo)
\chapter{Resultados y Discusión}
\input{capitulos/resultados_discusion/main.tex}

% Conclusiones
\chapter{Conclusiones}
\input{capitulos/conclusiones/main.tex}

% Planificación temporal y presupuesto
\chapter{Planificación Temporal y Presupuesto}
\input{capitulos/planificacion_presupuesto/main.tex}

% Bibliografía
\newpage
\addcontentsline{toc}{chapter}{Bibliografía}
\printbibliography

\end{document}


% Índice (paginado)
\clearpage
\pagestyle{simple}
% \newpage
\tableofcontents

% Introducción (donde se incluya los antecedentes y justificación)
\clearpage
\pagestyle{myfancy}
\newpage
\chapter{Introducción}
\documentclass[a4paper,11pt,twoside]{report}
\usepackage[left=25mm,right=25mm,top=25mm,bottom=25mm,includehead,includefoot,headsep=15mm,footskip=15mm]{geometry}
\usepackage{graphicx}
\usepackage{fancyhdr}
\usepackage{titlesec}
\usepackage[spanish]{babel}
\usepackage[utf8]{inputenc}
\usepackage{amsmath}
\usepackage{setspace}
\usepackage{svg}
\usepackage{hyperref}
\usepackage[backend=biber,style=numeric]{biblatex}
\addbibresource{references.bib}
\hypersetup{
    colorlinks=true,
    linkcolor=blue,      % color of internal links (sections, etc.)
    urlcolor=blue,       % color of external links
    pdftitle={Optimización energética de sistema híbrido con bomba de calor, suelo radiante, fotovoltaica y almacenamiento para vivienda},    % title
    pdfauthor={Luis D. Aranda Sánchez},     % author
    pdfkeywords={palabra1, palabra2, código1, etc.} % list of keywords
}

% Font change to Arial
\usepackage{helvet}
\renewcommand{\familydefault}{\sfdefault}

% Chapter titles in uppercase and larger font
\titleformat{\chapter}[hang]{\large\bfseries}{\thechapter.}{1em}{\MakeUppercase}
\titleformat{\section}[hang]{\bfseries}{\thesection.}{1em}{}
\titleformat{\subsection}[hang]{\bfseries}{\thesubsection.}{1em}{}

% Fancyhdr setup
\setlength{\headheight}{14.30174pt} % Adjust to recommended value, slightly larger for safety
\fancyhf{} % Clear all headers and footers
\fancyhead[LE]{\nouppercase{\leftmark}}
\fancyhead[RO]{Optimización energética para vivienda}
\fancyfoot[LE]{\thepage}
\fancyfoot[RE]{Escuela Técnica Superior de Ingenieros Industriales (UPM)}
\fancyfoot[LO]{Luis D. Aranda Sánchez}
\fancyfoot[RO]{\thepage}
\renewcommand{\headrulewidth}{0.4pt}
\renewcommand{\footrulewidth}{0.4pt}

\fancypagestyle{myfancy}{
    \fancyhf{} % Clear all headers and footers
    \fancyhead[LE]{\nouppercase{\leftmark}}
    \fancyhead[RO]{Optimización energética para vivienda}
    \fancyfoot[LE]{\thepage}
    \fancyfoot[RE]{Escuela Técnica Superior de Ingenieros Industriales (UPM)}
    \fancyfoot[LO]{Luis D. Aranda Sánchez}
    \fancyfoot[RO]{\thepage}
    \renewcommand{\headrulewidth}{0.4pt}
    \renewcommand{\footrulewidth}{0.4pt}
}

\fancypagestyle{simple}{
    \fancyhf{} % Clear all headers and footers
    \renewcommand{\headrulewidth}{0pt}
    \renewcommand{\footrulewidth}{0pt}
}

% Line spacing
\setstretch{1.2}

% Document starts here
\begin{document}

% Portada
\begin{titlepage}
    \centering
    {\scshape\LARGE Universidad Politécnica de Madrid \par}
    \vspace{1cm}
    {\scshape\Large Escuela Técnica Superior de Ingenieros Industriales\par}
    \vspace{1.5cm}
    {\huge\bfseries Optimización energética de sistema híbrido con bomba de calor, suelo radiante, fotovoltaica y almacenamiento para vivienda \par}
    \vspace{1.5cm}
    {\Large\bfseries Trabajo de Fin de Máster\par}
    \vspace{0.5cm}
    {\large Máster Universitario en Ingeniería de la Energía \par}
    \vspace{2cm}
    {\Large Luis D. Aranda Sánchez\par}
    \vfill
    Director: Javier Rodríguez Martín
    \vfill
    {\large Septiembre 6, 2024\par}
\end{titlepage}

% Resumen (máximo de 5 páginas, incluyendo al final Palabras clave)
\clearpage
\pagestyle{simple}
% \newpage
\chapter*{Resumen}
\addcontentsline{toc}{chapter}{Resumen}
\input{capitulos/resumen/main.tex}

% Índice (paginado)
\clearpage
\pagestyle{simple}
% \newpage
\tableofcontents

% Introducción (donde se incluya los antecedentes y justificación)
\clearpage
\pagestyle{myfancy}
\newpage
\chapter{Introducción}
\input{capitulos/introduccion/main.tex}

% Objetivos
\chapter{Objetivos}
\input{capitulos/objetivos/main.tex}

% Metodología
\chapter{Metodología}
\input{capitulos/metodologia/main.tex}

% Resultados y discusión (incluyendo la valoración de impactos y de aspectos de responsabilidad legal, ética y profesional relacionados con el trabajo)
\chapter{Resultados y Discusión}
\input{capitulos/resultados_discusion/main.tex}

% Conclusiones
\chapter{Conclusiones}
\input{capitulos/conclusiones/main.tex}

% Planificación temporal y presupuesto
\chapter{Planificación Temporal y Presupuesto}
\input{capitulos/planificacion_presupuesto/main.tex}

% Bibliografía
\newpage
\addcontentsline{toc}{chapter}{Bibliografía}
\printbibliography

\end{document}


% Objetivos
\chapter{Objetivos}
\documentclass[a4paper,11pt,twoside]{report}
\usepackage[left=25mm,right=25mm,top=25mm,bottom=25mm,includehead,includefoot,headsep=15mm,footskip=15mm]{geometry}
\usepackage{graphicx}
\usepackage{fancyhdr}
\usepackage{titlesec}
\usepackage[spanish]{babel}
\usepackage[utf8]{inputenc}
\usepackage{amsmath}
\usepackage{setspace}
\usepackage{svg}
\usepackage{hyperref}
\usepackage[backend=biber,style=numeric]{biblatex}
\addbibresource{references.bib}
\hypersetup{
    colorlinks=true,
    linkcolor=blue,      % color of internal links (sections, etc.)
    urlcolor=blue,       % color of external links
    pdftitle={Optimización energética de sistema híbrido con bomba de calor, suelo radiante, fotovoltaica y almacenamiento para vivienda},    % title
    pdfauthor={Luis D. Aranda Sánchez},     % author
    pdfkeywords={palabra1, palabra2, código1, etc.} % list of keywords
}

% Font change to Arial
\usepackage{helvet}
\renewcommand{\familydefault}{\sfdefault}

% Chapter titles in uppercase and larger font
\titleformat{\chapter}[hang]{\large\bfseries}{\thechapter.}{1em}{\MakeUppercase}
\titleformat{\section}[hang]{\bfseries}{\thesection.}{1em}{}
\titleformat{\subsection}[hang]{\bfseries}{\thesubsection.}{1em}{}

% Fancyhdr setup
\setlength{\headheight}{14.30174pt} % Adjust to recommended value, slightly larger for safety
\fancyhf{} % Clear all headers and footers
\fancyhead[LE]{\nouppercase{\leftmark}}
\fancyhead[RO]{Optimización energética para vivienda}
\fancyfoot[LE]{\thepage}
\fancyfoot[RE]{Escuela Técnica Superior de Ingenieros Industriales (UPM)}
\fancyfoot[LO]{Luis D. Aranda Sánchez}
\fancyfoot[RO]{\thepage}
\renewcommand{\headrulewidth}{0.4pt}
\renewcommand{\footrulewidth}{0.4pt}

\fancypagestyle{myfancy}{
    \fancyhf{} % Clear all headers and footers
    \fancyhead[LE]{\nouppercase{\leftmark}}
    \fancyhead[RO]{Optimización energética para vivienda}
    \fancyfoot[LE]{\thepage}
    \fancyfoot[RE]{Escuela Técnica Superior de Ingenieros Industriales (UPM)}
    \fancyfoot[LO]{Luis D. Aranda Sánchez}
    \fancyfoot[RO]{\thepage}
    \renewcommand{\headrulewidth}{0.4pt}
    \renewcommand{\footrulewidth}{0.4pt}
}

\fancypagestyle{simple}{
    \fancyhf{} % Clear all headers and footers
    \renewcommand{\headrulewidth}{0pt}
    \renewcommand{\footrulewidth}{0pt}
}

% Line spacing
\setstretch{1.2}

% Document starts here
\begin{document}

% Portada
\begin{titlepage}
    \centering
    {\scshape\LARGE Universidad Politécnica de Madrid \par}
    \vspace{1cm}
    {\scshape\Large Escuela Técnica Superior de Ingenieros Industriales\par}
    \vspace{1.5cm}
    {\huge\bfseries Optimización energética de sistema híbrido con bomba de calor, suelo radiante, fotovoltaica y almacenamiento para vivienda \par}
    \vspace{1.5cm}
    {\Large\bfseries Trabajo de Fin de Máster\par}
    \vspace{0.5cm}
    {\large Máster Universitario en Ingeniería de la Energía \par}
    \vspace{2cm}
    {\Large Luis D. Aranda Sánchez\par}
    \vfill
    Director: Javier Rodríguez Martín
    \vfill
    {\large Septiembre 6, 2024\par}
\end{titlepage}

% Resumen (máximo de 5 páginas, incluyendo al final Palabras clave)
\clearpage
\pagestyle{simple}
% \newpage
\chapter*{Resumen}
\addcontentsline{toc}{chapter}{Resumen}
\input{capitulos/resumen/main.tex}

% Índice (paginado)
\clearpage
\pagestyle{simple}
% \newpage
\tableofcontents

% Introducción (donde se incluya los antecedentes y justificación)
\clearpage
\pagestyle{myfancy}
\newpage
\chapter{Introducción}
\input{capitulos/introduccion/main.tex}

% Objetivos
\chapter{Objetivos}
\input{capitulos/objetivos/main.tex}

% Metodología
\chapter{Metodología}
\input{capitulos/metodologia/main.tex}

% Resultados y discusión (incluyendo la valoración de impactos y de aspectos de responsabilidad legal, ética y profesional relacionados con el trabajo)
\chapter{Resultados y Discusión}
\input{capitulos/resultados_discusion/main.tex}

% Conclusiones
\chapter{Conclusiones}
\input{capitulos/conclusiones/main.tex}

% Planificación temporal y presupuesto
\chapter{Planificación Temporal y Presupuesto}
\input{capitulos/planificacion_presupuesto/main.tex}

% Bibliografía
\newpage
\addcontentsline{toc}{chapter}{Bibliografía}
\printbibliography

\end{document}


% Metodología
\chapter{Metodología}
\documentclass[a4paper,11pt,twoside]{report}
\usepackage[left=25mm,right=25mm,top=25mm,bottom=25mm,includehead,includefoot,headsep=15mm,footskip=15mm]{geometry}
\usepackage{graphicx}
\usepackage{fancyhdr}
\usepackage{titlesec}
\usepackage[spanish]{babel}
\usepackage[utf8]{inputenc}
\usepackage{amsmath}
\usepackage{setspace}
\usepackage{svg}
\usepackage{hyperref}
\usepackage[backend=biber,style=numeric]{biblatex}
\addbibresource{references.bib}
\hypersetup{
    colorlinks=true,
    linkcolor=blue,      % color of internal links (sections, etc.)
    urlcolor=blue,       % color of external links
    pdftitle={Optimización energética de sistema híbrido con bomba de calor, suelo radiante, fotovoltaica y almacenamiento para vivienda},    % title
    pdfauthor={Luis D. Aranda Sánchez},     % author
    pdfkeywords={palabra1, palabra2, código1, etc.} % list of keywords
}

% Font change to Arial
\usepackage{helvet}
\renewcommand{\familydefault}{\sfdefault}

% Chapter titles in uppercase and larger font
\titleformat{\chapter}[hang]{\large\bfseries}{\thechapter.}{1em}{\MakeUppercase}
\titleformat{\section}[hang]{\bfseries}{\thesection.}{1em}{}
\titleformat{\subsection}[hang]{\bfseries}{\thesubsection.}{1em}{}

% Fancyhdr setup
\setlength{\headheight}{14.30174pt} % Adjust to recommended value, slightly larger for safety
\fancyhf{} % Clear all headers and footers
\fancyhead[LE]{\nouppercase{\leftmark}}
\fancyhead[RO]{Optimización energética para vivienda}
\fancyfoot[LE]{\thepage}
\fancyfoot[RE]{Escuela Técnica Superior de Ingenieros Industriales (UPM)}
\fancyfoot[LO]{Luis D. Aranda Sánchez}
\fancyfoot[RO]{\thepage}
\renewcommand{\headrulewidth}{0.4pt}
\renewcommand{\footrulewidth}{0.4pt}

\fancypagestyle{myfancy}{
    \fancyhf{} % Clear all headers and footers
    \fancyhead[LE]{\nouppercase{\leftmark}}
    \fancyhead[RO]{Optimización energética para vivienda}
    \fancyfoot[LE]{\thepage}
    \fancyfoot[RE]{Escuela Técnica Superior de Ingenieros Industriales (UPM)}
    \fancyfoot[LO]{Luis D. Aranda Sánchez}
    \fancyfoot[RO]{\thepage}
    \renewcommand{\headrulewidth}{0.4pt}
    \renewcommand{\footrulewidth}{0.4pt}
}

\fancypagestyle{simple}{
    \fancyhf{} % Clear all headers and footers
    \renewcommand{\headrulewidth}{0pt}
    \renewcommand{\footrulewidth}{0pt}
}

% Line spacing
\setstretch{1.2}

% Document starts here
\begin{document}

% Portada
\begin{titlepage}
    \centering
    {\scshape\LARGE Universidad Politécnica de Madrid \par}
    \vspace{1cm}
    {\scshape\Large Escuela Técnica Superior de Ingenieros Industriales\par}
    \vspace{1.5cm}
    {\huge\bfseries Optimización energética de sistema híbrido con bomba de calor, suelo radiante, fotovoltaica y almacenamiento para vivienda \par}
    \vspace{1.5cm}
    {\Large\bfseries Trabajo de Fin de Máster\par}
    \vspace{0.5cm}
    {\large Máster Universitario en Ingeniería de la Energía \par}
    \vspace{2cm}
    {\Large Luis D. Aranda Sánchez\par}
    \vfill
    Director: Javier Rodríguez Martín
    \vfill
    {\large Septiembre 6, 2024\par}
\end{titlepage}

% Resumen (máximo de 5 páginas, incluyendo al final Palabras clave)
\clearpage
\pagestyle{simple}
% \newpage
\chapter*{Resumen}
\addcontentsline{toc}{chapter}{Resumen}
\input{capitulos/resumen/main.tex}

% Índice (paginado)
\clearpage
\pagestyle{simple}
% \newpage
\tableofcontents

% Introducción (donde se incluya los antecedentes y justificación)
\clearpage
\pagestyle{myfancy}
\newpage
\chapter{Introducción}
\input{capitulos/introduccion/main.tex}

% Objetivos
\chapter{Objetivos}
\input{capitulos/objetivos/main.tex}

% Metodología
\chapter{Metodología}
\input{capitulos/metodologia/main.tex}

% Resultados y discusión (incluyendo la valoración de impactos y de aspectos de responsabilidad legal, ética y profesional relacionados con el trabajo)
\chapter{Resultados y Discusión}
\input{capitulos/resultados_discusion/main.tex}

% Conclusiones
\chapter{Conclusiones}
\input{capitulos/conclusiones/main.tex}

% Planificación temporal y presupuesto
\chapter{Planificación Temporal y Presupuesto}
\input{capitulos/planificacion_presupuesto/main.tex}

% Bibliografía
\newpage
\addcontentsline{toc}{chapter}{Bibliografía}
\printbibliography

\end{document}


% Resultados y discusión (incluyendo la valoración de impactos y de aspectos de responsabilidad legal, ética y profesional relacionados con el trabajo)
\chapter{Resultados y Discusión}
\documentclass[a4paper,11pt,twoside]{report}
\usepackage[left=25mm,right=25mm,top=25mm,bottom=25mm,includehead,includefoot,headsep=15mm,footskip=15mm]{geometry}
\usepackage{graphicx}
\usepackage{fancyhdr}
\usepackage{titlesec}
\usepackage[spanish]{babel}
\usepackage[utf8]{inputenc}
\usepackage{amsmath}
\usepackage{setspace}
\usepackage{svg}
\usepackage{hyperref}
\usepackage[backend=biber,style=numeric]{biblatex}
\addbibresource{references.bib}
\hypersetup{
    colorlinks=true,
    linkcolor=blue,      % color of internal links (sections, etc.)
    urlcolor=blue,       % color of external links
    pdftitle={Optimización energética de sistema híbrido con bomba de calor, suelo radiante, fotovoltaica y almacenamiento para vivienda},    % title
    pdfauthor={Luis D. Aranda Sánchez},     % author
    pdfkeywords={palabra1, palabra2, código1, etc.} % list of keywords
}

% Font change to Arial
\usepackage{helvet}
\renewcommand{\familydefault}{\sfdefault}

% Chapter titles in uppercase and larger font
\titleformat{\chapter}[hang]{\large\bfseries}{\thechapter.}{1em}{\MakeUppercase}
\titleformat{\section}[hang]{\bfseries}{\thesection.}{1em}{}
\titleformat{\subsection}[hang]{\bfseries}{\thesubsection.}{1em}{}

% Fancyhdr setup
\setlength{\headheight}{14.30174pt} % Adjust to recommended value, slightly larger for safety
\fancyhf{} % Clear all headers and footers
\fancyhead[LE]{\nouppercase{\leftmark}}
\fancyhead[RO]{Optimización energética para vivienda}
\fancyfoot[LE]{\thepage}
\fancyfoot[RE]{Escuela Técnica Superior de Ingenieros Industriales (UPM)}
\fancyfoot[LO]{Luis D. Aranda Sánchez}
\fancyfoot[RO]{\thepage}
\renewcommand{\headrulewidth}{0.4pt}
\renewcommand{\footrulewidth}{0.4pt}

\fancypagestyle{myfancy}{
    \fancyhf{} % Clear all headers and footers
    \fancyhead[LE]{\nouppercase{\leftmark}}
    \fancyhead[RO]{Optimización energética para vivienda}
    \fancyfoot[LE]{\thepage}
    \fancyfoot[RE]{Escuela Técnica Superior de Ingenieros Industriales (UPM)}
    \fancyfoot[LO]{Luis D. Aranda Sánchez}
    \fancyfoot[RO]{\thepage}
    \renewcommand{\headrulewidth}{0.4pt}
    \renewcommand{\footrulewidth}{0.4pt}
}

\fancypagestyle{simple}{
    \fancyhf{} % Clear all headers and footers
    \renewcommand{\headrulewidth}{0pt}
    \renewcommand{\footrulewidth}{0pt}
}

% Line spacing
\setstretch{1.2}

% Document starts here
\begin{document}

% Portada
\begin{titlepage}
    \centering
    {\scshape\LARGE Universidad Politécnica de Madrid \par}
    \vspace{1cm}
    {\scshape\Large Escuela Técnica Superior de Ingenieros Industriales\par}
    \vspace{1.5cm}
    {\huge\bfseries Optimización energética de sistema híbrido con bomba de calor, suelo radiante, fotovoltaica y almacenamiento para vivienda \par}
    \vspace{1.5cm}
    {\Large\bfseries Trabajo de Fin de Máster\par}
    \vspace{0.5cm}
    {\large Máster Universitario en Ingeniería de la Energía \par}
    \vspace{2cm}
    {\Large Luis D. Aranda Sánchez\par}
    \vfill
    Director: Javier Rodríguez Martín
    \vfill
    {\large Septiembre 6, 2024\par}
\end{titlepage}

% Resumen (máximo de 5 páginas, incluyendo al final Palabras clave)
\clearpage
\pagestyle{simple}
% \newpage
\chapter*{Resumen}
\addcontentsline{toc}{chapter}{Resumen}
\input{capitulos/resumen/main.tex}

% Índice (paginado)
\clearpage
\pagestyle{simple}
% \newpage
\tableofcontents

% Introducción (donde se incluya los antecedentes y justificación)
\clearpage
\pagestyle{myfancy}
\newpage
\chapter{Introducción}
\input{capitulos/introduccion/main.tex}

% Objetivos
\chapter{Objetivos}
\input{capitulos/objetivos/main.tex}

% Metodología
\chapter{Metodología}
\input{capitulos/metodologia/main.tex}

% Resultados y discusión (incluyendo la valoración de impactos y de aspectos de responsabilidad legal, ética y profesional relacionados con el trabajo)
\chapter{Resultados y Discusión}
\input{capitulos/resultados_discusion/main.tex}

% Conclusiones
\chapter{Conclusiones}
\input{capitulos/conclusiones/main.tex}

% Planificación temporal y presupuesto
\chapter{Planificación Temporal y Presupuesto}
\input{capitulos/planificacion_presupuesto/main.tex}

% Bibliografía
\newpage
\addcontentsline{toc}{chapter}{Bibliografía}
\printbibliography

\end{document}


% Conclusiones
\chapter{Conclusiones}
\documentclass[a4paper,11pt,twoside]{report}
\usepackage[left=25mm,right=25mm,top=25mm,bottom=25mm,includehead,includefoot,headsep=15mm,footskip=15mm]{geometry}
\usepackage{graphicx}
\usepackage{fancyhdr}
\usepackage{titlesec}
\usepackage[spanish]{babel}
\usepackage[utf8]{inputenc}
\usepackage{amsmath}
\usepackage{setspace}
\usepackage{svg}
\usepackage{hyperref}
\usepackage[backend=biber,style=numeric]{biblatex}
\addbibresource{references.bib}
\hypersetup{
    colorlinks=true,
    linkcolor=blue,      % color of internal links (sections, etc.)
    urlcolor=blue,       % color of external links
    pdftitle={Optimización energética de sistema híbrido con bomba de calor, suelo radiante, fotovoltaica y almacenamiento para vivienda},    % title
    pdfauthor={Luis D. Aranda Sánchez},     % author
    pdfkeywords={palabra1, palabra2, código1, etc.} % list of keywords
}

% Font change to Arial
\usepackage{helvet}
\renewcommand{\familydefault}{\sfdefault}

% Chapter titles in uppercase and larger font
\titleformat{\chapter}[hang]{\large\bfseries}{\thechapter.}{1em}{\MakeUppercase}
\titleformat{\section}[hang]{\bfseries}{\thesection.}{1em}{}
\titleformat{\subsection}[hang]{\bfseries}{\thesubsection.}{1em}{}

% Fancyhdr setup
\setlength{\headheight}{14.30174pt} % Adjust to recommended value, slightly larger for safety
\fancyhf{} % Clear all headers and footers
\fancyhead[LE]{\nouppercase{\leftmark}}
\fancyhead[RO]{Optimización energética para vivienda}
\fancyfoot[LE]{\thepage}
\fancyfoot[RE]{Escuela Técnica Superior de Ingenieros Industriales (UPM)}
\fancyfoot[LO]{Luis D. Aranda Sánchez}
\fancyfoot[RO]{\thepage}
\renewcommand{\headrulewidth}{0.4pt}
\renewcommand{\footrulewidth}{0.4pt}

\fancypagestyle{myfancy}{
    \fancyhf{} % Clear all headers and footers
    \fancyhead[LE]{\nouppercase{\leftmark}}
    \fancyhead[RO]{Optimización energética para vivienda}
    \fancyfoot[LE]{\thepage}
    \fancyfoot[RE]{Escuela Técnica Superior de Ingenieros Industriales (UPM)}
    \fancyfoot[LO]{Luis D. Aranda Sánchez}
    \fancyfoot[RO]{\thepage}
    \renewcommand{\headrulewidth}{0.4pt}
    \renewcommand{\footrulewidth}{0.4pt}
}

\fancypagestyle{simple}{
    \fancyhf{} % Clear all headers and footers
    \renewcommand{\headrulewidth}{0pt}
    \renewcommand{\footrulewidth}{0pt}
}

% Line spacing
\setstretch{1.2}

% Document starts here
\begin{document}

% Portada
\begin{titlepage}
    \centering
    {\scshape\LARGE Universidad Politécnica de Madrid \par}
    \vspace{1cm}
    {\scshape\Large Escuela Técnica Superior de Ingenieros Industriales\par}
    \vspace{1.5cm}
    {\huge\bfseries Optimización energética de sistema híbrido con bomba de calor, suelo radiante, fotovoltaica y almacenamiento para vivienda \par}
    \vspace{1.5cm}
    {\Large\bfseries Trabajo de Fin de Máster\par}
    \vspace{0.5cm}
    {\large Máster Universitario en Ingeniería de la Energía \par}
    \vspace{2cm}
    {\Large Luis D. Aranda Sánchez\par}
    \vfill
    Director: Javier Rodríguez Martín
    \vfill
    {\large Septiembre 6, 2024\par}
\end{titlepage}

% Resumen (máximo de 5 páginas, incluyendo al final Palabras clave)
\clearpage
\pagestyle{simple}
% \newpage
\chapter*{Resumen}
\addcontentsline{toc}{chapter}{Resumen}
\input{capitulos/resumen/main.tex}

% Índice (paginado)
\clearpage
\pagestyle{simple}
% \newpage
\tableofcontents

% Introducción (donde se incluya los antecedentes y justificación)
\clearpage
\pagestyle{myfancy}
\newpage
\chapter{Introducción}
\input{capitulos/introduccion/main.tex}

% Objetivos
\chapter{Objetivos}
\input{capitulos/objetivos/main.tex}

% Metodología
\chapter{Metodología}
\input{capitulos/metodologia/main.tex}

% Resultados y discusión (incluyendo la valoración de impactos y de aspectos de responsabilidad legal, ética y profesional relacionados con el trabajo)
\chapter{Resultados y Discusión}
\input{capitulos/resultados_discusion/main.tex}

% Conclusiones
\chapter{Conclusiones}
\input{capitulos/conclusiones/main.tex}

% Planificación temporal y presupuesto
\chapter{Planificación Temporal y Presupuesto}
\input{capitulos/planificacion_presupuesto/main.tex}

% Bibliografía
\newpage
\addcontentsline{toc}{chapter}{Bibliografía}
\printbibliography

\end{document}


% Planificación temporal y presupuesto
\chapter{Planificación Temporal y Presupuesto}
\documentclass[a4paper,11pt,twoside]{report}
\usepackage[left=25mm,right=25mm,top=25mm,bottom=25mm,includehead,includefoot,headsep=15mm,footskip=15mm]{geometry}
\usepackage{graphicx}
\usepackage{fancyhdr}
\usepackage{titlesec}
\usepackage[spanish]{babel}
\usepackage[utf8]{inputenc}
\usepackage{amsmath}
\usepackage{setspace}
\usepackage{svg}
\usepackage{hyperref}
\usepackage[backend=biber,style=numeric]{biblatex}
\addbibresource{references.bib}
\hypersetup{
    colorlinks=true,
    linkcolor=blue,      % color of internal links (sections, etc.)
    urlcolor=blue,       % color of external links
    pdftitle={Optimización energética de sistema híbrido con bomba de calor, suelo radiante, fotovoltaica y almacenamiento para vivienda},    % title
    pdfauthor={Luis D. Aranda Sánchez},     % author
    pdfkeywords={palabra1, palabra2, código1, etc.} % list of keywords
}

% Font change to Arial
\usepackage{helvet}
\renewcommand{\familydefault}{\sfdefault}

% Chapter titles in uppercase and larger font
\titleformat{\chapter}[hang]{\large\bfseries}{\thechapter.}{1em}{\MakeUppercase}
\titleformat{\section}[hang]{\bfseries}{\thesection.}{1em}{}
\titleformat{\subsection}[hang]{\bfseries}{\thesubsection.}{1em}{}

% Fancyhdr setup
\setlength{\headheight}{14.30174pt} % Adjust to recommended value, slightly larger for safety
\fancyhf{} % Clear all headers and footers
\fancyhead[LE]{\nouppercase{\leftmark}}
\fancyhead[RO]{Optimización energética para vivienda}
\fancyfoot[LE]{\thepage}
\fancyfoot[RE]{Escuela Técnica Superior de Ingenieros Industriales (UPM)}
\fancyfoot[LO]{Luis D. Aranda Sánchez}
\fancyfoot[RO]{\thepage}
\renewcommand{\headrulewidth}{0.4pt}
\renewcommand{\footrulewidth}{0.4pt}

\fancypagestyle{myfancy}{
    \fancyhf{} % Clear all headers and footers
    \fancyhead[LE]{\nouppercase{\leftmark}}
    \fancyhead[RO]{Optimización energética para vivienda}
    \fancyfoot[LE]{\thepage}
    \fancyfoot[RE]{Escuela Técnica Superior de Ingenieros Industriales (UPM)}
    \fancyfoot[LO]{Luis D. Aranda Sánchez}
    \fancyfoot[RO]{\thepage}
    \renewcommand{\headrulewidth}{0.4pt}
    \renewcommand{\footrulewidth}{0.4pt}
}

\fancypagestyle{simple}{
    \fancyhf{} % Clear all headers and footers
    \renewcommand{\headrulewidth}{0pt}
    \renewcommand{\footrulewidth}{0pt}
}

% Line spacing
\setstretch{1.2}

% Document starts here
\begin{document}

% Portada
\begin{titlepage}
    \centering
    {\scshape\LARGE Universidad Politécnica de Madrid \par}
    \vspace{1cm}
    {\scshape\Large Escuela Técnica Superior de Ingenieros Industriales\par}
    \vspace{1.5cm}
    {\huge\bfseries Optimización energética de sistema híbrido con bomba de calor, suelo radiante, fotovoltaica y almacenamiento para vivienda \par}
    \vspace{1.5cm}
    {\Large\bfseries Trabajo de Fin de Máster\par}
    \vspace{0.5cm}
    {\large Máster Universitario en Ingeniería de la Energía \par}
    \vspace{2cm}
    {\Large Luis D. Aranda Sánchez\par}
    \vfill
    Director: Javier Rodríguez Martín
    \vfill
    {\large Septiembre 6, 2024\par}
\end{titlepage}

% Resumen (máximo de 5 páginas, incluyendo al final Palabras clave)
\clearpage
\pagestyle{simple}
% \newpage
\chapter*{Resumen}
\addcontentsline{toc}{chapter}{Resumen}
\input{capitulos/resumen/main.tex}

% Índice (paginado)
\clearpage
\pagestyle{simple}
% \newpage
\tableofcontents

% Introducción (donde se incluya los antecedentes y justificación)
\clearpage
\pagestyle{myfancy}
\newpage
\chapter{Introducción}
\input{capitulos/introduccion/main.tex}

% Objetivos
\chapter{Objetivos}
\input{capitulos/objetivos/main.tex}

% Metodología
\chapter{Metodología}
\input{capitulos/metodologia/main.tex}

% Resultados y discusión (incluyendo la valoración de impactos y de aspectos de responsabilidad legal, ética y profesional relacionados con el trabajo)
\chapter{Resultados y Discusión}
\input{capitulos/resultados_discusion/main.tex}

% Conclusiones
\chapter{Conclusiones}
\input{capitulos/conclusiones/main.tex}

% Planificación temporal y presupuesto
\chapter{Planificación Temporal y Presupuesto}
\input{capitulos/planificacion_presupuesto/main.tex}

% Bibliografía
\newpage
\addcontentsline{toc}{chapter}{Bibliografía}
\printbibliography

\end{document}


% Bibliografía
\newpage
\addcontentsline{toc}{chapter}{Bibliografía}
\printbibliography

\end{document}


% Objetivos
\chapter{Objetivos}
\documentclass[a4paper,11pt,twoside]{report}
\usepackage[left=25mm,right=25mm,top=25mm,bottom=25mm,includehead,includefoot,headsep=15mm,footskip=15mm]{geometry}
\usepackage{graphicx}
\usepackage{fancyhdr}
\usepackage{titlesec}
\usepackage[spanish]{babel}
\usepackage[utf8]{inputenc}
\usepackage{amsmath}
\usepackage{setspace}
\usepackage{svg}
\usepackage{hyperref}
\usepackage[backend=biber,style=numeric]{biblatex}
\addbibresource{references.bib}
\hypersetup{
    colorlinks=true,
    linkcolor=blue,      % color of internal links (sections, etc.)
    urlcolor=blue,       % color of external links
    pdftitle={Optimización energética de sistema híbrido con bomba de calor, suelo radiante, fotovoltaica y almacenamiento para vivienda},    % title
    pdfauthor={Luis D. Aranda Sánchez},     % author
    pdfkeywords={palabra1, palabra2, código1, etc.} % list of keywords
}

% Font change to Arial
\usepackage{helvet}
\renewcommand{\familydefault}{\sfdefault}

% Chapter titles in uppercase and larger font
\titleformat{\chapter}[hang]{\large\bfseries}{\thechapter.}{1em}{\MakeUppercase}
\titleformat{\section}[hang]{\bfseries}{\thesection.}{1em}{}
\titleformat{\subsection}[hang]{\bfseries}{\thesubsection.}{1em}{}

% Fancyhdr setup
\setlength{\headheight}{14.30174pt} % Adjust to recommended value, slightly larger for safety
\fancyhf{} % Clear all headers and footers
\fancyhead[LE]{\nouppercase{\leftmark}}
\fancyhead[RO]{Optimización energética para vivienda}
\fancyfoot[LE]{\thepage}
\fancyfoot[RE]{Escuela Técnica Superior de Ingenieros Industriales (UPM)}
\fancyfoot[LO]{Luis D. Aranda Sánchez}
\fancyfoot[RO]{\thepage}
\renewcommand{\headrulewidth}{0.4pt}
\renewcommand{\footrulewidth}{0.4pt}

\fancypagestyle{myfancy}{
    \fancyhf{} % Clear all headers and footers
    \fancyhead[LE]{\nouppercase{\leftmark}}
    \fancyhead[RO]{Optimización energética para vivienda}
    \fancyfoot[LE]{\thepage}
    \fancyfoot[RE]{Escuela Técnica Superior de Ingenieros Industriales (UPM)}
    \fancyfoot[LO]{Luis D. Aranda Sánchez}
    \fancyfoot[RO]{\thepage}
    \renewcommand{\headrulewidth}{0.4pt}
    \renewcommand{\footrulewidth}{0.4pt}
}

\fancypagestyle{simple}{
    \fancyhf{} % Clear all headers and footers
    \renewcommand{\headrulewidth}{0pt}
    \renewcommand{\footrulewidth}{0pt}
}

% Line spacing
\setstretch{1.2}

% Document starts here
\begin{document}

% Portada
\begin{titlepage}
    \centering
    {\scshape\LARGE Universidad Politécnica de Madrid \par}
    \vspace{1cm}
    {\scshape\Large Escuela Técnica Superior de Ingenieros Industriales\par}
    \vspace{1.5cm}
    {\huge\bfseries Optimización energética de sistema híbrido con bomba de calor, suelo radiante, fotovoltaica y almacenamiento para vivienda \par}
    \vspace{1.5cm}
    {\Large\bfseries Trabajo de Fin de Máster\par}
    \vspace{0.5cm}
    {\large Máster Universitario en Ingeniería de la Energía \par}
    \vspace{2cm}
    {\Large Luis D. Aranda Sánchez\par}
    \vfill
    Director: Javier Rodríguez Martín
    \vfill
    {\large Septiembre 6, 2024\par}
\end{titlepage}

% Resumen (máximo de 5 páginas, incluyendo al final Palabras clave)
\clearpage
\pagestyle{simple}
% \newpage
\chapter*{Resumen}
\addcontentsline{toc}{chapter}{Resumen}
\documentclass[a4paper,11pt,twoside]{report}
\usepackage[left=25mm,right=25mm,top=25mm,bottom=25mm,includehead,includefoot,headsep=15mm,footskip=15mm]{geometry}
\usepackage{graphicx}
\usepackage{fancyhdr}
\usepackage{titlesec}
\usepackage[spanish]{babel}
\usepackage[utf8]{inputenc}
\usepackage{amsmath}
\usepackage{setspace}
\usepackage{svg}
\usepackage{hyperref}
\usepackage[backend=biber,style=numeric]{biblatex}
\addbibresource{references.bib}
\hypersetup{
    colorlinks=true,
    linkcolor=blue,      % color of internal links (sections, etc.)
    urlcolor=blue,       % color of external links
    pdftitle={Optimización energética de sistema híbrido con bomba de calor, suelo radiante, fotovoltaica y almacenamiento para vivienda},    % title
    pdfauthor={Luis D. Aranda Sánchez},     % author
    pdfkeywords={palabra1, palabra2, código1, etc.} % list of keywords
}

% Font change to Arial
\usepackage{helvet}
\renewcommand{\familydefault}{\sfdefault}

% Chapter titles in uppercase and larger font
\titleformat{\chapter}[hang]{\large\bfseries}{\thechapter.}{1em}{\MakeUppercase}
\titleformat{\section}[hang]{\bfseries}{\thesection.}{1em}{}
\titleformat{\subsection}[hang]{\bfseries}{\thesubsection.}{1em}{}

% Fancyhdr setup
\setlength{\headheight}{14.30174pt} % Adjust to recommended value, slightly larger for safety
\fancyhf{} % Clear all headers and footers
\fancyhead[LE]{\nouppercase{\leftmark}}
\fancyhead[RO]{Optimización energética para vivienda}
\fancyfoot[LE]{\thepage}
\fancyfoot[RE]{Escuela Técnica Superior de Ingenieros Industriales (UPM)}
\fancyfoot[LO]{Luis D. Aranda Sánchez}
\fancyfoot[RO]{\thepage}
\renewcommand{\headrulewidth}{0.4pt}
\renewcommand{\footrulewidth}{0.4pt}

\fancypagestyle{myfancy}{
    \fancyhf{} % Clear all headers and footers
    \fancyhead[LE]{\nouppercase{\leftmark}}
    \fancyhead[RO]{Optimización energética para vivienda}
    \fancyfoot[LE]{\thepage}
    \fancyfoot[RE]{Escuela Técnica Superior de Ingenieros Industriales (UPM)}
    \fancyfoot[LO]{Luis D. Aranda Sánchez}
    \fancyfoot[RO]{\thepage}
    \renewcommand{\headrulewidth}{0.4pt}
    \renewcommand{\footrulewidth}{0.4pt}
}

\fancypagestyle{simple}{
    \fancyhf{} % Clear all headers and footers
    \renewcommand{\headrulewidth}{0pt}
    \renewcommand{\footrulewidth}{0pt}
}

% Line spacing
\setstretch{1.2}

% Document starts here
\begin{document}

% Portada
\begin{titlepage}
    \centering
    {\scshape\LARGE Universidad Politécnica de Madrid \par}
    \vspace{1cm}
    {\scshape\Large Escuela Técnica Superior de Ingenieros Industriales\par}
    \vspace{1.5cm}
    {\huge\bfseries Optimización energética de sistema híbrido con bomba de calor, suelo radiante, fotovoltaica y almacenamiento para vivienda \par}
    \vspace{1.5cm}
    {\Large\bfseries Trabajo de Fin de Máster\par}
    \vspace{0.5cm}
    {\large Máster Universitario en Ingeniería de la Energía \par}
    \vspace{2cm}
    {\Large Luis D. Aranda Sánchez\par}
    \vfill
    Director: Javier Rodríguez Martín
    \vfill
    {\large Septiembre 6, 2024\par}
\end{titlepage}

% Resumen (máximo de 5 páginas, incluyendo al final Palabras clave)
\clearpage
\pagestyle{simple}
% \newpage
\chapter*{Resumen}
\addcontentsline{toc}{chapter}{Resumen}
\input{capitulos/resumen/main.tex}

% Índice (paginado)
\clearpage
\pagestyle{simple}
% \newpage
\tableofcontents

% Introducción (donde se incluya los antecedentes y justificación)
\clearpage
\pagestyle{myfancy}
\newpage
\chapter{Introducción}
\input{capitulos/introduccion/main.tex}

% Objetivos
\chapter{Objetivos}
\input{capitulos/objetivos/main.tex}

% Metodología
\chapter{Metodología}
\input{capitulos/metodologia/main.tex}

% Resultados y discusión (incluyendo la valoración de impactos y de aspectos de responsabilidad legal, ética y profesional relacionados con el trabajo)
\chapter{Resultados y Discusión}
\input{capitulos/resultados_discusion/main.tex}

% Conclusiones
\chapter{Conclusiones}
\input{capitulos/conclusiones/main.tex}

% Planificación temporal y presupuesto
\chapter{Planificación Temporal y Presupuesto}
\input{capitulos/planificacion_presupuesto/main.tex}

% Bibliografía
\newpage
\addcontentsline{toc}{chapter}{Bibliografía}
\printbibliography

\end{document}


% Índice (paginado)
\clearpage
\pagestyle{simple}
% \newpage
\tableofcontents

% Introducción (donde se incluya los antecedentes y justificación)
\clearpage
\pagestyle{myfancy}
\newpage
\chapter{Introducción}
\documentclass[a4paper,11pt,twoside]{report}
\usepackage[left=25mm,right=25mm,top=25mm,bottom=25mm,includehead,includefoot,headsep=15mm,footskip=15mm]{geometry}
\usepackage{graphicx}
\usepackage{fancyhdr}
\usepackage{titlesec}
\usepackage[spanish]{babel}
\usepackage[utf8]{inputenc}
\usepackage{amsmath}
\usepackage{setspace}
\usepackage{svg}
\usepackage{hyperref}
\usepackage[backend=biber,style=numeric]{biblatex}
\addbibresource{references.bib}
\hypersetup{
    colorlinks=true,
    linkcolor=blue,      % color of internal links (sections, etc.)
    urlcolor=blue,       % color of external links
    pdftitle={Optimización energética de sistema híbrido con bomba de calor, suelo radiante, fotovoltaica y almacenamiento para vivienda},    % title
    pdfauthor={Luis D. Aranda Sánchez},     % author
    pdfkeywords={palabra1, palabra2, código1, etc.} % list of keywords
}

% Font change to Arial
\usepackage{helvet}
\renewcommand{\familydefault}{\sfdefault}

% Chapter titles in uppercase and larger font
\titleformat{\chapter}[hang]{\large\bfseries}{\thechapter.}{1em}{\MakeUppercase}
\titleformat{\section}[hang]{\bfseries}{\thesection.}{1em}{}
\titleformat{\subsection}[hang]{\bfseries}{\thesubsection.}{1em}{}

% Fancyhdr setup
\setlength{\headheight}{14.30174pt} % Adjust to recommended value, slightly larger for safety
\fancyhf{} % Clear all headers and footers
\fancyhead[LE]{\nouppercase{\leftmark}}
\fancyhead[RO]{Optimización energética para vivienda}
\fancyfoot[LE]{\thepage}
\fancyfoot[RE]{Escuela Técnica Superior de Ingenieros Industriales (UPM)}
\fancyfoot[LO]{Luis D. Aranda Sánchez}
\fancyfoot[RO]{\thepage}
\renewcommand{\headrulewidth}{0.4pt}
\renewcommand{\footrulewidth}{0.4pt}

\fancypagestyle{myfancy}{
    \fancyhf{} % Clear all headers and footers
    \fancyhead[LE]{\nouppercase{\leftmark}}
    \fancyhead[RO]{Optimización energética para vivienda}
    \fancyfoot[LE]{\thepage}
    \fancyfoot[RE]{Escuela Técnica Superior de Ingenieros Industriales (UPM)}
    \fancyfoot[LO]{Luis D. Aranda Sánchez}
    \fancyfoot[RO]{\thepage}
    \renewcommand{\headrulewidth}{0.4pt}
    \renewcommand{\footrulewidth}{0.4pt}
}

\fancypagestyle{simple}{
    \fancyhf{} % Clear all headers and footers
    \renewcommand{\headrulewidth}{0pt}
    \renewcommand{\footrulewidth}{0pt}
}

% Line spacing
\setstretch{1.2}

% Document starts here
\begin{document}

% Portada
\begin{titlepage}
    \centering
    {\scshape\LARGE Universidad Politécnica de Madrid \par}
    \vspace{1cm}
    {\scshape\Large Escuela Técnica Superior de Ingenieros Industriales\par}
    \vspace{1.5cm}
    {\huge\bfseries Optimización energética de sistema híbrido con bomba de calor, suelo radiante, fotovoltaica y almacenamiento para vivienda \par}
    \vspace{1.5cm}
    {\Large\bfseries Trabajo de Fin de Máster\par}
    \vspace{0.5cm}
    {\large Máster Universitario en Ingeniería de la Energía \par}
    \vspace{2cm}
    {\Large Luis D. Aranda Sánchez\par}
    \vfill
    Director: Javier Rodríguez Martín
    \vfill
    {\large Septiembre 6, 2024\par}
\end{titlepage}

% Resumen (máximo de 5 páginas, incluyendo al final Palabras clave)
\clearpage
\pagestyle{simple}
% \newpage
\chapter*{Resumen}
\addcontentsline{toc}{chapter}{Resumen}
\input{capitulos/resumen/main.tex}

% Índice (paginado)
\clearpage
\pagestyle{simple}
% \newpage
\tableofcontents

% Introducción (donde se incluya los antecedentes y justificación)
\clearpage
\pagestyle{myfancy}
\newpage
\chapter{Introducción}
\input{capitulos/introduccion/main.tex}

% Objetivos
\chapter{Objetivos}
\input{capitulos/objetivos/main.tex}

% Metodología
\chapter{Metodología}
\input{capitulos/metodologia/main.tex}

% Resultados y discusión (incluyendo la valoración de impactos y de aspectos de responsabilidad legal, ética y profesional relacionados con el trabajo)
\chapter{Resultados y Discusión}
\input{capitulos/resultados_discusion/main.tex}

% Conclusiones
\chapter{Conclusiones}
\input{capitulos/conclusiones/main.tex}

% Planificación temporal y presupuesto
\chapter{Planificación Temporal y Presupuesto}
\input{capitulos/planificacion_presupuesto/main.tex}

% Bibliografía
\newpage
\addcontentsline{toc}{chapter}{Bibliografía}
\printbibliography

\end{document}


% Objetivos
\chapter{Objetivos}
\documentclass[a4paper,11pt,twoside]{report}
\usepackage[left=25mm,right=25mm,top=25mm,bottom=25mm,includehead,includefoot,headsep=15mm,footskip=15mm]{geometry}
\usepackage{graphicx}
\usepackage{fancyhdr}
\usepackage{titlesec}
\usepackage[spanish]{babel}
\usepackage[utf8]{inputenc}
\usepackage{amsmath}
\usepackage{setspace}
\usepackage{svg}
\usepackage{hyperref}
\usepackage[backend=biber,style=numeric]{biblatex}
\addbibresource{references.bib}
\hypersetup{
    colorlinks=true,
    linkcolor=blue,      % color of internal links (sections, etc.)
    urlcolor=blue,       % color of external links
    pdftitle={Optimización energética de sistema híbrido con bomba de calor, suelo radiante, fotovoltaica y almacenamiento para vivienda},    % title
    pdfauthor={Luis D. Aranda Sánchez},     % author
    pdfkeywords={palabra1, palabra2, código1, etc.} % list of keywords
}

% Font change to Arial
\usepackage{helvet}
\renewcommand{\familydefault}{\sfdefault}

% Chapter titles in uppercase and larger font
\titleformat{\chapter}[hang]{\large\bfseries}{\thechapter.}{1em}{\MakeUppercase}
\titleformat{\section}[hang]{\bfseries}{\thesection.}{1em}{}
\titleformat{\subsection}[hang]{\bfseries}{\thesubsection.}{1em}{}

% Fancyhdr setup
\setlength{\headheight}{14.30174pt} % Adjust to recommended value, slightly larger for safety
\fancyhf{} % Clear all headers and footers
\fancyhead[LE]{\nouppercase{\leftmark}}
\fancyhead[RO]{Optimización energética para vivienda}
\fancyfoot[LE]{\thepage}
\fancyfoot[RE]{Escuela Técnica Superior de Ingenieros Industriales (UPM)}
\fancyfoot[LO]{Luis D. Aranda Sánchez}
\fancyfoot[RO]{\thepage}
\renewcommand{\headrulewidth}{0.4pt}
\renewcommand{\footrulewidth}{0.4pt}

\fancypagestyle{myfancy}{
    \fancyhf{} % Clear all headers and footers
    \fancyhead[LE]{\nouppercase{\leftmark}}
    \fancyhead[RO]{Optimización energética para vivienda}
    \fancyfoot[LE]{\thepage}
    \fancyfoot[RE]{Escuela Técnica Superior de Ingenieros Industriales (UPM)}
    \fancyfoot[LO]{Luis D. Aranda Sánchez}
    \fancyfoot[RO]{\thepage}
    \renewcommand{\headrulewidth}{0.4pt}
    \renewcommand{\footrulewidth}{0.4pt}
}

\fancypagestyle{simple}{
    \fancyhf{} % Clear all headers and footers
    \renewcommand{\headrulewidth}{0pt}
    \renewcommand{\footrulewidth}{0pt}
}

% Line spacing
\setstretch{1.2}

% Document starts here
\begin{document}

% Portada
\begin{titlepage}
    \centering
    {\scshape\LARGE Universidad Politécnica de Madrid \par}
    \vspace{1cm}
    {\scshape\Large Escuela Técnica Superior de Ingenieros Industriales\par}
    \vspace{1.5cm}
    {\huge\bfseries Optimización energética de sistema híbrido con bomba de calor, suelo radiante, fotovoltaica y almacenamiento para vivienda \par}
    \vspace{1.5cm}
    {\Large\bfseries Trabajo de Fin de Máster\par}
    \vspace{0.5cm}
    {\large Máster Universitario en Ingeniería de la Energía \par}
    \vspace{2cm}
    {\Large Luis D. Aranda Sánchez\par}
    \vfill
    Director: Javier Rodríguez Martín
    \vfill
    {\large Septiembre 6, 2024\par}
\end{titlepage}

% Resumen (máximo de 5 páginas, incluyendo al final Palabras clave)
\clearpage
\pagestyle{simple}
% \newpage
\chapter*{Resumen}
\addcontentsline{toc}{chapter}{Resumen}
\input{capitulos/resumen/main.tex}

% Índice (paginado)
\clearpage
\pagestyle{simple}
% \newpage
\tableofcontents

% Introducción (donde se incluya los antecedentes y justificación)
\clearpage
\pagestyle{myfancy}
\newpage
\chapter{Introducción}
\input{capitulos/introduccion/main.tex}

% Objetivos
\chapter{Objetivos}
\input{capitulos/objetivos/main.tex}

% Metodología
\chapter{Metodología}
\input{capitulos/metodologia/main.tex}

% Resultados y discusión (incluyendo la valoración de impactos y de aspectos de responsabilidad legal, ética y profesional relacionados con el trabajo)
\chapter{Resultados y Discusión}
\input{capitulos/resultados_discusion/main.tex}

% Conclusiones
\chapter{Conclusiones}
\input{capitulos/conclusiones/main.tex}

% Planificación temporal y presupuesto
\chapter{Planificación Temporal y Presupuesto}
\input{capitulos/planificacion_presupuesto/main.tex}

% Bibliografía
\newpage
\addcontentsline{toc}{chapter}{Bibliografía}
\printbibliography

\end{document}


% Metodología
\chapter{Metodología}
\documentclass[a4paper,11pt,twoside]{report}
\usepackage[left=25mm,right=25mm,top=25mm,bottom=25mm,includehead,includefoot,headsep=15mm,footskip=15mm]{geometry}
\usepackage{graphicx}
\usepackage{fancyhdr}
\usepackage{titlesec}
\usepackage[spanish]{babel}
\usepackage[utf8]{inputenc}
\usepackage{amsmath}
\usepackage{setspace}
\usepackage{svg}
\usepackage{hyperref}
\usepackage[backend=biber,style=numeric]{biblatex}
\addbibresource{references.bib}
\hypersetup{
    colorlinks=true,
    linkcolor=blue,      % color of internal links (sections, etc.)
    urlcolor=blue,       % color of external links
    pdftitle={Optimización energética de sistema híbrido con bomba de calor, suelo radiante, fotovoltaica y almacenamiento para vivienda},    % title
    pdfauthor={Luis D. Aranda Sánchez},     % author
    pdfkeywords={palabra1, palabra2, código1, etc.} % list of keywords
}

% Font change to Arial
\usepackage{helvet}
\renewcommand{\familydefault}{\sfdefault}

% Chapter titles in uppercase and larger font
\titleformat{\chapter}[hang]{\large\bfseries}{\thechapter.}{1em}{\MakeUppercase}
\titleformat{\section}[hang]{\bfseries}{\thesection.}{1em}{}
\titleformat{\subsection}[hang]{\bfseries}{\thesubsection.}{1em}{}

% Fancyhdr setup
\setlength{\headheight}{14.30174pt} % Adjust to recommended value, slightly larger for safety
\fancyhf{} % Clear all headers and footers
\fancyhead[LE]{\nouppercase{\leftmark}}
\fancyhead[RO]{Optimización energética para vivienda}
\fancyfoot[LE]{\thepage}
\fancyfoot[RE]{Escuela Técnica Superior de Ingenieros Industriales (UPM)}
\fancyfoot[LO]{Luis D. Aranda Sánchez}
\fancyfoot[RO]{\thepage}
\renewcommand{\headrulewidth}{0.4pt}
\renewcommand{\footrulewidth}{0.4pt}

\fancypagestyle{myfancy}{
    \fancyhf{} % Clear all headers and footers
    \fancyhead[LE]{\nouppercase{\leftmark}}
    \fancyhead[RO]{Optimización energética para vivienda}
    \fancyfoot[LE]{\thepage}
    \fancyfoot[RE]{Escuela Técnica Superior de Ingenieros Industriales (UPM)}
    \fancyfoot[LO]{Luis D. Aranda Sánchez}
    \fancyfoot[RO]{\thepage}
    \renewcommand{\headrulewidth}{0.4pt}
    \renewcommand{\footrulewidth}{0.4pt}
}

\fancypagestyle{simple}{
    \fancyhf{} % Clear all headers and footers
    \renewcommand{\headrulewidth}{0pt}
    \renewcommand{\footrulewidth}{0pt}
}

% Line spacing
\setstretch{1.2}

% Document starts here
\begin{document}

% Portada
\begin{titlepage}
    \centering
    {\scshape\LARGE Universidad Politécnica de Madrid \par}
    \vspace{1cm}
    {\scshape\Large Escuela Técnica Superior de Ingenieros Industriales\par}
    \vspace{1.5cm}
    {\huge\bfseries Optimización energética de sistema híbrido con bomba de calor, suelo radiante, fotovoltaica y almacenamiento para vivienda \par}
    \vspace{1.5cm}
    {\Large\bfseries Trabajo de Fin de Máster\par}
    \vspace{0.5cm}
    {\large Máster Universitario en Ingeniería de la Energía \par}
    \vspace{2cm}
    {\Large Luis D. Aranda Sánchez\par}
    \vfill
    Director: Javier Rodríguez Martín
    \vfill
    {\large Septiembre 6, 2024\par}
\end{titlepage}

% Resumen (máximo de 5 páginas, incluyendo al final Palabras clave)
\clearpage
\pagestyle{simple}
% \newpage
\chapter*{Resumen}
\addcontentsline{toc}{chapter}{Resumen}
\input{capitulos/resumen/main.tex}

% Índice (paginado)
\clearpage
\pagestyle{simple}
% \newpage
\tableofcontents

% Introducción (donde se incluya los antecedentes y justificación)
\clearpage
\pagestyle{myfancy}
\newpage
\chapter{Introducción}
\input{capitulos/introduccion/main.tex}

% Objetivos
\chapter{Objetivos}
\input{capitulos/objetivos/main.tex}

% Metodología
\chapter{Metodología}
\input{capitulos/metodologia/main.tex}

% Resultados y discusión (incluyendo la valoración de impactos y de aspectos de responsabilidad legal, ética y profesional relacionados con el trabajo)
\chapter{Resultados y Discusión}
\input{capitulos/resultados_discusion/main.tex}

% Conclusiones
\chapter{Conclusiones}
\input{capitulos/conclusiones/main.tex}

% Planificación temporal y presupuesto
\chapter{Planificación Temporal y Presupuesto}
\input{capitulos/planificacion_presupuesto/main.tex}

% Bibliografía
\newpage
\addcontentsline{toc}{chapter}{Bibliografía}
\printbibliography

\end{document}


% Resultados y discusión (incluyendo la valoración de impactos y de aspectos de responsabilidad legal, ética y profesional relacionados con el trabajo)
\chapter{Resultados y Discusión}
\documentclass[a4paper,11pt,twoside]{report}
\usepackage[left=25mm,right=25mm,top=25mm,bottom=25mm,includehead,includefoot,headsep=15mm,footskip=15mm]{geometry}
\usepackage{graphicx}
\usepackage{fancyhdr}
\usepackage{titlesec}
\usepackage[spanish]{babel}
\usepackage[utf8]{inputenc}
\usepackage{amsmath}
\usepackage{setspace}
\usepackage{svg}
\usepackage{hyperref}
\usepackage[backend=biber,style=numeric]{biblatex}
\addbibresource{references.bib}
\hypersetup{
    colorlinks=true,
    linkcolor=blue,      % color of internal links (sections, etc.)
    urlcolor=blue,       % color of external links
    pdftitle={Optimización energética de sistema híbrido con bomba de calor, suelo radiante, fotovoltaica y almacenamiento para vivienda},    % title
    pdfauthor={Luis D. Aranda Sánchez},     % author
    pdfkeywords={palabra1, palabra2, código1, etc.} % list of keywords
}

% Font change to Arial
\usepackage{helvet}
\renewcommand{\familydefault}{\sfdefault}

% Chapter titles in uppercase and larger font
\titleformat{\chapter}[hang]{\large\bfseries}{\thechapter.}{1em}{\MakeUppercase}
\titleformat{\section}[hang]{\bfseries}{\thesection.}{1em}{}
\titleformat{\subsection}[hang]{\bfseries}{\thesubsection.}{1em}{}

% Fancyhdr setup
\setlength{\headheight}{14.30174pt} % Adjust to recommended value, slightly larger for safety
\fancyhf{} % Clear all headers and footers
\fancyhead[LE]{\nouppercase{\leftmark}}
\fancyhead[RO]{Optimización energética para vivienda}
\fancyfoot[LE]{\thepage}
\fancyfoot[RE]{Escuela Técnica Superior de Ingenieros Industriales (UPM)}
\fancyfoot[LO]{Luis D. Aranda Sánchez}
\fancyfoot[RO]{\thepage}
\renewcommand{\headrulewidth}{0.4pt}
\renewcommand{\footrulewidth}{0.4pt}

\fancypagestyle{myfancy}{
    \fancyhf{} % Clear all headers and footers
    \fancyhead[LE]{\nouppercase{\leftmark}}
    \fancyhead[RO]{Optimización energética para vivienda}
    \fancyfoot[LE]{\thepage}
    \fancyfoot[RE]{Escuela Técnica Superior de Ingenieros Industriales (UPM)}
    \fancyfoot[LO]{Luis D. Aranda Sánchez}
    \fancyfoot[RO]{\thepage}
    \renewcommand{\headrulewidth}{0.4pt}
    \renewcommand{\footrulewidth}{0.4pt}
}

\fancypagestyle{simple}{
    \fancyhf{} % Clear all headers and footers
    \renewcommand{\headrulewidth}{0pt}
    \renewcommand{\footrulewidth}{0pt}
}

% Line spacing
\setstretch{1.2}

% Document starts here
\begin{document}

% Portada
\begin{titlepage}
    \centering
    {\scshape\LARGE Universidad Politécnica de Madrid \par}
    \vspace{1cm}
    {\scshape\Large Escuela Técnica Superior de Ingenieros Industriales\par}
    \vspace{1.5cm}
    {\huge\bfseries Optimización energética de sistema híbrido con bomba de calor, suelo radiante, fotovoltaica y almacenamiento para vivienda \par}
    \vspace{1.5cm}
    {\Large\bfseries Trabajo de Fin de Máster\par}
    \vspace{0.5cm}
    {\large Máster Universitario en Ingeniería de la Energía \par}
    \vspace{2cm}
    {\Large Luis D. Aranda Sánchez\par}
    \vfill
    Director: Javier Rodríguez Martín
    \vfill
    {\large Septiembre 6, 2024\par}
\end{titlepage}

% Resumen (máximo de 5 páginas, incluyendo al final Palabras clave)
\clearpage
\pagestyle{simple}
% \newpage
\chapter*{Resumen}
\addcontentsline{toc}{chapter}{Resumen}
\input{capitulos/resumen/main.tex}

% Índice (paginado)
\clearpage
\pagestyle{simple}
% \newpage
\tableofcontents

% Introducción (donde se incluya los antecedentes y justificación)
\clearpage
\pagestyle{myfancy}
\newpage
\chapter{Introducción}
\input{capitulos/introduccion/main.tex}

% Objetivos
\chapter{Objetivos}
\input{capitulos/objetivos/main.tex}

% Metodología
\chapter{Metodología}
\input{capitulos/metodologia/main.tex}

% Resultados y discusión (incluyendo la valoración de impactos y de aspectos de responsabilidad legal, ética y profesional relacionados con el trabajo)
\chapter{Resultados y Discusión}
\input{capitulos/resultados_discusion/main.tex}

% Conclusiones
\chapter{Conclusiones}
\input{capitulos/conclusiones/main.tex}

% Planificación temporal y presupuesto
\chapter{Planificación Temporal y Presupuesto}
\input{capitulos/planificacion_presupuesto/main.tex}

% Bibliografía
\newpage
\addcontentsline{toc}{chapter}{Bibliografía}
\printbibliography

\end{document}


% Conclusiones
\chapter{Conclusiones}
\documentclass[a4paper,11pt,twoside]{report}
\usepackage[left=25mm,right=25mm,top=25mm,bottom=25mm,includehead,includefoot,headsep=15mm,footskip=15mm]{geometry}
\usepackage{graphicx}
\usepackage{fancyhdr}
\usepackage{titlesec}
\usepackage[spanish]{babel}
\usepackage[utf8]{inputenc}
\usepackage{amsmath}
\usepackage{setspace}
\usepackage{svg}
\usepackage{hyperref}
\usepackage[backend=biber,style=numeric]{biblatex}
\addbibresource{references.bib}
\hypersetup{
    colorlinks=true,
    linkcolor=blue,      % color of internal links (sections, etc.)
    urlcolor=blue,       % color of external links
    pdftitle={Optimización energética de sistema híbrido con bomba de calor, suelo radiante, fotovoltaica y almacenamiento para vivienda},    % title
    pdfauthor={Luis D. Aranda Sánchez},     % author
    pdfkeywords={palabra1, palabra2, código1, etc.} % list of keywords
}

% Font change to Arial
\usepackage{helvet}
\renewcommand{\familydefault}{\sfdefault}

% Chapter titles in uppercase and larger font
\titleformat{\chapter}[hang]{\large\bfseries}{\thechapter.}{1em}{\MakeUppercase}
\titleformat{\section}[hang]{\bfseries}{\thesection.}{1em}{}
\titleformat{\subsection}[hang]{\bfseries}{\thesubsection.}{1em}{}

% Fancyhdr setup
\setlength{\headheight}{14.30174pt} % Adjust to recommended value, slightly larger for safety
\fancyhf{} % Clear all headers and footers
\fancyhead[LE]{\nouppercase{\leftmark}}
\fancyhead[RO]{Optimización energética para vivienda}
\fancyfoot[LE]{\thepage}
\fancyfoot[RE]{Escuela Técnica Superior de Ingenieros Industriales (UPM)}
\fancyfoot[LO]{Luis D. Aranda Sánchez}
\fancyfoot[RO]{\thepage}
\renewcommand{\headrulewidth}{0.4pt}
\renewcommand{\footrulewidth}{0.4pt}

\fancypagestyle{myfancy}{
    \fancyhf{} % Clear all headers and footers
    \fancyhead[LE]{\nouppercase{\leftmark}}
    \fancyhead[RO]{Optimización energética para vivienda}
    \fancyfoot[LE]{\thepage}
    \fancyfoot[RE]{Escuela Técnica Superior de Ingenieros Industriales (UPM)}
    \fancyfoot[LO]{Luis D. Aranda Sánchez}
    \fancyfoot[RO]{\thepage}
    \renewcommand{\headrulewidth}{0.4pt}
    \renewcommand{\footrulewidth}{0.4pt}
}

\fancypagestyle{simple}{
    \fancyhf{} % Clear all headers and footers
    \renewcommand{\headrulewidth}{0pt}
    \renewcommand{\footrulewidth}{0pt}
}

% Line spacing
\setstretch{1.2}

% Document starts here
\begin{document}

% Portada
\begin{titlepage}
    \centering
    {\scshape\LARGE Universidad Politécnica de Madrid \par}
    \vspace{1cm}
    {\scshape\Large Escuela Técnica Superior de Ingenieros Industriales\par}
    \vspace{1.5cm}
    {\huge\bfseries Optimización energética de sistema híbrido con bomba de calor, suelo radiante, fotovoltaica y almacenamiento para vivienda \par}
    \vspace{1.5cm}
    {\Large\bfseries Trabajo de Fin de Máster\par}
    \vspace{0.5cm}
    {\large Máster Universitario en Ingeniería de la Energía \par}
    \vspace{2cm}
    {\Large Luis D. Aranda Sánchez\par}
    \vfill
    Director: Javier Rodríguez Martín
    \vfill
    {\large Septiembre 6, 2024\par}
\end{titlepage}

% Resumen (máximo de 5 páginas, incluyendo al final Palabras clave)
\clearpage
\pagestyle{simple}
% \newpage
\chapter*{Resumen}
\addcontentsline{toc}{chapter}{Resumen}
\input{capitulos/resumen/main.tex}

% Índice (paginado)
\clearpage
\pagestyle{simple}
% \newpage
\tableofcontents

% Introducción (donde se incluya los antecedentes y justificación)
\clearpage
\pagestyle{myfancy}
\newpage
\chapter{Introducción}
\input{capitulos/introduccion/main.tex}

% Objetivos
\chapter{Objetivos}
\input{capitulos/objetivos/main.tex}

% Metodología
\chapter{Metodología}
\input{capitulos/metodologia/main.tex}

% Resultados y discusión (incluyendo la valoración de impactos y de aspectos de responsabilidad legal, ética y profesional relacionados con el trabajo)
\chapter{Resultados y Discusión}
\input{capitulos/resultados_discusion/main.tex}

% Conclusiones
\chapter{Conclusiones}
\input{capitulos/conclusiones/main.tex}

% Planificación temporal y presupuesto
\chapter{Planificación Temporal y Presupuesto}
\input{capitulos/planificacion_presupuesto/main.tex}

% Bibliografía
\newpage
\addcontentsline{toc}{chapter}{Bibliografía}
\printbibliography

\end{document}


% Planificación temporal y presupuesto
\chapter{Planificación Temporal y Presupuesto}
\documentclass[a4paper,11pt,twoside]{report}
\usepackage[left=25mm,right=25mm,top=25mm,bottom=25mm,includehead,includefoot,headsep=15mm,footskip=15mm]{geometry}
\usepackage{graphicx}
\usepackage{fancyhdr}
\usepackage{titlesec}
\usepackage[spanish]{babel}
\usepackage[utf8]{inputenc}
\usepackage{amsmath}
\usepackage{setspace}
\usepackage{svg}
\usepackage{hyperref}
\usepackage[backend=biber,style=numeric]{biblatex}
\addbibresource{references.bib}
\hypersetup{
    colorlinks=true,
    linkcolor=blue,      % color of internal links (sections, etc.)
    urlcolor=blue,       % color of external links
    pdftitle={Optimización energética de sistema híbrido con bomba de calor, suelo radiante, fotovoltaica y almacenamiento para vivienda},    % title
    pdfauthor={Luis D. Aranda Sánchez},     % author
    pdfkeywords={palabra1, palabra2, código1, etc.} % list of keywords
}

% Font change to Arial
\usepackage{helvet}
\renewcommand{\familydefault}{\sfdefault}

% Chapter titles in uppercase and larger font
\titleformat{\chapter}[hang]{\large\bfseries}{\thechapter.}{1em}{\MakeUppercase}
\titleformat{\section}[hang]{\bfseries}{\thesection.}{1em}{}
\titleformat{\subsection}[hang]{\bfseries}{\thesubsection.}{1em}{}

% Fancyhdr setup
\setlength{\headheight}{14.30174pt} % Adjust to recommended value, slightly larger for safety
\fancyhf{} % Clear all headers and footers
\fancyhead[LE]{\nouppercase{\leftmark}}
\fancyhead[RO]{Optimización energética para vivienda}
\fancyfoot[LE]{\thepage}
\fancyfoot[RE]{Escuela Técnica Superior de Ingenieros Industriales (UPM)}
\fancyfoot[LO]{Luis D. Aranda Sánchez}
\fancyfoot[RO]{\thepage}
\renewcommand{\headrulewidth}{0.4pt}
\renewcommand{\footrulewidth}{0.4pt}

\fancypagestyle{myfancy}{
    \fancyhf{} % Clear all headers and footers
    \fancyhead[LE]{\nouppercase{\leftmark}}
    \fancyhead[RO]{Optimización energética para vivienda}
    \fancyfoot[LE]{\thepage}
    \fancyfoot[RE]{Escuela Técnica Superior de Ingenieros Industriales (UPM)}
    \fancyfoot[LO]{Luis D. Aranda Sánchez}
    \fancyfoot[RO]{\thepage}
    \renewcommand{\headrulewidth}{0.4pt}
    \renewcommand{\footrulewidth}{0.4pt}
}

\fancypagestyle{simple}{
    \fancyhf{} % Clear all headers and footers
    \renewcommand{\headrulewidth}{0pt}
    \renewcommand{\footrulewidth}{0pt}
}

% Line spacing
\setstretch{1.2}

% Document starts here
\begin{document}

% Portada
\begin{titlepage}
    \centering
    {\scshape\LARGE Universidad Politécnica de Madrid \par}
    \vspace{1cm}
    {\scshape\Large Escuela Técnica Superior de Ingenieros Industriales\par}
    \vspace{1.5cm}
    {\huge\bfseries Optimización energética de sistema híbrido con bomba de calor, suelo radiante, fotovoltaica y almacenamiento para vivienda \par}
    \vspace{1.5cm}
    {\Large\bfseries Trabajo de Fin de Máster\par}
    \vspace{0.5cm}
    {\large Máster Universitario en Ingeniería de la Energía \par}
    \vspace{2cm}
    {\Large Luis D. Aranda Sánchez\par}
    \vfill
    Director: Javier Rodríguez Martín
    \vfill
    {\large Septiembre 6, 2024\par}
\end{titlepage}

% Resumen (máximo de 5 páginas, incluyendo al final Palabras clave)
\clearpage
\pagestyle{simple}
% \newpage
\chapter*{Resumen}
\addcontentsline{toc}{chapter}{Resumen}
\input{capitulos/resumen/main.tex}

% Índice (paginado)
\clearpage
\pagestyle{simple}
% \newpage
\tableofcontents

% Introducción (donde se incluya los antecedentes y justificación)
\clearpage
\pagestyle{myfancy}
\newpage
\chapter{Introducción}
\input{capitulos/introduccion/main.tex}

% Objetivos
\chapter{Objetivos}
\input{capitulos/objetivos/main.tex}

% Metodología
\chapter{Metodología}
\input{capitulos/metodologia/main.tex}

% Resultados y discusión (incluyendo la valoración de impactos y de aspectos de responsabilidad legal, ética y profesional relacionados con el trabajo)
\chapter{Resultados y Discusión}
\input{capitulos/resultados_discusion/main.tex}

% Conclusiones
\chapter{Conclusiones}
\input{capitulos/conclusiones/main.tex}

% Planificación temporal y presupuesto
\chapter{Planificación Temporal y Presupuesto}
\input{capitulos/planificacion_presupuesto/main.tex}

% Bibliografía
\newpage
\addcontentsline{toc}{chapter}{Bibliografía}
\printbibliography

\end{document}


% Bibliografía
\newpage
\addcontentsline{toc}{chapter}{Bibliografía}
\printbibliography

\end{document}


% Metodología
\chapter{Metodología}
\documentclass[a4paper,11pt,twoside]{report}
\usepackage[left=25mm,right=25mm,top=25mm,bottom=25mm,includehead,includefoot,headsep=15mm,footskip=15mm]{geometry}
\usepackage{graphicx}
\usepackage{fancyhdr}
\usepackage{titlesec}
\usepackage[spanish]{babel}
\usepackage[utf8]{inputenc}
\usepackage{amsmath}
\usepackage{setspace}
\usepackage{svg}
\usepackage{hyperref}
\usepackage[backend=biber,style=numeric]{biblatex}
\addbibresource{references.bib}
\hypersetup{
    colorlinks=true,
    linkcolor=blue,      % color of internal links (sections, etc.)
    urlcolor=blue,       % color of external links
    pdftitle={Optimización energética de sistema híbrido con bomba de calor, suelo radiante, fotovoltaica y almacenamiento para vivienda},    % title
    pdfauthor={Luis D. Aranda Sánchez},     % author
    pdfkeywords={palabra1, palabra2, código1, etc.} % list of keywords
}

% Font change to Arial
\usepackage{helvet}
\renewcommand{\familydefault}{\sfdefault}

% Chapter titles in uppercase and larger font
\titleformat{\chapter}[hang]{\large\bfseries}{\thechapter.}{1em}{\MakeUppercase}
\titleformat{\section}[hang]{\bfseries}{\thesection.}{1em}{}
\titleformat{\subsection}[hang]{\bfseries}{\thesubsection.}{1em}{}

% Fancyhdr setup
\setlength{\headheight}{14.30174pt} % Adjust to recommended value, slightly larger for safety
\fancyhf{} % Clear all headers and footers
\fancyhead[LE]{\nouppercase{\leftmark}}
\fancyhead[RO]{Optimización energética para vivienda}
\fancyfoot[LE]{\thepage}
\fancyfoot[RE]{Escuela Técnica Superior de Ingenieros Industriales (UPM)}
\fancyfoot[LO]{Luis D. Aranda Sánchez}
\fancyfoot[RO]{\thepage}
\renewcommand{\headrulewidth}{0.4pt}
\renewcommand{\footrulewidth}{0.4pt}

\fancypagestyle{myfancy}{
    \fancyhf{} % Clear all headers and footers
    \fancyhead[LE]{\nouppercase{\leftmark}}
    \fancyhead[RO]{Optimización energética para vivienda}
    \fancyfoot[LE]{\thepage}
    \fancyfoot[RE]{Escuela Técnica Superior de Ingenieros Industriales (UPM)}
    \fancyfoot[LO]{Luis D. Aranda Sánchez}
    \fancyfoot[RO]{\thepage}
    \renewcommand{\headrulewidth}{0.4pt}
    \renewcommand{\footrulewidth}{0.4pt}
}

\fancypagestyle{simple}{
    \fancyhf{} % Clear all headers and footers
    \renewcommand{\headrulewidth}{0pt}
    \renewcommand{\footrulewidth}{0pt}
}

% Line spacing
\setstretch{1.2}

% Document starts here
\begin{document}

% Portada
\begin{titlepage}
    \centering
    {\scshape\LARGE Universidad Politécnica de Madrid \par}
    \vspace{1cm}
    {\scshape\Large Escuela Técnica Superior de Ingenieros Industriales\par}
    \vspace{1.5cm}
    {\huge\bfseries Optimización energética de sistema híbrido con bomba de calor, suelo radiante, fotovoltaica y almacenamiento para vivienda \par}
    \vspace{1.5cm}
    {\Large\bfseries Trabajo de Fin de Máster\par}
    \vspace{0.5cm}
    {\large Máster Universitario en Ingeniería de la Energía \par}
    \vspace{2cm}
    {\Large Luis D. Aranda Sánchez\par}
    \vfill
    Director: Javier Rodríguez Martín
    \vfill
    {\large Septiembre 6, 2024\par}
\end{titlepage}

% Resumen (máximo de 5 páginas, incluyendo al final Palabras clave)
\clearpage
\pagestyle{simple}
% \newpage
\chapter*{Resumen}
\addcontentsline{toc}{chapter}{Resumen}
\documentclass[a4paper,11pt,twoside]{report}
\usepackage[left=25mm,right=25mm,top=25mm,bottom=25mm,includehead,includefoot,headsep=15mm,footskip=15mm]{geometry}
\usepackage{graphicx}
\usepackage{fancyhdr}
\usepackage{titlesec}
\usepackage[spanish]{babel}
\usepackage[utf8]{inputenc}
\usepackage{amsmath}
\usepackage{setspace}
\usepackage{svg}
\usepackage{hyperref}
\usepackage[backend=biber,style=numeric]{biblatex}
\addbibresource{references.bib}
\hypersetup{
    colorlinks=true,
    linkcolor=blue,      % color of internal links (sections, etc.)
    urlcolor=blue,       % color of external links
    pdftitle={Optimización energética de sistema híbrido con bomba de calor, suelo radiante, fotovoltaica y almacenamiento para vivienda},    % title
    pdfauthor={Luis D. Aranda Sánchez},     % author
    pdfkeywords={palabra1, palabra2, código1, etc.} % list of keywords
}

% Font change to Arial
\usepackage{helvet}
\renewcommand{\familydefault}{\sfdefault}

% Chapter titles in uppercase and larger font
\titleformat{\chapter}[hang]{\large\bfseries}{\thechapter.}{1em}{\MakeUppercase}
\titleformat{\section}[hang]{\bfseries}{\thesection.}{1em}{}
\titleformat{\subsection}[hang]{\bfseries}{\thesubsection.}{1em}{}

% Fancyhdr setup
\setlength{\headheight}{14.30174pt} % Adjust to recommended value, slightly larger for safety
\fancyhf{} % Clear all headers and footers
\fancyhead[LE]{\nouppercase{\leftmark}}
\fancyhead[RO]{Optimización energética para vivienda}
\fancyfoot[LE]{\thepage}
\fancyfoot[RE]{Escuela Técnica Superior de Ingenieros Industriales (UPM)}
\fancyfoot[LO]{Luis D. Aranda Sánchez}
\fancyfoot[RO]{\thepage}
\renewcommand{\headrulewidth}{0.4pt}
\renewcommand{\footrulewidth}{0.4pt}

\fancypagestyle{myfancy}{
    \fancyhf{} % Clear all headers and footers
    \fancyhead[LE]{\nouppercase{\leftmark}}
    \fancyhead[RO]{Optimización energética para vivienda}
    \fancyfoot[LE]{\thepage}
    \fancyfoot[RE]{Escuela Técnica Superior de Ingenieros Industriales (UPM)}
    \fancyfoot[LO]{Luis D. Aranda Sánchez}
    \fancyfoot[RO]{\thepage}
    \renewcommand{\headrulewidth}{0.4pt}
    \renewcommand{\footrulewidth}{0.4pt}
}

\fancypagestyle{simple}{
    \fancyhf{} % Clear all headers and footers
    \renewcommand{\headrulewidth}{0pt}
    \renewcommand{\footrulewidth}{0pt}
}

% Line spacing
\setstretch{1.2}

% Document starts here
\begin{document}

% Portada
\begin{titlepage}
    \centering
    {\scshape\LARGE Universidad Politécnica de Madrid \par}
    \vspace{1cm}
    {\scshape\Large Escuela Técnica Superior de Ingenieros Industriales\par}
    \vspace{1.5cm}
    {\huge\bfseries Optimización energética de sistema híbrido con bomba de calor, suelo radiante, fotovoltaica y almacenamiento para vivienda \par}
    \vspace{1.5cm}
    {\Large\bfseries Trabajo de Fin de Máster\par}
    \vspace{0.5cm}
    {\large Máster Universitario en Ingeniería de la Energía \par}
    \vspace{2cm}
    {\Large Luis D. Aranda Sánchez\par}
    \vfill
    Director: Javier Rodríguez Martín
    \vfill
    {\large Septiembre 6, 2024\par}
\end{titlepage}

% Resumen (máximo de 5 páginas, incluyendo al final Palabras clave)
\clearpage
\pagestyle{simple}
% \newpage
\chapter*{Resumen}
\addcontentsline{toc}{chapter}{Resumen}
\input{capitulos/resumen/main.tex}

% Índice (paginado)
\clearpage
\pagestyle{simple}
% \newpage
\tableofcontents

% Introducción (donde se incluya los antecedentes y justificación)
\clearpage
\pagestyle{myfancy}
\newpage
\chapter{Introducción}
\input{capitulos/introduccion/main.tex}

% Objetivos
\chapter{Objetivos}
\input{capitulos/objetivos/main.tex}

% Metodología
\chapter{Metodología}
\input{capitulos/metodologia/main.tex}

% Resultados y discusión (incluyendo la valoración de impactos y de aspectos de responsabilidad legal, ética y profesional relacionados con el trabajo)
\chapter{Resultados y Discusión}
\input{capitulos/resultados_discusion/main.tex}

% Conclusiones
\chapter{Conclusiones}
\input{capitulos/conclusiones/main.tex}

% Planificación temporal y presupuesto
\chapter{Planificación Temporal y Presupuesto}
\input{capitulos/planificacion_presupuesto/main.tex}

% Bibliografía
\newpage
\addcontentsline{toc}{chapter}{Bibliografía}
\printbibliography

\end{document}


% Índice (paginado)
\clearpage
\pagestyle{simple}
% \newpage
\tableofcontents

% Introducción (donde se incluya los antecedentes y justificación)
\clearpage
\pagestyle{myfancy}
\newpage
\chapter{Introducción}
\documentclass[a4paper,11pt,twoside]{report}
\usepackage[left=25mm,right=25mm,top=25mm,bottom=25mm,includehead,includefoot,headsep=15mm,footskip=15mm]{geometry}
\usepackage{graphicx}
\usepackage{fancyhdr}
\usepackage{titlesec}
\usepackage[spanish]{babel}
\usepackage[utf8]{inputenc}
\usepackage{amsmath}
\usepackage{setspace}
\usepackage{svg}
\usepackage{hyperref}
\usepackage[backend=biber,style=numeric]{biblatex}
\addbibresource{references.bib}
\hypersetup{
    colorlinks=true,
    linkcolor=blue,      % color of internal links (sections, etc.)
    urlcolor=blue,       % color of external links
    pdftitle={Optimización energética de sistema híbrido con bomba de calor, suelo radiante, fotovoltaica y almacenamiento para vivienda},    % title
    pdfauthor={Luis D. Aranda Sánchez},     % author
    pdfkeywords={palabra1, palabra2, código1, etc.} % list of keywords
}

% Font change to Arial
\usepackage{helvet}
\renewcommand{\familydefault}{\sfdefault}

% Chapter titles in uppercase and larger font
\titleformat{\chapter}[hang]{\large\bfseries}{\thechapter.}{1em}{\MakeUppercase}
\titleformat{\section}[hang]{\bfseries}{\thesection.}{1em}{}
\titleformat{\subsection}[hang]{\bfseries}{\thesubsection.}{1em}{}

% Fancyhdr setup
\setlength{\headheight}{14.30174pt} % Adjust to recommended value, slightly larger for safety
\fancyhf{} % Clear all headers and footers
\fancyhead[LE]{\nouppercase{\leftmark}}
\fancyhead[RO]{Optimización energética para vivienda}
\fancyfoot[LE]{\thepage}
\fancyfoot[RE]{Escuela Técnica Superior de Ingenieros Industriales (UPM)}
\fancyfoot[LO]{Luis D. Aranda Sánchez}
\fancyfoot[RO]{\thepage}
\renewcommand{\headrulewidth}{0.4pt}
\renewcommand{\footrulewidth}{0.4pt}

\fancypagestyle{myfancy}{
    \fancyhf{} % Clear all headers and footers
    \fancyhead[LE]{\nouppercase{\leftmark}}
    \fancyhead[RO]{Optimización energética para vivienda}
    \fancyfoot[LE]{\thepage}
    \fancyfoot[RE]{Escuela Técnica Superior de Ingenieros Industriales (UPM)}
    \fancyfoot[LO]{Luis D. Aranda Sánchez}
    \fancyfoot[RO]{\thepage}
    \renewcommand{\headrulewidth}{0.4pt}
    \renewcommand{\footrulewidth}{0.4pt}
}

\fancypagestyle{simple}{
    \fancyhf{} % Clear all headers and footers
    \renewcommand{\headrulewidth}{0pt}
    \renewcommand{\footrulewidth}{0pt}
}

% Line spacing
\setstretch{1.2}

% Document starts here
\begin{document}

% Portada
\begin{titlepage}
    \centering
    {\scshape\LARGE Universidad Politécnica de Madrid \par}
    \vspace{1cm}
    {\scshape\Large Escuela Técnica Superior de Ingenieros Industriales\par}
    \vspace{1.5cm}
    {\huge\bfseries Optimización energética de sistema híbrido con bomba de calor, suelo radiante, fotovoltaica y almacenamiento para vivienda \par}
    \vspace{1.5cm}
    {\Large\bfseries Trabajo de Fin de Máster\par}
    \vspace{0.5cm}
    {\large Máster Universitario en Ingeniería de la Energía \par}
    \vspace{2cm}
    {\Large Luis D. Aranda Sánchez\par}
    \vfill
    Director: Javier Rodríguez Martín
    \vfill
    {\large Septiembre 6, 2024\par}
\end{titlepage}

% Resumen (máximo de 5 páginas, incluyendo al final Palabras clave)
\clearpage
\pagestyle{simple}
% \newpage
\chapter*{Resumen}
\addcontentsline{toc}{chapter}{Resumen}
\input{capitulos/resumen/main.tex}

% Índice (paginado)
\clearpage
\pagestyle{simple}
% \newpage
\tableofcontents

% Introducción (donde se incluya los antecedentes y justificación)
\clearpage
\pagestyle{myfancy}
\newpage
\chapter{Introducción}
\input{capitulos/introduccion/main.tex}

% Objetivos
\chapter{Objetivos}
\input{capitulos/objetivos/main.tex}

% Metodología
\chapter{Metodología}
\input{capitulos/metodologia/main.tex}

% Resultados y discusión (incluyendo la valoración de impactos y de aspectos de responsabilidad legal, ética y profesional relacionados con el trabajo)
\chapter{Resultados y Discusión}
\input{capitulos/resultados_discusion/main.tex}

% Conclusiones
\chapter{Conclusiones}
\input{capitulos/conclusiones/main.tex}

% Planificación temporal y presupuesto
\chapter{Planificación Temporal y Presupuesto}
\input{capitulos/planificacion_presupuesto/main.tex}

% Bibliografía
\newpage
\addcontentsline{toc}{chapter}{Bibliografía}
\printbibliography

\end{document}


% Objetivos
\chapter{Objetivos}
\documentclass[a4paper,11pt,twoside]{report}
\usepackage[left=25mm,right=25mm,top=25mm,bottom=25mm,includehead,includefoot,headsep=15mm,footskip=15mm]{geometry}
\usepackage{graphicx}
\usepackage{fancyhdr}
\usepackage{titlesec}
\usepackage[spanish]{babel}
\usepackage[utf8]{inputenc}
\usepackage{amsmath}
\usepackage{setspace}
\usepackage{svg}
\usepackage{hyperref}
\usepackage[backend=biber,style=numeric]{biblatex}
\addbibresource{references.bib}
\hypersetup{
    colorlinks=true,
    linkcolor=blue,      % color of internal links (sections, etc.)
    urlcolor=blue,       % color of external links
    pdftitle={Optimización energética de sistema híbrido con bomba de calor, suelo radiante, fotovoltaica y almacenamiento para vivienda},    % title
    pdfauthor={Luis D. Aranda Sánchez},     % author
    pdfkeywords={palabra1, palabra2, código1, etc.} % list of keywords
}

% Font change to Arial
\usepackage{helvet}
\renewcommand{\familydefault}{\sfdefault}

% Chapter titles in uppercase and larger font
\titleformat{\chapter}[hang]{\large\bfseries}{\thechapter.}{1em}{\MakeUppercase}
\titleformat{\section}[hang]{\bfseries}{\thesection.}{1em}{}
\titleformat{\subsection}[hang]{\bfseries}{\thesubsection.}{1em}{}

% Fancyhdr setup
\setlength{\headheight}{14.30174pt} % Adjust to recommended value, slightly larger for safety
\fancyhf{} % Clear all headers and footers
\fancyhead[LE]{\nouppercase{\leftmark}}
\fancyhead[RO]{Optimización energética para vivienda}
\fancyfoot[LE]{\thepage}
\fancyfoot[RE]{Escuela Técnica Superior de Ingenieros Industriales (UPM)}
\fancyfoot[LO]{Luis D. Aranda Sánchez}
\fancyfoot[RO]{\thepage}
\renewcommand{\headrulewidth}{0.4pt}
\renewcommand{\footrulewidth}{0.4pt}

\fancypagestyle{myfancy}{
    \fancyhf{} % Clear all headers and footers
    \fancyhead[LE]{\nouppercase{\leftmark}}
    \fancyhead[RO]{Optimización energética para vivienda}
    \fancyfoot[LE]{\thepage}
    \fancyfoot[RE]{Escuela Técnica Superior de Ingenieros Industriales (UPM)}
    \fancyfoot[LO]{Luis D. Aranda Sánchez}
    \fancyfoot[RO]{\thepage}
    \renewcommand{\headrulewidth}{0.4pt}
    \renewcommand{\footrulewidth}{0.4pt}
}

\fancypagestyle{simple}{
    \fancyhf{} % Clear all headers and footers
    \renewcommand{\headrulewidth}{0pt}
    \renewcommand{\footrulewidth}{0pt}
}

% Line spacing
\setstretch{1.2}

% Document starts here
\begin{document}

% Portada
\begin{titlepage}
    \centering
    {\scshape\LARGE Universidad Politécnica de Madrid \par}
    \vspace{1cm}
    {\scshape\Large Escuela Técnica Superior de Ingenieros Industriales\par}
    \vspace{1.5cm}
    {\huge\bfseries Optimización energética de sistema híbrido con bomba de calor, suelo radiante, fotovoltaica y almacenamiento para vivienda \par}
    \vspace{1.5cm}
    {\Large\bfseries Trabajo de Fin de Máster\par}
    \vspace{0.5cm}
    {\large Máster Universitario en Ingeniería de la Energía \par}
    \vspace{2cm}
    {\Large Luis D. Aranda Sánchez\par}
    \vfill
    Director: Javier Rodríguez Martín
    \vfill
    {\large Septiembre 6, 2024\par}
\end{titlepage}

% Resumen (máximo de 5 páginas, incluyendo al final Palabras clave)
\clearpage
\pagestyle{simple}
% \newpage
\chapter*{Resumen}
\addcontentsline{toc}{chapter}{Resumen}
\input{capitulos/resumen/main.tex}

% Índice (paginado)
\clearpage
\pagestyle{simple}
% \newpage
\tableofcontents

% Introducción (donde se incluya los antecedentes y justificación)
\clearpage
\pagestyle{myfancy}
\newpage
\chapter{Introducción}
\input{capitulos/introduccion/main.tex}

% Objetivos
\chapter{Objetivos}
\input{capitulos/objetivos/main.tex}

% Metodología
\chapter{Metodología}
\input{capitulos/metodologia/main.tex}

% Resultados y discusión (incluyendo la valoración de impactos y de aspectos de responsabilidad legal, ética y profesional relacionados con el trabajo)
\chapter{Resultados y Discusión}
\input{capitulos/resultados_discusion/main.tex}

% Conclusiones
\chapter{Conclusiones}
\input{capitulos/conclusiones/main.tex}

% Planificación temporal y presupuesto
\chapter{Planificación Temporal y Presupuesto}
\input{capitulos/planificacion_presupuesto/main.tex}

% Bibliografía
\newpage
\addcontentsline{toc}{chapter}{Bibliografía}
\printbibliography

\end{document}


% Metodología
\chapter{Metodología}
\documentclass[a4paper,11pt,twoside]{report}
\usepackage[left=25mm,right=25mm,top=25mm,bottom=25mm,includehead,includefoot,headsep=15mm,footskip=15mm]{geometry}
\usepackage{graphicx}
\usepackage{fancyhdr}
\usepackage{titlesec}
\usepackage[spanish]{babel}
\usepackage[utf8]{inputenc}
\usepackage{amsmath}
\usepackage{setspace}
\usepackage{svg}
\usepackage{hyperref}
\usepackage[backend=biber,style=numeric]{biblatex}
\addbibresource{references.bib}
\hypersetup{
    colorlinks=true,
    linkcolor=blue,      % color of internal links (sections, etc.)
    urlcolor=blue,       % color of external links
    pdftitle={Optimización energética de sistema híbrido con bomba de calor, suelo radiante, fotovoltaica y almacenamiento para vivienda},    % title
    pdfauthor={Luis D. Aranda Sánchez},     % author
    pdfkeywords={palabra1, palabra2, código1, etc.} % list of keywords
}

% Font change to Arial
\usepackage{helvet}
\renewcommand{\familydefault}{\sfdefault}

% Chapter titles in uppercase and larger font
\titleformat{\chapter}[hang]{\large\bfseries}{\thechapter.}{1em}{\MakeUppercase}
\titleformat{\section}[hang]{\bfseries}{\thesection.}{1em}{}
\titleformat{\subsection}[hang]{\bfseries}{\thesubsection.}{1em}{}

% Fancyhdr setup
\setlength{\headheight}{14.30174pt} % Adjust to recommended value, slightly larger for safety
\fancyhf{} % Clear all headers and footers
\fancyhead[LE]{\nouppercase{\leftmark}}
\fancyhead[RO]{Optimización energética para vivienda}
\fancyfoot[LE]{\thepage}
\fancyfoot[RE]{Escuela Técnica Superior de Ingenieros Industriales (UPM)}
\fancyfoot[LO]{Luis D. Aranda Sánchez}
\fancyfoot[RO]{\thepage}
\renewcommand{\headrulewidth}{0.4pt}
\renewcommand{\footrulewidth}{0.4pt}

\fancypagestyle{myfancy}{
    \fancyhf{} % Clear all headers and footers
    \fancyhead[LE]{\nouppercase{\leftmark}}
    \fancyhead[RO]{Optimización energética para vivienda}
    \fancyfoot[LE]{\thepage}
    \fancyfoot[RE]{Escuela Técnica Superior de Ingenieros Industriales (UPM)}
    \fancyfoot[LO]{Luis D. Aranda Sánchez}
    \fancyfoot[RO]{\thepage}
    \renewcommand{\headrulewidth}{0.4pt}
    \renewcommand{\footrulewidth}{0.4pt}
}

\fancypagestyle{simple}{
    \fancyhf{} % Clear all headers and footers
    \renewcommand{\headrulewidth}{0pt}
    \renewcommand{\footrulewidth}{0pt}
}

% Line spacing
\setstretch{1.2}

% Document starts here
\begin{document}

% Portada
\begin{titlepage}
    \centering
    {\scshape\LARGE Universidad Politécnica de Madrid \par}
    \vspace{1cm}
    {\scshape\Large Escuela Técnica Superior de Ingenieros Industriales\par}
    \vspace{1.5cm}
    {\huge\bfseries Optimización energética de sistema híbrido con bomba de calor, suelo radiante, fotovoltaica y almacenamiento para vivienda \par}
    \vspace{1.5cm}
    {\Large\bfseries Trabajo de Fin de Máster\par}
    \vspace{0.5cm}
    {\large Máster Universitario en Ingeniería de la Energía \par}
    \vspace{2cm}
    {\Large Luis D. Aranda Sánchez\par}
    \vfill
    Director: Javier Rodríguez Martín
    \vfill
    {\large Septiembre 6, 2024\par}
\end{titlepage}

% Resumen (máximo de 5 páginas, incluyendo al final Palabras clave)
\clearpage
\pagestyle{simple}
% \newpage
\chapter*{Resumen}
\addcontentsline{toc}{chapter}{Resumen}
\input{capitulos/resumen/main.tex}

% Índice (paginado)
\clearpage
\pagestyle{simple}
% \newpage
\tableofcontents

% Introducción (donde se incluya los antecedentes y justificación)
\clearpage
\pagestyle{myfancy}
\newpage
\chapter{Introducción}
\input{capitulos/introduccion/main.tex}

% Objetivos
\chapter{Objetivos}
\input{capitulos/objetivos/main.tex}

% Metodología
\chapter{Metodología}
\input{capitulos/metodologia/main.tex}

% Resultados y discusión (incluyendo la valoración de impactos y de aspectos de responsabilidad legal, ética y profesional relacionados con el trabajo)
\chapter{Resultados y Discusión}
\input{capitulos/resultados_discusion/main.tex}

% Conclusiones
\chapter{Conclusiones}
\input{capitulos/conclusiones/main.tex}

% Planificación temporal y presupuesto
\chapter{Planificación Temporal y Presupuesto}
\input{capitulos/planificacion_presupuesto/main.tex}

% Bibliografía
\newpage
\addcontentsline{toc}{chapter}{Bibliografía}
\printbibliography

\end{document}


% Resultados y discusión (incluyendo la valoración de impactos y de aspectos de responsabilidad legal, ética y profesional relacionados con el trabajo)
\chapter{Resultados y Discusión}
\documentclass[a4paper,11pt,twoside]{report}
\usepackage[left=25mm,right=25mm,top=25mm,bottom=25mm,includehead,includefoot,headsep=15mm,footskip=15mm]{geometry}
\usepackage{graphicx}
\usepackage{fancyhdr}
\usepackage{titlesec}
\usepackage[spanish]{babel}
\usepackage[utf8]{inputenc}
\usepackage{amsmath}
\usepackage{setspace}
\usepackage{svg}
\usepackage{hyperref}
\usepackage[backend=biber,style=numeric]{biblatex}
\addbibresource{references.bib}
\hypersetup{
    colorlinks=true,
    linkcolor=blue,      % color of internal links (sections, etc.)
    urlcolor=blue,       % color of external links
    pdftitle={Optimización energética de sistema híbrido con bomba de calor, suelo radiante, fotovoltaica y almacenamiento para vivienda},    % title
    pdfauthor={Luis D. Aranda Sánchez},     % author
    pdfkeywords={palabra1, palabra2, código1, etc.} % list of keywords
}

% Font change to Arial
\usepackage{helvet}
\renewcommand{\familydefault}{\sfdefault}

% Chapter titles in uppercase and larger font
\titleformat{\chapter}[hang]{\large\bfseries}{\thechapter.}{1em}{\MakeUppercase}
\titleformat{\section}[hang]{\bfseries}{\thesection.}{1em}{}
\titleformat{\subsection}[hang]{\bfseries}{\thesubsection.}{1em}{}

% Fancyhdr setup
\setlength{\headheight}{14.30174pt} % Adjust to recommended value, slightly larger for safety
\fancyhf{} % Clear all headers and footers
\fancyhead[LE]{\nouppercase{\leftmark}}
\fancyhead[RO]{Optimización energética para vivienda}
\fancyfoot[LE]{\thepage}
\fancyfoot[RE]{Escuela Técnica Superior de Ingenieros Industriales (UPM)}
\fancyfoot[LO]{Luis D. Aranda Sánchez}
\fancyfoot[RO]{\thepage}
\renewcommand{\headrulewidth}{0.4pt}
\renewcommand{\footrulewidth}{0.4pt}

\fancypagestyle{myfancy}{
    \fancyhf{} % Clear all headers and footers
    \fancyhead[LE]{\nouppercase{\leftmark}}
    \fancyhead[RO]{Optimización energética para vivienda}
    \fancyfoot[LE]{\thepage}
    \fancyfoot[RE]{Escuela Técnica Superior de Ingenieros Industriales (UPM)}
    \fancyfoot[LO]{Luis D. Aranda Sánchez}
    \fancyfoot[RO]{\thepage}
    \renewcommand{\headrulewidth}{0.4pt}
    \renewcommand{\footrulewidth}{0.4pt}
}

\fancypagestyle{simple}{
    \fancyhf{} % Clear all headers and footers
    \renewcommand{\headrulewidth}{0pt}
    \renewcommand{\footrulewidth}{0pt}
}

% Line spacing
\setstretch{1.2}

% Document starts here
\begin{document}

% Portada
\begin{titlepage}
    \centering
    {\scshape\LARGE Universidad Politécnica de Madrid \par}
    \vspace{1cm}
    {\scshape\Large Escuela Técnica Superior de Ingenieros Industriales\par}
    \vspace{1.5cm}
    {\huge\bfseries Optimización energética de sistema híbrido con bomba de calor, suelo radiante, fotovoltaica y almacenamiento para vivienda \par}
    \vspace{1.5cm}
    {\Large\bfseries Trabajo de Fin de Máster\par}
    \vspace{0.5cm}
    {\large Máster Universitario en Ingeniería de la Energía \par}
    \vspace{2cm}
    {\Large Luis D. Aranda Sánchez\par}
    \vfill
    Director: Javier Rodríguez Martín
    \vfill
    {\large Septiembre 6, 2024\par}
\end{titlepage}

% Resumen (máximo de 5 páginas, incluyendo al final Palabras clave)
\clearpage
\pagestyle{simple}
% \newpage
\chapter*{Resumen}
\addcontentsline{toc}{chapter}{Resumen}
\input{capitulos/resumen/main.tex}

% Índice (paginado)
\clearpage
\pagestyle{simple}
% \newpage
\tableofcontents

% Introducción (donde se incluya los antecedentes y justificación)
\clearpage
\pagestyle{myfancy}
\newpage
\chapter{Introducción}
\input{capitulos/introduccion/main.tex}

% Objetivos
\chapter{Objetivos}
\input{capitulos/objetivos/main.tex}

% Metodología
\chapter{Metodología}
\input{capitulos/metodologia/main.tex}

% Resultados y discusión (incluyendo la valoración de impactos y de aspectos de responsabilidad legal, ética y profesional relacionados con el trabajo)
\chapter{Resultados y Discusión}
\input{capitulos/resultados_discusion/main.tex}

% Conclusiones
\chapter{Conclusiones}
\input{capitulos/conclusiones/main.tex}

% Planificación temporal y presupuesto
\chapter{Planificación Temporal y Presupuesto}
\input{capitulos/planificacion_presupuesto/main.tex}

% Bibliografía
\newpage
\addcontentsline{toc}{chapter}{Bibliografía}
\printbibliography

\end{document}


% Conclusiones
\chapter{Conclusiones}
\documentclass[a4paper,11pt,twoside]{report}
\usepackage[left=25mm,right=25mm,top=25mm,bottom=25mm,includehead,includefoot,headsep=15mm,footskip=15mm]{geometry}
\usepackage{graphicx}
\usepackage{fancyhdr}
\usepackage{titlesec}
\usepackage[spanish]{babel}
\usepackage[utf8]{inputenc}
\usepackage{amsmath}
\usepackage{setspace}
\usepackage{svg}
\usepackage{hyperref}
\usepackage[backend=biber,style=numeric]{biblatex}
\addbibresource{references.bib}
\hypersetup{
    colorlinks=true,
    linkcolor=blue,      % color of internal links (sections, etc.)
    urlcolor=blue,       % color of external links
    pdftitle={Optimización energética de sistema híbrido con bomba de calor, suelo radiante, fotovoltaica y almacenamiento para vivienda},    % title
    pdfauthor={Luis D. Aranda Sánchez},     % author
    pdfkeywords={palabra1, palabra2, código1, etc.} % list of keywords
}

% Font change to Arial
\usepackage{helvet}
\renewcommand{\familydefault}{\sfdefault}

% Chapter titles in uppercase and larger font
\titleformat{\chapter}[hang]{\large\bfseries}{\thechapter.}{1em}{\MakeUppercase}
\titleformat{\section}[hang]{\bfseries}{\thesection.}{1em}{}
\titleformat{\subsection}[hang]{\bfseries}{\thesubsection.}{1em}{}

% Fancyhdr setup
\setlength{\headheight}{14.30174pt} % Adjust to recommended value, slightly larger for safety
\fancyhf{} % Clear all headers and footers
\fancyhead[LE]{\nouppercase{\leftmark}}
\fancyhead[RO]{Optimización energética para vivienda}
\fancyfoot[LE]{\thepage}
\fancyfoot[RE]{Escuela Técnica Superior de Ingenieros Industriales (UPM)}
\fancyfoot[LO]{Luis D. Aranda Sánchez}
\fancyfoot[RO]{\thepage}
\renewcommand{\headrulewidth}{0.4pt}
\renewcommand{\footrulewidth}{0.4pt}

\fancypagestyle{myfancy}{
    \fancyhf{} % Clear all headers and footers
    \fancyhead[LE]{\nouppercase{\leftmark}}
    \fancyhead[RO]{Optimización energética para vivienda}
    \fancyfoot[LE]{\thepage}
    \fancyfoot[RE]{Escuela Técnica Superior de Ingenieros Industriales (UPM)}
    \fancyfoot[LO]{Luis D. Aranda Sánchez}
    \fancyfoot[RO]{\thepage}
    \renewcommand{\headrulewidth}{0.4pt}
    \renewcommand{\footrulewidth}{0.4pt}
}

\fancypagestyle{simple}{
    \fancyhf{} % Clear all headers and footers
    \renewcommand{\headrulewidth}{0pt}
    \renewcommand{\footrulewidth}{0pt}
}

% Line spacing
\setstretch{1.2}

% Document starts here
\begin{document}

% Portada
\begin{titlepage}
    \centering
    {\scshape\LARGE Universidad Politécnica de Madrid \par}
    \vspace{1cm}
    {\scshape\Large Escuela Técnica Superior de Ingenieros Industriales\par}
    \vspace{1.5cm}
    {\huge\bfseries Optimización energética de sistema híbrido con bomba de calor, suelo radiante, fotovoltaica y almacenamiento para vivienda \par}
    \vspace{1.5cm}
    {\Large\bfseries Trabajo de Fin de Máster\par}
    \vspace{0.5cm}
    {\large Máster Universitario en Ingeniería de la Energía \par}
    \vspace{2cm}
    {\Large Luis D. Aranda Sánchez\par}
    \vfill
    Director: Javier Rodríguez Martín
    \vfill
    {\large Septiembre 6, 2024\par}
\end{titlepage}

% Resumen (máximo de 5 páginas, incluyendo al final Palabras clave)
\clearpage
\pagestyle{simple}
% \newpage
\chapter*{Resumen}
\addcontentsline{toc}{chapter}{Resumen}
\input{capitulos/resumen/main.tex}

% Índice (paginado)
\clearpage
\pagestyle{simple}
% \newpage
\tableofcontents

% Introducción (donde se incluya los antecedentes y justificación)
\clearpage
\pagestyle{myfancy}
\newpage
\chapter{Introducción}
\input{capitulos/introduccion/main.tex}

% Objetivos
\chapter{Objetivos}
\input{capitulos/objetivos/main.tex}

% Metodología
\chapter{Metodología}
\input{capitulos/metodologia/main.tex}

% Resultados y discusión (incluyendo la valoración de impactos y de aspectos de responsabilidad legal, ética y profesional relacionados con el trabajo)
\chapter{Resultados y Discusión}
\input{capitulos/resultados_discusion/main.tex}

% Conclusiones
\chapter{Conclusiones}
\input{capitulos/conclusiones/main.tex}

% Planificación temporal y presupuesto
\chapter{Planificación Temporal y Presupuesto}
\input{capitulos/planificacion_presupuesto/main.tex}

% Bibliografía
\newpage
\addcontentsline{toc}{chapter}{Bibliografía}
\printbibliography

\end{document}


% Planificación temporal y presupuesto
\chapter{Planificación Temporal y Presupuesto}
\documentclass[a4paper,11pt,twoside]{report}
\usepackage[left=25mm,right=25mm,top=25mm,bottom=25mm,includehead,includefoot,headsep=15mm,footskip=15mm]{geometry}
\usepackage{graphicx}
\usepackage{fancyhdr}
\usepackage{titlesec}
\usepackage[spanish]{babel}
\usepackage[utf8]{inputenc}
\usepackage{amsmath}
\usepackage{setspace}
\usepackage{svg}
\usepackage{hyperref}
\usepackage[backend=biber,style=numeric]{biblatex}
\addbibresource{references.bib}
\hypersetup{
    colorlinks=true,
    linkcolor=blue,      % color of internal links (sections, etc.)
    urlcolor=blue,       % color of external links
    pdftitle={Optimización energética de sistema híbrido con bomba de calor, suelo radiante, fotovoltaica y almacenamiento para vivienda},    % title
    pdfauthor={Luis D. Aranda Sánchez},     % author
    pdfkeywords={palabra1, palabra2, código1, etc.} % list of keywords
}

% Font change to Arial
\usepackage{helvet}
\renewcommand{\familydefault}{\sfdefault}

% Chapter titles in uppercase and larger font
\titleformat{\chapter}[hang]{\large\bfseries}{\thechapter.}{1em}{\MakeUppercase}
\titleformat{\section}[hang]{\bfseries}{\thesection.}{1em}{}
\titleformat{\subsection}[hang]{\bfseries}{\thesubsection.}{1em}{}

% Fancyhdr setup
\setlength{\headheight}{14.30174pt} % Adjust to recommended value, slightly larger for safety
\fancyhf{} % Clear all headers and footers
\fancyhead[LE]{\nouppercase{\leftmark}}
\fancyhead[RO]{Optimización energética para vivienda}
\fancyfoot[LE]{\thepage}
\fancyfoot[RE]{Escuela Técnica Superior de Ingenieros Industriales (UPM)}
\fancyfoot[LO]{Luis D. Aranda Sánchez}
\fancyfoot[RO]{\thepage}
\renewcommand{\headrulewidth}{0.4pt}
\renewcommand{\footrulewidth}{0.4pt}

\fancypagestyle{myfancy}{
    \fancyhf{} % Clear all headers and footers
    \fancyhead[LE]{\nouppercase{\leftmark}}
    \fancyhead[RO]{Optimización energética para vivienda}
    \fancyfoot[LE]{\thepage}
    \fancyfoot[RE]{Escuela Técnica Superior de Ingenieros Industriales (UPM)}
    \fancyfoot[LO]{Luis D. Aranda Sánchez}
    \fancyfoot[RO]{\thepage}
    \renewcommand{\headrulewidth}{0.4pt}
    \renewcommand{\footrulewidth}{0.4pt}
}

\fancypagestyle{simple}{
    \fancyhf{} % Clear all headers and footers
    \renewcommand{\headrulewidth}{0pt}
    \renewcommand{\footrulewidth}{0pt}
}

% Line spacing
\setstretch{1.2}

% Document starts here
\begin{document}

% Portada
\begin{titlepage}
    \centering
    {\scshape\LARGE Universidad Politécnica de Madrid \par}
    \vspace{1cm}
    {\scshape\Large Escuela Técnica Superior de Ingenieros Industriales\par}
    \vspace{1.5cm}
    {\huge\bfseries Optimización energética de sistema híbrido con bomba de calor, suelo radiante, fotovoltaica y almacenamiento para vivienda \par}
    \vspace{1.5cm}
    {\Large\bfseries Trabajo de Fin de Máster\par}
    \vspace{0.5cm}
    {\large Máster Universitario en Ingeniería de la Energía \par}
    \vspace{2cm}
    {\Large Luis D. Aranda Sánchez\par}
    \vfill
    Director: Javier Rodríguez Martín
    \vfill
    {\large Septiembre 6, 2024\par}
\end{titlepage}

% Resumen (máximo de 5 páginas, incluyendo al final Palabras clave)
\clearpage
\pagestyle{simple}
% \newpage
\chapter*{Resumen}
\addcontentsline{toc}{chapter}{Resumen}
\input{capitulos/resumen/main.tex}

% Índice (paginado)
\clearpage
\pagestyle{simple}
% \newpage
\tableofcontents

% Introducción (donde se incluya los antecedentes y justificación)
\clearpage
\pagestyle{myfancy}
\newpage
\chapter{Introducción}
\input{capitulos/introduccion/main.tex}

% Objetivos
\chapter{Objetivos}
\input{capitulos/objetivos/main.tex}

% Metodología
\chapter{Metodología}
\input{capitulos/metodologia/main.tex}

% Resultados y discusión (incluyendo la valoración de impactos y de aspectos de responsabilidad legal, ética y profesional relacionados con el trabajo)
\chapter{Resultados y Discusión}
\input{capitulos/resultados_discusion/main.tex}

% Conclusiones
\chapter{Conclusiones}
\input{capitulos/conclusiones/main.tex}

% Planificación temporal y presupuesto
\chapter{Planificación Temporal y Presupuesto}
\input{capitulos/planificacion_presupuesto/main.tex}

% Bibliografía
\newpage
\addcontentsline{toc}{chapter}{Bibliografía}
\printbibliography

\end{document}


% Bibliografía
\newpage
\addcontentsline{toc}{chapter}{Bibliografía}
\printbibliography

\end{document}


% Resultados y discusión (incluyendo la valoración de impactos y de aspectos de responsabilidad legal, ética y profesional relacionados con el trabajo)
\chapter{Resultados y Discusión}
\documentclass[a4paper,11pt,twoside]{report}
\usepackage[left=25mm,right=25mm,top=25mm,bottom=25mm,includehead,includefoot,headsep=15mm,footskip=15mm]{geometry}
\usepackage{graphicx}
\usepackage{fancyhdr}
\usepackage{titlesec}
\usepackage[spanish]{babel}
\usepackage[utf8]{inputenc}
\usepackage{amsmath}
\usepackage{setspace}
\usepackage{svg}
\usepackage{hyperref}
\usepackage[backend=biber,style=numeric]{biblatex}
\addbibresource{references.bib}
\hypersetup{
    colorlinks=true,
    linkcolor=blue,      % color of internal links (sections, etc.)
    urlcolor=blue,       % color of external links
    pdftitle={Optimización energética de sistema híbrido con bomba de calor, suelo radiante, fotovoltaica y almacenamiento para vivienda},    % title
    pdfauthor={Luis D. Aranda Sánchez},     % author
    pdfkeywords={palabra1, palabra2, código1, etc.} % list of keywords
}

% Font change to Arial
\usepackage{helvet}
\renewcommand{\familydefault}{\sfdefault}

% Chapter titles in uppercase and larger font
\titleformat{\chapter}[hang]{\large\bfseries}{\thechapter.}{1em}{\MakeUppercase}
\titleformat{\section}[hang]{\bfseries}{\thesection.}{1em}{}
\titleformat{\subsection}[hang]{\bfseries}{\thesubsection.}{1em}{}

% Fancyhdr setup
\setlength{\headheight}{14.30174pt} % Adjust to recommended value, slightly larger for safety
\fancyhf{} % Clear all headers and footers
\fancyhead[LE]{\nouppercase{\leftmark}}
\fancyhead[RO]{Optimización energética para vivienda}
\fancyfoot[LE]{\thepage}
\fancyfoot[RE]{Escuela Técnica Superior de Ingenieros Industriales (UPM)}
\fancyfoot[LO]{Luis D. Aranda Sánchez}
\fancyfoot[RO]{\thepage}
\renewcommand{\headrulewidth}{0.4pt}
\renewcommand{\footrulewidth}{0.4pt}

\fancypagestyle{myfancy}{
    \fancyhf{} % Clear all headers and footers
    \fancyhead[LE]{\nouppercase{\leftmark}}
    \fancyhead[RO]{Optimización energética para vivienda}
    \fancyfoot[LE]{\thepage}
    \fancyfoot[RE]{Escuela Técnica Superior de Ingenieros Industriales (UPM)}
    \fancyfoot[LO]{Luis D. Aranda Sánchez}
    \fancyfoot[RO]{\thepage}
    \renewcommand{\headrulewidth}{0.4pt}
    \renewcommand{\footrulewidth}{0.4pt}
}

\fancypagestyle{simple}{
    \fancyhf{} % Clear all headers and footers
    \renewcommand{\headrulewidth}{0pt}
    \renewcommand{\footrulewidth}{0pt}
}

% Line spacing
\setstretch{1.2}

% Document starts here
\begin{document}

% Portada
\begin{titlepage}
    \centering
    {\scshape\LARGE Universidad Politécnica de Madrid \par}
    \vspace{1cm}
    {\scshape\Large Escuela Técnica Superior de Ingenieros Industriales\par}
    \vspace{1.5cm}
    {\huge\bfseries Optimización energética de sistema híbrido con bomba de calor, suelo radiante, fotovoltaica y almacenamiento para vivienda \par}
    \vspace{1.5cm}
    {\Large\bfseries Trabajo de Fin de Máster\par}
    \vspace{0.5cm}
    {\large Máster Universitario en Ingeniería de la Energía \par}
    \vspace{2cm}
    {\Large Luis D. Aranda Sánchez\par}
    \vfill
    Director: Javier Rodríguez Martín
    \vfill
    {\large Septiembre 6, 2024\par}
\end{titlepage}

% Resumen (máximo de 5 páginas, incluyendo al final Palabras clave)
\clearpage
\pagestyle{simple}
% \newpage
\chapter*{Resumen}
\addcontentsline{toc}{chapter}{Resumen}
\documentclass[a4paper,11pt,twoside]{report}
\usepackage[left=25mm,right=25mm,top=25mm,bottom=25mm,includehead,includefoot,headsep=15mm,footskip=15mm]{geometry}
\usepackage{graphicx}
\usepackage{fancyhdr}
\usepackage{titlesec}
\usepackage[spanish]{babel}
\usepackage[utf8]{inputenc}
\usepackage{amsmath}
\usepackage{setspace}
\usepackage{svg}
\usepackage{hyperref}
\usepackage[backend=biber,style=numeric]{biblatex}
\addbibresource{references.bib}
\hypersetup{
    colorlinks=true,
    linkcolor=blue,      % color of internal links (sections, etc.)
    urlcolor=blue,       % color of external links
    pdftitle={Optimización energética de sistema híbrido con bomba de calor, suelo radiante, fotovoltaica y almacenamiento para vivienda},    % title
    pdfauthor={Luis D. Aranda Sánchez},     % author
    pdfkeywords={palabra1, palabra2, código1, etc.} % list of keywords
}

% Font change to Arial
\usepackage{helvet}
\renewcommand{\familydefault}{\sfdefault}

% Chapter titles in uppercase and larger font
\titleformat{\chapter}[hang]{\large\bfseries}{\thechapter.}{1em}{\MakeUppercase}
\titleformat{\section}[hang]{\bfseries}{\thesection.}{1em}{}
\titleformat{\subsection}[hang]{\bfseries}{\thesubsection.}{1em}{}

% Fancyhdr setup
\setlength{\headheight}{14.30174pt} % Adjust to recommended value, slightly larger for safety
\fancyhf{} % Clear all headers and footers
\fancyhead[LE]{\nouppercase{\leftmark}}
\fancyhead[RO]{Optimización energética para vivienda}
\fancyfoot[LE]{\thepage}
\fancyfoot[RE]{Escuela Técnica Superior de Ingenieros Industriales (UPM)}
\fancyfoot[LO]{Luis D. Aranda Sánchez}
\fancyfoot[RO]{\thepage}
\renewcommand{\headrulewidth}{0.4pt}
\renewcommand{\footrulewidth}{0.4pt}

\fancypagestyle{myfancy}{
    \fancyhf{} % Clear all headers and footers
    \fancyhead[LE]{\nouppercase{\leftmark}}
    \fancyhead[RO]{Optimización energética para vivienda}
    \fancyfoot[LE]{\thepage}
    \fancyfoot[RE]{Escuela Técnica Superior de Ingenieros Industriales (UPM)}
    \fancyfoot[LO]{Luis D. Aranda Sánchez}
    \fancyfoot[RO]{\thepage}
    \renewcommand{\headrulewidth}{0.4pt}
    \renewcommand{\footrulewidth}{0.4pt}
}

\fancypagestyle{simple}{
    \fancyhf{} % Clear all headers and footers
    \renewcommand{\headrulewidth}{0pt}
    \renewcommand{\footrulewidth}{0pt}
}

% Line spacing
\setstretch{1.2}

% Document starts here
\begin{document}

% Portada
\begin{titlepage}
    \centering
    {\scshape\LARGE Universidad Politécnica de Madrid \par}
    \vspace{1cm}
    {\scshape\Large Escuela Técnica Superior de Ingenieros Industriales\par}
    \vspace{1.5cm}
    {\huge\bfseries Optimización energética de sistema híbrido con bomba de calor, suelo radiante, fotovoltaica y almacenamiento para vivienda \par}
    \vspace{1.5cm}
    {\Large\bfseries Trabajo de Fin de Máster\par}
    \vspace{0.5cm}
    {\large Máster Universitario en Ingeniería de la Energía \par}
    \vspace{2cm}
    {\Large Luis D. Aranda Sánchez\par}
    \vfill
    Director: Javier Rodríguez Martín
    \vfill
    {\large Septiembre 6, 2024\par}
\end{titlepage}

% Resumen (máximo de 5 páginas, incluyendo al final Palabras clave)
\clearpage
\pagestyle{simple}
% \newpage
\chapter*{Resumen}
\addcontentsline{toc}{chapter}{Resumen}
\input{capitulos/resumen/main.tex}

% Índice (paginado)
\clearpage
\pagestyle{simple}
% \newpage
\tableofcontents

% Introducción (donde se incluya los antecedentes y justificación)
\clearpage
\pagestyle{myfancy}
\newpage
\chapter{Introducción}
\input{capitulos/introduccion/main.tex}

% Objetivos
\chapter{Objetivos}
\input{capitulos/objetivos/main.tex}

% Metodología
\chapter{Metodología}
\input{capitulos/metodologia/main.tex}

% Resultados y discusión (incluyendo la valoración de impactos y de aspectos de responsabilidad legal, ética y profesional relacionados con el trabajo)
\chapter{Resultados y Discusión}
\input{capitulos/resultados_discusion/main.tex}

% Conclusiones
\chapter{Conclusiones}
\input{capitulos/conclusiones/main.tex}

% Planificación temporal y presupuesto
\chapter{Planificación Temporal y Presupuesto}
\input{capitulos/planificacion_presupuesto/main.tex}

% Bibliografía
\newpage
\addcontentsline{toc}{chapter}{Bibliografía}
\printbibliography

\end{document}


% Índice (paginado)
\clearpage
\pagestyle{simple}
% \newpage
\tableofcontents

% Introducción (donde se incluya los antecedentes y justificación)
\clearpage
\pagestyle{myfancy}
\newpage
\chapter{Introducción}
\documentclass[a4paper,11pt,twoside]{report}
\usepackage[left=25mm,right=25mm,top=25mm,bottom=25mm,includehead,includefoot,headsep=15mm,footskip=15mm]{geometry}
\usepackage{graphicx}
\usepackage{fancyhdr}
\usepackage{titlesec}
\usepackage[spanish]{babel}
\usepackage[utf8]{inputenc}
\usepackage{amsmath}
\usepackage{setspace}
\usepackage{svg}
\usepackage{hyperref}
\usepackage[backend=biber,style=numeric]{biblatex}
\addbibresource{references.bib}
\hypersetup{
    colorlinks=true,
    linkcolor=blue,      % color of internal links (sections, etc.)
    urlcolor=blue,       % color of external links
    pdftitle={Optimización energética de sistema híbrido con bomba de calor, suelo radiante, fotovoltaica y almacenamiento para vivienda},    % title
    pdfauthor={Luis D. Aranda Sánchez},     % author
    pdfkeywords={palabra1, palabra2, código1, etc.} % list of keywords
}

% Font change to Arial
\usepackage{helvet}
\renewcommand{\familydefault}{\sfdefault}

% Chapter titles in uppercase and larger font
\titleformat{\chapter}[hang]{\large\bfseries}{\thechapter.}{1em}{\MakeUppercase}
\titleformat{\section}[hang]{\bfseries}{\thesection.}{1em}{}
\titleformat{\subsection}[hang]{\bfseries}{\thesubsection.}{1em}{}

% Fancyhdr setup
\setlength{\headheight}{14.30174pt} % Adjust to recommended value, slightly larger for safety
\fancyhf{} % Clear all headers and footers
\fancyhead[LE]{\nouppercase{\leftmark}}
\fancyhead[RO]{Optimización energética para vivienda}
\fancyfoot[LE]{\thepage}
\fancyfoot[RE]{Escuela Técnica Superior de Ingenieros Industriales (UPM)}
\fancyfoot[LO]{Luis D. Aranda Sánchez}
\fancyfoot[RO]{\thepage}
\renewcommand{\headrulewidth}{0.4pt}
\renewcommand{\footrulewidth}{0.4pt}

\fancypagestyle{myfancy}{
    \fancyhf{} % Clear all headers and footers
    \fancyhead[LE]{\nouppercase{\leftmark}}
    \fancyhead[RO]{Optimización energética para vivienda}
    \fancyfoot[LE]{\thepage}
    \fancyfoot[RE]{Escuela Técnica Superior de Ingenieros Industriales (UPM)}
    \fancyfoot[LO]{Luis D. Aranda Sánchez}
    \fancyfoot[RO]{\thepage}
    \renewcommand{\headrulewidth}{0.4pt}
    \renewcommand{\footrulewidth}{0.4pt}
}

\fancypagestyle{simple}{
    \fancyhf{} % Clear all headers and footers
    \renewcommand{\headrulewidth}{0pt}
    \renewcommand{\footrulewidth}{0pt}
}

% Line spacing
\setstretch{1.2}

% Document starts here
\begin{document}

% Portada
\begin{titlepage}
    \centering
    {\scshape\LARGE Universidad Politécnica de Madrid \par}
    \vspace{1cm}
    {\scshape\Large Escuela Técnica Superior de Ingenieros Industriales\par}
    \vspace{1.5cm}
    {\huge\bfseries Optimización energética de sistema híbrido con bomba de calor, suelo radiante, fotovoltaica y almacenamiento para vivienda \par}
    \vspace{1.5cm}
    {\Large\bfseries Trabajo de Fin de Máster\par}
    \vspace{0.5cm}
    {\large Máster Universitario en Ingeniería de la Energía \par}
    \vspace{2cm}
    {\Large Luis D. Aranda Sánchez\par}
    \vfill
    Director: Javier Rodríguez Martín
    \vfill
    {\large Septiembre 6, 2024\par}
\end{titlepage}

% Resumen (máximo de 5 páginas, incluyendo al final Palabras clave)
\clearpage
\pagestyle{simple}
% \newpage
\chapter*{Resumen}
\addcontentsline{toc}{chapter}{Resumen}
\input{capitulos/resumen/main.tex}

% Índice (paginado)
\clearpage
\pagestyle{simple}
% \newpage
\tableofcontents

% Introducción (donde se incluya los antecedentes y justificación)
\clearpage
\pagestyle{myfancy}
\newpage
\chapter{Introducción}
\input{capitulos/introduccion/main.tex}

% Objetivos
\chapter{Objetivos}
\input{capitulos/objetivos/main.tex}

% Metodología
\chapter{Metodología}
\input{capitulos/metodologia/main.tex}

% Resultados y discusión (incluyendo la valoración de impactos y de aspectos de responsabilidad legal, ética y profesional relacionados con el trabajo)
\chapter{Resultados y Discusión}
\input{capitulos/resultados_discusion/main.tex}

% Conclusiones
\chapter{Conclusiones}
\input{capitulos/conclusiones/main.tex}

% Planificación temporal y presupuesto
\chapter{Planificación Temporal y Presupuesto}
\input{capitulos/planificacion_presupuesto/main.tex}

% Bibliografía
\newpage
\addcontentsline{toc}{chapter}{Bibliografía}
\printbibliography

\end{document}


% Objetivos
\chapter{Objetivos}
\documentclass[a4paper,11pt,twoside]{report}
\usepackage[left=25mm,right=25mm,top=25mm,bottom=25mm,includehead,includefoot,headsep=15mm,footskip=15mm]{geometry}
\usepackage{graphicx}
\usepackage{fancyhdr}
\usepackage{titlesec}
\usepackage[spanish]{babel}
\usepackage[utf8]{inputenc}
\usepackage{amsmath}
\usepackage{setspace}
\usepackage{svg}
\usepackage{hyperref}
\usepackage[backend=biber,style=numeric]{biblatex}
\addbibresource{references.bib}
\hypersetup{
    colorlinks=true,
    linkcolor=blue,      % color of internal links (sections, etc.)
    urlcolor=blue,       % color of external links
    pdftitle={Optimización energética de sistema híbrido con bomba de calor, suelo radiante, fotovoltaica y almacenamiento para vivienda},    % title
    pdfauthor={Luis D. Aranda Sánchez},     % author
    pdfkeywords={palabra1, palabra2, código1, etc.} % list of keywords
}

% Font change to Arial
\usepackage{helvet}
\renewcommand{\familydefault}{\sfdefault}

% Chapter titles in uppercase and larger font
\titleformat{\chapter}[hang]{\large\bfseries}{\thechapter.}{1em}{\MakeUppercase}
\titleformat{\section}[hang]{\bfseries}{\thesection.}{1em}{}
\titleformat{\subsection}[hang]{\bfseries}{\thesubsection.}{1em}{}

% Fancyhdr setup
\setlength{\headheight}{14.30174pt} % Adjust to recommended value, slightly larger for safety
\fancyhf{} % Clear all headers and footers
\fancyhead[LE]{\nouppercase{\leftmark}}
\fancyhead[RO]{Optimización energética para vivienda}
\fancyfoot[LE]{\thepage}
\fancyfoot[RE]{Escuela Técnica Superior de Ingenieros Industriales (UPM)}
\fancyfoot[LO]{Luis D. Aranda Sánchez}
\fancyfoot[RO]{\thepage}
\renewcommand{\headrulewidth}{0.4pt}
\renewcommand{\footrulewidth}{0.4pt}

\fancypagestyle{myfancy}{
    \fancyhf{} % Clear all headers and footers
    \fancyhead[LE]{\nouppercase{\leftmark}}
    \fancyhead[RO]{Optimización energética para vivienda}
    \fancyfoot[LE]{\thepage}
    \fancyfoot[RE]{Escuela Técnica Superior de Ingenieros Industriales (UPM)}
    \fancyfoot[LO]{Luis D. Aranda Sánchez}
    \fancyfoot[RO]{\thepage}
    \renewcommand{\headrulewidth}{0.4pt}
    \renewcommand{\footrulewidth}{0.4pt}
}

\fancypagestyle{simple}{
    \fancyhf{} % Clear all headers and footers
    \renewcommand{\headrulewidth}{0pt}
    \renewcommand{\footrulewidth}{0pt}
}

% Line spacing
\setstretch{1.2}

% Document starts here
\begin{document}

% Portada
\begin{titlepage}
    \centering
    {\scshape\LARGE Universidad Politécnica de Madrid \par}
    \vspace{1cm}
    {\scshape\Large Escuela Técnica Superior de Ingenieros Industriales\par}
    \vspace{1.5cm}
    {\huge\bfseries Optimización energética de sistema híbrido con bomba de calor, suelo radiante, fotovoltaica y almacenamiento para vivienda \par}
    \vspace{1.5cm}
    {\Large\bfseries Trabajo de Fin de Máster\par}
    \vspace{0.5cm}
    {\large Máster Universitario en Ingeniería de la Energía \par}
    \vspace{2cm}
    {\Large Luis D. Aranda Sánchez\par}
    \vfill
    Director: Javier Rodríguez Martín
    \vfill
    {\large Septiembre 6, 2024\par}
\end{titlepage}

% Resumen (máximo de 5 páginas, incluyendo al final Palabras clave)
\clearpage
\pagestyle{simple}
% \newpage
\chapter*{Resumen}
\addcontentsline{toc}{chapter}{Resumen}
\input{capitulos/resumen/main.tex}

% Índice (paginado)
\clearpage
\pagestyle{simple}
% \newpage
\tableofcontents

% Introducción (donde se incluya los antecedentes y justificación)
\clearpage
\pagestyle{myfancy}
\newpage
\chapter{Introducción}
\input{capitulos/introduccion/main.tex}

% Objetivos
\chapter{Objetivos}
\input{capitulos/objetivos/main.tex}

% Metodología
\chapter{Metodología}
\input{capitulos/metodologia/main.tex}

% Resultados y discusión (incluyendo la valoración de impactos y de aspectos de responsabilidad legal, ética y profesional relacionados con el trabajo)
\chapter{Resultados y Discusión}
\input{capitulos/resultados_discusion/main.tex}

% Conclusiones
\chapter{Conclusiones}
\input{capitulos/conclusiones/main.tex}

% Planificación temporal y presupuesto
\chapter{Planificación Temporal y Presupuesto}
\input{capitulos/planificacion_presupuesto/main.tex}

% Bibliografía
\newpage
\addcontentsline{toc}{chapter}{Bibliografía}
\printbibliography

\end{document}


% Metodología
\chapter{Metodología}
\documentclass[a4paper,11pt,twoside]{report}
\usepackage[left=25mm,right=25mm,top=25mm,bottom=25mm,includehead,includefoot,headsep=15mm,footskip=15mm]{geometry}
\usepackage{graphicx}
\usepackage{fancyhdr}
\usepackage{titlesec}
\usepackage[spanish]{babel}
\usepackage[utf8]{inputenc}
\usepackage{amsmath}
\usepackage{setspace}
\usepackage{svg}
\usepackage{hyperref}
\usepackage[backend=biber,style=numeric]{biblatex}
\addbibresource{references.bib}
\hypersetup{
    colorlinks=true,
    linkcolor=blue,      % color of internal links (sections, etc.)
    urlcolor=blue,       % color of external links
    pdftitle={Optimización energética de sistema híbrido con bomba de calor, suelo radiante, fotovoltaica y almacenamiento para vivienda},    % title
    pdfauthor={Luis D. Aranda Sánchez},     % author
    pdfkeywords={palabra1, palabra2, código1, etc.} % list of keywords
}

% Font change to Arial
\usepackage{helvet}
\renewcommand{\familydefault}{\sfdefault}

% Chapter titles in uppercase and larger font
\titleformat{\chapter}[hang]{\large\bfseries}{\thechapter.}{1em}{\MakeUppercase}
\titleformat{\section}[hang]{\bfseries}{\thesection.}{1em}{}
\titleformat{\subsection}[hang]{\bfseries}{\thesubsection.}{1em}{}

% Fancyhdr setup
\setlength{\headheight}{14.30174pt} % Adjust to recommended value, slightly larger for safety
\fancyhf{} % Clear all headers and footers
\fancyhead[LE]{\nouppercase{\leftmark}}
\fancyhead[RO]{Optimización energética para vivienda}
\fancyfoot[LE]{\thepage}
\fancyfoot[RE]{Escuela Técnica Superior de Ingenieros Industriales (UPM)}
\fancyfoot[LO]{Luis D. Aranda Sánchez}
\fancyfoot[RO]{\thepage}
\renewcommand{\headrulewidth}{0.4pt}
\renewcommand{\footrulewidth}{0.4pt}

\fancypagestyle{myfancy}{
    \fancyhf{} % Clear all headers and footers
    \fancyhead[LE]{\nouppercase{\leftmark}}
    \fancyhead[RO]{Optimización energética para vivienda}
    \fancyfoot[LE]{\thepage}
    \fancyfoot[RE]{Escuela Técnica Superior de Ingenieros Industriales (UPM)}
    \fancyfoot[LO]{Luis D. Aranda Sánchez}
    \fancyfoot[RO]{\thepage}
    \renewcommand{\headrulewidth}{0.4pt}
    \renewcommand{\footrulewidth}{0.4pt}
}

\fancypagestyle{simple}{
    \fancyhf{} % Clear all headers and footers
    \renewcommand{\headrulewidth}{0pt}
    \renewcommand{\footrulewidth}{0pt}
}

% Line spacing
\setstretch{1.2}

% Document starts here
\begin{document}

% Portada
\begin{titlepage}
    \centering
    {\scshape\LARGE Universidad Politécnica de Madrid \par}
    \vspace{1cm}
    {\scshape\Large Escuela Técnica Superior de Ingenieros Industriales\par}
    \vspace{1.5cm}
    {\huge\bfseries Optimización energética de sistema híbrido con bomba de calor, suelo radiante, fotovoltaica y almacenamiento para vivienda \par}
    \vspace{1.5cm}
    {\Large\bfseries Trabajo de Fin de Máster\par}
    \vspace{0.5cm}
    {\large Máster Universitario en Ingeniería de la Energía \par}
    \vspace{2cm}
    {\Large Luis D. Aranda Sánchez\par}
    \vfill
    Director: Javier Rodríguez Martín
    \vfill
    {\large Septiembre 6, 2024\par}
\end{titlepage}

% Resumen (máximo de 5 páginas, incluyendo al final Palabras clave)
\clearpage
\pagestyle{simple}
% \newpage
\chapter*{Resumen}
\addcontentsline{toc}{chapter}{Resumen}
\input{capitulos/resumen/main.tex}

% Índice (paginado)
\clearpage
\pagestyle{simple}
% \newpage
\tableofcontents

% Introducción (donde se incluya los antecedentes y justificación)
\clearpage
\pagestyle{myfancy}
\newpage
\chapter{Introducción}
\input{capitulos/introduccion/main.tex}

% Objetivos
\chapter{Objetivos}
\input{capitulos/objetivos/main.tex}

% Metodología
\chapter{Metodología}
\input{capitulos/metodologia/main.tex}

% Resultados y discusión (incluyendo la valoración de impactos y de aspectos de responsabilidad legal, ética y profesional relacionados con el trabajo)
\chapter{Resultados y Discusión}
\input{capitulos/resultados_discusion/main.tex}

% Conclusiones
\chapter{Conclusiones}
\input{capitulos/conclusiones/main.tex}

% Planificación temporal y presupuesto
\chapter{Planificación Temporal y Presupuesto}
\input{capitulos/planificacion_presupuesto/main.tex}

% Bibliografía
\newpage
\addcontentsline{toc}{chapter}{Bibliografía}
\printbibliography

\end{document}


% Resultados y discusión (incluyendo la valoración de impactos y de aspectos de responsabilidad legal, ética y profesional relacionados con el trabajo)
\chapter{Resultados y Discusión}
\documentclass[a4paper,11pt,twoside]{report}
\usepackage[left=25mm,right=25mm,top=25mm,bottom=25mm,includehead,includefoot,headsep=15mm,footskip=15mm]{geometry}
\usepackage{graphicx}
\usepackage{fancyhdr}
\usepackage{titlesec}
\usepackage[spanish]{babel}
\usepackage[utf8]{inputenc}
\usepackage{amsmath}
\usepackage{setspace}
\usepackage{svg}
\usepackage{hyperref}
\usepackage[backend=biber,style=numeric]{biblatex}
\addbibresource{references.bib}
\hypersetup{
    colorlinks=true,
    linkcolor=blue,      % color of internal links (sections, etc.)
    urlcolor=blue,       % color of external links
    pdftitle={Optimización energética de sistema híbrido con bomba de calor, suelo radiante, fotovoltaica y almacenamiento para vivienda},    % title
    pdfauthor={Luis D. Aranda Sánchez},     % author
    pdfkeywords={palabra1, palabra2, código1, etc.} % list of keywords
}

% Font change to Arial
\usepackage{helvet}
\renewcommand{\familydefault}{\sfdefault}

% Chapter titles in uppercase and larger font
\titleformat{\chapter}[hang]{\large\bfseries}{\thechapter.}{1em}{\MakeUppercase}
\titleformat{\section}[hang]{\bfseries}{\thesection.}{1em}{}
\titleformat{\subsection}[hang]{\bfseries}{\thesubsection.}{1em}{}

% Fancyhdr setup
\setlength{\headheight}{14.30174pt} % Adjust to recommended value, slightly larger for safety
\fancyhf{} % Clear all headers and footers
\fancyhead[LE]{\nouppercase{\leftmark}}
\fancyhead[RO]{Optimización energética para vivienda}
\fancyfoot[LE]{\thepage}
\fancyfoot[RE]{Escuela Técnica Superior de Ingenieros Industriales (UPM)}
\fancyfoot[LO]{Luis D. Aranda Sánchez}
\fancyfoot[RO]{\thepage}
\renewcommand{\headrulewidth}{0.4pt}
\renewcommand{\footrulewidth}{0.4pt}

\fancypagestyle{myfancy}{
    \fancyhf{} % Clear all headers and footers
    \fancyhead[LE]{\nouppercase{\leftmark}}
    \fancyhead[RO]{Optimización energética para vivienda}
    \fancyfoot[LE]{\thepage}
    \fancyfoot[RE]{Escuela Técnica Superior de Ingenieros Industriales (UPM)}
    \fancyfoot[LO]{Luis D. Aranda Sánchez}
    \fancyfoot[RO]{\thepage}
    \renewcommand{\headrulewidth}{0.4pt}
    \renewcommand{\footrulewidth}{0.4pt}
}

\fancypagestyle{simple}{
    \fancyhf{} % Clear all headers and footers
    \renewcommand{\headrulewidth}{0pt}
    \renewcommand{\footrulewidth}{0pt}
}

% Line spacing
\setstretch{1.2}

% Document starts here
\begin{document}

% Portada
\begin{titlepage}
    \centering
    {\scshape\LARGE Universidad Politécnica de Madrid \par}
    \vspace{1cm}
    {\scshape\Large Escuela Técnica Superior de Ingenieros Industriales\par}
    \vspace{1.5cm}
    {\huge\bfseries Optimización energética de sistema híbrido con bomba de calor, suelo radiante, fotovoltaica y almacenamiento para vivienda \par}
    \vspace{1.5cm}
    {\Large\bfseries Trabajo de Fin de Máster\par}
    \vspace{0.5cm}
    {\large Máster Universitario en Ingeniería de la Energía \par}
    \vspace{2cm}
    {\Large Luis D. Aranda Sánchez\par}
    \vfill
    Director: Javier Rodríguez Martín
    \vfill
    {\large Septiembre 6, 2024\par}
\end{titlepage}

% Resumen (máximo de 5 páginas, incluyendo al final Palabras clave)
\clearpage
\pagestyle{simple}
% \newpage
\chapter*{Resumen}
\addcontentsline{toc}{chapter}{Resumen}
\input{capitulos/resumen/main.tex}

% Índice (paginado)
\clearpage
\pagestyle{simple}
% \newpage
\tableofcontents

% Introducción (donde se incluya los antecedentes y justificación)
\clearpage
\pagestyle{myfancy}
\newpage
\chapter{Introducción}
\input{capitulos/introduccion/main.tex}

% Objetivos
\chapter{Objetivos}
\input{capitulos/objetivos/main.tex}

% Metodología
\chapter{Metodología}
\input{capitulos/metodologia/main.tex}

% Resultados y discusión (incluyendo la valoración de impactos y de aspectos de responsabilidad legal, ética y profesional relacionados con el trabajo)
\chapter{Resultados y Discusión}
\input{capitulos/resultados_discusion/main.tex}

% Conclusiones
\chapter{Conclusiones}
\input{capitulos/conclusiones/main.tex}

% Planificación temporal y presupuesto
\chapter{Planificación Temporal y Presupuesto}
\input{capitulos/planificacion_presupuesto/main.tex}

% Bibliografía
\newpage
\addcontentsline{toc}{chapter}{Bibliografía}
\printbibliography

\end{document}


% Conclusiones
\chapter{Conclusiones}
\documentclass[a4paper,11pt,twoside]{report}
\usepackage[left=25mm,right=25mm,top=25mm,bottom=25mm,includehead,includefoot,headsep=15mm,footskip=15mm]{geometry}
\usepackage{graphicx}
\usepackage{fancyhdr}
\usepackage{titlesec}
\usepackage[spanish]{babel}
\usepackage[utf8]{inputenc}
\usepackage{amsmath}
\usepackage{setspace}
\usepackage{svg}
\usepackage{hyperref}
\usepackage[backend=biber,style=numeric]{biblatex}
\addbibresource{references.bib}
\hypersetup{
    colorlinks=true,
    linkcolor=blue,      % color of internal links (sections, etc.)
    urlcolor=blue,       % color of external links
    pdftitle={Optimización energética de sistema híbrido con bomba de calor, suelo radiante, fotovoltaica y almacenamiento para vivienda},    % title
    pdfauthor={Luis D. Aranda Sánchez},     % author
    pdfkeywords={palabra1, palabra2, código1, etc.} % list of keywords
}

% Font change to Arial
\usepackage{helvet}
\renewcommand{\familydefault}{\sfdefault}

% Chapter titles in uppercase and larger font
\titleformat{\chapter}[hang]{\large\bfseries}{\thechapter.}{1em}{\MakeUppercase}
\titleformat{\section}[hang]{\bfseries}{\thesection.}{1em}{}
\titleformat{\subsection}[hang]{\bfseries}{\thesubsection.}{1em}{}

% Fancyhdr setup
\setlength{\headheight}{14.30174pt} % Adjust to recommended value, slightly larger for safety
\fancyhf{} % Clear all headers and footers
\fancyhead[LE]{\nouppercase{\leftmark}}
\fancyhead[RO]{Optimización energética para vivienda}
\fancyfoot[LE]{\thepage}
\fancyfoot[RE]{Escuela Técnica Superior de Ingenieros Industriales (UPM)}
\fancyfoot[LO]{Luis D. Aranda Sánchez}
\fancyfoot[RO]{\thepage}
\renewcommand{\headrulewidth}{0.4pt}
\renewcommand{\footrulewidth}{0.4pt}

\fancypagestyle{myfancy}{
    \fancyhf{} % Clear all headers and footers
    \fancyhead[LE]{\nouppercase{\leftmark}}
    \fancyhead[RO]{Optimización energética para vivienda}
    \fancyfoot[LE]{\thepage}
    \fancyfoot[RE]{Escuela Técnica Superior de Ingenieros Industriales (UPM)}
    \fancyfoot[LO]{Luis D. Aranda Sánchez}
    \fancyfoot[RO]{\thepage}
    \renewcommand{\headrulewidth}{0.4pt}
    \renewcommand{\footrulewidth}{0.4pt}
}

\fancypagestyle{simple}{
    \fancyhf{} % Clear all headers and footers
    \renewcommand{\headrulewidth}{0pt}
    \renewcommand{\footrulewidth}{0pt}
}

% Line spacing
\setstretch{1.2}

% Document starts here
\begin{document}

% Portada
\begin{titlepage}
    \centering
    {\scshape\LARGE Universidad Politécnica de Madrid \par}
    \vspace{1cm}
    {\scshape\Large Escuela Técnica Superior de Ingenieros Industriales\par}
    \vspace{1.5cm}
    {\huge\bfseries Optimización energética de sistema híbrido con bomba de calor, suelo radiante, fotovoltaica y almacenamiento para vivienda \par}
    \vspace{1.5cm}
    {\Large\bfseries Trabajo de Fin de Máster\par}
    \vspace{0.5cm}
    {\large Máster Universitario en Ingeniería de la Energía \par}
    \vspace{2cm}
    {\Large Luis D. Aranda Sánchez\par}
    \vfill
    Director: Javier Rodríguez Martín
    \vfill
    {\large Septiembre 6, 2024\par}
\end{titlepage}

% Resumen (máximo de 5 páginas, incluyendo al final Palabras clave)
\clearpage
\pagestyle{simple}
% \newpage
\chapter*{Resumen}
\addcontentsline{toc}{chapter}{Resumen}
\input{capitulos/resumen/main.tex}

% Índice (paginado)
\clearpage
\pagestyle{simple}
% \newpage
\tableofcontents

% Introducción (donde se incluya los antecedentes y justificación)
\clearpage
\pagestyle{myfancy}
\newpage
\chapter{Introducción}
\input{capitulos/introduccion/main.tex}

% Objetivos
\chapter{Objetivos}
\input{capitulos/objetivos/main.tex}

% Metodología
\chapter{Metodología}
\input{capitulos/metodologia/main.tex}

% Resultados y discusión (incluyendo la valoración de impactos y de aspectos de responsabilidad legal, ética y profesional relacionados con el trabajo)
\chapter{Resultados y Discusión}
\input{capitulos/resultados_discusion/main.tex}

% Conclusiones
\chapter{Conclusiones}
\input{capitulos/conclusiones/main.tex}

% Planificación temporal y presupuesto
\chapter{Planificación Temporal y Presupuesto}
\input{capitulos/planificacion_presupuesto/main.tex}

% Bibliografía
\newpage
\addcontentsline{toc}{chapter}{Bibliografía}
\printbibliography

\end{document}


% Planificación temporal y presupuesto
\chapter{Planificación Temporal y Presupuesto}
\documentclass[a4paper,11pt,twoside]{report}
\usepackage[left=25mm,right=25mm,top=25mm,bottom=25mm,includehead,includefoot,headsep=15mm,footskip=15mm]{geometry}
\usepackage{graphicx}
\usepackage{fancyhdr}
\usepackage{titlesec}
\usepackage[spanish]{babel}
\usepackage[utf8]{inputenc}
\usepackage{amsmath}
\usepackage{setspace}
\usepackage{svg}
\usepackage{hyperref}
\usepackage[backend=biber,style=numeric]{biblatex}
\addbibresource{references.bib}
\hypersetup{
    colorlinks=true,
    linkcolor=blue,      % color of internal links (sections, etc.)
    urlcolor=blue,       % color of external links
    pdftitle={Optimización energética de sistema híbrido con bomba de calor, suelo radiante, fotovoltaica y almacenamiento para vivienda},    % title
    pdfauthor={Luis D. Aranda Sánchez},     % author
    pdfkeywords={palabra1, palabra2, código1, etc.} % list of keywords
}

% Font change to Arial
\usepackage{helvet}
\renewcommand{\familydefault}{\sfdefault}

% Chapter titles in uppercase and larger font
\titleformat{\chapter}[hang]{\large\bfseries}{\thechapter.}{1em}{\MakeUppercase}
\titleformat{\section}[hang]{\bfseries}{\thesection.}{1em}{}
\titleformat{\subsection}[hang]{\bfseries}{\thesubsection.}{1em}{}

% Fancyhdr setup
\setlength{\headheight}{14.30174pt} % Adjust to recommended value, slightly larger for safety
\fancyhf{} % Clear all headers and footers
\fancyhead[LE]{\nouppercase{\leftmark}}
\fancyhead[RO]{Optimización energética para vivienda}
\fancyfoot[LE]{\thepage}
\fancyfoot[RE]{Escuela Técnica Superior de Ingenieros Industriales (UPM)}
\fancyfoot[LO]{Luis D. Aranda Sánchez}
\fancyfoot[RO]{\thepage}
\renewcommand{\headrulewidth}{0.4pt}
\renewcommand{\footrulewidth}{0.4pt}

\fancypagestyle{myfancy}{
    \fancyhf{} % Clear all headers and footers
    \fancyhead[LE]{\nouppercase{\leftmark}}
    \fancyhead[RO]{Optimización energética para vivienda}
    \fancyfoot[LE]{\thepage}
    \fancyfoot[RE]{Escuela Técnica Superior de Ingenieros Industriales (UPM)}
    \fancyfoot[LO]{Luis D. Aranda Sánchez}
    \fancyfoot[RO]{\thepage}
    \renewcommand{\headrulewidth}{0.4pt}
    \renewcommand{\footrulewidth}{0.4pt}
}

\fancypagestyle{simple}{
    \fancyhf{} % Clear all headers and footers
    \renewcommand{\headrulewidth}{0pt}
    \renewcommand{\footrulewidth}{0pt}
}

% Line spacing
\setstretch{1.2}

% Document starts here
\begin{document}

% Portada
\begin{titlepage}
    \centering
    {\scshape\LARGE Universidad Politécnica de Madrid \par}
    \vspace{1cm}
    {\scshape\Large Escuela Técnica Superior de Ingenieros Industriales\par}
    \vspace{1.5cm}
    {\huge\bfseries Optimización energética de sistema híbrido con bomba de calor, suelo radiante, fotovoltaica y almacenamiento para vivienda \par}
    \vspace{1.5cm}
    {\Large\bfseries Trabajo de Fin de Máster\par}
    \vspace{0.5cm}
    {\large Máster Universitario en Ingeniería de la Energía \par}
    \vspace{2cm}
    {\Large Luis D. Aranda Sánchez\par}
    \vfill
    Director: Javier Rodríguez Martín
    \vfill
    {\large Septiembre 6, 2024\par}
\end{titlepage}

% Resumen (máximo de 5 páginas, incluyendo al final Palabras clave)
\clearpage
\pagestyle{simple}
% \newpage
\chapter*{Resumen}
\addcontentsline{toc}{chapter}{Resumen}
\input{capitulos/resumen/main.tex}

% Índice (paginado)
\clearpage
\pagestyle{simple}
% \newpage
\tableofcontents

% Introducción (donde se incluya los antecedentes y justificación)
\clearpage
\pagestyle{myfancy}
\newpage
\chapter{Introducción}
\input{capitulos/introduccion/main.tex}

% Objetivos
\chapter{Objetivos}
\input{capitulos/objetivos/main.tex}

% Metodología
\chapter{Metodología}
\input{capitulos/metodologia/main.tex}

% Resultados y discusión (incluyendo la valoración de impactos y de aspectos de responsabilidad legal, ética y profesional relacionados con el trabajo)
\chapter{Resultados y Discusión}
\input{capitulos/resultados_discusion/main.tex}

% Conclusiones
\chapter{Conclusiones}
\input{capitulos/conclusiones/main.tex}

% Planificación temporal y presupuesto
\chapter{Planificación Temporal y Presupuesto}
\input{capitulos/planificacion_presupuesto/main.tex}

% Bibliografía
\newpage
\addcontentsline{toc}{chapter}{Bibliografía}
\printbibliography

\end{document}


% Bibliografía
\newpage
\addcontentsline{toc}{chapter}{Bibliografía}
\printbibliography

\end{document}


% Conclusiones
\chapter{Conclusiones}
\documentclass[a4paper,11pt,twoside]{report}
\usepackage[left=25mm,right=25mm,top=25mm,bottom=25mm,includehead,includefoot,headsep=15mm,footskip=15mm]{geometry}
\usepackage{graphicx}
\usepackage{fancyhdr}
\usepackage{titlesec}
\usepackage[spanish]{babel}
\usepackage[utf8]{inputenc}
\usepackage{amsmath}
\usepackage{setspace}
\usepackage{svg}
\usepackage{hyperref}
\usepackage[backend=biber,style=numeric]{biblatex}
\addbibresource{references.bib}
\hypersetup{
    colorlinks=true,
    linkcolor=blue,      % color of internal links (sections, etc.)
    urlcolor=blue,       % color of external links
    pdftitle={Optimización energética de sistema híbrido con bomba de calor, suelo radiante, fotovoltaica y almacenamiento para vivienda},    % title
    pdfauthor={Luis D. Aranda Sánchez},     % author
    pdfkeywords={palabra1, palabra2, código1, etc.} % list of keywords
}

% Font change to Arial
\usepackage{helvet}
\renewcommand{\familydefault}{\sfdefault}

% Chapter titles in uppercase and larger font
\titleformat{\chapter}[hang]{\large\bfseries}{\thechapter.}{1em}{\MakeUppercase}
\titleformat{\section}[hang]{\bfseries}{\thesection.}{1em}{}
\titleformat{\subsection}[hang]{\bfseries}{\thesubsection.}{1em}{}

% Fancyhdr setup
\setlength{\headheight}{14.30174pt} % Adjust to recommended value, slightly larger for safety
\fancyhf{} % Clear all headers and footers
\fancyhead[LE]{\nouppercase{\leftmark}}
\fancyhead[RO]{Optimización energética para vivienda}
\fancyfoot[LE]{\thepage}
\fancyfoot[RE]{Escuela Técnica Superior de Ingenieros Industriales (UPM)}
\fancyfoot[LO]{Luis D. Aranda Sánchez}
\fancyfoot[RO]{\thepage}
\renewcommand{\headrulewidth}{0.4pt}
\renewcommand{\footrulewidth}{0.4pt}

\fancypagestyle{myfancy}{
    \fancyhf{} % Clear all headers and footers
    \fancyhead[LE]{\nouppercase{\leftmark}}
    \fancyhead[RO]{Optimización energética para vivienda}
    \fancyfoot[LE]{\thepage}
    \fancyfoot[RE]{Escuela Técnica Superior de Ingenieros Industriales (UPM)}
    \fancyfoot[LO]{Luis D. Aranda Sánchez}
    \fancyfoot[RO]{\thepage}
    \renewcommand{\headrulewidth}{0.4pt}
    \renewcommand{\footrulewidth}{0.4pt}
}

\fancypagestyle{simple}{
    \fancyhf{} % Clear all headers and footers
    \renewcommand{\headrulewidth}{0pt}
    \renewcommand{\footrulewidth}{0pt}
}

% Line spacing
\setstretch{1.2}

% Document starts here
\begin{document}

% Portada
\begin{titlepage}
    \centering
    {\scshape\LARGE Universidad Politécnica de Madrid \par}
    \vspace{1cm}
    {\scshape\Large Escuela Técnica Superior de Ingenieros Industriales\par}
    \vspace{1.5cm}
    {\huge\bfseries Optimización energética de sistema híbrido con bomba de calor, suelo radiante, fotovoltaica y almacenamiento para vivienda \par}
    \vspace{1.5cm}
    {\Large\bfseries Trabajo de Fin de Máster\par}
    \vspace{0.5cm}
    {\large Máster Universitario en Ingeniería de la Energía \par}
    \vspace{2cm}
    {\Large Luis D. Aranda Sánchez\par}
    \vfill
    Director: Javier Rodríguez Martín
    \vfill
    {\large Septiembre 6, 2024\par}
\end{titlepage}

% Resumen (máximo de 5 páginas, incluyendo al final Palabras clave)
\clearpage
\pagestyle{simple}
% \newpage
\chapter*{Resumen}
\addcontentsline{toc}{chapter}{Resumen}
\documentclass[a4paper,11pt,twoside]{report}
\usepackage[left=25mm,right=25mm,top=25mm,bottom=25mm,includehead,includefoot,headsep=15mm,footskip=15mm]{geometry}
\usepackage{graphicx}
\usepackage{fancyhdr}
\usepackage{titlesec}
\usepackage[spanish]{babel}
\usepackage[utf8]{inputenc}
\usepackage{amsmath}
\usepackage{setspace}
\usepackage{svg}
\usepackage{hyperref}
\usepackage[backend=biber,style=numeric]{biblatex}
\addbibresource{references.bib}
\hypersetup{
    colorlinks=true,
    linkcolor=blue,      % color of internal links (sections, etc.)
    urlcolor=blue,       % color of external links
    pdftitle={Optimización energética de sistema híbrido con bomba de calor, suelo radiante, fotovoltaica y almacenamiento para vivienda},    % title
    pdfauthor={Luis D. Aranda Sánchez},     % author
    pdfkeywords={palabra1, palabra2, código1, etc.} % list of keywords
}

% Font change to Arial
\usepackage{helvet}
\renewcommand{\familydefault}{\sfdefault}

% Chapter titles in uppercase and larger font
\titleformat{\chapter}[hang]{\large\bfseries}{\thechapter.}{1em}{\MakeUppercase}
\titleformat{\section}[hang]{\bfseries}{\thesection.}{1em}{}
\titleformat{\subsection}[hang]{\bfseries}{\thesubsection.}{1em}{}

% Fancyhdr setup
\setlength{\headheight}{14.30174pt} % Adjust to recommended value, slightly larger for safety
\fancyhf{} % Clear all headers and footers
\fancyhead[LE]{\nouppercase{\leftmark}}
\fancyhead[RO]{Optimización energética para vivienda}
\fancyfoot[LE]{\thepage}
\fancyfoot[RE]{Escuela Técnica Superior de Ingenieros Industriales (UPM)}
\fancyfoot[LO]{Luis D. Aranda Sánchez}
\fancyfoot[RO]{\thepage}
\renewcommand{\headrulewidth}{0.4pt}
\renewcommand{\footrulewidth}{0.4pt}

\fancypagestyle{myfancy}{
    \fancyhf{} % Clear all headers and footers
    \fancyhead[LE]{\nouppercase{\leftmark}}
    \fancyhead[RO]{Optimización energética para vivienda}
    \fancyfoot[LE]{\thepage}
    \fancyfoot[RE]{Escuela Técnica Superior de Ingenieros Industriales (UPM)}
    \fancyfoot[LO]{Luis D. Aranda Sánchez}
    \fancyfoot[RO]{\thepage}
    \renewcommand{\headrulewidth}{0.4pt}
    \renewcommand{\footrulewidth}{0.4pt}
}

\fancypagestyle{simple}{
    \fancyhf{} % Clear all headers and footers
    \renewcommand{\headrulewidth}{0pt}
    \renewcommand{\footrulewidth}{0pt}
}

% Line spacing
\setstretch{1.2}

% Document starts here
\begin{document}

% Portada
\begin{titlepage}
    \centering
    {\scshape\LARGE Universidad Politécnica de Madrid \par}
    \vspace{1cm}
    {\scshape\Large Escuela Técnica Superior de Ingenieros Industriales\par}
    \vspace{1.5cm}
    {\huge\bfseries Optimización energética de sistema híbrido con bomba de calor, suelo radiante, fotovoltaica y almacenamiento para vivienda \par}
    \vspace{1.5cm}
    {\Large\bfseries Trabajo de Fin de Máster\par}
    \vspace{0.5cm}
    {\large Máster Universitario en Ingeniería de la Energía \par}
    \vspace{2cm}
    {\Large Luis D. Aranda Sánchez\par}
    \vfill
    Director: Javier Rodríguez Martín
    \vfill
    {\large Septiembre 6, 2024\par}
\end{titlepage}

% Resumen (máximo de 5 páginas, incluyendo al final Palabras clave)
\clearpage
\pagestyle{simple}
% \newpage
\chapter*{Resumen}
\addcontentsline{toc}{chapter}{Resumen}
\input{capitulos/resumen/main.tex}

% Índice (paginado)
\clearpage
\pagestyle{simple}
% \newpage
\tableofcontents

% Introducción (donde se incluya los antecedentes y justificación)
\clearpage
\pagestyle{myfancy}
\newpage
\chapter{Introducción}
\input{capitulos/introduccion/main.tex}

% Objetivos
\chapter{Objetivos}
\input{capitulos/objetivos/main.tex}

% Metodología
\chapter{Metodología}
\input{capitulos/metodologia/main.tex}

% Resultados y discusión (incluyendo la valoración de impactos y de aspectos de responsabilidad legal, ética y profesional relacionados con el trabajo)
\chapter{Resultados y Discusión}
\input{capitulos/resultados_discusion/main.tex}

% Conclusiones
\chapter{Conclusiones}
\input{capitulos/conclusiones/main.tex}

% Planificación temporal y presupuesto
\chapter{Planificación Temporal y Presupuesto}
\input{capitulos/planificacion_presupuesto/main.tex}

% Bibliografía
\newpage
\addcontentsline{toc}{chapter}{Bibliografía}
\printbibliography

\end{document}


% Índice (paginado)
\clearpage
\pagestyle{simple}
% \newpage
\tableofcontents

% Introducción (donde se incluya los antecedentes y justificación)
\clearpage
\pagestyle{myfancy}
\newpage
\chapter{Introducción}
\documentclass[a4paper,11pt,twoside]{report}
\usepackage[left=25mm,right=25mm,top=25mm,bottom=25mm,includehead,includefoot,headsep=15mm,footskip=15mm]{geometry}
\usepackage{graphicx}
\usepackage{fancyhdr}
\usepackage{titlesec}
\usepackage[spanish]{babel}
\usepackage[utf8]{inputenc}
\usepackage{amsmath}
\usepackage{setspace}
\usepackage{svg}
\usepackage{hyperref}
\usepackage[backend=biber,style=numeric]{biblatex}
\addbibresource{references.bib}
\hypersetup{
    colorlinks=true,
    linkcolor=blue,      % color of internal links (sections, etc.)
    urlcolor=blue,       % color of external links
    pdftitle={Optimización energética de sistema híbrido con bomba de calor, suelo radiante, fotovoltaica y almacenamiento para vivienda},    % title
    pdfauthor={Luis D. Aranda Sánchez},     % author
    pdfkeywords={palabra1, palabra2, código1, etc.} % list of keywords
}

% Font change to Arial
\usepackage{helvet}
\renewcommand{\familydefault}{\sfdefault}

% Chapter titles in uppercase and larger font
\titleformat{\chapter}[hang]{\large\bfseries}{\thechapter.}{1em}{\MakeUppercase}
\titleformat{\section}[hang]{\bfseries}{\thesection.}{1em}{}
\titleformat{\subsection}[hang]{\bfseries}{\thesubsection.}{1em}{}

% Fancyhdr setup
\setlength{\headheight}{14.30174pt} % Adjust to recommended value, slightly larger for safety
\fancyhf{} % Clear all headers and footers
\fancyhead[LE]{\nouppercase{\leftmark}}
\fancyhead[RO]{Optimización energética para vivienda}
\fancyfoot[LE]{\thepage}
\fancyfoot[RE]{Escuela Técnica Superior de Ingenieros Industriales (UPM)}
\fancyfoot[LO]{Luis D. Aranda Sánchez}
\fancyfoot[RO]{\thepage}
\renewcommand{\headrulewidth}{0.4pt}
\renewcommand{\footrulewidth}{0.4pt}

\fancypagestyle{myfancy}{
    \fancyhf{} % Clear all headers and footers
    \fancyhead[LE]{\nouppercase{\leftmark}}
    \fancyhead[RO]{Optimización energética para vivienda}
    \fancyfoot[LE]{\thepage}
    \fancyfoot[RE]{Escuela Técnica Superior de Ingenieros Industriales (UPM)}
    \fancyfoot[LO]{Luis D. Aranda Sánchez}
    \fancyfoot[RO]{\thepage}
    \renewcommand{\headrulewidth}{0.4pt}
    \renewcommand{\footrulewidth}{0.4pt}
}

\fancypagestyle{simple}{
    \fancyhf{} % Clear all headers and footers
    \renewcommand{\headrulewidth}{0pt}
    \renewcommand{\footrulewidth}{0pt}
}

% Line spacing
\setstretch{1.2}

% Document starts here
\begin{document}

% Portada
\begin{titlepage}
    \centering
    {\scshape\LARGE Universidad Politécnica de Madrid \par}
    \vspace{1cm}
    {\scshape\Large Escuela Técnica Superior de Ingenieros Industriales\par}
    \vspace{1.5cm}
    {\huge\bfseries Optimización energética de sistema híbrido con bomba de calor, suelo radiante, fotovoltaica y almacenamiento para vivienda \par}
    \vspace{1.5cm}
    {\Large\bfseries Trabajo de Fin de Máster\par}
    \vspace{0.5cm}
    {\large Máster Universitario en Ingeniería de la Energía \par}
    \vspace{2cm}
    {\Large Luis D. Aranda Sánchez\par}
    \vfill
    Director: Javier Rodríguez Martín
    \vfill
    {\large Septiembre 6, 2024\par}
\end{titlepage}

% Resumen (máximo de 5 páginas, incluyendo al final Palabras clave)
\clearpage
\pagestyle{simple}
% \newpage
\chapter*{Resumen}
\addcontentsline{toc}{chapter}{Resumen}
\input{capitulos/resumen/main.tex}

% Índice (paginado)
\clearpage
\pagestyle{simple}
% \newpage
\tableofcontents

% Introducción (donde se incluya los antecedentes y justificación)
\clearpage
\pagestyle{myfancy}
\newpage
\chapter{Introducción}
\input{capitulos/introduccion/main.tex}

% Objetivos
\chapter{Objetivos}
\input{capitulos/objetivos/main.tex}

% Metodología
\chapter{Metodología}
\input{capitulos/metodologia/main.tex}

% Resultados y discusión (incluyendo la valoración de impactos y de aspectos de responsabilidad legal, ética y profesional relacionados con el trabajo)
\chapter{Resultados y Discusión}
\input{capitulos/resultados_discusion/main.tex}

% Conclusiones
\chapter{Conclusiones}
\input{capitulos/conclusiones/main.tex}

% Planificación temporal y presupuesto
\chapter{Planificación Temporal y Presupuesto}
\input{capitulos/planificacion_presupuesto/main.tex}

% Bibliografía
\newpage
\addcontentsline{toc}{chapter}{Bibliografía}
\printbibliography

\end{document}


% Objetivos
\chapter{Objetivos}
\documentclass[a4paper,11pt,twoside]{report}
\usepackage[left=25mm,right=25mm,top=25mm,bottom=25mm,includehead,includefoot,headsep=15mm,footskip=15mm]{geometry}
\usepackage{graphicx}
\usepackage{fancyhdr}
\usepackage{titlesec}
\usepackage[spanish]{babel}
\usepackage[utf8]{inputenc}
\usepackage{amsmath}
\usepackage{setspace}
\usepackage{svg}
\usepackage{hyperref}
\usepackage[backend=biber,style=numeric]{biblatex}
\addbibresource{references.bib}
\hypersetup{
    colorlinks=true,
    linkcolor=blue,      % color of internal links (sections, etc.)
    urlcolor=blue,       % color of external links
    pdftitle={Optimización energética de sistema híbrido con bomba de calor, suelo radiante, fotovoltaica y almacenamiento para vivienda},    % title
    pdfauthor={Luis D. Aranda Sánchez},     % author
    pdfkeywords={palabra1, palabra2, código1, etc.} % list of keywords
}

% Font change to Arial
\usepackage{helvet}
\renewcommand{\familydefault}{\sfdefault}

% Chapter titles in uppercase and larger font
\titleformat{\chapter}[hang]{\large\bfseries}{\thechapter.}{1em}{\MakeUppercase}
\titleformat{\section}[hang]{\bfseries}{\thesection.}{1em}{}
\titleformat{\subsection}[hang]{\bfseries}{\thesubsection.}{1em}{}

% Fancyhdr setup
\setlength{\headheight}{14.30174pt} % Adjust to recommended value, slightly larger for safety
\fancyhf{} % Clear all headers and footers
\fancyhead[LE]{\nouppercase{\leftmark}}
\fancyhead[RO]{Optimización energética para vivienda}
\fancyfoot[LE]{\thepage}
\fancyfoot[RE]{Escuela Técnica Superior de Ingenieros Industriales (UPM)}
\fancyfoot[LO]{Luis D. Aranda Sánchez}
\fancyfoot[RO]{\thepage}
\renewcommand{\headrulewidth}{0.4pt}
\renewcommand{\footrulewidth}{0.4pt}

\fancypagestyle{myfancy}{
    \fancyhf{} % Clear all headers and footers
    \fancyhead[LE]{\nouppercase{\leftmark}}
    \fancyhead[RO]{Optimización energética para vivienda}
    \fancyfoot[LE]{\thepage}
    \fancyfoot[RE]{Escuela Técnica Superior de Ingenieros Industriales (UPM)}
    \fancyfoot[LO]{Luis D. Aranda Sánchez}
    \fancyfoot[RO]{\thepage}
    \renewcommand{\headrulewidth}{0.4pt}
    \renewcommand{\footrulewidth}{0.4pt}
}

\fancypagestyle{simple}{
    \fancyhf{} % Clear all headers and footers
    \renewcommand{\headrulewidth}{0pt}
    \renewcommand{\footrulewidth}{0pt}
}

% Line spacing
\setstretch{1.2}

% Document starts here
\begin{document}

% Portada
\begin{titlepage}
    \centering
    {\scshape\LARGE Universidad Politécnica de Madrid \par}
    \vspace{1cm}
    {\scshape\Large Escuela Técnica Superior de Ingenieros Industriales\par}
    \vspace{1.5cm}
    {\huge\bfseries Optimización energética de sistema híbrido con bomba de calor, suelo radiante, fotovoltaica y almacenamiento para vivienda \par}
    \vspace{1.5cm}
    {\Large\bfseries Trabajo de Fin de Máster\par}
    \vspace{0.5cm}
    {\large Máster Universitario en Ingeniería de la Energía \par}
    \vspace{2cm}
    {\Large Luis D. Aranda Sánchez\par}
    \vfill
    Director: Javier Rodríguez Martín
    \vfill
    {\large Septiembre 6, 2024\par}
\end{titlepage}

% Resumen (máximo de 5 páginas, incluyendo al final Palabras clave)
\clearpage
\pagestyle{simple}
% \newpage
\chapter*{Resumen}
\addcontentsline{toc}{chapter}{Resumen}
\input{capitulos/resumen/main.tex}

% Índice (paginado)
\clearpage
\pagestyle{simple}
% \newpage
\tableofcontents

% Introducción (donde se incluya los antecedentes y justificación)
\clearpage
\pagestyle{myfancy}
\newpage
\chapter{Introducción}
\input{capitulos/introduccion/main.tex}

% Objetivos
\chapter{Objetivos}
\input{capitulos/objetivos/main.tex}

% Metodología
\chapter{Metodología}
\input{capitulos/metodologia/main.tex}

% Resultados y discusión (incluyendo la valoración de impactos y de aspectos de responsabilidad legal, ética y profesional relacionados con el trabajo)
\chapter{Resultados y Discusión}
\input{capitulos/resultados_discusion/main.tex}

% Conclusiones
\chapter{Conclusiones}
\input{capitulos/conclusiones/main.tex}

% Planificación temporal y presupuesto
\chapter{Planificación Temporal y Presupuesto}
\input{capitulos/planificacion_presupuesto/main.tex}

% Bibliografía
\newpage
\addcontentsline{toc}{chapter}{Bibliografía}
\printbibliography

\end{document}


% Metodología
\chapter{Metodología}
\documentclass[a4paper,11pt,twoside]{report}
\usepackage[left=25mm,right=25mm,top=25mm,bottom=25mm,includehead,includefoot,headsep=15mm,footskip=15mm]{geometry}
\usepackage{graphicx}
\usepackage{fancyhdr}
\usepackage{titlesec}
\usepackage[spanish]{babel}
\usepackage[utf8]{inputenc}
\usepackage{amsmath}
\usepackage{setspace}
\usepackage{svg}
\usepackage{hyperref}
\usepackage[backend=biber,style=numeric]{biblatex}
\addbibresource{references.bib}
\hypersetup{
    colorlinks=true,
    linkcolor=blue,      % color of internal links (sections, etc.)
    urlcolor=blue,       % color of external links
    pdftitle={Optimización energética de sistema híbrido con bomba de calor, suelo radiante, fotovoltaica y almacenamiento para vivienda},    % title
    pdfauthor={Luis D. Aranda Sánchez},     % author
    pdfkeywords={palabra1, palabra2, código1, etc.} % list of keywords
}

% Font change to Arial
\usepackage{helvet}
\renewcommand{\familydefault}{\sfdefault}

% Chapter titles in uppercase and larger font
\titleformat{\chapter}[hang]{\large\bfseries}{\thechapter.}{1em}{\MakeUppercase}
\titleformat{\section}[hang]{\bfseries}{\thesection.}{1em}{}
\titleformat{\subsection}[hang]{\bfseries}{\thesubsection.}{1em}{}

% Fancyhdr setup
\setlength{\headheight}{14.30174pt} % Adjust to recommended value, slightly larger for safety
\fancyhf{} % Clear all headers and footers
\fancyhead[LE]{\nouppercase{\leftmark}}
\fancyhead[RO]{Optimización energética para vivienda}
\fancyfoot[LE]{\thepage}
\fancyfoot[RE]{Escuela Técnica Superior de Ingenieros Industriales (UPM)}
\fancyfoot[LO]{Luis D. Aranda Sánchez}
\fancyfoot[RO]{\thepage}
\renewcommand{\headrulewidth}{0.4pt}
\renewcommand{\footrulewidth}{0.4pt}

\fancypagestyle{myfancy}{
    \fancyhf{} % Clear all headers and footers
    \fancyhead[LE]{\nouppercase{\leftmark}}
    \fancyhead[RO]{Optimización energética para vivienda}
    \fancyfoot[LE]{\thepage}
    \fancyfoot[RE]{Escuela Técnica Superior de Ingenieros Industriales (UPM)}
    \fancyfoot[LO]{Luis D. Aranda Sánchez}
    \fancyfoot[RO]{\thepage}
    \renewcommand{\headrulewidth}{0.4pt}
    \renewcommand{\footrulewidth}{0.4pt}
}

\fancypagestyle{simple}{
    \fancyhf{} % Clear all headers and footers
    \renewcommand{\headrulewidth}{0pt}
    \renewcommand{\footrulewidth}{0pt}
}

% Line spacing
\setstretch{1.2}

% Document starts here
\begin{document}

% Portada
\begin{titlepage}
    \centering
    {\scshape\LARGE Universidad Politécnica de Madrid \par}
    \vspace{1cm}
    {\scshape\Large Escuela Técnica Superior de Ingenieros Industriales\par}
    \vspace{1.5cm}
    {\huge\bfseries Optimización energética de sistema híbrido con bomba de calor, suelo radiante, fotovoltaica y almacenamiento para vivienda \par}
    \vspace{1.5cm}
    {\Large\bfseries Trabajo de Fin de Máster\par}
    \vspace{0.5cm}
    {\large Máster Universitario en Ingeniería de la Energía \par}
    \vspace{2cm}
    {\Large Luis D. Aranda Sánchez\par}
    \vfill
    Director: Javier Rodríguez Martín
    \vfill
    {\large Septiembre 6, 2024\par}
\end{titlepage}

% Resumen (máximo de 5 páginas, incluyendo al final Palabras clave)
\clearpage
\pagestyle{simple}
% \newpage
\chapter*{Resumen}
\addcontentsline{toc}{chapter}{Resumen}
\input{capitulos/resumen/main.tex}

% Índice (paginado)
\clearpage
\pagestyle{simple}
% \newpage
\tableofcontents

% Introducción (donde se incluya los antecedentes y justificación)
\clearpage
\pagestyle{myfancy}
\newpage
\chapter{Introducción}
\input{capitulos/introduccion/main.tex}

% Objetivos
\chapter{Objetivos}
\input{capitulos/objetivos/main.tex}

% Metodología
\chapter{Metodología}
\input{capitulos/metodologia/main.tex}

% Resultados y discusión (incluyendo la valoración de impactos y de aspectos de responsabilidad legal, ética y profesional relacionados con el trabajo)
\chapter{Resultados y Discusión}
\input{capitulos/resultados_discusion/main.tex}

% Conclusiones
\chapter{Conclusiones}
\input{capitulos/conclusiones/main.tex}

% Planificación temporal y presupuesto
\chapter{Planificación Temporal y Presupuesto}
\input{capitulos/planificacion_presupuesto/main.tex}

% Bibliografía
\newpage
\addcontentsline{toc}{chapter}{Bibliografía}
\printbibliography

\end{document}


% Resultados y discusión (incluyendo la valoración de impactos y de aspectos de responsabilidad legal, ética y profesional relacionados con el trabajo)
\chapter{Resultados y Discusión}
\documentclass[a4paper,11pt,twoside]{report}
\usepackage[left=25mm,right=25mm,top=25mm,bottom=25mm,includehead,includefoot,headsep=15mm,footskip=15mm]{geometry}
\usepackage{graphicx}
\usepackage{fancyhdr}
\usepackage{titlesec}
\usepackage[spanish]{babel}
\usepackage[utf8]{inputenc}
\usepackage{amsmath}
\usepackage{setspace}
\usepackage{svg}
\usepackage{hyperref}
\usepackage[backend=biber,style=numeric]{biblatex}
\addbibresource{references.bib}
\hypersetup{
    colorlinks=true,
    linkcolor=blue,      % color of internal links (sections, etc.)
    urlcolor=blue,       % color of external links
    pdftitle={Optimización energética de sistema híbrido con bomba de calor, suelo radiante, fotovoltaica y almacenamiento para vivienda},    % title
    pdfauthor={Luis D. Aranda Sánchez},     % author
    pdfkeywords={palabra1, palabra2, código1, etc.} % list of keywords
}

% Font change to Arial
\usepackage{helvet}
\renewcommand{\familydefault}{\sfdefault}

% Chapter titles in uppercase and larger font
\titleformat{\chapter}[hang]{\large\bfseries}{\thechapter.}{1em}{\MakeUppercase}
\titleformat{\section}[hang]{\bfseries}{\thesection.}{1em}{}
\titleformat{\subsection}[hang]{\bfseries}{\thesubsection.}{1em}{}

% Fancyhdr setup
\setlength{\headheight}{14.30174pt} % Adjust to recommended value, slightly larger for safety
\fancyhf{} % Clear all headers and footers
\fancyhead[LE]{\nouppercase{\leftmark}}
\fancyhead[RO]{Optimización energética para vivienda}
\fancyfoot[LE]{\thepage}
\fancyfoot[RE]{Escuela Técnica Superior de Ingenieros Industriales (UPM)}
\fancyfoot[LO]{Luis D. Aranda Sánchez}
\fancyfoot[RO]{\thepage}
\renewcommand{\headrulewidth}{0.4pt}
\renewcommand{\footrulewidth}{0.4pt}

\fancypagestyle{myfancy}{
    \fancyhf{} % Clear all headers and footers
    \fancyhead[LE]{\nouppercase{\leftmark}}
    \fancyhead[RO]{Optimización energética para vivienda}
    \fancyfoot[LE]{\thepage}
    \fancyfoot[RE]{Escuela Técnica Superior de Ingenieros Industriales (UPM)}
    \fancyfoot[LO]{Luis D. Aranda Sánchez}
    \fancyfoot[RO]{\thepage}
    \renewcommand{\headrulewidth}{0.4pt}
    \renewcommand{\footrulewidth}{0.4pt}
}

\fancypagestyle{simple}{
    \fancyhf{} % Clear all headers and footers
    \renewcommand{\headrulewidth}{0pt}
    \renewcommand{\footrulewidth}{0pt}
}

% Line spacing
\setstretch{1.2}

% Document starts here
\begin{document}

% Portada
\begin{titlepage}
    \centering
    {\scshape\LARGE Universidad Politécnica de Madrid \par}
    \vspace{1cm}
    {\scshape\Large Escuela Técnica Superior de Ingenieros Industriales\par}
    \vspace{1.5cm}
    {\huge\bfseries Optimización energética de sistema híbrido con bomba de calor, suelo radiante, fotovoltaica y almacenamiento para vivienda \par}
    \vspace{1.5cm}
    {\Large\bfseries Trabajo de Fin de Máster\par}
    \vspace{0.5cm}
    {\large Máster Universitario en Ingeniería de la Energía \par}
    \vspace{2cm}
    {\Large Luis D. Aranda Sánchez\par}
    \vfill
    Director: Javier Rodríguez Martín
    \vfill
    {\large Septiembre 6, 2024\par}
\end{titlepage}

% Resumen (máximo de 5 páginas, incluyendo al final Palabras clave)
\clearpage
\pagestyle{simple}
% \newpage
\chapter*{Resumen}
\addcontentsline{toc}{chapter}{Resumen}
\input{capitulos/resumen/main.tex}

% Índice (paginado)
\clearpage
\pagestyle{simple}
% \newpage
\tableofcontents

% Introducción (donde se incluya los antecedentes y justificación)
\clearpage
\pagestyle{myfancy}
\newpage
\chapter{Introducción}
\input{capitulos/introduccion/main.tex}

% Objetivos
\chapter{Objetivos}
\input{capitulos/objetivos/main.tex}

% Metodología
\chapter{Metodología}
\input{capitulos/metodologia/main.tex}

% Resultados y discusión (incluyendo la valoración de impactos y de aspectos de responsabilidad legal, ética y profesional relacionados con el trabajo)
\chapter{Resultados y Discusión}
\input{capitulos/resultados_discusion/main.tex}

% Conclusiones
\chapter{Conclusiones}
\input{capitulos/conclusiones/main.tex}

% Planificación temporal y presupuesto
\chapter{Planificación Temporal y Presupuesto}
\input{capitulos/planificacion_presupuesto/main.tex}

% Bibliografía
\newpage
\addcontentsline{toc}{chapter}{Bibliografía}
\printbibliography

\end{document}


% Conclusiones
\chapter{Conclusiones}
\documentclass[a4paper,11pt,twoside]{report}
\usepackage[left=25mm,right=25mm,top=25mm,bottom=25mm,includehead,includefoot,headsep=15mm,footskip=15mm]{geometry}
\usepackage{graphicx}
\usepackage{fancyhdr}
\usepackage{titlesec}
\usepackage[spanish]{babel}
\usepackage[utf8]{inputenc}
\usepackage{amsmath}
\usepackage{setspace}
\usepackage{svg}
\usepackage{hyperref}
\usepackage[backend=biber,style=numeric]{biblatex}
\addbibresource{references.bib}
\hypersetup{
    colorlinks=true,
    linkcolor=blue,      % color of internal links (sections, etc.)
    urlcolor=blue,       % color of external links
    pdftitle={Optimización energética de sistema híbrido con bomba de calor, suelo radiante, fotovoltaica y almacenamiento para vivienda},    % title
    pdfauthor={Luis D. Aranda Sánchez},     % author
    pdfkeywords={palabra1, palabra2, código1, etc.} % list of keywords
}

% Font change to Arial
\usepackage{helvet}
\renewcommand{\familydefault}{\sfdefault}

% Chapter titles in uppercase and larger font
\titleformat{\chapter}[hang]{\large\bfseries}{\thechapter.}{1em}{\MakeUppercase}
\titleformat{\section}[hang]{\bfseries}{\thesection.}{1em}{}
\titleformat{\subsection}[hang]{\bfseries}{\thesubsection.}{1em}{}

% Fancyhdr setup
\setlength{\headheight}{14.30174pt} % Adjust to recommended value, slightly larger for safety
\fancyhf{} % Clear all headers and footers
\fancyhead[LE]{\nouppercase{\leftmark}}
\fancyhead[RO]{Optimización energética para vivienda}
\fancyfoot[LE]{\thepage}
\fancyfoot[RE]{Escuela Técnica Superior de Ingenieros Industriales (UPM)}
\fancyfoot[LO]{Luis D. Aranda Sánchez}
\fancyfoot[RO]{\thepage}
\renewcommand{\headrulewidth}{0.4pt}
\renewcommand{\footrulewidth}{0.4pt}

\fancypagestyle{myfancy}{
    \fancyhf{} % Clear all headers and footers
    \fancyhead[LE]{\nouppercase{\leftmark}}
    \fancyhead[RO]{Optimización energética para vivienda}
    \fancyfoot[LE]{\thepage}
    \fancyfoot[RE]{Escuela Técnica Superior de Ingenieros Industriales (UPM)}
    \fancyfoot[LO]{Luis D. Aranda Sánchez}
    \fancyfoot[RO]{\thepage}
    \renewcommand{\headrulewidth}{0.4pt}
    \renewcommand{\footrulewidth}{0.4pt}
}

\fancypagestyle{simple}{
    \fancyhf{} % Clear all headers and footers
    \renewcommand{\headrulewidth}{0pt}
    \renewcommand{\footrulewidth}{0pt}
}

% Line spacing
\setstretch{1.2}

% Document starts here
\begin{document}

% Portada
\begin{titlepage}
    \centering
    {\scshape\LARGE Universidad Politécnica de Madrid \par}
    \vspace{1cm}
    {\scshape\Large Escuela Técnica Superior de Ingenieros Industriales\par}
    \vspace{1.5cm}
    {\huge\bfseries Optimización energética de sistema híbrido con bomba de calor, suelo radiante, fotovoltaica y almacenamiento para vivienda \par}
    \vspace{1.5cm}
    {\Large\bfseries Trabajo de Fin de Máster\par}
    \vspace{0.5cm}
    {\large Máster Universitario en Ingeniería de la Energía \par}
    \vspace{2cm}
    {\Large Luis D. Aranda Sánchez\par}
    \vfill
    Director: Javier Rodríguez Martín
    \vfill
    {\large Septiembre 6, 2024\par}
\end{titlepage}

% Resumen (máximo de 5 páginas, incluyendo al final Palabras clave)
\clearpage
\pagestyle{simple}
% \newpage
\chapter*{Resumen}
\addcontentsline{toc}{chapter}{Resumen}
\input{capitulos/resumen/main.tex}

% Índice (paginado)
\clearpage
\pagestyle{simple}
% \newpage
\tableofcontents

% Introducción (donde se incluya los antecedentes y justificación)
\clearpage
\pagestyle{myfancy}
\newpage
\chapter{Introducción}
\input{capitulos/introduccion/main.tex}

% Objetivos
\chapter{Objetivos}
\input{capitulos/objetivos/main.tex}

% Metodología
\chapter{Metodología}
\input{capitulos/metodologia/main.tex}

% Resultados y discusión (incluyendo la valoración de impactos y de aspectos de responsabilidad legal, ética y profesional relacionados con el trabajo)
\chapter{Resultados y Discusión}
\input{capitulos/resultados_discusion/main.tex}

% Conclusiones
\chapter{Conclusiones}
\input{capitulos/conclusiones/main.tex}

% Planificación temporal y presupuesto
\chapter{Planificación Temporal y Presupuesto}
\input{capitulos/planificacion_presupuesto/main.tex}

% Bibliografía
\newpage
\addcontentsline{toc}{chapter}{Bibliografía}
\printbibliography

\end{document}


% Planificación temporal y presupuesto
\chapter{Planificación Temporal y Presupuesto}
\documentclass[a4paper,11pt,twoside]{report}
\usepackage[left=25mm,right=25mm,top=25mm,bottom=25mm,includehead,includefoot,headsep=15mm,footskip=15mm]{geometry}
\usepackage{graphicx}
\usepackage{fancyhdr}
\usepackage{titlesec}
\usepackage[spanish]{babel}
\usepackage[utf8]{inputenc}
\usepackage{amsmath}
\usepackage{setspace}
\usepackage{svg}
\usepackage{hyperref}
\usepackage[backend=biber,style=numeric]{biblatex}
\addbibresource{references.bib}
\hypersetup{
    colorlinks=true,
    linkcolor=blue,      % color of internal links (sections, etc.)
    urlcolor=blue,       % color of external links
    pdftitle={Optimización energética de sistema híbrido con bomba de calor, suelo radiante, fotovoltaica y almacenamiento para vivienda},    % title
    pdfauthor={Luis D. Aranda Sánchez},     % author
    pdfkeywords={palabra1, palabra2, código1, etc.} % list of keywords
}

% Font change to Arial
\usepackage{helvet}
\renewcommand{\familydefault}{\sfdefault}

% Chapter titles in uppercase and larger font
\titleformat{\chapter}[hang]{\large\bfseries}{\thechapter.}{1em}{\MakeUppercase}
\titleformat{\section}[hang]{\bfseries}{\thesection.}{1em}{}
\titleformat{\subsection}[hang]{\bfseries}{\thesubsection.}{1em}{}

% Fancyhdr setup
\setlength{\headheight}{14.30174pt} % Adjust to recommended value, slightly larger for safety
\fancyhf{} % Clear all headers and footers
\fancyhead[LE]{\nouppercase{\leftmark}}
\fancyhead[RO]{Optimización energética para vivienda}
\fancyfoot[LE]{\thepage}
\fancyfoot[RE]{Escuela Técnica Superior de Ingenieros Industriales (UPM)}
\fancyfoot[LO]{Luis D. Aranda Sánchez}
\fancyfoot[RO]{\thepage}
\renewcommand{\headrulewidth}{0.4pt}
\renewcommand{\footrulewidth}{0.4pt}

\fancypagestyle{myfancy}{
    \fancyhf{} % Clear all headers and footers
    \fancyhead[LE]{\nouppercase{\leftmark}}
    \fancyhead[RO]{Optimización energética para vivienda}
    \fancyfoot[LE]{\thepage}
    \fancyfoot[RE]{Escuela Técnica Superior de Ingenieros Industriales (UPM)}
    \fancyfoot[LO]{Luis D. Aranda Sánchez}
    \fancyfoot[RO]{\thepage}
    \renewcommand{\headrulewidth}{0.4pt}
    \renewcommand{\footrulewidth}{0.4pt}
}

\fancypagestyle{simple}{
    \fancyhf{} % Clear all headers and footers
    \renewcommand{\headrulewidth}{0pt}
    \renewcommand{\footrulewidth}{0pt}
}

% Line spacing
\setstretch{1.2}

% Document starts here
\begin{document}

% Portada
\begin{titlepage}
    \centering
    {\scshape\LARGE Universidad Politécnica de Madrid \par}
    \vspace{1cm}
    {\scshape\Large Escuela Técnica Superior de Ingenieros Industriales\par}
    \vspace{1.5cm}
    {\huge\bfseries Optimización energética de sistema híbrido con bomba de calor, suelo radiante, fotovoltaica y almacenamiento para vivienda \par}
    \vspace{1.5cm}
    {\Large\bfseries Trabajo de Fin de Máster\par}
    \vspace{0.5cm}
    {\large Máster Universitario en Ingeniería de la Energía \par}
    \vspace{2cm}
    {\Large Luis D. Aranda Sánchez\par}
    \vfill
    Director: Javier Rodríguez Martín
    \vfill
    {\large Septiembre 6, 2024\par}
\end{titlepage}

% Resumen (máximo de 5 páginas, incluyendo al final Palabras clave)
\clearpage
\pagestyle{simple}
% \newpage
\chapter*{Resumen}
\addcontentsline{toc}{chapter}{Resumen}
\input{capitulos/resumen/main.tex}

% Índice (paginado)
\clearpage
\pagestyle{simple}
% \newpage
\tableofcontents

% Introducción (donde se incluya los antecedentes y justificación)
\clearpage
\pagestyle{myfancy}
\newpage
\chapter{Introducción}
\input{capitulos/introduccion/main.tex}

% Objetivos
\chapter{Objetivos}
\input{capitulos/objetivos/main.tex}

% Metodología
\chapter{Metodología}
\input{capitulos/metodologia/main.tex}

% Resultados y discusión (incluyendo la valoración de impactos y de aspectos de responsabilidad legal, ética y profesional relacionados con el trabajo)
\chapter{Resultados y Discusión}
\input{capitulos/resultados_discusion/main.tex}

% Conclusiones
\chapter{Conclusiones}
\input{capitulos/conclusiones/main.tex}

% Planificación temporal y presupuesto
\chapter{Planificación Temporal y Presupuesto}
\input{capitulos/planificacion_presupuesto/main.tex}

% Bibliografía
\newpage
\addcontentsline{toc}{chapter}{Bibliografía}
\printbibliography

\end{document}


% Bibliografía
\newpage
\addcontentsline{toc}{chapter}{Bibliografía}
\printbibliography

\end{document}


% Planificación temporal y presupuesto
\chapter{Planificación Temporal y Presupuesto}
\documentclass[a4paper,11pt,twoside]{report}
\usepackage[left=25mm,right=25mm,top=25mm,bottom=25mm,includehead,includefoot,headsep=15mm,footskip=15mm]{geometry}
\usepackage{graphicx}
\usepackage{fancyhdr}
\usepackage{titlesec}
\usepackage[spanish]{babel}
\usepackage[utf8]{inputenc}
\usepackage{amsmath}
\usepackage{setspace}
\usepackage{svg}
\usepackage{hyperref}
\usepackage[backend=biber,style=numeric]{biblatex}
\addbibresource{references.bib}
\hypersetup{
    colorlinks=true,
    linkcolor=blue,      % color of internal links (sections, etc.)
    urlcolor=blue,       % color of external links
    pdftitle={Optimización energética de sistema híbrido con bomba de calor, suelo radiante, fotovoltaica y almacenamiento para vivienda},    % title
    pdfauthor={Luis D. Aranda Sánchez},     % author
    pdfkeywords={palabra1, palabra2, código1, etc.} % list of keywords
}

% Font change to Arial
\usepackage{helvet}
\renewcommand{\familydefault}{\sfdefault}

% Chapter titles in uppercase and larger font
\titleformat{\chapter}[hang]{\large\bfseries}{\thechapter.}{1em}{\MakeUppercase}
\titleformat{\section}[hang]{\bfseries}{\thesection.}{1em}{}
\titleformat{\subsection}[hang]{\bfseries}{\thesubsection.}{1em}{}

% Fancyhdr setup
\setlength{\headheight}{14.30174pt} % Adjust to recommended value, slightly larger for safety
\fancyhf{} % Clear all headers and footers
\fancyhead[LE]{\nouppercase{\leftmark}}
\fancyhead[RO]{Optimización energética para vivienda}
\fancyfoot[LE]{\thepage}
\fancyfoot[RE]{Escuela Técnica Superior de Ingenieros Industriales (UPM)}
\fancyfoot[LO]{Luis D. Aranda Sánchez}
\fancyfoot[RO]{\thepage}
\renewcommand{\headrulewidth}{0.4pt}
\renewcommand{\footrulewidth}{0.4pt}

\fancypagestyle{myfancy}{
    \fancyhf{} % Clear all headers and footers
    \fancyhead[LE]{\nouppercase{\leftmark}}
    \fancyhead[RO]{Optimización energética para vivienda}
    \fancyfoot[LE]{\thepage}
    \fancyfoot[RE]{Escuela Técnica Superior de Ingenieros Industriales (UPM)}
    \fancyfoot[LO]{Luis D. Aranda Sánchez}
    \fancyfoot[RO]{\thepage}
    \renewcommand{\headrulewidth}{0.4pt}
    \renewcommand{\footrulewidth}{0.4pt}
}

\fancypagestyle{simple}{
    \fancyhf{} % Clear all headers and footers
    \renewcommand{\headrulewidth}{0pt}
    \renewcommand{\footrulewidth}{0pt}
}

% Line spacing
\setstretch{1.2}

% Document starts here
\begin{document}

% Portada
\begin{titlepage}
    \centering
    {\scshape\LARGE Universidad Politécnica de Madrid \par}
    \vspace{1cm}
    {\scshape\Large Escuela Técnica Superior de Ingenieros Industriales\par}
    \vspace{1.5cm}
    {\huge\bfseries Optimización energética de sistema híbrido con bomba de calor, suelo radiante, fotovoltaica y almacenamiento para vivienda \par}
    \vspace{1.5cm}
    {\Large\bfseries Trabajo de Fin de Máster\par}
    \vspace{0.5cm}
    {\large Máster Universitario en Ingeniería de la Energía \par}
    \vspace{2cm}
    {\Large Luis D. Aranda Sánchez\par}
    \vfill
    Director: Javier Rodríguez Martín
    \vfill
    {\large Septiembre 6, 2024\par}
\end{titlepage}

% Resumen (máximo de 5 páginas, incluyendo al final Palabras clave)
\clearpage
\pagestyle{simple}
% \newpage
\chapter*{Resumen}
\addcontentsline{toc}{chapter}{Resumen}
\documentclass[a4paper,11pt,twoside]{report}
\usepackage[left=25mm,right=25mm,top=25mm,bottom=25mm,includehead,includefoot,headsep=15mm,footskip=15mm]{geometry}
\usepackage{graphicx}
\usepackage{fancyhdr}
\usepackage{titlesec}
\usepackage[spanish]{babel}
\usepackage[utf8]{inputenc}
\usepackage{amsmath}
\usepackage{setspace}
\usepackage{svg}
\usepackage{hyperref}
\usepackage[backend=biber,style=numeric]{biblatex}
\addbibresource{references.bib}
\hypersetup{
    colorlinks=true,
    linkcolor=blue,      % color of internal links (sections, etc.)
    urlcolor=blue,       % color of external links
    pdftitle={Optimización energética de sistema híbrido con bomba de calor, suelo radiante, fotovoltaica y almacenamiento para vivienda},    % title
    pdfauthor={Luis D. Aranda Sánchez},     % author
    pdfkeywords={palabra1, palabra2, código1, etc.} % list of keywords
}

% Font change to Arial
\usepackage{helvet}
\renewcommand{\familydefault}{\sfdefault}

% Chapter titles in uppercase and larger font
\titleformat{\chapter}[hang]{\large\bfseries}{\thechapter.}{1em}{\MakeUppercase}
\titleformat{\section}[hang]{\bfseries}{\thesection.}{1em}{}
\titleformat{\subsection}[hang]{\bfseries}{\thesubsection.}{1em}{}

% Fancyhdr setup
\setlength{\headheight}{14.30174pt} % Adjust to recommended value, slightly larger for safety
\fancyhf{} % Clear all headers and footers
\fancyhead[LE]{\nouppercase{\leftmark}}
\fancyhead[RO]{Optimización energética para vivienda}
\fancyfoot[LE]{\thepage}
\fancyfoot[RE]{Escuela Técnica Superior de Ingenieros Industriales (UPM)}
\fancyfoot[LO]{Luis D. Aranda Sánchez}
\fancyfoot[RO]{\thepage}
\renewcommand{\headrulewidth}{0.4pt}
\renewcommand{\footrulewidth}{0.4pt}

\fancypagestyle{myfancy}{
    \fancyhf{} % Clear all headers and footers
    \fancyhead[LE]{\nouppercase{\leftmark}}
    \fancyhead[RO]{Optimización energética para vivienda}
    \fancyfoot[LE]{\thepage}
    \fancyfoot[RE]{Escuela Técnica Superior de Ingenieros Industriales (UPM)}
    \fancyfoot[LO]{Luis D. Aranda Sánchez}
    \fancyfoot[RO]{\thepage}
    \renewcommand{\headrulewidth}{0.4pt}
    \renewcommand{\footrulewidth}{0.4pt}
}

\fancypagestyle{simple}{
    \fancyhf{} % Clear all headers and footers
    \renewcommand{\headrulewidth}{0pt}
    \renewcommand{\footrulewidth}{0pt}
}

% Line spacing
\setstretch{1.2}

% Document starts here
\begin{document}

% Portada
\begin{titlepage}
    \centering
    {\scshape\LARGE Universidad Politécnica de Madrid \par}
    \vspace{1cm}
    {\scshape\Large Escuela Técnica Superior de Ingenieros Industriales\par}
    \vspace{1.5cm}
    {\huge\bfseries Optimización energética de sistema híbrido con bomba de calor, suelo radiante, fotovoltaica y almacenamiento para vivienda \par}
    \vspace{1.5cm}
    {\Large\bfseries Trabajo de Fin de Máster\par}
    \vspace{0.5cm}
    {\large Máster Universitario en Ingeniería de la Energía \par}
    \vspace{2cm}
    {\Large Luis D. Aranda Sánchez\par}
    \vfill
    Director: Javier Rodríguez Martín
    \vfill
    {\large Septiembre 6, 2024\par}
\end{titlepage}

% Resumen (máximo de 5 páginas, incluyendo al final Palabras clave)
\clearpage
\pagestyle{simple}
% \newpage
\chapter*{Resumen}
\addcontentsline{toc}{chapter}{Resumen}
\input{capitulos/resumen/main.tex}

% Índice (paginado)
\clearpage
\pagestyle{simple}
% \newpage
\tableofcontents

% Introducción (donde se incluya los antecedentes y justificación)
\clearpage
\pagestyle{myfancy}
\newpage
\chapter{Introducción}
\input{capitulos/introduccion/main.tex}

% Objetivos
\chapter{Objetivos}
\input{capitulos/objetivos/main.tex}

% Metodología
\chapter{Metodología}
\input{capitulos/metodologia/main.tex}

% Resultados y discusión (incluyendo la valoración de impactos y de aspectos de responsabilidad legal, ética y profesional relacionados con el trabajo)
\chapter{Resultados y Discusión}
\input{capitulos/resultados_discusion/main.tex}

% Conclusiones
\chapter{Conclusiones}
\input{capitulos/conclusiones/main.tex}

% Planificación temporal y presupuesto
\chapter{Planificación Temporal y Presupuesto}
\input{capitulos/planificacion_presupuesto/main.tex}

% Bibliografía
\newpage
\addcontentsline{toc}{chapter}{Bibliografía}
\printbibliography

\end{document}


% Índice (paginado)
\clearpage
\pagestyle{simple}
% \newpage
\tableofcontents

% Introducción (donde se incluya los antecedentes y justificación)
\clearpage
\pagestyle{myfancy}
\newpage
\chapter{Introducción}
\documentclass[a4paper,11pt,twoside]{report}
\usepackage[left=25mm,right=25mm,top=25mm,bottom=25mm,includehead,includefoot,headsep=15mm,footskip=15mm]{geometry}
\usepackage{graphicx}
\usepackage{fancyhdr}
\usepackage{titlesec}
\usepackage[spanish]{babel}
\usepackage[utf8]{inputenc}
\usepackage{amsmath}
\usepackage{setspace}
\usepackage{svg}
\usepackage{hyperref}
\usepackage[backend=biber,style=numeric]{biblatex}
\addbibresource{references.bib}
\hypersetup{
    colorlinks=true,
    linkcolor=blue,      % color of internal links (sections, etc.)
    urlcolor=blue,       % color of external links
    pdftitle={Optimización energética de sistema híbrido con bomba de calor, suelo radiante, fotovoltaica y almacenamiento para vivienda},    % title
    pdfauthor={Luis D. Aranda Sánchez},     % author
    pdfkeywords={palabra1, palabra2, código1, etc.} % list of keywords
}

% Font change to Arial
\usepackage{helvet}
\renewcommand{\familydefault}{\sfdefault}

% Chapter titles in uppercase and larger font
\titleformat{\chapter}[hang]{\large\bfseries}{\thechapter.}{1em}{\MakeUppercase}
\titleformat{\section}[hang]{\bfseries}{\thesection.}{1em}{}
\titleformat{\subsection}[hang]{\bfseries}{\thesubsection.}{1em}{}

% Fancyhdr setup
\setlength{\headheight}{14.30174pt} % Adjust to recommended value, slightly larger for safety
\fancyhf{} % Clear all headers and footers
\fancyhead[LE]{\nouppercase{\leftmark}}
\fancyhead[RO]{Optimización energética para vivienda}
\fancyfoot[LE]{\thepage}
\fancyfoot[RE]{Escuela Técnica Superior de Ingenieros Industriales (UPM)}
\fancyfoot[LO]{Luis D. Aranda Sánchez}
\fancyfoot[RO]{\thepage}
\renewcommand{\headrulewidth}{0.4pt}
\renewcommand{\footrulewidth}{0.4pt}

\fancypagestyle{myfancy}{
    \fancyhf{} % Clear all headers and footers
    \fancyhead[LE]{\nouppercase{\leftmark}}
    \fancyhead[RO]{Optimización energética para vivienda}
    \fancyfoot[LE]{\thepage}
    \fancyfoot[RE]{Escuela Técnica Superior de Ingenieros Industriales (UPM)}
    \fancyfoot[LO]{Luis D. Aranda Sánchez}
    \fancyfoot[RO]{\thepage}
    \renewcommand{\headrulewidth}{0.4pt}
    \renewcommand{\footrulewidth}{0.4pt}
}

\fancypagestyle{simple}{
    \fancyhf{} % Clear all headers and footers
    \renewcommand{\headrulewidth}{0pt}
    \renewcommand{\footrulewidth}{0pt}
}

% Line spacing
\setstretch{1.2}

% Document starts here
\begin{document}

% Portada
\begin{titlepage}
    \centering
    {\scshape\LARGE Universidad Politécnica de Madrid \par}
    \vspace{1cm}
    {\scshape\Large Escuela Técnica Superior de Ingenieros Industriales\par}
    \vspace{1.5cm}
    {\huge\bfseries Optimización energética de sistema híbrido con bomba de calor, suelo radiante, fotovoltaica y almacenamiento para vivienda \par}
    \vspace{1.5cm}
    {\Large\bfseries Trabajo de Fin de Máster\par}
    \vspace{0.5cm}
    {\large Máster Universitario en Ingeniería de la Energía \par}
    \vspace{2cm}
    {\Large Luis D. Aranda Sánchez\par}
    \vfill
    Director: Javier Rodríguez Martín
    \vfill
    {\large Septiembre 6, 2024\par}
\end{titlepage}

% Resumen (máximo de 5 páginas, incluyendo al final Palabras clave)
\clearpage
\pagestyle{simple}
% \newpage
\chapter*{Resumen}
\addcontentsline{toc}{chapter}{Resumen}
\input{capitulos/resumen/main.tex}

% Índice (paginado)
\clearpage
\pagestyle{simple}
% \newpage
\tableofcontents

% Introducción (donde se incluya los antecedentes y justificación)
\clearpage
\pagestyle{myfancy}
\newpage
\chapter{Introducción}
\input{capitulos/introduccion/main.tex}

% Objetivos
\chapter{Objetivos}
\input{capitulos/objetivos/main.tex}

% Metodología
\chapter{Metodología}
\input{capitulos/metodologia/main.tex}

% Resultados y discusión (incluyendo la valoración de impactos y de aspectos de responsabilidad legal, ética y profesional relacionados con el trabajo)
\chapter{Resultados y Discusión}
\input{capitulos/resultados_discusion/main.tex}

% Conclusiones
\chapter{Conclusiones}
\input{capitulos/conclusiones/main.tex}

% Planificación temporal y presupuesto
\chapter{Planificación Temporal y Presupuesto}
\input{capitulos/planificacion_presupuesto/main.tex}

% Bibliografía
\newpage
\addcontentsline{toc}{chapter}{Bibliografía}
\printbibliography

\end{document}


% Objetivos
\chapter{Objetivos}
\documentclass[a4paper,11pt,twoside]{report}
\usepackage[left=25mm,right=25mm,top=25mm,bottom=25mm,includehead,includefoot,headsep=15mm,footskip=15mm]{geometry}
\usepackage{graphicx}
\usepackage{fancyhdr}
\usepackage{titlesec}
\usepackage[spanish]{babel}
\usepackage[utf8]{inputenc}
\usepackage{amsmath}
\usepackage{setspace}
\usepackage{svg}
\usepackage{hyperref}
\usepackage[backend=biber,style=numeric]{biblatex}
\addbibresource{references.bib}
\hypersetup{
    colorlinks=true,
    linkcolor=blue,      % color of internal links (sections, etc.)
    urlcolor=blue,       % color of external links
    pdftitle={Optimización energética de sistema híbrido con bomba de calor, suelo radiante, fotovoltaica y almacenamiento para vivienda},    % title
    pdfauthor={Luis D. Aranda Sánchez},     % author
    pdfkeywords={palabra1, palabra2, código1, etc.} % list of keywords
}

% Font change to Arial
\usepackage{helvet}
\renewcommand{\familydefault}{\sfdefault}

% Chapter titles in uppercase and larger font
\titleformat{\chapter}[hang]{\large\bfseries}{\thechapter.}{1em}{\MakeUppercase}
\titleformat{\section}[hang]{\bfseries}{\thesection.}{1em}{}
\titleformat{\subsection}[hang]{\bfseries}{\thesubsection.}{1em}{}

% Fancyhdr setup
\setlength{\headheight}{14.30174pt} % Adjust to recommended value, slightly larger for safety
\fancyhf{} % Clear all headers and footers
\fancyhead[LE]{\nouppercase{\leftmark}}
\fancyhead[RO]{Optimización energética para vivienda}
\fancyfoot[LE]{\thepage}
\fancyfoot[RE]{Escuela Técnica Superior de Ingenieros Industriales (UPM)}
\fancyfoot[LO]{Luis D. Aranda Sánchez}
\fancyfoot[RO]{\thepage}
\renewcommand{\headrulewidth}{0.4pt}
\renewcommand{\footrulewidth}{0.4pt}

\fancypagestyle{myfancy}{
    \fancyhf{} % Clear all headers and footers
    \fancyhead[LE]{\nouppercase{\leftmark}}
    \fancyhead[RO]{Optimización energética para vivienda}
    \fancyfoot[LE]{\thepage}
    \fancyfoot[RE]{Escuela Técnica Superior de Ingenieros Industriales (UPM)}
    \fancyfoot[LO]{Luis D. Aranda Sánchez}
    \fancyfoot[RO]{\thepage}
    \renewcommand{\headrulewidth}{0.4pt}
    \renewcommand{\footrulewidth}{0.4pt}
}

\fancypagestyle{simple}{
    \fancyhf{} % Clear all headers and footers
    \renewcommand{\headrulewidth}{0pt}
    \renewcommand{\footrulewidth}{0pt}
}

% Line spacing
\setstretch{1.2}

% Document starts here
\begin{document}

% Portada
\begin{titlepage}
    \centering
    {\scshape\LARGE Universidad Politécnica de Madrid \par}
    \vspace{1cm}
    {\scshape\Large Escuela Técnica Superior de Ingenieros Industriales\par}
    \vspace{1.5cm}
    {\huge\bfseries Optimización energética de sistema híbrido con bomba de calor, suelo radiante, fotovoltaica y almacenamiento para vivienda \par}
    \vspace{1.5cm}
    {\Large\bfseries Trabajo de Fin de Máster\par}
    \vspace{0.5cm}
    {\large Máster Universitario en Ingeniería de la Energía \par}
    \vspace{2cm}
    {\Large Luis D. Aranda Sánchez\par}
    \vfill
    Director: Javier Rodríguez Martín
    \vfill
    {\large Septiembre 6, 2024\par}
\end{titlepage}

% Resumen (máximo de 5 páginas, incluyendo al final Palabras clave)
\clearpage
\pagestyle{simple}
% \newpage
\chapter*{Resumen}
\addcontentsline{toc}{chapter}{Resumen}
\input{capitulos/resumen/main.tex}

% Índice (paginado)
\clearpage
\pagestyle{simple}
% \newpage
\tableofcontents

% Introducción (donde se incluya los antecedentes y justificación)
\clearpage
\pagestyle{myfancy}
\newpage
\chapter{Introducción}
\input{capitulos/introduccion/main.tex}

% Objetivos
\chapter{Objetivos}
\input{capitulos/objetivos/main.tex}

% Metodología
\chapter{Metodología}
\input{capitulos/metodologia/main.tex}

% Resultados y discusión (incluyendo la valoración de impactos y de aspectos de responsabilidad legal, ética y profesional relacionados con el trabajo)
\chapter{Resultados y Discusión}
\input{capitulos/resultados_discusion/main.tex}

% Conclusiones
\chapter{Conclusiones}
\input{capitulos/conclusiones/main.tex}

% Planificación temporal y presupuesto
\chapter{Planificación Temporal y Presupuesto}
\input{capitulos/planificacion_presupuesto/main.tex}

% Bibliografía
\newpage
\addcontentsline{toc}{chapter}{Bibliografía}
\printbibliography

\end{document}


% Metodología
\chapter{Metodología}
\documentclass[a4paper,11pt,twoside]{report}
\usepackage[left=25mm,right=25mm,top=25mm,bottom=25mm,includehead,includefoot,headsep=15mm,footskip=15mm]{geometry}
\usepackage{graphicx}
\usepackage{fancyhdr}
\usepackage{titlesec}
\usepackage[spanish]{babel}
\usepackage[utf8]{inputenc}
\usepackage{amsmath}
\usepackage{setspace}
\usepackage{svg}
\usepackage{hyperref}
\usepackage[backend=biber,style=numeric]{biblatex}
\addbibresource{references.bib}
\hypersetup{
    colorlinks=true,
    linkcolor=blue,      % color of internal links (sections, etc.)
    urlcolor=blue,       % color of external links
    pdftitle={Optimización energética de sistema híbrido con bomba de calor, suelo radiante, fotovoltaica y almacenamiento para vivienda},    % title
    pdfauthor={Luis D. Aranda Sánchez},     % author
    pdfkeywords={palabra1, palabra2, código1, etc.} % list of keywords
}

% Font change to Arial
\usepackage{helvet}
\renewcommand{\familydefault}{\sfdefault}

% Chapter titles in uppercase and larger font
\titleformat{\chapter}[hang]{\large\bfseries}{\thechapter.}{1em}{\MakeUppercase}
\titleformat{\section}[hang]{\bfseries}{\thesection.}{1em}{}
\titleformat{\subsection}[hang]{\bfseries}{\thesubsection.}{1em}{}

% Fancyhdr setup
\setlength{\headheight}{14.30174pt} % Adjust to recommended value, slightly larger for safety
\fancyhf{} % Clear all headers and footers
\fancyhead[LE]{\nouppercase{\leftmark}}
\fancyhead[RO]{Optimización energética para vivienda}
\fancyfoot[LE]{\thepage}
\fancyfoot[RE]{Escuela Técnica Superior de Ingenieros Industriales (UPM)}
\fancyfoot[LO]{Luis D. Aranda Sánchez}
\fancyfoot[RO]{\thepage}
\renewcommand{\headrulewidth}{0.4pt}
\renewcommand{\footrulewidth}{0.4pt}

\fancypagestyle{myfancy}{
    \fancyhf{} % Clear all headers and footers
    \fancyhead[LE]{\nouppercase{\leftmark}}
    \fancyhead[RO]{Optimización energética para vivienda}
    \fancyfoot[LE]{\thepage}
    \fancyfoot[RE]{Escuela Técnica Superior de Ingenieros Industriales (UPM)}
    \fancyfoot[LO]{Luis D. Aranda Sánchez}
    \fancyfoot[RO]{\thepage}
    \renewcommand{\headrulewidth}{0.4pt}
    \renewcommand{\footrulewidth}{0.4pt}
}

\fancypagestyle{simple}{
    \fancyhf{} % Clear all headers and footers
    \renewcommand{\headrulewidth}{0pt}
    \renewcommand{\footrulewidth}{0pt}
}

% Line spacing
\setstretch{1.2}

% Document starts here
\begin{document}

% Portada
\begin{titlepage}
    \centering
    {\scshape\LARGE Universidad Politécnica de Madrid \par}
    \vspace{1cm}
    {\scshape\Large Escuela Técnica Superior de Ingenieros Industriales\par}
    \vspace{1.5cm}
    {\huge\bfseries Optimización energética de sistema híbrido con bomba de calor, suelo radiante, fotovoltaica y almacenamiento para vivienda \par}
    \vspace{1.5cm}
    {\Large\bfseries Trabajo de Fin de Máster\par}
    \vspace{0.5cm}
    {\large Máster Universitario en Ingeniería de la Energía \par}
    \vspace{2cm}
    {\Large Luis D. Aranda Sánchez\par}
    \vfill
    Director: Javier Rodríguez Martín
    \vfill
    {\large Septiembre 6, 2024\par}
\end{titlepage}

% Resumen (máximo de 5 páginas, incluyendo al final Palabras clave)
\clearpage
\pagestyle{simple}
% \newpage
\chapter*{Resumen}
\addcontentsline{toc}{chapter}{Resumen}
\input{capitulos/resumen/main.tex}

% Índice (paginado)
\clearpage
\pagestyle{simple}
% \newpage
\tableofcontents

% Introducción (donde se incluya los antecedentes y justificación)
\clearpage
\pagestyle{myfancy}
\newpage
\chapter{Introducción}
\input{capitulos/introduccion/main.tex}

% Objetivos
\chapter{Objetivos}
\input{capitulos/objetivos/main.tex}

% Metodología
\chapter{Metodología}
\input{capitulos/metodologia/main.tex}

% Resultados y discusión (incluyendo la valoración de impactos y de aspectos de responsabilidad legal, ética y profesional relacionados con el trabajo)
\chapter{Resultados y Discusión}
\input{capitulos/resultados_discusion/main.tex}

% Conclusiones
\chapter{Conclusiones}
\input{capitulos/conclusiones/main.tex}

% Planificación temporal y presupuesto
\chapter{Planificación Temporal y Presupuesto}
\input{capitulos/planificacion_presupuesto/main.tex}

% Bibliografía
\newpage
\addcontentsline{toc}{chapter}{Bibliografía}
\printbibliography

\end{document}


% Resultados y discusión (incluyendo la valoración de impactos y de aspectos de responsabilidad legal, ética y profesional relacionados con el trabajo)
\chapter{Resultados y Discusión}
\documentclass[a4paper,11pt,twoside]{report}
\usepackage[left=25mm,right=25mm,top=25mm,bottom=25mm,includehead,includefoot,headsep=15mm,footskip=15mm]{geometry}
\usepackage{graphicx}
\usepackage{fancyhdr}
\usepackage{titlesec}
\usepackage[spanish]{babel}
\usepackage[utf8]{inputenc}
\usepackage{amsmath}
\usepackage{setspace}
\usepackage{svg}
\usepackage{hyperref}
\usepackage[backend=biber,style=numeric]{biblatex}
\addbibresource{references.bib}
\hypersetup{
    colorlinks=true,
    linkcolor=blue,      % color of internal links (sections, etc.)
    urlcolor=blue,       % color of external links
    pdftitle={Optimización energética de sistema híbrido con bomba de calor, suelo radiante, fotovoltaica y almacenamiento para vivienda},    % title
    pdfauthor={Luis D. Aranda Sánchez},     % author
    pdfkeywords={palabra1, palabra2, código1, etc.} % list of keywords
}

% Font change to Arial
\usepackage{helvet}
\renewcommand{\familydefault}{\sfdefault}

% Chapter titles in uppercase and larger font
\titleformat{\chapter}[hang]{\large\bfseries}{\thechapter.}{1em}{\MakeUppercase}
\titleformat{\section}[hang]{\bfseries}{\thesection.}{1em}{}
\titleformat{\subsection}[hang]{\bfseries}{\thesubsection.}{1em}{}

% Fancyhdr setup
\setlength{\headheight}{14.30174pt} % Adjust to recommended value, slightly larger for safety
\fancyhf{} % Clear all headers and footers
\fancyhead[LE]{\nouppercase{\leftmark}}
\fancyhead[RO]{Optimización energética para vivienda}
\fancyfoot[LE]{\thepage}
\fancyfoot[RE]{Escuela Técnica Superior de Ingenieros Industriales (UPM)}
\fancyfoot[LO]{Luis D. Aranda Sánchez}
\fancyfoot[RO]{\thepage}
\renewcommand{\headrulewidth}{0.4pt}
\renewcommand{\footrulewidth}{0.4pt}

\fancypagestyle{myfancy}{
    \fancyhf{} % Clear all headers and footers
    \fancyhead[LE]{\nouppercase{\leftmark}}
    \fancyhead[RO]{Optimización energética para vivienda}
    \fancyfoot[LE]{\thepage}
    \fancyfoot[RE]{Escuela Técnica Superior de Ingenieros Industriales (UPM)}
    \fancyfoot[LO]{Luis D. Aranda Sánchez}
    \fancyfoot[RO]{\thepage}
    \renewcommand{\headrulewidth}{0.4pt}
    \renewcommand{\footrulewidth}{0.4pt}
}

\fancypagestyle{simple}{
    \fancyhf{} % Clear all headers and footers
    \renewcommand{\headrulewidth}{0pt}
    \renewcommand{\footrulewidth}{0pt}
}

% Line spacing
\setstretch{1.2}

% Document starts here
\begin{document}

% Portada
\begin{titlepage}
    \centering
    {\scshape\LARGE Universidad Politécnica de Madrid \par}
    \vspace{1cm}
    {\scshape\Large Escuela Técnica Superior de Ingenieros Industriales\par}
    \vspace{1.5cm}
    {\huge\bfseries Optimización energética de sistema híbrido con bomba de calor, suelo radiante, fotovoltaica y almacenamiento para vivienda \par}
    \vspace{1.5cm}
    {\Large\bfseries Trabajo de Fin de Máster\par}
    \vspace{0.5cm}
    {\large Máster Universitario en Ingeniería de la Energía \par}
    \vspace{2cm}
    {\Large Luis D. Aranda Sánchez\par}
    \vfill
    Director: Javier Rodríguez Martín
    \vfill
    {\large Septiembre 6, 2024\par}
\end{titlepage}

% Resumen (máximo de 5 páginas, incluyendo al final Palabras clave)
\clearpage
\pagestyle{simple}
% \newpage
\chapter*{Resumen}
\addcontentsline{toc}{chapter}{Resumen}
\input{capitulos/resumen/main.tex}

% Índice (paginado)
\clearpage
\pagestyle{simple}
% \newpage
\tableofcontents

% Introducción (donde se incluya los antecedentes y justificación)
\clearpage
\pagestyle{myfancy}
\newpage
\chapter{Introducción}
\input{capitulos/introduccion/main.tex}

% Objetivos
\chapter{Objetivos}
\input{capitulos/objetivos/main.tex}

% Metodología
\chapter{Metodología}
\input{capitulos/metodologia/main.tex}

% Resultados y discusión (incluyendo la valoración de impactos y de aspectos de responsabilidad legal, ética y profesional relacionados con el trabajo)
\chapter{Resultados y Discusión}
\input{capitulos/resultados_discusion/main.tex}

% Conclusiones
\chapter{Conclusiones}
\input{capitulos/conclusiones/main.tex}

% Planificación temporal y presupuesto
\chapter{Planificación Temporal y Presupuesto}
\input{capitulos/planificacion_presupuesto/main.tex}

% Bibliografía
\newpage
\addcontentsline{toc}{chapter}{Bibliografía}
\printbibliography

\end{document}


% Conclusiones
\chapter{Conclusiones}
\documentclass[a4paper,11pt,twoside]{report}
\usepackage[left=25mm,right=25mm,top=25mm,bottom=25mm,includehead,includefoot,headsep=15mm,footskip=15mm]{geometry}
\usepackage{graphicx}
\usepackage{fancyhdr}
\usepackage{titlesec}
\usepackage[spanish]{babel}
\usepackage[utf8]{inputenc}
\usepackage{amsmath}
\usepackage{setspace}
\usepackage{svg}
\usepackage{hyperref}
\usepackage[backend=biber,style=numeric]{biblatex}
\addbibresource{references.bib}
\hypersetup{
    colorlinks=true,
    linkcolor=blue,      % color of internal links (sections, etc.)
    urlcolor=blue,       % color of external links
    pdftitle={Optimización energética de sistema híbrido con bomba de calor, suelo radiante, fotovoltaica y almacenamiento para vivienda},    % title
    pdfauthor={Luis D. Aranda Sánchez},     % author
    pdfkeywords={palabra1, palabra2, código1, etc.} % list of keywords
}

% Font change to Arial
\usepackage{helvet}
\renewcommand{\familydefault}{\sfdefault}

% Chapter titles in uppercase and larger font
\titleformat{\chapter}[hang]{\large\bfseries}{\thechapter.}{1em}{\MakeUppercase}
\titleformat{\section}[hang]{\bfseries}{\thesection.}{1em}{}
\titleformat{\subsection}[hang]{\bfseries}{\thesubsection.}{1em}{}

% Fancyhdr setup
\setlength{\headheight}{14.30174pt} % Adjust to recommended value, slightly larger for safety
\fancyhf{} % Clear all headers and footers
\fancyhead[LE]{\nouppercase{\leftmark}}
\fancyhead[RO]{Optimización energética para vivienda}
\fancyfoot[LE]{\thepage}
\fancyfoot[RE]{Escuela Técnica Superior de Ingenieros Industriales (UPM)}
\fancyfoot[LO]{Luis D. Aranda Sánchez}
\fancyfoot[RO]{\thepage}
\renewcommand{\headrulewidth}{0.4pt}
\renewcommand{\footrulewidth}{0.4pt}

\fancypagestyle{myfancy}{
    \fancyhf{} % Clear all headers and footers
    \fancyhead[LE]{\nouppercase{\leftmark}}
    \fancyhead[RO]{Optimización energética para vivienda}
    \fancyfoot[LE]{\thepage}
    \fancyfoot[RE]{Escuela Técnica Superior de Ingenieros Industriales (UPM)}
    \fancyfoot[LO]{Luis D. Aranda Sánchez}
    \fancyfoot[RO]{\thepage}
    \renewcommand{\headrulewidth}{0.4pt}
    \renewcommand{\footrulewidth}{0.4pt}
}

\fancypagestyle{simple}{
    \fancyhf{} % Clear all headers and footers
    \renewcommand{\headrulewidth}{0pt}
    \renewcommand{\footrulewidth}{0pt}
}

% Line spacing
\setstretch{1.2}

% Document starts here
\begin{document}

% Portada
\begin{titlepage}
    \centering
    {\scshape\LARGE Universidad Politécnica de Madrid \par}
    \vspace{1cm}
    {\scshape\Large Escuela Técnica Superior de Ingenieros Industriales\par}
    \vspace{1.5cm}
    {\huge\bfseries Optimización energética de sistema híbrido con bomba de calor, suelo radiante, fotovoltaica y almacenamiento para vivienda \par}
    \vspace{1.5cm}
    {\Large\bfseries Trabajo de Fin de Máster\par}
    \vspace{0.5cm}
    {\large Máster Universitario en Ingeniería de la Energía \par}
    \vspace{2cm}
    {\Large Luis D. Aranda Sánchez\par}
    \vfill
    Director: Javier Rodríguez Martín
    \vfill
    {\large Septiembre 6, 2024\par}
\end{titlepage}

% Resumen (máximo de 5 páginas, incluyendo al final Palabras clave)
\clearpage
\pagestyle{simple}
% \newpage
\chapter*{Resumen}
\addcontentsline{toc}{chapter}{Resumen}
\input{capitulos/resumen/main.tex}

% Índice (paginado)
\clearpage
\pagestyle{simple}
% \newpage
\tableofcontents

% Introducción (donde se incluya los antecedentes y justificación)
\clearpage
\pagestyle{myfancy}
\newpage
\chapter{Introducción}
\input{capitulos/introduccion/main.tex}

% Objetivos
\chapter{Objetivos}
\input{capitulos/objetivos/main.tex}

% Metodología
\chapter{Metodología}
\input{capitulos/metodologia/main.tex}

% Resultados y discusión (incluyendo la valoración de impactos y de aspectos de responsabilidad legal, ética y profesional relacionados con el trabajo)
\chapter{Resultados y Discusión}
\input{capitulos/resultados_discusion/main.tex}

% Conclusiones
\chapter{Conclusiones}
\input{capitulos/conclusiones/main.tex}

% Planificación temporal y presupuesto
\chapter{Planificación Temporal y Presupuesto}
\input{capitulos/planificacion_presupuesto/main.tex}

% Bibliografía
\newpage
\addcontentsline{toc}{chapter}{Bibliografía}
\printbibliography

\end{document}


% Planificación temporal y presupuesto
\chapter{Planificación Temporal y Presupuesto}
\documentclass[a4paper,11pt,twoside]{report}
\usepackage[left=25mm,right=25mm,top=25mm,bottom=25mm,includehead,includefoot,headsep=15mm,footskip=15mm]{geometry}
\usepackage{graphicx}
\usepackage{fancyhdr}
\usepackage{titlesec}
\usepackage[spanish]{babel}
\usepackage[utf8]{inputenc}
\usepackage{amsmath}
\usepackage{setspace}
\usepackage{svg}
\usepackage{hyperref}
\usepackage[backend=biber,style=numeric]{biblatex}
\addbibresource{references.bib}
\hypersetup{
    colorlinks=true,
    linkcolor=blue,      % color of internal links (sections, etc.)
    urlcolor=blue,       % color of external links
    pdftitle={Optimización energética de sistema híbrido con bomba de calor, suelo radiante, fotovoltaica y almacenamiento para vivienda},    % title
    pdfauthor={Luis D. Aranda Sánchez},     % author
    pdfkeywords={palabra1, palabra2, código1, etc.} % list of keywords
}

% Font change to Arial
\usepackage{helvet}
\renewcommand{\familydefault}{\sfdefault}

% Chapter titles in uppercase and larger font
\titleformat{\chapter}[hang]{\large\bfseries}{\thechapter.}{1em}{\MakeUppercase}
\titleformat{\section}[hang]{\bfseries}{\thesection.}{1em}{}
\titleformat{\subsection}[hang]{\bfseries}{\thesubsection.}{1em}{}

% Fancyhdr setup
\setlength{\headheight}{14.30174pt} % Adjust to recommended value, slightly larger for safety
\fancyhf{} % Clear all headers and footers
\fancyhead[LE]{\nouppercase{\leftmark}}
\fancyhead[RO]{Optimización energética para vivienda}
\fancyfoot[LE]{\thepage}
\fancyfoot[RE]{Escuela Técnica Superior de Ingenieros Industriales (UPM)}
\fancyfoot[LO]{Luis D. Aranda Sánchez}
\fancyfoot[RO]{\thepage}
\renewcommand{\headrulewidth}{0.4pt}
\renewcommand{\footrulewidth}{0.4pt}

\fancypagestyle{myfancy}{
    \fancyhf{} % Clear all headers and footers
    \fancyhead[LE]{\nouppercase{\leftmark}}
    \fancyhead[RO]{Optimización energética para vivienda}
    \fancyfoot[LE]{\thepage}
    \fancyfoot[RE]{Escuela Técnica Superior de Ingenieros Industriales (UPM)}
    \fancyfoot[LO]{Luis D. Aranda Sánchez}
    \fancyfoot[RO]{\thepage}
    \renewcommand{\headrulewidth}{0.4pt}
    \renewcommand{\footrulewidth}{0.4pt}
}

\fancypagestyle{simple}{
    \fancyhf{} % Clear all headers and footers
    \renewcommand{\headrulewidth}{0pt}
    \renewcommand{\footrulewidth}{0pt}
}

% Line spacing
\setstretch{1.2}

% Document starts here
\begin{document}

% Portada
\begin{titlepage}
    \centering
    {\scshape\LARGE Universidad Politécnica de Madrid \par}
    \vspace{1cm}
    {\scshape\Large Escuela Técnica Superior de Ingenieros Industriales\par}
    \vspace{1.5cm}
    {\huge\bfseries Optimización energética de sistema híbrido con bomba de calor, suelo radiante, fotovoltaica y almacenamiento para vivienda \par}
    \vspace{1.5cm}
    {\Large\bfseries Trabajo de Fin de Máster\par}
    \vspace{0.5cm}
    {\large Máster Universitario en Ingeniería de la Energía \par}
    \vspace{2cm}
    {\Large Luis D. Aranda Sánchez\par}
    \vfill
    Director: Javier Rodríguez Martín
    \vfill
    {\large Septiembre 6, 2024\par}
\end{titlepage}

% Resumen (máximo de 5 páginas, incluyendo al final Palabras clave)
\clearpage
\pagestyle{simple}
% \newpage
\chapter*{Resumen}
\addcontentsline{toc}{chapter}{Resumen}
\input{capitulos/resumen/main.tex}

% Índice (paginado)
\clearpage
\pagestyle{simple}
% \newpage
\tableofcontents

% Introducción (donde se incluya los antecedentes y justificación)
\clearpage
\pagestyle{myfancy}
\newpage
\chapter{Introducción}
\input{capitulos/introduccion/main.tex}

% Objetivos
\chapter{Objetivos}
\input{capitulos/objetivos/main.tex}

% Metodología
\chapter{Metodología}
\input{capitulos/metodologia/main.tex}

% Resultados y discusión (incluyendo la valoración de impactos y de aspectos de responsabilidad legal, ética y profesional relacionados con el trabajo)
\chapter{Resultados y Discusión}
\input{capitulos/resultados_discusion/main.tex}

% Conclusiones
\chapter{Conclusiones}
\input{capitulos/conclusiones/main.tex}

% Planificación temporal y presupuesto
\chapter{Planificación Temporal y Presupuesto}
\input{capitulos/planificacion_presupuesto/main.tex}

% Bibliografía
\newpage
\addcontentsline{toc}{chapter}{Bibliografía}
\printbibliography

\end{document}


% Bibliografía
\newpage
\addcontentsline{toc}{chapter}{Bibliografía}
\printbibliography

\end{document}


% Bibliografía
\newpage
\addcontentsline{toc}{chapter}{Bibliografía}
\printbibliography

\end{document}


% Objetivos
\cleardoublepage
\chapter{Objetivos}
\documentclass[a4paper,11pt,twoside]{report}
\usepackage[left=25mm,right=25mm,top=25mm,bottom=25mm,includehead,includefoot,headsep=15mm,footskip=15mm]{geometry}
\usepackage{graphicx}
\usepackage{fancyhdr}
\usepackage{titlesec}
\usepackage[spanish]{babel}
\usepackage[utf8]{inputenc}
\usepackage{amsmath}
\usepackage{setspace}
\usepackage{svg}
\usepackage{hyperref}
\usepackage[backend=biber,style=numeric]{biblatex}
\addbibresource{references.bib}
\hypersetup{
    colorlinks=true,
    linkcolor=blue,      % color of internal links (sections, etc.)
    urlcolor=blue,       % color of external links
    pdftitle={Optimización energética de sistema híbrido con bomba de calor, suelo radiante, fotovoltaica y almacenamiento para vivienda},    % title
    pdfauthor={Luis D. Aranda Sánchez},     % author
    pdfkeywords={palabra1, palabra2, código1, etc.} % list of keywords
}

% Font change to Arial
\usepackage{helvet}
\renewcommand{\familydefault}{\sfdefault}

% Chapter titles in uppercase and larger font
\titleformat{\chapter}[hang]{\large\bfseries}{\thechapter.}{1em}{\MakeUppercase}
\titleformat{\section}[hang]{\bfseries}{\thesection.}{1em}{}
\titleformat{\subsection}[hang]{\bfseries}{\thesubsection.}{1em}{}

% Fancyhdr setup
\setlength{\headheight}{14.30174pt} % Adjust to recommended value, slightly larger for safety
\fancyhf{} % Clear all headers and footers
\fancyhead[LE]{\nouppercase{\leftmark}}
\fancyhead[RO]{Optimización energética para vivienda}
\fancyfoot[LE]{\thepage}
\fancyfoot[RE]{Escuela Técnica Superior de Ingenieros Industriales (UPM)}
\fancyfoot[LO]{Luis D. Aranda Sánchez}
\fancyfoot[RO]{\thepage}
\renewcommand{\headrulewidth}{0.4pt}
\renewcommand{\footrulewidth}{0.4pt}

\fancypagestyle{myfancy}{
    \fancyhf{} % Clear all headers and footers
    \fancyhead[LE]{\nouppercase{\leftmark}}
    \fancyhead[RO]{Optimización energética para vivienda}
    \fancyfoot[LE]{\thepage}
    \fancyfoot[RE]{Escuela Técnica Superior de Ingenieros Industriales (UPM)}
    \fancyfoot[LO]{Luis D. Aranda Sánchez}
    \fancyfoot[RO]{\thepage}
    \renewcommand{\headrulewidth}{0.4pt}
    \renewcommand{\footrulewidth}{0.4pt}
}

\fancypagestyle{simple}{
    \fancyhf{} % Clear all headers and footers
    \renewcommand{\headrulewidth}{0pt}
    \renewcommand{\footrulewidth}{0pt}
}

% Line spacing
\setstretch{1.2}

% Document starts here
\begin{document}

% Portada
\begin{titlepage}
    \centering
    {\scshape\LARGE Universidad Politécnica de Madrid \par}
    \vspace{1cm}
    {\scshape\Large Escuela Técnica Superior de Ingenieros Industriales\par}
    \vspace{1.5cm}
    {\huge\bfseries Optimización energética de sistema híbrido con bomba de calor, suelo radiante, fotovoltaica y almacenamiento para vivienda \par}
    \vspace{1.5cm}
    {\Large\bfseries Trabajo de Fin de Máster\par}
    \vspace{0.5cm}
    {\large Máster Universitario en Ingeniería de la Energía \par}
    \vspace{2cm}
    {\Large Luis D. Aranda Sánchez\par}
    \vfill
    Director: Javier Rodríguez Martín
    \vfill
    {\large Septiembre 6, 2024\par}
\end{titlepage}

% Resumen (máximo de 5 páginas, incluyendo al final Palabras clave)
\clearpage
\pagestyle{simple}
% \newpage
\chapter*{Resumen}
\addcontentsline{toc}{chapter}{Resumen}
\documentclass[a4paper,11pt,twoside]{report}
\usepackage[left=25mm,right=25mm,top=25mm,bottom=25mm,includehead,includefoot,headsep=15mm,footskip=15mm]{geometry}
\usepackage{graphicx}
\usepackage{fancyhdr}
\usepackage{titlesec}
\usepackage[spanish]{babel}
\usepackage[utf8]{inputenc}
\usepackage{amsmath}
\usepackage{setspace}
\usepackage{svg}
\usepackage{hyperref}
\usepackage[backend=biber,style=numeric]{biblatex}
\addbibresource{references.bib}
\hypersetup{
    colorlinks=true,
    linkcolor=blue,      % color of internal links (sections, etc.)
    urlcolor=blue,       % color of external links
    pdftitle={Optimización energética de sistema híbrido con bomba de calor, suelo radiante, fotovoltaica y almacenamiento para vivienda},    % title
    pdfauthor={Luis D. Aranda Sánchez},     % author
    pdfkeywords={palabra1, palabra2, código1, etc.} % list of keywords
}

% Font change to Arial
\usepackage{helvet}
\renewcommand{\familydefault}{\sfdefault}

% Chapter titles in uppercase and larger font
\titleformat{\chapter}[hang]{\large\bfseries}{\thechapter.}{1em}{\MakeUppercase}
\titleformat{\section}[hang]{\bfseries}{\thesection.}{1em}{}
\titleformat{\subsection}[hang]{\bfseries}{\thesubsection.}{1em}{}

% Fancyhdr setup
\setlength{\headheight}{14.30174pt} % Adjust to recommended value, slightly larger for safety
\fancyhf{} % Clear all headers and footers
\fancyhead[LE]{\nouppercase{\leftmark}}
\fancyhead[RO]{Optimización energética para vivienda}
\fancyfoot[LE]{\thepage}
\fancyfoot[RE]{Escuela Técnica Superior de Ingenieros Industriales (UPM)}
\fancyfoot[LO]{Luis D. Aranda Sánchez}
\fancyfoot[RO]{\thepage}
\renewcommand{\headrulewidth}{0.4pt}
\renewcommand{\footrulewidth}{0.4pt}

\fancypagestyle{myfancy}{
    \fancyhf{} % Clear all headers and footers
    \fancyhead[LE]{\nouppercase{\leftmark}}
    \fancyhead[RO]{Optimización energética para vivienda}
    \fancyfoot[LE]{\thepage}
    \fancyfoot[RE]{Escuela Técnica Superior de Ingenieros Industriales (UPM)}
    \fancyfoot[LO]{Luis D. Aranda Sánchez}
    \fancyfoot[RO]{\thepage}
    \renewcommand{\headrulewidth}{0.4pt}
    \renewcommand{\footrulewidth}{0.4pt}
}

\fancypagestyle{simple}{
    \fancyhf{} % Clear all headers and footers
    \renewcommand{\headrulewidth}{0pt}
    \renewcommand{\footrulewidth}{0pt}
}

% Line spacing
\setstretch{1.2}

% Document starts here
\begin{document}

% Portada
\begin{titlepage}
    \centering
    {\scshape\LARGE Universidad Politécnica de Madrid \par}
    \vspace{1cm}
    {\scshape\Large Escuela Técnica Superior de Ingenieros Industriales\par}
    \vspace{1.5cm}
    {\huge\bfseries Optimización energética de sistema híbrido con bomba de calor, suelo radiante, fotovoltaica y almacenamiento para vivienda \par}
    \vspace{1.5cm}
    {\Large\bfseries Trabajo de Fin de Máster\par}
    \vspace{0.5cm}
    {\large Máster Universitario en Ingeniería de la Energía \par}
    \vspace{2cm}
    {\Large Luis D. Aranda Sánchez\par}
    \vfill
    Director: Javier Rodríguez Martín
    \vfill
    {\large Septiembre 6, 2024\par}
\end{titlepage}

% Resumen (máximo de 5 páginas, incluyendo al final Palabras clave)
\clearpage
\pagestyle{simple}
% \newpage
\chapter*{Resumen}
\addcontentsline{toc}{chapter}{Resumen}
\documentclass[a4paper,11pt,twoside]{report}
\usepackage[left=25mm,right=25mm,top=25mm,bottom=25mm,includehead,includefoot,headsep=15mm,footskip=15mm]{geometry}
\usepackage{graphicx}
\usepackage{fancyhdr}
\usepackage{titlesec}
\usepackage[spanish]{babel}
\usepackage[utf8]{inputenc}
\usepackage{amsmath}
\usepackage{setspace}
\usepackage{svg}
\usepackage{hyperref}
\usepackage[backend=biber,style=numeric]{biblatex}
\addbibresource{references.bib}
\hypersetup{
    colorlinks=true,
    linkcolor=blue,      % color of internal links (sections, etc.)
    urlcolor=blue,       % color of external links
    pdftitle={Optimización energética de sistema híbrido con bomba de calor, suelo radiante, fotovoltaica y almacenamiento para vivienda},    % title
    pdfauthor={Luis D. Aranda Sánchez},     % author
    pdfkeywords={palabra1, palabra2, código1, etc.} % list of keywords
}

% Font change to Arial
\usepackage{helvet}
\renewcommand{\familydefault}{\sfdefault}

% Chapter titles in uppercase and larger font
\titleformat{\chapter}[hang]{\large\bfseries}{\thechapter.}{1em}{\MakeUppercase}
\titleformat{\section}[hang]{\bfseries}{\thesection.}{1em}{}
\titleformat{\subsection}[hang]{\bfseries}{\thesubsection.}{1em}{}

% Fancyhdr setup
\setlength{\headheight}{14.30174pt} % Adjust to recommended value, slightly larger for safety
\fancyhf{} % Clear all headers and footers
\fancyhead[LE]{\nouppercase{\leftmark}}
\fancyhead[RO]{Optimización energética para vivienda}
\fancyfoot[LE]{\thepage}
\fancyfoot[RE]{Escuela Técnica Superior de Ingenieros Industriales (UPM)}
\fancyfoot[LO]{Luis D. Aranda Sánchez}
\fancyfoot[RO]{\thepage}
\renewcommand{\headrulewidth}{0.4pt}
\renewcommand{\footrulewidth}{0.4pt}

\fancypagestyle{myfancy}{
    \fancyhf{} % Clear all headers and footers
    \fancyhead[LE]{\nouppercase{\leftmark}}
    \fancyhead[RO]{Optimización energética para vivienda}
    \fancyfoot[LE]{\thepage}
    \fancyfoot[RE]{Escuela Técnica Superior de Ingenieros Industriales (UPM)}
    \fancyfoot[LO]{Luis D. Aranda Sánchez}
    \fancyfoot[RO]{\thepage}
    \renewcommand{\headrulewidth}{0.4pt}
    \renewcommand{\footrulewidth}{0.4pt}
}

\fancypagestyle{simple}{
    \fancyhf{} % Clear all headers and footers
    \renewcommand{\headrulewidth}{0pt}
    \renewcommand{\footrulewidth}{0pt}
}

% Line spacing
\setstretch{1.2}

% Document starts here
\begin{document}

% Portada
\begin{titlepage}
    \centering
    {\scshape\LARGE Universidad Politécnica de Madrid \par}
    \vspace{1cm}
    {\scshape\Large Escuela Técnica Superior de Ingenieros Industriales\par}
    \vspace{1.5cm}
    {\huge\bfseries Optimización energética de sistema híbrido con bomba de calor, suelo radiante, fotovoltaica y almacenamiento para vivienda \par}
    \vspace{1.5cm}
    {\Large\bfseries Trabajo de Fin de Máster\par}
    \vspace{0.5cm}
    {\large Máster Universitario en Ingeniería de la Energía \par}
    \vspace{2cm}
    {\Large Luis D. Aranda Sánchez\par}
    \vfill
    Director: Javier Rodríguez Martín
    \vfill
    {\large Septiembre 6, 2024\par}
\end{titlepage}

% Resumen (máximo de 5 páginas, incluyendo al final Palabras clave)
\clearpage
\pagestyle{simple}
% \newpage
\chapter*{Resumen}
\addcontentsline{toc}{chapter}{Resumen}
\input{capitulos/resumen/main.tex}

% Índice (paginado)
\clearpage
\pagestyle{simple}
% \newpage
\tableofcontents

% Introducción (donde se incluya los antecedentes y justificación)
\clearpage
\pagestyle{myfancy}
\newpage
\chapter{Introducción}
\input{capitulos/introduccion/main.tex}

% Objetivos
\chapter{Objetivos}
\input{capitulos/objetivos/main.tex}

% Metodología
\chapter{Metodología}
\input{capitulos/metodologia/main.tex}

% Resultados y discusión (incluyendo la valoración de impactos y de aspectos de responsabilidad legal, ética y profesional relacionados con el trabajo)
\chapter{Resultados y Discusión}
\input{capitulos/resultados_discusion/main.tex}

% Conclusiones
\chapter{Conclusiones}
\input{capitulos/conclusiones/main.tex}

% Planificación temporal y presupuesto
\chapter{Planificación Temporal y Presupuesto}
\input{capitulos/planificacion_presupuesto/main.tex}

% Bibliografía
\newpage
\addcontentsline{toc}{chapter}{Bibliografía}
\printbibliography

\end{document}


% Índice (paginado)
\clearpage
\pagestyle{simple}
% \newpage
\tableofcontents

% Introducción (donde se incluya los antecedentes y justificación)
\clearpage
\pagestyle{myfancy}
\newpage
\chapter{Introducción}
\documentclass[a4paper,11pt,twoside]{report}
\usepackage[left=25mm,right=25mm,top=25mm,bottom=25mm,includehead,includefoot,headsep=15mm,footskip=15mm]{geometry}
\usepackage{graphicx}
\usepackage{fancyhdr}
\usepackage{titlesec}
\usepackage[spanish]{babel}
\usepackage[utf8]{inputenc}
\usepackage{amsmath}
\usepackage{setspace}
\usepackage{svg}
\usepackage{hyperref}
\usepackage[backend=biber,style=numeric]{biblatex}
\addbibresource{references.bib}
\hypersetup{
    colorlinks=true,
    linkcolor=blue,      % color of internal links (sections, etc.)
    urlcolor=blue,       % color of external links
    pdftitle={Optimización energética de sistema híbrido con bomba de calor, suelo radiante, fotovoltaica y almacenamiento para vivienda},    % title
    pdfauthor={Luis D. Aranda Sánchez},     % author
    pdfkeywords={palabra1, palabra2, código1, etc.} % list of keywords
}

% Font change to Arial
\usepackage{helvet}
\renewcommand{\familydefault}{\sfdefault}

% Chapter titles in uppercase and larger font
\titleformat{\chapter}[hang]{\large\bfseries}{\thechapter.}{1em}{\MakeUppercase}
\titleformat{\section}[hang]{\bfseries}{\thesection.}{1em}{}
\titleformat{\subsection}[hang]{\bfseries}{\thesubsection.}{1em}{}

% Fancyhdr setup
\setlength{\headheight}{14.30174pt} % Adjust to recommended value, slightly larger for safety
\fancyhf{} % Clear all headers and footers
\fancyhead[LE]{\nouppercase{\leftmark}}
\fancyhead[RO]{Optimización energética para vivienda}
\fancyfoot[LE]{\thepage}
\fancyfoot[RE]{Escuela Técnica Superior de Ingenieros Industriales (UPM)}
\fancyfoot[LO]{Luis D. Aranda Sánchez}
\fancyfoot[RO]{\thepage}
\renewcommand{\headrulewidth}{0.4pt}
\renewcommand{\footrulewidth}{0.4pt}

\fancypagestyle{myfancy}{
    \fancyhf{} % Clear all headers and footers
    \fancyhead[LE]{\nouppercase{\leftmark}}
    \fancyhead[RO]{Optimización energética para vivienda}
    \fancyfoot[LE]{\thepage}
    \fancyfoot[RE]{Escuela Técnica Superior de Ingenieros Industriales (UPM)}
    \fancyfoot[LO]{Luis D. Aranda Sánchez}
    \fancyfoot[RO]{\thepage}
    \renewcommand{\headrulewidth}{0.4pt}
    \renewcommand{\footrulewidth}{0.4pt}
}

\fancypagestyle{simple}{
    \fancyhf{} % Clear all headers and footers
    \renewcommand{\headrulewidth}{0pt}
    \renewcommand{\footrulewidth}{0pt}
}

% Line spacing
\setstretch{1.2}

% Document starts here
\begin{document}

% Portada
\begin{titlepage}
    \centering
    {\scshape\LARGE Universidad Politécnica de Madrid \par}
    \vspace{1cm}
    {\scshape\Large Escuela Técnica Superior de Ingenieros Industriales\par}
    \vspace{1.5cm}
    {\huge\bfseries Optimización energética de sistema híbrido con bomba de calor, suelo radiante, fotovoltaica y almacenamiento para vivienda \par}
    \vspace{1.5cm}
    {\Large\bfseries Trabajo de Fin de Máster\par}
    \vspace{0.5cm}
    {\large Máster Universitario en Ingeniería de la Energía \par}
    \vspace{2cm}
    {\Large Luis D. Aranda Sánchez\par}
    \vfill
    Director: Javier Rodríguez Martín
    \vfill
    {\large Septiembre 6, 2024\par}
\end{titlepage}

% Resumen (máximo de 5 páginas, incluyendo al final Palabras clave)
\clearpage
\pagestyle{simple}
% \newpage
\chapter*{Resumen}
\addcontentsline{toc}{chapter}{Resumen}
\input{capitulos/resumen/main.tex}

% Índice (paginado)
\clearpage
\pagestyle{simple}
% \newpage
\tableofcontents

% Introducción (donde se incluya los antecedentes y justificación)
\clearpage
\pagestyle{myfancy}
\newpage
\chapter{Introducción}
\input{capitulos/introduccion/main.tex}

% Objetivos
\chapter{Objetivos}
\input{capitulos/objetivos/main.tex}

% Metodología
\chapter{Metodología}
\input{capitulos/metodologia/main.tex}

% Resultados y discusión (incluyendo la valoración de impactos y de aspectos de responsabilidad legal, ética y profesional relacionados con el trabajo)
\chapter{Resultados y Discusión}
\input{capitulos/resultados_discusion/main.tex}

% Conclusiones
\chapter{Conclusiones}
\input{capitulos/conclusiones/main.tex}

% Planificación temporal y presupuesto
\chapter{Planificación Temporal y Presupuesto}
\input{capitulos/planificacion_presupuesto/main.tex}

% Bibliografía
\newpage
\addcontentsline{toc}{chapter}{Bibliografía}
\printbibliography

\end{document}


% Objetivos
\chapter{Objetivos}
\documentclass[a4paper,11pt,twoside]{report}
\usepackage[left=25mm,right=25mm,top=25mm,bottom=25mm,includehead,includefoot,headsep=15mm,footskip=15mm]{geometry}
\usepackage{graphicx}
\usepackage{fancyhdr}
\usepackage{titlesec}
\usepackage[spanish]{babel}
\usepackage[utf8]{inputenc}
\usepackage{amsmath}
\usepackage{setspace}
\usepackage{svg}
\usepackage{hyperref}
\usepackage[backend=biber,style=numeric]{biblatex}
\addbibresource{references.bib}
\hypersetup{
    colorlinks=true,
    linkcolor=blue,      % color of internal links (sections, etc.)
    urlcolor=blue,       % color of external links
    pdftitle={Optimización energética de sistema híbrido con bomba de calor, suelo radiante, fotovoltaica y almacenamiento para vivienda},    % title
    pdfauthor={Luis D. Aranda Sánchez},     % author
    pdfkeywords={palabra1, palabra2, código1, etc.} % list of keywords
}

% Font change to Arial
\usepackage{helvet}
\renewcommand{\familydefault}{\sfdefault}

% Chapter titles in uppercase and larger font
\titleformat{\chapter}[hang]{\large\bfseries}{\thechapter.}{1em}{\MakeUppercase}
\titleformat{\section}[hang]{\bfseries}{\thesection.}{1em}{}
\titleformat{\subsection}[hang]{\bfseries}{\thesubsection.}{1em}{}

% Fancyhdr setup
\setlength{\headheight}{14.30174pt} % Adjust to recommended value, slightly larger for safety
\fancyhf{} % Clear all headers and footers
\fancyhead[LE]{\nouppercase{\leftmark}}
\fancyhead[RO]{Optimización energética para vivienda}
\fancyfoot[LE]{\thepage}
\fancyfoot[RE]{Escuela Técnica Superior de Ingenieros Industriales (UPM)}
\fancyfoot[LO]{Luis D. Aranda Sánchez}
\fancyfoot[RO]{\thepage}
\renewcommand{\headrulewidth}{0.4pt}
\renewcommand{\footrulewidth}{0.4pt}

\fancypagestyle{myfancy}{
    \fancyhf{} % Clear all headers and footers
    \fancyhead[LE]{\nouppercase{\leftmark}}
    \fancyhead[RO]{Optimización energética para vivienda}
    \fancyfoot[LE]{\thepage}
    \fancyfoot[RE]{Escuela Técnica Superior de Ingenieros Industriales (UPM)}
    \fancyfoot[LO]{Luis D. Aranda Sánchez}
    \fancyfoot[RO]{\thepage}
    \renewcommand{\headrulewidth}{0.4pt}
    \renewcommand{\footrulewidth}{0.4pt}
}

\fancypagestyle{simple}{
    \fancyhf{} % Clear all headers and footers
    \renewcommand{\headrulewidth}{0pt}
    \renewcommand{\footrulewidth}{0pt}
}

% Line spacing
\setstretch{1.2}

% Document starts here
\begin{document}

% Portada
\begin{titlepage}
    \centering
    {\scshape\LARGE Universidad Politécnica de Madrid \par}
    \vspace{1cm}
    {\scshape\Large Escuela Técnica Superior de Ingenieros Industriales\par}
    \vspace{1.5cm}
    {\huge\bfseries Optimización energética de sistema híbrido con bomba de calor, suelo radiante, fotovoltaica y almacenamiento para vivienda \par}
    \vspace{1.5cm}
    {\Large\bfseries Trabajo de Fin de Máster\par}
    \vspace{0.5cm}
    {\large Máster Universitario en Ingeniería de la Energía \par}
    \vspace{2cm}
    {\Large Luis D. Aranda Sánchez\par}
    \vfill
    Director: Javier Rodríguez Martín
    \vfill
    {\large Septiembre 6, 2024\par}
\end{titlepage}

% Resumen (máximo de 5 páginas, incluyendo al final Palabras clave)
\clearpage
\pagestyle{simple}
% \newpage
\chapter*{Resumen}
\addcontentsline{toc}{chapter}{Resumen}
\input{capitulos/resumen/main.tex}

% Índice (paginado)
\clearpage
\pagestyle{simple}
% \newpage
\tableofcontents

% Introducción (donde se incluya los antecedentes y justificación)
\clearpage
\pagestyle{myfancy}
\newpage
\chapter{Introducción}
\input{capitulos/introduccion/main.tex}

% Objetivos
\chapter{Objetivos}
\input{capitulos/objetivos/main.tex}

% Metodología
\chapter{Metodología}
\input{capitulos/metodologia/main.tex}

% Resultados y discusión (incluyendo la valoración de impactos y de aspectos de responsabilidad legal, ética y profesional relacionados con el trabajo)
\chapter{Resultados y Discusión}
\input{capitulos/resultados_discusion/main.tex}

% Conclusiones
\chapter{Conclusiones}
\input{capitulos/conclusiones/main.tex}

% Planificación temporal y presupuesto
\chapter{Planificación Temporal y Presupuesto}
\input{capitulos/planificacion_presupuesto/main.tex}

% Bibliografía
\newpage
\addcontentsline{toc}{chapter}{Bibliografía}
\printbibliography

\end{document}


% Metodología
\chapter{Metodología}
\documentclass[a4paper,11pt,twoside]{report}
\usepackage[left=25mm,right=25mm,top=25mm,bottom=25mm,includehead,includefoot,headsep=15mm,footskip=15mm]{geometry}
\usepackage{graphicx}
\usepackage{fancyhdr}
\usepackage{titlesec}
\usepackage[spanish]{babel}
\usepackage[utf8]{inputenc}
\usepackage{amsmath}
\usepackage{setspace}
\usepackage{svg}
\usepackage{hyperref}
\usepackage[backend=biber,style=numeric]{biblatex}
\addbibresource{references.bib}
\hypersetup{
    colorlinks=true,
    linkcolor=blue,      % color of internal links (sections, etc.)
    urlcolor=blue,       % color of external links
    pdftitle={Optimización energética de sistema híbrido con bomba de calor, suelo radiante, fotovoltaica y almacenamiento para vivienda},    % title
    pdfauthor={Luis D. Aranda Sánchez},     % author
    pdfkeywords={palabra1, palabra2, código1, etc.} % list of keywords
}

% Font change to Arial
\usepackage{helvet}
\renewcommand{\familydefault}{\sfdefault}

% Chapter titles in uppercase and larger font
\titleformat{\chapter}[hang]{\large\bfseries}{\thechapter.}{1em}{\MakeUppercase}
\titleformat{\section}[hang]{\bfseries}{\thesection.}{1em}{}
\titleformat{\subsection}[hang]{\bfseries}{\thesubsection.}{1em}{}

% Fancyhdr setup
\setlength{\headheight}{14.30174pt} % Adjust to recommended value, slightly larger for safety
\fancyhf{} % Clear all headers and footers
\fancyhead[LE]{\nouppercase{\leftmark}}
\fancyhead[RO]{Optimización energética para vivienda}
\fancyfoot[LE]{\thepage}
\fancyfoot[RE]{Escuela Técnica Superior de Ingenieros Industriales (UPM)}
\fancyfoot[LO]{Luis D. Aranda Sánchez}
\fancyfoot[RO]{\thepage}
\renewcommand{\headrulewidth}{0.4pt}
\renewcommand{\footrulewidth}{0.4pt}

\fancypagestyle{myfancy}{
    \fancyhf{} % Clear all headers and footers
    \fancyhead[LE]{\nouppercase{\leftmark}}
    \fancyhead[RO]{Optimización energética para vivienda}
    \fancyfoot[LE]{\thepage}
    \fancyfoot[RE]{Escuela Técnica Superior de Ingenieros Industriales (UPM)}
    \fancyfoot[LO]{Luis D. Aranda Sánchez}
    \fancyfoot[RO]{\thepage}
    \renewcommand{\headrulewidth}{0.4pt}
    \renewcommand{\footrulewidth}{0.4pt}
}

\fancypagestyle{simple}{
    \fancyhf{} % Clear all headers and footers
    \renewcommand{\headrulewidth}{0pt}
    \renewcommand{\footrulewidth}{0pt}
}

% Line spacing
\setstretch{1.2}

% Document starts here
\begin{document}

% Portada
\begin{titlepage}
    \centering
    {\scshape\LARGE Universidad Politécnica de Madrid \par}
    \vspace{1cm}
    {\scshape\Large Escuela Técnica Superior de Ingenieros Industriales\par}
    \vspace{1.5cm}
    {\huge\bfseries Optimización energética de sistema híbrido con bomba de calor, suelo radiante, fotovoltaica y almacenamiento para vivienda \par}
    \vspace{1.5cm}
    {\Large\bfseries Trabajo de Fin de Máster\par}
    \vspace{0.5cm}
    {\large Máster Universitario en Ingeniería de la Energía \par}
    \vspace{2cm}
    {\Large Luis D. Aranda Sánchez\par}
    \vfill
    Director: Javier Rodríguez Martín
    \vfill
    {\large Septiembre 6, 2024\par}
\end{titlepage}

% Resumen (máximo de 5 páginas, incluyendo al final Palabras clave)
\clearpage
\pagestyle{simple}
% \newpage
\chapter*{Resumen}
\addcontentsline{toc}{chapter}{Resumen}
\input{capitulos/resumen/main.tex}

% Índice (paginado)
\clearpage
\pagestyle{simple}
% \newpage
\tableofcontents

% Introducción (donde se incluya los antecedentes y justificación)
\clearpage
\pagestyle{myfancy}
\newpage
\chapter{Introducción}
\input{capitulos/introduccion/main.tex}

% Objetivos
\chapter{Objetivos}
\input{capitulos/objetivos/main.tex}

% Metodología
\chapter{Metodología}
\input{capitulos/metodologia/main.tex}

% Resultados y discusión (incluyendo la valoración de impactos y de aspectos de responsabilidad legal, ética y profesional relacionados con el trabajo)
\chapter{Resultados y Discusión}
\input{capitulos/resultados_discusion/main.tex}

% Conclusiones
\chapter{Conclusiones}
\input{capitulos/conclusiones/main.tex}

% Planificación temporal y presupuesto
\chapter{Planificación Temporal y Presupuesto}
\input{capitulos/planificacion_presupuesto/main.tex}

% Bibliografía
\newpage
\addcontentsline{toc}{chapter}{Bibliografía}
\printbibliography

\end{document}


% Resultados y discusión (incluyendo la valoración de impactos y de aspectos de responsabilidad legal, ética y profesional relacionados con el trabajo)
\chapter{Resultados y Discusión}
\documentclass[a4paper,11pt,twoside]{report}
\usepackage[left=25mm,right=25mm,top=25mm,bottom=25mm,includehead,includefoot,headsep=15mm,footskip=15mm]{geometry}
\usepackage{graphicx}
\usepackage{fancyhdr}
\usepackage{titlesec}
\usepackage[spanish]{babel}
\usepackage[utf8]{inputenc}
\usepackage{amsmath}
\usepackage{setspace}
\usepackage{svg}
\usepackage{hyperref}
\usepackage[backend=biber,style=numeric]{biblatex}
\addbibresource{references.bib}
\hypersetup{
    colorlinks=true,
    linkcolor=blue,      % color of internal links (sections, etc.)
    urlcolor=blue,       % color of external links
    pdftitle={Optimización energética de sistema híbrido con bomba de calor, suelo radiante, fotovoltaica y almacenamiento para vivienda},    % title
    pdfauthor={Luis D. Aranda Sánchez},     % author
    pdfkeywords={palabra1, palabra2, código1, etc.} % list of keywords
}

% Font change to Arial
\usepackage{helvet}
\renewcommand{\familydefault}{\sfdefault}

% Chapter titles in uppercase and larger font
\titleformat{\chapter}[hang]{\large\bfseries}{\thechapter.}{1em}{\MakeUppercase}
\titleformat{\section}[hang]{\bfseries}{\thesection.}{1em}{}
\titleformat{\subsection}[hang]{\bfseries}{\thesubsection.}{1em}{}

% Fancyhdr setup
\setlength{\headheight}{14.30174pt} % Adjust to recommended value, slightly larger for safety
\fancyhf{} % Clear all headers and footers
\fancyhead[LE]{\nouppercase{\leftmark}}
\fancyhead[RO]{Optimización energética para vivienda}
\fancyfoot[LE]{\thepage}
\fancyfoot[RE]{Escuela Técnica Superior de Ingenieros Industriales (UPM)}
\fancyfoot[LO]{Luis D. Aranda Sánchez}
\fancyfoot[RO]{\thepage}
\renewcommand{\headrulewidth}{0.4pt}
\renewcommand{\footrulewidth}{0.4pt}

\fancypagestyle{myfancy}{
    \fancyhf{} % Clear all headers and footers
    \fancyhead[LE]{\nouppercase{\leftmark}}
    \fancyhead[RO]{Optimización energética para vivienda}
    \fancyfoot[LE]{\thepage}
    \fancyfoot[RE]{Escuela Técnica Superior de Ingenieros Industriales (UPM)}
    \fancyfoot[LO]{Luis D. Aranda Sánchez}
    \fancyfoot[RO]{\thepage}
    \renewcommand{\headrulewidth}{0.4pt}
    \renewcommand{\footrulewidth}{0.4pt}
}

\fancypagestyle{simple}{
    \fancyhf{} % Clear all headers and footers
    \renewcommand{\headrulewidth}{0pt}
    \renewcommand{\footrulewidth}{0pt}
}

% Line spacing
\setstretch{1.2}

% Document starts here
\begin{document}

% Portada
\begin{titlepage}
    \centering
    {\scshape\LARGE Universidad Politécnica de Madrid \par}
    \vspace{1cm}
    {\scshape\Large Escuela Técnica Superior de Ingenieros Industriales\par}
    \vspace{1.5cm}
    {\huge\bfseries Optimización energética de sistema híbrido con bomba de calor, suelo radiante, fotovoltaica y almacenamiento para vivienda \par}
    \vspace{1.5cm}
    {\Large\bfseries Trabajo de Fin de Máster\par}
    \vspace{0.5cm}
    {\large Máster Universitario en Ingeniería de la Energía \par}
    \vspace{2cm}
    {\Large Luis D. Aranda Sánchez\par}
    \vfill
    Director: Javier Rodríguez Martín
    \vfill
    {\large Septiembre 6, 2024\par}
\end{titlepage}

% Resumen (máximo de 5 páginas, incluyendo al final Palabras clave)
\clearpage
\pagestyle{simple}
% \newpage
\chapter*{Resumen}
\addcontentsline{toc}{chapter}{Resumen}
\input{capitulos/resumen/main.tex}

% Índice (paginado)
\clearpage
\pagestyle{simple}
% \newpage
\tableofcontents

% Introducción (donde se incluya los antecedentes y justificación)
\clearpage
\pagestyle{myfancy}
\newpage
\chapter{Introducción}
\input{capitulos/introduccion/main.tex}

% Objetivos
\chapter{Objetivos}
\input{capitulos/objetivos/main.tex}

% Metodología
\chapter{Metodología}
\input{capitulos/metodologia/main.tex}

% Resultados y discusión (incluyendo la valoración de impactos y de aspectos de responsabilidad legal, ética y profesional relacionados con el trabajo)
\chapter{Resultados y Discusión}
\input{capitulos/resultados_discusion/main.tex}

% Conclusiones
\chapter{Conclusiones}
\input{capitulos/conclusiones/main.tex}

% Planificación temporal y presupuesto
\chapter{Planificación Temporal y Presupuesto}
\input{capitulos/planificacion_presupuesto/main.tex}

% Bibliografía
\newpage
\addcontentsline{toc}{chapter}{Bibliografía}
\printbibliography

\end{document}


% Conclusiones
\chapter{Conclusiones}
\documentclass[a4paper,11pt,twoside]{report}
\usepackage[left=25mm,right=25mm,top=25mm,bottom=25mm,includehead,includefoot,headsep=15mm,footskip=15mm]{geometry}
\usepackage{graphicx}
\usepackage{fancyhdr}
\usepackage{titlesec}
\usepackage[spanish]{babel}
\usepackage[utf8]{inputenc}
\usepackage{amsmath}
\usepackage{setspace}
\usepackage{svg}
\usepackage{hyperref}
\usepackage[backend=biber,style=numeric]{biblatex}
\addbibresource{references.bib}
\hypersetup{
    colorlinks=true,
    linkcolor=blue,      % color of internal links (sections, etc.)
    urlcolor=blue,       % color of external links
    pdftitle={Optimización energética de sistema híbrido con bomba de calor, suelo radiante, fotovoltaica y almacenamiento para vivienda},    % title
    pdfauthor={Luis D. Aranda Sánchez},     % author
    pdfkeywords={palabra1, palabra2, código1, etc.} % list of keywords
}

% Font change to Arial
\usepackage{helvet}
\renewcommand{\familydefault}{\sfdefault}

% Chapter titles in uppercase and larger font
\titleformat{\chapter}[hang]{\large\bfseries}{\thechapter.}{1em}{\MakeUppercase}
\titleformat{\section}[hang]{\bfseries}{\thesection.}{1em}{}
\titleformat{\subsection}[hang]{\bfseries}{\thesubsection.}{1em}{}

% Fancyhdr setup
\setlength{\headheight}{14.30174pt} % Adjust to recommended value, slightly larger for safety
\fancyhf{} % Clear all headers and footers
\fancyhead[LE]{\nouppercase{\leftmark}}
\fancyhead[RO]{Optimización energética para vivienda}
\fancyfoot[LE]{\thepage}
\fancyfoot[RE]{Escuela Técnica Superior de Ingenieros Industriales (UPM)}
\fancyfoot[LO]{Luis D. Aranda Sánchez}
\fancyfoot[RO]{\thepage}
\renewcommand{\headrulewidth}{0.4pt}
\renewcommand{\footrulewidth}{0.4pt}

\fancypagestyle{myfancy}{
    \fancyhf{} % Clear all headers and footers
    \fancyhead[LE]{\nouppercase{\leftmark}}
    \fancyhead[RO]{Optimización energética para vivienda}
    \fancyfoot[LE]{\thepage}
    \fancyfoot[RE]{Escuela Técnica Superior de Ingenieros Industriales (UPM)}
    \fancyfoot[LO]{Luis D. Aranda Sánchez}
    \fancyfoot[RO]{\thepage}
    \renewcommand{\headrulewidth}{0.4pt}
    \renewcommand{\footrulewidth}{0.4pt}
}

\fancypagestyle{simple}{
    \fancyhf{} % Clear all headers and footers
    \renewcommand{\headrulewidth}{0pt}
    \renewcommand{\footrulewidth}{0pt}
}

% Line spacing
\setstretch{1.2}

% Document starts here
\begin{document}

% Portada
\begin{titlepage}
    \centering
    {\scshape\LARGE Universidad Politécnica de Madrid \par}
    \vspace{1cm}
    {\scshape\Large Escuela Técnica Superior de Ingenieros Industriales\par}
    \vspace{1.5cm}
    {\huge\bfseries Optimización energética de sistema híbrido con bomba de calor, suelo radiante, fotovoltaica y almacenamiento para vivienda \par}
    \vspace{1.5cm}
    {\Large\bfseries Trabajo de Fin de Máster\par}
    \vspace{0.5cm}
    {\large Máster Universitario en Ingeniería de la Energía \par}
    \vspace{2cm}
    {\Large Luis D. Aranda Sánchez\par}
    \vfill
    Director: Javier Rodríguez Martín
    \vfill
    {\large Septiembre 6, 2024\par}
\end{titlepage}

% Resumen (máximo de 5 páginas, incluyendo al final Palabras clave)
\clearpage
\pagestyle{simple}
% \newpage
\chapter*{Resumen}
\addcontentsline{toc}{chapter}{Resumen}
\input{capitulos/resumen/main.tex}

% Índice (paginado)
\clearpage
\pagestyle{simple}
% \newpage
\tableofcontents

% Introducción (donde se incluya los antecedentes y justificación)
\clearpage
\pagestyle{myfancy}
\newpage
\chapter{Introducción}
\input{capitulos/introduccion/main.tex}

% Objetivos
\chapter{Objetivos}
\input{capitulos/objetivos/main.tex}

% Metodología
\chapter{Metodología}
\input{capitulos/metodologia/main.tex}

% Resultados y discusión (incluyendo la valoración de impactos y de aspectos de responsabilidad legal, ética y profesional relacionados con el trabajo)
\chapter{Resultados y Discusión}
\input{capitulos/resultados_discusion/main.tex}

% Conclusiones
\chapter{Conclusiones}
\input{capitulos/conclusiones/main.tex}

% Planificación temporal y presupuesto
\chapter{Planificación Temporal y Presupuesto}
\input{capitulos/planificacion_presupuesto/main.tex}

% Bibliografía
\newpage
\addcontentsline{toc}{chapter}{Bibliografía}
\printbibliography

\end{document}


% Planificación temporal y presupuesto
\chapter{Planificación Temporal y Presupuesto}
\documentclass[a4paper,11pt,twoside]{report}
\usepackage[left=25mm,right=25mm,top=25mm,bottom=25mm,includehead,includefoot,headsep=15mm,footskip=15mm]{geometry}
\usepackage{graphicx}
\usepackage{fancyhdr}
\usepackage{titlesec}
\usepackage[spanish]{babel}
\usepackage[utf8]{inputenc}
\usepackage{amsmath}
\usepackage{setspace}
\usepackage{svg}
\usepackage{hyperref}
\usepackage[backend=biber,style=numeric]{biblatex}
\addbibresource{references.bib}
\hypersetup{
    colorlinks=true,
    linkcolor=blue,      % color of internal links (sections, etc.)
    urlcolor=blue,       % color of external links
    pdftitle={Optimización energética de sistema híbrido con bomba de calor, suelo radiante, fotovoltaica y almacenamiento para vivienda},    % title
    pdfauthor={Luis D. Aranda Sánchez},     % author
    pdfkeywords={palabra1, palabra2, código1, etc.} % list of keywords
}

% Font change to Arial
\usepackage{helvet}
\renewcommand{\familydefault}{\sfdefault}

% Chapter titles in uppercase and larger font
\titleformat{\chapter}[hang]{\large\bfseries}{\thechapter.}{1em}{\MakeUppercase}
\titleformat{\section}[hang]{\bfseries}{\thesection.}{1em}{}
\titleformat{\subsection}[hang]{\bfseries}{\thesubsection.}{1em}{}

% Fancyhdr setup
\setlength{\headheight}{14.30174pt} % Adjust to recommended value, slightly larger for safety
\fancyhf{} % Clear all headers and footers
\fancyhead[LE]{\nouppercase{\leftmark}}
\fancyhead[RO]{Optimización energética para vivienda}
\fancyfoot[LE]{\thepage}
\fancyfoot[RE]{Escuela Técnica Superior de Ingenieros Industriales (UPM)}
\fancyfoot[LO]{Luis D. Aranda Sánchez}
\fancyfoot[RO]{\thepage}
\renewcommand{\headrulewidth}{0.4pt}
\renewcommand{\footrulewidth}{0.4pt}

\fancypagestyle{myfancy}{
    \fancyhf{} % Clear all headers and footers
    \fancyhead[LE]{\nouppercase{\leftmark}}
    \fancyhead[RO]{Optimización energética para vivienda}
    \fancyfoot[LE]{\thepage}
    \fancyfoot[RE]{Escuela Técnica Superior de Ingenieros Industriales (UPM)}
    \fancyfoot[LO]{Luis D. Aranda Sánchez}
    \fancyfoot[RO]{\thepage}
    \renewcommand{\headrulewidth}{0.4pt}
    \renewcommand{\footrulewidth}{0.4pt}
}

\fancypagestyle{simple}{
    \fancyhf{} % Clear all headers and footers
    \renewcommand{\headrulewidth}{0pt}
    \renewcommand{\footrulewidth}{0pt}
}

% Line spacing
\setstretch{1.2}

% Document starts here
\begin{document}

% Portada
\begin{titlepage}
    \centering
    {\scshape\LARGE Universidad Politécnica de Madrid \par}
    \vspace{1cm}
    {\scshape\Large Escuela Técnica Superior de Ingenieros Industriales\par}
    \vspace{1.5cm}
    {\huge\bfseries Optimización energética de sistema híbrido con bomba de calor, suelo radiante, fotovoltaica y almacenamiento para vivienda \par}
    \vspace{1.5cm}
    {\Large\bfseries Trabajo de Fin de Máster\par}
    \vspace{0.5cm}
    {\large Máster Universitario en Ingeniería de la Energía \par}
    \vspace{2cm}
    {\Large Luis D. Aranda Sánchez\par}
    \vfill
    Director: Javier Rodríguez Martín
    \vfill
    {\large Septiembre 6, 2024\par}
\end{titlepage}

% Resumen (máximo de 5 páginas, incluyendo al final Palabras clave)
\clearpage
\pagestyle{simple}
% \newpage
\chapter*{Resumen}
\addcontentsline{toc}{chapter}{Resumen}
\input{capitulos/resumen/main.tex}

% Índice (paginado)
\clearpage
\pagestyle{simple}
% \newpage
\tableofcontents

% Introducción (donde se incluya los antecedentes y justificación)
\clearpage
\pagestyle{myfancy}
\newpage
\chapter{Introducción}
\input{capitulos/introduccion/main.tex}

% Objetivos
\chapter{Objetivos}
\input{capitulos/objetivos/main.tex}

% Metodología
\chapter{Metodología}
\input{capitulos/metodologia/main.tex}

% Resultados y discusión (incluyendo la valoración de impactos y de aspectos de responsabilidad legal, ética y profesional relacionados con el trabajo)
\chapter{Resultados y Discusión}
\input{capitulos/resultados_discusion/main.tex}

% Conclusiones
\chapter{Conclusiones}
\input{capitulos/conclusiones/main.tex}

% Planificación temporal y presupuesto
\chapter{Planificación Temporal y Presupuesto}
\input{capitulos/planificacion_presupuesto/main.tex}

% Bibliografía
\newpage
\addcontentsline{toc}{chapter}{Bibliografía}
\printbibliography

\end{document}


% Bibliografía
\newpage
\addcontentsline{toc}{chapter}{Bibliografía}
\printbibliography

\end{document}


% Índice (paginado)
\clearpage
\pagestyle{simple}
% \newpage
\tableofcontents

% Introducción (donde se incluya los antecedentes y justificación)
\clearpage
\pagestyle{myfancy}
\newpage
\chapter{Introducción}
\documentclass[a4paper,11pt,twoside]{report}
\usepackage[left=25mm,right=25mm,top=25mm,bottom=25mm,includehead,includefoot,headsep=15mm,footskip=15mm]{geometry}
\usepackage{graphicx}
\usepackage{fancyhdr}
\usepackage{titlesec}
\usepackage[spanish]{babel}
\usepackage[utf8]{inputenc}
\usepackage{amsmath}
\usepackage{setspace}
\usepackage{svg}
\usepackage{hyperref}
\usepackage[backend=biber,style=numeric]{biblatex}
\addbibresource{references.bib}
\hypersetup{
    colorlinks=true,
    linkcolor=blue,      % color of internal links (sections, etc.)
    urlcolor=blue,       % color of external links
    pdftitle={Optimización energética de sistema híbrido con bomba de calor, suelo radiante, fotovoltaica y almacenamiento para vivienda},    % title
    pdfauthor={Luis D. Aranda Sánchez},     % author
    pdfkeywords={palabra1, palabra2, código1, etc.} % list of keywords
}

% Font change to Arial
\usepackage{helvet}
\renewcommand{\familydefault}{\sfdefault}

% Chapter titles in uppercase and larger font
\titleformat{\chapter}[hang]{\large\bfseries}{\thechapter.}{1em}{\MakeUppercase}
\titleformat{\section}[hang]{\bfseries}{\thesection.}{1em}{}
\titleformat{\subsection}[hang]{\bfseries}{\thesubsection.}{1em}{}

% Fancyhdr setup
\setlength{\headheight}{14.30174pt} % Adjust to recommended value, slightly larger for safety
\fancyhf{} % Clear all headers and footers
\fancyhead[LE]{\nouppercase{\leftmark}}
\fancyhead[RO]{Optimización energética para vivienda}
\fancyfoot[LE]{\thepage}
\fancyfoot[RE]{Escuela Técnica Superior de Ingenieros Industriales (UPM)}
\fancyfoot[LO]{Luis D. Aranda Sánchez}
\fancyfoot[RO]{\thepage}
\renewcommand{\headrulewidth}{0.4pt}
\renewcommand{\footrulewidth}{0.4pt}

\fancypagestyle{myfancy}{
    \fancyhf{} % Clear all headers and footers
    \fancyhead[LE]{\nouppercase{\leftmark}}
    \fancyhead[RO]{Optimización energética para vivienda}
    \fancyfoot[LE]{\thepage}
    \fancyfoot[RE]{Escuela Técnica Superior de Ingenieros Industriales (UPM)}
    \fancyfoot[LO]{Luis D. Aranda Sánchez}
    \fancyfoot[RO]{\thepage}
    \renewcommand{\headrulewidth}{0.4pt}
    \renewcommand{\footrulewidth}{0.4pt}
}

\fancypagestyle{simple}{
    \fancyhf{} % Clear all headers and footers
    \renewcommand{\headrulewidth}{0pt}
    \renewcommand{\footrulewidth}{0pt}
}

% Line spacing
\setstretch{1.2}

% Document starts here
\begin{document}

% Portada
\begin{titlepage}
    \centering
    {\scshape\LARGE Universidad Politécnica de Madrid \par}
    \vspace{1cm}
    {\scshape\Large Escuela Técnica Superior de Ingenieros Industriales\par}
    \vspace{1.5cm}
    {\huge\bfseries Optimización energética de sistema híbrido con bomba de calor, suelo radiante, fotovoltaica y almacenamiento para vivienda \par}
    \vspace{1.5cm}
    {\Large\bfseries Trabajo de Fin de Máster\par}
    \vspace{0.5cm}
    {\large Máster Universitario en Ingeniería de la Energía \par}
    \vspace{2cm}
    {\Large Luis D. Aranda Sánchez\par}
    \vfill
    Director: Javier Rodríguez Martín
    \vfill
    {\large Septiembre 6, 2024\par}
\end{titlepage}

% Resumen (máximo de 5 páginas, incluyendo al final Palabras clave)
\clearpage
\pagestyle{simple}
% \newpage
\chapter*{Resumen}
\addcontentsline{toc}{chapter}{Resumen}
\documentclass[a4paper,11pt,twoside]{report}
\usepackage[left=25mm,right=25mm,top=25mm,bottom=25mm,includehead,includefoot,headsep=15mm,footskip=15mm]{geometry}
\usepackage{graphicx}
\usepackage{fancyhdr}
\usepackage{titlesec}
\usepackage[spanish]{babel}
\usepackage[utf8]{inputenc}
\usepackage{amsmath}
\usepackage{setspace}
\usepackage{svg}
\usepackage{hyperref}
\usepackage[backend=biber,style=numeric]{biblatex}
\addbibresource{references.bib}
\hypersetup{
    colorlinks=true,
    linkcolor=blue,      % color of internal links (sections, etc.)
    urlcolor=blue,       % color of external links
    pdftitle={Optimización energética de sistema híbrido con bomba de calor, suelo radiante, fotovoltaica y almacenamiento para vivienda},    % title
    pdfauthor={Luis D. Aranda Sánchez},     % author
    pdfkeywords={palabra1, palabra2, código1, etc.} % list of keywords
}

% Font change to Arial
\usepackage{helvet}
\renewcommand{\familydefault}{\sfdefault}

% Chapter titles in uppercase and larger font
\titleformat{\chapter}[hang]{\large\bfseries}{\thechapter.}{1em}{\MakeUppercase}
\titleformat{\section}[hang]{\bfseries}{\thesection.}{1em}{}
\titleformat{\subsection}[hang]{\bfseries}{\thesubsection.}{1em}{}

% Fancyhdr setup
\setlength{\headheight}{14.30174pt} % Adjust to recommended value, slightly larger for safety
\fancyhf{} % Clear all headers and footers
\fancyhead[LE]{\nouppercase{\leftmark}}
\fancyhead[RO]{Optimización energética para vivienda}
\fancyfoot[LE]{\thepage}
\fancyfoot[RE]{Escuela Técnica Superior de Ingenieros Industriales (UPM)}
\fancyfoot[LO]{Luis D. Aranda Sánchez}
\fancyfoot[RO]{\thepage}
\renewcommand{\headrulewidth}{0.4pt}
\renewcommand{\footrulewidth}{0.4pt}

\fancypagestyle{myfancy}{
    \fancyhf{} % Clear all headers and footers
    \fancyhead[LE]{\nouppercase{\leftmark}}
    \fancyhead[RO]{Optimización energética para vivienda}
    \fancyfoot[LE]{\thepage}
    \fancyfoot[RE]{Escuela Técnica Superior de Ingenieros Industriales (UPM)}
    \fancyfoot[LO]{Luis D. Aranda Sánchez}
    \fancyfoot[RO]{\thepage}
    \renewcommand{\headrulewidth}{0.4pt}
    \renewcommand{\footrulewidth}{0.4pt}
}

\fancypagestyle{simple}{
    \fancyhf{} % Clear all headers and footers
    \renewcommand{\headrulewidth}{0pt}
    \renewcommand{\footrulewidth}{0pt}
}

% Line spacing
\setstretch{1.2}

% Document starts here
\begin{document}

% Portada
\begin{titlepage}
    \centering
    {\scshape\LARGE Universidad Politécnica de Madrid \par}
    \vspace{1cm}
    {\scshape\Large Escuela Técnica Superior de Ingenieros Industriales\par}
    \vspace{1.5cm}
    {\huge\bfseries Optimización energética de sistema híbrido con bomba de calor, suelo radiante, fotovoltaica y almacenamiento para vivienda \par}
    \vspace{1.5cm}
    {\Large\bfseries Trabajo de Fin de Máster\par}
    \vspace{0.5cm}
    {\large Máster Universitario en Ingeniería de la Energía \par}
    \vspace{2cm}
    {\Large Luis D. Aranda Sánchez\par}
    \vfill
    Director: Javier Rodríguez Martín
    \vfill
    {\large Septiembre 6, 2024\par}
\end{titlepage}

% Resumen (máximo de 5 páginas, incluyendo al final Palabras clave)
\clearpage
\pagestyle{simple}
% \newpage
\chapter*{Resumen}
\addcontentsline{toc}{chapter}{Resumen}
\input{capitulos/resumen/main.tex}

% Índice (paginado)
\clearpage
\pagestyle{simple}
% \newpage
\tableofcontents

% Introducción (donde se incluya los antecedentes y justificación)
\clearpage
\pagestyle{myfancy}
\newpage
\chapter{Introducción}
\input{capitulos/introduccion/main.tex}

% Objetivos
\chapter{Objetivos}
\input{capitulos/objetivos/main.tex}

% Metodología
\chapter{Metodología}
\input{capitulos/metodologia/main.tex}

% Resultados y discusión (incluyendo la valoración de impactos y de aspectos de responsabilidad legal, ética y profesional relacionados con el trabajo)
\chapter{Resultados y Discusión}
\input{capitulos/resultados_discusion/main.tex}

% Conclusiones
\chapter{Conclusiones}
\input{capitulos/conclusiones/main.tex}

% Planificación temporal y presupuesto
\chapter{Planificación Temporal y Presupuesto}
\input{capitulos/planificacion_presupuesto/main.tex}

% Bibliografía
\newpage
\addcontentsline{toc}{chapter}{Bibliografía}
\printbibliography

\end{document}


% Índice (paginado)
\clearpage
\pagestyle{simple}
% \newpage
\tableofcontents

% Introducción (donde se incluya los antecedentes y justificación)
\clearpage
\pagestyle{myfancy}
\newpage
\chapter{Introducción}
\documentclass[a4paper,11pt,twoside]{report}
\usepackage[left=25mm,right=25mm,top=25mm,bottom=25mm,includehead,includefoot,headsep=15mm,footskip=15mm]{geometry}
\usepackage{graphicx}
\usepackage{fancyhdr}
\usepackage{titlesec}
\usepackage[spanish]{babel}
\usepackage[utf8]{inputenc}
\usepackage{amsmath}
\usepackage{setspace}
\usepackage{svg}
\usepackage{hyperref}
\usepackage[backend=biber,style=numeric]{biblatex}
\addbibresource{references.bib}
\hypersetup{
    colorlinks=true,
    linkcolor=blue,      % color of internal links (sections, etc.)
    urlcolor=blue,       % color of external links
    pdftitle={Optimización energética de sistema híbrido con bomba de calor, suelo radiante, fotovoltaica y almacenamiento para vivienda},    % title
    pdfauthor={Luis D. Aranda Sánchez},     % author
    pdfkeywords={palabra1, palabra2, código1, etc.} % list of keywords
}

% Font change to Arial
\usepackage{helvet}
\renewcommand{\familydefault}{\sfdefault}

% Chapter titles in uppercase and larger font
\titleformat{\chapter}[hang]{\large\bfseries}{\thechapter.}{1em}{\MakeUppercase}
\titleformat{\section}[hang]{\bfseries}{\thesection.}{1em}{}
\titleformat{\subsection}[hang]{\bfseries}{\thesubsection.}{1em}{}

% Fancyhdr setup
\setlength{\headheight}{14.30174pt} % Adjust to recommended value, slightly larger for safety
\fancyhf{} % Clear all headers and footers
\fancyhead[LE]{\nouppercase{\leftmark}}
\fancyhead[RO]{Optimización energética para vivienda}
\fancyfoot[LE]{\thepage}
\fancyfoot[RE]{Escuela Técnica Superior de Ingenieros Industriales (UPM)}
\fancyfoot[LO]{Luis D. Aranda Sánchez}
\fancyfoot[RO]{\thepage}
\renewcommand{\headrulewidth}{0.4pt}
\renewcommand{\footrulewidth}{0.4pt}

\fancypagestyle{myfancy}{
    \fancyhf{} % Clear all headers and footers
    \fancyhead[LE]{\nouppercase{\leftmark}}
    \fancyhead[RO]{Optimización energética para vivienda}
    \fancyfoot[LE]{\thepage}
    \fancyfoot[RE]{Escuela Técnica Superior de Ingenieros Industriales (UPM)}
    \fancyfoot[LO]{Luis D. Aranda Sánchez}
    \fancyfoot[RO]{\thepage}
    \renewcommand{\headrulewidth}{0.4pt}
    \renewcommand{\footrulewidth}{0.4pt}
}

\fancypagestyle{simple}{
    \fancyhf{} % Clear all headers and footers
    \renewcommand{\headrulewidth}{0pt}
    \renewcommand{\footrulewidth}{0pt}
}

% Line spacing
\setstretch{1.2}

% Document starts here
\begin{document}

% Portada
\begin{titlepage}
    \centering
    {\scshape\LARGE Universidad Politécnica de Madrid \par}
    \vspace{1cm}
    {\scshape\Large Escuela Técnica Superior de Ingenieros Industriales\par}
    \vspace{1.5cm}
    {\huge\bfseries Optimización energética de sistema híbrido con bomba de calor, suelo radiante, fotovoltaica y almacenamiento para vivienda \par}
    \vspace{1.5cm}
    {\Large\bfseries Trabajo de Fin de Máster\par}
    \vspace{0.5cm}
    {\large Máster Universitario en Ingeniería de la Energía \par}
    \vspace{2cm}
    {\Large Luis D. Aranda Sánchez\par}
    \vfill
    Director: Javier Rodríguez Martín
    \vfill
    {\large Septiembre 6, 2024\par}
\end{titlepage}

% Resumen (máximo de 5 páginas, incluyendo al final Palabras clave)
\clearpage
\pagestyle{simple}
% \newpage
\chapter*{Resumen}
\addcontentsline{toc}{chapter}{Resumen}
\input{capitulos/resumen/main.tex}

% Índice (paginado)
\clearpage
\pagestyle{simple}
% \newpage
\tableofcontents

% Introducción (donde se incluya los antecedentes y justificación)
\clearpage
\pagestyle{myfancy}
\newpage
\chapter{Introducción}
\input{capitulos/introduccion/main.tex}

% Objetivos
\chapter{Objetivos}
\input{capitulos/objetivos/main.tex}

% Metodología
\chapter{Metodología}
\input{capitulos/metodologia/main.tex}

% Resultados y discusión (incluyendo la valoración de impactos y de aspectos de responsabilidad legal, ética y profesional relacionados con el trabajo)
\chapter{Resultados y Discusión}
\input{capitulos/resultados_discusion/main.tex}

% Conclusiones
\chapter{Conclusiones}
\input{capitulos/conclusiones/main.tex}

% Planificación temporal y presupuesto
\chapter{Planificación Temporal y Presupuesto}
\input{capitulos/planificacion_presupuesto/main.tex}

% Bibliografía
\newpage
\addcontentsline{toc}{chapter}{Bibliografía}
\printbibliography

\end{document}


% Objetivos
\chapter{Objetivos}
\documentclass[a4paper,11pt,twoside]{report}
\usepackage[left=25mm,right=25mm,top=25mm,bottom=25mm,includehead,includefoot,headsep=15mm,footskip=15mm]{geometry}
\usepackage{graphicx}
\usepackage{fancyhdr}
\usepackage{titlesec}
\usepackage[spanish]{babel}
\usepackage[utf8]{inputenc}
\usepackage{amsmath}
\usepackage{setspace}
\usepackage{svg}
\usepackage{hyperref}
\usepackage[backend=biber,style=numeric]{biblatex}
\addbibresource{references.bib}
\hypersetup{
    colorlinks=true,
    linkcolor=blue,      % color of internal links (sections, etc.)
    urlcolor=blue,       % color of external links
    pdftitle={Optimización energética de sistema híbrido con bomba de calor, suelo radiante, fotovoltaica y almacenamiento para vivienda},    % title
    pdfauthor={Luis D. Aranda Sánchez},     % author
    pdfkeywords={palabra1, palabra2, código1, etc.} % list of keywords
}

% Font change to Arial
\usepackage{helvet}
\renewcommand{\familydefault}{\sfdefault}

% Chapter titles in uppercase and larger font
\titleformat{\chapter}[hang]{\large\bfseries}{\thechapter.}{1em}{\MakeUppercase}
\titleformat{\section}[hang]{\bfseries}{\thesection.}{1em}{}
\titleformat{\subsection}[hang]{\bfseries}{\thesubsection.}{1em}{}

% Fancyhdr setup
\setlength{\headheight}{14.30174pt} % Adjust to recommended value, slightly larger for safety
\fancyhf{} % Clear all headers and footers
\fancyhead[LE]{\nouppercase{\leftmark}}
\fancyhead[RO]{Optimización energética para vivienda}
\fancyfoot[LE]{\thepage}
\fancyfoot[RE]{Escuela Técnica Superior de Ingenieros Industriales (UPM)}
\fancyfoot[LO]{Luis D. Aranda Sánchez}
\fancyfoot[RO]{\thepage}
\renewcommand{\headrulewidth}{0.4pt}
\renewcommand{\footrulewidth}{0.4pt}

\fancypagestyle{myfancy}{
    \fancyhf{} % Clear all headers and footers
    \fancyhead[LE]{\nouppercase{\leftmark}}
    \fancyhead[RO]{Optimización energética para vivienda}
    \fancyfoot[LE]{\thepage}
    \fancyfoot[RE]{Escuela Técnica Superior de Ingenieros Industriales (UPM)}
    \fancyfoot[LO]{Luis D. Aranda Sánchez}
    \fancyfoot[RO]{\thepage}
    \renewcommand{\headrulewidth}{0.4pt}
    \renewcommand{\footrulewidth}{0.4pt}
}

\fancypagestyle{simple}{
    \fancyhf{} % Clear all headers and footers
    \renewcommand{\headrulewidth}{0pt}
    \renewcommand{\footrulewidth}{0pt}
}

% Line spacing
\setstretch{1.2}

% Document starts here
\begin{document}

% Portada
\begin{titlepage}
    \centering
    {\scshape\LARGE Universidad Politécnica de Madrid \par}
    \vspace{1cm}
    {\scshape\Large Escuela Técnica Superior de Ingenieros Industriales\par}
    \vspace{1.5cm}
    {\huge\bfseries Optimización energética de sistema híbrido con bomba de calor, suelo radiante, fotovoltaica y almacenamiento para vivienda \par}
    \vspace{1.5cm}
    {\Large\bfseries Trabajo de Fin de Máster\par}
    \vspace{0.5cm}
    {\large Máster Universitario en Ingeniería de la Energía \par}
    \vspace{2cm}
    {\Large Luis D. Aranda Sánchez\par}
    \vfill
    Director: Javier Rodríguez Martín
    \vfill
    {\large Septiembre 6, 2024\par}
\end{titlepage}

% Resumen (máximo de 5 páginas, incluyendo al final Palabras clave)
\clearpage
\pagestyle{simple}
% \newpage
\chapter*{Resumen}
\addcontentsline{toc}{chapter}{Resumen}
\input{capitulos/resumen/main.tex}

% Índice (paginado)
\clearpage
\pagestyle{simple}
% \newpage
\tableofcontents

% Introducción (donde se incluya los antecedentes y justificación)
\clearpage
\pagestyle{myfancy}
\newpage
\chapter{Introducción}
\input{capitulos/introduccion/main.tex}

% Objetivos
\chapter{Objetivos}
\input{capitulos/objetivos/main.tex}

% Metodología
\chapter{Metodología}
\input{capitulos/metodologia/main.tex}

% Resultados y discusión (incluyendo la valoración de impactos y de aspectos de responsabilidad legal, ética y profesional relacionados con el trabajo)
\chapter{Resultados y Discusión}
\input{capitulos/resultados_discusion/main.tex}

% Conclusiones
\chapter{Conclusiones}
\input{capitulos/conclusiones/main.tex}

% Planificación temporal y presupuesto
\chapter{Planificación Temporal y Presupuesto}
\input{capitulos/planificacion_presupuesto/main.tex}

% Bibliografía
\newpage
\addcontentsline{toc}{chapter}{Bibliografía}
\printbibliography

\end{document}


% Metodología
\chapter{Metodología}
\documentclass[a4paper,11pt,twoside]{report}
\usepackage[left=25mm,right=25mm,top=25mm,bottom=25mm,includehead,includefoot,headsep=15mm,footskip=15mm]{geometry}
\usepackage{graphicx}
\usepackage{fancyhdr}
\usepackage{titlesec}
\usepackage[spanish]{babel}
\usepackage[utf8]{inputenc}
\usepackage{amsmath}
\usepackage{setspace}
\usepackage{svg}
\usepackage{hyperref}
\usepackage[backend=biber,style=numeric]{biblatex}
\addbibresource{references.bib}
\hypersetup{
    colorlinks=true,
    linkcolor=blue,      % color of internal links (sections, etc.)
    urlcolor=blue,       % color of external links
    pdftitle={Optimización energética de sistema híbrido con bomba de calor, suelo radiante, fotovoltaica y almacenamiento para vivienda},    % title
    pdfauthor={Luis D. Aranda Sánchez},     % author
    pdfkeywords={palabra1, palabra2, código1, etc.} % list of keywords
}

% Font change to Arial
\usepackage{helvet}
\renewcommand{\familydefault}{\sfdefault}

% Chapter titles in uppercase and larger font
\titleformat{\chapter}[hang]{\large\bfseries}{\thechapter.}{1em}{\MakeUppercase}
\titleformat{\section}[hang]{\bfseries}{\thesection.}{1em}{}
\titleformat{\subsection}[hang]{\bfseries}{\thesubsection.}{1em}{}

% Fancyhdr setup
\setlength{\headheight}{14.30174pt} % Adjust to recommended value, slightly larger for safety
\fancyhf{} % Clear all headers and footers
\fancyhead[LE]{\nouppercase{\leftmark}}
\fancyhead[RO]{Optimización energética para vivienda}
\fancyfoot[LE]{\thepage}
\fancyfoot[RE]{Escuela Técnica Superior de Ingenieros Industriales (UPM)}
\fancyfoot[LO]{Luis D. Aranda Sánchez}
\fancyfoot[RO]{\thepage}
\renewcommand{\headrulewidth}{0.4pt}
\renewcommand{\footrulewidth}{0.4pt}

\fancypagestyle{myfancy}{
    \fancyhf{} % Clear all headers and footers
    \fancyhead[LE]{\nouppercase{\leftmark}}
    \fancyhead[RO]{Optimización energética para vivienda}
    \fancyfoot[LE]{\thepage}
    \fancyfoot[RE]{Escuela Técnica Superior de Ingenieros Industriales (UPM)}
    \fancyfoot[LO]{Luis D. Aranda Sánchez}
    \fancyfoot[RO]{\thepage}
    \renewcommand{\headrulewidth}{0.4pt}
    \renewcommand{\footrulewidth}{0.4pt}
}

\fancypagestyle{simple}{
    \fancyhf{} % Clear all headers and footers
    \renewcommand{\headrulewidth}{0pt}
    \renewcommand{\footrulewidth}{0pt}
}

% Line spacing
\setstretch{1.2}

% Document starts here
\begin{document}

% Portada
\begin{titlepage}
    \centering
    {\scshape\LARGE Universidad Politécnica de Madrid \par}
    \vspace{1cm}
    {\scshape\Large Escuela Técnica Superior de Ingenieros Industriales\par}
    \vspace{1.5cm}
    {\huge\bfseries Optimización energética de sistema híbrido con bomba de calor, suelo radiante, fotovoltaica y almacenamiento para vivienda \par}
    \vspace{1.5cm}
    {\Large\bfseries Trabajo de Fin de Máster\par}
    \vspace{0.5cm}
    {\large Máster Universitario en Ingeniería de la Energía \par}
    \vspace{2cm}
    {\Large Luis D. Aranda Sánchez\par}
    \vfill
    Director: Javier Rodríguez Martín
    \vfill
    {\large Septiembre 6, 2024\par}
\end{titlepage}

% Resumen (máximo de 5 páginas, incluyendo al final Palabras clave)
\clearpage
\pagestyle{simple}
% \newpage
\chapter*{Resumen}
\addcontentsline{toc}{chapter}{Resumen}
\input{capitulos/resumen/main.tex}

% Índice (paginado)
\clearpage
\pagestyle{simple}
% \newpage
\tableofcontents

% Introducción (donde se incluya los antecedentes y justificación)
\clearpage
\pagestyle{myfancy}
\newpage
\chapter{Introducción}
\input{capitulos/introduccion/main.tex}

% Objetivos
\chapter{Objetivos}
\input{capitulos/objetivos/main.tex}

% Metodología
\chapter{Metodología}
\input{capitulos/metodologia/main.tex}

% Resultados y discusión (incluyendo la valoración de impactos y de aspectos de responsabilidad legal, ética y profesional relacionados con el trabajo)
\chapter{Resultados y Discusión}
\input{capitulos/resultados_discusion/main.tex}

% Conclusiones
\chapter{Conclusiones}
\input{capitulos/conclusiones/main.tex}

% Planificación temporal y presupuesto
\chapter{Planificación Temporal y Presupuesto}
\input{capitulos/planificacion_presupuesto/main.tex}

% Bibliografía
\newpage
\addcontentsline{toc}{chapter}{Bibliografía}
\printbibliography

\end{document}


% Resultados y discusión (incluyendo la valoración de impactos y de aspectos de responsabilidad legal, ética y profesional relacionados con el trabajo)
\chapter{Resultados y Discusión}
\documentclass[a4paper,11pt,twoside]{report}
\usepackage[left=25mm,right=25mm,top=25mm,bottom=25mm,includehead,includefoot,headsep=15mm,footskip=15mm]{geometry}
\usepackage{graphicx}
\usepackage{fancyhdr}
\usepackage{titlesec}
\usepackage[spanish]{babel}
\usepackage[utf8]{inputenc}
\usepackage{amsmath}
\usepackage{setspace}
\usepackage{svg}
\usepackage{hyperref}
\usepackage[backend=biber,style=numeric]{biblatex}
\addbibresource{references.bib}
\hypersetup{
    colorlinks=true,
    linkcolor=blue,      % color of internal links (sections, etc.)
    urlcolor=blue,       % color of external links
    pdftitle={Optimización energética de sistema híbrido con bomba de calor, suelo radiante, fotovoltaica y almacenamiento para vivienda},    % title
    pdfauthor={Luis D. Aranda Sánchez},     % author
    pdfkeywords={palabra1, palabra2, código1, etc.} % list of keywords
}

% Font change to Arial
\usepackage{helvet}
\renewcommand{\familydefault}{\sfdefault}

% Chapter titles in uppercase and larger font
\titleformat{\chapter}[hang]{\large\bfseries}{\thechapter.}{1em}{\MakeUppercase}
\titleformat{\section}[hang]{\bfseries}{\thesection.}{1em}{}
\titleformat{\subsection}[hang]{\bfseries}{\thesubsection.}{1em}{}

% Fancyhdr setup
\setlength{\headheight}{14.30174pt} % Adjust to recommended value, slightly larger for safety
\fancyhf{} % Clear all headers and footers
\fancyhead[LE]{\nouppercase{\leftmark}}
\fancyhead[RO]{Optimización energética para vivienda}
\fancyfoot[LE]{\thepage}
\fancyfoot[RE]{Escuela Técnica Superior de Ingenieros Industriales (UPM)}
\fancyfoot[LO]{Luis D. Aranda Sánchez}
\fancyfoot[RO]{\thepage}
\renewcommand{\headrulewidth}{0.4pt}
\renewcommand{\footrulewidth}{0.4pt}

\fancypagestyle{myfancy}{
    \fancyhf{} % Clear all headers and footers
    \fancyhead[LE]{\nouppercase{\leftmark}}
    \fancyhead[RO]{Optimización energética para vivienda}
    \fancyfoot[LE]{\thepage}
    \fancyfoot[RE]{Escuela Técnica Superior de Ingenieros Industriales (UPM)}
    \fancyfoot[LO]{Luis D. Aranda Sánchez}
    \fancyfoot[RO]{\thepage}
    \renewcommand{\headrulewidth}{0.4pt}
    \renewcommand{\footrulewidth}{0.4pt}
}

\fancypagestyle{simple}{
    \fancyhf{} % Clear all headers and footers
    \renewcommand{\headrulewidth}{0pt}
    \renewcommand{\footrulewidth}{0pt}
}

% Line spacing
\setstretch{1.2}

% Document starts here
\begin{document}

% Portada
\begin{titlepage}
    \centering
    {\scshape\LARGE Universidad Politécnica de Madrid \par}
    \vspace{1cm}
    {\scshape\Large Escuela Técnica Superior de Ingenieros Industriales\par}
    \vspace{1.5cm}
    {\huge\bfseries Optimización energética de sistema híbrido con bomba de calor, suelo radiante, fotovoltaica y almacenamiento para vivienda \par}
    \vspace{1.5cm}
    {\Large\bfseries Trabajo de Fin de Máster\par}
    \vspace{0.5cm}
    {\large Máster Universitario en Ingeniería de la Energía \par}
    \vspace{2cm}
    {\Large Luis D. Aranda Sánchez\par}
    \vfill
    Director: Javier Rodríguez Martín
    \vfill
    {\large Septiembre 6, 2024\par}
\end{titlepage}

% Resumen (máximo de 5 páginas, incluyendo al final Palabras clave)
\clearpage
\pagestyle{simple}
% \newpage
\chapter*{Resumen}
\addcontentsline{toc}{chapter}{Resumen}
\input{capitulos/resumen/main.tex}

% Índice (paginado)
\clearpage
\pagestyle{simple}
% \newpage
\tableofcontents

% Introducción (donde se incluya los antecedentes y justificación)
\clearpage
\pagestyle{myfancy}
\newpage
\chapter{Introducción}
\input{capitulos/introduccion/main.tex}

% Objetivos
\chapter{Objetivos}
\input{capitulos/objetivos/main.tex}

% Metodología
\chapter{Metodología}
\input{capitulos/metodologia/main.tex}

% Resultados y discusión (incluyendo la valoración de impactos y de aspectos de responsabilidad legal, ética y profesional relacionados con el trabajo)
\chapter{Resultados y Discusión}
\input{capitulos/resultados_discusion/main.tex}

% Conclusiones
\chapter{Conclusiones}
\input{capitulos/conclusiones/main.tex}

% Planificación temporal y presupuesto
\chapter{Planificación Temporal y Presupuesto}
\input{capitulos/planificacion_presupuesto/main.tex}

% Bibliografía
\newpage
\addcontentsline{toc}{chapter}{Bibliografía}
\printbibliography

\end{document}


% Conclusiones
\chapter{Conclusiones}
\documentclass[a4paper,11pt,twoside]{report}
\usepackage[left=25mm,right=25mm,top=25mm,bottom=25mm,includehead,includefoot,headsep=15mm,footskip=15mm]{geometry}
\usepackage{graphicx}
\usepackage{fancyhdr}
\usepackage{titlesec}
\usepackage[spanish]{babel}
\usepackage[utf8]{inputenc}
\usepackage{amsmath}
\usepackage{setspace}
\usepackage{svg}
\usepackage{hyperref}
\usepackage[backend=biber,style=numeric]{biblatex}
\addbibresource{references.bib}
\hypersetup{
    colorlinks=true,
    linkcolor=blue,      % color of internal links (sections, etc.)
    urlcolor=blue,       % color of external links
    pdftitle={Optimización energética de sistema híbrido con bomba de calor, suelo radiante, fotovoltaica y almacenamiento para vivienda},    % title
    pdfauthor={Luis D. Aranda Sánchez},     % author
    pdfkeywords={palabra1, palabra2, código1, etc.} % list of keywords
}

% Font change to Arial
\usepackage{helvet}
\renewcommand{\familydefault}{\sfdefault}

% Chapter titles in uppercase and larger font
\titleformat{\chapter}[hang]{\large\bfseries}{\thechapter.}{1em}{\MakeUppercase}
\titleformat{\section}[hang]{\bfseries}{\thesection.}{1em}{}
\titleformat{\subsection}[hang]{\bfseries}{\thesubsection.}{1em}{}

% Fancyhdr setup
\setlength{\headheight}{14.30174pt} % Adjust to recommended value, slightly larger for safety
\fancyhf{} % Clear all headers and footers
\fancyhead[LE]{\nouppercase{\leftmark}}
\fancyhead[RO]{Optimización energética para vivienda}
\fancyfoot[LE]{\thepage}
\fancyfoot[RE]{Escuela Técnica Superior de Ingenieros Industriales (UPM)}
\fancyfoot[LO]{Luis D. Aranda Sánchez}
\fancyfoot[RO]{\thepage}
\renewcommand{\headrulewidth}{0.4pt}
\renewcommand{\footrulewidth}{0.4pt}

\fancypagestyle{myfancy}{
    \fancyhf{} % Clear all headers and footers
    \fancyhead[LE]{\nouppercase{\leftmark}}
    \fancyhead[RO]{Optimización energética para vivienda}
    \fancyfoot[LE]{\thepage}
    \fancyfoot[RE]{Escuela Técnica Superior de Ingenieros Industriales (UPM)}
    \fancyfoot[LO]{Luis D. Aranda Sánchez}
    \fancyfoot[RO]{\thepage}
    \renewcommand{\headrulewidth}{0.4pt}
    \renewcommand{\footrulewidth}{0.4pt}
}

\fancypagestyle{simple}{
    \fancyhf{} % Clear all headers and footers
    \renewcommand{\headrulewidth}{0pt}
    \renewcommand{\footrulewidth}{0pt}
}

% Line spacing
\setstretch{1.2}

% Document starts here
\begin{document}

% Portada
\begin{titlepage}
    \centering
    {\scshape\LARGE Universidad Politécnica de Madrid \par}
    \vspace{1cm}
    {\scshape\Large Escuela Técnica Superior de Ingenieros Industriales\par}
    \vspace{1.5cm}
    {\huge\bfseries Optimización energética de sistema híbrido con bomba de calor, suelo radiante, fotovoltaica y almacenamiento para vivienda \par}
    \vspace{1.5cm}
    {\Large\bfseries Trabajo de Fin de Máster\par}
    \vspace{0.5cm}
    {\large Máster Universitario en Ingeniería de la Energía \par}
    \vspace{2cm}
    {\Large Luis D. Aranda Sánchez\par}
    \vfill
    Director: Javier Rodríguez Martín
    \vfill
    {\large Septiembre 6, 2024\par}
\end{titlepage}

% Resumen (máximo de 5 páginas, incluyendo al final Palabras clave)
\clearpage
\pagestyle{simple}
% \newpage
\chapter*{Resumen}
\addcontentsline{toc}{chapter}{Resumen}
\input{capitulos/resumen/main.tex}

% Índice (paginado)
\clearpage
\pagestyle{simple}
% \newpage
\tableofcontents

% Introducción (donde se incluya los antecedentes y justificación)
\clearpage
\pagestyle{myfancy}
\newpage
\chapter{Introducción}
\input{capitulos/introduccion/main.tex}

% Objetivos
\chapter{Objetivos}
\input{capitulos/objetivos/main.tex}

% Metodología
\chapter{Metodología}
\input{capitulos/metodologia/main.tex}

% Resultados y discusión (incluyendo la valoración de impactos y de aspectos de responsabilidad legal, ética y profesional relacionados con el trabajo)
\chapter{Resultados y Discusión}
\input{capitulos/resultados_discusion/main.tex}

% Conclusiones
\chapter{Conclusiones}
\input{capitulos/conclusiones/main.tex}

% Planificación temporal y presupuesto
\chapter{Planificación Temporal y Presupuesto}
\input{capitulos/planificacion_presupuesto/main.tex}

% Bibliografía
\newpage
\addcontentsline{toc}{chapter}{Bibliografía}
\printbibliography

\end{document}


% Planificación temporal y presupuesto
\chapter{Planificación Temporal y Presupuesto}
\documentclass[a4paper,11pt,twoside]{report}
\usepackage[left=25mm,right=25mm,top=25mm,bottom=25mm,includehead,includefoot,headsep=15mm,footskip=15mm]{geometry}
\usepackage{graphicx}
\usepackage{fancyhdr}
\usepackage{titlesec}
\usepackage[spanish]{babel}
\usepackage[utf8]{inputenc}
\usepackage{amsmath}
\usepackage{setspace}
\usepackage{svg}
\usepackage{hyperref}
\usepackage[backend=biber,style=numeric]{biblatex}
\addbibresource{references.bib}
\hypersetup{
    colorlinks=true,
    linkcolor=blue,      % color of internal links (sections, etc.)
    urlcolor=blue,       % color of external links
    pdftitle={Optimización energética de sistema híbrido con bomba de calor, suelo radiante, fotovoltaica y almacenamiento para vivienda},    % title
    pdfauthor={Luis D. Aranda Sánchez},     % author
    pdfkeywords={palabra1, palabra2, código1, etc.} % list of keywords
}

% Font change to Arial
\usepackage{helvet}
\renewcommand{\familydefault}{\sfdefault}

% Chapter titles in uppercase and larger font
\titleformat{\chapter}[hang]{\large\bfseries}{\thechapter.}{1em}{\MakeUppercase}
\titleformat{\section}[hang]{\bfseries}{\thesection.}{1em}{}
\titleformat{\subsection}[hang]{\bfseries}{\thesubsection.}{1em}{}

% Fancyhdr setup
\setlength{\headheight}{14.30174pt} % Adjust to recommended value, slightly larger for safety
\fancyhf{} % Clear all headers and footers
\fancyhead[LE]{\nouppercase{\leftmark}}
\fancyhead[RO]{Optimización energética para vivienda}
\fancyfoot[LE]{\thepage}
\fancyfoot[RE]{Escuela Técnica Superior de Ingenieros Industriales (UPM)}
\fancyfoot[LO]{Luis D. Aranda Sánchez}
\fancyfoot[RO]{\thepage}
\renewcommand{\headrulewidth}{0.4pt}
\renewcommand{\footrulewidth}{0.4pt}

\fancypagestyle{myfancy}{
    \fancyhf{} % Clear all headers and footers
    \fancyhead[LE]{\nouppercase{\leftmark}}
    \fancyhead[RO]{Optimización energética para vivienda}
    \fancyfoot[LE]{\thepage}
    \fancyfoot[RE]{Escuela Técnica Superior de Ingenieros Industriales (UPM)}
    \fancyfoot[LO]{Luis D. Aranda Sánchez}
    \fancyfoot[RO]{\thepage}
    \renewcommand{\headrulewidth}{0.4pt}
    \renewcommand{\footrulewidth}{0.4pt}
}

\fancypagestyle{simple}{
    \fancyhf{} % Clear all headers and footers
    \renewcommand{\headrulewidth}{0pt}
    \renewcommand{\footrulewidth}{0pt}
}

% Line spacing
\setstretch{1.2}

% Document starts here
\begin{document}

% Portada
\begin{titlepage}
    \centering
    {\scshape\LARGE Universidad Politécnica de Madrid \par}
    \vspace{1cm}
    {\scshape\Large Escuela Técnica Superior de Ingenieros Industriales\par}
    \vspace{1.5cm}
    {\huge\bfseries Optimización energética de sistema híbrido con bomba de calor, suelo radiante, fotovoltaica y almacenamiento para vivienda \par}
    \vspace{1.5cm}
    {\Large\bfseries Trabajo de Fin de Máster\par}
    \vspace{0.5cm}
    {\large Máster Universitario en Ingeniería de la Energía \par}
    \vspace{2cm}
    {\Large Luis D. Aranda Sánchez\par}
    \vfill
    Director: Javier Rodríguez Martín
    \vfill
    {\large Septiembre 6, 2024\par}
\end{titlepage}

% Resumen (máximo de 5 páginas, incluyendo al final Palabras clave)
\clearpage
\pagestyle{simple}
% \newpage
\chapter*{Resumen}
\addcontentsline{toc}{chapter}{Resumen}
\input{capitulos/resumen/main.tex}

% Índice (paginado)
\clearpage
\pagestyle{simple}
% \newpage
\tableofcontents

% Introducción (donde se incluya los antecedentes y justificación)
\clearpage
\pagestyle{myfancy}
\newpage
\chapter{Introducción}
\input{capitulos/introduccion/main.tex}

% Objetivos
\chapter{Objetivos}
\input{capitulos/objetivos/main.tex}

% Metodología
\chapter{Metodología}
\input{capitulos/metodologia/main.tex}

% Resultados y discusión (incluyendo la valoración de impactos y de aspectos de responsabilidad legal, ética y profesional relacionados con el trabajo)
\chapter{Resultados y Discusión}
\input{capitulos/resultados_discusion/main.tex}

% Conclusiones
\chapter{Conclusiones}
\input{capitulos/conclusiones/main.tex}

% Planificación temporal y presupuesto
\chapter{Planificación Temporal y Presupuesto}
\input{capitulos/planificacion_presupuesto/main.tex}

% Bibliografía
\newpage
\addcontentsline{toc}{chapter}{Bibliografía}
\printbibliography

\end{document}


% Bibliografía
\newpage
\addcontentsline{toc}{chapter}{Bibliografía}
\printbibliography

\end{document}


% Objetivos
\chapter{Objetivos}
\documentclass[a4paper,11pt,twoside]{report}
\usepackage[left=25mm,right=25mm,top=25mm,bottom=25mm,includehead,includefoot,headsep=15mm,footskip=15mm]{geometry}
\usepackage{graphicx}
\usepackage{fancyhdr}
\usepackage{titlesec}
\usepackage[spanish]{babel}
\usepackage[utf8]{inputenc}
\usepackage{amsmath}
\usepackage{setspace}
\usepackage{svg}
\usepackage{hyperref}
\usepackage[backend=biber,style=numeric]{biblatex}
\addbibresource{references.bib}
\hypersetup{
    colorlinks=true,
    linkcolor=blue,      % color of internal links (sections, etc.)
    urlcolor=blue,       % color of external links
    pdftitle={Optimización energética de sistema híbrido con bomba de calor, suelo radiante, fotovoltaica y almacenamiento para vivienda},    % title
    pdfauthor={Luis D. Aranda Sánchez},     % author
    pdfkeywords={palabra1, palabra2, código1, etc.} % list of keywords
}

% Font change to Arial
\usepackage{helvet}
\renewcommand{\familydefault}{\sfdefault}

% Chapter titles in uppercase and larger font
\titleformat{\chapter}[hang]{\large\bfseries}{\thechapter.}{1em}{\MakeUppercase}
\titleformat{\section}[hang]{\bfseries}{\thesection.}{1em}{}
\titleformat{\subsection}[hang]{\bfseries}{\thesubsection.}{1em}{}

% Fancyhdr setup
\setlength{\headheight}{14.30174pt} % Adjust to recommended value, slightly larger for safety
\fancyhf{} % Clear all headers and footers
\fancyhead[LE]{\nouppercase{\leftmark}}
\fancyhead[RO]{Optimización energética para vivienda}
\fancyfoot[LE]{\thepage}
\fancyfoot[RE]{Escuela Técnica Superior de Ingenieros Industriales (UPM)}
\fancyfoot[LO]{Luis D. Aranda Sánchez}
\fancyfoot[RO]{\thepage}
\renewcommand{\headrulewidth}{0.4pt}
\renewcommand{\footrulewidth}{0.4pt}

\fancypagestyle{myfancy}{
    \fancyhf{} % Clear all headers and footers
    \fancyhead[LE]{\nouppercase{\leftmark}}
    \fancyhead[RO]{Optimización energética para vivienda}
    \fancyfoot[LE]{\thepage}
    \fancyfoot[RE]{Escuela Técnica Superior de Ingenieros Industriales (UPM)}
    \fancyfoot[LO]{Luis D. Aranda Sánchez}
    \fancyfoot[RO]{\thepage}
    \renewcommand{\headrulewidth}{0.4pt}
    \renewcommand{\footrulewidth}{0.4pt}
}

\fancypagestyle{simple}{
    \fancyhf{} % Clear all headers and footers
    \renewcommand{\headrulewidth}{0pt}
    \renewcommand{\footrulewidth}{0pt}
}

% Line spacing
\setstretch{1.2}

% Document starts here
\begin{document}

% Portada
\begin{titlepage}
    \centering
    {\scshape\LARGE Universidad Politécnica de Madrid \par}
    \vspace{1cm}
    {\scshape\Large Escuela Técnica Superior de Ingenieros Industriales\par}
    \vspace{1.5cm}
    {\huge\bfseries Optimización energética de sistema híbrido con bomba de calor, suelo radiante, fotovoltaica y almacenamiento para vivienda \par}
    \vspace{1.5cm}
    {\Large\bfseries Trabajo de Fin de Máster\par}
    \vspace{0.5cm}
    {\large Máster Universitario en Ingeniería de la Energía \par}
    \vspace{2cm}
    {\Large Luis D. Aranda Sánchez\par}
    \vfill
    Director: Javier Rodríguez Martín
    \vfill
    {\large Septiembre 6, 2024\par}
\end{titlepage}

% Resumen (máximo de 5 páginas, incluyendo al final Palabras clave)
\clearpage
\pagestyle{simple}
% \newpage
\chapter*{Resumen}
\addcontentsline{toc}{chapter}{Resumen}
\documentclass[a4paper,11pt,twoside]{report}
\usepackage[left=25mm,right=25mm,top=25mm,bottom=25mm,includehead,includefoot,headsep=15mm,footskip=15mm]{geometry}
\usepackage{graphicx}
\usepackage{fancyhdr}
\usepackage{titlesec}
\usepackage[spanish]{babel}
\usepackage[utf8]{inputenc}
\usepackage{amsmath}
\usepackage{setspace}
\usepackage{svg}
\usepackage{hyperref}
\usepackage[backend=biber,style=numeric]{biblatex}
\addbibresource{references.bib}
\hypersetup{
    colorlinks=true,
    linkcolor=blue,      % color of internal links (sections, etc.)
    urlcolor=blue,       % color of external links
    pdftitle={Optimización energética de sistema híbrido con bomba de calor, suelo radiante, fotovoltaica y almacenamiento para vivienda},    % title
    pdfauthor={Luis D. Aranda Sánchez},     % author
    pdfkeywords={palabra1, palabra2, código1, etc.} % list of keywords
}

% Font change to Arial
\usepackage{helvet}
\renewcommand{\familydefault}{\sfdefault}

% Chapter titles in uppercase and larger font
\titleformat{\chapter}[hang]{\large\bfseries}{\thechapter.}{1em}{\MakeUppercase}
\titleformat{\section}[hang]{\bfseries}{\thesection.}{1em}{}
\titleformat{\subsection}[hang]{\bfseries}{\thesubsection.}{1em}{}

% Fancyhdr setup
\setlength{\headheight}{14.30174pt} % Adjust to recommended value, slightly larger for safety
\fancyhf{} % Clear all headers and footers
\fancyhead[LE]{\nouppercase{\leftmark}}
\fancyhead[RO]{Optimización energética para vivienda}
\fancyfoot[LE]{\thepage}
\fancyfoot[RE]{Escuela Técnica Superior de Ingenieros Industriales (UPM)}
\fancyfoot[LO]{Luis D. Aranda Sánchez}
\fancyfoot[RO]{\thepage}
\renewcommand{\headrulewidth}{0.4pt}
\renewcommand{\footrulewidth}{0.4pt}

\fancypagestyle{myfancy}{
    \fancyhf{} % Clear all headers and footers
    \fancyhead[LE]{\nouppercase{\leftmark}}
    \fancyhead[RO]{Optimización energética para vivienda}
    \fancyfoot[LE]{\thepage}
    \fancyfoot[RE]{Escuela Técnica Superior de Ingenieros Industriales (UPM)}
    \fancyfoot[LO]{Luis D. Aranda Sánchez}
    \fancyfoot[RO]{\thepage}
    \renewcommand{\headrulewidth}{0.4pt}
    \renewcommand{\footrulewidth}{0.4pt}
}

\fancypagestyle{simple}{
    \fancyhf{} % Clear all headers and footers
    \renewcommand{\headrulewidth}{0pt}
    \renewcommand{\footrulewidth}{0pt}
}

% Line spacing
\setstretch{1.2}

% Document starts here
\begin{document}

% Portada
\begin{titlepage}
    \centering
    {\scshape\LARGE Universidad Politécnica de Madrid \par}
    \vspace{1cm}
    {\scshape\Large Escuela Técnica Superior de Ingenieros Industriales\par}
    \vspace{1.5cm}
    {\huge\bfseries Optimización energética de sistema híbrido con bomba de calor, suelo radiante, fotovoltaica y almacenamiento para vivienda \par}
    \vspace{1.5cm}
    {\Large\bfseries Trabajo de Fin de Máster\par}
    \vspace{0.5cm}
    {\large Máster Universitario en Ingeniería de la Energía \par}
    \vspace{2cm}
    {\Large Luis D. Aranda Sánchez\par}
    \vfill
    Director: Javier Rodríguez Martín
    \vfill
    {\large Septiembre 6, 2024\par}
\end{titlepage}

% Resumen (máximo de 5 páginas, incluyendo al final Palabras clave)
\clearpage
\pagestyle{simple}
% \newpage
\chapter*{Resumen}
\addcontentsline{toc}{chapter}{Resumen}
\input{capitulos/resumen/main.tex}

% Índice (paginado)
\clearpage
\pagestyle{simple}
% \newpage
\tableofcontents

% Introducción (donde se incluya los antecedentes y justificación)
\clearpage
\pagestyle{myfancy}
\newpage
\chapter{Introducción}
\input{capitulos/introduccion/main.tex}

% Objetivos
\chapter{Objetivos}
\input{capitulos/objetivos/main.tex}

% Metodología
\chapter{Metodología}
\input{capitulos/metodologia/main.tex}

% Resultados y discusión (incluyendo la valoración de impactos y de aspectos de responsabilidad legal, ética y profesional relacionados con el trabajo)
\chapter{Resultados y Discusión}
\input{capitulos/resultados_discusion/main.tex}

% Conclusiones
\chapter{Conclusiones}
\input{capitulos/conclusiones/main.tex}

% Planificación temporal y presupuesto
\chapter{Planificación Temporal y Presupuesto}
\input{capitulos/planificacion_presupuesto/main.tex}

% Bibliografía
\newpage
\addcontentsline{toc}{chapter}{Bibliografía}
\printbibliography

\end{document}


% Índice (paginado)
\clearpage
\pagestyle{simple}
% \newpage
\tableofcontents

% Introducción (donde se incluya los antecedentes y justificación)
\clearpage
\pagestyle{myfancy}
\newpage
\chapter{Introducción}
\documentclass[a4paper,11pt,twoside]{report}
\usepackage[left=25mm,right=25mm,top=25mm,bottom=25mm,includehead,includefoot,headsep=15mm,footskip=15mm]{geometry}
\usepackage{graphicx}
\usepackage{fancyhdr}
\usepackage{titlesec}
\usepackage[spanish]{babel}
\usepackage[utf8]{inputenc}
\usepackage{amsmath}
\usepackage{setspace}
\usepackage{svg}
\usepackage{hyperref}
\usepackage[backend=biber,style=numeric]{biblatex}
\addbibresource{references.bib}
\hypersetup{
    colorlinks=true,
    linkcolor=blue,      % color of internal links (sections, etc.)
    urlcolor=blue,       % color of external links
    pdftitle={Optimización energética de sistema híbrido con bomba de calor, suelo radiante, fotovoltaica y almacenamiento para vivienda},    % title
    pdfauthor={Luis D. Aranda Sánchez},     % author
    pdfkeywords={palabra1, palabra2, código1, etc.} % list of keywords
}

% Font change to Arial
\usepackage{helvet}
\renewcommand{\familydefault}{\sfdefault}

% Chapter titles in uppercase and larger font
\titleformat{\chapter}[hang]{\large\bfseries}{\thechapter.}{1em}{\MakeUppercase}
\titleformat{\section}[hang]{\bfseries}{\thesection.}{1em}{}
\titleformat{\subsection}[hang]{\bfseries}{\thesubsection.}{1em}{}

% Fancyhdr setup
\setlength{\headheight}{14.30174pt} % Adjust to recommended value, slightly larger for safety
\fancyhf{} % Clear all headers and footers
\fancyhead[LE]{\nouppercase{\leftmark}}
\fancyhead[RO]{Optimización energética para vivienda}
\fancyfoot[LE]{\thepage}
\fancyfoot[RE]{Escuela Técnica Superior de Ingenieros Industriales (UPM)}
\fancyfoot[LO]{Luis D. Aranda Sánchez}
\fancyfoot[RO]{\thepage}
\renewcommand{\headrulewidth}{0.4pt}
\renewcommand{\footrulewidth}{0.4pt}

\fancypagestyle{myfancy}{
    \fancyhf{} % Clear all headers and footers
    \fancyhead[LE]{\nouppercase{\leftmark}}
    \fancyhead[RO]{Optimización energética para vivienda}
    \fancyfoot[LE]{\thepage}
    \fancyfoot[RE]{Escuela Técnica Superior de Ingenieros Industriales (UPM)}
    \fancyfoot[LO]{Luis D. Aranda Sánchez}
    \fancyfoot[RO]{\thepage}
    \renewcommand{\headrulewidth}{0.4pt}
    \renewcommand{\footrulewidth}{0.4pt}
}

\fancypagestyle{simple}{
    \fancyhf{} % Clear all headers and footers
    \renewcommand{\headrulewidth}{0pt}
    \renewcommand{\footrulewidth}{0pt}
}

% Line spacing
\setstretch{1.2}

% Document starts here
\begin{document}

% Portada
\begin{titlepage}
    \centering
    {\scshape\LARGE Universidad Politécnica de Madrid \par}
    \vspace{1cm}
    {\scshape\Large Escuela Técnica Superior de Ingenieros Industriales\par}
    \vspace{1.5cm}
    {\huge\bfseries Optimización energética de sistema híbrido con bomba de calor, suelo radiante, fotovoltaica y almacenamiento para vivienda \par}
    \vspace{1.5cm}
    {\Large\bfseries Trabajo de Fin de Máster\par}
    \vspace{0.5cm}
    {\large Máster Universitario en Ingeniería de la Energía \par}
    \vspace{2cm}
    {\Large Luis D. Aranda Sánchez\par}
    \vfill
    Director: Javier Rodríguez Martín
    \vfill
    {\large Septiembre 6, 2024\par}
\end{titlepage}

% Resumen (máximo de 5 páginas, incluyendo al final Palabras clave)
\clearpage
\pagestyle{simple}
% \newpage
\chapter*{Resumen}
\addcontentsline{toc}{chapter}{Resumen}
\input{capitulos/resumen/main.tex}

% Índice (paginado)
\clearpage
\pagestyle{simple}
% \newpage
\tableofcontents

% Introducción (donde se incluya los antecedentes y justificación)
\clearpage
\pagestyle{myfancy}
\newpage
\chapter{Introducción}
\input{capitulos/introduccion/main.tex}

% Objetivos
\chapter{Objetivos}
\input{capitulos/objetivos/main.tex}

% Metodología
\chapter{Metodología}
\input{capitulos/metodologia/main.tex}

% Resultados y discusión (incluyendo la valoración de impactos y de aspectos de responsabilidad legal, ética y profesional relacionados con el trabajo)
\chapter{Resultados y Discusión}
\input{capitulos/resultados_discusion/main.tex}

% Conclusiones
\chapter{Conclusiones}
\input{capitulos/conclusiones/main.tex}

% Planificación temporal y presupuesto
\chapter{Planificación Temporal y Presupuesto}
\input{capitulos/planificacion_presupuesto/main.tex}

% Bibliografía
\newpage
\addcontentsline{toc}{chapter}{Bibliografía}
\printbibliography

\end{document}


% Objetivos
\chapter{Objetivos}
\documentclass[a4paper,11pt,twoside]{report}
\usepackage[left=25mm,right=25mm,top=25mm,bottom=25mm,includehead,includefoot,headsep=15mm,footskip=15mm]{geometry}
\usepackage{graphicx}
\usepackage{fancyhdr}
\usepackage{titlesec}
\usepackage[spanish]{babel}
\usepackage[utf8]{inputenc}
\usepackage{amsmath}
\usepackage{setspace}
\usepackage{svg}
\usepackage{hyperref}
\usepackage[backend=biber,style=numeric]{biblatex}
\addbibresource{references.bib}
\hypersetup{
    colorlinks=true,
    linkcolor=blue,      % color of internal links (sections, etc.)
    urlcolor=blue,       % color of external links
    pdftitle={Optimización energética de sistema híbrido con bomba de calor, suelo radiante, fotovoltaica y almacenamiento para vivienda},    % title
    pdfauthor={Luis D. Aranda Sánchez},     % author
    pdfkeywords={palabra1, palabra2, código1, etc.} % list of keywords
}

% Font change to Arial
\usepackage{helvet}
\renewcommand{\familydefault}{\sfdefault}

% Chapter titles in uppercase and larger font
\titleformat{\chapter}[hang]{\large\bfseries}{\thechapter.}{1em}{\MakeUppercase}
\titleformat{\section}[hang]{\bfseries}{\thesection.}{1em}{}
\titleformat{\subsection}[hang]{\bfseries}{\thesubsection.}{1em}{}

% Fancyhdr setup
\setlength{\headheight}{14.30174pt} % Adjust to recommended value, slightly larger for safety
\fancyhf{} % Clear all headers and footers
\fancyhead[LE]{\nouppercase{\leftmark}}
\fancyhead[RO]{Optimización energética para vivienda}
\fancyfoot[LE]{\thepage}
\fancyfoot[RE]{Escuela Técnica Superior de Ingenieros Industriales (UPM)}
\fancyfoot[LO]{Luis D. Aranda Sánchez}
\fancyfoot[RO]{\thepage}
\renewcommand{\headrulewidth}{0.4pt}
\renewcommand{\footrulewidth}{0.4pt}

\fancypagestyle{myfancy}{
    \fancyhf{} % Clear all headers and footers
    \fancyhead[LE]{\nouppercase{\leftmark}}
    \fancyhead[RO]{Optimización energética para vivienda}
    \fancyfoot[LE]{\thepage}
    \fancyfoot[RE]{Escuela Técnica Superior de Ingenieros Industriales (UPM)}
    \fancyfoot[LO]{Luis D. Aranda Sánchez}
    \fancyfoot[RO]{\thepage}
    \renewcommand{\headrulewidth}{0.4pt}
    \renewcommand{\footrulewidth}{0.4pt}
}

\fancypagestyle{simple}{
    \fancyhf{} % Clear all headers and footers
    \renewcommand{\headrulewidth}{0pt}
    \renewcommand{\footrulewidth}{0pt}
}

% Line spacing
\setstretch{1.2}

% Document starts here
\begin{document}

% Portada
\begin{titlepage}
    \centering
    {\scshape\LARGE Universidad Politécnica de Madrid \par}
    \vspace{1cm}
    {\scshape\Large Escuela Técnica Superior de Ingenieros Industriales\par}
    \vspace{1.5cm}
    {\huge\bfseries Optimización energética de sistema híbrido con bomba de calor, suelo radiante, fotovoltaica y almacenamiento para vivienda \par}
    \vspace{1.5cm}
    {\Large\bfseries Trabajo de Fin de Máster\par}
    \vspace{0.5cm}
    {\large Máster Universitario en Ingeniería de la Energía \par}
    \vspace{2cm}
    {\Large Luis D. Aranda Sánchez\par}
    \vfill
    Director: Javier Rodríguez Martín
    \vfill
    {\large Septiembre 6, 2024\par}
\end{titlepage}

% Resumen (máximo de 5 páginas, incluyendo al final Palabras clave)
\clearpage
\pagestyle{simple}
% \newpage
\chapter*{Resumen}
\addcontentsline{toc}{chapter}{Resumen}
\input{capitulos/resumen/main.tex}

% Índice (paginado)
\clearpage
\pagestyle{simple}
% \newpage
\tableofcontents

% Introducción (donde se incluya los antecedentes y justificación)
\clearpage
\pagestyle{myfancy}
\newpage
\chapter{Introducción}
\input{capitulos/introduccion/main.tex}

% Objetivos
\chapter{Objetivos}
\input{capitulos/objetivos/main.tex}

% Metodología
\chapter{Metodología}
\input{capitulos/metodologia/main.tex}

% Resultados y discusión (incluyendo la valoración de impactos y de aspectos de responsabilidad legal, ética y profesional relacionados con el trabajo)
\chapter{Resultados y Discusión}
\input{capitulos/resultados_discusion/main.tex}

% Conclusiones
\chapter{Conclusiones}
\input{capitulos/conclusiones/main.tex}

% Planificación temporal y presupuesto
\chapter{Planificación Temporal y Presupuesto}
\input{capitulos/planificacion_presupuesto/main.tex}

% Bibliografía
\newpage
\addcontentsline{toc}{chapter}{Bibliografía}
\printbibliography

\end{document}


% Metodología
\chapter{Metodología}
\documentclass[a4paper,11pt,twoside]{report}
\usepackage[left=25mm,right=25mm,top=25mm,bottom=25mm,includehead,includefoot,headsep=15mm,footskip=15mm]{geometry}
\usepackage{graphicx}
\usepackage{fancyhdr}
\usepackage{titlesec}
\usepackage[spanish]{babel}
\usepackage[utf8]{inputenc}
\usepackage{amsmath}
\usepackage{setspace}
\usepackage{svg}
\usepackage{hyperref}
\usepackage[backend=biber,style=numeric]{biblatex}
\addbibresource{references.bib}
\hypersetup{
    colorlinks=true,
    linkcolor=blue,      % color of internal links (sections, etc.)
    urlcolor=blue,       % color of external links
    pdftitle={Optimización energética de sistema híbrido con bomba de calor, suelo radiante, fotovoltaica y almacenamiento para vivienda},    % title
    pdfauthor={Luis D. Aranda Sánchez},     % author
    pdfkeywords={palabra1, palabra2, código1, etc.} % list of keywords
}

% Font change to Arial
\usepackage{helvet}
\renewcommand{\familydefault}{\sfdefault}

% Chapter titles in uppercase and larger font
\titleformat{\chapter}[hang]{\large\bfseries}{\thechapter.}{1em}{\MakeUppercase}
\titleformat{\section}[hang]{\bfseries}{\thesection.}{1em}{}
\titleformat{\subsection}[hang]{\bfseries}{\thesubsection.}{1em}{}

% Fancyhdr setup
\setlength{\headheight}{14.30174pt} % Adjust to recommended value, slightly larger for safety
\fancyhf{} % Clear all headers and footers
\fancyhead[LE]{\nouppercase{\leftmark}}
\fancyhead[RO]{Optimización energética para vivienda}
\fancyfoot[LE]{\thepage}
\fancyfoot[RE]{Escuela Técnica Superior de Ingenieros Industriales (UPM)}
\fancyfoot[LO]{Luis D. Aranda Sánchez}
\fancyfoot[RO]{\thepage}
\renewcommand{\headrulewidth}{0.4pt}
\renewcommand{\footrulewidth}{0.4pt}

\fancypagestyle{myfancy}{
    \fancyhf{} % Clear all headers and footers
    \fancyhead[LE]{\nouppercase{\leftmark}}
    \fancyhead[RO]{Optimización energética para vivienda}
    \fancyfoot[LE]{\thepage}
    \fancyfoot[RE]{Escuela Técnica Superior de Ingenieros Industriales (UPM)}
    \fancyfoot[LO]{Luis D. Aranda Sánchez}
    \fancyfoot[RO]{\thepage}
    \renewcommand{\headrulewidth}{0.4pt}
    \renewcommand{\footrulewidth}{0.4pt}
}

\fancypagestyle{simple}{
    \fancyhf{} % Clear all headers and footers
    \renewcommand{\headrulewidth}{0pt}
    \renewcommand{\footrulewidth}{0pt}
}

% Line spacing
\setstretch{1.2}

% Document starts here
\begin{document}

% Portada
\begin{titlepage}
    \centering
    {\scshape\LARGE Universidad Politécnica de Madrid \par}
    \vspace{1cm}
    {\scshape\Large Escuela Técnica Superior de Ingenieros Industriales\par}
    \vspace{1.5cm}
    {\huge\bfseries Optimización energética de sistema híbrido con bomba de calor, suelo radiante, fotovoltaica y almacenamiento para vivienda \par}
    \vspace{1.5cm}
    {\Large\bfseries Trabajo de Fin de Máster\par}
    \vspace{0.5cm}
    {\large Máster Universitario en Ingeniería de la Energía \par}
    \vspace{2cm}
    {\Large Luis D. Aranda Sánchez\par}
    \vfill
    Director: Javier Rodríguez Martín
    \vfill
    {\large Septiembre 6, 2024\par}
\end{titlepage}

% Resumen (máximo de 5 páginas, incluyendo al final Palabras clave)
\clearpage
\pagestyle{simple}
% \newpage
\chapter*{Resumen}
\addcontentsline{toc}{chapter}{Resumen}
\input{capitulos/resumen/main.tex}

% Índice (paginado)
\clearpage
\pagestyle{simple}
% \newpage
\tableofcontents

% Introducción (donde se incluya los antecedentes y justificación)
\clearpage
\pagestyle{myfancy}
\newpage
\chapter{Introducción}
\input{capitulos/introduccion/main.tex}

% Objetivos
\chapter{Objetivos}
\input{capitulos/objetivos/main.tex}

% Metodología
\chapter{Metodología}
\input{capitulos/metodologia/main.tex}

% Resultados y discusión (incluyendo la valoración de impactos y de aspectos de responsabilidad legal, ética y profesional relacionados con el trabajo)
\chapter{Resultados y Discusión}
\input{capitulos/resultados_discusion/main.tex}

% Conclusiones
\chapter{Conclusiones}
\input{capitulos/conclusiones/main.tex}

% Planificación temporal y presupuesto
\chapter{Planificación Temporal y Presupuesto}
\input{capitulos/planificacion_presupuesto/main.tex}

% Bibliografía
\newpage
\addcontentsline{toc}{chapter}{Bibliografía}
\printbibliography

\end{document}


% Resultados y discusión (incluyendo la valoración de impactos y de aspectos de responsabilidad legal, ética y profesional relacionados con el trabajo)
\chapter{Resultados y Discusión}
\documentclass[a4paper,11pt,twoside]{report}
\usepackage[left=25mm,right=25mm,top=25mm,bottom=25mm,includehead,includefoot,headsep=15mm,footskip=15mm]{geometry}
\usepackage{graphicx}
\usepackage{fancyhdr}
\usepackage{titlesec}
\usepackage[spanish]{babel}
\usepackage[utf8]{inputenc}
\usepackage{amsmath}
\usepackage{setspace}
\usepackage{svg}
\usepackage{hyperref}
\usepackage[backend=biber,style=numeric]{biblatex}
\addbibresource{references.bib}
\hypersetup{
    colorlinks=true,
    linkcolor=blue,      % color of internal links (sections, etc.)
    urlcolor=blue,       % color of external links
    pdftitle={Optimización energética de sistema híbrido con bomba de calor, suelo radiante, fotovoltaica y almacenamiento para vivienda},    % title
    pdfauthor={Luis D. Aranda Sánchez},     % author
    pdfkeywords={palabra1, palabra2, código1, etc.} % list of keywords
}

% Font change to Arial
\usepackage{helvet}
\renewcommand{\familydefault}{\sfdefault}

% Chapter titles in uppercase and larger font
\titleformat{\chapter}[hang]{\large\bfseries}{\thechapter.}{1em}{\MakeUppercase}
\titleformat{\section}[hang]{\bfseries}{\thesection.}{1em}{}
\titleformat{\subsection}[hang]{\bfseries}{\thesubsection.}{1em}{}

% Fancyhdr setup
\setlength{\headheight}{14.30174pt} % Adjust to recommended value, slightly larger for safety
\fancyhf{} % Clear all headers and footers
\fancyhead[LE]{\nouppercase{\leftmark}}
\fancyhead[RO]{Optimización energética para vivienda}
\fancyfoot[LE]{\thepage}
\fancyfoot[RE]{Escuela Técnica Superior de Ingenieros Industriales (UPM)}
\fancyfoot[LO]{Luis D. Aranda Sánchez}
\fancyfoot[RO]{\thepage}
\renewcommand{\headrulewidth}{0.4pt}
\renewcommand{\footrulewidth}{0.4pt}

\fancypagestyle{myfancy}{
    \fancyhf{} % Clear all headers and footers
    \fancyhead[LE]{\nouppercase{\leftmark}}
    \fancyhead[RO]{Optimización energética para vivienda}
    \fancyfoot[LE]{\thepage}
    \fancyfoot[RE]{Escuela Técnica Superior de Ingenieros Industriales (UPM)}
    \fancyfoot[LO]{Luis D. Aranda Sánchez}
    \fancyfoot[RO]{\thepage}
    \renewcommand{\headrulewidth}{0.4pt}
    \renewcommand{\footrulewidth}{0.4pt}
}

\fancypagestyle{simple}{
    \fancyhf{} % Clear all headers and footers
    \renewcommand{\headrulewidth}{0pt}
    \renewcommand{\footrulewidth}{0pt}
}

% Line spacing
\setstretch{1.2}

% Document starts here
\begin{document}

% Portada
\begin{titlepage}
    \centering
    {\scshape\LARGE Universidad Politécnica de Madrid \par}
    \vspace{1cm}
    {\scshape\Large Escuela Técnica Superior de Ingenieros Industriales\par}
    \vspace{1.5cm}
    {\huge\bfseries Optimización energética de sistema híbrido con bomba de calor, suelo radiante, fotovoltaica y almacenamiento para vivienda \par}
    \vspace{1.5cm}
    {\Large\bfseries Trabajo de Fin de Máster\par}
    \vspace{0.5cm}
    {\large Máster Universitario en Ingeniería de la Energía \par}
    \vspace{2cm}
    {\Large Luis D. Aranda Sánchez\par}
    \vfill
    Director: Javier Rodríguez Martín
    \vfill
    {\large Septiembre 6, 2024\par}
\end{titlepage}

% Resumen (máximo de 5 páginas, incluyendo al final Palabras clave)
\clearpage
\pagestyle{simple}
% \newpage
\chapter*{Resumen}
\addcontentsline{toc}{chapter}{Resumen}
\input{capitulos/resumen/main.tex}

% Índice (paginado)
\clearpage
\pagestyle{simple}
% \newpage
\tableofcontents

% Introducción (donde se incluya los antecedentes y justificación)
\clearpage
\pagestyle{myfancy}
\newpage
\chapter{Introducción}
\input{capitulos/introduccion/main.tex}

% Objetivos
\chapter{Objetivos}
\input{capitulos/objetivos/main.tex}

% Metodología
\chapter{Metodología}
\input{capitulos/metodologia/main.tex}

% Resultados y discusión (incluyendo la valoración de impactos y de aspectos de responsabilidad legal, ética y profesional relacionados con el trabajo)
\chapter{Resultados y Discusión}
\input{capitulos/resultados_discusion/main.tex}

% Conclusiones
\chapter{Conclusiones}
\input{capitulos/conclusiones/main.tex}

% Planificación temporal y presupuesto
\chapter{Planificación Temporal y Presupuesto}
\input{capitulos/planificacion_presupuesto/main.tex}

% Bibliografía
\newpage
\addcontentsline{toc}{chapter}{Bibliografía}
\printbibliography

\end{document}


% Conclusiones
\chapter{Conclusiones}
\documentclass[a4paper,11pt,twoside]{report}
\usepackage[left=25mm,right=25mm,top=25mm,bottom=25mm,includehead,includefoot,headsep=15mm,footskip=15mm]{geometry}
\usepackage{graphicx}
\usepackage{fancyhdr}
\usepackage{titlesec}
\usepackage[spanish]{babel}
\usepackage[utf8]{inputenc}
\usepackage{amsmath}
\usepackage{setspace}
\usepackage{svg}
\usepackage{hyperref}
\usepackage[backend=biber,style=numeric]{biblatex}
\addbibresource{references.bib}
\hypersetup{
    colorlinks=true,
    linkcolor=blue,      % color of internal links (sections, etc.)
    urlcolor=blue,       % color of external links
    pdftitle={Optimización energética de sistema híbrido con bomba de calor, suelo radiante, fotovoltaica y almacenamiento para vivienda},    % title
    pdfauthor={Luis D. Aranda Sánchez},     % author
    pdfkeywords={palabra1, palabra2, código1, etc.} % list of keywords
}

% Font change to Arial
\usepackage{helvet}
\renewcommand{\familydefault}{\sfdefault}

% Chapter titles in uppercase and larger font
\titleformat{\chapter}[hang]{\large\bfseries}{\thechapter.}{1em}{\MakeUppercase}
\titleformat{\section}[hang]{\bfseries}{\thesection.}{1em}{}
\titleformat{\subsection}[hang]{\bfseries}{\thesubsection.}{1em}{}

% Fancyhdr setup
\setlength{\headheight}{14.30174pt} % Adjust to recommended value, slightly larger for safety
\fancyhf{} % Clear all headers and footers
\fancyhead[LE]{\nouppercase{\leftmark}}
\fancyhead[RO]{Optimización energética para vivienda}
\fancyfoot[LE]{\thepage}
\fancyfoot[RE]{Escuela Técnica Superior de Ingenieros Industriales (UPM)}
\fancyfoot[LO]{Luis D. Aranda Sánchez}
\fancyfoot[RO]{\thepage}
\renewcommand{\headrulewidth}{0.4pt}
\renewcommand{\footrulewidth}{0.4pt}

\fancypagestyle{myfancy}{
    \fancyhf{} % Clear all headers and footers
    \fancyhead[LE]{\nouppercase{\leftmark}}
    \fancyhead[RO]{Optimización energética para vivienda}
    \fancyfoot[LE]{\thepage}
    \fancyfoot[RE]{Escuela Técnica Superior de Ingenieros Industriales (UPM)}
    \fancyfoot[LO]{Luis D. Aranda Sánchez}
    \fancyfoot[RO]{\thepage}
    \renewcommand{\headrulewidth}{0.4pt}
    \renewcommand{\footrulewidth}{0.4pt}
}

\fancypagestyle{simple}{
    \fancyhf{} % Clear all headers and footers
    \renewcommand{\headrulewidth}{0pt}
    \renewcommand{\footrulewidth}{0pt}
}

% Line spacing
\setstretch{1.2}

% Document starts here
\begin{document}

% Portada
\begin{titlepage}
    \centering
    {\scshape\LARGE Universidad Politécnica de Madrid \par}
    \vspace{1cm}
    {\scshape\Large Escuela Técnica Superior de Ingenieros Industriales\par}
    \vspace{1.5cm}
    {\huge\bfseries Optimización energética de sistema híbrido con bomba de calor, suelo radiante, fotovoltaica y almacenamiento para vivienda \par}
    \vspace{1.5cm}
    {\Large\bfseries Trabajo de Fin de Máster\par}
    \vspace{0.5cm}
    {\large Máster Universitario en Ingeniería de la Energía \par}
    \vspace{2cm}
    {\Large Luis D. Aranda Sánchez\par}
    \vfill
    Director: Javier Rodríguez Martín
    \vfill
    {\large Septiembre 6, 2024\par}
\end{titlepage}

% Resumen (máximo de 5 páginas, incluyendo al final Palabras clave)
\clearpage
\pagestyle{simple}
% \newpage
\chapter*{Resumen}
\addcontentsline{toc}{chapter}{Resumen}
\input{capitulos/resumen/main.tex}

% Índice (paginado)
\clearpage
\pagestyle{simple}
% \newpage
\tableofcontents

% Introducción (donde se incluya los antecedentes y justificación)
\clearpage
\pagestyle{myfancy}
\newpage
\chapter{Introducción}
\input{capitulos/introduccion/main.tex}

% Objetivos
\chapter{Objetivos}
\input{capitulos/objetivos/main.tex}

% Metodología
\chapter{Metodología}
\input{capitulos/metodologia/main.tex}

% Resultados y discusión (incluyendo la valoración de impactos y de aspectos de responsabilidad legal, ética y profesional relacionados con el trabajo)
\chapter{Resultados y Discusión}
\input{capitulos/resultados_discusion/main.tex}

% Conclusiones
\chapter{Conclusiones}
\input{capitulos/conclusiones/main.tex}

% Planificación temporal y presupuesto
\chapter{Planificación Temporal y Presupuesto}
\input{capitulos/planificacion_presupuesto/main.tex}

% Bibliografía
\newpage
\addcontentsline{toc}{chapter}{Bibliografía}
\printbibliography

\end{document}


% Planificación temporal y presupuesto
\chapter{Planificación Temporal y Presupuesto}
\documentclass[a4paper,11pt,twoside]{report}
\usepackage[left=25mm,right=25mm,top=25mm,bottom=25mm,includehead,includefoot,headsep=15mm,footskip=15mm]{geometry}
\usepackage{graphicx}
\usepackage{fancyhdr}
\usepackage{titlesec}
\usepackage[spanish]{babel}
\usepackage[utf8]{inputenc}
\usepackage{amsmath}
\usepackage{setspace}
\usepackage{svg}
\usepackage{hyperref}
\usepackage[backend=biber,style=numeric]{biblatex}
\addbibresource{references.bib}
\hypersetup{
    colorlinks=true,
    linkcolor=blue,      % color of internal links (sections, etc.)
    urlcolor=blue,       % color of external links
    pdftitle={Optimización energética de sistema híbrido con bomba de calor, suelo radiante, fotovoltaica y almacenamiento para vivienda},    % title
    pdfauthor={Luis D. Aranda Sánchez},     % author
    pdfkeywords={palabra1, palabra2, código1, etc.} % list of keywords
}

% Font change to Arial
\usepackage{helvet}
\renewcommand{\familydefault}{\sfdefault}

% Chapter titles in uppercase and larger font
\titleformat{\chapter}[hang]{\large\bfseries}{\thechapter.}{1em}{\MakeUppercase}
\titleformat{\section}[hang]{\bfseries}{\thesection.}{1em}{}
\titleformat{\subsection}[hang]{\bfseries}{\thesubsection.}{1em}{}

% Fancyhdr setup
\setlength{\headheight}{14.30174pt} % Adjust to recommended value, slightly larger for safety
\fancyhf{} % Clear all headers and footers
\fancyhead[LE]{\nouppercase{\leftmark}}
\fancyhead[RO]{Optimización energética para vivienda}
\fancyfoot[LE]{\thepage}
\fancyfoot[RE]{Escuela Técnica Superior de Ingenieros Industriales (UPM)}
\fancyfoot[LO]{Luis D. Aranda Sánchez}
\fancyfoot[RO]{\thepage}
\renewcommand{\headrulewidth}{0.4pt}
\renewcommand{\footrulewidth}{0.4pt}

\fancypagestyle{myfancy}{
    \fancyhf{} % Clear all headers and footers
    \fancyhead[LE]{\nouppercase{\leftmark}}
    \fancyhead[RO]{Optimización energética para vivienda}
    \fancyfoot[LE]{\thepage}
    \fancyfoot[RE]{Escuela Técnica Superior de Ingenieros Industriales (UPM)}
    \fancyfoot[LO]{Luis D. Aranda Sánchez}
    \fancyfoot[RO]{\thepage}
    \renewcommand{\headrulewidth}{0.4pt}
    \renewcommand{\footrulewidth}{0.4pt}
}

\fancypagestyle{simple}{
    \fancyhf{} % Clear all headers and footers
    \renewcommand{\headrulewidth}{0pt}
    \renewcommand{\footrulewidth}{0pt}
}

% Line spacing
\setstretch{1.2}

% Document starts here
\begin{document}

% Portada
\begin{titlepage}
    \centering
    {\scshape\LARGE Universidad Politécnica de Madrid \par}
    \vspace{1cm}
    {\scshape\Large Escuela Técnica Superior de Ingenieros Industriales\par}
    \vspace{1.5cm}
    {\huge\bfseries Optimización energética de sistema híbrido con bomba de calor, suelo radiante, fotovoltaica y almacenamiento para vivienda \par}
    \vspace{1.5cm}
    {\Large\bfseries Trabajo de Fin de Máster\par}
    \vspace{0.5cm}
    {\large Máster Universitario en Ingeniería de la Energía \par}
    \vspace{2cm}
    {\Large Luis D. Aranda Sánchez\par}
    \vfill
    Director: Javier Rodríguez Martín
    \vfill
    {\large Septiembre 6, 2024\par}
\end{titlepage}

% Resumen (máximo de 5 páginas, incluyendo al final Palabras clave)
\clearpage
\pagestyle{simple}
% \newpage
\chapter*{Resumen}
\addcontentsline{toc}{chapter}{Resumen}
\input{capitulos/resumen/main.tex}

% Índice (paginado)
\clearpage
\pagestyle{simple}
% \newpage
\tableofcontents

% Introducción (donde se incluya los antecedentes y justificación)
\clearpage
\pagestyle{myfancy}
\newpage
\chapter{Introducción}
\input{capitulos/introduccion/main.tex}

% Objetivos
\chapter{Objetivos}
\input{capitulos/objetivos/main.tex}

% Metodología
\chapter{Metodología}
\input{capitulos/metodologia/main.tex}

% Resultados y discusión (incluyendo la valoración de impactos y de aspectos de responsabilidad legal, ética y profesional relacionados con el trabajo)
\chapter{Resultados y Discusión}
\input{capitulos/resultados_discusion/main.tex}

% Conclusiones
\chapter{Conclusiones}
\input{capitulos/conclusiones/main.tex}

% Planificación temporal y presupuesto
\chapter{Planificación Temporal y Presupuesto}
\input{capitulos/planificacion_presupuesto/main.tex}

% Bibliografía
\newpage
\addcontentsline{toc}{chapter}{Bibliografía}
\printbibliography

\end{document}


% Bibliografía
\newpage
\addcontentsline{toc}{chapter}{Bibliografía}
\printbibliography

\end{document}


% Metodología
\chapter{Metodología}
\documentclass[a4paper,11pt,twoside]{report}
\usepackage[left=25mm,right=25mm,top=25mm,bottom=25mm,includehead,includefoot,headsep=15mm,footskip=15mm]{geometry}
\usepackage{graphicx}
\usepackage{fancyhdr}
\usepackage{titlesec}
\usepackage[spanish]{babel}
\usepackage[utf8]{inputenc}
\usepackage{amsmath}
\usepackage{setspace}
\usepackage{svg}
\usepackage{hyperref}
\usepackage[backend=biber,style=numeric]{biblatex}
\addbibresource{references.bib}
\hypersetup{
    colorlinks=true,
    linkcolor=blue,      % color of internal links (sections, etc.)
    urlcolor=blue,       % color of external links
    pdftitle={Optimización energética de sistema híbrido con bomba de calor, suelo radiante, fotovoltaica y almacenamiento para vivienda},    % title
    pdfauthor={Luis D. Aranda Sánchez},     % author
    pdfkeywords={palabra1, palabra2, código1, etc.} % list of keywords
}

% Font change to Arial
\usepackage{helvet}
\renewcommand{\familydefault}{\sfdefault}

% Chapter titles in uppercase and larger font
\titleformat{\chapter}[hang]{\large\bfseries}{\thechapter.}{1em}{\MakeUppercase}
\titleformat{\section}[hang]{\bfseries}{\thesection.}{1em}{}
\titleformat{\subsection}[hang]{\bfseries}{\thesubsection.}{1em}{}

% Fancyhdr setup
\setlength{\headheight}{14.30174pt} % Adjust to recommended value, slightly larger for safety
\fancyhf{} % Clear all headers and footers
\fancyhead[LE]{\nouppercase{\leftmark}}
\fancyhead[RO]{Optimización energética para vivienda}
\fancyfoot[LE]{\thepage}
\fancyfoot[RE]{Escuela Técnica Superior de Ingenieros Industriales (UPM)}
\fancyfoot[LO]{Luis D. Aranda Sánchez}
\fancyfoot[RO]{\thepage}
\renewcommand{\headrulewidth}{0.4pt}
\renewcommand{\footrulewidth}{0.4pt}

\fancypagestyle{myfancy}{
    \fancyhf{} % Clear all headers and footers
    \fancyhead[LE]{\nouppercase{\leftmark}}
    \fancyhead[RO]{Optimización energética para vivienda}
    \fancyfoot[LE]{\thepage}
    \fancyfoot[RE]{Escuela Técnica Superior de Ingenieros Industriales (UPM)}
    \fancyfoot[LO]{Luis D. Aranda Sánchez}
    \fancyfoot[RO]{\thepage}
    \renewcommand{\headrulewidth}{0.4pt}
    \renewcommand{\footrulewidth}{0.4pt}
}

\fancypagestyle{simple}{
    \fancyhf{} % Clear all headers and footers
    \renewcommand{\headrulewidth}{0pt}
    \renewcommand{\footrulewidth}{0pt}
}

% Line spacing
\setstretch{1.2}

% Document starts here
\begin{document}

% Portada
\begin{titlepage}
    \centering
    {\scshape\LARGE Universidad Politécnica de Madrid \par}
    \vspace{1cm}
    {\scshape\Large Escuela Técnica Superior de Ingenieros Industriales\par}
    \vspace{1.5cm}
    {\huge\bfseries Optimización energética de sistema híbrido con bomba de calor, suelo radiante, fotovoltaica y almacenamiento para vivienda \par}
    \vspace{1.5cm}
    {\Large\bfseries Trabajo de Fin de Máster\par}
    \vspace{0.5cm}
    {\large Máster Universitario en Ingeniería de la Energía \par}
    \vspace{2cm}
    {\Large Luis D. Aranda Sánchez\par}
    \vfill
    Director: Javier Rodríguez Martín
    \vfill
    {\large Septiembre 6, 2024\par}
\end{titlepage}

% Resumen (máximo de 5 páginas, incluyendo al final Palabras clave)
\clearpage
\pagestyle{simple}
% \newpage
\chapter*{Resumen}
\addcontentsline{toc}{chapter}{Resumen}
\documentclass[a4paper,11pt,twoside]{report}
\usepackage[left=25mm,right=25mm,top=25mm,bottom=25mm,includehead,includefoot,headsep=15mm,footskip=15mm]{geometry}
\usepackage{graphicx}
\usepackage{fancyhdr}
\usepackage{titlesec}
\usepackage[spanish]{babel}
\usepackage[utf8]{inputenc}
\usepackage{amsmath}
\usepackage{setspace}
\usepackage{svg}
\usepackage{hyperref}
\usepackage[backend=biber,style=numeric]{biblatex}
\addbibresource{references.bib}
\hypersetup{
    colorlinks=true,
    linkcolor=blue,      % color of internal links (sections, etc.)
    urlcolor=blue,       % color of external links
    pdftitle={Optimización energética de sistema híbrido con bomba de calor, suelo radiante, fotovoltaica y almacenamiento para vivienda},    % title
    pdfauthor={Luis D. Aranda Sánchez},     % author
    pdfkeywords={palabra1, palabra2, código1, etc.} % list of keywords
}

% Font change to Arial
\usepackage{helvet}
\renewcommand{\familydefault}{\sfdefault}

% Chapter titles in uppercase and larger font
\titleformat{\chapter}[hang]{\large\bfseries}{\thechapter.}{1em}{\MakeUppercase}
\titleformat{\section}[hang]{\bfseries}{\thesection.}{1em}{}
\titleformat{\subsection}[hang]{\bfseries}{\thesubsection.}{1em}{}

% Fancyhdr setup
\setlength{\headheight}{14.30174pt} % Adjust to recommended value, slightly larger for safety
\fancyhf{} % Clear all headers and footers
\fancyhead[LE]{\nouppercase{\leftmark}}
\fancyhead[RO]{Optimización energética para vivienda}
\fancyfoot[LE]{\thepage}
\fancyfoot[RE]{Escuela Técnica Superior de Ingenieros Industriales (UPM)}
\fancyfoot[LO]{Luis D. Aranda Sánchez}
\fancyfoot[RO]{\thepage}
\renewcommand{\headrulewidth}{0.4pt}
\renewcommand{\footrulewidth}{0.4pt}

\fancypagestyle{myfancy}{
    \fancyhf{} % Clear all headers and footers
    \fancyhead[LE]{\nouppercase{\leftmark}}
    \fancyhead[RO]{Optimización energética para vivienda}
    \fancyfoot[LE]{\thepage}
    \fancyfoot[RE]{Escuela Técnica Superior de Ingenieros Industriales (UPM)}
    \fancyfoot[LO]{Luis D. Aranda Sánchez}
    \fancyfoot[RO]{\thepage}
    \renewcommand{\headrulewidth}{0.4pt}
    \renewcommand{\footrulewidth}{0.4pt}
}

\fancypagestyle{simple}{
    \fancyhf{} % Clear all headers and footers
    \renewcommand{\headrulewidth}{0pt}
    \renewcommand{\footrulewidth}{0pt}
}

% Line spacing
\setstretch{1.2}

% Document starts here
\begin{document}

% Portada
\begin{titlepage}
    \centering
    {\scshape\LARGE Universidad Politécnica de Madrid \par}
    \vspace{1cm}
    {\scshape\Large Escuela Técnica Superior de Ingenieros Industriales\par}
    \vspace{1.5cm}
    {\huge\bfseries Optimización energética de sistema híbrido con bomba de calor, suelo radiante, fotovoltaica y almacenamiento para vivienda \par}
    \vspace{1.5cm}
    {\Large\bfseries Trabajo de Fin de Máster\par}
    \vspace{0.5cm}
    {\large Máster Universitario en Ingeniería de la Energía \par}
    \vspace{2cm}
    {\Large Luis D. Aranda Sánchez\par}
    \vfill
    Director: Javier Rodríguez Martín
    \vfill
    {\large Septiembre 6, 2024\par}
\end{titlepage}

% Resumen (máximo de 5 páginas, incluyendo al final Palabras clave)
\clearpage
\pagestyle{simple}
% \newpage
\chapter*{Resumen}
\addcontentsline{toc}{chapter}{Resumen}
\input{capitulos/resumen/main.tex}

% Índice (paginado)
\clearpage
\pagestyle{simple}
% \newpage
\tableofcontents

% Introducción (donde se incluya los antecedentes y justificación)
\clearpage
\pagestyle{myfancy}
\newpage
\chapter{Introducción}
\input{capitulos/introduccion/main.tex}

% Objetivos
\chapter{Objetivos}
\input{capitulos/objetivos/main.tex}

% Metodología
\chapter{Metodología}
\input{capitulos/metodologia/main.tex}

% Resultados y discusión (incluyendo la valoración de impactos y de aspectos de responsabilidad legal, ética y profesional relacionados con el trabajo)
\chapter{Resultados y Discusión}
\input{capitulos/resultados_discusion/main.tex}

% Conclusiones
\chapter{Conclusiones}
\input{capitulos/conclusiones/main.tex}

% Planificación temporal y presupuesto
\chapter{Planificación Temporal y Presupuesto}
\input{capitulos/planificacion_presupuesto/main.tex}

% Bibliografía
\newpage
\addcontentsline{toc}{chapter}{Bibliografía}
\printbibliography

\end{document}


% Índice (paginado)
\clearpage
\pagestyle{simple}
% \newpage
\tableofcontents

% Introducción (donde se incluya los antecedentes y justificación)
\clearpage
\pagestyle{myfancy}
\newpage
\chapter{Introducción}
\documentclass[a4paper,11pt,twoside]{report}
\usepackage[left=25mm,right=25mm,top=25mm,bottom=25mm,includehead,includefoot,headsep=15mm,footskip=15mm]{geometry}
\usepackage{graphicx}
\usepackage{fancyhdr}
\usepackage{titlesec}
\usepackage[spanish]{babel}
\usepackage[utf8]{inputenc}
\usepackage{amsmath}
\usepackage{setspace}
\usepackage{svg}
\usepackage{hyperref}
\usepackage[backend=biber,style=numeric]{biblatex}
\addbibresource{references.bib}
\hypersetup{
    colorlinks=true,
    linkcolor=blue,      % color of internal links (sections, etc.)
    urlcolor=blue,       % color of external links
    pdftitle={Optimización energética de sistema híbrido con bomba de calor, suelo radiante, fotovoltaica y almacenamiento para vivienda},    % title
    pdfauthor={Luis D. Aranda Sánchez},     % author
    pdfkeywords={palabra1, palabra2, código1, etc.} % list of keywords
}

% Font change to Arial
\usepackage{helvet}
\renewcommand{\familydefault}{\sfdefault}

% Chapter titles in uppercase and larger font
\titleformat{\chapter}[hang]{\large\bfseries}{\thechapter.}{1em}{\MakeUppercase}
\titleformat{\section}[hang]{\bfseries}{\thesection.}{1em}{}
\titleformat{\subsection}[hang]{\bfseries}{\thesubsection.}{1em}{}

% Fancyhdr setup
\setlength{\headheight}{14.30174pt} % Adjust to recommended value, slightly larger for safety
\fancyhf{} % Clear all headers and footers
\fancyhead[LE]{\nouppercase{\leftmark}}
\fancyhead[RO]{Optimización energética para vivienda}
\fancyfoot[LE]{\thepage}
\fancyfoot[RE]{Escuela Técnica Superior de Ingenieros Industriales (UPM)}
\fancyfoot[LO]{Luis D. Aranda Sánchez}
\fancyfoot[RO]{\thepage}
\renewcommand{\headrulewidth}{0.4pt}
\renewcommand{\footrulewidth}{0.4pt}

\fancypagestyle{myfancy}{
    \fancyhf{} % Clear all headers and footers
    \fancyhead[LE]{\nouppercase{\leftmark}}
    \fancyhead[RO]{Optimización energética para vivienda}
    \fancyfoot[LE]{\thepage}
    \fancyfoot[RE]{Escuela Técnica Superior de Ingenieros Industriales (UPM)}
    \fancyfoot[LO]{Luis D. Aranda Sánchez}
    \fancyfoot[RO]{\thepage}
    \renewcommand{\headrulewidth}{0.4pt}
    \renewcommand{\footrulewidth}{0.4pt}
}

\fancypagestyle{simple}{
    \fancyhf{} % Clear all headers and footers
    \renewcommand{\headrulewidth}{0pt}
    \renewcommand{\footrulewidth}{0pt}
}

% Line spacing
\setstretch{1.2}

% Document starts here
\begin{document}

% Portada
\begin{titlepage}
    \centering
    {\scshape\LARGE Universidad Politécnica de Madrid \par}
    \vspace{1cm}
    {\scshape\Large Escuela Técnica Superior de Ingenieros Industriales\par}
    \vspace{1.5cm}
    {\huge\bfseries Optimización energética de sistema híbrido con bomba de calor, suelo radiante, fotovoltaica y almacenamiento para vivienda \par}
    \vspace{1.5cm}
    {\Large\bfseries Trabajo de Fin de Máster\par}
    \vspace{0.5cm}
    {\large Máster Universitario en Ingeniería de la Energía \par}
    \vspace{2cm}
    {\Large Luis D. Aranda Sánchez\par}
    \vfill
    Director: Javier Rodríguez Martín
    \vfill
    {\large Septiembre 6, 2024\par}
\end{titlepage}

% Resumen (máximo de 5 páginas, incluyendo al final Palabras clave)
\clearpage
\pagestyle{simple}
% \newpage
\chapter*{Resumen}
\addcontentsline{toc}{chapter}{Resumen}
\input{capitulos/resumen/main.tex}

% Índice (paginado)
\clearpage
\pagestyle{simple}
% \newpage
\tableofcontents

% Introducción (donde se incluya los antecedentes y justificación)
\clearpage
\pagestyle{myfancy}
\newpage
\chapter{Introducción}
\input{capitulos/introduccion/main.tex}

% Objetivos
\chapter{Objetivos}
\input{capitulos/objetivos/main.tex}

% Metodología
\chapter{Metodología}
\input{capitulos/metodologia/main.tex}

% Resultados y discusión (incluyendo la valoración de impactos y de aspectos de responsabilidad legal, ética y profesional relacionados con el trabajo)
\chapter{Resultados y Discusión}
\input{capitulos/resultados_discusion/main.tex}

% Conclusiones
\chapter{Conclusiones}
\input{capitulos/conclusiones/main.tex}

% Planificación temporal y presupuesto
\chapter{Planificación Temporal y Presupuesto}
\input{capitulos/planificacion_presupuesto/main.tex}

% Bibliografía
\newpage
\addcontentsline{toc}{chapter}{Bibliografía}
\printbibliography

\end{document}


% Objetivos
\chapter{Objetivos}
\documentclass[a4paper,11pt,twoside]{report}
\usepackage[left=25mm,right=25mm,top=25mm,bottom=25mm,includehead,includefoot,headsep=15mm,footskip=15mm]{geometry}
\usepackage{graphicx}
\usepackage{fancyhdr}
\usepackage{titlesec}
\usepackage[spanish]{babel}
\usepackage[utf8]{inputenc}
\usepackage{amsmath}
\usepackage{setspace}
\usepackage{svg}
\usepackage{hyperref}
\usepackage[backend=biber,style=numeric]{biblatex}
\addbibresource{references.bib}
\hypersetup{
    colorlinks=true,
    linkcolor=blue,      % color of internal links (sections, etc.)
    urlcolor=blue,       % color of external links
    pdftitle={Optimización energética de sistema híbrido con bomba de calor, suelo radiante, fotovoltaica y almacenamiento para vivienda},    % title
    pdfauthor={Luis D. Aranda Sánchez},     % author
    pdfkeywords={palabra1, palabra2, código1, etc.} % list of keywords
}

% Font change to Arial
\usepackage{helvet}
\renewcommand{\familydefault}{\sfdefault}

% Chapter titles in uppercase and larger font
\titleformat{\chapter}[hang]{\large\bfseries}{\thechapter.}{1em}{\MakeUppercase}
\titleformat{\section}[hang]{\bfseries}{\thesection.}{1em}{}
\titleformat{\subsection}[hang]{\bfseries}{\thesubsection.}{1em}{}

% Fancyhdr setup
\setlength{\headheight}{14.30174pt} % Adjust to recommended value, slightly larger for safety
\fancyhf{} % Clear all headers and footers
\fancyhead[LE]{\nouppercase{\leftmark}}
\fancyhead[RO]{Optimización energética para vivienda}
\fancyfoot[LE]{\thepage}
\fancyfoot[RE]{Escuela Técnica Superior de Ingenieros Industriales (UPM)}
\fancyfoot[LO]{Luis D. Aranda Sánchez}
\fancyfoot[RO]{\thepage}
\renewcommand{\headrulewidth}{0.4pt}
\renewcommand{\footrulewidth}{0.4pt}

\fancypagestyle{myfancy}{
    \fancyhf{} % Clear all headers and footers
    \fancyhead[LE]{\nouppercase{\leftmark}}
    \fancyhead[RO]{Optimización energética para vivienda}
    \fancyfoot[LE]{\thepage}
    \fancyfoot[RE]{Escuela Técnica Superior de Ingenieros Industriales (UPM)}
    \fancyfoot[LO]{Luis D. Aranda Sánchez}
    \fancyfoot[RO]{\thepage}
    \renewcommand{\headrulewidth}{0.4pt}
    \renewcommand{\footrulewidth}{0.4pt}
}

\fancypagestyle{simple}{
    \fancyhf{} % Clear all headers and footers
    \renewcommand{\headrulewidth}{0pt}
    \renewcommand{\footrulewidth}{0pt}
}

% Line spacing
\setstretch{1.2}

% Document starts here
\begin{document}

% Portada
\begin{titlepage}
    \centering
    {\scshape\LARGE Universidad Politécnica de Madrid \par}
    \vspace{1cm}
    {\scshape\Large Escuela Técnica Superior de Ingenieros Industriales\par}
    \vspace{1.5cm}
    {\huge\bfseries Optimización energética de sistema híbrido con bomba de calor, suelo radiante, fotovoltaica y almacenamiento para vivienda \par}
    \vspace{1.5cm}
    {\Large\bfseries Trabajo de Fin de Máster\par}
    \vspace{0.5cm}
    {\large Máster Universitario en Ingeniería de la Energía \par}
    \vspace{2cm}
    {\Large Luis D. Aranda Sánchez\par}
    \vfill
    Director: Javier Rodríguez Martín
    \vfill
    {\large Septiembre 6, 2024\par}
\end{titlepage}

% Resumen (máximo de 5 páginas, incluyendo al final Palabras clave)
\clearpage
\pagestyle{simple}
% \newpage
\chapter*{Resumen}
\addcontentsline{toc}{chapter}{Resumen}
\input{capitulos/resumen/main.tex}

% Índice (paginado)
\clearpage
\pagestyle{simple}
% \newpage
\tableofcontents

% Introducción (donde se incluya los antecedentes y justificación)
\clearpage
\pagestyle{myfancy}
\newpage
\chapter{Introducción}
\input{capitulos/introduccion/main.tex}

% Objetivos
\chapter{Objetivos}
\input{capitulos/objetivos/main.tex}

% Metodología
\chapter{Metodología}
\input{capitulos/metodologia/main.tex}

% Resultados y discusión (incluyendo la valoración de impactos y de aspectos de responsabilidad legal, ética y profesional relacionados con el trabajo)
\chapter{Resultados y Discusión}
\input{capitulos/resultados_discusion/main.tex}

% Conclusiones
\chapter{Conclusiones}
\input{capitulos/conclusiones/main.tex}

% Planificación temporal y presupuesto
\chapter{Planificación Temporal y Presupuesto}
\input{capitulos/planificacion_presupuesto/main.tex}

% Bibliografía
\newpage
\addcontentsline{toc}{chapter}{Bibliografía}
\printbibliography

\end{document}


% Metodología
\chapter{Metodología}
\documentclass[a4paper,11pt,twoside]{report}
\usepackage[left=25mm,right=25mm,top=25mm,bottom=25mm,includehead,includefoot,headsep=15mm,footskip=15mm]{geometry}
\usepackage{graphicx}
\usepackage{fancyhdr}
\usepackage{titlesec}
\usepackage[spanish]{babel}
\usepackage[utf8]{inputenc}
\usepackage{amsmath}
\usepackage{setspace}
\usepackage{svg}
\usepackage{hyperref}
\usepackage[backend=biber,style=numeric]{biblatex}
\addbibresource{references.bib}
\hypersetup{
    colorlinks=true,
    linkcolor=blue,      % color of internal links (sections, etc.)
    urlcolor=blue,       % color of external links
    pdftitle={Optimización energética de sistema híbrido con bomba de calor, suelo radiante, fotovoltaica y almacenamiento para vivienda},    % title
    pdfauthor={Luis D. Aranda Sánchez},     % author
    pdfkeywords={palabra1, palabra2, código1, etc.} % list of keywords
}

% Font change to Arial
\usepackage{helvet}
\renewcommand{\familydefault}{\sfdefault}

% Chapter titles in uppercase and larger font
\titleformat{\chapter}[hang]{\large\bfseries}{\thechapter.}{1em}{\MakeUppercase}
\titleformat{\section}[hang]{\bfseries}{\thesection.}{1em}{}
\titleformat{\subsection}[hang]{\bfseries}{\thesubsection.}{1em}{}

% Fancyhdr setup
\setlength{\headheight}{14.30174pt} % Adjust to recommended value, slightly larger for safety
\fancyhf{} % Clear all headers and footers
\fancyhead[LE]{\nouppercase{\leftmark}}
\fancyhead[RO]{Optimización energética para vivienda}
\fancyfoot[LE]{\thepage}
\fancyfoot[RE]{Escuela Técnica Superior de Ingenieros Industriales (UPM)}
\fancyfoot[LO]{Luis D. Aranda Sánchez}
\fancyfoot[RO]{\thepage}
\renewcommand{\headrulewidth}{0.4pt}
\renewcommand{\footrulewidth}{0.4pt}

\fancypagestyle{myfancy}{
    \fancyhf{} % Clear all headers and footers
    \fancyhead[LE]{\nouppercase{\leftmark}}
    \fancyhead[RO]{Optimización energética para vivienda}
    \fancyfoot[LE]{\thepage}
    \fancyfoot[RE]{Escuela Técnica Superior de Ingenieros Industriales (UPM)}
    \fancyfoot[LO]{Luis D. Aranda Sánchez}
    \fancyfoot[RO]{\thepage}
    \renewcommand{\headrulewidth}{0.4pt}
    \renewcommand{\footrulewidth}{0.4pt}
}

\fancypagestyle{simple}{
    \fancyhf{} % Clear all headers and footers
    \renewcommand{\headrulewidth}{0pt}
    \renewcommand{\footrulewidth}{0pt}
}

% Line spacing
\setstretch{1.2}

% Document starts here
\begin{document}

% Portada
\begin{titlepage}
    \centering
    {\scshape\LARGE Universidad Politécnica de Madrid \par}
    \vspace{1cm}
    {\scshape\Large Escuela Técnica Superior de Ingenieros Industriales\par}
    \vspace{1.5cm}
    {\huge\bfseries Optimización energética de sistema híbrido con bomba de calor, suelo radiante, fotovoltaica y almacenamiento para vivienda \par}
    \vspace{1.5cm}
    {\Large\bfseries Trabajo de Fin de Máster\par}
    \vspace{0.5cm}
    {\large Máster Universitario en Ingeniería de la Energía \par}
    \vspace{2cm}
    {\Large Luis D. Aranda Sánchez\par}
    \vfill
    Director: Javier Rodríguez Martín
    \vfill
    {\large Septiembre 6, 2024\par}
\end{titlepage}

% Resumen (máximo de 5 páginas, incluyendo al final Palabras clave)
\clearpage
\pagestyle{simple}
% \newpage
\chapter*{Resumen}
\addcontentsline{toc}{chapter}{Resumen}
\input{capitulos/resumen/main.tex}

% Índice (paginado)
\clearpage
\pagestyle{simple}
% \newpage
\tableofcontents

% Introducción (donde se incluya los antecedentes y justificación)
\clearpage
\pagestyle{myfancy}
\newpage
\chapter{Introducción}
\input{capitulos/introduccion/main.tex}

% Objetivos
\chapter{Objetivos}
\input{capitulos/objetivos/main.tex}

% Metodología
\chapter{Metodología}
\input{capitulos/metodologia/main.tex}

% Resultados y discusión (incluyendo la valoración de impactos y de aspectos de responsabilidad legal, ética y profesional relacionados con el trabajo)
\chapter{Resultados y Discusión}
\input{capitulos/resultados_discusion/main.tex}

% Conclusiones
\chapter{Conclusiones}
\input{capitulos/conclusiones/main.tex}

% Planificación temporal y presupuesto
\chapter{Planificación Temporal y Presupuesto}
\input{capitulos/planificacion_presupuesto/main.tex}

% Bibliografía
\newpage
\addcontentsline{toc}{chapter}{Bibliografía}
\printbibliography

\end{document}


% Resultados y discusión (incluyendo la valoración de impactos y de aspectos de responsabilidad legal, ética y profesional relacionados con el trabajo)
\chapter{Resultados y Discusión}
\documentclass[a4paper,11pt,twoside]{report}
\usepackage[left=25mm,right=25mm,top=25mm,bottom=25mm,includehead,includefoot,headsep=15mm,footskip=15mm]{geometry}
\usepackage{graphicx}
\usepackage{fancyhdr}
\usepackage{titlesec}
\usepackage[spanish]{babel}
\usepackage[utf8]{inputenc}
\usepackage{amsmath}
\usepackage{setspace}
\usepackage{svg}
\usepackage{hyperref}
\usepackage[backend=biber,style=numeric]{biblatex}
\addbibresource{references.bib}
\hypersetup{
    colorlinks=true,
    linkcolor=blue,      % color of internal links (sections, etc.)
    urlcolor=blue,       % color of external links
    pdftitle={Optimización energética de sistema híbrido con bomba de calor, suelo radiante, fotovoltaica y almacenamiento para vivienda},    % title
    pdfauthor={Luis D. Aranda Sánchez},     % author
    pdfkeywords={palabra1, palabra2, código1, etc.} % list of keywords
}

% Font change to Arial
\usepackage{helvet}
\renewcommand{\familydefault}{\sfdefault}

% Chapter titles in uppercase and larger font
\titleformat{\chapter}[hang]{\large\bfseries}{\thechapter.}{1em}{\MakeUppercase}
\titleformat{\section}[hang]{\bfseries}{\thesection.}{1em}{}
\titleformat{\subsection}[hang]{\bfseries}{\thesubsection.}{1em}{}

% Fancyhdr setup
\setlength{\headheight}{14.30174pt} % Adjust to recommended value, slightly larger for safety
\fancyhf{} % Clear all headers and footers
\fancyhead[LE]{\nouppercase{\leftmark}}
\fancyhead[RO]{Optimización energética para vivienda}
\fancyfoot[LE]{\thepage}
\fancyfoot[RE]{Escuela Técnica Superior de Ingenieros Industriales (UPM)}
\fancyfoot[LO]{Luis D. Aranda Sánchez}
\fancyfoot[RO]{\thepage}
\renewcommand{\headrulewidth}{0.4pt}
\renewcommand{\footrulewidth}{0.4pt}

\fancypagestyle{myfancy}{
    \fancyhf{} % Clear all headers and footers
    \fancyhead[LE]{\nouppercase{\leftmark}}
    \fancyhead[RO]{Optimización energética para vivienda}
    \fancyfoot[LE]{\thepage}
    \fancyfoot[RE]{Escuela Técnica Superior de Ingenieros Industriales (UPM)}
    \fancyfoot[LO]{Luis D. Aranda Sánchez}
    \fancyfoot[RO]{\thepage}
    \renewcommand{\headrulewidth}{0.4pt}
    \renewcommand{\footrulewidth}{0.4pt}
}

\fancypagestyle{simple}{
    \fancyhf{} % Clear all headers and footers
    \renewcommand{\headrulewidth}{0pt}
    \renewcommand{\footrulewidth}{0pt}
}

% Line spacing
\setstretch{1.2}

% Document starts here
\begin{document}

% Portada
\begin{titlepage}
    \centering
    {\scshape\LARGE Universidad Politécnica de Madrid \par}
    \vspace{1cm}
    {\scshape\Large Escuela Técnica Superior de Ingenieros Industriales\par}
    \vspace{1.5cm}
    {\huge\bfseries Optimización energética de sistema híbrido con bomba de calor, suelo radiante, fotovoltaica y almacenamiento para vivienda \par}
    \vspace{1.5cm}
    {\Large\bfseries Trabajo de Fin de Máster\par}
    \vspace{0.5cm}
    {\large Máster Universitario en Ingeniería de la Energía \par}
    \vspace{2cm}
    {\Large Luis D. Aranda Sánchez\par}
    \vfill
    Director: Javier Rodríguez Martín
    \vfill
    {\large Septiembre 6, 2024\par}
\end{titlepage}

% Resumen (máximo de 5 páginas, incluyendo al final Palabras clave)
\clearpage
\pagestyle{simple}
% \newpage
\chapter*{Resumen}
\addcontentsline{toc}{chapter}{Resumen}
\input{capitulos/resumen/main.tex}

% Índice (paginado)
\clearpage
\pagestyle{simple}
% \newpage
\tableofcontents

% Introducción (donde se incluya los antecedentes y justificación)
\clearpage
\pagestyle{myfancy}
\newpage
\chapter{Introducción}
\input{capitulos/introduccion/main.tex}

% Objetivos
\chapter{Objetivos}
\input{capitulos/objetivos/main.tex}

% Metodología
\chapter{Metodología}
\input{capitulos/metodologia/main.tex}

% Resultados y discusión (incluyendo la valoración de impactos y de aspectos de responsabilidad legal, ética y profesional relacionados con el trabajo)
\chapter{Resultados y Discusión}
\input{capitulos/resultados_discusion/main.tex}

% Conclusiones
\chapter{Conclusiones}
\input{capitulos/conclusiones/main.tex}

% Planificación temporal y presupuesto
\chapter{Planificación Temporal y Presupuesto}
\input{capitulos/planificacion_presupuesto/main.tex}

% Bibliografía
\newpage
\addcontentsline{toc}{chapter}{Bibliografía}
\printbibliography

\end{document}


% Conclusiones
\chapter{Conclusiones}
\documentclass[a4paper,11pt,twoside]{report}
\usepackage[left=25mm,right=25mm,top=25mm,bottom=25mm,includehead,includefoot,headsep=15mm,footskip=15mm]{geometry}
\usepackage{graphicx}
\usepackage{fancyhdr}
\usepackage{titlesec}
\usepackage[spanish]{babel}
\usepackage[utf8]{inputenc}
\usepackage{amsmath}
\usepackage{setspace}
\usepackage{svg}
\usepackage{hyperref}
\usepackage[backend=biber,style=numeric]{biblatex}
\addbibresource{references.bib}
\hypersetup{
    colorlinks=true,
    linkcolor=blue,      % color of internal links (sections, etc.)
    urlcolor=blue,       % color of external links
    pdftitle={Optimización energética de sistema híbrido con bomba de calor, suelo radiante, fotovoltaica y almacenamiento para vivienda},    % title
    pdfauthor={Luis D. Aranda Sánchez},     % author
    pdfkeywords={palabra1, palabra2, código1, etc.} % list of keywords
}

% Font change to Arial
\usepackage{helvet}
\renewcommand{\familydefault}{\sfdefault}

% Chapter titles in uppercase and larger font
\titleformat{\chapter}[hang]{\large\bfseries}{\thechapter.}{1em}{\MakeUppercase}
\titleformat{\section}[hang]{\bfseries}{\thesection.}{1em}{}
\titleformat{\subsection}[hang]{\bfseries}{\thesubsection.}{1em}{}

% Fancyhdr setup
\setlength{\headheight}{14.30174pt} % Adjust to recommended value, slightly larger for safety
\fancyhf{} % Clear all headers and footers
\fancyhead[LE]{\nouppercase{\leftmark}}
\fancyhead[RO]{Optimización energética para vivienda}
\fancyfoot[LE]{\thepage}
\fancyfoot[RE]{Escuela Técnica Superior de Ingenieros Industriales (UPM)}
\fancyfoot[LO]{Luis D. Aranda Sánchez}
\fancyfoot[RO]{\thepage}
\renewcommand{\headrulewidth}{0.4pt}
\renewcommand{\footrulewidth}{0.4pt}

\fancypagestyle{myfancy}{
    \fancyhf{} % Clear all headers and footers
    \fancyhead[LE]{\nouppercase{\leftmark}}
    \fancyhead[RO]{Optimización energética para vivienda}
    \fancyfoot[LE]{\thepage}
    \fancyfoot[RE]{Escuela Técnica Superior de Ingenieros Industriales (UPM)}
    \fancyfoot[LO]{Luis D. Aranda Sánchez}
    \fancyfoot[RO]{\thepage}
    \renewcommand{\headrulewidth}{0.4pt}
    \renewcommand{\footrulewidth}{0.4pt}
}

\fancypagestyle{simple}{
    \fancyhf{} % Clear all headers and footers
    \renewcommand{\headrulewidth}{0pt}
    \renewcommand{\footrulewidth}{0pt}
}

% Line spacing
\setstretch{1.2}

% Document starts here
\begin{document}

% Portada
\begin{titlepage}
    \centering
    {\scshape\LARGE Universidad Politécnica de Madrid \par}
    \vspace{1cm}
    {\scshape\Large Escuela Técnica Superior de Ingenieros Industriales\par}
    \vspace{1.5cm}
    {\huge\bfseries Optimización energética de sistema híbrido con bomba de calor, suelo radiante, fotovoltaica y almacenamiento para vivienda \par}
    \vspace{1.5cm}
    {\Large\bfseries Trabajo de Fin de Máster\par}
    \vspace{0.5cm}
    {\large Máster Universitario en Ingeniería de la Energía \par}
    \vspace{2cm}
    {\Large Luis D. Aranda Sánchez\par}
    \vfill
    Director: Javier Rodríguez Martín
    \vfill
    {\large Septiembre 6, 2024\par}
\end{titlepage}

% Resumen (máximo de 5 páginas, incluyendo al final Palabras clave)
\clearpage
\pagestyle{simple}
% \newpage
\chapter*{Resumen}
\addcontentsline{toc}{chapter}{Resumen}
\input{capitulos/resumen/main.tex}

% Índice (paginado)
\clearpage
\pagestyle{simple}
% \newpage
\tableofcontents

% Introducción (donde se incluya los antecedentes y justificación)
\clearpage
\pagestyle{myfancy}
\newpage
\chapter{Introducción}
\input{capitulos/introduccion/main.tex}

% Objetivos
\chapter{Objetivos}
\input{capitulos/objetivos/main.tex}

% Metodología
\chapter{Metodología}
\input{capitulos/metodologia/main.tex}

% Resultados y discusión (incluyendo la valoración de impactos y de aspectos de responsabilidad legal, ética y profesional relacionados con el trabajo)
\chapter{Resultados y Discusión}
\input{capitulos/resultados_discusion/main.tex}

% Conclusiones
\chapter{Conclusiones}
\input{capitulos/conclusiones/main.tex}

% Planificación temporal y presupuesto
\chapter{Planificación Temporal y Presupuesto}
\input{capitulos/planificacion_presupuesto/main.tex}

% Bibliografía
\newpage
\addcontentsline{toc}{chapter}{Bibliografía}
\printbibliography

\end{document}


% Planificación temporal y presupuesto
\chapter{Planificación Temporal y Presupuesto}
\documentclass[a4paper,11pt,twoside]{report}
\usepackage[left=25mm,right=25mm,top=25mm,bottom=25mm,includehead,includefoot,headsep=15mm,footskip=15mm]{geometry}
\usepackage{graphicx}
\usepackage{fancyhdr}
\usepackage{titlesec}
\usepackage[spanish]{babel}
\usepackage[utf8]{inputenc}
\usepackage{amsmath}
\usepackage{setspace}
\usepackage{svg}
\usepackage{hyperref}
\usepackage[backend=biber,style=numeric]{biblatex}
\addbibresource{references.bib}
\hypersetup{
    colorlinks=true,
    linkcolor=blue,      % color of internal links (sections, etc.)
    urlcolor=blue,       % color of external links
    pdftitle={Optimización energética de sistema híbrido con bomba de calor, suelo radiante, fotovoltaica y almacenamiento para vivienda},    % title
    pdfauthor={Luis D. Aranda Sánchez},     % author
    pdfkeywords={palabra1, palabra2, código1, etc.} % list of keywords
}

% Font change to Arial
\usepackage{helvet}
\renewcommand{\familydefault}{\sfdefault}

% Chapter titles in uppercase and larger font
\titleformat{\chapter}[hang]{\large\bfseries}{\thechapter.}{1em}{\MakeUppercase}
\titleformat{\section}[hang]{\bfseries}{\thesection.}{1em}{}
\titleformat{\subsection}[hang]{\bfseries}{\thesubsection.}{1em}{}

% Fancyhdr setup
\setlength{\headheight}{14.30174pt} % Adjust to recommended value, slightly larger for safety
\fancyhf{} % Clear all headers and footers
\fancyhead[LE]{\nouppercase{\leftmark}}
\fancyhead[RO]{Optimización energética para vivienda}
\fancyfoot[LE]{\thepage}
\fancyfoot[RE]{Escuela Técnica Superior de Ingenieros Industriales (UPM)}
\fancyfoot[LO]{Luis D. Aranda Sánchez}
\fancyfoot[RO]{\thepage}
\renewcommand{\headrulewidth}{0.4pt}
\renewcommand{\footrulewidth}{0.4pt}

\fancypagestyle{myfancy}{
    \fancyhf{} % Clear all headers and footers
    \fancyhead[LE]{\nouppercase{\leftmark}}
    \fancyhead[RO]{Optimización energética para vivienda}
    \fancyfoot[LE]{\thepage}
    \fancyfoot[RE]{Escuela Técnica Superior de Ingenieros Industriales (UPM)}
    \fancyfoot[LO]{Luis D. Aranda Sánchez}
    \fancyfoot[RO]{\thepage}
    \renewcommand{\headrulewidth}{0.4pt}
    \renewcommand{\footrulewidth}{0.4pt}
}

\fancypagestyle{simple}{
    \fancyhf{} % Clear all headers and footers
    \renewcommand{\headrulewidth}{0pt}
    \renewcommand{\footrulewidth}{0pt}
}

% Line spacing
\setstretch{1.2}

% Document starts here
\begin{document}

% Portada
\begin{titlepage}
    \centering
    {\scshape\LARGE Universidad Politécnica de Madrid \par}
    \vspace{1cm}
    {\scshape\Large Escuela Técnica Superior de Ingenieros Industriales\par}
    \vspace{1.5cm}
    {\huge\bfseries Optimización energética de sistema híbrido con bomba de calor, suelo radiante, fotovoltaica y almacenamiento para vivienda \par}
    \vspace{1.5cm}
    {\Large\bfseries Trabajo de Fin de Máster\par}
    \vspace{0.5cm}
    {\large Máster Universitario en Ingeniería de la Energía \par}
    \vspace{2cm}
    {\Large Luis D. Aranda Sánchez\par}
    \vfill
    Director: Javier Rodríguez Martín
    \vfill
    {\large Septiembre 6, 2024\par}
\end{titlepage}

% Resumen (máximo de 5 páginas, incluyendo al final Palabras clave)
\clearpage
\pagestyle{simple}
% \newpage
\chapter*{Resumen}
\addcontentsline{toc}{chapter}{Resumen}
\input{capitulos/resumen/main.tex}

% Índice (paginado)
\clearpage
\pagestyle{simple}
% \newpage
\tableofcontents

% Introducción (donde se incluya los antecedentes y justificación)
\clearpage
\pagestyle{myfancy}
\newpage
\chapter{Introducción}
\input{capitulos/introduccion/main.tex}

% Objetivos
\chapter{Objetivos}
\input{capitulos/objetivos/main.tex}

% Metodología
\chapter{Metodología}
\input{capitulos/metodologia/main.tex}

% Resultados y discusión (incluyendo la valoración de impactos y de aspectos de responsabilidad legal, ética y profesional relacionados con el trabajo)
\chapter{Resultados y Discusión}
\input{capitulos/resultados_discusion/main.tex}

% Conclusiones
\chapter{Conclusiones}
\input{capitulos/conclusiones/main.tex}

% Planificación temporal y presupuesto
\chapter{Planificación Temporal y Presupuesto}
\input{capitulos/planificacion_presupuesto/main.tex}

% Bibliografía
\newpage
\addcontentsline{toc}{chapter}{Bibliografía}
\printbibliography

\end{document}


% Bibliografía
\newpage
\addcontentsline{toc}{chapter}{Bibliografía}
\printbibliography

\end{document}


% Resultados y discusión (incluyendo la valoración de impactos y de aspectos de responsabilidad legal, ética y profesional relacionados con el trabajo)
\chapter{Resultados y Discusión}
\documentclass[a4paper,11pt,twoside]{report}
\usepackage[left=25mm,right=25mm,top=25mm,bottom=25mm,includehead,includefoot,headsep=15mm,footskip=15mm]{geometry}
\usepackage{graphicx}
\usepackage{fancyhdr}
\usepackage{titlesec}
\usepackage[spanish]{babel}
\usepackage[utf8]{inputenc}
\usepackage{amsmath}
\usepackage{setspace}
\usepackage{svg}
\usepackage{hyperref}
\usepackage[backend=biber,style=numeric]{biblatex}
\addbibresource{references.bib}
\hypersetup{
    colorlinks=true,
    linkcolor=blue,      % color of internal links (sections, etc.)
    urlcolor=blue,       % color of external links
    pdftitle={Optimización energética de sistema híbrido con bomba de calor, suelo radiante, fotovoltaica y almacenamiento para vivienda},    % title
    pdfauthor={Luis D. Aranda Sánchez},     % author
    pdfkeywords={palabra1, palabra2, código1, etc.} % list of keywords
}

% Font change to Arial
\usepackage{helvet}
\renewcommand{\familydefault}{\sfdefault}

% Chapter titles in uppercase and larger font
\titleformat{\chapter}[hang]{\large\bfseries}{\thechapter.}{1em}{\MakeUppercase}
\titleformat{\section}[hang]{\bfseries}{\thesection.}{1em}{}
\titleformat{\subsection}[hang]{\bfseries}{\thesubsection.}{1em}{}

% Fancyhdr setup
\setlength{\headheight}{14.30174pt} % Adjust to recommended value, slightly larger for safety
\fancyhf{} % Clear all headers and footers
\fancyhead[LE]{\nouppercase{\leftmark}}
\fancyhead[RO]{Optimización energética para vivienda}
\fancyfoot[LE]{\thepage}
\fancyfoot[RE]{Escuela Técnica Superior de Ingenieros Industriales (UPM)}
\fancyfoot[LO]{Luis D. Aranda Sánchez}
\fancyfoot[RO]{\thepage}
\renewcommand{\headrulewidth}{0.4pt}
\renewcommand{\footrulewidth}{0.4pt}

\fancypagestyle{myfancy}{
    \fancyhf{} % Clear all headers and footers
    \fancyhead[LE]{\nouppercase{\leftmark}}
    \fancyhead[RO]{Optimización energética para vivienda}
    \fancyfoot[LE]{\thepage}
    \fancyfoot[RE]{Escuela Técnica Superior de Ingenieros Industriales (UPM)}
    \fancyfoot[LO]{Luis D. Aranda Sánchez}
    \fancyfoot[RO]{\thepage}
    \renewcommand{\headrulewidth}{0.4pt}
    \renewcommand{\footrulewidth}{0.4pt}
}

\fancypagestyle{simple}{
    \fancyhf{} % Clear all headers and footers
    \renewcommand{\headrulewidth}{0pt}
    \renewcommand{\footrulewidth}{0pt}
}

% Line spacing
\setstretch{1.2}

% Document starts here
\begin{document}

% Portada
\begin{titlepage}
    \centering
    {\scshape\LARGE Universidad Politécnica de Madrid \par}
    \vspace{1cm}
    {\scshape\Large Escuela Técnica Superior de Ingenieros Industriales\par}
    \vspace{1.5cm}
    {\huge\bfseries Optimización energética de sistema híbrido con bomba de calor, suelo radiante, fotovoltaica y almacenamiento para vivienda \par}
    \vspace{1.5cm}
    {\Large\bfseries Trabajo de Fin de Máster\par}
    \vspace{0.5cm}
    {\large Máster Universitario en Ingeniería de la Energía \par}
    \vspace{2cm}
    {\Large Luis D. Aranda Sánchez\par}
    \vfill
    Director: Javier Rodríguez Martín
    \vfill
    {\large Septiembre 6, 2024\par}
\end{titlepage}

% Resumen (máximo de 5 páginas, incluyendo al final Palabras clave)
\clearpage
\pagestyle{simple}
% \newpage
\chapter*{Resumen}
\addcontentsline{toc}{chapter}{Resumen}
\documentclass[a4paper,11pt,twoside]{report}
\usepackage[left=25mm,right=25mm,top=25mm,bottom=25mm,includehead,includefoot,headsep=15mm,footskip=15mm]{geometry}
\usepackage{graphicx}
\usepackage{fancyhdr}
\usepackage{titlesec}
\usepackage[spanish]{babel}
\usepackage[utf8]{inputenc}
\usepackage{amsmath}
\usepackage{setspace}
\usepackage{svg}
\usepackage{hyperref}
\usepackage[backend=biber,style=numeric]{biblatex}
\addbibresource{references.bib}
\hypersetup{
    colorlinks=true,
    linkcolor=blue,      % color of internal links (sections, etc.)
    urlcolor=blue,       % color of external links
    pdftitle={Optimización energética de sistema híbrido con bomba de calor, suelo radiante, fotovoltaica y almacenamiento para vivienda},    % title
    pdfauthor={Luis D. Aranda Sánchez},     % author
    pdfkeywords={palabra1, palabra2, código1, etc.} % list of keywords
}

% Font change to Arial
\usepackage{helvet}
\renewcommand{\familydefault}{\sfdefault}

% Chapter titles in uppercase and larger font
\titleformat{\chapter}[hang]{\large\bfseries}{\thechapter.}{1em}{\MakeUppercase}
\titleformat{\section}[hang]{\bfseries}{\thesection.}{1em}{}
\titleformat{\subsection}[hang]{\bfseries}{\thesubsection.}{1em}{}

% Fancyhdr setup
\setlength{\headheight}{14.30174pt} % Adjust to recommended value, slightly larger for safety
\fancyhf{} % Clear all headers and footers
\fancyhead[LE]{\nouppercase{\leftmark}}
\fancyhead[RO]{Optimización energética para vivienda}
\fancyfoot[LE]{\thepage}
\fancyfoot[RE]{Escuela Técnica Superior de Ingenieros Industriales (UPM)}
\fancyfoot[LO]{Luis D. Aranda Sánchez}
\fancyfoot[RO]{\thepage}
\renewcommand{\headrulewidth}{0.4pt}
\renewcommand{\footrulewidth}{0.4pt}

\fancypagestyle{myfancy}{
    \fancyhf{} % Clear all headers and footers
    \fancyhead[LE]{\nouppercase{\leftmark}}
    \fancyhead[RO]{Optimización energética para vivienda}
    \fancyfoot[LE]{\thepage}
    \fancyfoot[RE]{Escuela Técnica Superior de Ingenieros Industriales (UPM)}
    \fancyfoot[LO]{Luis D. Aranda Sánchez}
    \fancyfoot[RO]{\thepage}
    \renewcommand{\headrulewidth}{0.4pt}
    \renewcommand{\footrulewidth}{0.4pt}
}

\fancypagestyle{simple}{
    \fancyhf{} % Clear all headers and footers
    \renewcommand{\headrulewidth}{0pt}
    \renewcommand{\footrulewidth}{0pt}
}

% Line spacing
\setstretch{1.2}

% Document starts here
\begin{document}

% Portada
\begin{titlepage}
    \centering
    {\scshape\LARGE Universidad Politécnica de Madrid \par}
    \vspace{1cm}
    {\scshape\Large Escuela Técnica Superior de Ingenieros Industriales\par}
    \vspace{1.5cm}
    {\huge\bfseries Optimización energética de sistema híbrido con bomba de calor, suelo radiante, fotovoltaica y almacenamiento para vivienda \par}
    \vspace{1.5cm}
    {\Large\bfseries Trabajo de Fin de Máster\par}
    \vspace{0.5cm}
    {\large Máster Universitario en Ingeniería de la Energía \par}
    \vspace{2cm}
    {\Large Luis D. Aranda Sánchez\par}
    \vfill
    Director: Javier Rodríguez Martín
    \vfill
    {\large Septiembre 6, 2024\par}
\end{titlepage}

% Resumen (máximo de 5 páginas, incluyendo al final Palabras clave)
\clearpage
\pagestyle{simple}
% \newpage
\chapter*{Resumen}
\addcontentsline{toc}{chapter}{Resumen}
\input{capitulos/resumen/main.tex}

% Índice (paginado)
\clearpage
\pagestyle{simple}
% \newpage
\tableofcontents

% Introducción (donde se incluya los antecedentes y justificación)
\clearpage
\pagestyle{myfancy}
\newpage
\chapter{Introducción}
\input{capitulos/introduccion/main.tex}

% Objetivos
\chapter{Objetivos}
\input{capitulos/objetivos/main.tex}

% Metodología
\chapter{Metodología}
\input{capitulos/metodologia/main.tex}

% Resultados y discusión (incluyendo la valoración de impactos y de aspectos de responsabilidad legal, ética y profesional relacionados con el trabajo)
\chapter{Resultados y Discusión}
\input{capitulos/resultados_discusion/main.tex}

% Conclusiones
\chapter{Conclusiones}
\input{capitulos/conclusiones/main.tex}

% Planificación temporal y presupuesto
\chapter{Planificación Temporal y Presupuesto}
\input{capitulos/planificacion_presupuesto/main.tex}

% Bibliografía
\newpage
\addcontentsline{toc}{chapter}{Bibliografía}
\printbibliography

\end{document}


% Índice (paginado)
\clearpage
\pagestyle{simple}
% \newpage
\tableofcontents

% Introducción (donde se incluya los antecedentes y justificación)
\clearpage
\pagestyle{myfancy}
\newpage
\chapter{Introducción}
\documentclass[a4paper,11pt,twoside]{report}
\usepackage[left=25mm,right=25mm,top=25mm,bottom=25mm,includehead,includefoot,headsep=15mm,footskip=15mm]{geometry}
\usepackage{graphicx}
\usepackage{fancyhdr}
\usepackage{titlesec}
\usepackage[spanish]{babel}
\usepackage[utf8]{inputenc}
\usepackage{amsmath}
\usepackage{setspace}
\usepackage{svg}
\usepackage{hyperref}
\usepackage[backend=biber,style=numeric]{biblatex}
\addbibresource{references.bib}
\hypersetup{
    colorlinks=true,
    linkcolor=blue,      % color of internal links (sections, etc.)
    urlcolor=blue,       % color of external links
    pdftitle={Optimización energética de sistema híbrido con bomba de calor, suelo radiante, fotovoltaica y almacenamiento para vivienda},    % title
    pdfauthor={Luis D. Aranda Sánchez},     % author
    pdfkeywords={palabra1, palabra2, código1, etc.} % list of keywords
}

% Font change to Arial
\usepackage{helvet}
\renewcommand{\familydefault}{\sfdefault}

% Chapter titles in uppercase and larger font
\titleformat{\chapter}[hang]{\large\bfseries}{\thechapter.}{1em}{\MakeUppercase}
\titleformat{\section}[hang]{\bfseries}{\thesection.}{1em}{}
\titleformat{\subsection}[hang]{\bfseries}{\thesubsection.}{1em}{}

% Fancyhdr setup
\setlength{\headheight}{14.30174pt} % Adjust to recommended value, slightly larger for safety
\fancyhf{} % Clear all headers and footers
\fancyhead[LE]{\nouppercase{\leftmark}}
\fancyhead[RO]{Optimización energética para vivienda}
\fancyfoot[LE]{\thepage}
\fancyfoot[RE]{Escuela Técnica Superior de Ingenieros Industriales (UPM)}
\fancyfoot[LO]{Luis D. Aranda Sánchez}
\fancyfoot[RO]{\thepage}
\renewcommand{\headrulewidth}{0.4pt}
\renewcommand{\footrulewidth}{0.4pt}

\fancypagestyle{myfancy}{
    \fancyhf{} % Clear all headers and footers
    \fancyhead[LE]{\nouppercase{\leftmark}}
    \fancyhead[RO]{Optimización energética para vivienda}
    \fancyfoot[LE]{\thepage}
    \fancyfoot[RE]{Escuela Técnica Superior de Ingenieros Industriales (UPM)}
    \fancyfoot[LO]{Luis D. Aranda Sánchez}
    \fancyfoot[RO]{\thepage}
    \renewcommand{\headrulewidth}{0.4pt}
    \renewcommand{\footrulewidth}{0.4pt}
}

\fancypagestyle{simple}{
    \fancyhf{} % Clear all headers and footers
    \renewcommand{\headrulewidth}{0pt}
    \renewcommand{\footrulewidth}{0pt}
}

% Line spacing
\setstretch{1.2}

% Document starts here
\begin{document}

% Portada
\begin{titlepage}
    \centering
    {\scshape\LARGE Universidad Politécnica de Madrid \par}
    \vspace{1cm}
    {\scshape\Large Escuela Técnica Superior de Ingenieros Industriales\par}
    \vspace{1.5cm}
    {\huge\bfseries Optimización energética de sistema híbrido con bomba de calor, suelo radiante, fotovoltaica y almacenamiento para vivienda \par}
    \vspace{1.5cm}
    {\Large\bfseries Trabajo de Fin de Máster\par}
    \vspace{0.5cm}
    {\large Máster Universitario en Ingeniería de la Energía \par}
    \vspace{2cm}
    {\Large Luis D. Aranda Sánchez\par}
    \vfill
    Director: Javier Rodríguez Martín
    \vfill
    {\large Septiembre 6, 2024\par}
\end{titlepage}

% Resumen (máximo de 5 páginas, incluyendo al final Palabras clave)
\clearpage
\pagestyle{simple}
% \newpage
\chapter*{Resumen}
\addcontentsline{toc}{chapter}{Resumen}
\input{capitulos/resumen/main.tex}

% Índice (paginado)
\clearpage
\pagestyle{simple}
% \newpage
\tableofcontents

% Introducción (donde se incluya los antecedentes y justificación)
\clearpage
\pagestyle{myfancy}
\newpage
\chapter{Introducción}
\input{capitulos/introduccion/main.tex}

% Objetivos
\chapter{Objetivos}
\input{capitulos/objetivos/main.tex}

% Metodología
\chapter{Metodología}
\input{capitulos/metodologia/main.tex}

% Resultados y discusión (incluyendo la valoración de impactos y de aspectos de responsabilidad legal, ética y profesional relacionados con el trabajo)
\chapter{Resultados y Discusión}
\input{capitulos/resultados_discusion/main.tex}

% Conclusiones
\chapter{Conclusiones}
\input{capitulos/conclusiones/main.tex}

% Planificación temporal y presupuesto
\chapter{Planificación Temporal y Presupuesto}
\input{capitulos/planificacion_presupuesto/main.tex}

% Bibliografía
\newpage
\addcontentsline{toc}{chapter}{Bibliografía}
\printbibliography

\end{document}


% Objetivos
\chapter{Objetivos}
\documentclass[a4paper,11pt,twoside]{report}
\usepackage[left=25mm,right=25mm,top=25mm,bottom=25mm,includehead,includefoot,headsep=15mm,footskip=15mm]{geometry}
\usepackage{graphicx}
\usepackage{fancyhdr}
\usepackage{titlesec}
\usepackage[spanish]{babel}
\usepackage[utf8]{inputenc}
\usepackage{amsmath}
\usepackage{setspace}
\usepackage{svg}
\usepackage{hyperref}
\usepackage[backend=biber,style=numeric]{biblatex}
\addbibresource{references.bib}
\hypersetup{
    colorlinks=true,
    linkcolor=blue,      % color of internal links (sections, etc.)
    urlcolor=blue,       % color of external links
    pdftitle={Optimización energética de sistema híbrido con bomba de calor, suelo radiante, fotovoltaica y almacenamiento para vivienda},    % title
    pdfauthor={Luis D. Aranda Sánchez},     % author
    pdfkeywords={palabra1, palabra2, código1, etc.} % list of keywords
}

% Font change to Arial
\usepackage{helvet}
\renewcommand{\familydefault}{\sfdefault}

% Chapter titles in uppercase and larger font
\titleformat{\chapter}[hang]{\large\bfseries}{\thechapter.}{1em}{\MakeUppercase}
\titleformat{\section}[hang]{\bfseries}{\thesection.}{1em}{}
\titleformat{\subsection}[hang]{\bfseries}{\thesubsection.}{1em}{}

% Fancyhdr setup
\setlength{\headheight}{14.30174pt} % Adjust to recommended value, slightly larger for safety
\fancyhf{} % Clear all headers and footers
\fancyhead[LE]{\nouppercase{\leftmark}}
\fancyhead[RO]{Optimización energética para vivienda}
\fancyfoot[LE]{\thepage}
\fancyfoot[RE]{Escuela Técnica Superior de Ingenieros Industriales (UPM)}
\fancyfoot[LO]{Luis D. Aranda Sánchez}
\fancyfoot[RO]{\thepage}
\renewcommand{\headrulewidth}{0.4pt}
\renewcommand{\footrulewidth}{0.4pt}

\fancypagestyle{myfancy}{
    \fancyhf{} % Clear all headers and footers
    \fancyhead[LE]{\nouppercase{\leftmark}}
    \fancyhead[RO]{Optimización energética para vivienda}
    \fancyfoot[LE]{\thepage}
    \fancyfoot[RE]{Escuela Técnica Superior de Ingenieros Industriales (UPM)}
    \fancyfoot[LO]{Luis D. Aranda Sánchez}
    \fancyfoot[RO]{\thepage}
    \renewcommand{\headrulewidth}{0.4pt}
    \renewcommand{\footrulewidth}{0.4pt}
}

\fancypagestyle{simple}{
    \fancyhf{} % Clear all headers and footers
    \renewcommand{\headrulewidth}{0pt}
    \renewcommand{\footrulewidth}{0pt}
}

% Line spacing
\setstretch{1.2}

% Document starts here
\begin{document}

% Portada
\begin{titlepage}
    \centering
    {\scshape\LARGE Universidad Politécnica de Madrid \par}
    \vspace{1cm}
    {\scshape\Large Escuela Técnica Superior de Ingenieros Industriales\par}
    \vspace{1.5cm}
    {\huge\bfseries Optimización energética de sistema híbrido con bomba de calor, suelo radiante, fotovoltaica y almacenamiento para vivienda \par}
    \vspace{1.5cm}
    {\Large\bfseries Trabajo de Fin de Máster\par}
    \vspace{0.5cm}
    {\large Máster Universitario en Ingeniería de la Energía \par}
    \vspace{2cm}
    {\Large Luis D. Aranda Sánchez\par}
    \vfill
    Director: Javier Rodríguez Martín
    \vfill
    {\large Septiembre 6, 2024\par}
\end{titlepage}

% Resumen (máximo de 5 páginas, incluyendo al final Palabras clave)
\clearpage
\pagestyle{simple}
% \newpage
\chapter*{Resumen}
\addcontentsline{toc}{chapter}{Resumen}
\input{capitulos/resumen/main.tex}

% Índice (paginado)
\clearpage
\pagestyle{simple}
% \newpage
\tableofcontents

% Introducción (donde se incluya los antecedentes y justificación)
\clearpage
\pagestyle{myfancy}
\newpage
\chapter{Introducción}
\input{capitulos/introduccion/main.tex}

% Objetivos
\chapter{Objetivos}
\input{capitulos/objetivos/main.tex}

% Metodología
\chapter{Metodología}
\input{capitulos/metodologia/main.tex}

% Resultados y discusión (incluyendo la valoración de impactos y de aspectos de responsabilidad legal, ética y profesional relacionados con el trabajo)
\chapter{Resultados y Discusión}
\input{capitulos/resultados_discusion/main.tex}

% Conclusiones
\chapter{Conclusiones}
\input{capitulos/conclusiones/main.tex}

% Planificación temporal y presupuesto
\chapter{Planificación Temporal y Presupuesto}
\input{capitulos/planificacion_presupuesto/main.tex}

% Bibliografía
\newpage
\addcontentsline{toc}{chapter}{Bibliografía}
\printbibliography

\end{document}


% Metodología
\chapter{Metodología}
\documentclass[a4paper,11pt,twoside]{report}
\usepackage[left=25mm,right=25mm,top=25mm,bottom=25mm,includehead,includefoot,headsep=15mm,footskip=15mm]{geometry}
\usepackage{graphicx}
\usepackage{fancyhdr}
\usepackage{titlesec}
\usepackage[spanish]{babel}
\usepackage[utf8]{inputenc}
\usepackage{amsmath}
\usepackage{setspace}
\usepackage{svg}
\usepackage{hyperref}
\usepackage[backend=biber,style=numeric]{biblatex}
\addbibresource{references.bib}
\hypersetup{
    colorlinks=true,
    linkcolor=blue,      % color of internal links (sections, etc.)
    urlcolor=blue,       % color of external links
    pdftitle={Optimización energética de sistema híbrido con bomba de calor, suelo radiante, fotovoltaica y almacenamiento para vivienda},    % title
    pdfauthor={Luis D. Aranda Sánchez},     % author
    pdfkeywords={palabra1, palabra2, código1, etc.} % list of keywords
}

% Font change to Arial
\usepackage{helvet}
\renewcommand{\familydefault}{\sfdefault}

% Chapter titles in uppercase and larger font
\titleformat{\chapter}[hang]{\large\bfseries}{\thechapter.}{1em}{\MakeUppercase}
\titleformat{\section}[hang]{\bfseries}{\thesection.}{1em}{}
\titleformat{\subsection}[hang]{\bfseries}{\thesubsection.}{1em}{}

% Fancyhdr setup
\setlength{\headheight}{14.30174pt} % Adjust to recommended value, slightly larger for safety
\fancyhf{} % Clear all headers and footers
\fancyhead[LE]{\nouppercase{\leftmark}}
\fancyhead[RO]{Optimización energética para vivienda}
\fancyfoot[LE]{\thepage}
\fancyfoot[RE]{Escuela Técnica Superior de Ingenieros Industriales (UPM)}
\fancyfoot[LO]{Luis D. Aranda Sánchez}
\fancyfoot[RO]{\thepage}
\renewcommand{\headrulewidth}{0.4pt}
\renewcommand{\footrulewidth}{0.4pt}

\fancypagestyle{myfancy}{
    \fancyhf{} % Clear all headers and footers
    \fancyhead[LE]{\nouppercase{\leftmark}}
    \fancyhead[RO]{Optimización energética para vivienda}
    \fancyfoot[LE]{\thepage}
    \fancyfoot[RE]{Escuela Técnica Superior de Ingenieros Industriales (UPM)}
    \fancyfoot[LO]{Luis D. Aranda Sánchez}
    \fancyfoot[RO]{\thepage}
    \renewcommand{\headrulewidth}{0.4pt}
    \renewcommand{\footrulewidth}{0.4pt}
}

\fancypagestyle{simple}{
    \fancyhf{} % Clear all headers and footers
    \renewcommand{\headrulewidth}{0pt}
    \renewcommand{\footrulewidth}{0pt}
}

% Line spacing
\setstretch{1.2}

% Document starts here
\begin{document}

% Portada
\begin{titlepage}
    \centering
    {\scshape\LARGE Universidad Politécnica de Madrid \par}
    \vspace{1cm}
    {\scshape\Large Escuela Técnica Superior de Ingenieros Industriales\par}
    \vspace{1.5cm}
    {\huge\bfseries Optimización energética de sistema híbrido con bomba de calor, suelo radiante, fotovoltaica y almacenamiento para vivienda \par}
    \vspace{1.5cm}
    {\Large\bfseries Trabajo de Fin de Máster\par}
    \vspace{0.5cm}
    {\large Máster Universitario en Ingeniería de la Energía \par}
    \vspace{2cm}
    {\Large Luis D. Aranda Sánchez\par}
    \vfill
    Director: Javier Rodríguez Martín
    \vfill
    {\large Septiembre 6, 2024\par}
\end{titlepage}

% Resumen (máximo de 5 páginas, incluyendo al final Palabras clave)
\clearpage
\pagestyle{simple}
% \newpage
\chapter*{Resumen}
\addcontentsline{toc}{chapter}{Resumen}
\input{capitulos/resumen/main.tex}

% Índice (paginado)
\clearpage
\pagestyle{simple}
% \newpage
\tableofcontents

% Introducción (donde se incluya los antecedentes y justificación)
\clearpage
\pagestyle{myfancy}
\newpage
\chapter{Introducción}
\input{capitulos/introduccion/main.tex}

% Objetivos
\chapter{Objetivos}
\input{capitulos/objetivos/main.tex}

% Metodología
\chapter{Metodología}
\input{capitulos/metodologia/main.tex}

% Resultados y discusión (incluyendo la valoración de impactos y de aspectos de responsabilidad legal, ética y profesional relacionados con el trabajo)
\chapter{Resultados y Discusión}
\input{capitulos/resultados_discusion/main.tex}

% Conclusiones
\chapter{Conclusiones}
\input{capitulos/conclusiones/main.tex}

% Planificación temporal y presupuesto
\chapter{Planificación Temporal y Presupuesto}
\input{capitulos/planificacion_presupuesto/main.tex}

% Bibliografía
\newpage
\addcontentsline{toc}{chapter}{Bibliografía}
\printbibliography

\end{document}


% Resultados y discusión (incluyendo la valoración de impactos y de aspectos de responsabilidad legal, ética y profesional relacionados con el trabajo)
\chapter{Resultados y Discusión}
\documentclass[a4paper,11pt,twoside]{report}
\usepackage[left=25mm,right=25mm,top=25mm,bottom=25mm,includehead,includefoot,headsep=15mm,footskip=15mm]{geometry}
\usepackage{graphicx}
\usepackage{fancyhdr}
\usepackage{titlesec}
\usepackage[spanish]{babel}
\usepackage[utf8]{inputenc}
\usepackage{amsmath}
\usepackage{setspace}
\usepackage{svg}
\usepackage{hyperref}
\usepackage[backend=biber,style=numeric]{biblatex}
\addbibresource{references.bib}
\hypersetup{
    colorlinks=true,
    linkcolor=blue,      % color of internal links (sections, etc.)
    urlcolor=blue,       % color of external links
    pdftitle={Optimización energética de sistema híbrido con bomba de calor, suelo radiante, fotovoltaica y almacenamiento para vivienda},    % title
    pdfauthor={Luis D. Aranda Sánchez},     % author
    pdfkeywords={palabra1, palabra2, código1, etc.} % list of keywords
}

% Font change to Arial
\usepackage{helvet}
\renewcommand{\familydefault}{\sfdefault}

% Chapter titles in uppercase and larger font
\titleformat{\chapter}[hang]{\large\bfseries}{\thechapter.}{1em}{\MakeUppercase}
\titleformat{\section}[hang]{\bfseries}{\thesection.}{1em}{}
\titleformat{\subsection}[hang]{\bfseries}{\thesubsection.}{1em}{}

% Fancyhdr setup
\setlength{\headheight}{14.30174pt} % Adjust to recommended value, slightly larger for safety
\fancyhf{} % Clear all headers and footers
\fancyhead[LE]{\nouppercase{\leftmark}}
\fancyhead[RO]{Optimización energética para vivienda}
\fancyfoot[LE]{\thepage}
\fancyfoot[RE]{Escuela Técnica Superior de Ingenieros Industriales (UPM)}
\fancyfoot[LO]{Luis D. Aranda Sánchez}
\fancyfoot[RO]{\thepage}
\renewcommand{\headrulewidth}{0.4pt}
\renewcommand{\footrulewidth}{0.4pt}

\fancypagestyle{myfancy}{
    \fancyhf{} % Clear all headers and footers
    \fancyhead[LE]{\nouppercase{\leftmark}}
    \fancyhead[RO]{Optimización energética para vivienda}
    \fancyfoot[LE]{\thepage}
    \fancyfoot[RE]{Escuela Técnica Superior de Ingenieros Industriales (UPM)}
    \fancyfoot[LO]{Luis D. Aranda Sánchez}
    \fancyfoot[RO]{\thepage}
    \renewcommand{\headrulewidth}{0.4pt}
    \renewcommand{\footrulewidth}{0.4pt}
}

\fancypagestyle{simple}{
    \fancyhf{} % Clear all headers and footers
    \renewcommand{\headrulewidth}{0pt}
    \renewcommand{\footrulewidth}{0pt}
}

% Line spacing
\setstretch{1.2}

% Document starts here
\begin{document}

% Portada
\begin{titlepage}
    \centering
    {\scshape\LARGE Universidad Politécnica de Madrid \par}
    \vspace{1cm}
    {\scshape\Large Escuela Técnica Superior de Ingenieros Industriales\par}
    \vspace{1.5cm}
    {\huge\bfseries Optimización energética de sistema híbrido con bomba de calor, suelo radiante, fotovoltaica y almacenamiento para vivienda \par}
    \vspace{1.5cm}
    {\Large\bfseries Trabajo de Fin de Máster\par}
    \vspace{0.5cm}
    {\large Máster Universitario en Ingeniería de la Energía \par}
    \vspace{2cm}
    {\Large Luis D. Aranda Sánchez\par}
    \vfill
    Director: Javier Rodríguez Martín
    \vfill
    {\large Septiembre 6, 2024\par}
\end{titlepage}

% Resumen (máximo de 5 páginas, incluyendo al final Palabras clave)
\clearpage
\pagestyle{simple}
% \newpage
\chapter*{Resumen}
\addcontentsline{toc}{chapter}{Resumen}
\input{capitulos/resumen/main.tex}

% Índice (paginado)
\clearpage
\pagestyle{simple}
% \newpage
\tableofcontents

% Introducción (donde se incluya los antecedentes y justificación)
\clearpage
\pagestyle{myfancy}
\newpage
\chapter{Introducción}
\input{capitulos/introduccion/main.tex}

% Objetivos
\chapter{Objetivos}
\input{capitulos/objetivos/main.tex}

% Metodología
\chapter{Metodología}
\input{capitulos/metodologia/main.tex}

% Resultados y discusión (incluyendo la valoración de impactos y de aspectos de responsabilidad legal, ética y profesional relacionados con el trabajo)
\chapter{Resultados y Discusión}
\input{capitulos/resultados_discusion/main.tex}

% Conclusiones
\chapter{Conclusiones}
\input{capitulos/conclusiones/main.tex}

% Planificación temporal y presupuesto
\chapter{Planificación Temporal y Presupuesto}
\input{capitulos/planificacion_presupuesto/main.tex}

% Bibliografía
\newpage
\addcontentsline{toc}{chapter}{Bibliografía}
\printbibliography

\end{document}


% Conclusiones
\chapter{Conclusiones}
\documentclass[a4paper,11pt,twoside]{report}
\usepackage[left=25mm,right=25mm,top=25mm,bottom=25mm,includehead,includefoot,headsep=15mm,footskip=15mm]{geometry}
\usepackage{graphicx}
\usepackage{fancyhdr}
\usepackage{titlesec}
\usepackage[spanish]{babel}
\usepackage[utf8]{inputenc}
\usepackage{amsmath}
\usepackage{setspace}
\usepackage{svg}
\usepackage{hyperref}
\usepackage[backend=biber,style=numeric]{biblatex}
\addbibresource{references.bib}
\hypersetup{
    colorlinks=true,
    linkcolor=blue,      % color of internal links (sections, etc.)
    urlcolor=blue,       % color of external links
    pdftitle={Optimización energética de sistema híbrido con bomba de calor, suelo radiante, fotovoltaica y almacenamiento para vivienda},    % title
    pdfauthor={Luis D. Aranda Sánchez},     % author
    pdfkeywords={palabra1, palabra2, código1, etc.} % list of keywords
}

% Font change to Arial
\usepackage{helvet}
\renewcommand{\familydefault}{\sfdefault}

% Chapter titles in uppercase and larger font
\titleformat{\chapter}[hang]{\large\bfseries}{\thechapter.}{1em}{\MakeUppercase}
\titleformat{\section}[hang]{\bfseries}{\thesection.}{1em}{}
\titleformat{\subsection}[hang]{\bfseries}{\thesubsection.}{1em}{}

% Fancyhdr setup
\setlength{\headheight}{14.30174pt} % Adjust to recommended value, slightly larger for safety
\fancyhf{} % Clear all headers and footers
\fancyhead[LE]{\nouppercase{\leftmark}}
\fancyhead[RO]{Optimización energética para vivienda}
\fancyfoot[LE]{\thepage}
\fancyfoot[RE]{Escuela Técnica Superior de Ingenieros Industriales (UPM)}
\fancyfoot[LO]{Luis D. Aranda Sánchez}
\fancyfoot[RO]{\thepage}
\renewcommand{\headrulewidth}{0.4pt}
\renewcommand{\footrulewidth}{0.4pt}

\fancypagestyle{myfancy}{
    \fancyhf{} % Clear all headers and footers
    \fancyhead[LE]{\nouppercase{\leftmark}}
    \fancyhead[RO]{Optimización energética para vivienda}
    \fancyfoot[LE]{\thepage}
    \fancyfoot[RE]{Escuela Técnica Superior de Ingenieros Industriales (UPM)}
    \fancyfoot[LO]{Luis D. Aranda Sánchez}
    \fancyfoot[RO]{\thepage}
    \renewcommand{\headrulewidth}{0.4pt}
    \renewcommand{\footrulewidth}{0.4pt}
}

\fancypagestyle{simple}{
    \fancyhf{} % Clear all headers and footers
    \renewcommand{\headrulewidth}{0pt}
    \renewcommand{\footrulewidth}{0pt}
}

% Line spacing
\setstretch{1.2}

% Document starts here
\begin{document}

% Portada
\begin{titlepage}
    \centering
    {\scshape\LARGE Universidad Politécnica de Madrid \par}
    \vspace{1cm}
    {\scshape\Large Escuela Técnica Superior de Ingenieros Industriales\par}
    \vspace{1.5cm}
    {\huge\bfseries Optimización energética de sistema híbrido con bomba de calor, suelo radiante, fotovoltaica y almacenamiento para vivienda \par}
    \vspace{1.5cm}
    {\Large\bfseries Trabajo de Fin de Máster\par}
    \vspace{0.5cm}
    {\large Máster Universitario en Ingeniería de la Energía \par}
    \vspace{2cm}
    {\Large Luis D. Aranda Sánchez\par}
    \vfill
    Director: Javier Rodríguez Martín
    \vfill
    {\large Septiembre 6, 2024\par}
\end{titlepage}

% Resumen (máximo de 5 páginas, incluyendo al final Palabras clave)
\clearpage
\pagestyle{simple}
% \newpage
\chapter*{Resumen}
\addcontentsline{toc}{chapter}{Resumen}
\input{capitulos/resumen/main.tex}

% Índice (paginado)
\clearpage
\pagestyle{simple}
% \newpage
\tableofcontents

% Introducción (donde se incluya los antecedentes y justificación)
\clearpage
\pagestyle{myfancy}
\newpage
\chapter{Introducción}
\input{capitulos/introduccion/main.tex}

% Objetivos
\chapter{Objetivos}
\input{capitulos/objetivos/main.tex}

% Metodología
\chapter{Metodología}
\input{capitulos/metodologia/main.tex}

% Resultados y discusión (incluyendo la valoración de impactos y de aspectos de responsabilidad legal, ética y profesional relacionados con el trabajo)
\chapter{Resultados y Discusión}
\input{capitulos/resultados_discusion/main.tex}

% Conclusiones
\chapter{Conclusiones}
\input{capitulos/conclusiones/main.tex}

% Planificación temporal y presupuesto
\chapter{Planificación Temporal y Presupuesto}
\input{capitulos/planificacion_presupuesto/main.tex}

% Bibliografía
\newpage
\addcontentsline{toc}{chapter}{Bibliografía}
\printbibliography

\end{document}


% Planificación temporal y presupuesto
\chapter{Planificación Temporal y Presupuesto}
\documentclass[a4paper,11pt,twoside]{report}
\usepackage[left=25mm,right=25mm,top=25mm,bottom=25mm,includehead,includefoot,headsep=15mm,footskip=15mm]{geometry}
\usepackage{graphicx}
\usepackage{fancyhdr}
\usepackage{titlesec}
\usepackage[spanish]{babel}
\usepackage[utf8]{inputenc}
\usepackage{amsmath}
\usepackage{setspace}
\usepackage{svg}
\usepackage{hyperref}
\usepackage[backend=biber,style=numeric]{biblatex}
\addbibresource{references.bib}
\hypersetup{
    colorlinks=true,
    linkcolor=blue,      % color of internal links (sections, etc.)
    urlcolor=blue,       % color of external links
    pdftitle={Optimización energética de sistema híbrido con bomba de calor, suelo radiante, fotovoltaica y almacenamiento para vivienda},    % title
    pdfauthor={Luis D. Aranda Sánchez},     % author
    pdfkeywords={palabra1, palabra2, código1, etc.} % list of keywords
}

% Font change to Arial
\usepackage{helvet}
\renewcommand{\familydefault}{\sfdefault}

% Chapter titles in uppercase and larger font
\titleformat{\chapter}[hang]{\large\bfseries}{\thechapter.}{1em}{\MakeUppercase}
\titleformat{\section}[hang]{\bfseries}{\thesection.}{1em}{}
\titleformat{\subsection}[hang]{\bfseries}{\thesubsection.}{1em}{}

% Fancyhdr setup
\setlength{\headheight}{14.30174pt} % Adjust to recommended value, slightly larger for safety
\fancyhf{} % Clear all headers and footers
\fancyhead[LE]{\nouppercase{\leftmark}}
\fancyhead[RO]{Optimización energética para vivienda}
\fancyfoot[LE]{\thepage}
\fancyfoot[RE]{Escuela Técnica Superior de Ingenieros Industriales (UPM)}
\fancyfoot[LO]{Luis D. Aranda Sánchez}
\fancyfoot[RO]{\thepage}
\renewcommand{\headrulewidth}{0.4pt}
\renewcommand{\footrulewidth}{0.4pt}

\fancypagestyle{myfancy}{
    \fancyhf{} % Clear all headers and footers
    \fancyhead[LE]{\nouppercase{\leftmark}}
    \fancyhead[RO]{Optimización energética para vivienda}
    \fancyfoot[LE]{\thepage}
    \fancyfoot[RE]{Escuela Técnica Superior de Ingenieros Industriales (UPM)}
    \fancyfoot[LO]{Luis D. Aranda Sánchez}
    \fancyfoot[RO]{\thepage}
    \renewcommand{\headrulewidth}{0.4pt}
    \renewcommand{\footrulewidth}{0.4pt}
}

\fancypagestyle{simple}{
    \fancyhf{} % Clear all headers and footers
    \renewcommand{\headrulewidth}{0pt}
    \renewcommand{\footrulewidth}{0pt}
}

% Line spacing
\setstretch{1.2}

% Document starts here
\begin{document}

% Portada
\begin{titlepage}
    \centering
    {\scshape\LARGE Universidad Politécnica de Madrid \par}
    \vspace{1cm}
    {\scshape\Large Escuela Técnica Superior de Ingenieros Industriales\par}
    \vspace{1.5cm}
    {\huge\bfseries Optimización energética de sistema híbrido con bomba de calor, suelo radiante, fotovoltaica y almacenamiento para vivienda \par}
    \vspace{1.5cm}
    {\Large\bfseries Trabajo de Fin de Máster\par}
    \vspace{0.5cm}
    {\large Máster Universitario en Ingeniería de la Energía \par}
    \vspace{2cm}
    {\Large Luis D. Aranda Sánchez\par}
    \vfill
    Director: Javier Rodríguez Martín
    \vfill
    {\large Septiembre 6, 2024\par}
\end{titlepage}

% Resumen (máximo de 5 páginas, incluyendo al final Palabras clave)
\clearpage
\pagestyle{simple}
% \newpage
\chapter*{Resumen}
\addcontentsline{toc}{chapter}{Resumen}
\input{capitulos/resumen/main.tex}

% Índice (paginado)
\clearpage
\pagestyle{simple}
% \newpage
\tableofcontents

% Introducción (donde se incluya los antecedentes y justificación)
\clearpage
\pagestyle{myfancy}
\newpage
\chapter{Introducción}
\input{capitulos/introduccion/main.tex}

% Objetivos
\chapter{Objetivos}
\input{capitulos/objetivos/main.tex}

% Metodología
\chapter{Metodología}
\input{capitulos/metodologia/main.tex}

% Resultados y discusión (incluyendo la valoración de impactos y de aspectos de responsabilidad legal, ética y profesional relacionados con el trabajo)
\chapter{Resultados y Discusión}
\input{capitulos/resultados_discusion/main.tex}

% Conclusiones
\chapter{Conclusiones}
\input{capitulos/conclusiones/main.tex}

% Planificación temporal y presupuesto
\chapter{Planificación Temporal y Presupuesto}
\input{capitulos/planificacion_presupuesto/main.tex}

% Bibliografía
\newpage
\addcontentsline{toc}{chapter}{Bibliografía}
\printbibliography

\end{document}


% Bibliografía
\newpage
\addcontentsline{toc}{chapter}{Bibliografía}
\printbibliography

\end{document}


% Conclusiones
\chapter{Conclusiones}
\documentclass[a4paper,11pt,twoside]{report}
\usepackage[left=25mm,right=25mm,top=25mm,bottom=25mm,includehead,includefoot,headsep=15mm,footskip=15mm]{geometry}
\usepackage{graphicx}
\usepackage{fancyhdr}
\usepackage{titlesec}
\usepackage[spanish]{babel}
\usepackage[utf8]{inputenc}
\usepackage{amsmath}
\usepackage{setspace}
\usepackage{svg}
\usepackage{hyperref}
\usepackage[backend=biber,style=numeric]{biblatex}
\addbibresource{references.bib}
\hypersetup{
    colorlinks=true,
    linkcolor=blue,      % color of internal links (sections, etc.)
    urlcolor=blue,       % color of external links
    pdftitle={Optimización energética de sistema híbrido con bomba de calor, suelo radiante, fotovoltaica y almacenamiento para vivienda},    % title
    pdfauthor={Luis D. Aranda Sánchez},     % author
    pdfkeywords={palabra1, palabra2, código1, etc.} % list of keywords
}

% Font change to Arial
\usepackage{helvet}
\renewcommand{\familydefault}{\sfdefault}

% Chapter titles in uppercase and larger font
\titleformat{\chapter}[hang]{\large\bfseries}{\thechapter.}{1em}{\MakeUppercase}
\titleformat{\section}[hang]{\bfseries}{\thesection.}{1em}{}
\titleformat{\subsection}[hang]{\bfseries}{\thesubsection.}{1em}{}

% Fancyhdr setup
\setlength{\headheight}{14.30174pt} % Adjust to recommended value, slightly larger for safety
\fancyhf{} % Clear all headers and footers
\fancyhead[LE]{\nouppercase{\leftmark}}
\fancyhead[RO]{Optimización energética para vivienda}
\fancyfoot[LE]{\thepage}
\fancyfoot[RE]{Escuela Técnica Superior de Ingenieros Industriales (UPM)}
\fancyfoot[LO]{Luis D. Aranda Sánchez}
\fancyfoot[RO]{\thepage}
\renewcommand{\headrulewidth}{0.4pt}
\renewcommand{\footrulewidth}{0.4pt}

\fancypagestyle{myfancy}{
    \fancyhf{} % Clear all headers and footers
    \fancyhead[LE]{\nouppercase{\leftmark}}
    \fancyhead[RO]{Optimización energética para vivienda}
    \fancyfoot[LE]{\thepage}
    \fancyfoot[RE]{Escuela Técnica Superior de Ingenieros Industriales (UPM)}
    \fancyfoot[LO]{Luis D. Aranda Sánchez}
    \fancyfoot[RO]{\thepage}
    \renewcommand{\headrulewidth}{0.4pt}
    \renewcommand{\footrulewidth}{0.4pt}
}

\fancypagestyle{simple}{
    \fancyhf{} % Clear all headers and footers
    \renewcommand{\headrulewidth}{0pt}
    \renewcommand{\footrulewidth}{0pt}
}

% Line spacing
\setstretch{1.2}

% Document starts here
\begin{document}

% Portada
\begin{titlepage}
    \centering
    {\scshape\LARGE Universidad Politécnica de Madrid \par}
    \vspace{1cm}
    {\scshape\Large Escuela Técnica Superior de Ingenieros Industriales\par}
    \vspace{1.5cm}
    {\huge\bfseries Optimización energética de sistema híbrido con bomba de calor, suelo radiante, fotovoltaica y almacenamiento para vivienda \par}
    \vspace{1.5cm}
    {\Large\bfseries Trabajo de Fin de Máster\par}
    \vspace{0.5cm}
    {\large Máster Universitario en Ingeniería de la Energía \par}
    \vspace{2cm}
    {\Large Luis D. Aranda Sánchez\par}
    \vfill
    Director: Javier Rodríguez Martín
    \vfill
    {\large Septiembre 6, 2024\par}
\end{titlepage}

% Resumen (máximo de 5 páginas, incluyendo al final Palabras clave)
\clearpage
\pagestyle{simple}
% \newpage
\chapter*{Resumen}
\addcontentsline{toc}{chapter}{Resumen}
\documentclass[a4paper,11pt,twoside]{report}
\usepackage[left=25mm,right=25mm,top=25mm,bottom=25mm,includehead,includefoot,headsep=15mm,footskip=15mm]{geometry}
\usepackage{graphicx}
\usepackage{fancyhdr}
\usepackage{titlesec}
\usepackage[spanish]{babel}
\usepackage[utf8]{inputenc}
\usepackage{amsmath}
\usepackage{setspace}
\usepackage{svg}
\usepackage{hyperref}
\usepackage[backend=biber,style=numeric]{biblatex}
\addbibresource{references.bib}
\hypersetup{
    colorlinks=true,
    linkcolor=blue,      % color of internal links (sections, etc.)
    urlcolor=blue,       % color of external links
    pdftitle={Optimización energética de sistema híbrido con bomba de calor, suelo radiante, fotovoltaica y almacenamiento para vivienda},    % title
    pdfauthor={Luis D. Aranda Sánchez},     % author
    pdfkeywords={palabra1, palabra2, código1, etc.} % list of keywords
}

% Font change to Arial
\usepackage{helvet}
\renewcommand{\familydefault}{\sfdefault}

% Chapter titles in uppercase and larger font
\titleformat{\chapter}[hang]{\large\bfseries}{\thechapter.}{1em}{\MakeUppercase}
\titleformat{\section}[hang]{\bfseries}{\thesection.}{1em}{}
\titleformat{\subsection}[hang]{\bfseries}{\thesubsection.}{1em}{}

% Fancyhdr setup
\setlength{\headheight}{14.30174pt} % Adjust to recommended value, slightly larger for safety
\fancyhf{} % Clear all headers and footers
\fancyhead[LE]{\nouppercase{\leftmark}}
\fancyhead[RO]{Optimización energética para vivienda}
\fancyfoot[LE]{\thepage}
\fancyfoot[RE]{Escuela Técnica Superior de Ingenieros Industriales (UPM)}
\fancyfoot[LO]{Luis D. Aranda Sánchez}
\fancyfoot[RO]{\thepage}
\renewcommand{\headrulewidth}{0.4pt}
\renewcommand{\footrulewidth}{0.4pt}

\fancypagestyle{myfancy}{
    \fancyhf{} % Clear all headers and footers
    \fancyhead[LE]{\nouppercase{\leftmark}}
    \fancyhead[RO]{Optimización energética para vivienda}
    \fancyfoot[LE]{\thepage}
    \fancyfoot[RE]{Escuela Técnica Superior de Ingenieros Industriales (UPM)}
    \fancyfoot[LO]{Luis D. Aranda Sánchez}
    \fancyfoot[RO]{\thepage}
    \renewcommand{\headrulewidth}{0.4pt}
    \renewcommand{\footrulewidth}{0.4pt}
}

\fancypagestyle{simple}{
    \fancyhf{} % Clear all headers and footers
    \renewcommand{\headrulewidth}{0pt}
    \renewcommand{\footrulewidth}{0pt}
}

% Line spacing
\setstretch{1.2}

% Document starts here
\begin{document}

% Portada
\begin{titlepage}
    \centering
    {\scshape\LARGE Universidad Politécnica de Madrid \par}
    \vspace{1cm}
    {\scshape\Large Escuela Técnica Superior de Ingenieros Industriales\par}
    \vspace{1.5cm}
    {\huge\bfseries Optimización energética de sistema híbrido con bomba de calor, suelo radiante, fotovoltaica y almacenamiento para vivienda \par}
    \vspace{1.5cm}
    {\Large\bfseries Trabajo de Fin de Máster\par}
    \vspace{0.5cm}
    {\large Máster Universitario en Ingeniería de la Energía \par}
    \vspace{2cm}
    {\Large Luis D. Aranda Sánchez\par}
    \vfill
    Director: Javier Rodríguez Martín
    \vfill
    {\large Septiembre 6, 2024\par}
\end{titlepage}

% Resumen (máximo de 5 páginas, incluyendo al final Palabras clave)
\clearpage
\pagestyle{simple}
% \newpage
\chapter*{Resumen}
\addcontentsline{toc}{chapter}{Resumen}
\input{capitulos/resumen/main.tex}

% Índice (paginado)
\clearpage
\pagestyle{simple}
% \newpage
\tableofcontents

% Introducción (donde se incluya los antecedentes y justificación)
\clearpage
\pagestyle{myfancy}
\newpage
\chapter{Introducción}
\input{capitulos/introduccion/main.tex}

% Objetivos
\chapter{Objetivos}
\input{capitulos/objetivos/main.tex}

% Metodología
\chapter{Metodología}
\input{capitulos/metodologia/main.tex}

% Resultados y discusión (incluyendo la valoración de impactos y de aspectos de responsabilidad legal, ética y profesional relacionados con el trabajo)
\chapter{Resultados y Discusión}
\input{capitulos/resultados_discusion/main.tex}

% Conclusiones
\chapter{Conclusiones}
\input{capitulos/conclusiones/main.tex}

% Planificación temporal y presupuesto
\chapter{Planificación Temporal y Presupuesto}
\input{capitulos/planificacion_presupuesto/main.tex}

% Bibliografía
\newpage
\addcontentsline{toc}{chapter}{Bibliografía}
\printbibliography

\end{document}


% Índice (paginado)
\clearpage
\pagestyle{simple}
% \newpage
\tableofcontents

% Introducción (donde se incluya los antecedentes y justificación)
\clearpage
\pagestyle{myfancy}
\newpage
\chapter{Introducción}
\documentclass[a4paper,11pt,twoside]{report}
\usepackage[left=25mm,right=25mm,top=25mm,bottom=25mm,includehead,includefoot,headsep=15mm,footskip=15mm]{geometry}
\usepackage{graphicx}
\usepackage{fancyhdr}
\usepackage{titlesec}
\usepackage[spanish]{babel}
\usepackage[utf8]{inputenc}
\usepackage{amsmath}
\usepackage{setspace}
\usepackage{svg}
\usepackage{hyperref}
\usepackage[backend=biber,style=numeric]{biblatex}
\addbibresource{references.bib}
\hypersetup{
    colorlinks=true,
    linkcolor=blue,      % color of internal links (sections, etc.)
    urlcolor=blue,       % color of external links
    pdftitle={Optimización energética de sistema híbrido con bomba de calor, suelo radiante, fotovoltaica y almacenamiento para vivienda},    % title
    pdfauthor={Luis D. Aranda Sánchez},     % author
    pdfkeywords={palabra1, palabra2, código1, etc.} % list of keywords
}

% Font change to Arial
\usepackage{helvet}
\renewcommand{\familydefault}{\sfdefault}

% Chapter titles in uppercase and larger font
\titleformat{\chapter}[hang]{\large\bfseries}{\thechapter.}{1em}{\MakeUppercase}
\titleformat{\section}[hang]{\bfseries}{\thesection.}{1em}{}
\titleformat{\subsection}[hang]{\bfseries}{\thesubsection.}{1em}{}

% Fancyhdr setup
\setlength{\headheight}{14.30174pt} % Adjust to recommended value, slightly larger for safety
\fancyhf{} % Clear all headers and footers
\fancyhead[LE]{\nouppercase{\leftmark}}
\fancyhead[RO]{Optimización energética para vivienda}
\fancyfoot[LE]{\thepage}
\fancyfoot[RE]{Escuela Técnica Superior de Ingenieros Industriales (UPM)}
\fancyfoot[LO]{Luis D. Aranda Sánchez}
\fancyfoot[RO]{\thepage}
\renewcommand{\headrulewidth}{0.4pt}
\renewcommand{\footrulewidth}{0.4pt}

\fancypagestyle{myfancy}{
    \fancyhf{} % Clear all headers and footers
    \fancyhead[LE]{\nouppercase{\leftmark}}
    \fancyhead[RO]{Optimización energética para vivienda}
    \fancyfoot[LE]{\thepage}
    \fancyfoot[RE]{Escuela Técnica Superior de Ingenieros Industriales (UPM)}
    \fancyfoot[LO]{Luis D. Aranda Sánchez}
    \fancyfoot[RO]{\thepage}
    \renewcommand{\headrulewidth}{0.4pt}
    \renewcommand{\footrulewidth}{0.4pt}
}

\fancypagestyle{simple}{
    \fancyhf{} % Clear all headers and footers
    \renewcommand{\headrulewidth}{0pt}
    \renewcommand{\footrulewidth}{0pt}
}

% Line spacing
\setstretch{1.2}

% Document starts here
\begin{document}

% Portada
\begin{titlepage}
    \centering
    {\scshape\LARGE Universidad Politécnica de Madrid \par}
    \vspace{1cm}
    {\scshape\Large Escuela Técnica Superior de Ingenieros Industriales\par}
    \vspace{1.5cm}
    {\huge\bfseries Optimización energética de sistema híbrido con bomba de calor, suelo radiante, fotovoltaica y almacenamiento para vivienda \par}
    \vspace{1.5cm}
    {\Large\bfseries Trabajo de Fin de Máster\par}
    \vspace{0.5cm}
    {\large Máster Universitario en Ingeniería de la Energía \par}
    \vspace{2cm}
    {\Large Luis D. Aranda Sánchez\par}
    \vfill
    Director: Javier Rodríguez Martín
    \vfill
    {\large Septiembre 6, 2024\par}
\end{titlepage}

% Resumen (máximo de 5 páginas, incluyendo al final Palabras clave)
\clearpage
\pagestyle{simple}
% \newpage
\chapter*{Resumen}
\addcontentsline{toc}{chapter}{Resumen}
\input{capitulos/resumen/main.tex}

% Índice (paginado)
\clearpage
\pagestyle{simple}
% \newpage
\tableofcontents

% Introducción (donde se incluya los antecedentes y justificación)
\clearpage
\pagestyle{myfancy}
\newpage
\chapter{Introducción}
\input{capitulos/introduccion/main.tex}

% Objetivos
\chapter{Objetivos}
\input{capitulos/objetivos/main.tex}

% Metodología
\chapter{Metodología}
\input{capitulos/metodologia/main.tex}

% Resultados y discusión (incluyendo la valoración de impactos y de aspectos de responsabilidad legal, ética y profesional relacionados con el trabajo)
\chapter{Resultados y Discusión}
\input{capitulos/resultados_discusion/main.tex}

% Conclusiones
\chapter{Conclusiones}
\input{capitulos/conclusiones/main.tex}

% Planificación temporal y presupuesto
\chapter{Planificación Temporal y Presupuesto}
\input{capitulos/planificacion_presupuesto/main.tex}

% Bibliografía
\newpage
\addcontentsline{toc}{chapter}{Bibliografía}
\printbibliography

\end{document}


% Objetivos
\chapter{Objetivos}
\documentclass[a4paper,11pt,twoside]{report}
\usepackage[left=25mm,right=25mm,top=25mm,bottom=25mm,includehead,includefoot,headsep=15mm,footskip=15mm]{geometry}
\usepackage{graphicx}
\usepackage{fancyhdr}
\usepackage{titlesec}
\usepackage[spanish]{babel}
\usepackage[utf8]{inputenc}
\usepackage{amsmath}
\usepackage{setspace}
\usepackage{svg}
\usepackage{hyperref}
\usepackage[backend=biber,style=numeric]{biblatex}
\addbibresource{references.bib}
\hypersetup{
    colorlinks=true,
    linkcolor=blue,      % color of internal links (sections, etc.)
    urlcolor=blue,       % color of external links
    pdftitle={Optimización energética de sistema híbrido con bomba de calor, suelo radiante, fotovoltaica y almacenamiento para vivienda},    % title
    pdfauthor={Luis D. Aranda Sánchez},     % author
    pdfkeywords={palabra1, palabra2, código1, etc.} % list of keywords
}

% Font change to Arial
\usepackage{helvet}
\renewcommand{\familydefault}{\sfdefault}

% Chapter titles in uppercase and larger font
\titleformat{\chapter}[hang]{\large\bfseries}{\thechapter.}{1em}{\MakeUppercase}
\titleformat{\section}[hang]{\bfseries}{\thesection.}{1em}{}
\titleformat{\subsection}[hang]{\bfseries}{\thesubsection.}{1em}{}

% Fancyhdr setup
\setlength{\headheight}{14.30174pt} % Adjust to recommended value, slightly larger for safety
\fancyhf{} % Clear all headers and footers
\fancyhead[LE]{\nouppercase{\leftmark}}
\fancyhead[RO]{Optimización energética para vivienda}
\fancyfoot[LE]{\thepage}
\fancyfoot[RE]{Escuela Técnica Superior de Ingenieros Industriales (UPM)}
\fancyfoot[LO]{Luis D. Aranda Sánchez}
\fancyfoot[RO]{\thepage}
\renewcommand{\headrulewidth}{0.4pt}
\renewcommand{\footrulewidth}{0.4pt}

\fancypagestyle{myfancy}{
    \fancyhf{} % Clear all headers and footers
    \fancyhead[LE]{\nouppercase{\leftmark}}
    \fancyhead[RO]{Optimización energética para vivienda}
    \fancyfoot[LE]{\thepage}
    \fancyfoot[RE]{Escuela Técnica Superior de Ingenieros Industriales (UPM)}
    \fancyfoot[LO]{Luis D. Aranda Sánchez}
    \fancyfoot[RO]{\thepage}
    \renewcommand{\headrulewidth}{0.4pt}
    \renewcommand{\footrulewidth}{0.4pt}
}

\fancypagestyle{simple}{
    \fancyhf{} % Clear all headers and footers
    \renewcommand{\headrulewidth}{0pt}
    \renewcommand{\footrulewidth}{0pt}
}

% Line spacing
\setstretch{1.2}

% Document starts here
\begin{document}

% Portada
\begin{titlepage}
    \centering
    {\scshape\LARGE Universidad Politécnica de Madrid \par}
    \vspace{1cm}
    {\scshape\Large Escuela Técnica Superior de Ingenieros Industriales\par}
    \vspace{1.5cm}
    {\huge\bfseries Optimización energética de sistema híbrido con bomba de calor, suelo radiante, fotovoltaica y almacenamiento para vivienda \par}
    \vspace{1.5cm}
    {\Large\bfseries Trabajo de Fin de Máster\par}
    \vspace{0.5cm}
    {\large Máster Universitario en Ingeniería de la Energía \par}
    \vspace{2cm}
    {\Large Luis D. Aranda Sánchez\par}
    \vfill
    Director: Javier Rodríguez Martín
    \vfill
    {\large Septiembre 6, 2024\par}
\end{titlepage}

% Resumen (máximo de 5 páginas, incluyendo al final Palabras clave)
\clearpage
\pagestyle{simple}
% \newpage
\chapter*{Resumen}
\addcontentsline{toc}{chapter}{Resumen}
\input{capitulos/resumen/main.tex}

% Índice (paginado)
\clearpage
\pagestyle{simple}
% \newpage
\tableofcontents

% Introducción (donde se incluya los antecedentes y justificación)
\clearpage
\pagestyle{myfancy}
\newpage
\chapter{Introducción}
\input{capitulos/introduccion/main.tex}

% Objetivos
\chapter{Objetivos}
\input{capitulos/objetivos/main.tex}

% Metodología
\chapter{Metodología}
\input{capitulos/metodologia/main.tex}

% Resultados y discusión (incluyendo la valoración de impactos y de aspectos de responsabilidad legal, ética y profesional relacionados con el trabajo)
\chapter{Resultados y Discusión}
\input{capitulos/resultados_discusion/main.tex}

% Conclusiones
\chapter{Conclusiones}
\input{capitulos/conclusiones/main.tex}

% Planificación temporal y presupuesto
\chapter{Planificación Temporal y Presupuesto}
\input{capitulos/planificacion_presupuesto/main.tex}

% Bibliografía
\newpage
\addcontentsline{toc}{chapter}{Bibliografía}
\printbibliography

\end{document}


% Metodología
\chapter{Metodología}
\documentclass[a4paper,11pt,twoside]{report}
\usepackage[left=25mm,right=25mm,top=25mm,bottom=25mm,includehead,includefoot,headsep=15mm,footskip=15mm]{geometry}
\usepackage{graphicx}
\usepackage{fancyhdr}
\usepackage{titlesec}
\usepackage[spanish]{babel}
\usepackage[utf8]{inputenc}
\usepackage{amsmath}
\usepackage{setspace}
\usepackage{svg}
\usepackage{hyperref}
\usepackage[backend=biber,style=numeric]{biblatex}
\addbibresource{references.bib}
\hypersetup{
    colorlinks=true,
    linkcolor=blue,      % color of internal links (sections, etc.)
    urlcolor=blue,       % color of external links
    pdftitle={Optimización energética de sistema híbrido con bomba de calor, suelo radiante, fotovoltaica y almacenamiento para vivienda},    % title
    pdfauthor={Luis D. Aranda Sánchez},     % author
    pdfkeywords={palabra1, palabra2, código1, etc.} % list of keywords
}

% Font change to Arial
\usepackage{helvet}
\renewcommand{\familydefault}{\sfdefault}

% Chapter titles in uppercase and larger font
\titleformat{\chapter}[hang]{\large\bfseries}{\thechapter.}{1em}{\MakeUppercase}
\titleformat{\section}[hang]{\bfseries}{\thesection.}{1em}{}
\titleformat{\subsection}[hang]{\bfseries}{\thesubsection.}{1em}{}

% Fancyhdr setup
\setlength{\headheight}{14.30174pt} % Adjust to recommended value, slightly larger for safety
\fancyhf{} % Clear all headers and footers
\fancyhead[LE]{\nouppercase{\leftmark}}
\fancyhead[RO]{Optimización energética para vivienda}
\fancyfoot[LE]{\thepage}
\fancyfoot[RE]{Escuela Técnica Superior de Ingenieros Industriales (UPM)}
\fancyfoot[LO]{Luis D. Aranda Sánchez}
\fancyfoot[RO]{\thepage}
\renewcommand{\headrulewidth}{0.4pt}
\renewcommand{\footrulewidth}{0.4pt}

\fancypagestyle{myfancy}{
    \fancyhf{} % Clear all headers and footers
    \fancyhead[LE]{\nouppercase{\leftmark}}
    \fancyhead[RO]{Optimización energética para vivienda}
    \fancyfoot[LE]{\thepage}
    \fancyfoot[RE]{Escuela Técnica Superior de Ingenieros Industriales (UPM)}
    \fancyfoot[LO]{Luis D. Aranda Sánchez}
    \fancyfoot[RO]{\thepage}
    \renewcommand{\headrulewidth}{0.4pt}
    \renewcommand{\footrulewidth}{0.4pt}
}

\fancypagestyle{simple}{
    \fancyhf{} % Clear all headers and footers
    \renewcommand{\headrulewidth}{0pt}
    \renewcommand{\footrulewidth}{0pt}
}

% Line spacing
\setstretch{1.2}

% Document starts here
\begin{document}

% Portada
\begin{titlepage}
    \centering
    {\scshape\LARGE Universidad Politécnica de Madrid \par}
    \vspace{1cm}
    {\scshape\Large Escuela Técnica Superior de Ingenieros Industriales\par}
    \vspace{1.5cm}
    {\huge\bfseries Optimización energética de sistema híbrido con bomba de calor, suelo radiante, fotovoltaica y almacenamiento para vivienda \par}
    \vspace{1.5cm}
    {\Large\bfseries Trabajo de Fin de Máster\par}
    \vspace{0.5cm}
    {\large Máster Universitario en Ingeniería de la Energía \par}
    \vspace{2cm}
    {\Large Luis D. Aranda Sánchez\par}
    \vfill
    Director: Javier Rodríguez Martín
    \vfill
    {\large Septiembre 6, 2024\par}
\end{titlepage}

% Resumen (máximo de 5 páginas, incluyendo al final Palabras clave)
\clearpage
\pagestyle{simple}
% \newpage
\chapter*{Resumen}
\addcontentsline{toc}{chapter}{Resumen}
\input{capitulos/resumen/main.tex}

% Índice (paginado)
\clearpage
\pagestyle{simple}
% \newpage
\tableofcontents

% Introducción (donde se incluya los antecedentes y justificación)
\clearpage
\pagestyle{myfancy}
\newpage
\chapter{Introducción}
\input{capitulos/introduccion/main.tex}

% Objetivos
\chapter{Objetivos}
\input{capitulos/objetivos/main.tex}

% Metodología
\chapter{Metodología}
\input{capitulos/metodologia/main.tex}

% Resultados y discusión (incluyendo la valoración de impactos y de aspectos de responsabilidad legal, ética y profesional relacionados con el trabajo)
\chapter{Resultados y Discusión}
\input{capitulos/resultados_discusion/main.tex}

% Conclusiones
\chapter{Conclusiones}
\input{capitulos/conclusiones/main.tex}

% Planificación temporal y presupuesto
\chapter{Planificación Temporal y Presupuesto}
\input{capitulos/planificacion_presupuesto/main.tex}

% Bibliografía
\newpage
\addcontentsline{toc}{chapter}{Bibliografía}
\printbibliography

\end{document}


% Resultados y discusión (incluyendo la valoración de impactos y de aspectos de responsabilidad legal, ética y profesional relacionados con el trabajo)
\chapter{Resultados y Discusión}
\documentclass[a4paper,11pt,twoside]{report}
\usepackage[left=25mm,right=25mm,top=25mm,bottom=25mm,includehead,includefoot,headsep=15mm,footskip=15mm]{geometry}
\usepackage{graphicx}
\usepackage{fancyhdr}
\usepackage{titlesec}
\usepackage[spanish]{babel}
\usepackage[utf8]{inputenc}
\usepackage{amsmath}
\usepackage{setspace}
\usepackage{svg}
\usepackage{hyperref}
\usepackage[backend=biber,style=numeric]{biblatex}
\addbibresource{references.bib}
\hypersetup{
    colorlinks=true,
    linkcolor=blue,      % color of internal links (sections, etc.)
    urlcolor=blue,       % color of external links
    pdftitle={Optimización energética de sistema híbrido con bomba de calor, suelo radiante, fotovoltaica y almacenamiento para vivienda},    % title
    pdfauthor={Luis D. Aranda Sánchez},     % author
    pdfkeywords={palabra1, palabra2, código1, etc.} % list of keywords
}

% Font change to Arial
\usepackage{helvet}
\renewcommand{\familydefault}{\sfdefault}

% Chapter titles in uppercase and larger font
\titleformat{\chapter}[hang]{\large\bfseries}{\thechapter.}{1em}{\MakeUppercase}
\titleformat{\section}[hang]{\bfseries}{\thesection.}{1em}{}
\titleformat{\subsection}[hang]{\bfseries}{\thesubsection.}{1em}{}

% Fancyhdr setup
\setlength{\headheight}{14.30174pt} % Adjust to recommended value, slightly larger for safety
\fancyhf{} % Clear all headers and footers
\fancyhead[LE]{\nouppercase{\leftmark}}
\fancyhead[RO]{Optimización energética para vivienda}
\fancyfoot[LE]{\thepage}
\fancyfoot[RE]{Escuela Técnica Superior de Ingenieros Industriales (UPM)}
\fancyfoot[LO]{Luis D. Aranda Sánchez}
\fancyfoot[RO]{\thepage}
\renewcommand{\headrulewidth}{0.4pt}
\renewcommand{\footrulewidth}{0.4pt}

\fancypagestyle{myfancy}{
    \fancyhf{} % Clear all headers and footers
    \fancyhead[LE]{\nouppercase{\leftmark}}
    \fancyhead[RO]{Optimización energética para vivienda}
    \fancyfoot[LE]{\thepage}
    \fancyfoot[RE]{Escuela Técnica Superior de Ingenieros Industriales (UPM)}
    \fancyfoot[LO]{Luis D. Aranda Sánchez}
    \fancyfoot[RO]{\thepage}
    \renewcommand{\headrulewidth}{0.4pt}
    \renewcommand{\footrulewidth}{0.4pt}
}

\fancypagestyle{simple}{
    \fancyhf{} % Clear all headers and footers
    \renewcommand{\headrulewidth}{0pt}
    \renewcommand{\footrulewidth}{0pt}
}

% Line spacing
\setstretch{1.2}

% Document starts here
\begin{document}

% Portada
\begin{titlepage}
    \centering
    {\scshape\LARGE Universidad Politécnica de Madrid \par}
    \vspace{1cm}
    {\scshape\Large Escuela Técnica Superior de Ingenieros Industriales\par}
    \vspace{1.5cm}
    {\huge\bfseries Optimización energética de sistema híbrido con bomba de calor, suelo radiante, fotovoltaica y almacenamiento para vivienda \par}
    \vspace{1.5cm}
    {\Large\bfseries Trabajo de Fin de Máster\par}
    \vspace{0.5cm}
    {\large Máster Universitario en Ingeniería de la Energía \par}
    \vspace{2cm}
    {\Large Luis D. Aranda Sánchez\par}
    \vfill
    Director: Javier Rodríguez Martín
    \vfill
    {\large Septiembre 6, 2024\par}
\end{titlepage}

% Resumen (máximo de 5 páginas, incluyendo al final Palabras clave)
\clearpage
\pagestyle{simple}
% \newpage
\chapter*{Resumen}
\addcontentsline{toc}{chapter}{Resumen}
\input{capitulos/resumen/main.tex}

% Índice (paginado)
\clearpage
\pagestyle{simple}
% \newpage
\tableofcontents

% Introducción (donde se incluya los antecedentes y justificación)
\clearpage
\pagestyle{myfancy}
\newpage
\chapter{Introducción}
\input{capitulos/introduccion/main.tex}

% Objetivos
\chapter{Objetivos}
\input{capitulos/objetivos/main.tex}

% Metodología
\chapter{Metodología}
\input{capitulos/metodologia/main.tex}

% Resultados y discusión (incluyendo la valoración de impactos y de aspectos de responsabilidad legal, ética y profesional relacionados con el trabajo)
\chapter{Resultados y Discusión}
\input{capitulos/resultados_discusion/main.tex}

% Conclusiones
\chapter{Conclusiones}
\input{capitulos/conclusiones/main.tex}

% Planificación temporal y presupuesto
\chapter{Planificación Temporal y Presupuesto}
\input{capitulos/planificacion_presupuesto/main.tex}

% Bibliografía
\newpage
\addcontentsline{toc}{chapter}{Bibliografía}
\printbibliography

\end{document}


% Conclusiones
\chapter{Conclusiones}
\documentclass[a4paper,11pt,twoside]{report}
\usepackage[left=25mm,right=25mm,top=25mm,bottom=25mm,includehead,includefoot,headsep=15mm,footskip=15mm]{geometry}
\usepackage{graphicx}
\usepackage{fancyhdr}
\usepackage{titlesec}
\usepackage[spanish]{babel}
\usepackage[utf8]{inputenc}
\usepackage{amsmath}
\usepackage{setspace}
\usepackage{svg}
\usepackage{hyperref}
\usepackage[backend=biber,style=numeric]{biblatex}
\addbibresource{references.bib}
\hypersetup{
    colorlinks=true,
    linkcolor=blue,      % color of internal links (sections, etc.)
    urlcolor=blue,       % color of external links
    pdftitle={Optimización energética de sistema híbrido con bomba de calor, suelo radiante, fotovoltaica y almacenamiento para vivienda},    % title
    pdfauthor={Luis D. Aranda Sánchez},     % author
    pdfkeywords={palabra1, palabra2, código1, etc.} % list of keywords
}

% Font change to Arial
\usepackage{helvet}
\renewcommand{\familydefault}{\sfdefault}

% Chapter titles in uppercase and larger font
\titleformat{\chapter}[hang]{\large\bfseries}{\thechapter.}{1em}{\MakeUppercase}
\titleformat{\section}[hang]{\bfseries}{\thesection.}{1em}{}
\titleformat{\subsection}[hang]{\bfseries}{\thesubsection.}{1em}{}

% Fancyhdr setup
\setlength{\headheight}{14.30174pt} % Adjust to recommended value, slightly larger for safety
\fancyhf{} % Clear all headers and footers
\fancyhead[LE]{\nouppercase{\leftmark}}
\fancyhead[RO]{Optimización energética para vivienda}
\fancyfoot[LE]{\thepage}
\fancyfoot[RE]{Escuela Técnica Superior de Ingenieros Industriales (UPM)}
\fancyfoot[LO]{Luis D. Aranda Sánchez}
\fancyfoot[RO]{\thepage}
\renewcommand{\headrulewidth}{0.4pt}
\renewcommand{\footrulewidth}{0.4pt}

\fancypagestyle{myfancy}{
    \fancyhf{} % Clear all headers and footers
    \fancyhead[LE]{\nouppercase{\leftmark}}
    \fancyhead[RO]{Optimización energética para vivienda}
    \fancyfoot[LE]{\thepage}
    \fancyfoot[RE]{Escuela Técnica Superior de Ingenieros Industriales (UPM)}
    \fancyfoot[LO]{Luis D. Aranda Sánchez}
    \fancyfoot[RO]{\thepage}
    \renewcommand{\headrulewidth}{0.4pt}
    \renewcommand{\footrulewidth}{0.4pt}
}

\fancypagestyle{simple}{
    \fancyhf{} % Clear all headers and footers
    \renewcommand{\headrulewidth}{0pt}
    \renewcommand{\footrulewidth}{0pt}
}

% Line spacing
\setstretch{1.2}

% Document starts here
\begin{document}

% Portada
\begin{titlepage}
    \centering
    {\scshape\LARGE Universidad Politécnica de Madrid \par}
    \vspace{1cm}
    {\scshape\Large Escuela Técnica Superior de Ingenieros Industriales\par}
    \vspace{1.5cm}
    {\huge\bfseries Optimización energética de sistema híbrido con bomba de calor, suelo radiante, fotovoltaica y almacenamiento para vivienda \par}
    \vspace{1.5cm}
    {\Large\bfseries Trabajo de Fin de Máster\par}
    \vspace{0.5cm}
    {\large Máster Universitario en Ingeniería de la Energía \par}
    \vspace{2cm}
    {\Large Luis D. Aranda Sánchez\par}
    \vfill
    Director: Javier Rodríguez Martín
    \vfill
    {\large Septiembre 6, 2024\par}
\end{titlepage}

% Resumen (máximo de 5 páginas, incluyendo al final Palabras clave)
\clearpage
\pagestyle{simple}
% \newpage
\chapter*{Resumen}
\addcontentsline{toc}{chapter}{Resumen}
\input{capitulos/resumen/main.tex}

% Índice (paginado)
\clearpage
\pagestyle{simple}
% \newpage
\tableofcontents

% Introducción (donde se incluya los antecedentes y justificación)
\clearpage
\pagestyle{myfancy}
\newpage
\chapter{Introducción}
\input{capitulos/introduccion/main.tex}

% Objetivos
\chapter{Objetivos}
\input{capitulos/objetivos/main.tex}

% Metodología
\chapter{Metodología}
\input{capitulos/metodologia/main.tex}

% Resultados y discusión (incluyendo la valoración de impactos y de aspectos de responsabilidad legal, ética y profesional relacionados con el trabajo)
\chapter{Resultados y Discusión}
\input{capitulos/resultados_discusion/main.tex}

% Conclusiones
\chapter{Conclusiones}
\input{capitulos/conclusiones/main.tex}

% Planificación temporal y presupuesto
\chapter{Planificación Temporal y Presupuesto}
\input{capitulos/planificacion_presupuesto/main.tex}

% Bibliografía
\newpage
\addcontentsline{toc}{chapter}{Bibliografía}
\printbibliography

\end{document}


% Planificación temporal y presupuesto
\chapter{Planificación Temporal y Presupuesto}
\documentclass[a4paper,11pt,twoside]{report}
\usepackage[left=25mm,right=25mm,top=25mm,bottom=25mm,includehead,includefoot,headsep=15mm,footskip=15mm]{geometry}
\usepackage{graphicx}
\usepackage{fancyhdr}
\usepackage{titlesec}
\usepackage[spanish]{babel}
\usepackage[utf8]{inputenc}
\usepackage{amsmath}
\usepackage{setspace}
\usepackage{svg}
\usepackage{hyperref}
\usepackage[backend=biber,style=numeric]{biblatex}
\addbibresource{references.bib}
\hypersetup{
    colorlinks=true,
    linkcolor=blue,      % color of internal links (sections, etc.)
    urlcolor=blue,       % color of external links
    pdftitle={Optimización energética de sistema híbrido con bomba de calor, suelo radiante, fotovoltaica y almacenamiento para vivienda},    % title
    pdfauthor={Luis D. Aranda Sánchez},     % author
    pdfkeywords={palabra1, palabra2, código1, etc.} % list of keywords
}

% Font change to Arial
\usepackage{helvet}
\renewcommand{\familydefault}{\sfdefault}

% Chapter titles in uppercase and larger font
\titleformat{\chapter}[hang]{\large\bfseries}{\thechapter.}{1em}{\MakeUppercase}
\titleformat{\section}[hang]{\bfseries}{\thesection.}{1em}{}
\titleformat{\subsection}[hang]{\bfseries}{\thesubsection.}{1em}{}

% Fancyhdr setup
\setlength{\headheight}{14.30174pt} % Adjust to recommended value, slightly larger for safety
\fancyhf{} % Clear all headers and footers
\fancyhead[LE]{\nouppercase{\leftmark}}
\fancyhead[RO]{Optimización energética para vivienda}
\fancyfoot[LE]{\thepage}
\fancyfoot[RE]{Escuela Técnica Superior de Ingenieros Industriales (UPM)}
\fancyfoot[LO]{Luis D. Aranda Sánchez}
\fancyfoot[RO]{\thepage}
\renewcommand{\headrulewidth}{0.4pt}
\renewcommand{\footrulewidth}{0.4pt}

\fancypagestyle{myfancy}{
    \fancyhf{} % Clear all headers and footers
    \fancyhead[LE]{\nouppercase{\leftmark}}
    \fancyhead[RO]{Optimización energética para vivienda}
    \fancyfoot[LE]{\thepage}
    \fancyfoot[RE]{Escuela Técnica Superior de Ingenieros Industriales (UPM)}
    \fancyfoot[LO]{Luis D. Aranda Sánchez}
    \fancyfoot[RO]{\thepage}
    \renewcommand{\headrulewidth}{0.4pt}
    \renewcommand{\footrulewidth}{0.4pt}
}

\fancypagestyle{simple}{
    \fancyhf{} % Clear all headers and footers
    \renewcommand{\headrulewidth}{0pt}
    \renewcommand{\footrulewidth}{0pt}
}

% Line spacing
\setstretch{1.2}

% Document starts here
\begin{document}

% Portada
\begin{titlepage}
    \centering
    {\scshape\LARGE Universidad Politécnica de Madrid \par}
    \vspace{1cm}
    {\scshape\Large Escuela Técnica Superior de Ingenieros Industriales\par}
    \vspace{1.5cm}
    {\huge\bfseries Optimización energética de sistema híbrido con bomba de calor, suelo radiante, fotovoltaica y almacenamiento para vivienda \par}
    \vspace{1.5cm}
    {\Large\bfseries Trabajo de Fin de Máster\par}
    \vspace{0.5cm}
    {\large Máster Universitario en Ingeniería de la Energía \par}
    \vspace{2cm}
    {\Large Luis D. Aranda Sánchez\par}
    \vfill
    Director: Javier Rodríguez Martín
    \vfill
    {\large Septiembre 6, 2024\par}
\end{titlepage}

% Resumen (máximo de 5 páginas, incluyendo al final Palabras clave)
\clearpage
\pagestyle{simple}
% \newpage
\chapter*{Resumen}
\addcontentsline{toc}{chapter}{Resumen}
\input{capitulos/resumen/main.tex}

% Índice (paginado)
\clearpage
\pagestyle{simple}
% \newpage
\tableofcontents

% Introducción (donde se incluya los antecedentes y justificación)
\clearpage
\pagestyle{myfancy}
\newpage
\chapter{Introducción}
\input{capitulos/introduccion/main.tex}

% Objetivos
\chapter{Objetivos}
\input{capitulos/objetivos/main.tex}

% Metodología
\chapter{Metodología}
\input{capitulos/metodologia/main.tex}

% Resultados y discusión (incluyendo la valoración de impactos y de aspectos de responsabilidad legal, ética y profesional relacionados con el trabajo)
\chapter{Resultados y Discusión}
\input{capitulos/resultados_discusion/main.tex}

% Conclusiones
\chapter{Conclusiones}
\input{capitulos/conclusiones/main.tex}

% Planificación temporal y presupuesto
\chapter{Planificación Temporal y Presupuesto}
\input{capitulos/planificacion_presupuesto/main.tex}

% Bibliografía
\newpage
\addcontentsline{toc}{chapter}{Bibliografía}
\printbibliography

\end{document}


% Bibliografía
\newpage
\addcontentsline{toc}{chapter}{Bibliografía}
\printbibliography

\end{document}


% Planificación temporal y presupuesto
\chapter{Planificación Temporal y Presupuesto}
\documentclass[a4paper,11pt,twoside]{report}
\usepackage[left=25mm,right=25mm,top=25mm,bottom=25mm,includehead,includefoot,headsep=15mm,footskip=15mm]{geometry}
\usepackage{graphicx}
\usepackage{fancyhdr}
\usepackage{titlesec}
\usepackage[spanish]{babel}
\usepackage[utf8]{inputenc}
\usepackage{amsmath}
\usepackage{setspace}
\usepackage{svg}
\usepackage{hyperref}
\usepackage[backend=biber,style=numeric]{biblatex}
\addbibresource{references.bib}
\hypersetup{
    colorlinks=true,
    linkcolor=blue,      % color of internal links (sections, etc.)
    urlcolor=blue,       % color of external links
    pdftitle={Optimización energética de sistema híbrido con bomba de calor, suelo radiante, fotovoltaica y almacenamiento para vivienda},    % title
    pdfauthor={Luis D. Aranda Sánchez},     % author
    pdfkeywords={palabra1, palabra2, código1, etc.} % list of keywords
}

% Font change to Arial
\usepackage{helvet}
\renewcommand{\familydefault}{\sfdefault}

% Chapter titles in uppercase and larger font
\titleformat{\chapter}[hang]{\large\bfseries}{\thechapter.}{1em}{\MakeUppercase}
\titleformat{\section}[hang]{\bfseries}{\thesection.}{1em}{}
\titleformat{\subsection}[hang]{\bfseries}{\thesubsection.}{1em}{}

% Fancyhdr setup
\setlength{\headheight}{14.30174pt} % Adjust to recommended value, slightly larger for safety
\fancyhf{} % Clear all headers and footers
\fancyhead[LE]{\nouppercase{\leftmark}}
\fancyhead[RO]{Optimización energética para vivienda}
\fancyfoot[LE]{\thepage}
\fancyfoot[RE]{Escuela Técnica Superior de Ingenieros Industriales (UPM)}
\fancyfoot[LO]{Luis D. Aranda Sánchez}
\fancyfoot[RO]{\thepage}
\renewcommand{\headrulewidth}{0.4pt}
\renewcommand{\footrulewidth}{0.4pt}

\fancypagestyle{myfancy}{
    \fancyhf{} % Clear all headers and footers
    \fancyhead[LE]{\nouppercase{\leftmark}}
    \fancyhead[RO]{Optimización energética para vivienda}
    \fancyfoot[LE]{\thepage}
    \fancyfoot[RE]{Escuela Técnica Superior de Ingenieros Industriales (UPM)}
    \fancyfoot[LO]{Luis D. Aranda Sánchez}
    \fancyfoot[RO]{\thepage}
    \renewcommand{\headrulewidth}{0.4pt}
    \renewcommand{\footrulewidth}{0.4pt}
}

\fancypagestyle{simple}{
    \fancyhf{} % Clear all headers and footers
    \renewcommand{\headrulewidth}{0pt}
    \renewcommand{\footrulewidth}{0pt}
}

% Line spacing
\setstretch{1.2}

% Document starts here
\begin{document}

% Portada
\begin{titlepage}
    \centering
    {\scshape\LARGE Universidad Politécnica de Madrid \par}
    \vspace{1cm}
    {\scshape\Large Escuela Técnica Superior de Ingenieros Industriales\par}
    \vspace{1.5cm}
    {\huge\bfseries Optimización energética de sistema híbrido con bomba de calor, suelo radiante, fotovoltaica y almacenamiento para vivienda \par}
    \vspace{1.5cm}
    {\Large\bfseries Trabajo de Fin de Máster\par}
    \vspace{0.5cm}
    {\large Máster Universitario en Ingeniería de la Energía \par}
    \vspace{2cm}
    {\Large Luis D. Aranda Sánchez\par}
    \vfill
    Director: Javier Rodríguez Martín
    \vfill
    {\large Septiembre 6, 2024\par}
\end{titlepage}

% Resumen (máximo de 5 páginas, incluyendo al final Palabras clave)
\clearpage
\pagestyle{simple}
% \newpage
\chapter*{Resumen}
\addcontentsline{toc}{chapter}{Resumen}
\documentclass[a4paper,11pt,twoside]{report}
\usepackage[left=25mm,right=25mm,top=25mm,bottom=25mm,includehead,includefoot,headsep=15mm,footskip=15mm]{geometry}
\usepackage{graphicx}
\usepackage{fancyhdr}
\usepackage{titlesec}
\usepackage[spanish]{babel}
\usepackage[utf8]{inputenc}
\usepackage{amsmath}
\usepackage{setspace}
\usepackage{svg}
\usepackage{hyperref}
\usepackage[backend=biber,style=numeric]{biblatex}
\addbibresource{references.bib}
\hypersetup{
    colorlinks=true,
    linkcolor=blue,      % color of internal links (sections, etc.)
    urlcolor=blue,       % color of external links
    pdftitle={Optimización energética de sistema híbrido con bomba de calor, suelo radiante, fotovoltaica y almacenamiento para vivienda},    % title
    pdfauthor={Luis D. Aranda Sánchez},     % author
    pdfkeywords={palabra1, palabra2, código1, etc.} % list of keywords
}

% Font change to Arial
\usepackage{helvet}
\renewcommand{\familydefault}{\sfdefault}

% Chapter titles in uppercase and larger font
\titleformat{\chapter}[hang]{\large\bfseries}{\thechapter.}{1em}{\MakeUppercase}
\titleformat{\section}[hang]{\bfseries}{\thesection.}{1em}{}
\titleformat{\subsection}[hang]{\bfseries}{\thesubsection.}{1em}{}

% Fancyhdr setup
\setlength{\headheight}{14.30174pt} % Adjust to recommended value, slightly larger for safety
\fancyhf{} % Clear all headers and footers
\fancyhead[LE]{\nouppercase{\leftmark}}
\fancyhead[RO]{Optimización energética para vivienda}
\fancyfoot[LE]{\thepage}
\fancyfoot[RE]{Escuela Técnica Superior de Ingenieros Industriales (UPM)}
\fancyfoot[LO]{Luis D. Aranda Sánchez}
\fancyfoot[RO]{\thepage}
\renewcommand{\headrulewidth}{0.4pt}
\renewcommand{\footrulewidth}{0.4pt}

\fancypagestyle{myfancy}{
    \fancyhf{} % Clear all headers and footers
    \fancyhead[LE]{\nouppercase{\leftmark}}
    \fancyhead[RO]{Optimización energética para vivienda}
    \fancyfoot[LE]{\thepage}
    \fancyfoot[RE]{Escuela Técnica Superior de Ingenieros Industriales (UPM)}
    \fancyfoot[LO]{Luis D. Aranda Sánchez}
    \fancyfoot[RO]{\thepage}
    \renewcommand{\headrulewidth}{0.4pt}
    \renewcommand{\footrulewidth}{0.4pt}
}

\fancypagestyle{simple}{
    \fancyhf{} % Clear all headers and footers
    \renewcommand{\headrulewidth}{0pt}
    \renewcommand{\footrulewidth}{0pt}
}

% Line spacing
\setstretch{1.2}

% Document starts here
\begin{document}

% Portada
\begin{titlepage}
    \centering
    {\scshape\LARGE Universidad Politécnica de Madrid \par}
    \vspace{1cm}
    {\scshape\Large Escuela Técnica Superior de Ingenieros Industriales\par}
    \vspace{1.5cm}
    {\huge\bfseries Optimización energética de sistema híbrido con bomba de calor, suelo radiante, fotovoltaica y almacenamiento para vivienda \par}
    \vspace{1.5cm}
    {\Large\bfseries Trabajo de Fin de Máster\par}
    \vspace{0.5cm}
    {\large Máster Universitario en Ingeniería de la Energía \par}
    \vspace{2cm}
    {\Large Luis D. Aranda Sánchez\par}
    \vfill
    Director: Javier Rodríguez Martín
    \vfill
    {\large Septiembre 6, 2024\par}
\end{titlepage}

% Resumen (máximo de 5 páginas, incluyendo al final Palabras clave)
\clearpage
\pagestyle{simple}
% \newpage
\chapter*{Resumen}
\addcontentsline{toc}{chapter}{Resumen}
\input{capitulos/resumen/main.tex}

% Índice (paginado)
\clearpage
\pagestyle{simple}
% \newpage
\tableofcontents

% Introducción (donde se incluya los antecedentes y justificación)
\clearpage
\pagestyle{myfancy}
\newpage
\chapter{Introducción}
\input{capitulos/introduccion/main.tex}

% Objetivos
\chapter{Objetivos}
\input{capitulos/objetivos/main.tex}

% Metodología
\chapter{Metodología}
\input{capitulos/metodologia/main.tex}

% Resultados y discusión (incluyendo la valoración de impactos y de aspectos de responsabilidad legal, ética y profesional relacionados con el trabajo)
\chapter{Resultados y Discusión}
\input{capitulos/resultados_discusion/main.tex}

% Conclusiones
\chapter{Conclusiones}
\input{capitulos/conclusiones/main.tex}

% Planificación temporal y presupuesto
\chapter{Planificación Temporal y Presupuesto}
\input{capitulos/planificacion_presupuesto/main.tex}

% Bibliografía
\newpage
\addcontentsline{toc}{chapter}{Bibliografía}
\printbibliography

\end{document}


% Índice (paginado)
\clearpage
\pagestyle{simple}
% \newpage
\tableofcontents

% Introducción (donde se incluya los antecedentes y justificación)
\clearpage
\pagestyle{myfancy}
\newpage
\chapter{Introducción}
\documentclass[a4paper,11pt,twoside]{report}
\usepackage[left=25mm,right=25mm,top=25mm,bottom=25mm,includehead,includefoot,headsep=15mm,footskip=15mm]{geometry}
\usepackage{graphicx}
\usepackage{fancyhdr}
\usepackage{titlesec}
\usepackage[spanish]{babel}
\usepackage[utf8]{inputenc}
\usepackage{amsmath}
\usepackage{setspace}
\usepackage{svg}
\usepackage{hyperref}
\usepackage[backend=biber,style=numeric]{biblatex}
\addbibresource{references.bib}
\hypersetup{
    colorlinks=true,
    linkcolor=blue,      % color of internal links (sections, etc.)
    urlcolor=blue,       % color of external links
    pdftitle={Optimización energética de sistema híbrido con bomba de calor, suelo radiante, fotovoltaica y almacenamiento para vivienda},    % title
    pdfauthor={Luis D. Aranda Sánchez},     % author
    pdfkeywords={palabra1, palabra2, código1, etc.} % list of keywords
}

% Font change to Arial
\usepackage{helvet}
\renewcommand{\familydefault}{\sfdefault}

% Chapter titles in uppercase and larger font
\titleformat{\chapter}[hang]{\large\bfseries}{\thechapter.}{1em}{\MakeUppercase}
\titleformat{\section}[hang]{\bfseries}{\thesection.}{1em}{}
\titleformat{\subsection}[hang]{\bfseries}{\thesubsection.}{1em}{}

% Fancyhdr setup
\setlength{\headheight}{14.30174pt} % Adjust to recommended value, slightly larger for safety
\fancyhf{} % Clear all headers and footers
\fancyhead[LE]{\nouppercase{\leftmark}}
\fancyhead[RO]{Optimización energética para vivienda}
\fancyfoot[LE]{\thepage}
\fancyfoot[RE]{Escuela Técnica Superior de Ingenieros Industriales (UPM)}
\fancyfoot[LO]{Luis D. Aranda Sánchez}
\fancyfoot[RO]{\thepage}
\renewcommand{\headrulewidth}{0.4pt}
\renewcommand{\footrulewidth}{0.4pt}

\fancypagestyle{myfancy}{
    \fancyhf{} % Clear all headers and footers
    \fancyhead[LE]{\nouppercase{\leftmark}}
    \fancyhead[RO]{Optimización energética para vivienda}
    \fancyfoot[LE]{\thepage}
    \fancyfoot[RE]{Escuela Técnica Superior de Ingenieros Industriales (UPM)}
    \fancyfoot[LO]{Luis D. Aranda Sánchez}
    \fancyfoot[RO]{\thepage}
    \renewcommand{\headrulewidth}{0.4pt}
    \renewcommand{\footrulewidth}{0.4pt}
}

\fancypagestyle{simple}{
    \fancyhf{} % Clear all headers and footers
    \renewcommand{\headrulewidth}{0pt}
    \renewcommand{\footrulewidth}{0pt}
}

% Line spacing
\setstretch{1.2}

% Document starts here
\begin{document}

% Portada
\begin{titlepage}
    \centering
    {\scshape\LARGE Universidad Politécnica de Madrid \par}
    \vspace{1cm}
    {\scshape\Large Escuela Técnica Superior de Ingenieros Industriales\par}
    \vspace{1.5cm}
    {\huge\bfseries Optimización energética de sistema híbrido con bomba de calor, suelo radiante, fotovoltaica y almacenamiento para vivienda \par}
    \vspace{1.5cm}
    {\Large\bfseries Trabajo de Fin de Máster\par}
    \vspace{0.5cm}
    {\large Máster Universitario en Ingeniería de la Energía \par}
    \vspace{2cm}
    {\Large Luis D. Aranda Sánchez\par}
    \vfill
    Director: Javier Rodríguez Martín
    \vfill
    {\large Septiembre 6, 2024\par}
\end{titlepage}

% Resumen (máximo de 5 páginas, incluyendo al final Palabras clave)
\clearpage
\pagestyle{simple}
% \newpage
\chapter*{Resumen}
\addcontentsline{toc}{chapter}{Resumen}
\input{capitulos/resumen/main.tex}

% Índice (paginado)
\clearpage
\pagestyle{simple}
% \newpage
\tableofcontents

% Introducción (donde se incluya los antecedentes y justificación)
\clearpage
\pagestyle{myfancy}
\newpage
\chapter{Introducción}
\input{capitulos/introduccion/main.tex}

% Objetivos
\chapter{Objetivos}
\input{capitulos/objetivos/main.tex}

% Metodología
\chapter{Metodología}
\input{capitulos/metodologia/main.tex}

% Resultados y discusión (incluyendo la valoración de impactos y de aspectos de responsabilidad legal, ética y profesional relacionados con el trabajo)
\chapter{Resultados y Discusión}
\input{capitulos/resultados_discusion/main.tex}

% Conclusiones
\chapter{Conclusiones}
\input{capitulos/conclusiones/main.tex}

% Planificación temporal y presupuesto
\chapter{Planificación Temporal y Presupuesto}
\input{capitulos/planificacion_presupuesto/main.tex}

% Bibliografía
\newpage
\addcontentsline{toc}{chapter}{Bibliografía}
\printbibliography

\end{document}


% Objetivos
\chapter{Objetivos}
\documentclass[a4paper,11pt,twoside]{report}
\usepackage[left=25mm,right=25mm,top=25mm,bottom=25mm,includehead,includefoot,headsep=15mm,footskip=15mm]{geometry}
\usepackage{graphicx}
\usepackage{fancyhdr}
\usepackage{titlesec}
\usepackage[spanish]{babel}
\usepackage[utf8]{inputenc}
\usepackage{amsmath}
\usepackage{setspace}
\usepackage{svg}
\usepackage{hyperref}
\usepackage[backend=biber,style=numeric]{biblatex}
\addbibresource{references.bib}
\hypersetup{
    colorlinks=true,
    linkcolor=blue,      % color of internal links (sections, etc.)
    urlcolor=blue,       % color of external links
    pdftitle={Optimización energética de sistema híbrido con bomba de calor, suelo radiante, fotovoltaica y almacenamiento para vivienda},    % title
    pdfauthor={Luis D. Aranda Sánchez},     % author
    pdfkeywords={palabra1, palabra2, código1, etc.} % list of keywords
}

% Font change to Arial
\usepackage{helvet}
\renewcommand{\familydefault}{\sfdefault}

% Chapter titles in uppercase and larger font
\titleformat{\chapter}[hang]{\large\bfseries}{\thechapter.}{1em}{\MakeUppercase}
\titleformat{\section}[hang]{\bfseries}{\thesection.}{1em}{}
\titleformat{\subsection}[hang]{\bfseries}{\thesubsection.}{1em}{}

% Fancyhdr setup
\setlength{\headheight}{14.30174pt} % Adjust to recommended value, slightly larger for safety
\fancyhf{} % Clear all headers and footers
\fancyhead[LE]{\nouppercase{\leftmark}}
\fancyhead[RO]{Optimización energética para vivienda}
\fancyfoot[LE]{\thepage}
\fancyfoot[RE]{Escuela Técnica Superior de Ingenieros Industriales (UPM)}
\fancyfoot[LO]{Luis D. Aranda Sánchez}
\fancyfoot[RO]{\thepage}
\renewcommand{\headrulewidth}{0.4pt}
\renewcommand{\footrulewidth}{0.4pt}

\fancypagestyle{myfancy}{
    \fancyhf{} % Clear all headers and footers
    \fancyhead[LE]{\nouppercase{\leftmark}}
    \fancyhead[RO]{Optimización energética para vivienda}
    \fancyfoot[LE]{\thepage}
    \fancyfoot[RE]{Escuela Técnica Superior de Ingenieros Industriales (UPM)}
    \fancyfoot[LO]{Luis D. Aranda Sánchez}
    \fancyfoot[RO]{\thepage}
    \renewcommand{\headrulewidth}{0.4pt}
    \renewcommand{\footrulewidth}{0.4pt}
}

\fancypagestyle{simple}{
    \fancyhf{} % Clear all headers and footers
    \renewcommand{\headrulewidth}{0pt}
    \renewcommand{\footrulewidth}{0pt}
}

% Line spacing
\setstretch{1.2}

% Document starts here
\begin{document}

% Portada
\begin{titlepage}
    \centering
    {\scshape\LARGE Universidad Politécnica de Madrid \par}
    \vspace{1cm}
    {\scshape\Large Escuela Técnica Superior de Ingenieros Industriales\par}
    \vspace{1.5cm}
    {\huge\bfseries Optimización energética de sistema híbrido con bomba de calor, suelo radiante, fotovoltaica y almacenamiento para vivienda \par}
    \vspace{1.5cm}
    {\Large\bfseries Trabajo de Fin de Máster\par}
    \vspace{0.5cm}
    {\large Máster Universitario en Ingeniería de la Energía \par}
    \vspace{2cm}
    {\Large Luis D. Aranda Sánchez\par}
    \vfill
    Director: Javier Rodríguez Martín
    \vfill
    {\large Septiembre 6, 2024\par}
\end{titlepage}

% Resumen (máximo de 5 páginas, incluyendo al final Palabras clave)
\clearpage
\pagestyle{simple}
% \newpage
\chapter*{Resumen}
\addcontentsline{toc}{chapter}{Resumen}
\input{capitulos/resumen/main.tex}

% Índice (paginado)
\clearpage
\pagestyle{simple}
% \newpage
\tableofcontents

% Introducción (donde se incluya los antecedentes y justificación)
\clearpage
\pagestyle{myfancy}
\newpage
\chapter{Introducción}
\input{capitulos/introduccion/main.tex}

% Objetivos
\chapter{Objetivos}
\input{capitulos/objetivos/main.tex}

% Metodología
\chapter{Metodología}
\input{capitulos/metodologia/main.tex}

% Resultados y discusión (incluyendo la valoración de impactos y de aspectos de responsabilidad legal, ética y profesional relacionados con el trabajo)
\chapter{Resultados y Discusión}
\input{capitulos/resultados_discusion/main.tex}

% Conclusiones
\chapter{Conclusiones}
\input{capitulos/conclusiones/main.tex}

% Planificación temporal y presupuesto
\chapter{Planificación Temporal y Presupuesto}
\input{capitulos/planificacion_presupuesto/main.tex}

% Bibliografía
\newpage
\addcontentsline{toc}{chapter}{Bibliografía}
\printbibliography

\end{document}


% Metodología
\chapter{Metodología}
\documentclass[a4paper,11pt,twoside]{report}
\usepackage[left=25mm,right=25mm,top=25mm,bottom=25mm,includehead,includefoot,headsep=15mm,footskip=15mm]{geometry}
\usepackage{graphicx}
\usepackage{fancyhdr}
\usepackage{titlesec}
\usepackage[spanish]{babel}
\usepackage[utf8]{inputenc}
\usepackage{amsmath}
\usepackage{setspace}
\usepackage{svg}
\usepackage{hyperref}
\usepackage[backend=biber,style=numeric]{biblatex}
\addbibresource{references.bib}
\hypersetup{
    colorlinks=true,
    linkcolor=blue,      % color of internal links (sections, etc.)
    urlcolor=blue,       % color of external links
    pdftitle={Optimización energética de sistema híbrido con bomba de calor, suelo radiante, fotovoltaica y almacenamiento para vivienda},    % title
    pdfauthor={Luis D. Aranda Sánchez},     % author
    pdfkeywords={palabra1, palabra2, código1, etc.} % list of keywords
}

% Font change to Arial
\usepackage{helvet}
\renewcommand{\familydefault}{\sfdefault}

% Chapter titles in uppercase and larger font
\titleformat{\chapter}[hang]{\large\bfseries}{\thechapter.}{1em}{\MakeUppercase}
\titleformat{\section}[hang]{\bfseries}{\thesection.}{1em}{}
\titleformat{\subsection}[hang]{\bfseries}{\thesubsection.}{1em}{}

% Fancyhdr setup
\setlength{\headheight}{14.30174pt} % Adjust to recommended value, slightly larger for safety
\fancyhf{} % Clear all headers and footers
\fancyhead[LE]{\nouppercase{\leftmark}}
\fancyhead[RO]{Optimización energética para vivienda}
\fancyfoot[LE]{\thepage}
\fancyfoot[RE]{Escuela Técnica Superior de Ingenieros Industriales (UPM)}
\fancyfoot[LO]{Luis D. Aranda Sánchez}
\fancyfoot[RO]{\thepage}
\renewcommand{\headrulewidth}{0.4pt}
\renewcommand{\footrulewidth}{0.4pt}

\fancypagestyle{myfancy}{
    \fancyhf{} % Clear all headers and footers
    \fancyhead[LE]{\nouppercase{\leftmark}}
    \fancyhead[RO]{Optimización energética para vivienda}
    \fancyfoot[LE]{\thepage}
    \fancyfoot[RE]{Escuela Técnica Superior de Ingenieros Industriales (UPM)}
    \fancyfoot[LO]{Luis D. Aranda Sánchez}
    \fancyfoot[RO]{\thepage}
    \renewcommand{\headrulewidth}{0.4pt}
    \renewcommand{\footrulewidth}{0.4pt}
}

\fancypagestyle{simple}{
    \fancyhf{} % Clear all headers and footers
    \renewcommand{\headrulewidth}{0pt}
    \renewcommand{\footrulewidth}{0pt}
}

% Line spacing
\setstretch{1.2}

% Document starts here
\begin{document}

% Portada
\begin{titlepage}
    \centering
    {\scshape\LARGE Universidad Politécnica de Madrid \par}
    \vspace{1cm}
    {\scshape\Large Escuela Técnica Superior de Ingenieros Industriales\par}
    \vspace{1.5cm}
    {\huge\bfseries Optimización energética de sistema híbrido con bomba de calor, suelo radiante, fotovoltaica y almacenamiento para vivienda \par}
    \vspace{1.5cm}
    {\Large\bfseries Trabajo de Fin de Máster\par}
    \vspace{0.5cm}
    {\large Máster Universitario en Ingeniería de la Energía \par}
    \vspace{2cm}
    {\Large Luis D. Aranda Sánchez\par}
    \vfill
    Director: Javier Rodríguez Martín
    \vfill
    {\large Septiembre 6, 2024\par}
\end{titlepage}

% Resumen (máximo de 5 páginas, incluyendo al final Palabras clave)
\clearpage
\pagestyle{simple}
% \newpage
\chapter*{Resumen}
\addcontentsline{toc}{chapter}{Resumen}
\input{capitulos/resumen/main.tex}

% Índice (paginado)
\clearpage
\pagestyle{simple}
% \newpage
\tableofcontents

% Introducción (donde se incluya los antecedentes y justificación)
\clearpage
\pagestyle{myfancy}
\newpage
\chapter{Introducción}
\input{capitulos/introduccion/main.tex}

% Objetivos
\chapter{Objetivos}
\input{capitulos/objetivos/main.tex}

% Metodología
\chapter{Metodología}
\input{capitulos/metodologia/main.tex}

% Resultados y discusión (incluyendo la valoración de impactos y de aspectos de responsabilidad legal, ética y profesional relacionados con el trabajo)
\chapter{Resultados y Discusión}
\input{capitulos/resultados_discusion/main.tex}

% Conclusiones
\chapter{Conclusiones}
\input{capitulos/conclusiones/main.tex}

% Planificación temporal y presupuesto
\chapter{Planificación Temporal y Presupuesto}
\input{capitulos/planificacion_presupuesto/main.tex}

% Bibliografía
\newpage
\addcontentsline{toc}{chapter}{Bibliografía}
\printbibliography

\end{document}


% Resultados y discusión (incluyendo la valoración de impactos y de aspectos de responsabilidad legal, ética y profesional relacionados con el trabajo)
\chapter{Resultados y Discusión}
\documentclass[a4paper,11pt,twoside]{report}
\usepackage[left=25mm,right=25mm,top=25mm,bottom=25mm,includehead,includefoot,headsep=15mm,footskip=15mm]{geometry}
\usepackage{graphicx}
\usepackage{fancyhdr}
\usepackage{titlesec}
\usepackage[spanish]{babel}
\usepackage[utf8]{inputenc}
\usepackage{amsmath}
\usepackage{setspace}
\usepackage{svg}
\usepackage{hyperref}
\usepackage[backend=biber,style=numeric]{biblatex}
\addbibresource{references.bib}
\hypersetup{
    colorlinks=true,
    linkcolor=blue,      % color of internal links (sections, etc.)
    urlcolor=blue,       % color of external links
    pdftitle={Optimización energética de sistema híbrido con bomba de calor, suelo radiante, fotovoltaica y almacenamiento para vivienda},    % title
    pdfauthor={Luis D. Aranda Sánchez},     % author
    pdfkeywords={palabra1, palabra2, código1, etc.} % list of keywords
}

% Font change to Arial
\usepackage{helvet}
\renewcommand{\familydefault}{\sfdefault}

% Chapter titles in uppercase and larger font
\titleformat{\chapter}[hang]{\large\bfseries}{\thechapter.}{1em}{\MakeUppercase}
\titleformat{\section}[hang]{\bfseries}{\thesection.}{1em}{}
\titleformat{\subsection}[hang]{\bfseries}{\thesubsection.}{1em}{}

% Fancyhdr setup
\setlength{\headheight}{14.30174pt} % Adjust to recommended value, slightly larger for safety
\fancyhf{} % Clear all headers and footers
\fancyhead[LE]{\nouppercase{\leftmark}}
\fancyhead[RO]{Optimización energética para vivienda}
\fancyfoot[LE]{\thepage}
\fancyfoot[RE]{Escuela Técnica Superior de Ingenieros Industriales (UPM)}
\fancyfoot[LO]{Luis D. Aranda Sánchez}
\fancyfoot[RO]{\thepage}
\renewcommand{\headrulewidth}{0.4pt}
\renewcommand{\footrulewidth}{0.4pt}

\fancypagestyle{myfancy}{
    \fancyhf{} % Clear all headers and footers
    \fancyhead[LE]{\nouppercase{\leftmark}}
    \fancyhead[RO]{Optimización energética para vivienda}
    \fancyfoot[LE]{\thepage}
    \fancyfoot[RE]{Escuela Técnica Superior de Ingenieros Industriales (UPM)}
    \fancyfoot[LO]{Luis D. Aranda Sánchez}
    \fancyfoot[RO]{\thepage}
    \renewcommand{\headrulewidth}{0.4pt}
    \renewcommand{\footrulewidth}{0.4pt}
}

\fancypagestyle{simple}{
    \fancyhf{} % Clear all headers and footers
    \renewcommand{\headrulewidth}{0pt}
    \renewcommand{\footrulewidth}{0pt}
}

% Line spacing
\setstretch{1.2}

% Document starts here
\begin{document}

% Portada
\begin{titlepage}
    \centering
    {\scshape\LARGE Universidad Politécnica de Madrid \par}
    \vspace{1cm}
    {\scshape\Large Escuela Técnica Superior de Ingenieros Industriales\par}
    \vspace{1.5cm}
    {\huge\bfseries Optimización energética de sistema híbrido con bomba de calor, suelo radiante, fotovoltaica y almacenamiento para vivienda \par}
    \vspace{1.5cm}
    {\Large\bfseries Trabajo de Fin de Máster\par}
    \vspace{0.5cm}
    {\large Máster Universitario en Ingeniería de la Energía \par}
    \vspace{2cm}
    {\Large Luis D. Aranda Sánchez\par}
    \vfill
    Director: Javier Rodríguez Martín
    \vfill
    {\large Septiembre 6, 2024\par}
\end{titlepage}

% Resumen (máximo de 5 páginas, incluyendo al final Palabras clave)
\clearpage
\pagestyle{simple}
% \newpage
\chapter*{Resumen}
\addcontentsline{toc}{chapter}{Resumen}
\input{capitulos/resumen/main.tex}

% Índice (paginado)
\clearpage
\pagestyle{simple}
% \newpage
\tableofcontents

% Introducción (donde se incluya los antecedentes y justificación)
\clearpage
\pagestyle{myfancy}
\newpage
\chapter{Introducción}
\input{capitulos/introduccion/main.tex}

% Objetivos
\chapter{Objetivos}
\input{capitulos/objetivos/main.tex}

% Metodología
\chapter{Metodología}
\input{capitulos/metodologia/main.tex}

% Resultados y discusión (incluyendo la valoración de impactos y de aspectos de responsabilidad legal, ética y profesional relacionados con el trabajo)
\chapter{Resultados y Discusión}
\input{capitulos/resultados_discusion/main.tex}

% Conclusiones
\chapter{Conclusiones}
\input{capitulos/conclusiones/main.tex}

% Planificación temporal y presupuesto
\chapter{Planificación Temporal y Presupuesto}
\input{capitulos/planificacion_presupuesto/main.tex}

% Bibliografía
\newpage
\addcontentsline{toc}{chapter}{Bibliografía}
\printbibliography

\end{document}


% Conclusiones
\chapter{Conclusiones}
\documentclass[a4paper,11pt,twoside]{report}
\usepackage[left=25mm,right=25mm,top=25mm,bottom=25mm,includehead,includefoot,headsep=15mm,footskip=15mm]{geometry}
\usepackage{graphicx}
\usepackage{fancyhdr}
\usepackage{titlesec}
\usepackage[spanish]{babel}
\usepackage[utf8]{inputenc}
\usepackage{amsmath}
\usepackage{setspace}
\usepackage{svg}
\usepackage{hyperref}
\usepackage[backend=biber,style=numeric]{biblatex}
\addbibresource{references.bib}
\hypersetup{
    colorlinks=true,
    linkcolor=blue,      % color of internal links (sections, etc.)
    urlcolor=blue,       % color of external links
    pdftitle={Optimización energética de sistema híbrido con bomba de calor, suelo radiante, fotovoltaica y almacenamiento para vivienda},    % title
    pdfauthor={Luis D. Aranda Sánchez},     % author
    pdfkeywords={palabra1, palabra2, código1, etc.} % list of keywords
}

% Font change to Arial
\usepackage{helvet}
\renewcommand{\familydefault}{\sfdefault}

% Chapter titles in uppercase and larger font
\titleformat{\chapter}[hang]{\large\bfseries}{\thechapter.}{1em}{\MakeUppercase}
\titleformat{\section}[hang]{\bfseries}{\thesection.}{1em}{}
\titleformat{\subsection}[hang]{\bfseries}{\thesubsection.}{1em}{}

% Fancyhdr setup
\setlength{\headheight}{14.30174pt} % Adjust to recommended value, slightly larger for safety
\fancyhf{} % Clear all headers and footers
\fancyhead[LE]{\nouppercase{\leftmark}}
\fancyhead[RO]{Optimización energética para vivienda}
\fancyfoot[LE]{\thepage}
\fancyfoot[RE]{Escuela Técnica Superior de Ingenieros Industriales (UPM)}
\fancyfoot[LO]{Luis D. Aranda Sánchez}
\fancyfoot[RO]{\thepage}
\renewcommand{\headrulewidth}{0.4pt}
\renewcommand{\footrulewidth}{0.4pt}

\fancypagestyle{myfancy}{
    \fancyhf{} % Clear all headers and footers
    \fancyhead[LE]{\nouppercase{\leftmark}}
    \fancyhead[RO]{Optimización energética para vivienda}
    \fancyfoot[LE]{\thepage}
    \fancyfoot[RE]{Escuela Técnica Superior de Ingenieros Industriales (UPM)}
    \fancyfoot[LO]{Luis D. Aranda Sánchez}
    \fancyfoot[RO]{\thepage}
    \renewcommand{\headrulewidth}{0.4pt}
    \renewcommand{\footrulewidth}{0.4pt}
}

\fancypagestyle{simple}{
    \fancyhf{} % Clear all headers and footers
    \renewcommand{\headrulewidth}{0pt}
    \renewcommand{\footrulewidth}{0pt}
}

% Line spacing
\setstretch{1.2}

% Document starts here
\begin{document}

% Portada
\begin{titlepage}
    \centering
    {\scshape\LARGE Universidad Politécnica de Madrid \par}
    \vspace{1cm}
    {\scshape\Large Escuela Técnica Superior de Ingenieros Industriales\par}
    \vspace{1.5cm}
    {\huge\bfseries Optimización energética de sistema híbrido con bomba de calor, suelo radiante, fotovoltaica y almacenamiento para vivienda \par}
    \vspace{1.5cm}
    {\Large\bfseries Trabajo de Fin de Máster\par}
    \vspace{0.5cm}
    {\large Máster Universitario en Ingeniería de la Energía \par}
    \vspace{2cm}
    {\Large Luis D. Aranda Sánchez\par}
    \vfill
    Director: Javier Rodríguez Martín
    \vfill
    {\large Septiembre 6, 2024\par}
\end{titlepage}

% Resumen (máximo de 5 páginas, incluyendo al final Palabras clave)
\clearpage
\pagestyle{simple}
% \newpage
\chapter*{Resumen}
\addcontentsline{toc}{chapter}{Resumen}
\input{capitulos/resumen/main.tex}

% Índice (paginado)
\clearpage
\pagestyle{simple}
% \newpage
\tableofcontents

% Introducción (donde se incluya los antecedentes y justificación)
\clearpage
\pagestyle{myfancy}
\newpage
\chapter{Introducción}
\input{capitulos/introduccion/main.tex}

% Objetivos
\chapter{Objetivos}
\input{capitulos/objetivos/main.tex}

% Metodología
\chapter{Metodología}
\input{capitulos/metodologia/main.tex}

% Resultados y discusión (incluyendo la valoración de impactos y de aspectos de responsabilidad legal, ética y profesional relacionados con el trabajo)
\chapter{Resultados y Discusión}
\input{capitulos/resultados_discusion/main.tex}

% Conclusiones
\chapter{Conclusiones}
\input{capitulos/conclusiones/main.tex}

% Planificación temporal y presupuesto
\chapter{Planificación Temporal y Presupuesto}
\input{capitulos/planificacion_presupuesto/main.tex}

% Bibliografía
\newpage
\addcontentsline{toc}{chapter}{Bibliografía}
\printbibliography

\end{document}


% Planificación temporal y presupuesto
\chapter{Planificación Temporal y Presupuesto}
\documentclass[a4paper,11pt,twoside]{report}
\usepackage[left=25mm,right=25mm,top=25mm,bottom=25mm,includehead,includefoot,headsep=15mm,footskip=15mm]{geometry}
\usepackage{graphicx}
\usepackage{fancyhdr}
\usepackage{titlesec}
\usepackage[spanish]{babel}
\usepackage[utf8]{inputenc}
\usepackage{amsmath}
\usepackage{setspace}
\usepackage{svg}
\usepackage{hyperref}
\usepackage[backend=biber,style=numeric]{biblatex}
\addbibresource{references.bib}
\hypersetup{
    colorlinks=true,
    linkcolor=blue,      % color of internal links (sections, etc.)
    urlcolor=blue,       % color of external links
    pdftitle={Optimización energética de sistema híbrido con bomba de calor, suelo radiante, fotovoltaica y almacenamiento para vivienda},    % title
    pdfauthor={Luis D. Aranda Sánchez},     % author
    pdfkeywords={palabra1, palabra2, código1, etc.} % list of keywords
}

% Font change to Arial
\usepackage{helvet}
\renewcommand{\familydefault}{\sfdefault}

% Chapter titles in uppercase and larger font
\titleformat{\chapter}[hang]{\large\bfseries}{\thechapter.}{1em}{\MakeUppercase}
\titleformat{\section}[hang]{\bfseries}{\thesection.}{1em}{}
\titleformat{\subsection}[hang]{\bfseries}{\thesubsection.}{1em}{}

% Fancyhdr setup
\setlength{\headheight}{14.30174pt} % Adjust to recommended value, slightly larger for safety
\fancyhf{} % Clear all headers and footers
\fancyhead[LE]{\nouppercase{\leftmark}}
\fancyhead[RO]{Optimización energética para vivienda}
\fancyfoot[LE]{\thepage}
\fancyfoot[RE]{Escuela Técnica Superior de Ingenieros Industriales (UPM)}
\fancyfoot[LO]{Luis D. Aranda Sánchez}
\fancyfoot[RO]{\thepage}
\renewcommand{\headrulewidth}{0.4pt}
\renewcommand{\footrulewidth}{0.4pt}

\fancypagestyle{myfancy}{
    \fancyhf{} % Clear all headers and footers
    \fancyhead[LE]{\nouppercase{\leftmark}}
    \fancyhead[RO]{Optimización energética para vivienda}
    \fancyfoot[LE]{\thepage}
    \fancyfoot[RE]{Escuela Técnica Superior de Ingenieros Industriales (UPM)}
    \fancyfoot[LO]{Luis D. Aranda Sánchez}
    \fancyfoot[RO]{\thepage}
    \renewcommand{\headrulewidth}{0.4pt}
    \renewcommand{\footrulewidth}{0.4pt}
}

\fancypagestyle{simple}{
    \fancyhf{} % Clear all headers and footers
    \renewcommand{\headrulewidth}{0pt}
    \renewcommand{\footrulewidth}{0pt}
}

% Line spacing
\setstretch{1.2}

% Document starts here
\begin{document}

% Portada
\begin{titlepage}
    \centering
    {\scshape\LARGE Universidad Politécnica de Madrid \par}
    \vspace{1cm}
    {\scshape\Large Escuela Técnica Superior de Ingenieros Industriales\par}
    \vspace{1.5cm}
    {\huge\bfseries Optimización energética de sistema híbrido con bomba de calor, suelo radiante, fotovoltaica y almacenamiento para vivienda \par}
    \vspace{1.5cm}
    {\Large\bfseries Trabajo de Fin de Máster\par}
    \vspace{0.5cm}
    {\large Máster Universitario en Ingeniería de la Energía \par}
    \vspace{2cm}
    {\Large Luis D. Aranda Sánchez\par}
    \vfill
    Director: Javier Rodríguez Martín
    \vfill
    {\large Septiembre 6, 2024\par}
\end{titlepage}

% Resumen (máximo de 5 páginas, incluyendo al final Palabras clave)
\clearpage
\pagestyle{simple}
% \newpage
\chapter*{Resumen}
\addcontentsline{toc}{chapter}{Resumen}
\input{capitulos/resumen/main.tex}

% Índice (paginado)
\clearpage
\pagestyle{simple}
% \newpage
\tableofcontents

% Introducción (donde se incluya los antecedentes y justificación)
\clearpage
\pagestyle{myfancy}
\newpage
\chapter{Introducción}
\input{capitulos/introduccion/main.tex}

% Objetivos
\chapter{Objetivos}
\input{capitulos/objetivos/main.tex}

% Metodología
\chapter{Metodología}
\input{capitulos/metodologia/main.tex}

% Resultados y discusión (incluyendo la valoración de impactos y de aspectos de responsabilidad legal, ética y profesional relacionados con el trabajo)
\chapter{Resultados y Discusión}
\input{capitulos/resultados_discusion/main.tex}

% Conclusiones
\chapter{Conclusiones}
\input{capitulos/conclusiones/main.tex}

% Planificación temporal y presupuesto
\chapter{Planificación Temporal y Presupuesto}
\input{capitulos/planificacion_presupuesto/main.tex}

% Bibliografía
\newpage
\addcontentsline{toc}{chapter}{Bibliografía}
\printbibliography

\end{document}


% Bibliografía
\newpage
\addcontentsline{toc}{chapter}{Bibliografía}
\printbibliography

\end{document}


% Bibliografía
\newpage
\addcontentsline{toc}{chapter}{Bibliografía}
\printbibliography

\end{document}


% Metodología
\cleardoublepage
\chapter{Metodología}
\documentclass[a4paper,11pt,twoside]{report}
\usepackage[left=25mm,right=25mm,top=25mm,bottom=25mm,includehead,includefoot,headsep=15mm,footskip=15mm]{geometry}
\usepackage{graphicx}
\usepackage{fancyhdr}
\usepackage{titlesec}
\usepackage[spanish]{babel}
\usepackage[utf8]{inputenc}
\usepackage{amsmath}
\usepackage{setspace}
\usepackage{svg}
\usepackage{hyperref}
\usepackage[backend=biber,style=numeric]{biblatex}
\addbibresource{references.bib}
\hypersetup{
    colorlinks=true,
    linkcolor=blue,      % color of internal links (sections, etc.)
    urlcolor=blue,       % color of external links
    pdftitle={Optimización energética de sistema híbrido con bomba de calor, suelo radiante, fotovoltaica y almacenamiento para vivienda},    % title
    pdfauthor={Luis D. Aranda Sánchez},     % author
    pdfkeywords={palabra1, palabra2, código1, etc.} % list of keywords
}

% Font change to Arial
\usepackage{helvet}
\renewcommand{\familydefault}{\sfdefault}

% Chapter titles in uppercase and larger font
\titleformat{\chapter}[hang]{\large\bfseries}{\thechapter.}{1em}{\MakeUppercase}
\titleformat{\section}[hang]{\bfseries}{\thesection.}{1em}{}
\titleformat{\subsection}[hang]{\bfseries}{\thesubsection.}{1em}{}

% Fancyhdr setup
\setlength{\headheight}{14.30174pt} % Adjust to recommended value, slightly larger for safety
\fancyhf{} % Clear all headers and footers
\fancyhead[LE]{\nouppercase{\leftmark}}
\fancyhead[RO]{Optimización energética para vivienda}
\fancyfoot[LE]{\thepage}
\fancyfoot[RE]{Escuela Técnica Superior de Ingenieros Industriales (UPM)}
\fancyfoot[LO]{Luis D. Aranda Sánchez}
\fancyfoot[RO]{\thepage}
\renewcommand{\headrulewidth}{0.4pt}
\renewcommand{\footrulewidth}{0.4pt}

\fancypagestyle{myfancy}{
    \fancyhf{} % Clear all headers and footers
    \fancyhead[LE]{\nouppercase{\leftmark}}
    \fancyhead[RO]{Optimización energética para vivienda}
    \fancyfoot[LE]{\thepage}
    \fancyfoot[RE]{Escuela Técnica Superior de Ingenieros Industriales (UPM)}
    \fancyfoot[LO]{Luis D. Aranda Sánchez}
    \fancyfoot[RO]{\thepage}
    \renewcommand{\headrulewidth}{0.4pt}
    \renewcommand{\footrulewidth}{0.4pt}
}

\fancypagestyle{simple}{
    \fancyhf{} % Clear all headers and footers
    \renewcommand{\headrulewidth}{0pt}
    \renewcommand{\footrulewidth}{0pt}
}

% Line spacing
\setstretch{1.2}

% Document starts here
\begin{document}

% Portada
\begin{titlepage}
    \centering
    {\scshape\LARGE Universidad Politécnica de Madrid \par}
    \vspace{1cm}
    {\scshape\Large Escuela Técnica Superior de Ingenieros Industriales\par}
    \vspace{1.5cm}
    {\huge\bfseries Optimización energética de sistema híbrido con bomba de calor, suelo radiante, fotovoltaica y almacenamiento para vivienda \par}
    \vspace{1.5cm}
    {\Large\bfseries Trabajo de Fin de Máster\par}
    \vspace{0.5cm}
    {\large Máster Universitario en Ingeniería de la Energía \par}
    \vspace{2cm}
    {\Large Luis D. Aranda Sánchez\par}
    \vfill
    Director: Javier Rodríguez Martín
    \vfill
    {\large Septiembre 6, 2024\par}
\end{titlepage}

% Resumen (máximo de 5 páginas, incluyendo al final Palabras clave)
\clearpage
\pagestyle{simple}
% \newpage
\chapter*{Resumen}
\addcontentsline{toc}{chapter}{Resumen}
\documentclass[a4paper,11pt,twoside]{report}
\usepackage[left=25mm,right=25mm,top=25mm,bottom=25mm,includehead,includefoot,headsep=15mm,footskip=15mm]{geometry}
\usepackage{graphicx}
\usepackage{fancyhdr}
\usepackage{titlesec}
\usepackage[spanish]{babel}
\usepackage[utf8]{inputenc}
\usepackage{amsmath}
\usepackage{setspace}
\usepackage{svg}
\usepackage{hyperref}
\usepackage[backend=biber,style=numeric]{biblatex}
\addbibresource{references.bib}
\hypersetup{
    colorlinks=true,
    linkcolor=blue,      % color of internal links (sections, etc.)
    urlcolor=blue,       % color of external links
    pdftitle={Optimización energética de sistema híbrido con bomba de calor, suelo radiante, fotovoltaica y almacenamiento para vivienda},    % title
    pdfauthor={Luis D. Aranda Sánchez},     % author
    pdfkeywords={palabra1, palabra2, código1, etc.} % list of keywords
}

% Font change to Arial
\usepackage{helvet}
\renewcommand{\familydefault}{\sfdefault}

% Chapter titles in uppercase and larger font
\titleformat{\chapter}[hang]{\large\bfseries}{\thechapter.}{1em}{\MakeUppercase}
\titleformat{\section}[hang]{\bfseries}{\thesection.}{1em}{}
\titleformat{\subsection}[hang]{\bfseries}{\thesubsection.}{1em}{}

% Fancyhdr setup
\setlength{\headheight}{14.30174pt} % Adjust to recommended value, slightly larger for safety
\fancyhf{} % Clear all headers and footers
\fancyhead[LE]{\nouppercase{\leftmark}}
\fancyhead[RO]{Optimización energética para vivienda}
\fancyfoot[LE]{\thepage}
\fancyfoot[RE]{Escuela Técnica Superior de Ingenieros Industriales (UPM)}
\fancyfoot[LO]{Luis D. Aranda Sánchez}
\fancyfoot[RO]{\thepage}
\renewcommand{\headrulewidth}{0.4pt}
\renewcommand{\footrulewidth}{0.4pt}

\fancypagestyle{myfancy}{
    \fancyhf{} % Clear all headers and footers
    \fancyhead[LE]{\nouppercase{\leftmark}}
    \fancyhead[RO]{Optimización energética para vivienda}
    \fancyfoot[LE]{\thepage}
    \fancyfoot[RE]{Escuela Técnica Superior de Ingenieros Industriales (UPM)}
    \fancyfoot[LO]{Luis D. Aranda Sánchez}
    \fancyfoot[RO]{\thepage}
    \renewcommand{\headrulewidth}{0.4pt}
    \renewcommand{\footrulewidth}{0.4pt}
}

\fancypagestyle{simple}{
    \fancyhf{} % Clear all headers and footers
    \renewcommand{\headrulewidth}{0pt}
    \renewcommand{\footrulewidth}{0pt}
}

% Line spacing
\setstretch{1.2}

% Document starts here
\begin{document}

% Portada
\begin{titlepage}
    \centering
    {\scshape\LARGE Universidad Politécnica de Madrid \par}
    \vspace{1cm}
    {\scshape\Large Escuela Técnica Superior de Ingenieros Industriales\par}
    \vspace{1.5cm}
    {\huge\bfseries Optimización energética de sistema híbrido con bomba de calor, suelo radiante, fotovoltaica y almacenamiento para vivienda \par}
    \vspace{1.5cm}
    {\Large\bfseries Trabajo de Fin de Máster\par}
    \vspace{0.5cm}
    {\large Máster Universitario en Ingeniería de la Energía \par}
    \vspace{2cm}
    {\Large Luis D. Aranda Sánchez\par}
    \vfill
    Director: Javier Rodríguez Martín
    \vfill
    {\large Septiembre 6, 2024\par}
\end{titlepage}

% Resumen (máximo de 5 páginas, incluyendo al final Palabras clave)
\clearpage
\pagestyle{simple}
% \newpage
\chapter*{Resumen}
\addcontentsline{toc}{chapter}{Resumen}
\documentclass[a4paper,11pt,twoside]{report}
\usepackage[left=25mm,right=25mm,top=25mm,bottom=25mm,includehead,includefoot,headsep=15mm,footskip=15mm]{geometry}
\usepackage{graphicx}
\usepackage{fancyhdr}
\usepackage{titlesec}
\usepackage[spanish]{babel}
\usepackage[utf8]{inputenc}
\usepackage{amsmath}
\usepackage{setspace}
\usepackage{svg}
\usepackage{hyperref}
\usepackage[backend=biber,style=numeric]{biblatex}
\addbibresource{references.bib}
\hypersetup{
    colorlinks=true,
    linkcolor=blue,      % color of internal links (sections, etc.)
    urlcolor=blue,       % color of external links
    pdftitle={Optimización energética de sistema híbrido con bomba de calor, suelo radiante, fotovoltaica y almacenamiento para vivienda},    % title
    pdfauthor={Luis D. Aranda Sánchez},     % author
    pdfkeywords={palabra1, palabra2, código1, etc.} % list of keywords
}

% Font change to Arial
\usepackage{helvet}
\renewcommand{\familydefault}{\sfdefault}

% Chapter titles in uppercase and larger font
\titleformat{\chapter}[hang]{\large\bfseries}{\thechapter.}{1em}{\MakeUppercase}
\titleformat{\section}[hang]{\bfseries}{\thesection.}{1em}{}
\titleformat{\subsection}[hang]{\bfseries}{\thesubsection.}{1em}{}

% Fancyhdr setup
\setlength{\headheight}{14.30174pt} % Adjust to recommended value, slightly larger for safety
\fancyhf{} % Clear all headers and footers
\fancyhead[LE]{\nouppercase{\leftmark}}
\fancyhead[RO]{Optimización energética para vivienda}
\fancyfoot[LE]{\thepage}
\fancyfoot[RE]{Escuela Técnica Superior de Ingenieros Industriales (UPM)}
\fancyfoot[LO]{Luis D. Aranda Sánchez}
\fancyfoot[RO]{\thepage}
\renewcommand{\headrulewidth}{0.4pt}
\renewcommand{\footrulewidth}{0.4pt}

\fancypagestyle{myfancy}{
    \fancyhf{} % Clear all headers and footers
    \fancyhead[LE]{\nouppercase{\leftmark}}
    \fancyhead[RO]{Optimización energética para vivienda}
    \fancyfoot[LE]{\thepage}
    \fancyfoot[RE]{Escuela Técnica Superior de Ingenieros Industriales (UPM)}
    \fancyfoot[LO]{Luis D. Aranda Sánchez}
    \fancyfoot[RO]{\thepage}
    \renewcommand{\headrulewidth}{0.4pt}
    \renewcommand{\footrulewidth}{0.4pt}
}

\fancypagestyle{simple}{
    \fancyhf{} % Clear all headers and footers
    \renewcommand{\headrulewidth}{0pt}
    \renewcommand{\footrulewidth}{0pt}
}

% Line spacing
\setstretch{1.2}

% Document starts here
\begin{document}

% Portada
\begin{titlepage}
    \centering
    {\scshape\LARGE Universidad Politécnica de Madrid \par}
    \vspace{1cm}
    {\scshape\Large Escuela Técnica Superior de Ingenieros Industriales\par}
    \vspace{1.5cm}
    {\huge\bfseries Optimización energética de sistema híbrido con bomba de calor, suelo radiante, fotovoltaica y almacenamiento para vivienda \par}
    \vspace{1.5cm}
    {\Large\bfseries Trabajo de Fin de Máster\par}
    \vspace{0.5cm}
    {\large Máster Universitario en Ingeniería de la Energía \par}
    \vspace{2cm}
    {\Large Luis D. Aranda Sánchez\par}
    \vfill
    Director: Javier Rodríguez Martín
    \vfill
    {\large Septiembre 6, 2024\par}
\end{titlepage}

% Resumen (máximo de 5 páginas, incluyendo al final Palabras clave)
\clearpage
\pagestyle{simple}
% \newpage
\chapter*{Resumen}
\addcontentsline{toc}{chapter}{Resumen}
\input{capitulos/resumen/main.tex}

% Índice (paginado)
\clearpage
\pagestyle{simple}
% \newpage
\tableofcontents

% Introducción (donde se incluya los antecedentes y justificación)
\clearpage
\pagestyle{myfancy}
\newpage
\chapter{Introducción}
\input{capitulos/introduccion/main.tex}

% Objetivos
\chapter{Objetivos}
\input{capitulos/objetivos/main.tex}

% Metodología
\chapter{Metodología}
\input{capitulos/metodologia/main.tex}

% Resultados y discusión (incluyendo la valoración de impactos y de aspectos de responsabilidad legal, ética y profesional relacionados con el trabajo)
\chapter{Resultados y Discusión}
\input{capitulos/resultados_discusion/main.tex}

% Conclusiones
\chapter{Conclusiones}
\input{capitulos/conclusiones/main.tex}

% Planificación temporal y presupuesto
\chapter{Planificación Temporal y Presupuesto}
\input{capitulos/planificacion_presupuesto/main.tex}

% Bibliografía
\newpage
\addcontentsline{toc}{chapter}{Bibliografía}
\printbibliography

\end{document}


% Índice (paginado)
\clearpage
\pagestyle{simple}
% \newpage
\tableofcontents

% Introducción (donde se incluya los antecedentes y justificación)
\clearpage
\pagestyle{myfancy}
\newpage
\chapter{Introducción}
\documentclass[a4paper,11pt,twoside]{report}
\usepackage[left=25mm,right=25mm,top=25mm,bottom=25mm,includehead,includefoot,headsep=15mm,footskip=15mm]{geometry}
\usepackage{graphicx}
\usepackage{fancyhdr}
\usepackage{titlesec}
\usepackage[spanish]{babel}
\usepackage[utf8]{inputenc}
\usepackage{amsmath}
\usepackage{setspace}
\usepackage{svg}
\usepackage{hyperref}
\usepackage[backend=biber,style=numeric]{biblatex}
\addbibresource{references.bib}
\hypersetup{
    colorlinks=true,
    linkcolor=blue,      % color of internal links (sections, etc.)
    urlcolor=blue,       % color of external links
    pdftitle={Optimización energética de sistema híbrido con bomba de calor, suelo radiante, fotovoltaica y almacenamiento para vivienda},    % title
    pdfauthor={Luis D. Aranda Sánchez},     % author
    pdfkeywords={palabra1, palabra2, código1, etc.} % list of keywords
}

% Font change to Arial
\usepackage{helvet}
\renewcommand{\familydefault}{\sfdefault}

% Chapter titles in uppercase and larger font
\titleformat{\chapter}[hang]{\large\bfseries}{\thechapter.}{1em}{\MakeUppercase}
\titleformat{\section}[hang]{\bfseries}{\thesection.}{1em}{}
\titleformat{\subsection}[hang]{\bfseries}{\thesubsection.}{1em}{}

% Fancyhdr setup
\setlength{\headheight}{14.30174pt} % Adjust to recommended value, slightly larger for safety
\fancyhf{} % Clear all headers and footers
\fancyhead[LE]{\nouppercase{\leftmark}}
\fancyhead[RO]{Optimización energética para vivienda}
\fancyfoot[LE]{\thepage}
\fancyfoot[RE]{Escuela Técnica Superior de Ingenieros Industriales (UPM)}
\fancyfoot[LO]{Luis D. Aranda Sánchez}
\fancyfoot[RO]{\thepage}
\renewcommand{\headrulewidth}{0.4pt}
\renewcommand{\footrulewidth}{0.4pt}

\fancypagestyle{myfancy}{
    \fancyhf{} % Clear all headers and footers
    \fancyhead[LE]{\nouppercase{\leftmark}}
    \fancyhead[RO]{Optimización energética para vivienda}
    \fancyfoot[LE]{\thepage}
    \fancyfoot[RE]{Escuela Técnica Superior de Ingenieros Industriales (UPM)}
    \fancyfoot[LO]{Luis D. Aranda Sánchez}
    \fancyfoot[RO]{\thepage}
    \renewcommand{\headrulewidth}{0.4pt}
    \renewcommand{\footrulewidth}{0.4pt}
}

\fancypagestyle{simple}{
    \fancyhf{} % Clear all headers and footers
    \renewcommand{\headrulewidth}{0pt}
    \renewcommand{\footrulewidth}{0pt}
}

% Line spacing
\setstretch{1.2}

% Document starts here
\begin{document}

% Portada
\begin{titlepage}
    \centering
    {\scshape\LARGE Universidad Politécnica de Madrid \par}
    \vspace{1cm}
    {\scshape\Large Escuela Técnica Superior de Ingenieros Industriales\par}
    \vspace{1.5cm}
    {\huge\bfseries Optimización energética de sistema híbrido con bomba de calor, suelo radiante, fotovoltaica y almacenamiento para vivienda \par}
    \vspace{1.5cm}
    {\Large\bfseries Trabajo de Fin de Máster\par}
    \vspace{0.5cm}
    {\large Máster Universitario en Ingeniería de la Energía \par}
    \vspace{2cm}
    {\Large Luis D. Aranda Sánchez\par}
    \vfill
    Director: Javier Rodríguez Martín
    \vfill
    {\large Septiembre 6, 2024\par}
\end{titlepage}

% Resumen (máximo de 5 páginas, incluyendo al final Palabras clave)
\clearpage
\pagestyle{simple}
% \newpage
\chapter*{Resumen}
\addcontentsline{toc}{chapter}{Resumen}
\input{capitulos/resumen/main.tex}

% Índice (paginado)
\clearpage
\pagestyle{simple}
% \newpage
\tableofcontents

% Introducción (donde se incluya los antecedentes y justificación)
\clearpage
\pagestyle{myfancy}
\newpage
\chapter{Introducción}
\input{capitulos/introduccion/main.tex}

% Objetivos
\chapter{Objetivos}
\input{capitulos/objetivos/main.tex}

% Metodología
\chapter{Metodología}
\input{capitulos/metodologia/main.tex}

% Resultados y discusión (incluyendo la valoración de impactos y de aspectos de responsabilidad legal, ética y profesional relacionados con el trabajo)
\chapter{Resultados y Discusión}
\input{capitulos/resultados_discusion/main.tex}

% Conclusiones
\chapter{Conclusiones}
\input{capitulos/conclusiones/main.tex}

% Planificación temporal y presupuesto
\chapter{Planificación Temporal y Presupuesto}
\input{capitulos/planificacion_presupuesto/main.tex}

% Bibliografía
\newpage
\addcontentsline{toc}{chapter}{Bibliografía}
\printbibliography

\end{document}


% Objetivos
\chapter{Objetivos}
\documentclass[a4paper,11pt,twoside]{report}
\usepackage[left=25mm,right=25mm,top=25mm,bottom=25mm,includehead,includefoot,headsep=15mm,footskip=15mm]{geometry}
\usepackage{graphicx}
\usepackage{fancyhdr}
\usepackage{titlesec}
\usepackage[spanish]{babel}
\usepackage[utf8]{inputenc}
\usepackage{amsmath}
\usepackage{setspace}
\usepackage{svg}
\usepackage{hyperref}
\usepackage[backend=biber,style=numeric]{biblatex}
\addbibresource{references.bib}
\hypersetup{
    colorlinks=true,
    linkcolor=blue,      % color of internal links (sections, etc.)
    urlcolor=blue,       % color of external links
    pdftitle={Optimización energética de sistema híbrido con bomba de calor, suelo radiante, fotovoltaica y almacenamiento para vivienda},    % title
    pdfauthor={Luis D. Aranda Sánchez},     % author
    pdfkeywords={palabra1, palabra2, código1, etc.} % list of keywords
}

% Font change to Arial
\usepackage{helvet}
\renewcommand{\familydefault}{\sfdefault}

% Chapter titles in uppercase and larger font
\titleformat{\chapter}[hang]{\large\bfseries}{\thechapter.}{1em}{\MakeUppercase}
\titleformat{\section}[hang]{\bfseries}{\thesection.}{1em}{}
\titleformat{\subsection}[hang]{\bfseries}{\thesubsection.}{1em}{}

% Fancyhdr setup
\setlength{\headheight}{14.30174pt} % Adjust to recommended value, slightly larger for safety
\fancyhf{} % Clear all headers and footers
\fancyhead[LE]{\nouppercase{\leftmark}}
\fancyhead[RO]{Optimización energética para vivienda}
\fancyfoot[LE]{\thepage}
\fancyfoot[RE]{Escuela Técnica Superior de Ingenieros Industriales (UPM)}
\fancyfoot[LO]{Luis D. Aranda Sánchez}
\fancyfoot[RO]{\thepage}
\renewcommand{\headrulewidth}{0.4pt}
\renewcommand{\footrulewidth}{0.4pt}

\fancypagestyle{myfancy}{
    \fancyhf{} % Clear all headers and footers
    \fancyhead[LE]{\nouppercase{\leftmark}}
    \fancyhead[RO]{Optimización energética para vivienda}
    \fancyfoot[LE]{\thepage}
    \fancyfoot[RE]{Escuela Técnica Superior de Ingenieros Industriales (UPM)}
    \fancyfoot[LO]{Luis D. Aranda Sánchez}
    \fancyfoot[RO]{\thepage}
    \renewcommand{\headrulewidth}{0.4pt}
    \renewcommand{\footrulewidth}{0.4pt}
}

\fancypagestyle{simple}{
    \fancyhf{} % Clear all headers and footers
    \renewcommand{\headrulewidth}{0pt}
    \renewcommand{\footrulewidth}{0pt}
}

% Line spacing
\setstretch{1.2}

% Document starts here
\begin{document}

% Portada
\begin{titlepage}
    \centering
    {\scshape\LARGE Universidad Politécnica de Madrid \par}
    \vspace{1cm}
    {\scshape\Large Escuela Técnica Superior de Ingenieros Industriales\par}
    \vspace{1.5cm}
    {\huge\bfseries Optimización energética de sistema híbrido con bomba de calor, suelo radiante, fotovoltaica y almacenamiento para vivienda \par}
    \vspace{1.5cm}
    {\Large\bfseries Trabajo de Fin de Máster\par}
    \vspace{0.5cm}
    {\large Máster Universitario en Ingeniería de la Energía \par}
    \vspace{2cm}
    {\Large Luis D. Aranda Sánchez\par}
    \vfill
    Director: Javier Rodríguez Martín
    \vfill
    {\large Septiembre 6, 2024\par}
\end{titlepage}

% Resumen (máximo de 5 páginas, incluyendo al final Palabras clave)
\clearpage
\pagestyle{simple}
% \newpage
\chapter*{Resumen}
\addcontentsline{toc}{chapter}{Resumen}
\input{capitulos/resumen/main.tex}

% Índice (paginado)
\clearpage
\pagestyle{simple}
% \newpage
\tableofcontents

% Introducción (donde se incluya los antecedentes y justificación)
\clearpage
\pagestyle{myfancy}
\newpage
\chapter{Introducción}
\input{capitulos/introduccion/main.tex}

% Objetivos
\chapter{Objetivos}
\input{capitulos/objetivos/main.tex}

% Metodología
\chapter{Metodología}
\input{capitulos/metodologia/main.tex}

% Resultados y discusión (incluyendo la valoración de impactos y de aspectos de responsabilidad legal, ética y profesional relacionados con el trabajo)
\chapter{Resultados y Discusión}
\input{capitulos/resultados_discusion/main.tex}

% Conclusiones
\chapter{Conclusiones}
\input{capitulos/conclusiones/main.tex}

% Planificación temporal y presupuesto
\chapter{Planificación Temporal y Presupuesto}
\input{capitulos/planificacion_presupuesto/main.tex}

% Bibliografía
\newpage
\addcontentsline{toc}{chapter}{Bibliografía}
\printbibliography

\end{document}


% Metodología
\chapter{Metodología}
\documentclass[a4paper,11pt,twoside]{report}
\usepackage[left=25mm,right=25mm,top=25mm,bottom=25mm,includehead,includefoot,headsep=15mm,footskip=15mm]{geometry}
\usepackage{graphicx}
\usepackage{fancyhdr}
\usepackage{titlesec}
\usepackage[spanish]{babel}
\usepackage[utf8]{inputenc}
\usepackage{amsmath}
\usepackage{setspace}
\usepackage{svg}
\usepackage{hyperref}
\usepackage[backend=biber,style=numeric]{biblatex}
\addbibresource{references.bib}
\hypersetup{
    colorlinks=true,
    linkcolor=blue,      % color of internal links (sections, etc.)
    urlcolor=blue,       % color of external links
    pdftitle={Optimización energética de sistema híbrido con bomba de calor, suelo radiante, fotovoltaica y almacenamiento para vivienda},    % title
    pdfauthor={Luis D. Aranda Sánchez},     % author
    pdfkeywords={palabra1, palabra2, código1, etc.} % list of keywords
}

% Font change to Arial
\usepackage{helvet}
\renewcommand{\familydefault}{\sfdefault}

% Chapter titles in uppercase and larger font
\titleformat{\chapter}[hang]{\large\bfseries}{\thechapter.}{1em}{\MakeUppercase}
\titleformat{\section}[hang]{\bfseries}{\thesection.}{1em}{}
\titleformat{\subsection}[hang]{\bfseries}{\thesubsection.}{1em}{}

% Fancyhdr setup
\setlength{\headheight}{14.30174pt} % Adjust to recommended value, slightly larger for safety
\fancyhf{} % Clear all headers and footers
\fancyhead[LE]{\nouppercase{\leftmark}}
\fancyhead[RO]{Optimización energética para vivienda}
\fancyfoot[LE]{\thepage}
\fancyfoot[RE]{Escuela Técnica Superior de Ingenieros Industriales (UPM)}
\fancyfoot[LO]{Luis D. Aranda Sánchez}
\fancyfoot[RO]{\thepage}
\renewcommand{\headrulewidth}{0.4pt}
\renewcommand{\footrulewidth}{0.4pt}

\fancypagestyle{myfancy}{
    \fancyhf{} % Clear all headers and footers
    \fancyhead[LE]{\nouppercase{\leftmark}}
    \fancyhead[RO]{Optimización energética para vivienda}
    \fancyfoot[LE]{\thepage}
    \fancyfoot[RE]{Escuela Técnica Superior de Ingenieros Industriales (UPM)}
    \fancyfoot[LO]{Luis D. Aranda Sánchez}
    \fancyfoot[RO]{\thepage}
    \renewcommand{\headrulewidth}{0.4pt}
    \renewcommand{\footrulewidth}{0.4pt}
}

\fancypagestyle{simple}{
    \fancyhf{} % Clear all headers and footers
    \renewcommand{\headrulewidth}{0pt}
    \renewcommand{\footrulewidth}{0pt}
}

% Line spacing
\setstretch{1.2}

% Document starts here
\begin{document}

% Portada
\begin{titlepage}
    \centering
    {\scshape\LARGE Universidad Politécnica de Madrid \par}
    \vspace{1cm}
    {\scshape\Large Escuela Técnica Superior de Ingenieros Industriales\par}
    \vspace{1.5cm}
    {\huge\bfseries Optimización energética de sistema híbrido con bomba de calor, suelo radiante, fotovoltaica y almacenamiento para vivienda \par}
    \vspace{1.5cm}
    {\Large\bfseries Trabajo de Fin de Máster\par}
    \vspace{0.5cm}
    {\large Máster Universitario en Ingeniería de la Energía \par}
    \vspace{2cm}
    {\Large Luis D. Aranda Sánchez\par}
    \vfill
    Director: Javier Rodríguez Martín
    \vfill
    {\large Septiembre 6, 2024\par}
\end{titlepage}

% Resumen (máximo de 5 páginas, incluyendo al final Palabras clave)
\clearpage
\pagestyle{simple}
% \newpage
\chapter*{Resumen}
\addcontentsline{toc}{chapter}{Resumen}
\input{capitulos/resumen/main.tex}

% Índice (paginado)
\clearpage
\pagestyle{simple}
% \newpage
\tableofcontents

% Introducción (donde se incluya los antecedentes y justificación)
\clearpage
\pagestyle{myfancy}
\newpage
\chapter{Introducción}
\input{capitulos/introduccion/main.tex}

% Objetivos
\chapter{Objetivos}
\input{capitulos/objetivos/main.tex}

% Metodología
\chapter{Metodología}
\input{capitulos/metodologia/main.tex}

% Resultados y discusión (incluyendo la valoración de impactos y de aspectos de responsabilidad legal, ética y profesional relacionados con el trabajo)
\chapter{Resultados y Discusión}
\input{capitulos/resultados_discusion/main.tex}

% Conclusiones
\chapter{Conclusiones}
\input{capitulos/conclusiones/main.tex}

% Planificación temporal y presupuesto
\chapter{Planificación Temporal y Presupuesto}
\input{capitulos/planificacion_presupuesto/main.tex}

% Bibliografía
\newpage
\addcontentsline{toc}{chapter}{Bibliografía}
\printbibliography

\end{document}


% Resultados y discusión (incluyendo la valoración de impactos y de aspectos de responsabilidad legal, ética y profesional relacionados con el trabajo)
\chapter{Resultados y Discusión}
\documentclass[a4paper,11pt,twoside]{report}
\usepackage[left=25mm,right=25mm,top=25mm,bottom=25mm,includehead,includefoot,headsep=15mm,footskip=15mm]{geometry}
\usepackage{graphicx}
\usepackage{fancyhdr}
\usepackage{titlesec}
\usepackage[spanish]{babel}
\usepackage[utf8]{inputenc}
\usepackage{amsmath}
\usepackage{setspace}
\usepackage{svg}
\usepackage{hyperref}
\usepackage[backend=biber,style=numeric]{biblatex}
\addbibresource{references.bib}
\hypersetup{
    colorlinks=true,
    linkcolor=blue,      % color of internal links (sections, etc.)
    urlcolor=blue,       % color of external links
    pdftitle={Optimización energética de sistema híbrido con bomba de calor, suelo radiante, fotovoltaica y almacenamiento para vivienda},    % title
    pdfauthor={Luis D. Aranda Sánchez},     % author
    pdfkeywords={palabra1, palabra2, código1, etc.} % list of keywords
}

% Font change to Arial
\usepackage{helvet}
\renewcommand{\familydefault}{\sfdefault}

% Chapter titles in uppercase and larger font
\titleformat{\chapter}[hang]{\large\bfseries}{\thechapter.}{1em}{\MakeUppercase}
\titleformat{\section}[hang]{\bfseries}{\thesection.}{1em}{}
\titleformat{\subsection}[hang]{\bfseries}{\thesubsection.}{1em}{}

% Fancyhdr setup
\setlength{\headheight}{14.30174pt} % Adjust to recommended value, slightly larger for safety
\fancyhf{} % Clear all headers and footers
\fancyhead[LE]{\nouppercase{\leftmark}}
\fancyhead[RO]{Optimización energética para vivienda}
\fancyfoot[LE]{\thepage}
\fancyfoot[RE]{Escuela Técnica Superior de Ingenieros Industriales (UPM)}
\fancyfoot[LO]{Luis D. Aranda Sánchez}
\fancyfoot[RO]{\thepage}
\renewcommand{\headrulewidth}{0.4pt}
\renewcommand{\footrulewidth}{0.4pt}

\fancypagestyle{myfancy}{
    \fancyhf{} % Clear all headers and footers
    \fancyhead[LE]{\nouppercase{\leftmark}}
    \fancyhead[RO]{Optimización energética para vivienda}
    \fancyfoot[LE]{\thepage}
    \fancyfoot[RE]{Escuela Técnica Superior de Ingenieros Industriales (UPM)}
    \fancyfoot[LO]{Luis D. Aranda Sánchez}
    \fancyfoot[RO]{\thepage}
    \renewcommand{\headrulewidth}{0.4pt}
    \renewcommand{\footrulewidth}{0.4pt}
}

\fancypagestyle{simple}{
    \fancyhf{} % Clear all headers and footers
    \renewcommand{\headrulewidth}{0pt}
    \renewcommand{\footrulewidth}{0pt}
}

% Line spacing
\setstretch{1.2}

% Document starts here
\begin{document}

% Portada
\begin{titlepage}
    \centering
    {\scshape\LARGE Universidad Politécnica de Madrid \par}
    \vspace{1cm}
    {\scshape\Large Escuela Técnica Superior de Ingenieros Industriales\par}
    \vspace{1.5cm}
    {\huge\bfseries Optimización energética de sistema híbrido con bomba de calor, suelo radiante, fotovoltaica y almacenamiento para vivienda \par}
    \vspace{1.5cm}
    {\Large\bfseries Trabajo de Fin de Máster\par}
    \vspace{0.5cm}
    {\large Máster Universitario en Ingeniería de la Energía \par}
    \vspace{2cm}
    {\Large Luis D. Aranda Sánchez\par}
    \vfill
    Director: Javier Rodríguez Martín
    \vfill
    {\large Septiembre 6, 2024\par}
\end{titlepage}

% Resumen (máximo de 5 páginas, incluyendo al final Palabras clave)
\clearpage
\pagestyle{simple}
% \newpage
\chapter*{Resumen}
\addcontentsline{toc}{chapter}{Resumen}
\input{capitulos/resumen/main.tex}

% Índice (paginado)
\clearpage
\pagestyle{simple}
% \newpage
\tableofcontents

% Introducción (donde se incluya los antecedentes y justificación)
\clearpage
\pagestyle{myfancy}
\newpage
\chapter{Introducción}
\input{capitulos/introduccion/main.tex}

% Objetivos
\chapter{Objetivos}
\input{capitulos/objetivos/main.tex}

% Metodología
\chapter{Metodología}
\input{capitulos/metodologia/main.tex}

% Resultados y discusión (incluyendo la valoración de impactos y de aspectos de responsabilidad legal, ética y profesional relacionados con el trabajo)
\chapter{Resultados y Discusión}
\input{capitulos/resultados_discusion/main.tex}

% Conclusiones
\chapter{Conclusiones}
\input{capitulos/conclusiones/main.tex}

% Planificación temporal y presupuesto
\chapter{Planificación Temporal y Presupuesto}
\input{capitulos/planificacion_presupuesto/main.tex}

% Bibliografía
\newpage
\addcontentsline{toc}{chapter}{Bibliografía}
\printbibliography

\end{document}


% Conclusiones
\chapter{Conclusiones}
\documentclass[a4paper,11pt,twoside]{report}
\usepackage[left=25mm,right=25mm,top=25mm,bottom=25mm,includehead,includefoot,headsep=15mm,footskip=15mm]{geometry}
\usepackage{graphicx}
\usepackage{fancyhdr}
\usepackage{titlesec}
\usepackage[spanish]{babel}
\usepackage[utf8]{inputenc}
\usepackage{amsmath}
\usepackage{setspace}
\usepackage{svg}
\usepackage{hyperref}
\usepackage[backend=biber,style=numeric]{biblatex}
\addbibresource{references.bib}
\hypersetup{
    colorlinks=true,
    linkcolor=blue,      % color of internal links (sections, etc.)
    urlcolor=blue,       % color of external links
    pdftitle={Optimización energética de sistema híbrido con bomba de calor, suelo radiante, fotovoltaica y almacenamiento para vivienda},    % title
    pdfauthor={Luis D. Aranda Sánchez},     % author
    pdfkeywords={palabra1, palabra2, código1, etc.} % list of keywords
}

% Font change to Arial
\usepackage{helvet}
\renewcommand{\familydefault}{\sfdefault}

% Chapter titles in uppercase and larger font
\titleformat{\chapter}[hang]{\large\bfseries}{\thechapter.}{1em}{\MakeUppercase}
\titleformat{\section}[hang]{\bfseries}{\thesection.}{1em}{}
\titleformat{\subsection}[hang]{\bfseries}{\thesubsection.}{1em}{}

% Fancyhdr setup
\setlength{\headheight}{14.30174pt} % Adjust to recommended value, slightly larger for safety
\fancyhf{} % Clear all headers and footers
\fancyhead[LE]{\nouppercase{\leftmark}}
\fancyhead[RO]{Optimización energética para vivienda}
\fancyfoot[LE]{\thepage}
\fancyfoot[RE]{Escuela Técnica Superior de Ingenieros Industriales (UPM)}
\fancyfoot[LO]{Luis D. Aranda Sánchez}
\fancyfoot[RO]{\thepage}
\renewcommand{\headrulewidth}{0.4pt}
\renewcommand{\footrulewidth}{0.4pt}

\fancypagestyle{myfancy}{
    \fancyhf{} % Clear all headers and footers
    \fancyhead[LE]{\nouppercase{\leftmark}}
    \fancyhead[RO]{Optimización energética para vivienda}
    \fancyfoot[LE]{\thepage}
    \fancyfoot[RE]{Escuela Técnica Superior de Ingenieros Industriales (UPM)}
    \fancyfoot[LO]{Luis D. Aranda Sánchez}
    \fancyfoot[RO]{\thepage}
    \renewcommand{\headrulewidth}{0.4pt}
    \renewcommand{\footrulewidth}{0.4pt}
}

\fancypagestyle{simple}{
    \fancyhf{} % Clear all headers and footers
    \renewcommand{\headrulewidth}{0pt}
    \renewcommand{\footrulewidth}{0pt}
}

% Line spacing
\setstretch{1.2}

% Document starts here
\begin{document}

% Portada
\begin{titlepage}
    \centering
    {\scshape\LARGE Universidad Politécnica de Madrid \par}
    \vspace{1cm}
    {\scshape\Large Escuela Técnica Superior de Ingenieros Industriales\par}
    \vspace{1.5cm}
    {\huge\bfseries Optimización energética de sistema híbrido con bomba de calor, suelo radiante, fotovoltaica y almacenamiento para vivienda \par}
    \vspace{1.5cm}
    {\Large\bfseries Trabajo de Fin de Máster\par}
    \vspace{0.5cm}
    {\large Máster Universitario en Ingeniería de la Energía \par}
    \vspace{2cm}
    {\Large Luis D. Aranda Sánchez\par}
    \vfill
    Director: Javier Rodríguez Martín
    \vfill
    {\large Septiembre 6, 2024\par}
\end{titlepage}

% Resumen (máximo de 5 páginas, incluyendo al final Palabras clave)
\clearpage
\pagestyle{simple}
% \newpage
\chapter*{Resumen}
\addcontentsline{toc}{chapter}{Resumen}
\input{capitulos/resumen/main.tex}

% Índice (paginado)
\clearpage
\pagestyle{simple}
% \newpage
\tableofcontents

% Introducción (donde se incluya los antecedentes y justificación)
\clearpage
\pagestyle{myfancy}
\newpage
\chapter{Introducción}
\input{capitulos/introduccion/main.tex}

% Objetivos
\chapter{Objetivos}
\input{capitulos/objetivos/main.tex}

% Metodología
\chapter{Metodología}
\input{capitulos/metodologia/main.tex}

% Resultados y discusión (incluyendo la valoración de impactos y de aspectos de responsabilidad legal, ética y profesional relacionados con el trabajo)
\chapter{Resultados y Discusión}
\input{capitulos/resultados_discusion/main.tex}

% Conclusiones
\chapter{Conclusiones}
\input{capitulos/conclusiones/main.tex}

% Planificación temporal y presupuesto
\chapter{Planificación Temporal y Presupuesto}
\input{capitulos/planificacion_presupuesto/main.tex}

% Bibliografía
\newpage
\addcontentsline{toc}{chapter}{Bibliografía}
\printbibliography

\end{document}


% Planificación temporal y presupuesto
\chapter{Planificación Temporal y Presupuesto}
\documentclass[a4paper,11pt,twoside]{report}
\usepackage[left=25mm,right=25mm,top=25mm,bottom=25mm,includehead,includefoot,headsep=15mm,footskip=15mm]{geometry}
\usepackage{graphicx}
\usepackage{fancyhdr}
\usepackage{titlesec}
\usepackage[spanish]{babel}
\usepackage[utf8]{inputenc}
\usepackage{amsmath}
\usepackage{setspace}
\usepackage{svg}
\usepackage{hyperref}
\usepackage[backend=biber,style=numeric]{biblatex}
\addbibresource{references.bib}
\hypersetup{
    colorlinks=true,
    linkcolor=blue,      % color of internal links (sections, etc.)
    urlcolor=blue,       % color of external links
    pdftitle={Optimización energética de sistema híbrido con bomba de calor, suelo radiante, fotovoltaica y almacenamiento para vivienda},    % title
    pdfauthor={Luis D. Aranda Sánchez},     % author
    pdfkeywords={palabra1, palabra2, código1, etc.} % list of keywords
}

% Font change to Arial
\usepackage{helvet}
\renewcommand{\familydefault}{\sfdefault}

% Chapter titles in uppercase and larger font
\titleformat{\chapter}[hang]{\large\bfseries}{\thechapter.}{1em}{\MakeUppercase}
\titleformat{\section}[hang]{\bfseries}{\thesection.}{1em}{}
\titleformat{\subsection}[hang]{\bfseries}{\thesubsection.}{1em}{}

% Fancyhdr setup
\setlength{\headheight}{14.30174pt} % Adjust to recommended value, slightly larger for safety
\fancyhf{} % Clear all headers and footers
\fancyhead[LE]{\nouppercase{\leftmark}}
\fancyhead[RO]{Optimización energética para vivienda}
\fancyfoot[LE]{\thepage}
\fancyfoot[RE]{Escuela Técnica Superior de Ingenieros Industriales (UPM)}
\fancyfoot[LO]{Luis D. Aranda Sánchez}
\fancyfoot[RO]{\thepage}
\renewcommand{\headrulewidth}{0.4pt}
\renewcommand{\footrulewidth}{0.4pt}

\fancypagestyle{myfancy}{
    \fancyhf{} % Clear all headers and footers
    \fancyhead[LE]{\nouppercase{\leftmark}}
    \fancyhead[RO]{Optimización energética para vivienda}
    \fancyfoot[LE]{\thepage}
    \fancyfoot[RE]{Escuela Técnica Superior de Ingenieros Industriales (UPM)}
    \fancyfoot[LO]{Luis D. Aranda Sánchez}
    \fancyfoot[RO]{\thepage}
    \renewcommand{\headrulewidth}{0.4pt}
    \renewcommand{\footrulewidth}{0.4pt}
}

\fancypagestyle{simple}{
    \fancyhf{} % Clear all headers and footers
    \renewcommand{\headrulewidth}{0pt}
    \renewcommand{\footrulewidth}{0pt}
}

% Line spacing
\setstretch{1.2}

% Document starts here
\begin{document}

% Portada
\begin{titlepage}
    \centering
    {\scshape\LARGE Universidad Politécnica de Madrid \par}
    \vspace{1cm}
    {\scshape\Large Escuela Técnica Superior de Ingenieros Industriales\par}
    \vspace{1.5cm}
    {\huge\bfseries Optimización energética de sistema híbrido con bomba de calor, suelo radiante, fotovoltaica y almacenamiento para vivienda \par}
    \vspace{1.5cm}
    {\Large\bfseries Trabajo de Fin de Máster\par}
    \vspace{0.5cm}
    {\large Máster Universitario en Ingeniería de la Energía \par}
    \vspace{2cm}
    {\Large Luis D. Aranda Sánchez\par}
    \vfill
    Director: Javier Rodríguez Martín
    \vfill
    {\large Septiembre 6, 2024\par}
\end{titlepage}

% Resumen (máximo de 5 páginas, incluyendo al final Palabras clave)
\clearpage
\pagestyle{simple}
% \newpage
\chapter*{Resumen}
\addcontentsline{toc}{chapter}{Resumen}
\input{capitulos/resumen/main.tex}

% Índice (paginado)
\clearpage
\pagestyle{simple}
% \newpage
\tableofcontents

% Introducción (donde se incluya los antecedentes y justificación)
\clearpage
\pagestyle{myfancy}
\newpage
\chapter{Introducción}
\input{capitulos/introduccion/main.tex}

% Objetivos
\chapter{Objetivos}
\input{capitulos/objetivos/main.tex}

% Metodología
\chapter{Metodología}
\input{capitulos/metodologia/main.tex}

% Resultados y discusión (incluyendo la valoración de impactos y de aspectos de responsabilidad legal, ética y profesional relacionados con el trabajo)
\chapter{Resultados y Discusión}
\input{capitulos/resultados_discusion/main.tex}

% Conclusiones
\chapter{Conclusiones}
\input{capitulos/conclusiones/main.tex}

% Planificación temporal y presupuesto
\chapter{Planificación Temporal y Presupuesto}
\input{capitulos/planificacion_presupuesto/main.tex}

% Bibliografía
\newpage
\addcontentsline{toc}{chapter}{Bibliografía}
\printbibliography

\end{document}


% Bibliografía
\newpage
\addcontentsline{toc}{chapter}{Bibliografía}
\printbibliography

\end{document}


% Índice (paginado)
\clearpage
\pagestyle{simple}
% \newpage
\tableofcontents

% Introducción (donde se incluya los antecedentes y justificación)
\clearpage
\pagestyle{myfancy}
\newpage
\chapter{Introducción}
\documentclass[a4paper,11pt,twoside]{report}
\usepackage[left=25mm,right=25mm,top=25mm,bottom=25mm,includehead,includefoot,headsep=15mm,footskip=15mm]{geometry}
\usepackage{graphicx}
\usepackage{fancyhdr}
\usepackage{titlesec}
\usepackage[spanish]{babel}
\usepackage[utf8]{inputenc}
\usepackage{amsmath}
\usepackage{setspace}
\usepackage{svg}
\usepackage{hyperref}
\usepackage[backend=biber,style=numeric]{biblatex}
\addbibresource{references.bib}
\hypersetup{
    colorlinks=true,
    linkcolor=blue,      % color of internal links (sections, etc.)
    urlcolor=blue,       % color of external links
    pdftitle={Optimización energética de sistema híbrido con bomba de calor, suelo radiante, fotovoltaica y almacenamiento para vivienda},    % title
    pdfauthor={Luis D. Aranda Sánchez},     % author
    pdfkeywords={palabra1, palabra2, código1, etc.} % list of keywords
}

% Font change to Arial
\usepackage{helvet}
\renewcommand{\familydefault}{\sfdefault}

% Chapter titles in uppercase and larger font
\titleformat{\chapter}[hang]{\large\bfseries}{\thechapter.}{1em}{\MakeUppercase}
\titleformat{\section}[hang]{\bfseries}{\thesection.}{1em}{}
\titleformat{\subsection}[hang]{\bfseries}{\thesubsection.}{1em}{}

% Fancyhdr setup
\setlength{\headheight}{14.30174pt} % Adjust to recommended value, slightly larger for safety
\fancyhf{} % Clear all headers and footers
\fancyhead[LE]{\nouppercase{\leftmark}}
\fancyhead[RO]{Optimización energética para vivienda}
\fancyfoot[LE]{\thepage}
\fancyfoot[RE]{Escuela Técnica Superior de Ingenieros Industriales (UPM)}
\fancyfoot[LO]{Luis D. Aranda Sánchez}
\fancyfoot[RO]{\thepage}
\renewcommand{\headrulewidth}{0.4pt}
\renewcommand{\footrulewidth}{0.4pt}

\fancypagestyle{myfancy}{
    \fancyhf{} % Clear all headers and footers
    \fancyhead[LE]{\nouppercase{\leftmark}}
    \fancyhead[RO]{Optimización energética para vivienda}
    \fancyfoot[LE]{\thepage}
    \fancyfoot[RE]{Escuela Técnica Superior de Ingenieros Industriales (UPM)}
    \fancyfoot[LO]{Luis D. Aranda Sánchez}
    \fancyfoot[RO]{\thepage}
    \renewcommand{\headrulewidth}{0.4pt}
    \renewcommand{\footrulewidth}{0.4pt}
}

\fancypagestyle{simple}{
    \fancyhf{} % Clear all headers and footers
    \renewcommand{\headrulewidth}{0pt}
    \renewcommand{\footrulewidth}{0pt}
}

% Line spacing
\setstretch{1.2}

% Document starts here
\begin{document}

% Portada
\begin{titlepage}
    \centering
    {\scshape\LARGE Universidad Politécnica de Madrid \par}
    \vspace{1cm}
    {\scshape\Large Escuela Técnica Superior de Ingenieros Industriales\par}
    \vspace{1.5cm}
    {\huge\bfseries Optimización energética de sistema híbrido con bomba de calor, suelo radiante, fotovoltaica y almacenamiento para vivienda \par}
    \vspace{1.5cm}
    {\Large\bfseries Trabajo de Fin de Máster\par}
    \vspace{0.5cm}
    {\large Máster Universitario en Ingeniería de la Energía \par}
    \vspace{2cm}
    {\Large Luis D. Aranda Sánchez\par}
    \vfill
    Director: Javier Rodríguez Martín
    \vfill
    {\large Septiembre 6, 2024\par}
\end{titlepage}

% Resumen (máximo de 5 páginas, incluyendo al final Palabras clave)
\clearpage
\pagestyle{simple}
% \newpage
\chapter*{Resumen}
\addcontentsline{toc}{chapter}{Resumen}
\documentclass[a4paper,11pt,twoside]{report}
\usepackage[left=25mm,right=25mm,top=25mm,bottom=25mm,includehead,includefoot,headsep=15mm,footskip=15mm]{geometry}
\usepackage{graphicx}
\usepackage{fancyhdr}
\usepackage{titlesec}
\usepackage[spanish]{babel}
\usepackage[utf8]{inputenc}
\usepackage{amsmath}
\usepackage{setspace}
\usepackage{svg}
\usepackage{hyperref}
\usepackage[backend=biber,style=numeric]{biblatex}
\addbibresource{references.bib}
\hypersetup{
    colorlinks=true,
    linkcolor=blue,      % color of internal links (sections, etc.)
    urlcolor=blue,       % color of external links
    pdftitle={Optimización energética de sistema híbrido con bomba de calor, suelo radiante, fotovoltaica y almacenamiento para vivienda},    % title
    pdfauthor={Luis D. Aranda Sánchez},     % author
    pdfkeywords={palabra1, palabra2, código1, etc.} % list of keywords
}

% Font change to Arial
\usepackage{helvet}
\renewcommand{\familydefault}{\sfdefault}

% Chapter titles in uppercase and larger font
\titleformat{\chapter}[hang]{\large\bfseries}{\thechapter.}{1em}{\MakeUppercase}
\titleformat{\section}[hang]{\bfseries}{\thesection.}{1em}{}
\titleformat{\subsection}[hang]{\bfseries}{\thesubsection.}{1em}{}

% Fancyhdr setup
\setlength{\headheight}{14.30174pt} % Adjust to recommended value, slightly larger for safety
\fancyhf{} % Clear all headers and footers
\fancyhead[LE]{\nouppercase{\leftmark}}
\fancyhead[RO]{Optimización energética para vivienda}
\fancyfoot[LE]{\thepage}
\fancyfoot[RE]{Escuela Técnica Superior de Ingenieros Industriales (UPM)}
\fancyfoot[LO]{Luis D. Aranda Sánchez}
\fancyfoot[RO]{\thepage}
\renewcommand{\headrulewidth}{0.4pt}
\renewcommand{\footrulewidth}{0.4pt}

\fancypagestyle{myfancy}{
    \fancyhf{} % Clear all headers and footers
    \fancyhead[LE]{\nouppercase{\leftmark}}
    \fancyhead[RO]{Optimización energética para vivienda}
    \fancyfoot[LE]{\thepage}
    \fancyfoot[RE]{Escuela Técnica Superior de Ingenieros Industriales (UPM)}
    \fancyfoot[LO]{Luis D. Aranda Sánchez}
    \fancyfoot[RO]{\thepage}
    \renewcommand{\headrulewidth}{0.4pt}
    \renewcommand{\footrulewidth}{0.4pt}
}

\fancypagestyle{simple}{
    \fancyhf{} % Clear all headers and footers
    \renewcommand{\headrulewidth}{0pt}
    \renewcommand{\footrulewidth}{0pt}
}

% Line spacing
\setstretch{1.2}

% Document starts here
\begin{document}

% Portada
\begin{titlepage}
    \centering
    {\scshape\LARGE Universidad Politécnica de Madrid \par}
    \vspace{1cm}
    {\scshape\Large Escuela Técnica Superior de Ingenieros Industriales\par}
    \vspace{1.5cm}
    {\huge\bfseries Optimización energética de sistema híbrido con bomba de calor, suelo radiante, fotovoltaica y almacenamiento para vivienda \par}
    \vspace{1.5cm}
    {\Large\bfseries Trabajo de Fin de Máster\par}
    \vspace{0.5cm}
    {\large Máster Universitario en Ingeniería de la Energía \par}
    \vspace{2cm}
    {\Large Luis D. Aranda Sánchez\par}
    \vfill
    Director: Javier Rodríguez Martín
    \vfill
    {\large Septiembre 6, 2024\par}
\end{titlepage}

% Resumen (máximo de 5 páginas, incluyendo al final Palabras clave)
\clearpage
\pagestyle{simple}
% \newpage
\chapter*{Resumen}
\addcontentsline{toc}{chapter}{Resumen}
\input{capitulos/resumen/main.tex}

% Índice (paginado)
\clearpage
\pagestyle{simple}
% \newpage
\tableofcontents

% Introducción (donde se incluya los antecedentes y justificación)
\clearpage
\pagestyle{myfancy}
\newpage
\chapter{Introducción}
\input{capitulos/introduccion/main.tex}

% Objetivos
\chapter{Objetivos}
\input{capitulos/objetivos/main.tex}

% Metodología
\chapter{Metodología}
\input{capitulos/metodologia/main.tex}

% Resultados y discusión (incluyendo la valoración de impactos y de aspectos de responsabilidad legal, ética y profesional relacionados con el trabajo)
\chapter{Resultados y Discusión}
\input{capitulos/resultados_discusion/main.tex}

% Conclusiones
\chapter{Conclusiones}
\input{capitulos/conclusiones/main.tex}

% Planificación temporal y presupuesto
\chapter{Planificación Temporal y Presupuesto}
\input{capitulos/planificacion_presupuesto/main.tex}

% Bibliografía
\newpage
\addcontentsline{toc}{chapter}{Bibliografía}
\printbibliography

\end{document}


% Índice (paginado)
\clearpage
\pagestyle{simple}
% \newpage
\tableofcontents

% Introducción (donde se incluya los antecedentes y justificación)
\clearpage
\pagestyle{myfancy}
\newpage
\chapter{Introducción}
\documentclass[a4paper,11pt,twoside]{report}
\usepackage[left=25mm,right=25mm,top=25mm,bottom=25mm,includehead,includefoot,headsep=15mm,footskip=15mm]{geometry}
\usepackage{graphicx}
\usepackage{fancyhdr}
\usepackage{titlesec}
\usepackage[spanish]{babel}
\usepackage[utf8]{inputenc}
\usepackage{amsmath}
\usepackage{setspace}
\usepackage{svg}
\usepackage{hyperref}
\usepackage[backend=biber,style=numeric]{biblatex}
\addbibresource{references.bib}
\hypersetup{
    colorlinks=true,
    linkcolor=blue,      % color of internal links (sections, etc.)
    urlcolor=blue,       % color of external links
    pdftitle={Optimización energética de sistema híbrido con bomba de calor, suelo radiante, fotovoltaica y almacenamiento para vivienda},    % title
    pdfauthor={Luis D. Aranda Sánchez},     % author
    pdfkeywords={palabra1, palabra2, código1, etc.} % list of keywords
}

% Font change to Arial
\usepackage{helvet}
\renewcommand{\familydefault}{\sfdefault}

% Chapter titles in uppercase and larger font
\titleformat{\chapter}[hang]{\large\bfseries}{\thechapter.}{1em}{\MakeUppercase}
\titleformat{\section}[hang]{\bfseries}{\thesection.}{1em}{}
\titleformat{\subsection}[hang]{\bfseries}{\thesubsection.}{1em}{}

% Fancyhdr setup
\setlength{\headheight}{14.30174pt} % Adjust to recommended value, slightly larger for safety
\fancyhf{} % Clear all headers and footers
\fancyhead[LE]{\nouppercase{\leftmark}}
\fancyhead[RO]{Optimización energética para vivienda}
\fancyfoot[LE]{\thepage}
\fancyfoot[RE]{Escuela Técnica Superior de Ingenieros Industriales (UPM)}
\fancyfoot[LO]{Luis D. Aranda Sánchez}
\fancyfoot[RO]{\thepage}
\renewcommand{\headrulewidth}{0.4pt}
\renewcommand{\footrulewidth}{0.4pt}

\fancypagestyle{myfancy}{
    \fancyhf{} % Clear all headers and footers
    \fancyhead[LE]{\nouppercase{\leftmark}}
    \fancyhead[RO]{Optimización energética para vivienda}
    \fancyfoot[LE]{\thepage}
    \fancyfoot[RE]{Escuela Técnica Superior de Ingenieros Industriales (UPM)}
    \fancyfoot[LO]{Luis D. Aranda Sánchez}
    \fancyfoot[RO]{\thepage}
    \renewcommand{\headrulewidth}{0.4pt}
    \renewcommand{\footrulewidth}{0.4pt}
}

\fancypagestyle{simple}{
    \fancyhf{} % Clear all headers and footers
    \renewcommand{\headrulewidth}{0pt}
    \renewcommand{\footrulewidth}{0pt}
}

% Line spacing
\setstretch{1.2}

% Document starts here
\begin{document}

% Portada
\begin{titlepage}
    \centering
    {\scshape\LARGE Universidad Politécnica de Madrid \par}
    \vspace{1cm}
    {\scshape\Large Escuela Técnica Superior de Ingenieros Industriales\par}
    \vspace{1.5cm}
    {\huge\bfseries Optimización energética de sistema híbrido con bomba de calor, suelo radiante, fotovoltaica y almacenamiento para vivienda \par}
    \vspace{1.5cm}
    {\Large\bfseries Trabajo de Fin de Máster\par}
    \vspace{0.5cm}
    {\large Máster Universitario en Ingeniería de la Energía \par}
    \vspace{2cm}
    {\Large Luis D. Aranda Sánchez\par}
    \vfill
    Director: Javier Rodríguez Martín
    \vfill
    {\large Septiembre 6, 2024\par}
\end{titlepage}

% Resumen (máximo de 5 páginas, incluyendo al final Palabras clave)
\clearpage
\pagestyle{simple}
% \newpage
\chapter*{Resumen}
\addcontentsline{toc}{chapter}{Resumen}
\input{capitulos/resumen/main.tex}

% Índice (paginado)
\clearpage
\pagestyle{simple}
% \newpage
\tableofcontents

% Introducción (donde se incluya los antecedentes y justificación)
\clearpage
\pagestyle{myfancy}
\newpage
\chapter{Introducción}
\input{capitulos/introduccion/main.tex}

% Objetivos
\chapter{Objetivos}
\input{capitulos/objetivos/main.tex}

% Metodología
\chapter{Metodología}
\input{capitulos/metodologia/main.tex}

% Resultados y discusión (incluyendo la valoración de impactos y de aspectos de responsabilidad legal, ética y profesional relacionados con el trabajo)
\chapter{Resultados y Discusión}
\input{capitulos/resultados_discusion/main.tex}

% Conclusiones
\chapter{Conclusiones}
\input{capitulos/conclusiones/main.tex}

% Planificación temporal y presupuesto
\chapter{Planificación Temporal y Presupuesto}
\input{capitulos/planificacion_presupuesto/main.tex}

% Bibliografía
\newpage
\addcontentsline{toc}{chapter}{Bibliografía}
\printbibliography

\end{document}


% Objetivos
\chapter{Objetivos}
\documentclass[a4paper,11pt,twoside]{report}
\usepackage[left=25mm,right=25mm,top=25mm,bottom=25mm,includehead,includefoot,headsep=15mm,footskip=15mm]{geometry}
\usepackage{graphicx}
\usepackage{fancyhdr}
\usepackage{titlesec}
\usepackage[spanish]{babel}
\usepackage[utf8]{inputenc}
\usepackage{amsmath}
\usepackage{setspace}
\usepackage{svg}
\usepackage{hyperref}
\usepackage[backend=biber,style=numeric]{biblatex}
\addbibresource{references.bib}
\hypersetup{
    colorlinks=true,
    linkcolor=blue,      % color of internal links (sections, etc.)
    urlcolor=blue,       % color of external links
    pdftitle={Optimización energética de sistema híbrido con bomba de calor, suelo radiante, fotovoltaica y almacenamiento para vivienda},    % title
    pdfauthor={Luis D. Aranda Sánchez},     % author
    pdfkeywords={palabra1, palabra2, código1, etc.} % list of keywords
}

% Font change to Arial
\usepackage{helvet}
\renewcommand{\familydefault}{\sfdefault}

% Chapter titles in uppercase and larger font
\titleformat{\chapter}[hang]{\large\bfseries}{\thechapter.}{1em}{\MakeUppercase}
\titleformat{\section}[hang]{\bfseries}{\thesection.}{1em}{}
\titleformat{\subsection}[hang]{\bfseries}{\thesubsection.}{1em}{}

% Fancyhdr setup
\setlength{\headheight}{14.30174pt} % Adjust to recommended value, slightly larger for safety
\fancyhf{} % Clear all headers and footers
\fancyhead[LE]{\nouppercase{\leftmark}}
\fancyhead[RO]{Optimización energética para vivienda}
\fancyfoot[LE]{\thepage}
\fancyfoot[RE]{Escuela Técnica Superior de Ingenieros Industriales (UPM)}
\fancyfoot[LO]{Luis D. Aranda Sánchez}
\fancyfoot[RO]{\thepage}
\renewcommand{\headrulewidth}{0.4pt}
\renewcommand{\footrulewidth}{0.4pt}

\fancypagestyle{myfancy}{
    \fancyhf{} % Clear all headers and footers
    \fancyhead[LE]{\nouppercase{\leftmark}}
    \fancyhead[RO]{Optimización energética para vivienda}
    \fancyfoot[LE]{\thepage}
    \fancyfoot[RE]{Escuela Técnica Superior de Ingenieros Industriales (UPM)}
    \fancyfoot[LO]{Luis D. Aranda Sánchez}
    \fancyfoot[RO]{\thepage}
    \renewcommand{\headrulewidth}{0.4pt}
    \renewcommand{\footrulewidth}{0.4pt}
}

\fancypagestyle{simple}{
    \fancyhf{} % Clear all headers and footers
    \renewcommand{\headrulewidth}{0pt}
    \renewcommand{\footrulewidth}{0pt}
}

% Line spacing
\setstretch{1.2}

% Document starts here
\begin{document}

% Portada
\begin{titlepage}
    \centering
    {\scshape\LARGE Universidad Politécnica de Madrid \par}
    \vspace{1cm}
    {\scshape\Large Escuela Técnica Superior de Ingenieros Industriales\par}
    \vspace{1.5cm}
    {\huge\bfseries Optimización energética de sistema híbrido con bomba de calor, suelo radiante, fotovoltaica y almacenamiento para vivienda \par}
    \vspace{1.5cm}
    {\Large\bfseries Trabajo de Fin de Máster\par}
    \vspace{0.5cm}
    {\large Máster Universitario en Ingeniería de la Energía \par}
    \vspace{2cm}
    {\Large Luis D. Aranda Sánchez\par}
    \vfill
    Director: Javier Rodríguez Martín
    \vfill
    {\large Septiembre 6, 2024\par}
\end{titlepage}

% Resumen (máximo de 5 páginas, incluyendo al final Palabras clave)
\clearpage
\pagestyle{simple}
% \newpage
\chapter*{Resumen}
\addcontentsline{toc}{chapter}{Resumen}
\input{capitulos/resumen/main.tex}

% Índice (paginado)
\clearpage
\pagestyle{simple}
% \newpage
\tableofcontents

% Introducción (donde se incluya los antecedentes y justificación)
\clearpage
\pagestyle{myfancy}
\newpage
\chapter{Introducción}
\input{capitulos/introduccion/main.tex}

% Objetivos
\chapter{Objetivos}
\input{capitulos/objetivos/main.tex}

% Metodología
\chapter{Metodología}
\input{capitulos/metodologia/main.tex}

% Resultados y discusión (incluyendo la valoración de impactos y de aspectos de responsabilidad legal, ética y profesional relacionados con el trabajo)
\chapter{Resultados y Discusión}
\input{capitulos/resultados_discusion/main.tex}

% Conclusiones
\chapter{Conclusiones}
\input{capitulos/conclusiones/main.tex}

% Planificación temporal y presupuesto
\chapter{Planificación Temporal y Presupuesto}
\input{capitulos/planificacion_presupuesto/main.tex}

% Bibliografía
\newpage
\addcontentsline{toc}{chapter}{Bibliografía}
\printbibliography

\end{document}


% Metodología
\chapter{Metodología}
\documentclass[a4paper,11pt,twoside]{report}
\usepackage[left=25mm,right=25mm,top=25mm,bottom=25mm,includehead,includefoot,headsep=15mm,footskip=15mm]{geometry}
\usepackage{graphicx}
\usepackage{fancyhdr}
\usepackage{titlesec}
\usepackage[spanish]{babel}
\usepackage[utf8]{inputenc}
\usepackage{amsmath}
\usepackage{setspace}
\usepackage{svg}
\usepackage{hyperref}
\usepackage[backend=biber,style=numeric]{biblatex}
\addbibresource{references.bib}
\hypersetup{
    colorlinks=true,
    linkcolor=blue,      % color of internal links (sections, etc.)
    urlcolor=blue,       % color of external links
    pdftitle={Optimización energética de sistema híbrido con bomba de calor, suelo radiante, fotovoltaica y almacenamiento para vivienda},    % title
    pdfauthor={Luis D. Aranda Sánchez},     % author
    pdfkeywords={palabra1, palabra2, código1, etc.} % list of keywords
}

% Font change to Arial
\usepackage{helvet}
\renewcommand{\familydefault}{\sfdefault}

% Chapter titles in uppercase and larger font
\titleformat{\chapter}[hang]{\large\bfseries}{\thechapter.}{1em}{\MakeUppercase}
\titleformat{\section}[hang]{\bfseries}{\thesection.}{1em}{}
\titleformat{\subsection}[hang]{\bfseries}{\thesubsection.}{1em}{}

% Fancyhdr setup
\setlength{\headheight}{14.30174pt} % Adjust to recommended value, slightly larger for safety
\fancyhf{} % Clear all headers and footers
\fancyhead[LE]{\nouppercase{\leftmark}}
\fancyhead[RO]{Optimización energética para vivienda}
\fancyfoot[LE]{\thepage}
\fancyfoot[RE]{Escuela Técnica Superior de Ingenieros Industriales (UPM)}
\fancyfoot[LO]{Luis D. Aranda Sánchez}
\fancyfoot[RO]{\thepage}
\renewcommand{\headrulewidth}{0.4pt}
\renewcommand{\footrulewidth}{0.4pt}

\fancypagestyle{myfancy}{
    \fancyhf{} % Clear all headers and footers
    \fancyhead[LE]{\nouppercase{\leftmark}}
    \fancyhead[RO]{Optimización energética para vivienda}
    \fancyfoot[LE]{\thepage}
    \fancyfoot[RE]{Escuela Técnica Superior de Ingenieros Industriales (UPM)}
    \fancyfoot[LO]{Luis D. Aranda Sánchez}
    \fancyfoot[RO]{\thepage}
    \renewcommand{\headrulewidth}{0.4pt}
    \renewcommand{\footrulewidth}{0.4pt}
}

\fancypagestyle{simple}{
    \fancyhf{} % Clear all headers and footers
    \renewcommand{\headrulewidth}{0pt}
    \renewcommand{\footrulewidth}{0pt}
}

% Line spacing
\setstretch{1.2}

% Document starts here
\begin{document}

% Portada
\begin{titlepage}
    \centering
    {\scshape\LARGE Universidad Politécnica de Madrid \par}
    \vspace{1cm}
    {\scshape\Large Escuela Técnica Superior de Ingenieros Industriales\par}
    \vspace{1.5cm}
    {\huge\bfseries Optimización energética de sistema híbrido con bomba de calor, suelo radiante, fotovoltaica y almacenamiento para vivienda \par}
    \vspace{1.5cm}
    {\Large\bfseries Trabajo de Fin de Máster\par}
    \vspace{0.5cm}
    {\large Máster Universitario en Ingeniería de la Energía \par}
    \vspace{2cm}
    {\Large Luis D. Aranda Sánchez\par}
    \vfill
    Director: Javier Rodríguez Martín
    \vfill
    {\large Septiembre 6, 2024\par}
\end{titlepage}

% Resumen (máximo de 5 páginas, incluyendo al final Palabras clave)
\clearpage
\pagestyle{simple}
% \newpage
\chapter*{Resumen}
\addcontentsline{toc}{chapter}{Resumen}
\input{capitulos/resumen/main.tex}

% Índice (paginado)
\clearpage
\pagestyle{simple}
% \newpage
\tableofcontents

% Introducción (donde se incluya los antecedentes y justificación)
\clearpage
\pagestyle{myfancy}
\newpage
\chapter{Introducción}
\input{capitulos/introduccion/main.tex}

% Objetivos
\chapter{Objetivos}
\input{capitulos/objetivos/main.tex}

% Metodología
\chapter{Metodología}
\input{capitulos/metodologia/main.tex}

% Resultados y discusión (incluyendo la valoración de impactos y de aspectos de responsabilidad legal, ética y profesional relacionados con el trabajo)
\chapter{Resultados y Discusión}
\input{capitulos/resultados_discusion/main.tex}

% Conclusiones
\chapter{Conclusiones}
\input{capitulos/conclusiones/main.tex}

% Planificación temporal y presupuesto
\chapter{Planificación Temporal y Presupuesto}
\input{capitulos/planificacion_presupuesto/main.tex}

% Bibliografía
\newpage
\addcontentsline{toc}{chapter}{Bibliografía}
\printbibliography

\end{document}


% Resultados y discusión (incluyendo la valoración de impactos y de aspectos de responsabilidad legal, ética y profesional relacionados con el trabajo)
\chapter{Resultados y Discusión}
\documentclass[a4paper,11pt,twoside]{report}
\usepackage[left=25mm,right=25mm,top=25mm,bottom=25mm,includehead,includefoot,headsep=15mm,footskip=15mm]{geometry}
\usepackage{graphicx}
\usepackage{fancyhdr}
\usepackage{titlesec}
\usepackage[spanish]{babel}
\usepackage[utf8]{inputenc}
\usepackage{amsmath}
\usepackage{setspace}
\usepackage{svg}
\usepackage{hyperref}
\usepackage[backend=biber,style=numeric]{biblatex}
\addbibresource{references.bib}
\hypersetup{
    colorlinks=true,
    linkcolor=blue,      % color of internal links (sections, etc.)
    urlcolor=blue,       % color of external links
    pdftitle={Optimización energética de sistema híbrido con bomba de calor, suelo radiante, fotovoltaica y almacenamiento para vivienda},    % title
    pdfauthor={Luis D. Aranda Sánchez},     % author
    pdfkeywords={palabra1, palabra2, código1, etc.} % list of keywords
}

% Font change to Arial
\usepackage{helvet}
\renewcommand{\familydefault}{\sfdefault}

% Chapter titles in uppercase and larger font
\titleformat{\chapter}[hang]{\large\bfseries}{\thechapter.}{1em}{\MakeUppercase}
\titleformat{\section}[hang]{\bfseries}{\thesection.}{1em}{}
\titleformat{\subsection}[hang]{\bfseries}{\thesubsection.}{1em}{}

% Fancyhdr setup
\setlength{\headheight}{14.30174pt} % Adjust to recommended value, slightly larger for safety
\fancyhf{} % Clear all headers and footers
\fancyhead[LE]{\nouppercase{\leftmark}}
\fancyhead[RO]{Optimización energética para vivienda}
\fancyfoot[LE]{\thepage}
\fancyfoot[RE]{Escuela Técnica Superior de Ingenieros Industriales (UPM)}
\fancyfoot[LO]{Luis D. Aranda Sánchez}
\fancyfoot[RO]{\thepage}
\renewcommand{\headrulewidth}{0.4pt}
\renewcommand{\footrulewidth}{0.4pt}

\fancypagestyle{myfancy}{
    \fancyhf{} % Clear all headers and footers
    \fancyhead[LE]{\nouppercase{\leftmark}}
    \fancyhead[RO]{Optimización energética para vivienda}
    \fancyfoot[LE]{\thepage}
    \fancyfoot[RE]{Escuela Técnica Superior de Ingenieros Industriales (UPM)}
    \fancyfoot[LO]{Luis D. Aranda Sánchez}
    \fancyfoot[RO]{\thepage}
    \renewcommand{\headrulewidth}{0.4pt}
    \renewcommand{\footrulewidth}{0.4pt}
}

\fancypagestyle{simple}{
    \fancyhf{} % Clear all headers and footers
    \renewcommand{\headrulewidth}{0pt}
    \renewcommand{\footrulewidth}{0pt}
}

% Line spacing
\setstretch{1.2}

% Document starts here
\begin{document}

% Portada
\begin{titlepage}
    \centering
    {\scshape\LARGE Universidad Politécnica de Madrid \par}
    \vspace{1cm}
    {\scshape\Large Escuela Técnica Superior de Ingenieros Industriales\par}
    \vspace{1.5cm}
    {\huge\bfseries Optimización energética de sistema híbrido con bomba de calor, suelo radiante, fotovoltaica y almacenamiento para vivienda \par}
    \vspace{1.5cm}
    {\Large\bfseries Trabajo de Fin de Máster\par}
    \vspace{0.5cm}
    {\large Máster Universitario en Ingeniería de la Energía \par}
    \vspace{2cm}
    {\Large Luis D. Aranda Sánchez\par}
    \vfill
    Director: Javier Rodríguez Martín
    \vfill
    {\large Septiembre 6, 2024\par}
\end{titlepage}

% Resumen (máximo de 5 páginas, incluyendo al final Palabras clave)
\clearpage
\pagestyle{simple}
% \newpage
\chapter*{Resumen}
\addcontentsline{toc}{chapter}{Resumen}
\input{capitulos/resumen/main.tex}

% Índice (paginado)
\clearpage
\pagestyle{simple}
% \newpage
\tableofcontents

% Introducción (donde se incluya los antecedentes y justificación)
\clearpage
\pagestyle{myfancy}
\newpage
\chapter{Introducción}
\input{capitulos/introduccion/main.tex}

% Objetivos
\chapter{Objetivos}
\input{capitulos/objetivos/main.tex}

% Metodología
\chapter{Metodología}
\input{capitulos/metodologia/main.tex}

% Resultados y discusión (incluyendo la valoración de impactos y de aspectos de responsabilidad legal, ética y profesional relacionados con el trabajo)
\chapter{Resultados y Discusión}
\input{capitulos/resultados_discusion/main.tex}

% Conclusiones
\chapter{Conclusiones}
\input{capitulos/conclusiones/main.tex}

% Planificación temporal y presupuesto
\chapter{Planificación Temporal y Presupuesto}
\input{capitulos/planificacion_presupuesto/main.tex}

% Bibliografía
\newpage
\addcontentsline{toc}{chapter}{Bibliografía}
\printbibliography

\end{document}


% Conclusiones
\chapter{Conclusiones}
\documentclass[a4paper,11pt,twoside]{report}
\usepackage[left=25mm,right=25mm,top=25mm,bottom=25mm,includehead,includefoot,headsep=15mm,footskip=15mm]{geometry}
\usepackage{graphicx}
\usepackage{fancyhdr}
\usepackage{titlesec}
\usepackage[spanish]{babel}
\usepackage[utf8]{inputenc}
\usepackage{amsmath}
\usepackage{setspace}
\usepackage{svg}
\usepackage{hyperref}
\usepackage[backend=biber,style=numeric]{biblatex}
\addbibresource{references.bib}
\hypersetup{
    colorlinks=true,
    linkcolor=blue,      % color of internal links (sections, etc.)
    urlcolor=blue,       % color of external links
    pdftitle={Optimización energética de sistema híbrido con bomba de calor, suelo radiante, fotovoltaica y almacenamiento para vivienda},    % title
    pdfauthor={Luis D. Aranda Sánchez},     % author
    pdfkeywords={palabra1, palabra2, código1, etc.} % list of keywords
}

% Font change to Arial
\usepackage{helvet}
\renewcommand{\familydefault}{\sfdefault}

% Chapter titles in uppercase and larger font
\titleformat{\chapter}[hang]{\large\bfseries}{\thechapter.}{1em}{\MakeUppercase}
\titleformat{\section}[hang]{\bfseries}{\thesection.}{1em}{}
\titleformat{\subsection}[hang]{\bfseries}{\thesubsection.}{1em}{}

% Fancyhdr setup
\setlength{\headheight}{14.30174pt} % Adjust to recommended value, slightly larger for safety
\fancyhf{} % Clear all headers and footers
\fancyhead[LE]{\nouppercase{\leftmark}}
\fancyhead[RO]{Optimización energética para vivienda}
\fancyfoot[LE]{\thepage}
\fancyfoot[RE]{Escuela Técnica Superior de Ingenieros Industriales (UPM)}
\fancyfoot[LO]{Luis D. Aranda Sánchez}
\fancyfoot[RO]{\thepage}
\renewcommand{\headrulewidth}{0.4pt}
\renewcommand{\footrulewidth}{0.4pt}

\fancypagestyle{myfancy}{
    \fancyhf{} % Clear all headers and footers
    \fancyhead[LE]{\nouppercase{\leftmark}}
    \fancyhead[RO]{Optimización energética para vivienda}
    \fancyfoot[LE]{\thepage}
    \fancyfoot[RE]{Escuela Técnica Superior de Ingenieros Industriales (UPM)}
    \fancyfoot[LO]{Luis D. Aranda Sánchez}
    \fancyfoot[RO]{\thepage}
    \renewcommand{\headrulewidth}{0.4pt}
    \renewcommand{\footrulewidth}{0.4pt}
}

\fancypagestyle{simple}{
    \fancyhf{} % Clear all headers and footers
    \renewcommand{\headrulewidth}{0pt}
    \renewcommand{\footrulewidth}{0pt}
}

% Line spacing
\setstretch{1.2}

% Document starts here
\begin{document}

% Portada
\begin{titlepage}
    \centering
    {\scshape\LARGE Universidad Politécnica de Madrid \par}
    \vspace{1cm}
    {\scshape\Large Escuela Técnica Superior de Ingenieros Industriales\par}
    \vspace{1.5cm}
    {\huge\bfseries Optimización energética de sistema híbrido con bomba de calor, suelo radiante, fotovoltaica y almacenamiento para vivienda \par}
    \vspace{1.5cm}
    {\Large\bfseries Trabajo de Fin de Máster\par}
    \vspace{0.5cm}
    {\large Máster Universitario en Ingeniería de la Energía \par}
    \vspace{2cm}
    {\Large Luis D. Aranda Sánchez\par}
    \vfill
    Director: Javier Rodríguez Martín
    \vfill
    {\large Septiembre 6, 2024\par}
\end{titlepage}

% Resumen (máximo de 5 páginas, incluyendo al final Palabras clave)
\clearpage
\pagestyle{simple}
% \newpage
\chapter*{Resumen}
\addcontentsline{toc}{chapter}{Resumen}
\input{capitulos/resumen/main.tex}

% Índice (paginado)
\clearpage
\pagestyle{simple}
% \newpage
\tableofcontents

% Introducción (donde se incluya los antecedentes y justificación)
\clearpage
\pagestyle{myfancy}
\newpage
\chapter{Introducción}
\input{capitulos/introduccion/main.tex}

% Objetivos
\chapter{Objetivos}
\input{capitulos/objetivos/main.tex}

% Metodología
\chapter{Metodología}
\input{capitulos/metodologia/main.tex}

% Resultados y discusión (incluyendo la valoración de impactos y de aspectos de responsabilidad legal, ética y profesional relacionados con el trabajo)
\chapter{Resultados y Discusión}
\input{capitulos/resultados_discusion/main.tex}

% Conclusiones
\chapter{Conclusiones}
\input{capitulos/conclusiones/main.tex}

% Planificación temporal y presupuesto
\chapter{Planificación Temporal y Presupuesto}
\input{capitulos/planificacion_presupuesto/main.tex}

% Bibliografía
\newpage
\addcontentsline{toc}{chapter}{Bibliografía}
\printbibliography

\end{document}


% Planificación temporal y presupuesto
\chapter{Planificación Temporal y Presupuesto}
\documentclass[a4paper,11pt,twoside]{report}
\usepackage[left=25mm,right=25mm,top=25mm,bottom=25mm,includehead,includefoot,headsep=15mm,footskip=15mm]{geometry}
\usepackage{graphicx}
\usepackage{fancyhdr}
\usepackage{titlesec}
\usepackage[spanish]{babel}
\usepackage[utf8]{inputenc}
\usepackage{amsmath}
\usepackage{setspace}
\usepackage{svg}
\usepackage{hyperref}
\usepackage[backend=biber,style=numeric]{biblatex}
\addbibresource{references.bib}
\hypersetup{
    colorlinks=true,
    linkcolor=blue,      % color of internal links (sections, etc.)
    urlcolor=blue,       % color of external links
    pdftitle={Optimización energética de sistema híbrido con bomba de calor, suelo radiante, fotovoltaica y almacenamiento para vivienda},    % title
    pdfauthor={Luis D. Aranda Sánchez},     % author
    pdfkeywords={palabra1, palabra2, código1, etc.} % list of keywords
}

% Font change to Arial
\usepackage{helvet}
\renewcommand{\familydefault}{\sfdefault}

% Chapter titles in uppercase and larger font
\titleformat{\chapter}[hang]{\large\bfseries}{\thechapter.}{1em}{\MakeUppercase}
\titleformat{\section}[hang]{\bfseries}{\thesection.}{1em}{}
\titleformat{\subsection}[hang]{\bfseries}{\thesubsection.}{1em}{}

% Fancyhdr setup
\setlength{\headheight}{14.30174pt} % Adjust to recommended value, slightly larger for safety
\fancyhf{} % Clear all headers and footers
\fancyhead[LE]{\nouppercase{\leftmark}}
\fancyhead[RO]{Optimización energética para vivienda}
\fancyfoot[LE]{\thepage}
\fancyfoot[RE]{Escuela Técnica Superior de Ingenieros Industriales (UPM)}
\fancyfoot[LO]{Luis D. Aranda Sánchez}
\fancyfoot[RO]{\thepage}
\renewcommand{\headrulewidth}{0.4pt}
\renewcommand{\footrulewidth}{0.4pt}

\fancypagestyle{myfancy}{
    \fancyhf{} % Clear all headers and footers
    \fancyhead[LE]{\nouppercase{\leftmark}}
    \fancyhead[RO]{Optimización energética para vivienda}
    \fancyfoot[LE]{\thepage}
    \fancyfoot[RE]{Escuela Técnica Superior de Ingenieros Industriales (UPM)}
    \fancyfoot[LO]{Luis D. Aranda Sánchez}
    \fancyfoot[RO]{\thepage}
    \renewcommand{\headrulewidth}{0.4pt}
    \renewcommand{\footrulewidth}{0.4pt}
}

\fancypagestyle{simple}{
    \fancyhf{} % Clear all headers and footers
    \renewcommand{\headrulewidth}{0pt}
    \renewcommand{\footrulewidth}{0pt}
}

% Line spacing
\setstretch{1.2}

% Document starts here
\begin{document}

% Portada
\begin{titlepage}
    \centering
    {\scshape\LARGE Universidad Politécnica de Madrid \par}
    \vspace{1cm}
    {\scshape\Large Escuela Técnica Superior de Ingenieros Industriales\par}
    \vspace{1.5cm}
    {\huge\bfseries Optimización energética de sistema híbrido con bomba de calor, suelo radiante, fotovoltaica y almacenamiento para vivienda \par}
    \vspace{1.5cm}
    {\Large\bfseries Trabajo de Fin de Máster\par}
    \vspace{0.5cm}
    {\large Máster Universitario en Ingeniería de la Energía \par}
    \vspace{2cm}
    {\Large Luis D. Aranda Sánchez\par}
    \vfill
    Director: Javier Rodríguez Martín
    \vfill
    {\large Septiembre 6, 2024\par}
\end{titlepage}

% Resumen (máximo de 5 páginas, incluyendo al final Palabras clave)
\clearpage
\pagestyle{simple}
% \newpage
\chapter*{Resumen}
\addcontentsline{toc}{chapter}{Resumen}
\input{capitulos/resumen/main.tex}

% Índice (paginado)
\clearpage
\pagestyle{simple}
% \newpage
\tableofcontents

% Introducción (donde se incluya los antecedentes y justificación)
\clearpage
\pagestyle{myfancy}
\newpage
\chapter{Introducción}
\input{capitulos/introduccion/main.tex}

% Objetivos
\chapter{Objetivos}
\input{capitulos/objetivos/main.tex}

% Metodología
\chapter{Metodología}
\input{capitulos/metodologia/main.tex}

% Resultados y discusión (incluyendo la valoración de impactos y de aspectos de responsabilidad legal, ética y profesional relacionados con el trabajo)
\chapter{Resultados y Discusión}
\input{capitulos/resultados_discusion/main.tex}

% Conclusiones
\chapter{Conclusiones}
\input{capitulos/conclusiones/main.tex}

% Planificación temporal y presupuesto
\chapter{Planificación Temporal y Presupuesto}
\input{capitulos/planificacion_presupuesto/main.tex}

% Bibliografía
\newpage
\addcontentsline{toc}{chapter}{Bibliografía}
\printbibliography

\end{document}


% Bibliografía
\newpage
\addcontentsline{toc}{chapter}{Bibliografía}
\printbibliography

\end{document}


% Objetivos
\chapter{Objetivos}
\documentclass[a4paper,11pt,twoside]{report}
\usepackage[left=25mm,right=25mm,top=25mm,bottom=25mm,includehead,includefoot,headsep=15mm,footskip=15mm]{geometry}
\usepackage{graphicx}
\usepackage{fancyhdr}
\usepackage{titlesec}
\usepackage[spanish]{babel}
\usepackage[utf8]{inputenc}
\usepackage{amsmath}
\usepackage{setspace}
\usepackage{svg}
\usepackage{hyperref}
\usepackage[backend=biber,style=numeric]{biblatex}
\addbibresource{references.bib}
\hypersetup{
    colorlinks=true,
    linkcolor=blue,      % color of internal links (sections, etc.)
    urlcolor=blue,       % color of external links
    pdftitle={Optimización energética de sistema híbrido con bomba de calor, suelo radiante, fotovoltaica y almacenamiento para vivienda},    % title
    pdfauthor={Luis D. Aranda Sánchez},     % author
    pdfkeywords={palabra1, palabra2, código1, etc.} % list of keywords
}

% Font change to Arial
\usepackage{helvet}
\renewcommand{\familydefault}{\sfdefault}

% Chapter titles in uppercase and larger font
\titleformat{\chapter}[hang]{\large\bfseries}{\thechapter.}{1em}{\MakeUppercase}
\titleformat{\section}[hang]{\bfseries}{\thesection.}{1em}{}
\titleformat{\subsection}[hang]{\bfseries}{\thesubsection.}{1em}{}

% Fancyhdr setup
\setlength{\headheight}{14.30174pt} % Adjust to recommended value, slightly larger for safety
\fancyhf{} % Clear all headers and footers
\fancyhead[LE]{\nouppercase{\leftmark}}
\fancyhead[RO]{Optimización energética para vivienda}
\fancyfoot[LE]{\thepage}
\fancyfoot[RE]{Escuela Técnica Superior de Ingenieros Industriales (UPM)}
\fancyfoot[LO]{Luis D. Aranda Sánchez}
\fancyfoot[RO]{\thepage}
\renewcommand{\headrulewidth}{0.4pt}
\renewcommand{\footrulewidth}{0.4pt}

\fancypagestyle{myfancy}{
    \fancyhf{} % Clear all headers and footers
    \fancyhead[LE]{\nouppercase{\leftmark}}
    \fancyhead[RO]{Optimización energética para vivienda}
    \fancyfoot[LE]{\thepage}
    \fancyfoot[RE]{Escuela Técnica Superior de Ingenieros Industriales (UPM)}
    \fancyfoot[LO]{Luis D. Aranda Sánchez}
    \fancyfoot[RO]{\thepage}
    \renewcommand{\headrulewidth}{0.4pt}
    \renewcommand{\footrulewidth}{0.4pt}
}

\fancypagestyle{simple}{
    \fancyhf{} % Clear all headers and footers
    \renewcommand{\headrulewidth}{0pt}
    \renewcommand{\footrulewidth}{0pt}
}

% Line spacing
\setstretch{1.2}

% Document starts here
\begin{document}

% Portada
\begin{titlepage}
    \centering
    {\scshape\LARGE Universidad Politécnica de Madrid \par}
    \vspace{1cm}
    {\scshape\Large Escuela Técnica Superior de Ingenieros Industriales\par}
    \vspace{1.5cm}
    {\huge\bfseries Optimización energética de sistema híbrido con bomba de calor, suelo radiante, fotovoltaica y almacenamiento para vivienda \par}
    \vspace{1.5cm}
    {\Large\bfseries Trabajo de Fin de Máster\par}
    \vspace{0.5cm}
    {\large Máster Universitario en Ingeniería de la Energía \par}
    \vspace{2cm}
    {\Large Luis D. Aranda Sánchez\par}
    \vfill
    Director: Javier Rodríguez Martín
    \vfill
    {\large Septiembre 6, 2024\par}
\end{titlepage}

% Resumen (máximo de 5 páginas, incluyendo al final Palabras clave)
\clearpage
\pagestyle{simple}
% \newpage
\chapter*{Resumen}
\addcontentsline{toc}{chapter}{Resumen}
\documentclass[a4paper,11pt,twoside]{report}
\usepackage[left=25mm,right=25mm,top=25mm,bottom=25mm,includehead,includefoot,headsep=15mm,footskip=15mm]{geometry}
\usepackage{graphicx}
\usepackage{fancyhdr}
\usepackage{titlesec}
\usepackage[spanish]{babel}
\usepackage[utf8]{inputenc}
\usepackage{amsmath}
\usepackage{setspace}
\usepackage{svg}
\usepackage{hyperref}
\usepackage[backend=biber,style=numeric]{biblatex}
\addbibresource{references.bib}
\hypersetup{
    colorlinks=true,
    linkcolor=blue,      % color of internal links (sections, etc.)
    urlcolor=blue,       % color of external links
    pdftitle={Optimización energética de sistema híbrido con bomba de calor, suelo radiante, fotovoltaica y almacenamiento para vivienda},    % title
    pdfauthor={Luis D. Aranda Sánchez},     % author
    pdfkeywords={palabra1, palabra2, código1, etc.} % list of keywords
}

% Font change to Arial
\usepackage{helvet}
\renewcommand{\familydefault}{\sfdefault}

% Chapter titles in uppercase and larger font
\titleformat{\chapter}[hang]{\large\bfseries}{\thechapter.}{1em}{\MakeUppercase}
\titleformat{\section}[hang]{\bfseries}{\thesection.}{1em}{}
\titleformat{\subsection}[hang]{\bfseries}{\thesubsection.}{1em}{}

% Fancyhdr setup
\setlength{\headheight}{14.30174pt} % Adjust to recommended value, slightly larger for safety
\fancyhf{} % Clear all headers and footers
\fancyhead[LE]{\nouppercase{\leftmark}}
\fancyhead[RO]{Optimización energética para vivienda}
\fancyfoot[LE]{\thepage}
\fancyfoot[RE]{Escuela Técnica Superior de Ingenieros Industriales (UPM)}
\fancyfoot[LO]{Luis D. Aranda Sánchez}
\fancyfoot[RO]{\thepage}
\renewcommand{\headrulewidth}{0.4pt}
\renewcommand{\footrulewidth}{0.4pt}

\fancypagestyle{myfancy}{
    \fancyhf{} % Clear all headers and footers
    \fancyhead[LE]{\nouppercase{\leftmark}}
    \fancyhead[RO]{Optimización energética para vivienda}
    \fancyfoot[LE]{\thepage}
    \fancyfoot[RE]{Escuela Técnica Superior de Ingenieros Industriales (UPM)}
    \fancyfoot[LO]{Luis D. Aranda Sánchez}
    \fancyfoot[RO]{\thepage}
    \renewcommand{\headrulewidth}{0.4pt}
    \renewcommand{\footrulewidth}{0.4pt}
}

\fancypagestyle{simple}{
    \fancyhf{} % Clear all headers and footers
    \renewcommand{\headrulewidth}{0pt}
    \renewcommand{\footrulewidth}{0pt}
}

% Line spacing
\setstretch{1.2}

% Document starts here
\begin{document}

% Portada
\begin{titlepage}
    \centering
    {\scshape\LARGE Universidad Politécnica de Madrid \par}
    \vspace{1cm}
    {\scshape\Large Escuela Técnica Superior de Ingenieros Industriales\par}
    \vspace{1.5cm}
    {\huge\bfseries Optimización energética de sistema híbrido con bomba de calor, suelo radiante, fotovoltaica y almacenamiento para vivienda \par}
    \vspace{1.5cm}
    {\Large\bfseries Trabajo de Fin de Máster\par}
    \vspace{0.5cm}
    {\large Máster Universitario en Ingeniería de la Energía \par}
    \vspace{2cm}
    {\Large Luis D. Aranda Sánchez\par}
    \vfill
    Director: Javier Rodríguez Martín
    \vfill
    {\large Septiembre 6, 2024\par}
\end{titlepage}

% Resumen (máximo de 5 páginas, incluyendo al final Palabras clave)
\clearpage
\pagestyle{simple}
% \newpage
\chapter*{Resumen}
\addcontentsline{toc}{chapter}{Resumen}
\input{capitulos/resumen/main.tex}

% Índice (paginado)
\clearpage
\pagestyle{simple}
% \newpage
\tableofcontents

% Introducción (donde se incluya los antecedentes y justificación)
\clearpage
\pagestyle{myfancy}
\newpage
\chapter{Introducción}
\input{capitulos/introduccion/main.tex}

% Objetivos
\chapter{Objetivos}
\input{capitulos/objetivos/main.tex}

% Metodología
\chapter{Metodología}
\input{capitulos/metodologia/main.tex}

% Resultados y discusión (incluyendo la valoración de impactos y de aspectos de responsabilidad legal, ética y profesional relacionados con el trabajo)
\chapter{Resultados y Discusión}
\input{capitulos/resultados_discusion/main.tex}

% Conclusiones
\chapter{Conclusiones}
\input{capitulos/conclusiones/main.tex}

% Planificación temporal y presupuesto
\chapter{Planificación Temporal y Presupuesto}
\input{capitulos/planificacion_presupuesto/main.tex}

% Bibliografía
\newpage
\addcontentsline{toc}{chapter}{Bibliografía}
\printbibliography

\end{document}


% Índice (paginado)
\clearpage
\pagestyle{simple}
% \newpage
\tableofcontents

% Introducción (donde se incluya los antecedentes y justificación)
\clearpage
\pagestyle{myfancy}
\newpage
\chapter{Introducción}
\documentclass[a4paper,11pt,twoside]{report}
\usepackage[left=25mm,right=25mm,top=25mm,bottom=25mm,includehead,includefoot,headsep=15mm,footskip=15mm]{geometry}
\usepackage{graphicx}
\usepackage{fancyhdr}
\usepackage{titlesec}
\usepackage[spanish]{babel}
\usepackage[utf8]{inputenc}
\usepackage{amsmath}
\usepackage{setspace}
\usepackage{svg}
\usepackage{hyperref}
\usepackage[backend=biber,style=numeric]{biblatex}
\addbibresource{references.bib}
\hypersetup{
    colorlinks=true,
    linkcolor=blue,      % color of internal links (sections, etc.)
    urlcolor=blue,       % color of external links
    pdftitle={Optimización energética de sistema híbrido con bomba de calor, suelo radiante, fotovoltaica y almacenamiento para vivienda},    % title
    pdfauthor={Luis D. Aranda Sánchez},     % author
    pdfkeywords={palabra1, palabra2, código1, etc.} % list of keywords
}

% Font change to Arial
\usepackage{helvet}
\renewcommand{\familydefault}{\sfdefault}

% Chapter titles in uppercase and larger font
\titleformat{\chapter}[hang]{\large\bfseries}{\thechapter.}{1em}{\MakeUppercase}
\titleformat{\section}[hang]{\bfseries}{\thesection.}{1em}{}
\titleformat{\subsection}[hang]{\bfseries}{\thesubsection.}{1em}{}

% Fancyhdr setup
\setlength{\headheight}{14.30174pt} % Adjust to recommended value, slightly larger for safety
\fancyhf{} % Clear all headers and footers
\fancyhead[LE]{\nouppercase{\leftmark}}
\fancyhead[RO]{Optimización energética para vivienda}
\fancyfoot[LE]{\thepage}
\fancyfoot[RE]{Escuela Técnica Superior de Ingenieros Industriales (UPM)}
\fancyfoot[LO]{Luis D. Aranda Sánchez}
\fancyfoot[RO]{\thepage}
\renewcommand{\headrulewidth}{0.4pt}
\renewcommand{\footrulewidth}{0.4pt}

\fancypagestyle{myfancy}{
    \fancyhf{} % Clear all headers and footers
    \fancyhead[LE]{\nouppercase{\leftmark}}
    \fancyhead[RO]{Optimización energética para vivienda}
    \fancyfoot[LE]{\thepage}
    \fancyfoot[RE]{Escuela Técnica Superior de Ingenieros Industriales (UPM)}
    \fancyfoot[LO]{Luis D. Aranda Sánchez}
    \fancyfoot[RO]{\thepage}
    \renewcommand{\headrulewidth}{0.4pt}
    \renewcommand{\footrulewidth}{0.4pt}
}

\fancypagestyle{simple}{
    \fancyhf{} % Clear all headers and footers
    \renewcommand{\headrulewidth}{0pt}
    \renewcommand{\footrulewidth}{0pt}
}

% Line spacing
\setstretch{1.2}

% Document starts here
\begin{document}

% Portada
\begin{titlepage}
    \centering
    {\scshape\LARGE Universidad Politécnica de Madrid \par}
    \vspace{1cm}
    {\scshape\Large Escuela Técnica Superior de Ingenieros Industriales\par}
    \vspace{1.5cm}
    {\huge\bfseries Optimización energética de sistema híbrido con bomba de calor, suelo radiante, fotovoltaica y almacenamiento para vivienda \par}
    \vspace{1.5cm}
    {\Large\bfseries Trabajo de Fin de Máster\par}
    \vspace{0.5cm}
    {\large Máster Universitario en Ingeniería de la Energía \par}
    \vspace{2cm}
    {\Large Luis D. Aranda Sánchez\par}
    \vfill
    Director: Javier Rodríguez Martín
    \vfill
    {\large Septiembre 6, 2024\par}
\end{titlepage}

% Resumen (máximo de 5 páginas, incluyendo al final Palabras clave)
\clearpage
\pagestyle{simple}
% \newpage
\chapter*{Resumen}
\addcontentsline{toc}{chapter}{Resumen}
\input{capitulos/resumen/main.tex}

% Índice (paginado)
\clearpage
\pagestyle{simple}
% \newpage
\tableofcontents

% Introducción (donde se incluya los antecedentes y justificación)
\clearpage
\pagestyle{myfancy}
\newpage
\chapter{Introducción}
\input{capitulos/introduccion/main.tex}

% Objetivos
\chapter{Objetivos}
\input{capitulos/objetivos/main.tex}

% Metodología
\chapter{Metodología}
\input{capitulos/metodologia/main.tex}

% Resultados y discusión (incluyendo la valoración de impactos y de aspectos de responsabilidad legal, ética y profesional relacionados con el trabajo)
\chapter{Resultados y Discusión}
\input{capitulos/resultados_discusion/main.tex}

% Conclusiones
\chapter{Conclusiones}
\input{capitulos/conclusiones/main.tex}

% Planificación temporal y presupuesto
\chapter{Planificación Temporal y Presupuesto}
\input{capitulos/planificacion_presupuesto/main.tex}

% Bibliografía
\newpage
\addcontentsline{toc}{chapter}{Bibliografía}
\printbibliography

\end{document}


% Objetivos
\chapter{Objetivos}
\documentclass[a4paper,11pt,twoside]{report}
\usepackage[left=25mm,right=25mm,top=25mm,bottom=25mm,includehead,includefoot,headsep=15mm,footskip=15mm]{geometry}
\usepackage{graphicx}
\usepackage{fancyhdr}
\usepackage{titlesec}
\usepackage[spanish]{babel}
\usepackage[utf8]{inputenc}
\usepackage{amsmath}
\usepackage{setspace}
\usepackage{svg}
\usepackage{hyperref}
\usepackage[backend=biber,style=numeric]{biblatex}
\addbibresource{references.bib}
\hypersetup{
    colorlinks=true,
    linkcolor=blue,      % color of internal links (sections, etc.)
    urlcolor=blue,       % color of external links
    pdftitle={Optimización energética de sistema híbrido con bomba de calor, suelo radiante, fotovoltaica y almacenamiento para vivienda},    % title
    pdfauthor={Luis D. Aranda Sánchez},     % author
    pdfkeywords={palabra1, palabra2, código1, etc.} % list of keywords
}

% Font change to Arial
\usepackage{helvet}
\renewcommand{\familydefault}{\sfdefault}

% Chapter titles in uppercase and larger font
\titleformat{\chapter}[hang]{\large\bfseries}{\thechapter.}{1em}{\MakeUppercase}
\titleformat{\section}[hang]{\bfseries}{\thesection.}{1em}{}
\titleformat{\subsection}[hang]{\bfseries}{\thesubsection.}{1em}{}

% Fancyhdr setup
\setlength{\headheight}{14.30174pt} % Adjust to recommended value, slightly larger for safety
\fancyhf{} % Clear all headers and footers
\fancyhead[LE]{\nouppercase{\leftmark}}
\fancyhead[RO]{Optimización energética para vivienda}
\fancyfoot[LE]{\thepage}
\fancyfoot[RE]{Escuela Técnica Superior de Ingenieros Industriales (UPM)}
\fancyfoot[LO]{Luis D. Aranda Sánchez}
\fancyfoot[RO]{\thepage}
\renewcommand{\headrulewidth}{0.4pt}
\renewcommand{\footrulewidth}{0.4pt}

\fancypagestyle{myfancy}{
    \fancyhf{} % Clear all headers and footers
    \fancyhead[LE]{\nouppercase{\leftmark}}
    \fancyhead[RO]{Optimización energética para vivienda}
    \fancyfoot[LE]{\thepage}
    \fancyfoot[RE]{Escuela Técnica Superior de Ingenieros Industriales (UPM)}
    \fancyfoot[LO]{Luis D. Aranda Sánchez}
    \fancyfoot[RO]{\thepage}
    \renewcommand{\headrulewidth}{0.4pt}
    \renewcommand{\footrulewidth}{0.4pt}
}

\fancypagestyle{simple}{
    \fancyhf{} % Clear all headers and footers
    \renewcommand{\headrulewidth}{0pt}
    \renewcommand{\footrulewidth}{0pt}
}

% Line spacing
\setstretch{1.2}

% Document starts here
\begin{document}

% Portada
\begin{titlepage}
    \centering
    {\scshape\LARGE Universidad Politécnica de Madrid \par}
    \vspace{1cm}
    {\scshape\Large Escuela Técnica Superior de Ingenieros Industriales\par}
    \vspace{1.5cm}
    {\huge\bfseries Optimización energética de sistema híbrido con bomba de calor, suelo radiante, fotovoltaica y almacenamiento para vivienda \par}
    \vspace{1.5cm}
    {\Large\bfseries Trabajo de Fin de Máster\par}
    \vspace{0.5cm}
    {\large Máster Universitario en Ingeniería de la Energía \par}
    \vspace{2cm}
    {\Large Luis D. Aranda Sánchez\par}
    \vfill
    Director: Javier Rodríguez Martín
    \vfill
    {\large Septiembre 6, 2024\par}
\end{titlepage}

% Resumen (máximo de 5 páginas, incluyendo al final Palabras clave)
\clearpage
\pagestyle{simple}
% \newpage
\chapter*{Resumen}
\addcontentsline{toc}{chapter}{Resumen}
\input{capitulos/resumen/main.tex}

% Índice (paginado)
\clearpage
\pagestyle{simple}
% \newpage
\tableofcontents

% Introducción (donde se incluya los antecedentes y justificación)
\clearpage
\pagestyle{myfancy}
\newpage
\chapter{Introducción}
\input{capitulos/introduccion/main.tex}

% Objetivos
\chapter{Objetivos}
\input{capitulos/objetivos/main.tex}

% Metodología
\chapter{Metodología}
\input{capitulos/metodologia/main.tex}

% Resultados y discusión (incluyendo la valoración de impactos y de aspectos de responsabilidad legal, ética y profesional relacionados con el trabajo)
\chapter{Resultados y Discusión}
\input{capitulos/resultados_discusion/main.tex}

% Conclusiones
\chapter{Conclusiones}
\input{capitulos/conclusiones/main.tex}

% Planificación temporal y presupuesto
\chapter{Planificación Temporal y Presupuesto}
\input{capitulos/planificacion_presupuesto/main.tex}

% Bibliografía
\newpage
\addcontentsline{toc}{chapter}{Bibliografía}
\printbibliography

\end{document}


% Metodología
\chapter{Metodología}
\documentclass[a4paper,11pt,twoside]{report}
\usepackage[left=25mm,right=25mm,top=25mm,bottom=25mm,includehead,includefoot,headsep=15mm,footskip=15mm]{geometry}
\usepackage{graphicx}
\usepackage{fancyhdr}
\usepackage{titlesec}
\usepackage[spanish]{babel}
\usepackage[utf8]{inputenc}
\usepackage{amsmath}
\usepackage{setspace}
\usepackage{svg}
\usepackage{hyperref}
\usepackage[backend=biber,style=numeric]{biblatex}
\addbibresource{references.bib}
\hypersetup{
    colorlinks=true,
    linkcolor=blue,      % color of internal links (sections, etc.)
    urlcolor=blue,       % color of external links
    pdftitle={Optimización energética de sistema híbrido con bomba de calor, suelo radiante, fotovoltaica y almacenamiento para vivienda},    % title
    pdfauthor={Luis D. Aranda Sánchez},     % author
    pdfkeywords={palabra1, palabra2, código1, etc.} % list of keywords
}

% Font change to Arial
\usepackage{helvet}
\renewcommand{\familydefault}{\sfdefault}

% Chapter titles in uppercase and larger font
\titleformat{\chapter}[hang]{\large\bfseries}{\thechapter.}{1em}{\MakeUppercase}
\titleformat{\section}[hang]{\bfseries}{\thesection.}{1em}{}
\titleformat{\subsection}[hang]{\bfseries}{\thesubsection.}{1em}{}

% Fancyhdr setup
\setlength{\headheight}{14.30174pt} % Adjust to recommended value, slightly larger for safety
\fancyhf{} % Clear all headers and footers
\fancyhead[LE]{\nouppercase{\leftmark}}
\fancyhead[RO]{Optimización energética para vivienda}
\fancyfoot[LE]{\thepage}
\fancyfoot[RE]{Escuela Técnica Superior de Ingenieros Industriales (UPM)}
\fancyfoot[LO]{Luis D. Aranda Sánchez}
\fancyfoot[RO]{\thepage}
\renewcommand{\headrulewidth}{0.4pt}
\renewcommand{\footrulewidth}{0.4pt}

\fancypagestyle{myfancy}{
    \fancyhf{} % Clear all headers and footers
    \fancyhead[LE]{\nouppercase{\leftmark}}
    \fancyhead[RO]{Optimización energética para vivienda}
    \fancyfoot[LE]{\thepage}
    \fancyfoot[RE]{Escuela Técnica Superior de Ingenieros Industriales (UPM)}
    \fancyfoot[LO]{Luis D. Aranda Sánchez}
    \fancyfoot[RO]{\thepage}
    \renewcommand{\headrulewidth}{0.4pt}
    \renewcommand{\footrulewidth}{0.4pt}
}

\fancypagestyle{simple}{
    \fancyhf{} % Clear all headers and footers
    \renewcommand{\headrulewidth}{0pt}
    \renewcommand{\footrulewidth}{0pt}
}

% Line spacing
\setstretch{1.2}

% Document starts here
\begin{document}

% Portada
\begin{titlepage}
    \centering
    {\scshape\LARGE Universidad Politécnica de Madrid \par}
    \vspace{1cm}
    {\scshape\Large Escuela Técnica Superior de Ingenieros Industriales\par}
    \vspace{1.5cm}
    {\huge\bfseries Optimización energética de sistema híbrido con bomba de calor, suelo radiante, fotovoltaica y almacenamiento para vivienda \par}
    \vspace{1.5cm}
    {\Large\bfseries Trabajo de Fin de Máster\par}
    \vspace{0.5cm}
    {\large Máster Universitario en Ingeniería de la Energía \par}
    \vspace{2cm}
    {\Large Luis D. Aranda Sánchez\par}
    \vfill
    Director: Javier Rodríguez Martín
    \vfill
    {\large Septiembre 6, 2024\par}
\end{titlepage}

% Resumen (máximo de 5 páginas, incluyendo al final Palabras clave)
\clearpage
\pagestyle{simple}
% \newpage
\chapter*{Resumen}
\addcontentsline{toc}{chapter}{Resumen}
\input{capitulos/resumen/main.tex}

% Índice (paginado)
\clearpage
\pagestyle{simple}
% \newpage
\tableofcontents

% Introducción (donde se incluya los antecedentes y justificación)
\clearpage
\pagestyle{myfancy}
\newpage
\chapter{Introducción}
\input{capitulos/introduccion/main.tex}

% Objetivos
\chapter{Objetivos}
\input{capitulos/objetivos/main.tex}

% Metodología
\chapter{Metodología}
\input{capitulos/metodologia/main.tex}

% Resultados y discusión (incluyendo la valoración de impactos y de aspectos de responsabilidad legal, ética y profesional relacionados con el trabajo)
\chapter{Resultados y Discusión}
\input{capitulos/resultados_discusion/main.tex}

% Conclusiones
\chapter{Conclusiones}
\input{capitulos/conclusiones/main.tex}

% Planificación temporal y presupuesto
\chapter{Planificación Temporal y Presupuesto}
\input{capitulos/planificacion_presupuesto/main.tex}

% Bibliografía
\newpage
\addcontentsline{toc}{chapter}{Bibliografía}
\printbibliography

\end{document}


% Resultados y discusión (incluyendo la valoración de impactos y de aspectos de responsabilidad legal, ética y profesional relacionados con el trabajo)
\chapter{Resultados y Discusión}
\documentclass[a4paper,11pt,twoside]{report}
\usepackage[left=25mm,right=25mm,top=25mm,bottom=25mm,includehead,includefoot,headsep=15mm,footskip=15mm]{geometry}
\usepackage{graphicx}
\usepackage{fancyhdr}
\usepackage{titlesec}
\usepackage[spanish]{babel}
\usepackage[utf8]{inputenc}
\usepackage{amsmath}
\usepackage{setspace}
\usepackage{svg}
\usepackage{hyperref}
\usepackage[backend=biber,style=numeric]{biblatex}
\addbibresource{references.bib}
\hypersetup{
    colorlinks=true,
    linkcolor=blue,      % color of internal links (sections, etc.)
    urlcolor=blue,       % color of external links
    pdftitle={Optimización energética de sistema híbrido con bomba de calor, suelo radiante, fotovoltaica y almacenamiento para vivienda},    % title
    pdfauthor={Luis D. Aranda Sánchez},     % author
    pdfkeywords={palabra1, palabra2, código1, etc.} % list of keywords
}

% Font change to Arial
\usepackage{helvet}
\renewcommand{\familydefault}{\sfdefault}

% Chapter titles in uppercase and larger font
\titleformat{\chapter}[hang]{\large\bfseries}{\thechapter.}{1em}{\MakeUppercase}
\titleformat{\section}[hang]{\bfseries}{\thesection.}{1em}{}
\titleformat{\subsection}[hang]{\bfseries}{\thesubsection.}{1em}{}

% Fancyhdr setup
\setlength{\headheight}{14.30174pt} % Adjust to recommended value, slightly larger for safety
\fancyhf{} % Clear all headers and footers
\fancyhead[LE]{\nouppercase{\leftmark}}
\fancyhead[RO]{Optimización energética para vivienda}
\fancyfoot[LE]{\thepage}
\fancyfoot[RE]{Escuela Técnica Superior de Ingenieros Industriales (UPM)}
\fancyfoot[LO]{Luis D. Aranda Sánchez}
\fancyfoot[RO]{\thepage}
\renewcommand{\headrulewidth}{0.4pt}
\renewcommand{\footrulewidth}{0.4pt}

\fancypagestyle{myfancy}{
    \fancyhf{} % Clear all headers and footers
    \fancyhead[LE]{\nouppercase{\leftmark}}
    \fancyhead[RO]{Optimización energética para vivienda}
    \fancyfoot[LE]{\thepage}
    \fancyfoot[RE]{Escuela Técnica Superior de Ingenieros Industriales (UPM)}
    \fancyfoot[LO]{Luis D. Aranda Sánchez}
    \fancyfoot[RO]{\thepage}
    \renewcommand{\headrulewidth}{0.4pt}
    \renewcommand{\footrulewidth}{0.4pt}
}

\fancypagestyle{simple}{
    \fancyhf{} % Clear all headers and footers
    \renewcommand{\headrulewidth}{0pt}
    \renewcommand{\footrulewidth}{0pt}
}

% Line spacing
\setstretch{1.2}

% Document starts here
\begin{document}

% Portada
\begin{titlepage}
    \centering
    {\scshape\LARGE Universidad Politécnica de Madrid \par}
    \vspace{1cm}
    {\scshape\Large Escuela Técnica Superior de Ingenieros Industriales\par}
    \vspace{1.5cm}
    {\huge\bfseries Optimización energética de sistema híbrido con bomba de calor, suelo radiante, fotovoltaica y almacenamiento para vivienda \par}
    \vspace{1.5cm}
    {\Large\bfseries Trabajo de Fin de Máster\par}
    \vspace{0.5cm}
    {\large Máster Universitario en Ingeniería de la Energía \par}
    \vspace{2cm}
    {\Large Luis D. Aranda Sánchez\par}
    \vfill
    Director: Javier Rodríguez Martín
    \vfill
    {\large Septiembre 6, 2024\par}
\end{titlepage}

% Resumen (máximo de 5 páginas, incluyendo al final Palabras clave)
\clearpage
\pagestyle{simple}
% \newpage
\chapter*{Resumen}
\addcontentsline{toc}{chapter}{Resumen}
\input{capitulos/resumen/main.tex}

% Índice (paginado)
\clearpage
\pagestyle{simple}
% \newpage
\tableofcontents

% Introducción (donde se incluya los antecedentes y justificación)
\clearpage
\pagestyle{myfancy}
\newpage
\chapter{Introducción}
\input{capitulos/introduccion/main.tex}

% Objetivos
\chapter{Objetivos}
\input{capitulos/objetivos/main.tex}

% Metodología
\chapter{Metodología}
\input{capitulos/metodologia/main.tex}

% Resultados y discusión (incluyendo la valoración de impactos y de aspectos de responsabilidad legal, ética y profesional relacionados con el trabajo)
\chapter{Resultados y Discusión}
\input{capitulos/resultados_discusion/main.tex}

% Conclusiones
\chapter{Conclusiones}
\input{capitulos/conclusiones/main.tex}

% Planificación temporal y presupuesto
\chapter{Planificación Temporal y Presupuesto}
\input{capitulos/planificacion_presupuesto/main.tex}

% Bibliografía
\newpage
\addcontentsline{toc}{chapter}{Bibliografía}
\printbibliography

\end{document}


% Conclusiones
\chapter{Conclusiones}
\documentclass[a4paper,11pt,twoside]{report}
\usepackage[left=25mm,right=25mm,top=25mm,bottom=25mm,includehead,includefoot,headsep=15mm,footskip=15mm]{geometry}
\usepackage{graphicx}
\usepackage{fancyhdr}
\usepackage{titlesec}
\usepackage[spanish]{babel}
\usepackage[utf8]{inputenc}
\usepackage{amsmath}
\usepackage{setspace}
\usepackage{svg}
\usepackage{hyperref}
\usepackage[backend=biber,style=numeric]{biblatex}
\addbibresource{references.bib}
\hypersetup{
    colorlinks=true,
    linkcolor=blue,      % color of internal links (sections, etc.)
    urlcolor=blue,       % color of external links
    pdftitle={Optimización energética de sistema híbrido con bomba de calor, suelo radiante, fotovoltaica y almacenamiento para vivienda},    % title
    pdfauthor={Luis D. Aranda Sánchez},     % author
    pdfkeywords={palabra1, palabra2, código1, etc.} % list of keywords
}

% Font change to Arial
\usepackage{helvet}
\renewcommand{\familydefault}{\sfdefault}

% Chapter titles in uppercase and larger font
\titleformat{\chapter}[hang]{\large\bfseries}{\thechapter.}{1em}{\MakeUppercase}
\titleformat{\section}[hang]{\bfseries}{\thesection.}{1em}{}
\titleformat{\subsection}[hang]{\bfseries}{\thesubsection.}{1em}{}

% Fancyhdr setup
\setlength{\headheight}{14.30174pt} % Adjust to recommended value, slightly larger for safety
\fancyhf{} % Clear all headers and footers
\fancyhead[LE]{\nouppercase{\leftmark}}
\fancyhead[RO]{Optimización energética para vivienda}
\fancyfoot[LE]{\thepage}
\fancyfoot[RE]{Escuela Técnica Superior de Ingenieros Industriales (UPM)}
\fancyfoot[LO]{Luis D. Aranda Sánchez}
\fancyfoot[RO]{\thepage}
\renewcommand{\headrulewidth}{0.4pt}
\renewcommand{\footrulewidth}{0.4pt}

\fancypagestyle{myfancy}{
    \fancyhf{} % Clear all headers and footers
    \fancyhead[LE]{\nouppercase{\leftmark}}
    \fancyhead[RO]{Optimización energética para vivienda}
    \fancyfoot[LE]{\thepage}
    \fancyfoot[RE]{Escuela Técnica Superior de Ingenieros Industriales (UPM)}
    \fancyfoot[LO]{Luis D. Aranda Sánchez}
    \fancyfoot[RO]{\thepage}
    \renewcommand{\headrulewidth}{0.4pt}
    \renewcommand{\footrulewidth}{0.4pt}
}

\fancypagestyle{simple}{
    \fancyhf{} % Clear all headers and footers
    \renewcommand{\headrulewidth}{0pt}
    \renewcommand{\footrulewidth}{0pt}
}

% Line spacing
\setstretch{1.2}

% Document starts here
\begin{document}

% Portada
\begin{titlepage}
    \centering
    {\scshape\LARGE Universidad Politécnica de Madrid \par}
    \vspace{1cm}
    {\scshape\Large Escuela Técnica Superior de Ingenieros Industriales\par}
    \vspace{1.5cm}
    {\huge\bfseries Optimización energética de sistema híbrido con bomba de calor, suelo radiante, fotovoltaica y almacenamiento para vivienda \par}
    \vspace{1.5cm}
    {\Large\bfseries Trabajo de Fin de Máster\par}
    \vspace{0.5cm}
    {\large Máster Universitario en Ingeniería de la Energía \par}
    \vspace{2cm}
    {\Large Luis D. Aranda Sánchez\par}
    \vfill
    Director: Javier Rodríguez Martín
    \vfill
    {\large Septiembre 6, 2024\par}
\end{titlepage}

% Resumen (máximo de 5 páginas, incluyendo al final Palabras clave)
\clearpage
\pagestyle{simple}
% \newpage
\chapter*{Resumen}
\addcontentsline{toc}{chapter}{Resumen}
\input{capitulos/resumen/main.tex}

% Índice (paginado)
\clearpage
\pagestyle{simple}
% \newpage
\tableofcontents

% Introducción (donde se incluya los antecedentes y justificación)
\clearpage
\pagestyle{myfancy}
\newpage
\chapter{Introducción}
\input{capitulos/introduccion/main.tex}

% Objetivos
\chapter{Objetivos}
\input{capitulos/objetivos/main.tex}

% Metodología
\chapter{Metodología}
\input{capitulos/metodologia/main.tex}

% Resultados y discusión (incluyendo la valoración de impactos y de aspectos de responsabilidad legal, ética y profesional relacionados con el trabajo)
\chapter{Resultados y Discusión}
\input{capitulos/resultados_discusion/main.tex}

% Conclusiones
\chapter{Conclusiones}
\input{capitulos/conclusiones/main.tex}

% Planificación temporal y presupuesto
\chapter{Planificación Temporal y Presupuesto}
\input{capitulos/planificacion_presupuesto/main.tex}

% Bibliografía
\newpage
\addcontentsline{toc}{chapter}{Bibliografía}
\printbibliography

\end{document}


% Planificación temporal y presupuesto
\chapter{Planificación Temporal y Presupuesto}
\documentclass[a4paper,11pt,twoside]{report}
\usepackage[left=25mm,right=25mm,top=25mm,bottom=25mm,includehead,includefoot,headsep=15mm,footskip=15mm]{geometry}
\usepackage{graphicx}
\usepackage{fancyhdr}
\usepackage{titlesec}
\usepackage[spanish]{babel}
\usepackage[utf8]{inputenc}
\usepackage{amsmath}
\usepackage{setspace}
\usepackage{svg}
\usepackage{hyperref}
\usepackage[backend=biber,style=numeric]{biblatex}
\addbibresource{references.bib}
\hypersetup{
    colorlinks=true,
    linkcolor=blue,      % color of internal links (sections, etc.)
    urlcolor=blue,       % color of external links
    pdftitle={Optimización energética de sistema híbrido con bomba de calor, suelo radiante, fotovoltaica y almacenamiento para vivienda},    % title
    pdfauthor={Luis D. Aranda Sánchez},     % author
    pdfkeywords={palabra1, palabra2, código1, etc.} % list of keywords
}

% Font change to Arial
\usepackage{helvet}
\renewcommand{\familydefault}{\sfdefault}

% Chapter titles in uppercase and larger font
\titleformat{\chapter}[hang]{\large\bfseries}{\thechapter.}{1em}{\MakeUppercase}
\titleformat{\section}[hang]{\bfseries}{\thesection.}{1em}{}
\titleformat{\subsection}[hang]{\bfseries}{\thesubsection.}{1em}{}

% Fancyhdr setup
\setlength{\headheight}{14.30174pt} % Adjust to recommended value, slightly larger for safety
\fancyhf{} % Clear all headers and footers
\fancyhead[LE]{\nouppercase{\leftmark}}
\fancyhead[RO]{Optimización energética para vivienda}
\fancyfoot[LE]{\thepage}
\fancyfoot[RE]{Escuela Técnica Superior de Ingenieros Industriales (UPM)}
\fancyfoot[LO]{Luis D. Aranda Sánchez}
\fancyfoot[RO]{\thepage}
\renewcommand{\headrulewidth}{0.4pt}
\renewcommand{\footrulewidth}{0.4pt}

\fancypagestyle{myfancy}{
    \fancyhf{} % Clear all headers and footers
    \fancyhead[LE]{\nouppercase{\leftmark}}
    \fancyhead[RO]{Optimización energética para vivienda}
    \fancyfoot[LE]{\thepage}
    \fancyfoot[RE]{Escuela Técnica Superior de Ingenieros Industriales (UPM)}
    \fancyfoot[LO]{Luis D. Aranda Sánchez}
    \fancyfoot[RO]{\thepage}
    \renewcommand{\headrulewidth}{0.4pt}
    \renewcommand{\footrulewidth}{0.4pt}
}

\fancypagestyle{simple}{
    \fancyhf{} % Clear all headers and footers
    \renewcommand{\headrulewidth}{0pt}
    \renewcommand{\footrulewidth}{0pt}
}

% Line spacing
\setstretch{1.2}

% Document starts here
\begin{document}

% Portada
\begin{titlepage}
    \centering
    {\scshape\LARGE Universidad Politécnica de Madrid \par}
    \vspace{1cm}
    {\scshape\Large Escuela Técnica Superior de Ingenieros Industriales\par}
    \vspace{1.5cm}
    {\huge\bfseries Optimización energética de sistema híbrido con bomba de calor, suelo radiante, fotovoltaica y almacenamiento para vivienda \par}
    \vspace{1.5cm}
    {\Large\bfseries Trabajo de Fin de Máster\par}
    \vspace{0.5cm}
    {\large Máster Universitario en Ingeniería de la Energía \par}
    \vspace{2cm}
    {\Large Luis D. Aranda Sánchez\par}
    \vfill
    Director: Javier Rodríguez Martín
    \vfill
    {\large Septiembre 6, 2024\par}
\end{titlepage}

% Resumen (máximo de 5 páginas, incluyendo al final Palabras clave)
\clearpage
\pagestyle{simple}
% \newpage
\chapter*{Resumen}
\addcontentsline{toc}{chapter}{Resumen}
\input{capitulos/resumen/main.tex}

% Índice (paginado)
\clearpage
\pagestyle{simple}
% \newpage
\tableofcontents

% Introducción (donde se incluya los antecedentes y justificación)
\clearpage
\pagestyle{myfancy}
\newpage
\chapter{Introducción}
\input{capitulos/introduccion/main.tex}

% Objetivos
\chapter{Objetivos}
\input{capitulos/objetivos/main.tex}

% Metodología
\chapter{Metodología}
\input{capitulos/metodologia/main.tex}

% Resultados y discusión (incluyendo la valoración de impactos y de aspectos de responsabilidad legal, ética y profesional relacionados con el trabajo)
\chapter{Resultados y Discusión}
\input{capitulos/resultados_discusion/main.tex}

% Conclusiones
\chapter{Conclusiones}
\input{capitulos/conclusiones/main.tex}

% Planificación temporal y presupuesto
\chapter{Planificación Temporal y Presupuesto}
\input{capitulos/planificacion_presupuesto/main.tex}

% Bibliografía
\newpage
\addcontentsline{toc}{chapter}{Bibliografía}
\printbibliography

\end{document}


% Bibliografía
\newpage
\addcontentsline{toc}{chapter}{Bibliografía}
\printbibliography

\end{document}


% Metodología
\chapter{Metodología}
\documentclass[a4paper,11pt,twoside]{report}
\usepackage[left=25mm,right=25mm,top=25mm,bottom=25mm,includehead,includefoot,headsep=15mm,footskip=15mm]{geometry}
\usepackage{graphicx}
\usepackage{fancyhdr}
\usepackage{titlesec}
\usepackage[spanish]{babel}
\usepackage[utf8]{inputenc}
\usepackage{amsmath}
\usepackage{setspace}
\usepackage{svg}
\usepackage{hyperref}
\usepackage[backend=biber,style=numeric]{biblatex}
\addbibresource{references.bib}
\hypersetup{
    colorlinks=true,
    linkcolor=blue,      % color of internal links (sections, etc.)
    urlcolor=blue,       % color of external links
    pdftitle={Optimización energética de sistema híbrido con bomba de calor, suelo radiante, fotovoltaica y almacenamiento para vivienda},    % title
    pdfauthor={Luis D. Aranda Sánchez},     % author
    pdfkeywords={palabra1, palabra2, código1, etc.} % list of keywords
}

% Font change to Arial
\usepackage{helvet}
\renewcommand{\familydefault}{\sfdefault}

% Chapter titles in uppercase and larger font
\titleformat{\chapter}[hang]{\large\bfseries}{\thechapter.}{1em}{\MakeUppercase}
\titleformat{\section}[hang]{\bfseries}{\thesection.}{1em}{}
\titleformat{\subsection}[hang]{\bfseries}{\thesubsection.}{1em}{}

% Fancyhdr setup
\setlength{\headheight}{14.30174pt} % Adjust to recommended value, slightly larger for safety
\fancyhf{} % Clear all headers and footers
\fancyhead[LE]{\nouppercase{\leftmark}}
\fancyhead[RO]{Optimización energética para vivienda}
\fancyfoot[LE]{\thepage}
\fancyfoot[RE]{Escuela Técnica Superior de Ingenieros Industriales (UPM)}
\fancyfoot[LO]{Luis D. Aranda Sánchez}
\fancyfoot[RO]{\thepage}
\renewcommand{\headrulewidth}{0.4pt}
\renewcommand{\footrulewidth}{0.4pt}

\fancypagestyle{myfancy}{
    \fancyhf{} % Clear all headers and footers
    \fancyhead[LE]{\nouppercase{\leftmark}}
    \fancyhead[RO]{Optimización energética para vivienda}
    \fancyfoot[LE]{\thepage}
    \fancyfoot[RE]{Escuela Técnica Superior de Ingenieros Industriales (UPM)}
    \fancyfoot[LO]{Luis D. Aranda Sánchez}
    \fancyfoot[RO]{\thepage}
    \renewcommand{\headrulewidth}{0.4pt}
    \renewcommand{\footrulewidth}{0.4pt}
}

\fancypagestyle{simple}{
    \fancyhf{} % Clear all headers and footers
    \renewcommand{\headrulewidth}{0pt}
    \renewcommand{\footrulewidth}{0pt}
}

% Line spacing
\setstretch{1.2}

% Document starts here
\begin{document}

% Portada
\begin{titlepage}
    \centering
    {\scshape\LARGE Universidad Politécnica de Madrid \par}
    \vspace{1cm}
    {\scshape\Large Escuela Técnica Superior de Ingenieros Industriales\par}
    \vspace{1.5cm}
    {\huge\bfseries Optimización energética de sistema híbrido con bomba de calor, suelo radiante, fotovoltaica y almacenamiento para vivienda \par}
    \vspace{1.5cm}
    {\Large\bfseries Trabajo de Fin de Máster\par}
    \vspace{0.5cm}
    {\large Máster Universitario en Ingeniería de la Energía \par}
    \vspace{2cm}
    {\Large Luis D. Aranda Sánchez\par}
    \vfill
    Director: Javier Rodríguez Martín
    \vfill
    {\large Septiembre 6, 2024\par}
\end{titlepage}

% Resumen (máximo de 5 páginas, incluyendo al final Palabras clave)
\clearpage
\pagestyle{simple}
% \newpage
\chapter*{Resumen}
\addcontentsline{toc}{chapter}{Resumen}
\documentclass[a4paper,11pt,twoside]{report}
\usepackage[left=25mm,right=25mm,top=25mm,bottom=25mm,includehead,includefoot,headsep=15mm,footskip=15mm]{geometry}
\usepackage{graphicx}
\usepackage{fancyhdr}
\usepackage{titlesec}
\usepackage[spanish]{babel}
\usepackage[utf8]{inputenc}
\usepackage{amsmath}
\usepackage{setspace}
\usepackage{svg}
\usepackage{hyperref}
\usepackage[backend=biber,style=numeric]{biblatex}
\addbibresource{references.bib}
\hypersetup{
    colorlinks=true,
    linkcolor=blue,      % color of internal links (sections, etc.)
    urlcolor=blue,       % color of external links
    pdftitle={Optimización energética de sistema híbrido con bomba de calor, suelo radiante, fotovoltaica y almacenamiento para vivienda},    % title
    pdfauthor={Luis D. Aranda Sánchez},     % author
    pdfkeywords={palabra1, palabra2, código1, etc.} % list of keywords
}

% Font change to Arial
\usepackage{helvet}
\renewcommand{\familydefault}{\sfdefault}

% Chapter titles in uppercase and larger font
\titleformat{\chapter}[hang]{\large\bfseries}{\thechapter.}{1em}{\MakeUppercase}
\titleformat{\section}[hang]{\bfseries}{\thesection.}{1em}{}
\titleformat{\subsection}[hang]{\bfseries}{\thesubsection.}{1em}{}

% Fancyhdr setup
\setlength{\headheight}{14.30174pt} % Adjust to recommended value, slightly larger for safety
\fancyhf{} % Clear all headers and footers
\fancyhead[LE]{\nouppercase{\leftmark}}
\fancyhead[RO]{Optimización energética para vivienda}
\fancyfoot[LE]{\thepage}
\fancyfoot[RE]{Escuela Técnica Superior de Ingenieros Industriales (UPM)}
\fancyfoot[LO]{Luis D. Aranda Sánchez}
\fancyfoot[RO]{\thepage}
\renewcommand{\headrulewidth}{0.4pt}
\renewcommand{\footrulewidth}{0.4pt}

\fancypagestyle{myfancy}{
    \fancyhf{} % Clear all headers and footers
    \fancyhead[LE]{\nouppercase{\leftmark}}
    \fancyhead[RO]{Optimización energética para vivienda}
    \fancyfoot[LE]{\thepage}
    \fancyfoot[RE]{Escuela Técnica Superior de Ingenieros Industriales (UPM)}
    \fancyfoot[LO]{Luis D. Aranda Sánchez}
    \fancyfoot[RO]{\thepage}
    \renewcommand{\headrulewidth}{0.4pt}
    \renewcommand{\footrulewidth}{0.4pt}
}

\fancypagestyle{simple}{
    \fancyhf{} % Clear all headers and footers
    \renewcommand{\headrulewidth}{0pt}
    \renewcommand{\footrulewidth}{0pt}
}

% Line spacing
\setstretch{1.2}

% Document starts here
\begin{document}

% Portada
\begin{titlepage}
    \centering
    {\scshape\LARGE Universidad Politécnica de Madrid \par}
    \vspace{1cm}
    {\scshape\Large Escuela Técnica Superior de Ingenieros Industriales\par}
    \vspace{1.5cm}
    {\huge\bfseries Optimización energética de sistema híbrido con bomba de calor, suelo radiante, fotovoltaica y almacenamiento para vivienda \par}
    \vspace{1.5cm}
    {\Large\bfseries Trabajo de Fin de Máster\par}
    \vspace{0.5cm}
    {\large Máster Universitario en Ingeniería de la Energía \par}
    \vspace{2cm}
    {\Large Luis D. Aranda Sánchez\par}
    \vfill
    Director: Javier Rodríguez Martín
    \vfill
    {\large Septiembre 6, 2024\par}
\end{titlepage}

% Resumen (máximo de 5 páginas, incluyendo al final Palabras clave)
\clearpage
\pagestyle{simple}
% \newpage
\chapter*{Resumen}
\addcontentsline{toc}{chapter}{Resumen}
\input{capitulos/resumen/main.tex}

% Índice (paginado)
\clearpage
\pagestyle{simple}
% \newpage
\tableofcontents

% Introducción (donde se incluya los antecedentes y justificación)
\clearpage
\pagestyle{myfancy}
\newpage
\chapter{Introducción}
\input{capitulos/introduccion/main.tex}

% Objetivos
\chapter{Objetivos}
\input{capitulos/objetivos/main.tex}

% Metodología
\chapter{Metodología}
\input{capitulos/metodologia/main.tex}

% Resultados y discusión (incluyendo la valoración de impactos y de aspectos de responsabilidad legal, ética y profesional relacionados con el trabajo)
\chapter{Resultados y Discusión}
\input{capitulos/resultados_discusion/main.tex}

% Conclusiones
\chapter{Conclusiones}
\input{capitulos/conclusiones/main.tex}

% Planificación temporal y presupuesto
\chapter{Planificación Temporal y Presupuesto}
\input{capitulos/planificacion_presupuesto/main.tex}

% Bibliografía
\newpage
\addcontentsline{toc}{chapter}{Bibliografía}
\printbibliography

\end{document}


% Índice (paginado)
\clearpage
\pagestyle{simple}
% \newpage
\tableofcontents

% Introducción (donde se incluya los antecedentes y justificación)
\clearpage
\pagestyle{myfancy}
\newpage
\chapter{Introducción}
\documentclass[a4paper,11pt,twoside]{report}
\usepackage[left=25mm,right=25mm,top=25mm,bottom=25mm,includehead,includefoot,headsep=15mm,footskip=15mm]{geometry}
\usepackage{graphicx}
\usepackage{fancyhdr}
\usepackage{titlesec}
\usepackage[spanish]{babel}
\usepackage[utf8]{inputenc}
\usepackage{amsmath}
\usepackage{setspace}
\usepackage{svg}
\usepackage{hyperref}
\usepackage[backend=biber,style=numeric]{biblatex}
\addbibresource{references.bib}
\hypersetup{
    colorlinks=true,
    linkcolor=blue,      % color of internal links (sections, etc.)
    urlcolor=blue,       % color of external links
    pdftitle={Optimización energética de sistema híbrido con bomba de calor, suelo radiante, fotovoltaica y almacenamiento para vivienda},    % title
    pdfauthor={Luis D. Aranda Sánchez},     % author
    pdfkeywords={palabra1, palabra2, código1, etc.} % list of keywords
}

% Font change to Arial
\usepackage{helvet}
\renewcommand{\familydefault}{\sfdefault}

% Chapter titles in uppercase and larger font
\titleformat{\chapter}[hang]{\large\bfseries}{\thechapter.}{1em}{\MakeUppercase}
\titleformat{\section}[hang]{\bfseries}{\thesection.}{1em}{}
\titleformat{\subsection}[hang]{\bfseries}{\thesubsection.}{1em}{}

% Fancyhdr setup
\setlength{\headheight}{14.30174pt} % Adjust to recommended value, slightly larger for safety
\fancyhf{} % Clear all headers and footers
\fancyhead[LE]{\nouppercase{\leftmark}}
\fancyhead[RO]{Optimización energética para vivienda}
\fancyfoot[LE]{\thepage}
\fancyfoot[RE]{Escuela Técnica Superior de Ingenieros Industriales (UPM)}
\fancyfoot[LO]{Luis D. Aranda Sánchez}
\fancyfoot[RO]{\thepage}
\renewcommand{\headrulewidth}{0.4pt}
\renewcommand{\footrulewidth}{0.4pt}

\fancypagestyle{myfancy}{
    \fancyhf{} % Clear all headers and footers
    \fancyhead[LE]{\nouppercase{\leftmark}}
    \fancyhead[RO]{Optimización energética para vivienda}
    \fancyfoot[LE]{\thepage}
    \fancyfoot[RE]{Escuela Técnica Superior de Ingenieros Industriales (UPM)}
    \fancyfoot[LO]{Luis D. Aranda Sánchez}
    \fancyfoot[RO]{\thepage}
    \renewcommand{\headrulewidth}{0.4pt}
    \renewcommand{\footrulewidth}{0.4pt}
}

\fancypagestyle{simple}{
    \fancyhf{} % Clear all headers and footers
    \renewcommand{\headrulewidth}{0pt}
    \renewcommand{\footrulewidth}{0pt}
}

% Line spacing
\setstretch{1.2}

% Document starts here
\begin{document}

% Portada
\begin{titlepage}
    \centering
    {\scshape\LARGE Universidad Politécnica de Madrid \par}
    \vspace{1cm}
    {\scshape\Large Escuela Técnica Superior de Ingenieros Industriales\par}
    \vspace{1.5cm}
    {\huge\bfseries Optimización energética de sistema híbrido con bomba de calor, suelo radiante, fotovoltaica y almacenamiento para vivienda \par}
    \vspace{1.5cm}
    {\Large\bfseries Trabajo de Fin de Máster\par}
    \vspace{0.5cm}
    {\large Máster Universitario en Ingeniería de la Energía \par}
    \vspace{2cm}
    {\Large Luis D. Aranda Sánchez\par}
    \vfill
    Director: Javier Rodríguez Martín
    \vfill
    {\large Septiembre 6, 2024\par}
\end{titlepage}

% Resumen (máximo de 5 páginas, incluyendo al final Palabras clave)
\clearpage
\pagestyle{simple}
% \newpage
\chapter*{Resumen}
\addcontentsline{toc}{chapter}{Resumen}
\input{capitulos/resumen/main.tex}

% Índice (paginado)
\clearpage
\pagestyle{simple}
% \newpage
\tableofcontents

% Introducción (donde se incluya los antecedentes y justificación)
\clearpage
\pagestyle{myfancy}
\newpage
\chapter{Introducción}
\input{capitulos/introduccion/main.tex}

% Objetivos
\chapter{Objetivos}
\input{capitulos/objetivos/main.tex}

% Metodología
\chapter{Metodología}
\input{capitulos/metodologia/main.tex}

% Resultados y discusión (incluyendo la valoración de impactos y de aspectos de responsabilidad legal, ética y profesional relacionados con el trabajo)
\chapter{Resultados y Discusión}
\input{capitulos/resultados_discusion/main.tex}

% Conclusiones
\chapter{Conclusiones}
\input{capitulos/conclusiones/main.tex}

% Planificación temporal y presupuesto
\chapter{Planificación Temporal y Presupuesto}
\input{capitulos/planificacion_presupuesto/main.tex}

% Bibliografía
\newpage
\addcontentsline{toc}{chapter}{Bibliografía}
\printbibliography

\end{document}


% Objetivos
\chapter{Objetivos}
\documentclass[a4paper,11pt,twoside]{report}
\usepackage[left=25mm,right=25mm,top=25mm,bottom=25mm,includehead,includefoot,headsep=15mm,footskip=15mm]{geometry}
\usepackage{graphicx}
\usepackage{fancyhdr}
\usepackage{titlesec}
\usepackage[spanish]{babel}
\usepackage[utf8]{inputenc}
\usepackage{amsmath}
\usepackage{setspace}
\usepackage{svg}
\usepackage{hyperref}
\usepackage[backend=biber,style=numeric]{biblatex}
\addbibresource{references.bib}
\hypersetup{
    colorlinks=true,
    linkcolor=blue,      % color of internal links (sections, etc.)
    urlcolor=blue,       % color of external links
    pdftitle={Optimización energética de sistema híbrido con bomba de calor, suelo radiante, fotovoltaica y almacenamiento para vivienda},    % title
    pdfauthor={Luis D. Aranda Sánchez},     % author
    pdfkeywords={palabra1, palabra2, código1, etc.} % list of keywords
}

% Font change to Arial
\usepackage{helvet}
\renewcommand{\familydefault}{\sfdefault}

% Chapter titles in uppercase and larger font
\titleformat{\chapter}[hang]{\large\bfseries}{\thechapter.}{1em}{\MakeUppercase}
\titleformat{\section}[hang]{\bfseries}{\thesection.}{1em}{}
\titleformat{\subsection}[hang]{\bfseries}{\thesubsection.}{1em}{}

% Fancyhdr setup
\setlength{\headheight}{14.30174pt} % Adjust to recommended value, slightly larger for safety
\fancyhf{} % Clear all headers and footers
\fancyhead[LE]{\nouppercase{\leftmark}}
\fancyhead[RO]{Optimización energética para vivienda}
\fancyfoot[LE]{\thepage}
\fancyfoot[RE]{Escuela Técnica Superior de Ingenieros Industriales (UPM)}
\fancyfoot[LO]{Luis D. Aranda Sánchez}
\fancyfoot[RO]{\thepage}
\renewcommand{\headrulewidth}{0.4pt}
\renewcommand{\footrulewidth}{0.4pt}

\fancypagestyle{myfancy}{
    \fancyhf{} % Clear all headers and footers
    \fancyhead[LE]{\nouppercase{\leftmark}}
    \fancyhead[RO]{Optimización energética para vivienda}
    \fancyfoot[LE]{\thepage}
    \fancyfoot[RE]{Escuela Técnica Superior de Ingenieros Industriales (UPM)}
    \fancyfoot[LO]{Luis D. Aranda Sánchez}
    \fancyfoot[RO]{\thepage}
    \renewcommand{\headrulewidth}{0.4pt}
    \renewcommand{\footrulewidth}{0.4pt}
}

\fancypagestyle{simple}{
    \fancyhf{} % Clear all headers and footers
    \renewcommand{\headrulewidth}{0pt}
    \renewcommand{\footrulewidth}{0pt}
}

% Line spacing
\setstretch{1.2}

% Document starts here
\begin{document}

% Portada
\begin{titlepage}
    \centering
    {\scshape\LARGE Universidad Politécnica de Madrid \par}
    \vspace{1cm}
    {\scshape\Large Escuela Técnica Superior de Ingenieros Industriales\par}
    \vspace{1.5cm}
    {\huge\bfseries Optimización energética de sistema híbrido con bomba de calor, suelo radiante, fotovoltaica y almacenamiento para vivienda \par}
    \vspace{1.5cm}
    {\Large\bfseries Trabajo de Fin de Máster\par}
    \vspace{0.5cm}
    {\large Máster Universitario en Ingeniería de la Energía \par}
    \vspace{2cm}
    {\Large Luis D. Aranda Sánchez\par}
    \vfill
    Director: Javier Rodríguez Martín
    \vfill
    {\large Septiembre 6, 2024\par}
\end{titlepage}

% Resumen (máximo de 5 páginas, incluyendo al final Palabras clave)
\clearpage
\pagestyle{simple}
% \newpage
\chapter*{Resumen}
\addcontentsline{toc}{chapter}{Resumen}
\input{capitulos/resumen/main.tex}

% Índice (paginado)
\clearpage
\pagestyle{simple}
% \newpage
\tableofcontents

% Introducción (donde se incluya los antecedentes y justificación)
\clearpage
\pagestyle{myfancy}
\newpage
\chapter{Introducción}
\input{capitulos/introduccion/main.tex}

% Objetivos
\chapter{Objetivos}
\input{capitulos/objetivos/main.tex}

% Metodología
\chapter{Metodología}
\input{capitulos/metodologia/main.tex}

% Resultados y discusión (incluyendo la valoración de impactos y de aspectos de responsabilidad legal, ética y profesional relacionados con el trabajo)
\chapter{Resultados y Discusión}
\input{capitulos/resultados_discusion/main.tex}

% Conclusiones
\chapter{Conclusiones}
\input{capitulos/conclusiones/main.tex}

% Planificación temporal y presupuesto
\chapter{Planificación Temporal y Presupuesto}
\input{capitulos/planificacion_presupuesto/main.tex}

% Bibliografía
\newpage
\addcontentsline{toc}{chapter}{Bibliografía}
\printbibliography

\end{document}


% Metodología
\chapter{Metodología}
\documentclass[a4paper,11pt,twoside]{report}
\usepackage[left=25mm,right=25mm,top=25mm,bottom=25mm,includehead,includefoot,headsep=15mm,footskip=15mm]{geometry}
\usepackage{graphicx}
\usepackage{fancyhdr}
\usepackage{titlesec}
\usepackage[spanish]{babel}
\usepackage[utf8]{inputenc}
\usepackage{amsmath}
\usepackage{setspace}
\usepackage{svg}
\usepackage{hyperref}
\usepackage[backend=biber,style=numeric]{biblatex}
\addbibresource{references.bib}
\hypersetup{
    colorlinks=true,
    linkcolor=blue,      % color of internal links (sections, etc.)
    urlcolor=blue,       % color of external links
    pdftitle={Optimización energética de sistema híbrido con bomba de calor, suelo radiante, fotovoltaica y almacenamiento para vivienda},    % title
    pdfauthor={Luis D. Aranda Sánchez},     % author
    pdfkeywords={palabra1, palabra2, código1, etc.} % list of keywords
}

% Font change to Arial
\usepackage{helvet}
\renewcommand{\familydefault}{\sfdefault}

% Chapter titles in uppercase and larger font
\titleformat{\chapter}[hang]{\large\bfseries}{\thechapter.}{1em}{\MakeUppercase}
\titleformat{\section}[hang]{\bfseries}{\thesection.}{1em}{}
\titleformat{\subsection}[hang]{\bfseries}{\thesubsection.}{1em}{}

% Fancyhdr setup
\setlength{\headheight}{14.30174pt} % Adjust to recommended value, slightly larger for safety
\fancyhf{} % Clear all headers and footers
\fancyhead[LE]{\nouppercase{\leftmark}}
\fancyhead[RO]{Optimización energética para vivienda}
\fancyfoot[LE]{\thepage}
\fancyfoot[RE]{Escuela Técnica Superior de Ingenieros Industriales (UPM)}
\fancyfoot[LO]{Luis D. Aranda Sánchez}
\fancyfoot[RO]{\thepage}
\renewcommand{\headrulewidth}{0.4pt}
\renewcommand{\footrulewidth}{0.4pt}

\fancypagestyle{myfancy}{
    \fancyhf{} % Clear all headers and footers
    \fancyhead[LE]{\nouppercase{\leftmark}}
    \fancyhead[RO]{Optimización energética para vivienda}
    \fancyfoot[LE]{\thepage}
    \fancyfoot[RE]{Escuela Técnica Superior de Ingenieros Industriales (UPM)}
    \fancyfoot[LO]{Luis D. Aranda Sánchez}
    \fancyfoot[RO]{\thepage}
    \renewcommand{\headrulewidth}{0.4pt}
    \renewcommand{\footrulewidth}{0.4pt}
}

\fancypagestyle{simple}{
    \fancyhf{} % Clear all headers and footers
    \renewcommand{\headrulewidth}{0pt}
    \renewcommand{\footrulewidth}{0pt}
}

% Line spacing
\setstretch{1.2}

% Document starts here
\begin{document}

% Portada
\begin{titlepage}
    \centering
    {\scshape\LARGE Universidad Politécnica de Madrid \par}
    \vspace{1cm}
    {\scshape\Large Escuela Técnica Superior de Ingenieros Industriales\par}
    \vspace{1.5cm}
    {\huge\bfseries Optimización energética de sistema híbrido con bomba de calor, suelo radiante, fotovoltaica y almacenamiento para vivienda \par}
    \vspace{1.5cm}
    {\Large\bfseries Trabajo de Fin de Máster\par}
    \vspace{0.5cm}
    {\large Máster Universitario en Ingeniería de la Energía \par}
    \vspace{2cm}
    {\Large Luis D. Aranda Sánchez\par}
    \vfill
    Director: Javier Rodríguez Martín
    \vfill
    {\large Septiembre 6, 2024\par}
\end{titlepage}

% Resumen (máximo de 5 páginas, incluyendo al final Palabras clave)
\clearpage
\pagestyle{simple}
% \newpage
\chapter*{Resumen}
\addcontentsline{toc}{chapter}{Resumen}
\input{capitulos/resumen/main.tex}

% Índice (paginado)
\clearpage
\pagestyle{simple}
% \newpage
\tableofcontents

% Introducción (donde se incluya los antecedentes y justificación)
\clearpage
\pagestyle{myfancy}
\newpage
\chapter{Introducción}
\input{capitulos/introduccion/main.tex}

% Objetivos
\chapter{Objetivos}
\input{capitulos/objetivos/main.tex}

% Metodología
\chapter{Metodología}
\input{capitulos/metodologia/main.tex}

% Resultados y discusión (incluyendo la valoración de impactos y de aspectos de responsabilidad legal, ética y profesional relacionados con el trabajo)
\chapter{Resultados y Discusión}
\input{capitulos/resultados_discusion/main.tex}

% Conclusiones
\chapter{Conclusiones}
\input{capitulos/conclusiones/main.tex}

% Planificación temporal y presupuesto
\chapter{Planificación Temporal y Presupuesto}
\input{capitulos/planificacion_presupuesto/main.tex}

% Bibliografía
\newpage
\addcontentsline{toc}{chapter}{Bibliografía}
\printbibliography

\end{document}


% Resultados y discusión (incluyendo la valoración de impactos y de aspectos de responsabilidad legal, ética y profesional relacionados con el trabajo)
\chapter{Resultados y Discusión}
\documentclass[a4paper,11pt,twoside]{report}
\usepackage[left=25mm,right=25mm,top=25mm,bottom=25mm,includehead,includefoot,headsep=15mm,footskip=15mm]{geometry}
\usepackage{graphicx}
\usepackage{fancyhdr}
\usepackage{titlesec}
\usepackage[spanish]{babel}
\usepackage[utf8]{inputenc}
\usepackage{amsmath}
\usepackage{setspace}
\usepackage{svg}
\usepackage{hyperref}
\usepackage[backend=biber,style=numeric]{biblatex}
\addbibresource{references.bib}
\hypersetup{
    colorlinks=true,
    linkcolor=blue,      % color of internal links (sections, etc.)
    urlcolor=blue,       % color of external links
    pdftitle={Optimización energética de sistema híbrido con bomba de calor, suelo radiante, fotovoltaica y almacenamiento para vivienda},    % title
    pdfauthor={Luis D. Aranda Sánchez},     % author
    pdfkeywords={palabra1, palabra2, código1, etc.} % list of keywords
}

% Font change to Arial
\usepackage{helvet}
\renewcommand{\familydefault}{\sfdefault}

% Chapter titles in uppercase and larger font
\titleformat{\chapter}[hang]{\large\bfseries}{\thechapter.}{1em}{\MakeUppercase}
\titleformat{\section}[hang]{\bfseries}{\thesection.}{1em}{}
\titleformat{\subsection}[hang]{\bfseries}{\thesubsection.}{1em}{}

% Fancyhdr setup
\setlength{\headheight}{14.30174pt} % Adjust to recommended value, slightly larger for safety
\fancyhf{} % Clear all headers and footers
\fancyhead[LE]{\nouppercase{\leftmark}}
\fancyhead[RO]{Optimización energética para vivienda}
\fancyfoot[LE]{\thepage}
\fancyfoot[RE]{Escuela Técnica Superior de Ingenieros Industriales (UPM)}
\fancyfoot[LO]{Luis D. Aranda Sánchez}
\fancyfoot[RO]{\thepage}
\renewcommand{\headrulewidth}{0.4pt}
\renewcommand{\footrulewidth}{0.4pt}

\fancypagestyle{myfancy}{
    \fancyhf{} % Clear all headers and footers
    \fancyhead[LE]{\nouppercase{\leftmark}}
    \fancyhead[RO]{Optimización energética para vivienda}
    \fancyfoot[LE]{\thepage}
    \fancyfoot[RE]{Escuela Técnica Superior de Ingenieros Industriales (UPM)}
    \fancyfoot[LO]{Luis D. Aranda Sánchez}
    \fancyfoot[RO]{\thepage}
    \renewcommand{\headrulewidth}{0.4pt}
    \renewcommand{\footrulewidth}{0.4pt}
}

\fancypagestyle{simple}{
    \fancyhf{} % Clear all headers and footers
    \renewcommand{\headrulewidth}{0pt}
    \renewcommand{\footrulewidth}{0pt}
}

% Line spacing
\setstretch{1.2}

% Document starts here
\begin{document}

% Portada
\begin{titlepage}
    \centering
    {\scshape\LARGE Universidad Politécnica de Madrid \par}
    \vspace{1cm}
    {\scshape\Large Escuela Técnica Superior de Ingenieros Industriales\par}
    \vspace{1.5cm}
    {\huge\bfseries Optimización energética de sistema híbrido con bomba de calor, suelo radiante, fotovoltaica y almacenamiento para vivienda \par}
    \vspace{1.5cm}
    {\Large\bfseries Trabajo de Fin de Máster\par}
    \vspace{0.5cm}
    {\large Máster Universitario en Ingeniería de la Energía \par}
    \vspace{2cm}
    {\Large Luis D. Aranda Sánchez\par}
    \vfill
    Director: Javier Rodríguez Martín
    \vfill
    {\large Septiembre 6, 2024\par}
\end{titlepage}

% Resumen (máximo de 5 páginas, incluyendo al final Palabras clave)
\clearpage
\pagestyle{simple}
% \newpage
\chapter*{Resumen}
\addcontentsline{toc}{chapter}{Resumen}
\input{capitulos/resumen/main.tex}

% Índice (paginado)
\clearpage
\pagestyle{simple}
% \newpage
\tableofcontents

% Introducción (donde se incluya los antecedentes y justificación)
\clearpage
\pagestyle{myfancy}
\newpage
\chapter{Introducción}
\input{capitulos/introduccion/main.tex}

% Objetivos
\chapter{Objetivos}
\input{capitulos/objetivos/main.tex}

% Metodología
\chapter{Metodología}
\input{capitulos/metodologia/main.tex}

% Resultados y discusión (incluyendo la valoración de impactos y de aspectos de responsabilidad legal, ética y profesional relacionados con el trabajo)
\chapter{Resultados y Discusión}
\input{capitulos/resultados_discusion/main.tex}

% Conclusiones
\chapter{Conclusiones}
\input{capitulos/conclusiones/main.tex}

% Planificación temporal y presupuesto
\chapter{Planificación Temporal y Presupuesto}
\input{capitulos/planificacion_presupuesto/main.tex}

% Bibliografía
\newpage
\addcontentsline{toc}{chapter}{Bibliografía}
\printbibliography

\end{document}


% Conclusiones
\chapter{Conclusiones}
\documentclass[a4paper,11pt,twoside]{report}
\usepackage[left=25mm,right=25mm,top=25mm,bottom=25mm,includehead,includefoot,headsep=15mm,footskip=15mm]{geometry}
\usepackage{graphicx}
\usepackage{fancyhdr}
\usepackage{titlesec}
\usepackage[spanish]{babel}
\usepackage[utf8]{inputenc}
\usepackage{amsmath}
\usepackage{setspace}
\usepackage{svg}
\usepackage{hyperref}
\usepackage[backend=biber,style=numeric]{biblatex}
\addbibresource{references.bib}
\hypersetup{
    colorlinks=true,
    linkcolor=blue,      % color of internal links (sections, etc.)
    urlcolor=blue,       % color of external links
    pdftitle={Optimización energética de sistema híbrido con bomba de calor, suelo radiante, fotovoltaica y almacenamiento para vivienda},    % title
    pdfauthor={Luis D. Aranda Sánchez},     % author
    pdfkeywords={palabra1, palabra2, código1, etc.} % list of keywords
}

% Font change to Arial
\usepackage{helvet}
\renewcommand{\familydefault}{\sfdefault}

% Chapter titles in uppercase and larger font
\titleformat{\chapter}[hang]{\large\bfseries}{\thechapter.}{1em}{\MakeUppercase}
\titleformat{\section}[hang]{\bfseries}{\thesection.}{1em}{}
\titleformat{\subsection}[hang]{\bfseries}{\thesubsection.}{1em}{}

% Fancyhdr setup
\setlength{\headheight}{14.30174pt} % Adjust to recommended value, slightly larger for safety
\fancyhf{} % Clear all headers and footers
\fancyhead[LE]{\nouppercase{\leftmark}}
\fancyhead[RO]{Optimización energética para vivienda}
\fancyfoot[LE]{\thepage}
\fancyfoot[RE]{Escuela Técnica Superior de Ingenieros Industriales (UPM)}
\fancyfoot[LO]{Luis D. Aranda Sánchez}
\fancyfoot[RO]{\thepage}
\renewcommand{\headrulewidth}{0.4pt}
\renewcommand{\footrulewidth}{0.4pt}

\fancypagestyle{myfancy}{
    \fancyhf{} % Clear all headers and footers
    \fancyhead[LE]{\nouppercase{\leftmark}}
    \fancyhead[RO]{Optimización energética para vivienda}
    \fancyfoot[LE]{\thepage}
    \fancyfoot[RE]{Escuela Técnica Superior de Ingenieros Industriales (UPM)}
    \fancyfoot[LO]{Luis D. Aranda Sánchez}
    \fancyfoot[RO]{\thepage}
    \renewcommand{\headrulewidth}{0.4pt}
    \renewcommand{\footrulewidth}{0.4pt}
}

\fancypagestyle{simple}{
    \fancyhf{} % Clear all headers and footers
    \renewcommand{\headrulewidth}{0pt}
    \renewcommand{\footrulewidth}{0pt}
}

% Line spacing
\setstretch{1.2}

% Document starts here
\begin{document}

% Portada
\begin{titlepage}
    \centering
    {\scshape\LARGE Universidad Politécnica de Madrid \par}
    \vspace{1cm}
    {\scshape\Large Escuela Técnica Superior de Ingenieros Industriales\par}
    \vspace{1.5cm}
    {\huge\bfseries Optimización energética de sistema híbrido con bomba de calor, suelo radiante, fotovoltaica y almacenamiento para vivienda \par}
    \vspace{1.5cm}
    {\Large\bfseries Trabajo de Fin de Máster\par}
    \vspace{0.5cm}
    {\large Máster Universitario en Ingeniería de la Energía \par}
    \vspace{2cm}
    {\Large Luis D. Aranda Sánchez\par}
    \vfill
    Director: Javier Rodríguez Martín
    \vfill
    {\large Septiembre 6, 2024\par}
\end{titlepage}

% Resumen (máximo de 5 páginas, incluyendo al final Palabras clave)
\clearpage
\pagestyle{simple}
% \newpage
\chapter*{Resumen}
\addcontentsline{toc}{chapter}{Resumen}
\input{capitulos/resumen/main.tex}

% Índice (paginado)
\clearpage
\pagestyle{simple}
% \newpage
\tableofcontents

% Introducción (donde se incluya los antecedentes y justificación)
\clearpage
\pagestyle{myfancy}
\newpage
\chapter{Introducción}
\input{capitulos/introduccion/main.tex}

% Objetivos
\chapter{Objetivos}
\input{capitulos/objetivos/main.tex}

% Metodología
\chapter{Metodología}
\input{capitulos/metodologia/main.tex}

% Resultados y discusión (incluyendo la valoración de impactos y de aspectos de responsabilidad legal, ética y profesional relacionados con el trabajo)
\chapter{Resultados y Discusión}
\input{capitulos/resultados_discusion/main.tex}

% Conclusiones
\chapter{Conclusiones}
\input{capitulos/conclusiones/main.tex}

% Planificación temporal y presupuesto
\chapter{Planificación Temporal y Presupuesto}
\input{capitulos/planificacion_presupuesto/main.tex}

% Bibliografía
\newpage
\addcontentsline{toc}{chapter}{Bibliografía}
\printbibliography

\end{document}


% Planificación temporal y presupuesto
\chapter{Planificación Temporal y Presupuesto}
\documentclass[a4paper,11pt,twoside]{report}
\usepackage[left=25mm,right=25mm,top=25mm,bottom=25mm,includehead,includefoot,headsep=15mm,footskip=15mm]{geometry}
\usepackage{graphicx}
\usepackage{fancyhdr}
\usepackage{titlesec}
\usepackage[spanish]{babel}
\usepackage[utf8]{inputenc}
\usepackage{amsmath}
\usepackage{setspace}
\usepackage{svg}
\usepackage{hyperref}
\usepackage[backend=biber,style=numeric]{biblatex}
\addbibresource{references.bib}
\hypersetup{
    colorlinks=true,
    linkcolor=blue,      % color of internal links (sections, etc.)
    urlcolor=blue,       % color of external links
    pdftitle={Optimización energética de sistema híbrido con bomba de calor, suelo radiante, fotovoltaica y almacenamiento para vivienda},    % title
    pdfauthor={Luis D. Aranda Sánchez},     % author
    pdfkeywords={palabra1, palabra2, código1, etc.} % list of keywords
}

% Font change to Arial
\usepackage{helvet}
\renewcommand{\familydefault}{\sfdefault}

% Chapter titles in uppercase and larger font
\titleformat{\chapter}[hang]{\large\bfseries}{\thechapter.}{1em}{\MakeUppercase}
\titleformat{\section}[hang]{\bfseries}{\thesection.}{1em}{}
\titleformat{\subsection}[hang]{\bfseries}{\thesubsection.}{1em}{}

% Fancyhdr setup
\setlength{\headheight}{14.30174pt} % Adjust to recommended value, slightly larger for safety
\fancyhf{} % Clear all headers and footers
\fancyhead[LE]{\nouppercase{\leftmark}}
\fancyhead[RO]{Optimización energética para vivienda}
\fancyfoot[LE]{\thepage}
\fancyfoot[RE]{Escuela Técnica Superior de Ingenieros Industriales (UPM)}
\fancyfoot[LO]{Luis D. Aranda Sánchez}
\fancyfoot[RO]{\thepage}
\renewcommand{\headrulewidth}{0.4pt}
\renewcommand{\footrulewidth}{0.4pt}

\fancypagestyle{myfancy}{
    \fancyhf{} % Clear all headers and footers
    \fancyhead[LE]{\nouppercase{\leftmark}}
    \fancyhead[RO]{Optimización energética para vivienda}
    \fancyfoot[LE]{\thepage}
    \fancyfoot[RE]{Escuela Técnica Superior de Ingenieros Industriales (UPM)}
    \fancyfoot[LO]{Luis D. Aranda Sánchez}
    \fancyfoot[RO]{\thepage}
    \renewcommand{\headrulewidth}{0.4pt}
    \renewcommand{\footrulewidth}{0.4pt}
}

\fancypagestyle{simple}{
    \fancyhf{} % Clear all headers and footers
    \renewcommand{\headrulewidth}{0pt}
    \renewcommand{\footrulewidth}{0pt}
}

% Line spacing
\setstretch{1.2}

% Document starts here
\begin{document}

% Portada
\begin{titlepage}
    \centering
    {\scshape\LARGE Universidad Politécnica de Madrid \par}
    \vspace{1cm}
    {\scshape\Large Escuela Técnica Superior de Ingenieros Industriales\par}
    \vspace{1.5cm}
    {\huge\bfseries Optimización energética de sistema híbrido con bomba de calor, suelo radiante, fotovoltaica y almacenamiento para vivienda \par}
    \vspace{1.5cm}
    {\Large\bfseries Trabajo de Fin de Máster\par}
    \vspace{0.5cm}
    {\large Máster Universitario en Ingeniería de la Energía \par}
    \vspace{2cm}
    {\Large Luis D. Aranda Sánchez\par}
    \vfill
    Director: Javier Rodríguez Martín
    \vfill
    {\large Septiembre 6, 2024\par}
\end{titlepage}

% Resumen (máximo de 5 páginas, incluyendo al final Palabras clave)
\clearpage
\pagestyle{simple}
% \newpage
\chapter*{Resumen}
\addcontentsline{toc}{chapter}{Resumen}
\input{capitulos/resumen/main.tex}

% Índice (paginado)
\clearpage
\pagestyle{simple}
% \newpage
\tableofcontents

% Introducción (donde se incluya los antecedentes y justificación)
\clearpage
\pagestyle{myfancy}
\newpage
\chapter{Introducción}
\input{capitulos/introduccion/main.tex}

% Objetivos
\chapter{Objetivos}
\input{capitulos/objetivos/main.tex}

% Metodología
\chapter{Metodología}
\input{capitulos/metodologia/main.tex}

% Resultados y discusión (incluyendo la valoración de impactos y de aspectos de responsabilidad legal, ética y profesional relacionados con el trabajo)
\chapter{Resultados y Discusión}
\input{capitulos/resultados_discusion/main.tex}

% Conclusiones
\chapter{Conclusiones}
\input{capitulos/conclusiones/main.tex}

% Planificación temporal y presupuesto
\chapter{Planificación Temporal y Presupuesto}
\input{capitulos/planificacion_presupuesto/main.tex}

% Bibliografía
\newpage
\addcontentsline{toc}{chapter}{Bibliografía}
\printbibliography

\end{document}


% Bibliografía
\newpage
\addcontentsline{toc}{chapter}{Bibliografía}
\printbibliography

\end{document}


% Resultados y discusión (incluyendo la valoración de impactos y de aspectos de responsabilidad legal, ética y profesional relacionados con el trabajo)
\chapter{Resultados y Discusión}
\documentclass[a4paper,11pt,twoside]{report}
\usepackage[left=25mm,right=25mm,top=25mm,bottom=25mm,includehead,includefoot,headsep=15mm,footskip=15mm]{geometry}
\usepackage{graphicx}
\usepackage{fancyhdr}
\usepackage{titlesec}
\usepackage[spanish]{babel}
\usepackage[utf8]{inputenc}
\usepackage{amsmath}
\usepackage{setspace}
\usepackage{svg}
\usepackage{hyperref}
\usepackage[backend=biber,style=numeric]{biblatex}
\addbibresource{references.bib}
\hypersetup{
    colorlinks=true,
    linkcolor=blue,      % color of internal links (sections, etc.)
    urlcolor=blue,       % color of external links
    pdftitle={Optimización energética de sistema híbrido con bomba de calor, suelo radiante, fotovoltaica y almacenamiento para vivienda},    % title
    pdfauthor={Luis D. Aranda Sánchez},     % author
    pdfkeywords={palabra1, palabra2, código1, etc.} % list of keywords
}

% Font change to Arial
\usepackage{helvet}
\renewcommand{\familydefault}{\sfdefault}

% Chapter titles in uppercase and larger font
\titleformat{\chapter}[hang]{\large\bfseries}{\thechapter.}{1em}{\MakeUppercase}
\titleformat{\section}[hang]{\bfseries}{\thesection.}{1em}{}
\titleformat{\subsection}[hang]{\bfseries}{\thesubsection.}{1em}{}

% Fancyhdr setup
\setlength{\headheight}{14.30174pt} % Adjust to recommended value, slightly larger for safety
\fancyhf{} % Clear all headers and footers
\fancyhead[LE]{\nouppercase{\leftmark}}
\fancyhead[RO]{Optimización energética para vivienda}
\fancyfoot[LE]{\thepage}
\fancyfoot[RE]{Escuela Técnica Superior de Ingenieros Industriales (UPM)}
\fancyfoot[LO]{Luis D. Aranda Sánchez}
\fancyfoot[RO]{\thepage}
\renewcommand{\headrulewidth}{0.4pt}
\renewcommand{\footrulewidth}{0.4pt}

\fancypagestyle{myfancy}{
    \fancyhf{} % Clear all headers and footers
    \fancyhead[LE]{\nouppercase{\leftmark}}
    \fancyhead[RO]{Optimización energética para vivienda}
    \fancyfoot[LE]{\thepage}
    \fancyfoot[RE]{Escuela Técnica Superior de Ingenieros Industriales (UPM)}
    \fancyfoot[LO]{Luis D. Aranda Sánchez}
    \fancyfoot[RO]{\thepage}
    \renewcommand{\headrulewidth}{0.4pt}
    \renewcommand{\footrulewidth}{0.4pt}
}

\fancypagestyle{simple}{
    \fancyhf{} % Clear all headers and footers
    \renewcommand{\headrulewidth}{0pt}
    \renewcommand{\footrulewidth}{0pt}
}

% Line spacing
\setstretch{1.2}

% Document starts here
\begin{document}

% Portada
\begin{titlepage}
    \centering
    {\scshape\LARGE Universidad Politécnica de Madrid \par}
    \vspace{1cm}
    {\scshape\Large Escuela Técnica Superior de Ingenieros Industriales\par}
    \vspace{1.5cm}
    {\huge\bfseries Optimización energética de sistema híbrido con bomba de calor, suelo radiante, fotovoltaica y almacenamiento para vivienda \par}
    \vspace{1.5cm}
    {\Large\bfseries Trabajo de Fin de Máster\par}
    \vspace{0.5cm}
    {\large Máster Universitario en Ingeniería de la Energía \par}
    \vspace{2cm}
    {\Large Luis D. Aranda Sánchez\par}
    \vfill
    Director: Javier Rodríguez Martín
    \vfill
    {\large Septiembre 6, 2024\par}
\end{titlepage}

% Resumen (máximo de 5 páginas, incluyendo al final Palabras clave)
\clearpage
\pagestyle{simple}
% \newpage
\chapter*{Resumen}
\addcontentsline{toc}{chapter}{Resumen}
\documentclass[a4paper,11pt,twoside]{report}
\usepackage[left=25mm,right=25mm,top=25mm,bottom=25mm,includehead,includefoot,headsep=15mm,footskip=15mm]{geometry}
\usepackage{graphicx}
\usepackage{fancyhdr}
\usepackage{titlesec}
\usepackage[spanish]{babel}
\usepackage[utf8]{inputenc}
\usepackage{amsmath}
\usepackage{setspace}
\usepackage{svg}
\usepackage{hyperref}
\usepackage[backend=biber,style=numeric]{biblatex}
\addbibresource{references.bib}
\hypersetup{
    colorlinks=true,
    linkcolor=blue,      % color of internal links (sections, etc.)
    urlcolor=blue,       % color of external links
    pdftitle={Optimización energética de sistema híbrido con bomba de calor, suelo radiante, fotovoltaica y almacenamiento para vivienda},    % title
    pdfauthor={Luis D. Aranda Sánchez},     % author
    pdfkeywords={palabra1, palabra2, código1, etc.} % list of keywords
}

% Font change to Arial
\usepackage{helvet}
\renewcommand{\familydefault}{\sfdefault}

% Chapter titles in uppercase and larger font
\titleformat{\chapter}[hang]{\large\bfseries}{\thechapter.}{1em}{\MakeUppercase}
\titleformat{\section}[hang]{\bfseries}{\thesection.}{1em}{}
\titleformat{\subsection}[hang]{\bfseries}{\thesubsection.}{1em}{}

% Fancyhdr setup
\setlength{\headheight}{14.30174pt} % Adjust to recommended value, slightly larger for safety
\fancyhf{} % Clear all headers and footers
\fancyhead[LE]{\nouppercase{\leftmark}}
\fancyhead[RO]{Optimización energética para vivienda}
\fancyfoot[LE]{\thepage}
\fancyfoot[RE]{Escuela Técnica Superior de Ingenieros Industriales (UPM)}
\fancyfoot[LO]{Luis D. Aranda Sánchez}
\fancyfoot[RO]{\thepage}
\renewcommand{\headrulewidth}{0.4pt}
\renewcommand{\footrulewidth}{0.4pt}

\fancypagestyle{myfancy}{
    \fancyhf{} % Clear all headers and footers
    \fancyhead[LE]{\nouppercase{\leftmark}}
    \fancyhead[RO]{Optimización energética para vivienda}
    \fancyfoot[LE]{\thepage}
    \fancyfoot[RE]{Escuela Técnica Superior de Ingenieros Industriales (UPM)}
    \fancyfoot[LO]{Luis D. Aranda Sánchez}
    \fancyfoot[RO]{\thepage}
    \renewcommand{\headrulewidth}{0.4pt}
    \renewcommand{\footrulewidth}{0.4pt}
}

\fancypagestyle{simple}{
    \fancyhf{} % Clear all headers and footers
    \renewcommand{\headrulewidth}{0pt}
    \renewcommand{\footrulewidth}{0pt}
}

% Line spacing
\setstretch{1.2}

% Document starts here
\begin{document}

% Portada
\begin{titlepage}
    \centering
    {\scshape\LARGE Universidad Politécnica de Madrid \par}
    \vspace{1cm}
    {\scshape\Large Escuela Técnica Superior de Ingenieros Industriales\par}
    \vspace{1.5cm}
    {\huge\bfseries Optimización energética de sistema híbrido con bomba de calor, suelo radiante, fotovoltaica y almacenamiento para vivienda \par}
    \vspace{1.5cm}
    {\Large\bfseries Trabajo de Fin de Máster\par}
    \vspace{0.5cm}
    {\large Máster Universitario en Ingeniería de la Energía \par}
    \vspace{2cm}
    {\Large Luis D. Aranda Sánchez\par}
    \vfill
    Director: Javier Rodríguez Martín
    \vfill
    {\large Septiembre 6, 2024\par}
\end{titlepage}

% Resumen (máximo de 5 páginas, incluyendo al final Palabras clave)
\clearpage
\pagestyle{simple}
% \newpage
\chapter*{Resumen}
\addcontentsline{toc}{chapter}{Resumen}
\input{capitulos/resumen/main.tex}

% Índice (paginado)
\clearpage
\pagestyle{simple}
% \newpage
\tableofcontents

% Introducción (donde se incluya los antecedentes y justificación)
\clearpage
\pagestyle{myfancy}
\newpage
\chapter{Introducción}
\input{capitulos/introduccion/main.tex}

% Objetivos
\chapter{Objetivos}
\input{capitulos/objetivos/main.tex}

% Metodología
\chapter{Metodología}
\input{capitulos/metodologia/main.tex}

% Resultados y discusión (incluyendo la valoración de impactos y de aspectos de responsabilidad legal, ética y profesional relacionados con el trabajo)
\chapter{Resultados y Discusión}
\input{capitulos/resultados_discusion/main.tex}

% Conclusiones
\chapter{Conclusiones}
\input{capitulos/conclusiones/main.tex}

% Planificación temporal y presupuesto
\chapter{Planificación Temporal y Presupuesto}
\input{capitulos/planificacion_presupuesto/main.tex}

% Bibliografía
\newpage
\addcontentsline{toc}{chapter}{Bibliografía}
\printbibliography

\end{document}


% Índice (paginado)
\clearpage
\pagestyle{simple}
% \newpage
\tableofcontents

% Introducción (donde se incluya los antecedentes y justificación)
\clearpage
\pagestyle{myfancy}
\newpage
\chapter{Introducción}
\documentclass[a4paper,11pt,twoside]{report}
\usepackage[left=25mm,right=25mm,top=25mm,bottom=25mm,includehead,includefoot,headsep=15mm,footskip=15mm]{geometry}
\usepackage{graphicx}
\usepackage{fancyhdr}
\usepackage{titlesec}
\usepackage[spanish]{babel}
\usepackage[utf8]{inputenc}
\usepackage{amsmath}
\usepackage{setspace}
\usepackage{svg}
\usepackage{hyperref}
\usepackage[backend=biber,style=numeric]{biblatex}
\addbibresource{references.bib}
\hypersetup{
    colorlinks=true,
    linkcolor=blue,      % color of internal links (sections, etc.)
    urlcolor=blue,       % color of external links
    pdftitle={Optimización energética de sistema híbrido con bomba de calor, suelo radiante, fotovoltaica y almacenamiento para vivienda},    % title
    pdfauthor={Luis D. Aranda Sánchez},     % author
    pdfkeywords={palabra1, palabra2, código1, etc.} % list of keywords
}

% Font change to Arial
\usepackage{helvet}
\renewcommand{\familydefault}{\sfdefault}

% Chapter titles in uppercase and larger font
\titleformat{\chapter}[hang]{\large\bfseries}{\thechapter.}{1em}{\MakeUppercase}
\titleformat{\section}[hang]{\bfseries}{\thesection.}{1em}{}
\titleformat{\subsection}[hang]{\bfseries}{\thesubsection.}{1em}{}

% Fancyhdr setup
\setlength{\headheight}{14.30174pt} % Adjust to recommended value, slightly larger for safety
\fancyhf{} % Clear all headers and footers
\fancyhead[LE]{\nouppercase{\leftmark}}
\fancyhead[RO]{Optimización energética para vivienda}
\fancyfoot[LE]{\thepage}
\fancyfoot[RE]{Escuela Técnica Superior de Ingenieros Industriales (UPM)}
\fancyfoot[LO]{Luis D. Aranda Sánchez}
\fancyfoot[RO]{\thepage}
\renewcommand{\headrulewidth}{0.4pt}
\renewcommand{\footrulewidth}{0.4pt}

\fancypagestyle{myfancy}{
    \fancyhf{} % Clear all headers and footers
    \fancyhead[LE]{\nouppercase{\leftmark}}
    \fancyhead[RO]{Optimización energética para vivienda}
    \fancyfoot[LE]{\thepage}
    \fancyfoot[RE]{Escuela Técnica Superior de Ingenieros Industriales (UPM)}
    \fancyfoot[LO]{Luis D. Aranda Sánchez}
    \fancyfoot[RO]{\thepage}
    \renewcommand{\headrulewidth}{0.4pt}
    \renewcommand{\footrulewidth}{0.4pt}
}

\fancypagestyle{simple}{
    \fancyhf{} % Clear all headers and footers
    \renewcommand{\headrulewidth}{0pt}
    \renewcommand{\footrulewidth}{0pt}
}

% Line spacing
\setstretch{1.2}

% Document starts here
\begin{document}

% Portada
\begin{titlepage}
    \centering
    {\scshape\LARGE Universidad Politécnica de Madrid \par}
    \vspace{1cm}
    {\scshape\Large Escuela Técnica Superior de Ingenieros Industriales\par}
    \vspace{1.5cm}
    {\huge\bfseries Optimización energética de sistema híbrido con bomba de calor, suelo radiante, fotovoltaica y almacenamiento para vivienda \par}
    \vspace{1.5cm}
    {\Large\bfseries Trabajo de Fin de Máster\par}
    \vspace{0.5cm}
    {\large Máster Universitario en Ingeniería de la Energía \par}
    \vspace{2cm}
    {\Large Luis D. Aranda Sánchez\par}
    \vfill
    Director: Javier Rodríguez Martín
    \vfill
    {\large Septiembre 6, 2024\par}
\end{titlepage}

% Resumen (máximo de 5 páginas, incluyendo al final Palabras clave)
\clearpage
\pagestyle{simple}
% \newpage
\chapter*{Resumen}
\addcontentsline{toc}{chapter}{Resumen}
\input{capitulos/resumen/main.tex}

% Índice (paginado)
\clearpage
\pagestyle{simple}
% \newpage
\tableofcontents

% Introducción (donde se incluya los antecedentes y justificación)
\clearpage
\pagestyle{myfancy}
\newpage
\chapter{Introducción}
\input{capitulos/introduccion/main.tex}

% Objetivos
\chapter{Objetivos}
\input{capitulos/objetivos/main.tex}

% Metodología
\chapter{Metodología}
\input{capitulos/metodologia/main.tex}

% Resultados y discusión (incluyendo la valoración de impactos y de aspectos de responsabilidad legal, ética y profesional relacionados con el trabajo)
\chapter{Resultados y Discusión}
\input{capitulos/resultados_discusion/main.tex}

% Conclusiones
\chapter{Conclusiones}
\input{capitulos/conclusiones/main.tex}

% Planificación temporal y presupuesto
\chapter{Planificación Temporal y Presupuesto}
\input{capitulos/planificacion_presupuesto/main.tex}

% Bibliografía
\newpage
\addcontentsline{toc}{chapter}{Bibliografía}
\printbibliography

\end{document}


% Objetivos
\chapter{Objetivos}
\documentclass[a4paper,11pt,twoside]{report}
\usepackage[left=25mm,right=25mm,top=25mm,bottom=25mm,includehead,includefoot,headsep=15mm,footskip=15mm]{geometry}
\usepackage{graphicx}
\usepackage{fancyhdr}
\usepackage{titlesec}
\usepackage[spanish]{babel}
\usepackage[utf8]{inputenc}
\usepackage{amsmath}
\usepackage{setspace}
\usepackage{svg}
\usepackage{hyperref}
\usepackage[backend=biber,style=numeric]{biblatex}
\addbibresource{references.bib}
\hypersetup{
    colorlinks=true,
    linkcolor=blue,      % color of internal links (sections, etc.)
    urlcolor=blue,       % color of external links
    pdftitle={Optimización energética de sistema híbrido con bomba de calor, suelo radiante, fotovoltaica y almacenamiento para vivienda},    % title
    pdfauthor={Luis D. Aranda Sánchez},     % author
    pdfkeywords={palabra1, palabra2, código1, etc.} % list of keywords
}

% Font change to Arial
\usepackage{helvet}
\renewcommand{\familydefault}{\sfdefault}

% Chapter titles in uppercase and larger font
\titleformat{\chapter}[hang]{\large\bfseries}{\thechapter.}{1em}{\MakeUppercase}
\titleformat{\section}[hang]{\bfseries}{\thesection.}{1em}{}
\titleformat{\subsection}[hang]{\bfseries}{\thesubsection.}{1em}{}

% Fancyhdr setup
\setlength{\headheight}{14.30174pt} % Adjust to recommended value, slightly larger for safety
\fancyhf{} % Clear all headers and footers
\fancyhead[LE]{\nouppercase{\leftmark}}
\fancyhead[RO]{Optimización energética para vivienda}
\fancyfoot[LE]{\thepage}
\fancyfoot[RE]{Escuela Técnica Superior de Ingenieros Industriales (UPM)}
\fancyfoot[LO]{Luis D. Aranda Sánchez}
\fancyfoot[RO]{\thepage}
\renewcommand{\headrulewidth}{0.4pt}
\renewcommand{\footrulewidth}{0.4pt}

\fancypagestyle{myfancy}{
    \fancyhf{} % Clear all headers and footers
    \fancyhead[LE]{\nouppercase{\leftmark}}
    \fancyhead[RO]{Optimización energética para vivienda}
    \fancyfoot[LE]{\thepage}
    \fancyfoot[RE]{Escuela Técnica Superior de Ingenieros Industriales (UPM)}
    \fancyfoot[LO]{Luis D. Aranda Sánchez}
    \fancyfoot[RO]{\thepage}
    \renewcommand{\headrulewidth}{0.4pt}
    \renewcommand{\footrulewidth}{0.4pt}
}

\fancypagestyle{simple}{
    \fancyhf{} % Clear all headers and footers
    \renewcommand{\headrulewidth}{0pt}
    \renewcommand{\footrulewidth}{0pt}
}

% Line spacing
\setstretch{1.2}

% Document starts here
\begin{document}

% Portada
\begin{titlepage}
    \centering
    {\scshape\LARGE Universidad Politécnica de Madrid \par}
    \vspace{1cm}
    {\scshape\Large Escuela Técnica Superior de Ingenieros Industriales\par}
    \vspace{1.5cm}
    {\huge\bfseries Optimización energética de sistema híbrido con bomba de calor, suelo radiante, fotovoltaica y almacenamiento para vivienda \par}
    \vspace{1.5cm}
    {\Large\bfseries Trabajo de Fin de Máster\par}
    \vspace{0.5cm}
    {\large Máster Universitario en Ingeniería de la Energía \par}
    \vspace{2cm}
    {\Large Luis D. Aranda Sánchez\par}
    \vfill
    Director: Javier Rodríguez Martín
    \vfill
    {\large Septiembre 6, 2024\par}
\end{titlepage}

% Resumen (máximo de 5 páginas, incluyendo al final Palabras clave)
\clearpage
\pagestyle{simple}
% \newpage
\chapter*{Resumen}
\addcontentsline{toc}{chapter}{Resumen}
\input{capitulos/resumen/main.tex}

% Índice (paginado)
\clearpage
\pagestyle{simple}
% \newpage
\tableofcontents

% Introducción (donde se incluya los antecedentes y justificación)
\clearpage
\pagestyle{myfancy}
\newpage
\chapter{Introducción}
\input{capitulos/introduccion/main.tex}

% Objetivos
\chapter{Objetivos}
\input{capitulos/objetivos/main.tex}

% Metodología
\chapter{Metodología}
\input{capitulos/metodologia/main.tex}

% Resultados y discusión (incluyendo la valoración de impactos y de aspectos de responsabilidad legal, ética y profesional relacionados con el trabajo)
\chapter{Resultados y Discusión}
\input{capitulos/resultados_discusion/main.tex}

% Conclusiones
\chapter{Conclusiones}
\input{capitulos/conclusiones/main.tex}

% Planificación temporal y presupuesto
\chapter{Planificación Temporal y Presupuesto}
\input{capitulos/planificacion_presupuesto/main.tex}

% Bibliografía
\newpage
\addcontentsline{toc}{chapter}{Bibliografía}
\printbibliography

\end{document}


% Metodología
\chapter{Metodología}
\documentclass[a4paper,11pt,twoside]{report}
\usepackage[left=25mm,right=25mm,top=25mm,bottom=25mm,includehead,includefoot,headsep=15mm,footskip=15mm]{geometry}
\usepackage{graphicx}
\usepackage{fancyhdr}
\usepackage{titlesec}
\usepackage[spanish]{babel}
\usepackage[utf8]{inputenc}
\usepackage{amsmath}
\usepackage{setspace}
\usepackage{svg}
\usepackage{hyperref}
\usepackage[backend=biber,style=numeric]{biblatex}
\addbibresource{references.bib}
\hypersetup{
    colorlinks=true,
    linkcolor=blue,      % color of internal links (sections, etc.)
    urlcolor=blue,       % color of external links
    pdftitle={Optimización energética de sistema híbrido con bomba de calor, suelo radiante, fotovoltaica y almacenamiento para vivienda},    % title
    pdfauthor={Luis D. Aranda Sánchez},     % author
    pdfkeywords={palabra1, palabra2, código1, etc.} % list of keywords
}

% Font change to Arial
\usepackage{helvet}
\renewcommand{\familydefault}{\sfdefault}

% Chapter titles in uppercase and larger font
\titleformat{\chapter}[hang]{\large\bfseries}{\thechapter.}{1em}{\MakeUppercase}
\titleformat{\section}[hang]{\bfseries}{\thesection.}{1em}{}
\titleformat{\subsection}[hang]{\bfseries}{\thesubsection.}{1em}{}

% Fancyhdr setup
\setlength{\headheight}{14.30174pt} % Adjust to recommended value, slightly larger for safety
\fancyhf{} % Clear all headers and footers
\fancyhead[LE]{\nouppercase{\leftmark}}
\fancyhead[RO]{Optimización energética para vivienda}
\fancyfoot[LE]{\thepage}
\fancyfoot[RE]{Escuela Técnica Superior de Ingenieros Industriales (UPM)}
\fancyfoot[LO]{Luis D. Aranda Sánchez}
\fancyfoot[RO]{\thepage}
\renewcommand{\headrulewidth}{0.4pt}
\renewcommand{\footrulewidth}{0.4pt}

\fancypagestyle{myfancy}{
    \fancyhf{} % Clear all headers and footers
    \fancyhead[LE]{\nouppercase{\leftmark}}
    \fancyhead[RO]{Optimización energética para vivienda}
    \fancyfoot[LE]{\thepage}
    \fancyfoot[RE]{Escuela Técnica Superior de Ingenieros Industriales (UPM)}
    \fancyfoot[LO]{Luis D. Aranda Sánchez}
    \fancyfoot[RO]{\thepage}
    \renewcommand{\headrulewidth}{0.4pt}
    \renewcommand{\footrulewidth}{0.4pt}
}

\fancypagestyle{simple}{
    \fancyhf{} % Clear all headers and footers
    \renewcommand{\headrulewidth}{0pt}
    \renewcommand{\footrulewidth}{0pt}
}

% Line spacing
\setstretch{1.2}

% Document starts here
\begin{document}

% Portada
\begin{titlepage}
    \centering
    {\scshape\LARGE Universidad Politécnica de Madrid \par}
    \vspace{1cm}
    {\scshape\Large Escuela Técnica Superior de Ingenieros Industriales\par}
    \vspace{1.5cm}
    {\huge\bfseries Optimización energética de sistema híbrido con bomba de calor, suelo radiante, fotovoltaica y almacenamiento para vivienda \par}
    \vspace{1.5cm}
    {\Large\bfseries Trabajo de Fin de Máster\par}
    \vspace{0.5cm}
    {\large Máster Universitario en Ingeniería de la Energía \par}
    \vspace{2cm}
    {\Large Luis D. Aranda Sánchez\par}
    \vfill
    Director: Javier Rodríguez Martín
    \vfill
    {\large Septiembre 6, 2024\par}
\end{titlepage}

% Resumen (máximo de 5 páginas, incluyendo al final Palabras clave)
\clearpage
\pagestyle{simple}
% \newpage
\chapter*{Resumen}
\addcontentsline{toc}{chapter}{Resumen}
\input{capitulos/resumen/main.tex}

% Índice (paginado)
\clearpage
\pagestyle{simple}
% \newpage
\tableofcontents

% Introducción (donde se incluya los antecedentes y justificación)
\clearpage
\pagestyle{myfancy}
\newpage
\chapter{Introducción}
\input{capitulos/introduccion/main.tex}

% Objetivos
\chapter{Objetivos}
\input{capitulos/objetivos/main.tex}

% Metodología
\chapter{Metodología}
\input{capitulos/metodologia/main.tex}

% Resultados y discusión (incluyendo la valoración de impactos y de aspectos de responsabilidad legal, ética y profesional relacionados con el trabajo)
\chapter{Resultados y Discusión}
\input{capitulos/resultados_discusion/main.tex}

% Conclusiones
\chapter{Conclusiones}
\input{capitulos/conclusiones/main.tex}

% Planificación temporal y presupuesto
\chapter{Planificación Temporal y Presupuesto}
\input{capitulos/planificacion_presupuesto/main.tex}

% Bibliografía
\newpage
\addcontentsline{toc}{chapter}{Bibliografía}
\printbibliography

\end{document}


% Resultados y discusión (incluyendo la valoración de impactos y de aspectos de responsabilidad legal, ética y profesional relacionados con el trabajo)
\chapter{Resultados y Discusión}
\documentclass[a4paper,11pt,twoside]{report}
\usepackage[left=25mm,right=25mm,top=25mm,bottom=25mm,includehead,includefoot,headsep=15mm,footskip=15mm]{geometry}
\usepackage{graphicx}
\usepackage{fancyhdr}
\usepackage{titlesec}
\usepackage[spanish]{babel}
\usepackage[utf8]{inputenc}
\usepackage{amsmath}
\usepackage{setspace}
\usepackage{svg}
\usepackage{hyperref}
\usepackage[backend=biber,style=numeric]{biblatex}
\addbibresource{references.bib}
\hypersetup{
    colorlinks=true,
    linkcolor=blue,      % color of internal links (sections, etc.)
    urlcolor=blue,       % color of external links
    pdftitle={Optimización energética de sistema híbrido con bomba de calor, suelo radiante, fotovoltaica y almacenamiento para vivienda},    % title
    pdfauthor={Luis D. Aranda Sánchez},     % author
    pdfkeywords={palabra1, palabra2, código1, etc.} % list of keywords
}

% Font change to Arial
\usepackage{helvet}
\renewcommand{\familydefault}{\sfdefault}

% Chapter titles in uppercase and larger font
\titleformat{\chapter}[hang]{\large\bfseries}{\thechapter.}{1em}{\MakeUppercase}
\titleformat{\section}[hang]{\bfseries}{\thesection.}{1em}{}
\titleformat{\subsection}[hang]{\bfseries}{\thesubsection.}{1em}{}

% Fancyhdr setup
\setlength{\headheight}{14.30174pt} % Adjust to recommended value, slightly larger for safety
\fancyhf{} % Clear all headers and footers
\fancyhead[LE]{\nouppercase{\leftmark}}
\fancyhead[RO]{Optimización energética para vivienda}
\fancyfoot[LE]{\thepage}
\fancyfoot[RE]{Escuela Técnica Superior de Ingenieros Industriales (UPM)}
\fancyfoot[LO]{Luis D. Aranda Sánchez}
\fancyfoot[RO]{\thepage}
\renewcommand{\headrulewidth}{0.4pt}
\renewcommand{\footrulewidth}{0.4pt}

\fancypagestyle{myfancy}{
    \fancyhf{} % Clear all headers and footers
    \fancyhead[LE]{\nouppercase{\leftmark}}
    \fancyhead[RO]{Optimización energética para vivienda}
    \fancyfoot[LE]{\thepage}
    \fancyfoot[RE]{Escuela Técnica Superior de Ingenieros Industriales (UPM)}
    \fancyfoot[LO]{Luis D. Aranda Sánchez}
    \fancyfoot[RO]{\thepage}
    \renewcommand{\headrulewidth}{0.4pt}
    \renewcommand{\footrulewidth}{0.4pt}
}

\fancypagestyle{simple}{
    \fancyhf{} % Clear all headers and footers
    \renewcommand{\headrulewidth}{0pt}
    \renewcommand{\footrulewidth}{0pt}
}

% Line spacing
\setstretch{1.2}

% Document starts here
\begin{document}

% Portada
\begin{titlepage}
    \centering
    {\scshape\LARGE Universidad Politécnica de Madrid \par}
    \vspace{1cm}
    {\scshape\Large Escuela Técnica Superior de Ingenieros Industriales\par}
    \vspace{1.5cm}
    {\huge\bfseries Optimización energética de sistema híbrido con bomba de calor, suelo radiante, fotovoltaica y almacenamiento para vivienda \par}
    \vspace{1.5cm}
    {\Large\bfseries Trabajo de Fin de Máster\par}
    \vspace{0.5cm}
    {\large Máster Universitario en Ingeniería de la Energía \par}
    \vspace{2cm}
    {\Large Luis D. Aranda Sánchez\par}
    \vfill
    Director: Javier Rodríguez Martín
    \vfill
    {\large Septiembre 6, 2024\par}
\end{titlepage}

% Resumen (máximo de 5 páginas, incluyendo al final Palabras clave)
\clearpage
\pagestyle{simple}
% \newpage
\chapter*{Resumen}
\addcontentsline{toc}{chapter}{Resumen}
\input{capitulos/resumen/main.tex}

% Índice (paginado)
\clearpage
\pagestyle{simple}
% \newpage
\tableofcontents

% Introducción (donde se incluya los antecedentes y justificación)
\clearpage
\pagestyle{myfancy}
\newpage
\chapter{Introducción}
\input{capitulos/introduccion/main.tex}

% Objetivos
\chapter{Objetivos}
\input{capitulos/objetivos/main.tex}

% Metodología
\chapter{Metodología}
\input{capitulos/metodologia/main.tex}

% Resultados y discusión (incluyendo la valoración de impactos y de aspectos de responsabilidad legal, ética y profesional relacionados con el trabajo)
\chapter{Resultados y Discusión}
\input{capitulos/resultados_discusion/main.tex}

% Conclusiones
\chapter{Conclusiones}
\input{capitulos/conclusiones/main.tex}

% Planificación temporal y presupuesto
\chapter{Planificación Temporal y Presupuesto}
\input{capitulos/planificacion_presupuesto/main.tex}

% Bibliografía
\newpage
\addcontentsline{toc}{chapter}{Bibliografía}
\printbibliography

\end{document}


% Conclusiones
\chapter{Conclusiones}
\documentclass[a4paper,11pt,twoside]{report}
\usepackage[left=25mm,right=25mm,top=25mm,bottom=25mm,includehead,includefoot,headsep=15mm,footskip=15mm]{geometry}
\usepackage{graphicx}
\usepackage{fancyhdr}
\usepackage{titlesec}
\usepackage[spanish]{babel}
\usepackage[utf8]{inputenc}
\usepackage{amsmath}
\usepackage{setspace}
\usepackage{svg}
\usepackage{hyperref}
\usepackage[backend=biber,style=numeric]{biblatex}
\addbibresource{references.bib}
\hypersetup{
    colorlinks=true,
    linkcolor=blue,      % color of internal links (sections, etc.)
    urlcolor=blue,       % color of external links
    pdftitle={Optimización energética de sistema híbrido con bomba de calor, suelo radiante, fotovoltaica y almacenamiento para vivienda},    % title
    pdfauthor={Luis D. Aranda Sánchez},     % author
    pdfkeywords={palabra1, palabra2, código1, etc.} % list of keywords
}

% Font change to Arial
\usepackage{helvet}
\renewcommand{\familydefault}{\sfdefault}

% Chapter titles in uppercase and larger font
\titleformat{\chapter}[hang]{\large\bfseries}{\thechapter.}{1em}{\MakeUppercase}
\titleformat{\section}[hang]{\bfseries}{\thesection.}{1em}{}
\titleformat{\subsection}[hang]{\bfseries}{\thesubsection.}{1em}{}

% Fancyhdr setup
\setlength{\headheight}{14.30174pt} % Adjust to recommended value, slightly larger for safety
\fancyhf{} % Clear all headers and footers
\fancyhead[LE]{\nouppercase{\leftmark}}
\fancyhead[RO]{Optimización energética para vivienda}
\fancyfoot[LE]{\thepage}
\fancyfoot[RE]{Escuela Técnica Superior de Ingenieros Industriales (UPM)}
\fancyfoot[LO]{Luis D. Aranda Sánchez}
\fancyfoot[RO]{\thepage}
\renewcommand{\headrulewidth}{0.4pt}
\renewcommand{\footrulewidth}{0.4pt}

\fancypagestyle{myfancy}{
    \fancyhf{} % Clear all headers and footers
    \fancyhead[LE]{\nouppercase{\leftmark}}
    \fancyhead[RO]{Optimización energética para vivienda}
    \fancyfoot[LE]{\thepage}
    \fancyfoot[RE]{Escuela Técnica Superior de Ingenieros Industriales (UPM)}
    \fancyfoot[LO]{Luis D. Aranda Sánchez}
    \fancyfoot[RO]{\thepage}
    \renewcommand{\headrulewidth}{0.4pt}
    \renewcommand{\footrulewidth}{0.4pt}
}

\fancypagestyle{simple}{
    \fancyhf{} % Clear all headers and footers
    \renewcommand{\headrulewidth}{0pt}
    \renewcommand{\footrulewidth}{0pt}
}

% Line spacing
\setstretch{1.2}

% Document starts here
\begin{document}

% Portada
\begin{titlepage}
    \centering
    {\scshape\LARGE Universidad Politécnica de Madrid \par}
    \vspace{1cm}
    {\scshape\Large Escuela Técnica Superior de Ingenieros Industriales\par}
    \vspace{1.5cm}
    {\huge\bfseries Optimización energética de sistema híbrido con bomba de calor, suelo radiante, fotovoltaica y almacenamiento para vivienda \par}
    \vspace{1.5cm}
    {\Large\bfseries Trabajo de Fin de Máster\par}
    \vspace{0.5cm}
    {\large Máster Universitario en Ingeniería de la Energía \par}
    \vspace{2cm}
    {\Large Luis D. Aranda Sánchez\par}
    \vfill
    Director: Javier Rodríguez Martín
    \vfill
    {\large Septiembre 6, 2024\par}
\end{titlepage}

% Resumen (máximo de 5 páginas, incluyendo al final Palabras clave)
\clearpage
\pagestyle{simple}
% \newpage
\chapter*{Resumen}
\addcontentsline{toc}{chapter}{Resumen}
\input{capitulos/resumen/main.tex}

% Índice (paginado)
\clearpage
\pagestyle{simple}
% \newpage
\tableofcontents

% Introducción (donde se incluya los antecedentes y justificación)
\clearpage
\pagestyle{myfancy}
\newpage
\chapter{Introducción}
\input{capitulos/introduccion/main.tex}

% Objetivos
\chapter{Objetivos}
\input{capitulos/objetivos/main.tex}

% Metodología
\chapter{Metodología}
\input{capitulos/metodologia/main.tex}

% Resultados y discusión (incluyendo la valoración de impactos y de aspectos de responsabilidad legal, ética y profesional relacionados con el trabajo)
\chapter{Resultados y Discusión}
\input{capitulos/resultados_discusion/main.tex}

% Conclusiones
\chapter{Conclusiones}
\input{capitulos/conclusiones/main.tex}

% Planificación temporal y presupuesto
\chapter{Planificación Temporal y Presupuesto}
\input{capitulos/planificacion_presupuesto/main.tex}

% Bibliografía
\newpage
\addcontentsline{toc}{chapter}{Bibliografía}
\printbibliography

\end{document}


% Planificación temporal y presupuesto
\chapter{Planificación Temporal y Presupuesto}
\documentclass[a4paper,11pt,twoside]{report}
\usepackage[left=25mm,right=25mm,top=25mm,bottom=25mm,includehead,includefoot,headsep=15mm,footskip=15mm]{geometry}
\usepackage{graphicx}
\usepackage{fancyhdr}
\usepackage{titlesec}
\usepackage[spanish]{babel}
\usepackage[utf8]{inputenc}
\usepackage{amsmath}
\usepackage{setspace}
\usepackage{svg}
\usepackage{hyperref}
\usepackage[backend=biber,style=numeric]{biblatex}
\addbibresource{references.bib}
\hypersetup{
    colorlinks=true,
    linkcolor=blue,      % color of internal links (sections, etc.)
    urlcolor=blue,       % color of external links
    pdftitle={Optimización energética de sistema híbrido con bomba de calor, suelo radiante, fotovoltaica y almacenamiento para vivienda},    % title
    pdfauthor={Luis D. Aranda Sánchez},     % author
    pdfkeywords={palabra1, palabra2, código1, etc.} % list of keywords
}

% Font change to Arial
\usepackage{helvet}
\renewcommand{\familydefault}{\sfdefault}

% Chapter titles in uppercase and larger font
\titleformat{\chapter}[hang]{\large\bfseries}{\thechapter.}{1em}{\MakeUppercase}
\titleformat{\section}[hang]{\bfseries}{\thesection.}{1em}{}
\titleformat{\subsection}[hang]{\bfseries}{\thesubsection.}{1em}{}

% Fancyhdr setup
\setlength{\headheight}{14.30174pt} % Adjust to recommended value, slightly larger for safety
\fancyhf{} % Clear all headers and footers
\fancyhead[LE]{\nouppercase{\leftmark}}
\fancyhead[RO]{Optimización energética para vivienda}
\fancyfoot[LE]{\thepage}
\fancyfoot[RE]{Escuela Técnica Superior de Ingenieros Industriales (UPM)}
\fancyfoot[LO]{Luis D. Aranda Sánchez}
\fancyfoot[RO]{\thepage}
\renewcommand{\headrulewidth}{0.4pt}
\renewcommand{\footrulewidth}{0.4pt}

\fancypagestyle{myfancy}{
    \fancyhf{} % Clear all headers and footers
    \fancyhead[LE]{\nouppercase{\leftmark}}
    \fancyhead[RO]{Optimización energética para vivienda}
    \fancyfoot[LE]{\thepage}
    \fancyfoot[RE]{Escuela Técnica Superior de Ingenieros Industriales (UPM)}
    \fancyfoot[LO]{Luis D. Aranda Sánchez}
    \fancyfoot[RO]{\thepage}
    \renewcommand{\headrulewidth}{0.4pt}
    \renewcommand{\footrulewidth}{0.4pt}
}

\fancypagestyle{simple}{
    \fancyhf{} % Clear all headers and footers
    \renewcommand{\headrulewidth}{0pt}
    \renewcommand{\footrulewidth}{0pt}
}

% Line spacing
\setstretch{1.2}

% Document starts here
\begin{document}

% Portada
\begin{titlepage}
    \centering
    {\scshape\LARGE Universidad Politécnica de Madrid \par}
    \vspace{1cm}
    {\scshape\Large Escuela Técnica Superior de Ingenieros Industriales\par}
    \vspace{1.5cm}
    {\huge\bfseries Optimización energética de sistema híbrido con bomba de calor, suelo radiante, fotovoltaica y almacenamiento para vivienda \par}
    \vspace{1.5cm}
    {\Large\bfseries Trabajo de Fin de Máster\par}
    \vspace{0.5cm}
    {\large Máster Universitario en Ingeniería de la Energía \par}
    \vspace{2cm}
    {\Large Luis D. Aranda Sánchez\par}
    \vfill
    Director: Javier Rodríguez Martín
    \vfill
    {\large Septiembre 6, 2024\par}
\end{titlepage}

% Resumen (máximo de 5 páginas, incluyendo al final Palabras clave)
\clearpage
\pagestyle{simple}
% \newpage
\chapter*{Resumen}
\addcontentsline{toc}{chapter}{Resumen}
\input{capitulos/resumen/main.tex}

% Índice (paginado)
\clearpage
\pagestyle{simple}
% \newpage
\tableofcontents

% Introducción (donde se incluya los antecedentes y justificación)
\clearpage
\pagestyle{myfancy}
\newpage
\chapter{Introducción}
\input{capitulos/introduccion/main.tex}

% Objetivos
\chapter{Objetivos}
\input{capitulos/objetivos/main.tex}

% Metodología
\chapter{Metodología}
\input{capitulos/metodologia/main.tex}

% Resultados y discusión (incluyendo la valoración de impactos y de aspectos de responsabilidad legal, ética y profesional relacionados con el trabajo)
\chapter{Resultados y Discusión}
\input{capitulos/resultados_discusion/main.tex}

% Conclusiones
\chapter{Conclusiones}
\input{capitulos/conclusiones/main.tex}

% Planificación temporal y presupuesto
\chapter{Planificación Temporal y Presupuesto}
\input{capitulos/planificacion_presupuesto/main.tex}

% Bibliografía
\newpage
\addcontentsline{toc}{chapter}{Bibliografía}
\printbibliography

\end{document}


% Bibliografía
\newpage
\addcontentsline{toc}{chapter}{Bibliografía}
\printbibliography

\end{document}


% Conclusiones
\chapter{Conclusiones}
\documentclass[a4paper,11pt,twoside]{report}
\usepackage[left=25mm,right=25mm,top=25mm,bottom=25mm,includehead,includefoot,headsep=15mm,footskip=15mm]{geometry}
\usepackage{graphicx}
\usepackage{fancyhdr}
\usepackage{titlesec}
\usepackage[spanish]{babel}
\usepackage[utf8]{inputenc}
\usepackage{amsmath}
\usepackage{setspace}
\usepackage{svg}
\usepackage{hyperref}
\usepackage[backend=biber,style=numeric]{biblatex}
\addbibresource{references.bib}
\hypersetup{
    colorlinks=true,
    linkcolor=blue,      % color of internal links (sections, etc.)
    urlcolor=blue,       % color of external links
    pdftitle={Optimización energética de sistema híbrido con bomba de calor, suelo radiante, fotovoltaica y almacenamiento para vivienda},    % title
    pdfauthor={Luis D. Aranda Sánchez},     % author
    pdfkeywords={palabra1, palabra2, código1, etc.} % list of keywords
}

% Font change to Arial
\usepackage{helvet}
\renewcommand{\familydefault}{\sfdefault}

% Chapter titles in uppercase and larger font
\titleformat{\chapter}[hang]{\large\bfseries}{\thechapter.}{1em}{\MakeUppercase}
\titleformat{\section}[hang]{\bfseries}{\thesection.}{1em}{}
\titleformat{\subsection}[hang]{\bfseries}{\thesubsection.}{1em}{}

% Fancyhdr setup
\setlength{\headheight}{14.30174pt} % Adjust to recommended value, slightly larger for safety
\fancyhf{} % Clear all headers and footers
\fancyhead[LE]{\nouppercase{\leftmark}}
\fancyhead[RO]{Optimización energética para vivienda}
\fancyfoot[LE]{\thepage}
\fancyfoot[RE]{Escuela Técnica Superior de Ingenieros Industriales (UPM)}
\fancyfoot[LO]{Luis D. Aranda Sánchez}
\fancyfoot[RO]{\thepage}
\renewcommand{\headrulewidth}{0.4pt}
\renewcommand{\footrulewidth}{0.4pt}

\fancypagestyle{myfancy}{
    \fancyhf{} % Clear all headers and footers
    \fancyhead[LE]{\nouppercase{\leftmark}}
    \fancyhead[RO]{Optimización energética para vivienda}
    \fancyfoot[LE]{\thepage}
    \fancyfoot[RE]{Escuela Técnica Superior de Ingenieros Industriales (UPM)}
    \fancyfoot[LO]{Luis D. Aranda Sánchez}
    \fancyfoot[RO]{\thepage}
    \renewcommand{\headrulewidth}{0.4pt}
    \renewcommand{\footrulewidth}{0.4pt}
}

\fancypagestyle{simple}{
    \fancyhf{} % Clear all headers and footers
    \renewcommand{\headrulewidth}{0pt}
    \renewcommand{\footrulewidth}{0pt}
}

% Line spacing
\setstretch{1.2}

% Document starts here
\begin{document}

% Portada
\begin{titlepage}
    \centering
    {\scshape\LARGE Universidad Politécnica de Madrid \par}
    \vspace{1cm}
    {\scshape\Large Escuela Técnica Superior de Ingenieros Industriales\par}
    \vspace{1.5cm}
    {\huge\bfseries Optimización energética de sistema híbrido con bomba de calor, suelo radiante, fotovoltaica y almacenamiento para vivienda \par}
    \vspace{1.5cm}
    {\Large\bfseries Trabajo de Fin de Máster\par}
    \vspace{0.5cm}
    {\large Máster Universitario en Ingeniería de la Energía \par}
    \vspace{2cm}
    {\Large Luis D. Aranda Sánchez\par}
    \vfill
    Director: Javier Rodríguez Martín
    \vfill
    {\large Septiembre 6, 2024\par}
\end{titlepage}

% Resumen (máximo de 5 páginas, incluyendo al final Palabras clave)
\clearpage
\pagestyle{simple}
% \newpage
\chapter*{Resumen}
\addcontentsline{toc}{chapter}{Resumen}
\documentclass[a4paper,11pt,twoside]{report}
\usepackage[left=25mm,right=25mm,top=25mm,bottom=25mm,includehead,includefoot,headsep=15mm,footskip=15mm]{geometry}
\usepackage{graphicx}
\usepackage{fancyhdr}
\usepackage{titlesec}
\usepackage[spanish]{babel}
\usepackage[utf8]{inputenc}
\usepackage{amsmath}
\usepackage{setspace}
\usepackage{svg}
\usepackage{hyperref}
\usepackage[backend=biber,style=numeric]{biblatex}
\addbibresource{references.bib}
\hypersetup{
    colorlinks=true,
    linkcolor=blue,      % color of internal links (sections, etc.)
    urlcolor=blue,       % color of external links
    pdftitle={Optimización energética de sistema híbrido con bomba de calor, suelo radiante, fotovoltaica y almacenamiento para vivienda},    % title
    pdfauthor={Luis D. Aranda Sánchez},     % author
    pdfkeywords={palabra1, palabra2, código1, etc.} % list of keywords
}

% Font change to Arial
\usepackage{helvet}
\renewcommand{\familydefault}{\sfdefault}

% Chapter titles in uppercase and larger font
\titleformat{\chapter}[hang]{\large\bfseries}{\thechapter.}{1em}{\MakeUppercase}
\titleformat{\section}[hang]{\bfseries}{\thesection.}{1em}{}
\titleformat{\subsection}[hang]{\bfseries}{\thesubsection.}{1em}{}

% Fancyhdr setup
\setlength{\headheight}{14.30174pt} % Adjust to recommended value, slightly larger for safety
\fancyhf{} % Clear all headers and footers
\fancyhead[LE]{\nouppercase{\leftmark}}
\fancyhead[RO]{Optimización energética para vivienda}
\fancyfoot[LE]{\thepage}
\fancyfoot[RE]{Escuela Técnica Superior de Ingenieros Industriales (UPM)}
\fancyfoot[LO]{Luis D. Aranda Sánchez}
\fancyfoot[RO]{\thepage}
\renewcommand{\headrulewidth}{0.4pt}
\renewcommand{\footrulewidth}{0.4pt}

\fancypagestyle{myfancy}{
    \fancyhf{} % Clear all headers and footers
    \fancyhead[LE]{\nouppercase{\leftmark}}
    \fancyhead[RO]{Optimización energética para vivienda}
    \fancyfoot[LE]{\thepage}
    \fancyfoot[RE]{Escuela Técnica Superior de Ingenieros Industriales (UPM)}
    \fancyfoot[LO]{Luis D. Aranda Sánchez}
    \fancyfoot[RO]{\thepage}
    \renewcommand{\headrulewidth}{0.4pt}
    \renewcommand{\footrulewidth}{0.4pt}
}

\fancypagestyle{simple}{
    \fancyhf{} % Clear all headers and footers
    \renewcommand{\headrulewidth}{0pt}
    \renewcommand{\footrulewidth}{0pt}
}

% Line spacing
\setstretch{1.2}

% Document starts here
\begin{document}

% Portada
\begin{titlepage}
    \centering
    {\scshape\LARGE Universidad Politécnica de Madrid \par}
    \vspace{1cm}
    {\scshape\Large Escuela Técnica Superior de Ingenieros Industriales\par}
    \vspace{1.5cm}
    {\huge\bfseries Optimización energética de sistema híbrido con bomba de calor, suelo radiante, fotovoltaica y almacenamiento para vivienda \par}
    \vspace{1.5cm}
    {\Large\bfseries Trabajo de Fin de Máster\par}
    \vspace{0.5cm}
    {\large Máster Universitario en Ingeniería de la Energía \par}
    \vspace{2cm}
    {\Large Luis D. Aranda Sánchez\par}
    \vfill
    Director: Javier Rodríguez Martín
    \vfill
    {\large Septiembre 6, 2024\par}
\end{titlepage}

% Resumen (máximo de 5 páginas, incluyendo al final Palabras clave)
\clearpage
\pagestyle{simple}
% \newpage
\chapter*{Resumen}
\addcontentsline{toc}{chapter}{Resumen}
\input{capitulos/resumen/main.tex}

% Índice (paginado)
\clearpage
\pagestyle{simple}
% \newpage
\tableofcontents

% Introducción (donde se incluya los antecedentes y justificación)
\clearpage
\pagestyle{myfancy}
\newpage
\chapter{Introducción}
\input{capitulos/introduccion/main.tex}

% Objetivos
\chapter{Objetivos}
\input{capitulos/objetivos/main.tex}

% Metodología
\chapter{Metodología}
\input{capitulos/metodologia/main.tex}

% Resultados y discusión (incluyendo la valoración de impactos y de aspectos de responsabilidad legal, ética y profesional relacionados con el trabajo)
\chapter{Resultados y Discusión}
\input{capitulos/resultados_discusion/main.tex}

% Conclusiones
\chapter{Conclusiones}
\input{capitulos/conclusiones/main.tex}

% Planificación temporal y presupuesto
\chapter{Planificación Temporal y Presupuesto}
\input{capitulos/planificacion_presupuesto/main.tex}

% Bibliografía
\newpage
\addcontentsline{toc}{chapter}{Bibliografía}
\printbibliography

\end{document}


% Índice (paginado)
\clearpage
\pagestyle{simple}
% \newpage
\tableofcontents

% Introducción (donde se incluya los antecedentes y justificación)
\clearpage
\pagestyle{myfancy}
\newpage
\chapter{Introducción}
\documentclass[a4paper,11pt,twoside]{report}
\usepackage[left=25mm,right=25mm,top=25mm,bottom=25mm,includehead,includefoot,headsep=15mm,footskip=15mm]{geometry}
\usepackage{graphicx}
\usepackage{fancyhdr}
\usepackage{titlesec}
\usepackage[spanish]{babel}
\usepackage[utf8]{inputenc}
\usepackage{amsmath}
\usepackage{setspace}
\usepackage{svg}
\usepackage{hyperref}
\usepackage[backend=biber,style=numeric]{biblatex}
\addbibresource{references.bib}
\hypersetup{
    colorlinks=true,
    linkcolor=blue,      % color of internal links (sections, etc.)
    urlcolor=blue,       % color of external links
    pdftitle={Optimización energética de sistema híbrido con bomba de calor, suelo radiante, fotovoltaica y almacenamiento para vivienda},    % title
    pdfauthor={Luis D. Aranda Sánchez},     % author
    pdfkeywords={palabra1, palabra2, código1, etc.} % list of keywords
}

% Font change to Arial
\usepackage{helvet}
\renewcommand{\familydefault}{\sfdefault}

% Chapter titles in uppercase and larger font
\titleformat{\chapter}[hang]{\large\bfseries}{\thechapter.}{1em}{\MakeUppercase}
\titleformat{\section}[hang]{\bfseries}{\thesection.}{1em}{}
\titleformat{\subsection}[hang]{\bfseries}{\thesubsection.}{1em}{}

% Fancyhdr setup
\setlength{\headheight}{14.30174pt} % Adjust to recommended value, slightly larger for safety
\fancyhf{} % Clear all headers and footers
\fancyhead[LE]{\nouppercase{\leftmark}}
\fancyhead[RO]{Optimización energética para vivienda}
\fancyfoot[LE]{\thepage}
\fancyfoot[RE]{Escuela Técnica Superior de Ingenieros Industriales (UPM)}
\fancyfoot[LO]{Luis D. Aranda Sánchez}
\fancyfoot[RO]{\thepage}
\renewcommand{\headrulewidth}{0.4pt}
\renewcommand{\footrulewidth}{0.4pt}

\fancypagestyle{myfancy}{
    \fancyhf{} % Clear all headers and footers
    \fancyhead[LE]{\nouppercase{\leftmark}}
    \fancyhead[RO]{Optimización energética para vivienda}
    \fancyfoot[LE]{\thepage}
    \fancyfoot[RE]{Escuela Técnica Superior de Ingenieros Industriales (UPM)}
    \fancyfoot[LO]{Luis D. Aranda Sánchez}
    \fancyfoot[RO]{\thepage}
    \renewcommand{\headrulewidth}{0.4pt}
    \renewcommand{\footrulewidth}{0.4pt}
}

\fancypagestyle{simple}{
    \fancyhf{} % Clear all headers and footers
    \renewcommand{\headrulewidth}{0pt}
    \renewcommand{\footrulewidth}{0pt}
}

% Line spacing
\setstretch{1.2}

% Document starts here
\begin{document}

% Portada
\begin{titlepage}
    \centering
    {\scshape\LARGE Universidad Politécnica de Madrid \par}
    \vspace{1cm}
    {\scshape\Large Escuela Técnica Superior de Ingenieros Industriales\par}
    \vspace{1.5cm}
    {\huge\bfseries Optimización energética de sistema híbrido con bomba de calor, suelo radiante, fotovoltaica y almacenamiento para vivienda \par}
    \vspace{1.5cm}
    {\Large\bfseries Trabajo de Fin de Máster\par}
    \vspace{0.5cm}
    {\large Máster Universitario en Ingeniería de la Energía \par}
    \vspace{2cm}
    {\Large Luis D. Aranda Sánchez\par}
    \vfill
    Director: Javier Rodríguez Martín
    \vfill
    {\large Septiembre 6, 2024\par}
\end{titlepage}

% Resumen (máximo de 5 páginas, incluyendo al final Palabras clave)
\clearpage
\pagestyle{simple}
% \newpage
\chapter*{Resumen}
\addcontentsline{toc}{chapter}{Resumen}
\input{capitulos/resumen/main.tex}

% Índice (paginado)
\clearpage
\pagestyle{simple}
% \newpage
\tableofcontents

% Introducción (donde se incluya los antecedentes y justificación)
\clearpage
\pagestyle{myfancy}
\newpage
\chapter{Introducción}
\input{capitulos/introduccion/main.tex}

% Objetivos
\chapter{Objetivos}
\input{capitulos/objetivos/main.tex}

% Metodología
\chapter{Metodología}
\input{capitulos/metodologia/main.tex}

% Resultados y discusión (incluyendo la valoración de impactos y de aspectos de responsabilidad legal, ética y profesional relacionados con el trabajo)
\chapter{Resultados y Discusión}
\input{capitulos/resultados_discusion/main.tex}

% Conclusiones
\chapter{Conclusiones}
\input{capitulos/conclusiones/main.tex}

% Planificación temporal y presupuesto
\chapter{Planificación Temporal y Presupuesto}
\input{capitulos/planificacion_presupuesto/main.tex}

% Bibliografía
\newpage
\addcontentsline{toc}{chapter}{Bibliografía}
\printbibliography

\end{document}


% Objetivos
\chapter{Objetivos}
\documentclass[a4paper,11pt,twoside]{report}
\usepackage[left=25mm,right=25mm,top=25mm,bottom=25mm,includehead,includefoot,headsep=15mm,footskip=15mm]{geometry}
\usepackage{graphicx}
\usepackage{fancyhdr}
\usepackage{titlesec}
\usepackage[spanish]{babel}
\usepackage[utf8]{inputenc}
\usepackage{amsmath}
\usepackage{setspace}
\usepackage{svg}
\usepackage{hyperref}
\usepackage[backend=biber,style=numeric]{biblatex}
\addbibresource{references.bib}
\hypersetup{
    colorlinks=true,
    linkcolor=blue,      % color of internal links (sections, etc.)
    urlcolor=blue,       % color of external links
    pdftitle={Optimización energética de sistema híbrido con bomba de calor, suelo radiante, fotovoltaica y almacenamiento para vivienda},    % title
    pdfauthor={Luis D. Aranda Sánchez},     % author
    pdfkeywords={palabra1, palabra2, código1, etc.} % list of keywords
}

% Font change to Arial
\usepackage{helvet}
\renewcommand{\familydefault}{\sfdefault}

% Chapter titles in uppercase and larger font
\titleformat{\chapter}[hang]{\large\bfseries}{\thechapter.}{1em}{\MakeUppercase}
\titleformat{\section}[hang]{\bfseries}{\thesection.}{1em}{}
\titleformat{\subsection}[hang]{\bfseries}{\thesubsection.}{1em}{}

% Fancyhdr setup
\setlength{\headheight}{14.30174pt} % Adjust to recommended value, slightly larger for safety
\fancyhf{} % Clear all headers and footers
\fancyhead[LE]{\nouppercase{\leftmark}}
\fancyhead[RO]{Optimización energética para vivienda}
\fancyfoot[LE]{\thepage}
\fancyfoot[RE]{Escuela Técnica Superior de Ingenieros Industriales (UPM)}
\fancyfoot[LO]{Luis D. Aranda Sánchez}
\fancyfoot[RO]{\thepage}
\renewcommand{\headrulewidth}{0.4pt}
\renewcommand{\footrulewidth}{0.4pt}

\fancypagestyle{myfancy}{
    \fancyhf{} % Clear all headers and footers
    \fancyhead[LE]{\nouppercase{\leftmark}}
    \fancyhead[RO]{Optimización energética para vivienda}
    \fancyfoot[LE]{\thepage}
    \fancyfoot[RE]{Escuela Técnica Superior de Ingenieros Industriales (UPM)}
    \fancyfoot[LO]{Luis D. Aranda Sánchez}
    \fancyfoot[RO]{\thepage}
    \renewcommand{\headrulewidth}{0.4pt}
    \renewcommand{\footrulewidth}{0.4pt}
}

\fancypagestyle{simple}{
    \fancyhf{} % Clear all headers and footers
    \renewcommand{\headrulewidth}{0pt}
    \renewcommand{\footrulewidth}{0pt}
}

% Line spacing
\setstretch{1.2}

% Document starts here
\begin{document}

% Portada
\begin{titlepage}
    \centering
    {\scshape\LARGE Universidad Politécnica de Madrid \par}
    \vspace{1cm}
    {\scshape\Large Escuela Técnica Superior de Ingenieros Industriales\par}
    \vspace{1.5cm}
    {\huge\bfseries Optimización energética de sistema híbrido con bomba de calor, suelo radiante, fotovoltaica y almacenamiento para vivienda \par}
    \vspace{1.5cm}
    {\Large\bfseries Trabajo de Fin de Máster\par}
    \vspace{0.5cm}
    {\large Máster Universitario en Ingeniería de la Energía \par}
    \vspace{2cm}
    {\Large Luis D. Aranda Sánchez\par}
    \vfill
    Director: Javier Rodríguez Martín
    \vfill
    {\large Septiembre 6, 2024\par}
\end{titlepage}

% Resumen (máximo de 5 páginas, incluyendo al final Palabras clave)
\clearpage
\pagestyle{simple}
% \newpage
\chapter*{Resumen}
\addcontentsline{toc}{chapter}{Resumen}
\input{capitulos/resumen/main.tex}

% Índice (paginado)
\clearpage
\pagestyle{simple}
% \newpage
\tableofcontents

% Introducción (donde se incluya los antecedentes y justificación)
\clearpage
\pagestyle{myfancy}
\newpage
\chapter{Introducción}
\input{capitulos/introduccion/main.tex}

% Objetivos
\chapter{Objetivos}
\input{capitulos/objetivos/main.tex}

% Metodología
\chapter{Metodología}
\input{capitulos/metodologia/main.tex}

% Resultados y discusión (incluyendo la valoración de impactos y de aspectos de responsabilidad legal, ética y profesional relacionados con el trabajo)
\chapter{Resultados y Discusión}
\input{capitulos/resultados_discusion/main.tex}

% Conclusiones
\chapter{Conclusiones}
\input{capitulos/conclusiones/main.tex}

% Planificación temporal y presupuesto
\chapter{Planificación Temporal y Presupuesto}
\input{capitulos/planificacion_presupuesto/main.tex}

% Bibliografía
\newpage
\addcontentsline{toc}{chapter}{Bibliografía}
\printbibliography

\end{document}


% Metodología
\chapter{Metodología}
\documentclass[a4paper,11pt,twoside]{report}
\usepackage[left=25mm,right=25mm,top=25mm,bottom=25mm,includehead,includefoot,headsep=15mm,footskip=15mm]{geometry}
\usepackage{graphicx}
\usepackage{fancyhdr}
\usepackage{titlesec}
\usepackage[spanish]{babel}
\usepackage[utf8]{inputenc}
\usepackage{amsmath}
\usepackage{setspace}
\usepackage{svg}
\usepackage{hyperref}
\usepackage[backend=biber,style=numeric]{biblatex}
\addbibresource{references.bib}
\hypersetup{
    colorlinks=true,
    linkcolor=blue,      % color of internal links (sections, etc.)
    urlcolor=blue,       % color of external links
    pdftitle={Optimización energética de sistema híbrido con bomba de calor, suelo radiante, fotovoltaica y almacenamiento para vivienda},    % title
    pdfauthor={Luis D. Aranda Sánchez},     % author
    pdfkeywords={palabra1, palabra2, código1, etc.} % list of keywords
}

% Font change to Arial
\usepackage{helvet}
\renewcommand{\familydefault}{\sfdefault}

% Chapter titles in uppercase and larger font
\titleformat{\chapter}[hang]{\large\bfseries}{\thechapter.}{1em}{\MakeUppercase}
\titleformat{\section}[hang]{\bfseries}{\thesection.}{1em}{}
\titleformat{\subsection}[hang]{\bfseries}{\thesubsection.}{1em}{}

% Fancyhdr setup
\setlength{\headheight}{14.30174pt} % Adjust to recommended value, slightly larger for safety
\fancyhf{} % Clear all headers and footers
\fancyhead[LE]{\nouppercase{\leftmark}}
\fancyhead[RO]{Optimización energética para vivienda}
\fancyfoot[LE]{\thepage}
\fancyfoot[RE]{Escuela Técnica Superior de Ingenieros Industriales (UPM)}
\fancyfoot[LO]{Luis D. Aranda Sánchez}
\fancyfoot[RO]{\thepage}
\renewcommand{\headrulewidth}{0.4pt}
\renewcommand{\footrulewidth}{0.4pt}

\fancypagestyle{myfancy}{
    \fancyhf{} % Clear all headers and footers
    \fancyhead[LE]{\nouppercase{\leftmark}}
    \fancyhead[RO]{Optimización energética para vivienda}
    \fancyfoot[LE]{\thepage}
    \fancyfoot[RE]{Escuela Técnica Superior de Ingenieros Industriales (UPM)}
    \fancyfoot[LO]{Luis D. Aranda Sánchez}
    \fancyfoot[RO]{\thepage}
    \renewcommand{\headrulewidth}{0.4pt}
    \renewcommand{\footrulewidth}{0.4pt}
}

\fancypagestyle{simple}{
    \fancyhf{} % Clear all headers and footers
    \renewcommand{\headrulewidth}{0pt}
    \renewcommand{\footrulewidth}{0pt}
}

% Line spacing
\setstretch{1.2}

% Document starts here
\begin{document}

% Portada
\begin{titlepage}
    \centering
    {\scshape\LARGE Universidad Politécnica de Madrid \par}
    \vspace{1cm}
    {\scshape\Large Escuela Técnica Superior de Ingenieros Industriales\par}
    \vspace{1.5cm}
    {\huge\bfseries Optimización energética de sistema híbrido con bomba de calor, suelo radiante, fotovoltaica y almacenamiento para vivienda \par}
    \vspace{1.5cm}
    {\Large\bfseries Trabajo de Fin de Máster\par}
    \vspace{0.5cm}
    {\large Máster Universitario en Ingeniería de la Energía \par}
    \vspace{2cm}
    {\Large Luis D. Aranda Sánchez\par}
    \vfill
    Director: Javier Rodríguez Martín
    \vfill
    {\large Septiembre 6, 2024\par}
\end{titlepage}

% Resumen (máximo de 5 páginas, incluyendo al final Palabras clave)
\clearpage
\pagestyle{simple}
% \newpage
\chapter*{Resumen}
\addcontentsline{toc}{chapter}{Resumen}
\input{capitulos/resumen/main.tex}

% Índice (paginado)
\clearpage
\pagestyle{simple}
% \newpage
\tableofcontents

% Introducción (donde se incluya los antecedentes y justificación)
\clearpage
\pagestyle{myfancy}
\newpage
\chapter{Introducción}
\input{capitulos/introduccion/main.tex}

% Objetivos
\chapter{Objetivos}
\input{capitulos/objetivos/main.tex}

% Metodología
\chapter{Metodología}
\input{capitulos/metodologia/main.tex}

% Resultados y discusión (incluyendo la valoración de impactos y de aspectos de responsabilidad legal, ética y profesional relacionados con el trabajo)
\chapter{Resultados y Discusión}
\input{capitulos/resultados_discusion/main.tex}

% Conclusiones
\chapter{Conclusiones}
\input{capitulos/conclusiones/main.tex}

% Planificación temporal y presupuesto
\chapter{Planificación Temporal y Presupuesto}
\input{capitulos/planificacion_presupuesto/main.tex}

% Bibliografía
\newpage
\addcontentsline{toc}{chapter}{Bibliografía}
\printbibliography

\end{document}


% Resultados y discusión (incluyendo la valoración de impactos y de aspectos de responsabilidad legal, ética y profesional relacionados con el trabajo)
\chapter{Resultados y Discusión}
\documentclass[a4paper,11pt,twoside]{report}
\usepackage[left=25mm,right=25mm,top=25mm,bottom=25mm,includehead,includefoot,headsep=15mm,footskip=15mm]{geometry}
\usepackage{graphicx}
\usepackage{fancyhdr}
\usepackage{titlesec}
\usepackage[spanish]{babel}
\usepackage[utf8]{inputenc}
\usepackage{amsmath}
\usepackage{setspace}
\usepackage{svg}
\usepackage{hyperref}
\usepackage[backend=biber,style=numeric]{biblatex}
\addbibresource{references.bib}
\hypersetup{
    colorlinks=true,
    linkcolor=blue,      % color of internal links (sections, etc.)
    urlcolor=blue,       % color of external links
    pdftitle={Optimización energética de sistema híbrido con bomba de calor, suelo radiante, fotovoltaica y almacenamiento para vivienda},    % title
    pdfauthor={Luis D. Aranda Sánchez},     % author
    pdfkeywords={palabra1, palabra2, código1, etc.} % list of keywords
}

% Font change to Arial
\usepackage{helvet}
\renewcommand{\familydefault}{\sfdefault}

% Chapter titles in uppercase and larger font
\titleformat{\chapter}[hang]{\large\bfseries}{\thechapter.}{1em}{\MakeUppercase}
\titleformat{\section}[hang]{\bfseries}{\thesection.}{1em}{}
\titleformat{\subsection}[hang]{\bfseries}{\thesubsection.}{1em}{}

% Fancyhdr setup
\setlength{\headheight}{14.30174pt} % Adjust to recommended value, slightly larger for safety
\fancyhf{} % Clear all headers and footers
\fancyhead[LE]{\nouppercase{\leftmark}}
\fancyhead[RO]{Optimización energética para vivienda}
\fancyfoot[LE]{\thepage}
\fancyfoot[RE]{Escuela Técnica Superior de Ingenieros Industriales (UPM)}
\fancyfoot[LO]{Luis D. Aranda Sánchez}
\fancyfoot[RO]{\thepage}
\renewcommand{\headrulewidth}{0.4pt}
\renewcommand{\footrulewidth}{0.4pt}

\fancypagestyle{myfancy}{
    \fancyhf{} % Clear all headers and footers
    \fancyhead[LE]{\nouppercase{\leftmark}}
    \fancyhead[RO]{Optimización energética para vivienda}
    \fancyfoot[LE]{\thepage}
    \fancyfoot[RE]{Escuela Técnica Superior de Ingenieros Industriales (UPM)}
    \fancyfoot[LO]{Luis D. Aranda Sánchez}
    \fancyfoot[RO]{\thepage}
    \renewcommand{\headrulewidth}{0.4pt}
    \renewcommand{\footrulewidth}{0.4pt}
}

\fancypagestyle{simple}{
    \fancyhf{} % Clear all headers and footers
    \renewcommand{\headrulewidth}{0pt}
    \renewcommand{\footrulewidth}{0pt}
}

% Line spacing
\setstretch{1.2}

% Document starts here
\begin{document}

% Portada
\begin{titlepage}
    \centering
    {\scshape\LARGE Universidad Politécnica de Madrid \par}
    \vspace{1cm}
    {\scshape\Large Escuela Técnica Superior de Ingenieros Industriales\par}
    \vspace{1.5cm}
    {\huge\bfseries Optimización energética de sistema híbrido con bomba de calor, suelo radiante, fotovoltaica y almacenamiento para vivienda \par}
    \vspace{1.5cm}
    {\Large\bfseries Trabajo de Fin de Máster\par}
    \vspace{0.5cm}
    {\large Máster Universitario en Ingeniería de la Energía \par}
    \vspace{2cm}
    {\Large Luis D. Aranda Sánchez\par}
    \vfill
    Director: Javier Rodríguez Martín
    \vfill
    {\large Septiembre 6, 2024\par}
\end{titlepage}

% Resumen (máximo de 5 páginas, incluyendo al final Palabras clave)
\clearpage
\pagestyle{simple}
% \newpage
\chapter*{Resumen}
\addcontentsline{toc}{chapter}{Resumen}
\input{capitulos/resumen/main.tex}

% Índice (paginado)
\clearpage
\pagestyle{simple}
% \newpage
\tableofcontents

% Introducción (donde se incluya los antecedentes y justificación)
\clearpage
\pagestyle{myfancy}
\newpage
\chapter{Introducción}
\input{capitulos/introduccion/main.tex}

% Objetivos
\chapter{Objetivos}
\input{capitulos/objetivos/main.tex}

% Metodología
\chapter{Metodología}
\input{capitulos/metodologia/main.tex}

% Resultados y discusión (incluyendo la valoración de impactos y de aspectos de responsabilidad legal, ética y profesional relacionados con el trabajo)
\chapter{Resultados y Discusión}
\input{capitulos/resultados_discusion/main.tex}

% Conclusiones
\chapter{Conclusiones}
\input{capitulos/conclusiones/main.tex}

% Planificación temporal y presupuesto
\chapter{Planificación Temporal y Presupuesto}
\input{capitulos/planificacion_presupuesto/main.tex}

% Bibliografía
\newpage
\addcontentsline{toc}{chapter}{Bibliografía}
\printbibliography

\end{document}


% Conclusiones
\chapter{Conclusiones}
\documentclass[a4paper,11pt,twoside]{report}
\usepackage[left=25mm,right=25mm,top=25mm,bottom=25mm,includehead,includefoot,headsep=15mm,footskip=15mm]{geometry}
\usepackage{graphicx}
\usepackage{fancyhdr}
\usepackage{titlesec}
\usepackage[spanish]{babel}
\usepackage[utf8]{inputenc}
\usepackage{amsmath}
\usepackage{setspace}
\usepackage{svg}
\usepackage{hyperref}
\usepackage[backend=biber,style=numeric]{biblatex}
\addbibresource{references.bib}
\hypersetup{
    colorlinks=true,
    linkcolor=blue,      % color of internal links (sections, etc.)
    urlcolor=blue,       % color of external links
    pdftitle={Optimización energética de sistema híbrido con bomba de calor, suelo radiante, fotovoltaica y almacenamiento para vivienda},    % title
    pdfauthor={Luis D. Aranda Sánchez},     % author
    pdfkeywords={palabra1, palabra2, código1, etc.} % list of keywords
}

% Font change to Arial
\usepackage{helvet}
\renewcommand{\familydefault}{\sfdefault}

% Chapter titles in uppercase and larger font
\titleformat{\chapter}[hang]{\large\bfseries}{\thechapter.}{1em}{\MakeUppercase}
\titleformat{\section}[hang]{\bfseries}{\thesection.}{1em}{}
\titleformat{\subsection}[hang]{\bfseries}{\thesubsection.}{1em}{}

% Fancyhdr setup
\setlength{\headheight}{14.30174pt} % Adjust to recommended value, slightly larger for safety
\fancyhf{} % Clear all headers and footers
\fancyhead[LE]{\nouppercase{\leftmark}}
\fancyhead[RO]{Optimización energética para vivienda}
\fancyfoot[LE]{\thepage}
\fancyfoot[RE]{Escuela Técnica Superior de Ingenieros Industriales (UPM)}
\fancyfoot[LO]{Luis D. Aranda Sánchez}
\fancyfoot[RO]{\thepage}
\renewcommand{\headrulewidth}{0.4pt}
\renewcommand{\footrulewidth}{0.4pt}

\fancypagestyle{myfancy}{
    \fancyhf{} % Clear all headers and footers
    \fancyhead[LE]{\nouppercase{\leftmark}}
    \fancyhead[RO]{Optimización energética para vivienda}
    \fancyfoot[LE]{\thepage}
    \fancyfoot[RE]{Escuela Técnica Superior de Ingenieros Industriales (UPM)}
    \fancyfoot[LO]{Luis D. Aranda Sánchez}
    \fancyfoot[RO]{\thepage}
    \renewcommand{\headrulewidth}{0.4pt}
    \renewcommand{\footrulewidth}{0.4pt}
}

\fancypagestyle{simple}{
    \fancyhf{} % Clear all headers and footers
    \renewcommand{\headrulewidth}{0pt}
    \renewcommand{\footrulewidth}{0pt}
}

% Line spacing
\setstretch{1.2}

% Document starts here
\begin{document}

% Portada
\begin{titlepage}
    \centering
    {\scshape\LARGE Universidad Politécnica de Madrid \par}
    \vspace{1cm}
    {\scshape\Large Escuela Técnica Superior de Ingenieros Industriales\par}
    \vspace{1.5cm}
    {\huge\bfseries Optimización energética de sistema híbrido con bomba de calor, suelo radiante, fotovoltaica y almacenamiento para vivienda \par}
    \vspace{1.5cm}
    {\Large\bfseries Trabajo de Fin de Máster\par}
    \vspace{0.5cm}
    {\large Máster Universitario en Ingeniería de la Energía \par}
    \vspace{2cm}
    {\Large Luis D. Aranda Sánchez\par}
    \vfill
    Director: Javier Rodríguez Martín
    \vfill
    {\large Septiembre 6, 2024\par}
\end{titlepage}

% Resumen (máximo de 5 páginas, incluyendo al final Palabras clave)
\clearpage
\pagestyle{simple}
% \newpage
\chapter*{Resumen}
\addcontentsline{toc}{chapter}{Resumen}
\input{capitulos/resumen/main.tex}

% Índice (paginado)
\clearpage
\pagestyle{simple}
% \newpage
\tableofcontents

% Introducción (donde se incluya los antecedentes y justificación)
\clearpage
\pagestyle{myfancy}
\newpage
\chapter{Introducción}
\input{capitulos/introduccion/main.tex}

% Objetivos
\chapter{Objetivos}
\input{capitulos/objetivos/main.tex}

% Metodología
\chapter{Metodología}
\input{capitulos/metodologia/main.tex}

% Resultados y discusión (incluyendo la valoración de impactos y de aspectos de responsabilidad legal, ética y profesional relacionados con el trabajo)
\chapter{Resultados y Discusión}
\input{capitulos/resultados_discusion/main.tex}

% Conclusiones
\chapter{Conclusiones}
\input{capitulos/conclusiones/main.tex}

% Planificación temporal y presupuesto
\chapter{Planificación Temporal y Presupuesto}
\input{capitulos/planificacion_presupuesto/main.tex}

% Bibliografía
\newpage
\addcontentsline{toc}{chapter}{Bibliografía}
\printbibliography

\end{document}


% Planificación temporal y presupuesto
\chapter{Planificación Temporal y Presupuesto}
\documentclass[a4paper,11pt,twoside]{report}
\usepackage[left=25mm,right=25mm,top=25mm,bottom=25mm,includehead,includefoot,headsep=15mm,footskip=15mm]{geometry}
\usepackage{graphicx}
\usepackage{fancyhdr}
\usepackage{titlesec}
\usepackage[spanish]{babel}
\usepackage[utf8]{inputenc}
\usepackage{amsmath}
\usepackage{setspace}
\usepackage{svg}
\usepackage{hyperref}
\usepackage[backend=biber,style=numeric]{biblatex}
\addbibresource{references.bib}
\hypersetup{
    colorlinks=true,
    linkcolor=blue,      % color of internal links (sections, etc.)
    urlcolor=blue,       % color of external links
    pdftitle={Optimización energética de sistema híbrido con bomba de calor, suelo radiante, fotovoltaica y almacenamiento para vivienda},    % title
    pdfauthor={Luis D. Aranda Sánchez},     % author
    pdfkeywords={palabra1, palabra2, código1, etc.} % list of keywords
}

% Font change to Arial
\usepackage{helvet}
\renewcommand{\familydefault}{\sfdefault}

% Chapter titles in uppercase and larger font
\titleformat{\chapter}[hang]{\large\bfseries}{\thechapter.}{1em}{\MakeUppercase}
\titleformat{\section}[hang]{\bfseries}{\thesection.}{1em}{}
\titleformat{\subsection}[hang]{\bfseries}{\thesubsection.}{1em}{}

% Fancyhdr setup
\setlength{\headheight}{14.30174pt} % Adjust to recommended value, slightly larger for safety
\fancyhf{} % Clear all headers and footers
\fancyhead[LE]{\nouppercase{\leftmark}}
\fancyhead[RO]{Optimización energética para vivienda}
\fancyfoot[LE]{\thepage}
\fancyfoot[RE]{Escuela Técnica Superior de Ingenieros Industriales (UPM)}
\fancyfoot[LO]{Luis D. Aranda Sánchez}
\fancyfoot[RO]{\thepage}
\renewcommand{\headrulewidth}{0.4pt}
\renewcommand{\footrulewidth}{0.4pt}

\fancypagestyle{myfancy}{
    \fancyhf{} % Clear all headers and footers
    \fancyhead[LE]{\nouppercase{\leftmark}}
    \fancyhead[RO]{Optimización energética para vivienda}
    \fancyfoot[LE]{\thepage}
    \fancyfoot[RE]{Escuela Técnica Superior de Ingenieros Industriales (UPM)}
    \fancyfoot[LO]{Luis D. Aranda Sánchez}
    \fancyfoot[RO]{\thepage}
    \renewcommand{\headrulewidth}{0.4pt}
    \renewcommand{\footrulewidth}{0.4pt}
}

\fancypagestyle{simple}{
    \fancyhf{} % Clear all headers and footers
    \renewcommand{\headrulewidth}{0pt}
    \renewcommand{\footrulewidth}{0pt}
}

% Line spacing
\setstretch{1.2}

% Document starts here
\begin{document}

% Portada
\begin{titlepage}
    \centering
    {\scshape\LARGE Universidad Politécnica de Madrid \par}
    \vspace{1cm}
    {\scshape\Large Escuela Técnica Superior de Ingenieros Industriales\par}
    \vspace{1.5cm}
    {\huge\bfseries Optimización energética de sistema híbrido con bomba de calor, suelo radiante, fotovoltaica y almacenamiento para vivienda \par}
    \vspace{1.5cm}
    {\Large\bfseries Trabajo de Fin de Máster\par}
    \vspace{0.5cm}
    {\large Máster Universitario en Ingeniería de la Energía \par}
    \vspace{2cm}
    {\Large Luis D. Aranda Sánchez\par}
    \vfill
    Director: Javier Rodríguez Martín
    \vfill
    {\large Septiembre 6, 2024\par}
\end{titlepage}

% Resumen (máximo de 5 páginas, incluyendo al final Palabras clave)
\clearpage
\pagestyle{simple}
% \newpage
\chapter*{Resumen}
\addcontentsline{toc}{chapter}{Resumen}
\input{capitulos/resumen/main.tex}

% Índice (paginado)
\clearpage
\pagestyle{simple}
% \newpage
\tableofcontents

% Introducción (donde se incluya los antecedentes y justificación)
\clearpage
\pagestyle{myfancy}
\newpage
\chapter{Introducción}
\input{capitulos/introduccion/main.tex}

% Objetivos
\chapter{Objetivos}
\input{capitulos/objetivos/main.tex}

% Metodología
\chapter{Metodología}
\input{capitulos/metodologia/main.tex}

% Resultados y discusión (incluyendo la valoración de impactos y de aspectos de responsabilidad legal, ética y profesional relacionados con el trabajo)
\chapter{Resultados y Discusión}
\input{capitulos/resultados_discusion/main.tex}

% Conclusiones
\chapter{Conclusiones}
\input{capitulos/conclusiones/main.tex}

% Planificación temporal y presupuesto
\chapter{Planificación Temporal y Presupuesto}
\input{capitulos/planificacion_presupuesto/main.tex}

% Bibliografía
\newpage
\addcontentsline{toc}{chapter}{Bibliografía}
\printbibliography

\end{document}


% Bibliografía
\newpage
\addcontentsline{toc}{chapter}{Bibliografía}
\printbibliography

\end{document}


% Planificación temporal y presupuesto
\chapter{Planificación Temporal y Presupuesto}
\documentclass[a4paper,11pt,twoside]{report}
\usepackage[left=25mm,right=25mm,top=25mm,bottom=25mm,includehead,includefoot,headsep=15mm,footskip=15mm]{geometry}
\usepackage{graphicx}
\usepackage{fancyhdr}
\usepackage{titlesec}
\usepackage[spanish]{babel}
\usepackage[utf8]{inputenc}
\usepackage{amsmath}
\usepackage{setspace}
\usepackage{svg}
\usepackage{hyperref}
\usepackage[backend=biber,style=numeric]{biblatex}
\addbibresource{references.bib}
\hypersetup{
    colorlinks=true,
    linkcolor=blue,      % color of internal links (sections, etc.)
    urlcolor=blue,       % color of external links
    pdftitle={Optimización energética de sistema híbrido con bomba de calor, suelo radiante, fotovoltaica y almacenamiento para vivienda},    % title
    pdfauthor={Luis D. Aranda Sánchez},     % author
    pdfkeywords={palabra1, palabra2, código1, etc.} % list of keywords
}

% Font change to Arial
\usepackage{helvet}
\renewcommand{\familydefault}{\sfdefault}

% Chapter titles in uppercase and larger font
\titleformat{\chapter}[hang]{\large\bfseries}{\thechapter.}{1em}{\MakeUppercase}
\titleformat{\section}[hang]{\bfseries}{\thesection.}{1em}{}
\titleformat{\subsection}[hang]{\bfseries}{\thesubsection.}{1em}{}

% Fancyhdr setup
\setlength{\headheight}{14.30174pt} % Adjust to recommended value, slightly larger for safety
\fancyhf{} % Clear all headers and footers
\fancyhead[LE]{\nouppercase{\leftmark}}
\fancyhead[RO]{Optimización energética para vivienda}
\fancyfoot[LE]{\thepage}
\fancyfoot[RE]{Escuela Técnica Superior de Ingenieros Industriales (UPM)}
\fancyfoot[LO]{Luis D. Aranda Sánchez}
\fancyfoot[RO]{\thepage}
\renewcommand{\headrulewidth}{0.4pt}
\renewcommand{\footrulewidth}{0.4pt}

\fancypagestyle{myfancy}{
    \fancyhf{} % Clear all headers and footers
    \fancyhead[LE]{\nouppercase{\leftmark}}
    \fancyhead[RO]{Optimización energética para vivienda}
    \fancyfoot[LE]{\thepage}
    \fancyfoot[RE]{Escuela Técnica Superior de Ingenieros Industriales (UPM)}
    \fancyfoot[LO]{Luis D. Aranda Sánchez}
    \fancyfoot[RO]{\thepage}
    \renewcommand{\headrulewidth}{0.4pt}
    \renewcommand{\footrulewidth}{0.4pt}
}

\fancypagestyle{simple}{
    \fancyhf{} % Clear all headers and footers
    \renewcommand{\headrulewidth}{0pt}
    \renewcommand{\footrulewidth}{0pt}
}

% Line spacing
\setstretch{1.2}

% Document starts here
\begin{document}

% Portada
\begin{titlepage}
    \centering
    {\scshape\LARGE Universidad Politécnica de Madrid \par}
    \vspace{1cm}
    {\scshape\Large Escuela Técnica Superior de Ingenieros Industriales\par}
    \vspace{1.5cm}
    {\huge\bfseries Optimización energética de sistema híbrido con bomba de calor, suelo radiante, fotovoltaica y almacenamiento para vivienda \par}
    \vspace{1.5cm}
    {\Large\bfseries Trabajo de Fin de Máster\par}
    \vspace{0.5cm}
    {\large Máster Universitario en Ingeniería de la Energía \par}
    \vspace{2cm}
    {\Large Luis D. Aranda Sánchez\par}
    \vfill
    Director: Javier Rodríguez Martín
    \vfill
    {\large Septiembre 6, 2024\par}
\end{titlepage}

% Resumen (máximo de 5 páginas, incluyendo al final Palabras clave)
\clearpage
\pagestyle{simple}
% \newpage
\chapter*{Resumen}
\addcontentsline{toc}{chapter}{Resumen}
\documentclass[a4paper,11pt,twoside]{report}
\usepackage[left=25mm,right=25mm,top=25mm,bottom=25mm,includehead,includefoot,headsep=15mm,footskip=15mm]{geometry}
\usepackage{graphicx}
\usepackage{fancyhdr}
\usepackage{titlesec}
\usepackage[spanish]{babel}
\usepackage[utf8]{inputenc}
\usepackage{amsmath}
\usepackage{setspace}
\usepackage{svg}
\usepackage{hyperref}
\usepackage[backend=biber,style=numeric]{biblatex}
\addbibresource{references.bib}
\hypersetup{
    colorlinks=true,
    linkcolor=blue,      % color of internal links (sections, etc.)
    urlcolor=blue,       % color of external links
    pdftitle={Optimización energética de sistema híbrido con bomba de calor, suelo radiante, fotovoltaica y almacenamiento para vivienda},    % title
    pdfauthor={Luis D. Aranda Sánchez},     % author
    pdfkeywords={palabra1, palabra2, código1, etc.} % list of keywords
}

% Font change to Arial
\usepackage{helvet}
\renewcommand{\familydefault}{\sfdefault}

% Chapter titles in uppercase and larger font
\titleformat{\chapter}[hang]{\large\bfseries}{\thechapter.}{1em}{\MakeUppercase}
\titleformat{\section}[hang]{\bfseries}{\thesection.}{1em}{}
\titleformat{\subsection}[hang]{\bfseries}{\thesubsection.}{1em}{}

% Fancyhdr setup
\setlength{\headheight}{14.30174pt} % Adjust to recommended value, slightly larger for safety
\fancyhf{} % Clear all headers and footers
\fancyhead[LE]{\nouppercase{\leftmark}}
\fancyhead[RO]{Optimización energética para vivienda}
\fancyfoot[LE]{\thepage}
\fancyfoot[RE]{Escuela Técnica Superior de Ingenieros Industriales (UPM)}
\fancyfoot[LO]{Luis D. Aranda Sánchez}
\fancyfoot[RO]{\thepage}
\renewcommand{\headrulewidth}{0.4pt}
\renewcommand{\footrulewidth}{0.4pt}

\fancypagestyle{myfancy}{
    \fancyhf{} % Clear all headers and footers
    \fancyhead[LE]{\nouppercase{\leftmark}}
    \fancyhead[RO]{Optimización energética para vivienda}
    \fancyfoot[LE]{\thepage}
    \fancyfoot[RE]{Escuela Técnica Superior de Ingenieros Industriales (UPM)}
    \fancyfoot[LO]{Luis D. Aranda Sánchez}
    \fancyfoot[RO]{\thepage}
    \renewcommand{\headrulewidth}{0.4pt}
    \renewcommand{\footrulewidth}{0.4pt}
}

\fancypagestyle{simple}{
    \fancyhf{} % Clear all headers and footers
    \renewcommand{\headrulewidth}{0pt}
    \renewcommand{\footrulewidth}{0pt}
}

% Line spacing
\setstretch{1.2}

% Document starts here
\begin{document}

% Portada
\begin{titlepage}
    \centering
    {\scshape\LARGE Universidad Politécnica de Madrid \par}
    \vspace{1cm}
    {\scshape\Large Escuela Técnica Superior de Ingenieros Industriales\par}
    \vspace{1.5cm}
    {\huge\bfseries Optimización energética de sistema híbrido con bomba de calor, suelo radiante, fotovoltaica y almacenamiento para vivienda \par}
    \vspace{1.5cm}
    {\Large\bfseries Trabajo de Fin de Máster\par}
    \vspace{0.5cm}
    {\large Máster Universitario en Ingeniería de la Energía \par}
    \vspace{2cm}
    {\Large Luis D. Aranda Sánchez\par}
    \vfill
    Director: Javier Rodríguez Martín
    \vfill
    {\large Septiembre 6, 2024\par}
\end{titlepage}

% Resumen (máximo de 5 páginas, incluyendo al final Palabras clave)
\clearpage
\pagestyle{simple}
% \newpage
\chapter*{Resumen}
\addcontentsline{toc}{chapter}{Resumen}
\input{capitulos/resumen/main.tex}

% Índice (paginado)
\clearpage
\pagestyle{simple}
% \newpage
\tableofcontents

% Introducción (donde se incluya los antecedentes y justificación)
\clearpage
\pagestyle{myfancy}
\newpage
\chapter{Introducción}
\input{capitulos/introduccion/main.tex}

% Objetivos
\chapter{Objetivos}
\input{capitulos/objetivos/main.tex}

% Metodología
\chapter{Metodología}
\input{capitulos/metodologia/main.tex}

% Resultados y discusión (incluyendo la valoración de impactos y de aspectos de responsabilidad legal, ética y profesional relacionados con el trabajo)
\chapter{Resultados y Discusión}
\input{capitulos/resultados_discusion/main.tex}

% Conclusiones
\chapter{Conclusiones}
\input{capitulos/conclusiones/main.tex}

% Planificación temporal y presupuesto
\chapter{Planificación Temporal y Presupuesto}
\input{capitulos/planificacion_presupuesto/main.tex}

% Bibliografía
\newpage
\addcontentsline{toc}{chapter}{Bibliografía}
\printbibliography

\end{document}


% Índice (paginado)
\clearpage
\pagestyle{simple}
% \newpage
\tableofcontents

% Introducción (donde se incluya los antecedentes y justificación)
\clearpage
\pagestyle{myfancy}
\newpage
\chapter{Introducción}
\documentclass[a4paper,11pt,twoside]{report}
\usepackage[left=25mm,right=25mm,top=25mm,bottom=25mm,includehead,includefoot,headsep=15mm,footskip=15mm]{geometry}
\usepackage{graphicx}
\usepackage{fancyhdr}
\usepackage{titlesec}
\usepackage[spanish]{babel}
\usepackage[utf8]{inputenc}
\usepackage{amsmath}
\usepackage{setspace}
\usepackage{svg}
\usepackage{hyperref}
\usepackage[backend=biber,style=numeric]{biblatex}
\addbibresource{references.bib}
\hypersetup{
    colorlinks=true,
    linkcolor=blue,      % color of internal links (sections, etc.)
    urlcolor=blue,       % color of external links
    pdftitle={Optimización energética de sistema híbrido con bomba de calor, suelo radiante, fotovoltaica y almacenamiento para vivienda},    % title
    pdfauthor={Luis D. Aranda Sánchez},     % author
    pdfkeywords={palabra1, palabra2, código1, etc.} % list of keywords
}

% Font change to Arial
\usepackage{helvet}
\renewcommand{\familydefault}{\sfdefault}

% Chapter titles in uppercase and larger font
\titleformat{\chapter}[hang]{\large\bfseries}{\thechapter.}{1em}{\MakeUppercase}
\titleformat{\section}[hang]{\bfseries}{\thesection.}{1em}{}
\titleformat{\subsection}[hang]{\bfseries}{\thesubsection.}{1em}{}

% Fancyhdr setup
\setlength{\headheight}{14.30174pt} % Adjust to recommended value, slightly larger for safety
\fancyhf{} % Clear all headers and footers
\fancyhead[LE]{\nouppercase{\leftmark}}
\fancyhead[RO]{Optimización energética para vivienda}
\fancyfoot[LE]{\thepage}
\fancyfoot[RE]{Escuela Técnica Superior de Ingenieros Industriales (UPM)}
\fancyfoot[LO]{Luis D. Aranda Sánchez}
\fancyfoot[RO]{\thepage}
\renewcommand{\headrulewidth}{0.4pt}
\renewcommand{\footrulewidth}{0.4pt}

\fancypagestyle{myfancy}{
    \fancyhf{} % Clear all headers and footers
    \fancyhead[LE]{\nouppercase{\leftmark}}
    \fancyhead[RO]{Optimización energética para vivienda}
    \fancyfoot[LE]{\thepage}
    \fancyfoot[RE]{Escuela Técnica Superior de Ingenieros Industriales (UPM)}
    \fancyfoot[LO]{Luis D. Aranda Sánchez}
    \fancyfoot[RO]{\thepage}
    \renewcommand{\headrulewidth}{0.4pt}
    \renewcommand{\footrulewidth}{0.4pt}
}

\fancypagestyle{simple}{
    \fancyhf{} % Clear all headers and footers
    \renewcommand{\headrulewidth}{0pt}
    \renewcommand{\footrulewidth}{0pt}
}

% Line spacing
\setstretch{1.2}

% Document starts here
\begin{document}

% Portada
\begin{titlepage}
    \centering
    {\scshape\LARGE Universidad Politécnica de Madrid \par}
    \vspace{1cm}
    {\scshape\Large Escuela Técnica Superior de Ingenieros Industriales\par}
    \vspace{1.5cm}
    {\huge\bfseries Optimización energética de sistema híbrido con bomba de calor, suelo radiante, fotovoltaica y almacenamiento para vivienda \par}
    \vspace{1.5cm}
    {\Large\bfseries Trabajo de Fin de Máster\par}
    \vspace{0.5cm}
    {\large Máster Universitario en Ingeniería de la Energía \par}
    \vspace{2cm}
    {\Large Luis D. Aranda Sánchez\par}
    \vfill
    Director: Javier Rodríguez Martín
    \vfill
    {\large Septiembre 6, 2024\par}
\end{titlepage}

% Resumen (máximo de 5 páginas, incluyendo al final Palabras clave)
\clearpage
\pagestyle{simple}
% \newpage
\chapter*{Resumen}
\addcontentsline{toc}{chapter}{Resumen}
\input{capitulos/resumen/main.tex}

% Índice (paginado)
\clearpage
\pagestyle{simple}
% \newpage
\tableofcontents

% Introducción (donde se incluya los antecedentes y justificación)
\clearpage
\pagestyle{myfancy}
\newpage
\chapter{Introducción}
\input{capitulos/introduccion/main.tex}

% Objetivos
\chapter{Objetivos}
\input{capitulos/objetivos/main.tex}

% Metodología
\chapter{Metodología}
\input{capitulos/metodologia/main.tex}

% Resultados y discusión (incluyendo la valoración de impactos y de aspectos de responsabilidad legal, ética y profesional relacionados con el trabajo)
\chapter{Resultados y Discusión}
\input{capitulos/resultados_discusion/main.tex}

% Conclusiones
\chapter{Conclusiones}
\input{capitulos/conclusiones/main.tex}

% Planificación temporal y presupuesto
\chapter{Planificación Temporal y Presupuesto}
\input{capitulos/planificacion_presupuesto/main.tex}

% Bibliografía
\newpage
\addcontentsline{toc}{chapter}{Bibliografía}
\printbibliography

\end{document}


% Objetivos
\chapter{Objetivos}
\documentclass[a4paper,11pt,twoside]{report}
\usepackage[left=25mm,right=25mm,top=25mm,bottom=25mm,includehead,includefoot,headsep=15mm,footskip=15mm]{geometry}
\usepackage{graphicx}
\usepackage{fancyhdr}
\usepackage{titlesec}
\usepackage[spanish]{babel}
\usepackage[utf8]{inputenc}
\usepackage{amsmath}
\usepackage{setspace}
\usepackage{svg}
\usepackage{hyperref}
\usepackage[backend=biber,style=numeric]{biblatex}
\addbibresource{references.bib}
\hypersetup{
    colorlinks=true,
    linkcolor=blue,      % color of internal links (sections, etc.)
    urlcolor=blue,       % color of external links
    pdftitle={Optimización energética de sistema híbrido con bomba de calor, suelo radiante, fotovoltaica y almacenamiento para vivienda},    % title
    pdfauthor={Luis D. Aranda Sánchez},     % author
    pdfkeywords={palabra1, palabra2, código1, etc.} % list of keywords
}

% Font change to Arial
\usepackage{helvet}
\renewcommand{\familydefault}{\sfdefault}

% Chapter titles in uppercase and larger font
\titleformat{\chapter}[hang]{\large\bfseries}{\thechapter.}{1em}{\MakeUppercase}
\titleformat{\section}[hang]{\bfseries}{\thesection.}{1em}{}
\titleformat{\subsection}[hang]{\bfseries}{\thesubsection.}{1em}{}

% Fancyhdr setup
\setlength{\headheight}{14.30174pt} % Adjust to recommended value, slightly larger for safety
\fancyhf{} % Clear all headers and footers
\fancyhead[LE]{\nouppercase{\leftmark}}
\fancyhead[RO]{Optimización energética para vivienda}
\fancyfoot[LE]{\thepage}
\fancyfoot[RE]{Escuela Técnica Superior de Ingenieros Industriales (UPM)}
\fancyfoot[LO]{Luis D. Aranda Sánchez}
\fancyfoot[RO]{\thepage}
\renewcommand{\headrulewidth}{0.4pt}
\renewcommand{\footrulewidth}{0.4pt}

\fancypagestyle{myfancy}{
    \fancyhf{} % Clear all headers and footers
    \fancyhead[LE]{\nouppercase{\leftmark}}
    \fancyhead[RO]{Optimización energética para vivienda}
    \fancyfoot[LE]{\thepage}
    \fancyfoot[RE]{Escuela Técnica Superior de Ingenieros Industriales (UPM)}
    \fancyfoot[LO]{Luis D. Aranda Sánchez}
    \fancyfoot[RO]{\thepage}
    \renewcommand{\headrulewidth}{0.4pt}
    \renewcommand{\footrulewidth}{0.4pt}
}

\fancypagestyle{simple}{
    \fancyhf{} % Clear all headers and footers
    \renewcommand{\headrulewidth}{0pt}
    \renewcommand{\footrulewidth}{0pt}
}

% Line spacing
\setstretch{1.2}

% Document starts here
\begin{document}

% Portada
\begin{titlepage}
    \centering
    {\scshape\LARGE Universidad Politécnica de Madrid \par}
    \vspace{1cm}
    {\scshape\Large Escuela Técnica Superior de Ingenieros Industriales\par}
    \vspace{1.5cm}
    {\huge\bfseries Optimización energética de sistema híbrido con bomba de calor, suelo radiante, fotovoltaica y almacenamiento para vivienda \par}
    \vspace{1.5cm}
    {\Large\bfseries Trabajo de Fin de Máster\par}
    \vspace{0.5cm}
    {\large Máster Universitario en Ingeniería de la Energía \par}
    \vspace{2cm}
    {\Large Luis D. Aranda Sánchez\par}
    \vfill
    Director: Javier Rodríguez Martín
    \vfill
    {\large Septiembre 6, 2024\par}
\end{titlepage}

% Resumen (máximo de 5 páginas, incluyendo al final Palabras clave)
\clearpage
\pagestyle{simple}
% \newpage
\chapter*{Resumen}
\addcontentsline{toc}{chapter}{Resumen}
\input{capitulos/resumen/main.tex}

% Índice (paginado)
\clearpage
\pagestyle{simple}
% \newpage
\tableofcontents

% Introducción (donde se incluya los antecedentes y justificación)
\clearpage
\pagestyle{myfancy}
\newpage
\chapter{Introducción}
\input{capitulos/introduccion/main.tex}

% Objetivos
\chapter{Objetivos}
\input{capitulos/objetivos/main.tex}

% Metodología
\chapter{Metodología}
\input{capitulos/metodologia/main.tex}

% Resultados y discusión (incluyendo la valoración de impactos y de aspectos de responsabilidad legal, ética y profesional relacionados con el trabajo)
\chapter{Resultados y Discusión}
\input{capitulos/resultados_discusion/main.tex}

% Conclusiones
\chapter{Conclusiones}
\input{capitulos/conclusiones/main.tex}

% Planificación temporal y presupuesto
\chapter{Planificación Temporal y Presupuesto}
\input{capitulos/planificacion_presupuesto/main.tex}

% Bibliografía
\newpage
\addcontentsline{toc}{chapter}{Bibliografía}
\printbibliography

\end{document}


% Metodología
\chapter{Metodología}
\documentclass[a4paper,11pt,twoside]{report}
\usepackage[left=25mm,right=25mm,top=25mm,bottom=25mm,includehead,includefoot,headsep=15mm,footskip=15mm]{geometry}
\usepackage{graphicx}
\usepackage{fancyhdr}
\usepackage{titlesec}
\usepackage[spanish]{babel}
\usepackage[utf8]{inputenc}
\usepackage{amsmath}
\usepackage{setspace}
\usepackage{svg}
\usepackage{hyperref}
\usepackage[backend=biber,style=numeric]{biblatex}
\addbibresource{references.bib}
\hypersetup{
    colorlinks=true,
    linkcolor=blue,      % color of internal links (sections, etc.)
    urlcolor=blue,       % color of external links
    pdftitle={Optimización energética de sistema híbrido con bomba de calor, suelo radiante, fotovoltaica y almacenamiento para vivienda},    % title
    pdfauthor={Luis D. Aranda Sánchez},     % author
    pdfkeywords={palabra1, palabra2, código1, etc.} % list of keywords
}

% Font change to Arial
\usepackage{helvet}
\renewcommand{\familydefault}{\sfdefault}

% Chapter titles in uppercase and larger font
\titleformat{\chapter}[hang]{\large\bfseries}{\thechapter.}{1em}{\MakeUppercase}
\titleformat{\section}[hang]{\bfseries}{\thesection.}{1em}{}
\titleformat{\subsection}[hang]{\bfseries}{\thesubsection.}{1em}{}

% Fancyhdr setup
\setlength{\headheight}{14.30174pt} % Adjust to recommended value, slightly larger for safety
\fancyhf{} % Clear all headers and footers
\fancyhead[LE]{\nouppercase{\leftmark}}
\fancyhead[RO]{Optimización energética para vivienda}
\fancyfoot[LE]{\thepage}
\fancyfoot[RE]{Escuela Técnica Superior de Ingenieros Industriales (UPM)}
\fancyfoot[LO]{Luis D. Aranda Sánchez}
\fancyfoot[RO]{\thepage}
\renewcommand{\headrulewidth}{0.4pt}
\renewcommand{\footrulewidth}{0.4pt}

\fancypagestyle{myfancy}{
    \fancyhf{} % Clear all headers and footers
    \fancyhead[LE]{\nouppercase{\leftmark}}
    \fancyhead[RO]{Optimización energética para vivienda}
    \fancyfoot[LE]{\thepage}
    \fancyfoot[RE]{Escuela Técnica Superior de Ingenieros Industriales (UPM)}
    \fancyfoot[LO]{Luis D. Aranda Sánchez}
    \fancyfoot[RO]{\thepage}
    \renewcommand{\headrulewidth}{0.4pt}
    \renewcommand{\footrulewidth}{0.4pt}
}

\fancypagestyle{simple}{
    \fancyhf{} % Clear all headers and footers
    \renewcommand{\headrulewidth}{0pt}
    \renewcommand{\footrulewidth}{0pt}
}

% Line spacing
\setstretch{1.2}

% Document starts here
\begin{document}

% Portada
\begin{titlepage}
    \centering
    {\scshape\LARGE Universidad Politécnica de Madrid \par}
    \vspace{1cm}
    {\scshape\Large Escuela Técnica Superior de Ingenieros Industriales\par}
    \vspace{1.5cm}
    {\huge\bfseries Optimización energética de sistema híbrido con bomba de calor, suelo radiante, fotovoltaica y almacenamiento para vivienda \par}
    \vspace{1.5cm}
    {\Large\bfseries Trabajo de Fin de Máster\par}
    \vspace{0.5cm}
    {\large Máster Universitario en Ingeniería de la Energía \par}
    \vspace{2cm}
    {\Large Luis D. Aranda Sánchez\par}
    \vfill
    Director: Javier Rodríguez Martín
    \vfill
    {\large Septiembre 6, 2024\par}
\end{titlepage}

% Resumen (máximo de 5 páginas, incluyendo al final Palabras clave)
\clearpage
\pagestyle{simple}
% \newpage
\chapter*{Resumen}
\addcontentsline{toc}{chapter}{Resumen}
\input{capitulos/resumen/main.tex}

% Índice (paginado)
\clearpage
\pagestyle{simple}
% \newpage
\tableofcontents

% Introducción (donde se incluya los antecedentes y justificación)
\clearpage
\pagestyle{myfancy}
\newpage
\chapter{Introducción}
\input{capitulos/introduccion/main.tex}

% Objetivos
\chapter{Objetivos}
\input{capitulos/objetivos/main.tex}

% Metodología
\chapter{Metodología}
\input{capitulos/metodologia/main.tex}

% Resultados y discusión (incluyendo la valoración de impactos y de aspectos de responsabilidad legal, ética y profesional relacionados con el trabajo)
\chapter{Resultados y Discusión}
\input{capitulos/resultados_discusion/main.tex}

% Conclusiones
\chapter{Conclusiones}
\input{capitulos/conclusiones/main.tex}

% Planificación temporal y presupuesto
\chapter{Planificación Temporal y Presupuesto}
\input{capitulos/planificacion_presupuesto/main.tex}

% Bibliografía
\newpage
\addcontentsline{toc}{chapter}{Bibliografía}
\printbibliography

\end{document}


% Resultados y discusión (incluyendo la valoración de impactos y de aspectos de responsabilidad legal, ética y profesional relacionados con el trabajo)
\chapter{Resultados y Discusión}
\documentclass[a4paper,11pt,twoside]{report}
\usepackage[left=25mm,right=25mm,top=25mm,bottom=25mm,includehead,includefoot,headsep=15mm,footskip=15mm]{geometry}
\usepackage{graphicx}
\usepackage{fancyhdr}
\usepackage{titlesec}
\usepackage[spanish]{babel}
\usepackage[utf8]{inputenc}
\usepackage{amsmath}
\usepackage{setspace}
\usepackage{svg}
\usepackage{hyperref}
\usepackage[backend=biber,style=numeric]{biblatex}
\addbibresource{references.bib}
\hypersetup{
    colorlinks=true,
    linkcolor=blue,      % color of internal links (sections, etc.)
    urlcolor=blue,       % color of external links
    pdftitle={Optimización energética de sistema híbrido con bomba de calor, suelo radiante, fotovoltaica y almacenamiento para vivienda},    % title
    pdfauthor={Luis D. Aranda Sánchez},     % author
    pdfkeywords={palabra1, palabra2, código1, etc.} % list of keywords
}

% Font change to Arial
\usepackage{helvet}
\renewcommand{\familydefault}{\sfdefault}

% Chapter titles in uppercase and larger font
\titleformat{\chapter}[hang]{\large\bfseries}{\thechapter.}{1em}{\MakeUppercase}
\titleformat{\section}[hang]{\bfseries}{\thesection.}{1em}{}
\titleformat{\subsection}[hang]{\bfseries}{\thesubsection.}{1em}{}

% Fancyhdr setup
\setlength{\headheight}{14.30174pt} % Adjust to recommended value, slightly larger for safety
\fancyhf{} % Clear all headers and footers
\fancyhead[LE]{\nouppercase{\leftmark}}
\fancyhead[RO]{Optimización energética para vivienda}
\fancyfoot[LE]{\thepage}
\fancyfoot[RE]{Escuela Técnica Superior de Ingenieros Industriales (UPM)}
\fancyfoot[LO]{Luis D. Aranda Sánchez}
\fancyfoot[RO]{\thepage}
\renewcommand{\headrulewidth}{0.4pt}
\renewcommand{\footrulewidth}{0.4pt}

\fancypagestyle{myfancy}{
    \fancyhf{} % Clear all headers and footers
    \fancyhead[LE]{\nouppercase{\leftmark}}
    \fancyhead[RO]{Optimización energética para vivienda}
    \fancyfoot[LE]{\thepage}
    \fancyfoot[RE]{Escuela Técnica Superior de Ingenieros Industriales (UPM)}
    \fancyfoot[LO]{Luis D. Aranda Sánchez}
    \fancyfoot[RO]{\thepage}
    \renewcommand{\headrulewidth}{0.4pt}
    \renewcommand{\footrulewidth}{0.4pt}
}

\fancypagestyle{simple}{
    \fancyhf{} % Clear all headers and footers
    \renewcommand{\headrulewidth}{0pt}
    \renewcommand{\footrulewidth}{0pt}
}

% Line spacing
\setstretch{1.2}

% Document starts here
\begin{document}

% Portada
\begin{titlepage}
    \centering
    {\scshape\LARGE Universidad Politécnica de Madrid \par}
    \vspace{1cm}
    {\scshape\Large Escuela Técnica Superior de Ingenieros Industriales\par}
    \vspace{1.5cm}
    {\huge\bfseries Optimización energética de sistema híbrido con bomba de calor, suelo radiante, fotovoltaica y almacenamiento para vivienda \par}
    \vspace{1.5cm}
    {\Large\bfseries Trabajo de Fin de Máster\par}
    \vspace{0.5cm}
    {\large Máster Universitario en Ingeniería de la Energía \par}
    \vspace{2cm}
    {\Large Luis D. Aranda Sánchez\par}
    \vfill
    Director: Javier Rodríguez Martín
    \vfill
    {\large Septiembre 6, 2024\par}
\end{titlepage}

% Resumen (máximo de 5 páginas, incluyendo al final Palabras clave)
\clearpage
\pagestyle{simple}
% \newpage
\chapter*{Resumen}
\addcontentsline{toc}{chapter}{Resumen}
\input{capitulos/resumen/main.tex}

% Índice (paginado)
\clearpage
\pagestyle{simple}
% \newpage
\tableofcontents

% Introducción (donde se incluya los antecedentes y justificación)
\clearpage
\pagestyle{myfancy}
\newpage
\chapter{Introducción}
\input{capitulos/introduccion/main.tex}

% Objetivos
\chapter{Objetivos}
\input{capitulos/objetivos/main.tex}

% Metodología
\chapter{Metodología}
\input{capitulos/metodologia/main.tex}

% Resultados y discusión (incluyendo la valoración de impactos y de aspectos de responsabilidad legal, ética y profesional relacionados con el trabajo)
\chapter{Resultados y Discusión}
\input{capitulos/resultados_discusion/main.tex}

% Conclusiones
\chapter{Conclusiones}
\input{capitulos/conclusiones/main.tex}

% Planificación temporal y presupuesto
\chapter{Planificación Temporal y Presupuesto}
\input{capitulos/planificacion_presupuesto/main.tex}

% Bibliografía
\newpage
\addcontentsline{toc}{chapter}{Bibliografía}
\printbibliography

\end{document}


% Conclusiones
\chapter{Conclusiones}
\documentclass[a4paper,11pt,twoside]{report}
\usepackage[left=25mm,right=25mm,top=25mm,bottom=25mm,includehead,includefoot,headsep=15mm,footskip=15mm]{geometry}
\usepackage{graphicx}
\usepackage{fancyhdr}
\usepackage{titlesec}
\usepackage[spanish]{babel}
\usepackage[utf8]{inputenc}
\usepackage{amsmath}
\usepackage{setspace}
\usepackage{svg}
\usepackage{hyperref}
\usepackage[backend=biber,style=numeric]{biblatex}
\addbibresource{references.bib}
\hypersetup{
    colorlinks=true,
    linkcolor=blue,      % color of internal links (sections, etc.)
    urlcolor=blue,       % color of external links
    pdftitle={Optimización energética de sistema híbrido con bomba de calor, suelo radiante, fotovoltaica y almacenamiento para vivienda},    % title
    pdfauthor={Luis D. Aranda Sánchez},     % author
    pdfkeywords={palabra1, palabra2, código1, etc.} % list of keywords
}

% Font change to Arial
\usepackage{helvet}
\renewcommand{\familydefault}{\sfdefault}

% Chapter titles in uppercase and larger font
\titleformat{\chapter}[hang]{\large\bfseries}{\thechapter.}{1em}{\MakeUppercase}
\titleformat{\section}[hang]{\bfseries}{\thesection.}{1em}{}
\titleformat{\subsection}[hang]{\bfseries}{\thesubsection.}{1em}{}

% Fancyhdr setup
\setlength{\headheight}{14.30174pt} % Adjust to recommended value, slightly larger for safety
\fancyhf{} % Clear all headers and footers
\fancyhead[LE]{\nouppercase{\leftmark}}
\fancyhead[RO]{Optimización energética para vivienda}
\fancyfoot[LE]{\thepage}
\fancyfoot[RE]{Escuela Técnica Superior de Ingenieros Industriales (UPM)}
\fancyfoot[LO]{Luis D. Aranda Sánchez}
\fancyfoot[RO]{\thepage}
\renewcommand{\headrulewidth}{0.4pt}
\renewcommand{\footrulewidth}{0.4pt}

\fancypagestyle{myfancy}{
    \fancyhf{} % Clear all headers and footers
    \fancyhead[LE]{\nouppercase{\leftmark}}
    \fancyhead[RO]{Optimización energética para vivienda}
    \fancyfoot[LE]{\thepage}
    \fancyfoot[RE]{Escuela Técnica Superior de Ingenieros Industriales (UPM)}
    \fancyfoot[LO]{Luis D. Aranda Sánchez}
    \fancyfoot[RO]{\thepage}
    \renewcommand{\headrulewidth}{0.4pt}
    \renewcommand{\footrulewidth}{0.4pt}
}

\fancypagestyle{simple}{
    \fancyhf{} % Clear all headers and footers
    \renewcommand{\headrulewidth}{0pt}
    \renewcommand{\footrulewidth}{0pt}
}

% Line spacing
\setstretch{1.2}

% Document starts here
\begin{document}

% Portada
\begin{titlepage}
    \centering
    {\scshape\LARGE Universidad Politécnica de Madrid \par}
    \vspace{1cm}
    {\scshape\Large Escuela Técnica Superior de Ingenieros Industriales\par}
    \vspace{1.5cm}
    {\huge\bfseries Optimización energética de sistema híbrido con bomba de calor, suelo radiante, fotovoltaica y almacenamiento para vivienda \par}
    \vspace{1.5cm}
    {\Large\bfseries Trabajo de Fin de Máster\par}
    \vspace{0.5cm}
    {\large Máster Universitario en Ingeniería de la Energía \par}
    \vspace{2cm}
    {\Large Luis D. Aranda Sánchez\par}
    \vfill
    Director: Javier Rodríguez Martín
    \vfill
    {\large Septiembre 6, 2024\par}
\end{titlepage}

% Resumen (máximo de 5 páginas, incluyendo al final Palabras clave)
\clearpage
\pagestyle{simple}
% \newpage
\chapter*{Resumen}
\addcontentsline{toc}{chapter}{Resumen}
\input{capitulos/resumen/main.tex}

% Índice (paginado)
\clearpage
\pagestyle{simple}
% \newpage
\tableofcontents

% Introducción (donde se incluya los antecedentes y justificación)
\clearpage
\pagestyle{myfancy}
\newpage
\chapter{Introducción}
\input{capitulos/introduccion/main.tex}

% Objetivos
\chapter{Objetivos}
\input{capitulos/objetivos/main.tex}

% Metodología
\chapter{Metodología}
\input{capitulos/metodologia/main.tex}

% Resultados y discusión (incluyendo la valoración de impactos y de aspectos de responsabilidad legal, ética y profesional relacionados con el trabajo)
\chapter{Resultados y Discusión}
\input{capitulos/resultados_discusion/main.tex}

% Conclusiones
\chapter{Conclusiones}
\input{capitulos/conclusiones/main.tex}

% Planificación temporal y presupuesto
\chapter{Planificación Temporal y Presupuesto}
\input{capitulos/planificacion_presupuesto/main.tex}

% Bibliografía
\newpage
\addcontentsline{toc}{chapter}{Bibliografía}
\printbibliography

\end{document}


% Planificación temporal y presupuesto
\chapter{Planificación Temporal y Presupuesto}
\documentclass[a4paper,11pt,twoside]{report}
\usepackage[left=25mm,right=25mm,top=25mm,bottom=25mm,includehead,includefoot,headsep=15mm,footskip=15mm]{geometry}
\usepackage{graphicx}
\usepackage{fancyhdr}
\usepackage{titlesec}
\usepackage[spanish]{babel}
\usepackage[utf8]{inputenc}
\usepackage{amsmath}
\usepackage{setspace}
\usepackage{svg}
\usepackage{hyperref}
\usepackage[backend=biber,style=numeric]{biblatex}
\addbibresource{references.bib}
\hypersetup{
    colorlinks=true,
    linkcolor=blue,      % color of internal links (sections, etc.)
    urlcolor=blue,       % color of external links
    pdftitle={Optimización energética de sistema híbrido con bomba de calor, suelo radiante, fotovoltaica y almacenamiento para vivienda},    % title
    pdfauthor={Luis D. Aranda Sánchez},     % author
    pdfkeywords={palabra1, palabra2, código1, etc.} % list of keywords
}

% Font change to Arial
\usepackage{helvet}
\renewcommand{\familydefault}{\sfdefault}

% Chapter titles in uppercase and larger font
\titleformat{\chapter}[hang]{\large\bfseries}{\thechapter.}{1em}{\MakeUppercase}
\titleformat{\section}[hang]{\bfseries}{\thesection.}{1em}{}
\titleformat{\subsection}[hang]{\bfseries}{\thesubsection.}{1em}{}

% Fancyhdr setup
\setlength{\headheight}{14.30174pt} % Adjust to recommended value, slightly larger for safety
\fancyhf{} % Clear all headers and footers
\fancyhead[LE]{\nouppercase{\leftmark}}
\fancyhead[RO]{Optimización energética para vivienda}
\fancyfoot[LE]{\thepage}
\fancyfoot[RE]{Escuela Técnica Superior de Ingenieros Industriales (UPM)}
\fancyfoot[LO]{Luis D. Aranda Sánchez}
\fancyfoot[RO]{\thepage}
\renewcommand{\headrulewidth}{0.4pt}
\renewcommand{\footrulewidth}{0.4pt}

\fancypagestyle{myfancy}{
    \fancyhf{} % Clear all headers and footers
    \fancyhead[LE]{\nouppercase{\leftmark}}
    \fancyhead[RO]{Optimización energética para vivienda}
    \fancyfoot[LE]{\thepage}
    \fancyfoot[RE]{Escuela Técnica Superior de Ingenieros Industriales (UPM)}
    \fancyfoot[LO]{Luis D. Aranda Sánchez}
    \fancyfoot[RO]{\thepage}
    \renewcommand{\headrulewidth}{0.4pt}
    \renewcommand{\footrulewidth}{0.4pt}
}

\fancypagestyle{simple}{
    \fancyhf{} % Clear all headers and footers
    \renewcommand{\headrulewidth}{0pt}
    \renewcommand{\footrulewidth}{0pt}
}

% Line spacing
\setstretch{1.2}

% Document starts here
\begin{document}

% Portada
\begin{titlepage}
    \centering
    {\scshape\LARGE Universidad Politécnica de Madrid \par}
    \vspace{1cm}
    {\scshape\Large Escuela Técnica Superior de Ingenieros Industriales\par}
    \vspace{1.5cm}
    {\huge\bfseries Optimización energética de sistema híbrido con bomba de calor, suelo radiante, fotovoltaica y almacenamiento para vivienda \par}
    \vspace{1.5cm}
    {\Large\bfseries Trabajo de Fin de Máster\par}
    \vspace{0.5cm}
    {\large Máster Universitario en Ingeniería de la Energía \par}
    \vspace{2cm}
    {\Large Luis D. Aranda Sánchez\par}
    \vfill
    Director: Javier Rodríguez Martín
    \vfill
    {\large Septiembre 6, 2024\par}
\end{titlepage}

% Resumen (máximo de 5 páginas, incluyendo al final Palabras clave)
\clearpage
\pagestyle{simple}
% \newpage
\chapter*{Resumen}
\addcontentsline{toc}{chapter}{Resumen}
\input{capitulos/resumen/main.tex}

% Índice (paginado)
\clearpage
\pagestyle{simple}
% \newpage
\tableofcontents

% Introducción (donde se incluya los antecedentes y justificación)
\clearpage
\pagestyle{myfancy}
\newpage
\chapter{Introducción}
\input{capitulos/introduccion/main.tex}

% Objetivos
\chapter{Objetivos}
\input{capitulos/objetivos/main.tex}

% Metodología
\chapter{Metodología}
\input{capitulos/metodologia/main.tex}

% Resultados y discusión (incluyendo la valoración de impactos y de aspectos de responsabilidad legal, ética y profesional relacionados con el trabajo)
\chapter{Resultados y Discusión}
\input{capitulos/resultados_discusion/main.tex}

% Conclusiones
\chapter{Conclusiones}
\input{capitulos/conclusiones/main.tex}

% Planificación temporal y presupuesto
\chapter{Planificación Temporal y Presupuesto}
\input{capitulos/planificacion_presupuesto/main.tex}

% Bibliografía
\newpage
\addcontentsline{toc}{chapter}{Bibliografía}
\printbibliography

\end{document}


% Bibliografía
\newpage
\addcontentsline{toc}{chapter}{Bibliografía}
\printbibliography

\end{document}


% Bibliografía
\newpage
\addcontentsline{toc}{chapter}{Bibliografía}
\printbibliography

\end{document}


% Adquisición de datos
\cleardoublepage
\chapter{Adquisición de datos}
\documentclass[a4paper,11pt,twoside]{report}
\usepackage[left=25mm,right=25mm,top=25mm,bottom=25mm,includehead,includefoot,headsep=15mm,footskip=15mm]{geometry}
\usepackage{graphicx}
\usepackage{fancyhdr}
\usepackage{titlesec}
\usepackage[spanish]{babel}
\usepackage[utf8]{inputenc}
\usepackage{amsmath}
\usepackage{setspace}
\usepackage{svg}
\usepackage{hyperref}
\usepackage[backend=biber,style=numeric]{biblatex}
\addbibresource{references.bib}
\hypersetup{
    colorlinks=true,
    linkcolor=blue,      % color of internal links (sections, etc.)
    urlcolor=blue,       % color of external links
    pdftitle={Optimización energética de sistema híbrido con bomba de calor, suelo radiante, fotovoltaica y almacenamiento para vivienda},    % title
    pdfauthor={Luis D. Aranda Sánchez},     % author
    pdfkeywords={palabra1, palabra2, código1, etc.} % list of keywords
}

% Font change to Arial
\usepackage{helvet}
\renewcommand{\familydefault}{\sfdefault}

% Chapter titles in uppercase and larger font
\titleformat{\chapter}[hang]{\large\bfseries}{\thechapter.}{1em}{\MakeUppercase}
\titleformat{\section}[hang]{\bfseries}{\thesection.}{1em}{}
\titleformat{\subsection}[hang]{\bfseries}{\thesubsection.}{1em}{}

% Fancyhdr setup
\setlength{\headheight}{14.30174pt} % Adjust to recommended value, slightly larger for safety
\fancyhf{} % Clear all headers and footers
\fancyhead[LE]{\nouppercase{\leftmark}}
\fancyhead[RO]{Optimización energética para vivienda}
\fancyfoot[LE]{\thepage}
\fancyfoot[RE]{Escuela Técnica Superior de Ingenieros Industriales (UPM)}
\fancyfoot[LO]{Luis D. Aranda Sánchez}
\fancyfoot[RO]{\thepage}
\renewcommand{\headrulewidth}{0.4pt}
\renewcommand{\footrulewidth}{0.4pt}

\fancypagestyle{myfancy}{
    \fancyhf{} % Clear all headers and footers
    \fancyhead[LE]{\nouppercase{\leftmark}}
    \fancyhead[RO]{Optimización energética para vivienda}
    \fancyfoot[LE]{\thepage}
    \fancyfoot[RE]{Escuela Técnica Superior de Ingenieros Industriales (UPM)}
    \fancyfoot[LO]{Luis D. Aranda Sánchez}
    \fancyfoot[RO]{\thepage}
    \renewcommand{\headrulewidth}{0.4pt}
    \renewcommand{\footrulewidth}{0.4pt}
}

\fancypagestyle{simple}{
    \fancyhf{} % Clear all headers and footers
    \renewcommand{\headrulewidth}{0pt}
    \renewcommand{\footrulewidth}{0pt}
}

% Line spacing
\setstretch{1.2}

% Document starts here
\begin{document}

% Portada
\begin{titlepage}
    \centering
    {\scshape\LARGE Universidad Politécnica de Madrid \par}
    \vspace{1cm}
    {\scshape\Large Escuela Técnica Superior de Ingenieros Industriales\par}
    \vspace{1.5cm}
    {\huge\bfseries Optimización energética de sistema híbrido con bomba de calor, suelo radiante, fotovoltaica y almacenamiento para vivienda \par}
    \vspace{1.5cm}
    {\Large\bfseries Trabajo de Fin de Máster\par}
    \vspace{0.5cm}
    {\large Máster Universitario en Ingeniería de la Energía \par}
    \vspace{2cm}
    {\Large Luis D. Aranda Sánchez\par}
    \vfill
    Director: Javier Rodríguez Martín
    \vfill
    {\large Septiembre 6, 2024\par}
\end{titlepage}

% Resumen (máximo de 5 páginas, incluyendo al final Palabras clave)
\clearpage
\pagestyle{simple}
% \newpage
\chapter*{Resumen}
\addcontentsline{toc}{chapter}{Resumen}
\documentclass[a4paper,11pt,twoside]{report}
\usepackage[left=25mm,right=25mm,top=25mm,bottom=25mm,includehead,includefoot,headsep=15mm,footskip=15mm]{geometry}
\usepackage{graphicx}
\usepackage{fancyhdr}
\usepackage{titlesec}
\usepackage[spanish]{babel}
\usepackage[utf8]{inputenc}
\usepackage{amsmath}
\usepackage{setspace}
\usepackage{svg}
\usepackage{hyperref}
\usepackage[backend=biber,style=numeric]{biblatex}
\addbibresource{references.bib}
\hypersetup{
    colorlinks=true,
    linkcolor=blue,      % color of internal links (sections, etc.)
    urlcolor=blue,       % color of external links
    pdftitle={Optimización energética de sistema híbrido con bomba de calor, suelo radiante, fotovoltaica y almacenamiento para vivienda},    % title
    pdfauthor={Luis D. Aranda Sánchez},     % author
    pdfkeywords={palabra1, palabra2, código1, etc.} % list of keywords
}

% Font change to Arial
\usepackage{helvet}
\renewcommand{\familydefault}{\sfdefault}

% Chapter titles in uppercase and larger font
\titleformat{\chapter}[hang]{\large\bfseries}{\thechapter.}{1em}{\MakeUppercase}
\titleformat{\section}[hang]{\bfseries}{\thesection.}{1em}{}
\titleformat{\subsection}[hang]{\bfseries}{\thesubsection.}{1em}{}

% Fancyhdr setup
\setlength{\headheight}{14.30174pt} % Adjust to recommended value, slightly larger for safety
\fancyhf{} % Clear all headers and footers
\fancyhead[LE]{\nouppercase{\leftmark}}
\fancyhead[RO]{Optimización energética para vivienda}
\fancyfoot[LE]{\thepage}
\fancyfoot[RE]{Escuela Técnica Superior de Ingenieros Industriales (UPM)}
\fancyfoot[LO]{Luis D. Aranda Sánchez}
\fancyfoot[RO]{\thepage}
\renewcommand{\headrulewidth}{0.4pt}
\renewcommand{\footrulewidth}{0.4pt}

\fancypagestyle{myfancy}{
    \fancyhf{} % Clear all headers and footers
    \fancyhead[LE]{\nouppercase{\leftmark}}
    \fancyhead[RO]{Optimización energética para vivienda}
    \fancyfoot[LE]{\thepage}
    \fancyfoot[RE]{Escuela Técnica Superior de Ingenieros Industriales (UPM)}
    \fancyfoot[LO]{Luis D. Aranda Sánchez}
    \fancyfoot[RO]{\thepage}
    \renewcommand{\headrulewidth}{0.4pt}
    \renewcommand{\footrulewidth}{0.4pt}
}

\fancypagestyle{simple}{
    \fancyhf{} % Clear all headers and footers
    \renewcommand{\headrulewidth}{0pt}
    \renewcommand{\footrulewidth}{0pt}
}

% Line spacing
\setstretch{1.2}

% Document starts here
\begin{document}

% Portada
\begin{titlepage}
    \centering
    {\scshape\LARGE Universidad Politécnica de Madrid \par}
    \vspace{1cm}
    {\scshape\Large Escuela Técnica Superior de Ingenieros Industriales\par}
    \vspace{1.5cm}
    {\huge\bfseries Optimización energética de sistema híbrido con bomba de calor, suelo radiante, fotovoltaica y almacenamiento para vivienda \par}
    \vspace{1.5cm}
    {\Large\bfseries Trabajo de Fin de Máster\par}
    \vspace{0.5cm}
    {\large Máster Universitario en Ingeniería de la Energía \par}
    \vspace{2cm}
    {\Large Luis D. Aranda Sánchez\par}
    \vfill
    Director: Javier Rodríguez Martín
    \vfill
    {\large Septiembre 6, 2024\par}
\end{titlepage}

% Resumen (máximo de 5 páginas, incluyendo al final Palabras clave)
\clearpage
\pagestyle{simple}
% \newpage
\chapter*{Resumen}
\addcontentsline{toc}{chapter}{Resumen}
\documentclass[a4paper,11pt,twoside]{report}
\usepackage[left=25mm,right=25mm,top=25mm,bottom=25mm,includehead,includefoot,headsep=15mm,footskip=15mm]{geometry}
\usepackage{graphicx}
\usepackage{fancyhdr}
\usepackage{titlesec}
\usepackage[spanish]{babel}
\usepackage[utf8]{inputenc}
\usepackage{amsmath}
\usepackage{setspace}
\usepackage{svg}
\usepackage{hyperref}
\usepackage[backend=biber,style=numeric]{biblatex}
\addbibresource{references.bib}
\hypersetup{
    colorlinks=true,
    linkcolor=blue,      % color of internal links (sections, etc.)
    urlcolor=blue,       % color of external links
    pdftitle={Optimización energética de sistema híbrido con bomba de calor, suelo radiante, fotovoltaica y almacenamiento para vivienda},    % title
    pdfauthor={Luis D. Aranda Sánchez},     % author
    pdfkeywords={palabra1, palabra2, código1, etc.} % list of keywords
}

% Font change to Arial
\usepackage{helvet}
\renewcommand{\familydefault}{\sfdefault}

% Chapter titles in uppercase and larger font
\titleformat{\chapter}[hang]{\large\bfseries}{\thechapter.}{1em}{\MakeUppercase}
\titleformat{\section}[hang]{\bfseries}{\thesection.}{1em}{}
\titleformat{\subsection}[hang]{\bfseries}{\thesubsection.}{1em}{}

% Fancyhdr setup
\setlength{\headheight}{14.30174pt} % Adjust to recommended value, slightly larger for safety
\fancyhf{} % Clear all headers and footers
\fancyhead[LE]{\nouppercase{\leftmark}}
\fancyhead[RO]{Optimización energética para vivienda}
\fancyfoot[LE]{\thepage}
\fancyfoot[RE]{Escuela Técnica Superior de Ingenieros Industriales (UPM)}
\fancyfoot[LO]{Luis D. Aranda Sánchez}
\fancyfoot[RO]{\thepage}
\renewcommand{\headrulewidth}{0.4pt}
\renewcommand{\footrulewidth}{0.4pt}

\fancypagestyle{myfancy}{
    \fancyhf{} % Clear all headers and footers
    \fancyhead[LE]{\nouppercase{\leftmark}}
    \fancyhead[RO]{Optimización energética para vivienda}
    \fancyfoot[LE]{\thepage}
    \fancyfoot[RE]{Escuela Técnica Superior de Ingenieros Industriales (UPM)}
    \fancyfoot[LO]{Luis D. Aranda Sánchez}
    \fancyfoot[RO]{\thepage}
    \renewcommand{\headrulewidth}{0.4pt}
    \renewcommand{\footrulewidth}{0.4pt}
}

\fancypagestyle{simple}{
    \fancyhf{} % Clear all headers and footers
    \renewcommand{\headrulewidth}{0pt}
    \renewcommand{\footrulewidth}{0pt}
}

% Line spacing
\setstretch{1.2}

% Document starts here
\begin{document}

% Portada
\begin{titlepage}
    \centering
    {\scshape\LARGE Universidad Politécnica de Madrid \par}
    \vspace{1cm}
    {\scshape\Large Escuela Técnica Superior de Ingenieros Industriales\par}
    \vspace{1.5cm}
    {\huge\bfseries Optimización energética de sistema híbrido con bomba de calor, suelo radiante, fotovoltaica y almacenamiento para vivienda \par}
    \vspace{1.5cm}
    {\Large\bfseries Trabajo de Fin de Máster\par}
    \vspace{0.5cm}
    {\large Máster Universitario en Ingeniería de la Energía \par}
    \vspace{2cm}
    {\Large Luis D. Aranda Sánchez\par}
    \vfill
    Director: Javier Rodríguez Martín
    \vfill
    {\large Septiembre 6, 2024\par}
\end{titlepage}

% Resumen (máximo de 5 páginas, incluyendo al final Palabras clave)
\clearpage
\pagestyle{simple}
% \newpage
\chapter*{Resumen}
\addcontentsline{toc}{chapter}{Resumen}
\input{capitulos/resumen/main.tex}

% Índice (paginado)
\clearpage
\pagestyle{simple}
% \newpage
\tableofcontents

% Introducción (donde se incluya los antecedentes y justificación)
\clearpage
\pagestyle{myfancy}
\newpage
\chapter{Introducción}
\input{capitulos/introduccion/main.tex}

% Objetivos
\chapter{Objetivos}
\input{capitulos/objetivos/main.tex}

% Metodología
\chapter{Metodología}
\input{capitulos/metodologia/main.tex}

% Resultados y discusión (incluyendo la valoración de impactos y de aspectos de responsabilidad legal, ética y profesional relacionados con el trabajo)
\chapter{Resultados y Discusión}
\input{capitulos/resultados_discusion/main.tex}

% Conclusiones
\chapter{Conclusiones}
\input{capitulos/conclusiones/main.tex}

% Planificación temporal y presupuesto
\chapter{Planificación Temporal y Presupuesto}
\input{capitulos/planificacion_presupuesto/main.tex}

% Bibliografía
\newpage
\addcontentsline{toc}{chapter}{Bibliografía}
\printbibliography

\end{document}


% Índice (paginado)
\clearpage
\pagestyle{simple}
% \newpage
\tableofcontents

% Introducción (donde se incluya los antecedentes y justificación)
\clearpage
\pagestyle{myfancy}
\newpage
\chapter{Introducción}
\documentclass[a4paper,11pt,twoside]{report}
\usepackage[left=25mm,right=25mm,top=25mm,bottom=25mm,includehead,includefoot,headsep=15mm,footskip=15mm]{geometry}
\usepackage{graphicx}
\usepackage{fancyhdr}
\usepackage{titlesec}
\usepackage[spanish]{babel}
\usepackage[utf8]{inputenc}
\usepackage{amsmath}
\usepackage{setspace}
\usepackage{svg}
\usepackage{hyperref}
\usepackage[backend=biber,style=numeric]{biblatex}
\addbibresource{references.bib}
\hypersetup{
    colorlinks=true,
    linkcolor=blue,      % color of internal links (sections, etc.)
    urlcolor=blue,       % color of external links
    pdftitle={Optimización energética de sistema híbrido con bomba de calor, suelo radiante, fotovoltaica y almacenamiento para vivienda},    % title
    pdfauthor={Luis D. Aranda Sánchez},     % author
    pdfkeywords={palabra1, palabra2, código1, etc.} % list of keywords
}

% Font change to Arial
\usepackage{helvet}
\renewcommand{\familydefault}{\sfdefault}

% Chapter titles in uppercase and larger font
\titleformat{\chapter}[hang]{\large\bfseries}{\thechapter.}{1em}{\MakeUppercase}
\titleformat{\section}[hang]{\bfseries}{\thesection.}{1em}{}
\titleformat{\subsection}[hang]{\bfseries}{\thesubsection.}{1em}{}

% Fancyhdr setup
\setlength{\headheight}{14.30174pt} % Adjust to recommended value, slightly larger for safety
\fancyhf{} % Clear all headers and footers
\fancyhead[LE]{\nouppercase{\leftmark}}
\fancyhead[RO]{Optimización energética para vivienda}
\fancyfoot[LE]{\thepage}
\fancyfoot[RE]{Escuela Técnica Superior de Ingenieros Industriales (UPM)}
\fancyfoot[LO]{Luis D. Aranda Sánchez}
\fancyfoot[RO]{\thepage}
\renewcommand{\headrulewidth}{0.4pt}
\renewcommand{\footrulewidth}{0.4pt}

\fancypagestyle{myfancy}{
    \fancyhf{} % Clear all headers and footers
    \fancyhead[LE]{\nouppercase{\leftmark}}
    \fancyhead[RO]{Optimización energética para vivienda}
    \fancyfoot[LE]{\thepage}
    \fancyfoot[RE]{Escuela Técnica Superior de Ingenieros Industriales (UPM)}
    \fancyfoot[LO]{Luis D. Aranda Sánchez}
    \fancyfoot[RO]{\thepage}
    \renewcommand{\headrulewidth}{0.4pt}
    \renewcommand{\footrulewidth}{0.4pt}
}

\fancypagestyle{simple}{
    \fancyhf{} % Clear all headers and footers
    \renewcommand{\headrulewidth}{0pt}
    \renewcommand{\footrulewidth}{0pt}
}

% Line spacing
\setstretch{1.2}

% Document starts here
\begin{document}

% Portada
\begin{titlepage}
    \centering
    {\scshape\LARGE Universidad Politécnica de Madrid \par}
    \vspace{1cm}
    {\scshape\Large Escuela Técnica Superior de Ingenieros Industriales\par}
    \vspace{1.5cm}
    {\huge\bfseries Optimización energética de sistema híbrido con bomba de calor, suelo radiante, fotovoltaica y almacenamiento para vivienda \par}
    \vspace{1.5cm}
    {\Large\bfseries Trabajo de Fin de Máster\par}
    \vspace{0.5cm}
    {\large Máster Universitario en Ingeniería de la Energía \par}
    \vspace{2cm}
    {\Large Luis D. Aranda Sánchez\par}
    \vfill
    Director: Javier Rodríguez Martín
    \vfill
    {\large Septiembre 6, 2024\par}
\end{titlepage}

% Resumen (máximo de 5 páginas, incluyendo al final Palabras clave)
\clearpage
\pagestyle{simple}
% \newpage
\chapter*{Resumen}
\addcontentsline{toc}{chapter}{Resumen}
\input{capitulos/resumen/main.tex}

% Índice (paginado)
\clearpage
\pagestyle{simple}
% \newpage
\tableofcontents

% Introducción (donde se incluya los antecedentes y justificación)
\clearpage
\pagestyle{myfancy}
\newpage
\chapter{Introducción}
\input{capitulos/introduccion/main.tex}

% Objetivos
\chapter{Objetivos}
\input{capitulos/objetivos/main.tex}

% Metodología
\chapter{Metodología}
\input{capitulos/metodologia/main.tex}

% Resultados y discusión (incluyendo la valoración de impactos y de aspectos de responsabilidad legal, ética y profesional relacionados con el trabajo)
\chapter{Resultados y Discusión}
\input{capitulos/resultados_discusion/main.tex}

% Conclusiones
\chapter{Conclusiones}
\input{capitulos/conclusiones/main.tex}

% Planificación temporal y presupuesto
\chapter{Planificación Temporal y Presupuesto}
\input{capitulos/planificacion_presupuesto/main.tex}

% Bibliografía
\newpage
\addcontentsline{toc}{chapter}{Bibliografía}
\printbibliography

\end{document}


% Objetivos
\chapter{Objetivos}
\documentclass[a4paper,11pt,twoside]{report}
\usepackage[left=25mm,right=25mm,top=25mm,bottom=25mm,includehead,includefoot,headsep=15mm,footskip=15mm]{geometry}
\usepackage{graphicx}
\usepackage{fancyhdr}
\usepackage{titlesec}
\usepackage[spanish]{babel}
\usepackage[utf8]{inputenc}
\usepackage{amsmath}
\usepackage{setspace}
\usepackage{svg}
\usepackage{hyperref}
\usepackage[backend=biber,style=numeric]{biblatex}
\addbibresource{references.bib}
\hypersetup{
    colorlinks=true,
    linkcolor=blue,      % color of internal links (sections, etc.)
    urlcolor=blue,       % color of external links
    pdftitle={Optimización energética de sistema híbrido con bomba de calor, suelo radiante, fotovoltaica y almacenamiento para vivienda},    % title
    pdfauthor={Luis D. Aranda Sánchez},     % author
    pdfkeywords={palabra1, palabra2, código1, etc.} % list of keywords
}

% Font change to Arial
\usepackage{helvet}
\renewcommand{\familydefault}{\sfdefault}

% Chapter titles in uppercase and larger font
\titleformat{\chapter}[hang]{\large\bfseries}{\thechapter.}{1em}{\MakeUppercase}
\titleformat{\section}[hang]{\bfseries}{\thesection.}{1em}{}
\titleformat{\subsection}[hang]{\bfseries}{\thesubsection.}{1em}{}

% Fancyhdr setup
\setlength{\headheight}{14.30174pt} % Adjust to recommended value, slightly larger for safety
\fancyhf{} % Clear all headers and footers
\fancyhead[LE]{\nouppercase{\leftmark}}
\fancyhead[RO]{Optimización energética para vivienda}
\fancyfoot[LE]{\thepage}
\fancyfoot[RE]{Escuela Técnica Superior de Ingenieros Industriales (UPM)}
\fancyfoot[LO]{Luis D. Aranda Sánchez}
\fancyfoot[RO]{\thepage}
\renewcommand{\headrulewidth}{0.4pt}
\renewcommand{\footrulewidth}{0.4pt}

\fancypagestyle{myfancy}{
    \fancyhf{} % Clear all headers and footers
    \fancyhead[LE]{\nouppercase{\leftmark}}
    \fancyhead[RO]{Optimización energética para vivienda}
    \fancyfoot[LE]{\thepage}
    \fancyfoot[RE]{Escuela Técnica Superior de Ingenieros Industriales (UPM)}
    \fancyfoot[LO]{Luis D. Aranda Sánchez}
    \fancyfoot[RO]{\thepage}
    \renewcommand{\headrulewidth}{0.4pt}
    \renewcommand{\footrulewidth}{0.4pt}
}

\fancypagestyle{simple}{
    \fancyhf{} % Clear all headers and footers
    \renewcommand{\headrulewidth}{0pt}
    \renewcommand{\footrulewidth}{0pt}
}

% Line spacing
\setstretch{1.2}

% Document starts here
\begin{document}

% Portada
\begin{titlepage}
    \centering
    {\scshape\LARGE Universidad Politécnica de Madrid \par}
    \vspace{1cm}
    {\scshape\Large Escuela Técnica Superior de Ingenieros Industriales\par}
    \vspace{1.5cm}
    {\huge\bfseries Optimización energética de sistema híbrido con bomba de calor, suelo radiante, fotovoltaica y almacenamiento para vivienda \par}
    \vspace{1.5cm}
    {\Large\bfseries Trabajo de Fin de Máster\par}
    \vspace{0.5cm}
    {\large Máster Universitario en Ingeniería de la Energía \par}
    \vspace{2cm}
    {\Large Luis D. Aranda Sánchez\par}
    \vfill
    Director: Javier Rodríguez Martín
    \vfill
    {\large Septiembre 6, 2024\par}
\end{titlepage}

% Resumen (máximo de 5 páginas, incluyendo al final Palabras clave)
\clearpage
\pagestyle{simple}
% \newpage
\chapter*{Resumen}
\addcontentsline{toc}{chapter}{Resumen}
\input{capitulos/resumen/main.tex}

% Índice (paginado)
\clearpage
\pagestyle{simple}
% \newpage
\tableofcontents

% Introducción (donde se incluya los antecedentes y justificación)
\clearpage
\pagestyle{myfancy}
\newpage
\chapter{Introducción}
\input{capitulos/introduccion/main.tex}

% Objetivos
\chapter{Objetivos}
\input{capitulos/objetivos/main.tex}

% Metodología
\chapter{Metodología}
\input{capitulos/metodologia/main.tex}

% Resultados y discusión (incluyendo la valoración de impactos y de aspectos de responsabilidad legal, ética y profesional relacionados con el trabajo)
\chapter{Resultados y Discusión}
\input{capitulos/resultados_discusion/main.tex}

% Conclusiones
\chapter{Conclusiones}
\input{capitulos/conclusiones/main.tex}

% Planificación temporal y presupuesto
\chapter{Planificación Temporal y Presupuesto}
\input{capitulos/planificacion_presupuesto/main.tex}

% Bibliografía
\newpage
\addcontentsline{toc}{chapter}{Bibliografía}
\printbibliography

\end{document}


% Metodología
\chapter{Metodología}
\documentclass[a4paper,11pt,twoside]{report}
\usepackage[left=25mm,right=25mm,top=25mm,bottom=25mm,includehead,includefoot,headsep=15mm,footskip=15mm]{geometry}
\usepackage{graphicx}
\usepackage{fancyhdr}
\usepackage{titlesec}
\usepackage[spanish]{babel}
\usepackage[utf8]{inputenc}
\usepackage{amsmath}
\usepackage{setspace}
\usepackage{svg}
\usepackage{hyperref}
\usepackage[backend=biber,style=numeric]{biblatex}
\addbibresource{references.bib}
\hypersetup{
    colorlinks=true,
    linkcolor=blue,      % color of internal links (sections, etc.)
    urlcolor=blue,       % color of external links
    pdftitle={Optimización energética de sistema híbrido con bomba de calor, suelo radiante, fotovoltaica y almacenamiento para vivienda},    % title
    pdfauthor={Luis D. Aranda Sánchez},     % author
    pdfkeywords={palabra1, palabra2, código1, etc.} % list of keywords
}

% Font change to Arial
\usepackage{helvet}
\renewcommand{\familydefault}{\sfdefault}

% Chapter titles in uppercase and larger font
\titleformat{\chapter}[hang]{\large\bfseries}{\thechapter.}{1em}{\MakeUppercase}
\titleformat{\section}[hang]{\bfseries}{\thesection.}{1em}{}
\titleformat{\subsection}[hang]{\bfseries}{\thesubsection.}{1em}{}

% Fancyhdr setup
\setlength{\headheight}{14.30174pt} % Adjust to recommended value, slightly larger for safety
\fancyhf{} % Clear all headers and footers
\fancyhead[LE]{\nouppercase{\leftmark}}
\fancyhead[RO]{Optimización energética para vivienda}
\fancyfoot[LE]{\thepage}
\fancyfoot[RE]{Escuela Técnica Superior de Ingenieros Industriales (UPM)}
\fancyfoot[LO]{Luis D. Aranda Sánchez}
\fancyfoot[RO]{\thepage}
\renewcommand{\headrulewidth}{0.4pt}
\renewcommand{\footrulewidth}{0.4pt}

\fancypagestyle{myfancy}{
    \fancyhf{} % Clear all headers and footers
    \fancyhead[LE]{\nouppercase{\leftmark}}
    \fancyhead[RO]{Optimización energética para vivienda}
    \fancyfoot[LE]{\thepage}
    \fancyfoot[RE]{Escuela Técnica Superior de Ingenieros Industriales (UPM)}
    \fancyfoot[LO]{Luis D. Aranda Sánchez}
    \fancyfoot[RO]{\thepage}
    \renewcommand{\headrulewidth}{0.4pt}
    \renewcommand{\footrulewidth}{0.4pt}
}

\fancypagestyle{simple}{
    \fancyhf{} % Clear all headers and footers
    \renewcommand{\headrulewidth}{0pt}
    \renewcommand{\footrulewidth}{0pt}
}

% Line spacing
\setstretch{1.2}

% Document starts here
\begin{document}

% Portada
\begin{titlepage}
    \centering
    {\scshape\LARGE Universidad Politécnica de Madrid \par}
    \vspace{1cm}
    {\scshape\Large Escuela Técnica Superior de Ingenieros Industriales\par}
    \vspace{1.5cm}
    {\huge\bfseries Optimización energética de sistema híbrido con bomba de calor, suelo radiante, fotovoltaica y almacenamiento para vivienda \par}
    \vspace{1.5cm}
    {\Large\bfseries Trabajo de Fin de Máster\par}
    \vspace{0.5cm}
    {\large Máster Universitario en Ingeniería de la Energía \par}
    \vspace{2cm}
    {\Large Luis D. Aranda Sánchez\par}
    \vfill
    Director: Javier Rodríguez Martín
    \vfill
    {\large Septiembre 6, 2024\par}
\end{titlepage}

% Resumen (máximo de 5 páginas, incluyendo al final Palabras clave)
\clearpage
\pagestyle{simple}
% \newpage
\chapter*{Resumen}
\addcontentsline{toc}{chapter}{Resumen}
\input{capitulos/resumen/main.tex}

% Índice (paginado)
\clearpage
\pagestyle{simple}
% \newpage
\tableofcontents

% Introducción (donde se incluya los antecedentes y justificación)
\clearpage
\pagestyle{myfancy}
\newpage
\chapter{Introducción}
\input{capitulos/introduccion/main.tex}

% Objetivos
\chapter{Objetivos}
\input{capitulos/objetivos/main.tex}

% Metodología
\chapter{Metodología}
\input{capitulos/metodologia/main.tex}

% Resultados y discusión (incluyendo la valoración de impactos y de aspectos de responsabilidad legal, ética y profesional relacionados con el trabajo)
\chapter{Resultados y Discusión}
\input{capitulos/resultados_discusion/main.tex}

% Conclusiones
\chapter{Conclusiones}
\input{capitulos/conclusiones/main.tex}

% Planificación temporal y presupuesto
\chapter{Planificación Temporal y Presupuesto}
\input{capitulos/planificacion_presupuesto/main.tex}

% Bibliografía
\newpage
\addcontentsline{toc}{chapter}{Bibliografía}
\printbibliography

\end{document}


% Resultados y discusión (incluyendo la valoración de impactos y de aspectos de responsabilidad legal, ética y profesional relacionados con el trabajo)
\chapter{Resultados y Discusión}
\documentclass[a4paper,11pt,twoside]{report}
\usepackage[left=25mm,right=25mm,top=25mm,bottom=25mm,includehead,includefoot,headsep=15mm,footskip=15mm]{geometry}
\usepackage{graphicx}
\usepackage{fancyhdr}
\usepackage{titlesec}
\usepackage[spanish]{babel}
\usepackage[utf8]{inputenc}
\usepackage{amsmath}
\usepackage{setspace}
\usepackage{svg}
\usepackage{hyperref}
\usepackage[backend=biber,style=numeric]{biblatex}
\addbibresource{references.bib}
\hypersetup{
    colorlinks=true,
    linkcolor=blue,      % color of internal links (sections, etc.)
    urlcolor=blue,       % color of external links
    pdftitle={Optimización energética de sistema híbrido con bomba de calor, suelo radiante, fotovoltaica y almacenamiento para vivienda},    % title
    pdfauthor={Luis D. Aranda Sánchez},     % author
    pdfkeywords={palabra1, palabra2, código1, etc.} % list of keywords
}

% Font change to Arial
\usepackage{helvet}
\renewcommand{\familydefault}{\sfdefault}

% Chapter titles in uppercase and larger font
\titleformat{\chapter}[hang]{\large\bfseries}{\thechapter.}{1em}{\MakeUppercase}
\titleformat{\section}[hang]{\bfseries}{\thesection.}{1em}{}
\titleformat{\subsection}[hang]{\bfseries}{\thesubsection.}{1em}{}

% Fancyhdr setup
\setlength{\headheight}{14.30174pt} % Adjust to recommended value, slightly larger for safety
\fancyhf{} % Clear all headers and footers
\fancyhead[LE]{\nouppercase{\leftmark}}
\fancyhead[RO]{Optimización energética para vivienda}
\fancyfoot[LE]{\thepage}
\fancyfoot[RE]{Escuela Técnica Superior de Ingenieros Industriales (UPM)}
\fancyfoot[LO]{Luis D. Aranda Sánchez}
\fancyfoot[RO]{\thepage}
\renewcommand{\headrulewidth}{0.4pt}
\renewcommand{\footrulewidth}{0.4pt}

\fancypagestyle{myfancy}{
    \fancyhf{} % Clear all headers and footers
    \fancyhead[LE]{\nouppercase{\leftmark}}
    \fancyhead[RO]{Optimización energética para vivienda}
    \fancyfoot[LE]{\thepage}
    \fancyfoot[RE]{Escuela Técnica Superior de Ingenieros Industriales (UPM)}
    \fancyfoot[LO]{Luis D. Aranda Sánchez}
    \fancyfoot[RO]{\thepage}
    \renewcommand{\headrulewidth}{0.4pt}
    \renewcommand{\footrulewidth}{0.4pt}
}

\fancypagestyle{simple}{
    \fancyhf{} % Clear all headers and footers
    \renewcommand{\headrulewidth}{0pt}
    \renewcommand{\footrulewidth}{0pt}
}

% Line spacing
\setstretch{1.2}

% Document starts here
\begin{document}

% Portada
\begin{titlepage}
    \centering
    {\scshape\LARGE Universidad Politécnica de Madrid \par}
    \vspace{1cm}
    {\scshape\Large Escuela Técnica Superior de Ingenieros Industriales\par}
    \vspace{1.5cm}
    {\huge\bfseries Optimización energética de sistema híbrido con bomba de calor, suelo radiante, fotovoltaica y almacenamiento para vivienda \par}
    \vspace{1.5cm}
    {\Large\bfseries Trabajo de Fin de Máster\par}
    \vspace{0.5cm}
    {\large Máster Universitario en Ingeniería de la Energía \par}
    \vspace{2cm}
    {\Large Luis D. Aranda Sánchez\par}
    \vfill
    Director: Javier Rodríguez Martín
    \vfill
    {\large Septiembre 6, 2024\par}
\end{titlepage}

% Resumen (máximo de 5 páginas, incluyendo al final Palabras clave)
\clearpage
\pagestyle{simple}
% \newpage
\chapter*{Resumen}
\addcontentsline{toc}{chapter}{Resumen}
\input{capitulos/resumen/main.tex}

% Índice (paginado)
\clearpage
\pagestyle{simple}
% \newpage
\tableofcontents

% Introducción (donde se incluya los antecedentes y justificación)
\clearpage
\pagestyle{myfancy}
\newpage
\chapter{Introducción}
\input{capitulos/introduccion/main.tex}

% Objetivos
\chapter{Objetivos}
\input{capitulos/objetivos/main.tex}

% Metodología
\chapter{Metodología}
\input{capitulos/metodologia/main.tex}

% Resultados y discusión (incluyendo la valoración de impactos y de aspectos de responsabilidad legal, ética y profesional relacionados con el trabajo)
\chapter{Resultados y Discusión}
\input{capitulos/resultados_discusion/main.tex}

% Conclusiones
\chapter{Conclusiones}
\input{capitulos/conclusiones/main.tex}

% Planificación temporal y presupuesto
\chapter{Planificación Temporal y Presupuesto}
\input{capitulos/planificacion_presupuesto/main.tex}

% Bibliografía
\newpage
\addcontentsline{toc}{chapter}{Bibliografía}
\printbibliography

\end{document}


% Conclusiones
\chapter{Conclusiones}
\documentclass[a4paper,11pt,twoside]{report}
\usepackage[left=25mm,right=25mm,top=25mm,bottom=25mm,includehead,includefoot,headsep=15mm,footskip=15mm]{geometry}
\usepackage{graphicx}
\usepackage{fancyhdr}
\usepackage{titlesec}
\usepackage[spanish]{babel}
\usepackage[utf8]{inputenc}
\usepackage{amsmath}
\usepackage{setspace}
\usepackage{svg}
\usepackage{hyperref}
\usepackage[backend=biber,style=numeric]{biblatex}
\addbibresource{references.bib}
\hypersetup{
    colorlinks=true,
    linkcolor=blue,      % color of internal links (sections, etc.)
    urlcolor=blue,       % color of external links
    pdftitle={Optimización energética de sistema híbrido con bomba de calor, suelo radiante, fotovoltaica y almacenamiento para vivienda},    % title
    pdfauthor={Luis D. Aranda Sánchez},     % author
    pdfkeywords={palabra1, palabra2, código1, etc.} % list of keywords
}

% Font change to Arial
\usepackage{helvet}
\renewcommand{\familydefault}{\sfdefault}

% Chapter titles in uppercase and larger font
\titleformat{\chapter}[hang]{\large\bfseries}{\thechapter.}{1em}{\MakeUppercase}
\titleformat{\section}[hang]{\bfseries}{\thesection.}{1em}{}
\titleformat{\subsection}[hang]{\bfseries}{\thesubsection.}{1em}{}

% Fancyhdr setup
\setlength{\headheight}{14.30174pt} % Adjust to recommended value, slightly larger for safety
\fancyhf{} % Clear all headers and footers
\fancyhead[LE]{\nouppercase{\leftmark}}
\fancyhead[RO]{Optimización energética para vivienda}
\fancyfoot[LE]{\thepage}
\fancyfoot[RE]{Escuela Técnica Superior de Ingenieros Industriales (UPM)}
\fancyfoot[LO]{Luis D. Aranda Sánchez}
\fancyfoot[RO]{\thepage}
\renewcommand{\headrulewidth}{0.4pt}
\renewcommand{\footrulewidth}{0.4pt}

\fancypagestyle{myfancy}{
    \fancyhf{} % Clear all headers and footers
    \fancyhead[LE]{\nouppercase{\leftmark}}
    \fancyhead[RO]{Optimización energética para vivienda}
    \fancyfoot[LE]{\thepage}
    \fancyfoot[RE]{Escuela Técnica Superior de Ingenieros Industriales (UPM)}
    \fancyfoot[LO]{Luis D. Aranda Sánchez}
    \fancyfoot[RO]{\thepage}
    \renewcommand{\headrulewidth}{0.4pt}
    \renewcommand{\footrulewidth}{0.4pt}
}

\fancypagestyle{simple}{
    \fancyhf{} % Clear all headers and footers
    \renewcommand{\headrulewidth}{0pt}
    \renewcommand{\footrulewidth}{0pt}
}

% Line spacing
\setstretch{1.2}

% Document starts here
\begin{document}

% Portada
\begin{titlepage}
    \centering
    {\scshape\LARGE Universidad Politécnica de Madrid \par}
    \vspace{1cm}
    {\scshape\Large Escuela Técnica Superior de Ingenieros Industriales\par}
    \vspace{1.5cm}
    {\huge\bfseries Optimización energética de sistema híbrido con bomba de calor, suelo radiante, fotovoltaica y almacenamiento para vivienda \par}
    \vspace{1.5cm}
    {\Large\bfseries Trabajo de Fin de Máster\par}
    \vspace{0.5cm}
    {\large Máster Universitario en Ingeniería de la Energía \par}
    \vspace{2cm}
    {\Large Luis D. Aranda Sánchez\par}
    \vfill
    Director: Javier Rodríguez Martín
    \vfill
    {\large Septiembre 6, 2024\par}
\end{titlepage}

% Resumen (máximo de 5 páginas, incluyendo al final Palabras clave)
\clearpage
\pagestyle{simple}
% \newpage
\chapter*{Resumen}
\addcontentsline{toc}{chapter}{Resumen}
\input{capitulos/resumen/main.tex}

% Índice (paginado)
\clearpage
\pagestyle{simple}
% \newpage
\tableofcontents

% Introducción (donde se incluya los antecedentes y justificación)
\clearpage
\pagestyle{myfancy}
\newpage
\chapter{Introducción}
\input{capitulos/introduccion/main.tex}

% Objetivos
\chapter{Objetivos}
\input{capitulos/objetivos/main.tex}

% Metodología
\chapter{Metodología}
\input{capitulos/metodologia/main.tex}

% Resultados y discusión (incluyendo la valoración de impactos y de aspectos de responsabilidad legal, ética y profesional relacionados con el trabajo)
\chapter{Resultados y Discusión}
\input{capitulos/resultados_discusion/main.tex}

% Conclusiones
\chapter{Conclusiones}
\input{capitulos/conclusiones/main.tex}

% Planificación temporal y presupuesto
\chapter{Planificación Temporal y Presupuesto}
\input{capitulos/planificacion_presupuesto/main.tex}

% Bibliografía
\newpage
\addcontentsline{toc}{chapter}{Bibliografía}
\printbibliography

\end{document}


% Planificación temporal y presupuesto
\chapter{Planificación Temporal y Presupuesto}
\documentclass[a4paper,11pt,twoside]{report}
\usepackage[left=25mm,right=25mm,top=25mm,bottom=25mm,includehead,includefoot,headsep=15mm,footskip=15mm]{geometry}
\usepackage{graphicx}
\usepackage{fancyhdr}
\usepackage{titlesec}
\usepackage[spanish]{babel}
\usepackage[utf8]{inputenc}
\usepackage{amsmath}
\usepackage{setspace}
\usepackage{svg}
\usepackage{hyperref}
\usepackage[backend=biber,style=numeric]{biblatex}
\addbibresource{references.bib}
\hypersetup{
    colorlinks=true,
    linkcolor=blue,      % color of internal links (sections, etc.)
    urlcolor=blue,       % color of external links
    pdftitle={Optimización energética de sistema híbrido con bomba de calor, suelo radiante, fotovoltaica y almacenamiento para vivienda},    % title
    pdfauthor={Luis D. Aranda Sánchez},     % author
    pdfkeywords={palabra1, palabra2, código1, etc.} % list of keywords
}

% Font change to Arial
\usepackage{helvet}
\renewcommand{\familydefault}{\sfdefault}

% Chapter titles in uppercase and larger font
\titleformat{\chapter}[hang]{\large\bfseries}{\thechapter.}{1em}{\MakeUppercase}
\titleformat{\section}[hang]{\bfseries}{\thesection.}{1em}{}
\titleformat{\subsection}[hang]{\bfseries}{\thesubsection.}{1em}{}

% Fancyhdr setup
\setlength{\headheight}{14.30174pt} % Adjust to recommended value, slightly larger for safety
\fancyhf{} % Clear all headers and footers
\fancyhead[LE]{\nouppercase{\leftmark}}
\fancyhead[RO]{Optimización energética para vivienda}
\fancyfoot[LE]{\thepage}
\fancyfoot[RE]{Escuela Técnica Superior de Ingenieros Industriales (UPM)}
\fancyfoot[LO]{Luis D. Aranda Sánchez}
\fancyfoot[RO]{\thepage}
\renewcommand{\headrulewidth}{0.4pt}
\renewcommand{\footrulewidth}{0.4pt}

\fancypagestyle{myfancy}{
    \fancyhf{} % Clear all headers and footers
    \fancyhead[LE]{\nouppercase{\leftmark}}
    \fancyhead[RO]{Optimización energética para vivienda}
    \fancyfoot[LE]{\thepage}
    \fancyfoot[RE]{Escuela Técnica Superior de Ingenieros Industriales (UPM)}
    \fancyfoot[LO]{Luis D. Aranda Sánchez}
    \fancyfoot[RO]{\thepage}
    \renewcommand{\headrulewidth}{0.4pt}
    \renewcommand{\footrulewidth}{0.4pt}
}

\fancypagestyle{simple}{
    \fancyhf{} % Clear all headers and footers
    \renewcommand{\headrulewidth}{0pt}
    \renewcommand{\footrulewidth}{0pt}
}

% Line spacing
\setstretch{1.2}

% Document starts here
\begin{document}

% Portada
\begin{titlepage}
    \centering
    {\scshape\LARGE Universidad Politécnica de Madrid \par}
    \vspace{1cm}
    {\scshape\Large Escuela Técnica Superior de Ingenieros Industriales\par}
    \vspace{1.5cm}
    {\huge\bfseries Optimización energética de sistema híbrido con bomba de calor, suelo radiante, fotovoltaica y almacenamiento para vivienda \par}
    \vspace{1.5cm}
    {\Large\bfseries Trabajo de Fin de Máster\par}
    \vspace{0.5cm}
    {\large Máster Universitario en Ingeniería de la Energía \par}
    \vspace{2cm}
    {\Large Luis D. Aranda Sánchez\par}
    \vfill
    Director: Javier Rodríguez Martín
    \vfill
    {\large Septiembre 6, 2024\par}
\end{titlepage}

% Resumen (máximo de 5 páginas, incluyendo al final Palabras clave)
\clearpage
\pagestyle{simple}
% \newpage
\chapter*{Resumen}
\addcontentsline{toc}{chapter}{Resumen}
\input{capitulos/resumen/main.tex}

% Índice (paginado)
\clearpage
\pagestyle{simple}
% \newpage
\tableofcontents

% Introducción (donde se incluya los antecedentes y justificación)
\clearpage
\pagestyle{myfancy}
\newpage
\chapter{Introducción}
\input{capitulos/introduccion/main.tex}

% Objetivos
\chapter{Objetivos}
\input{capitulos/objetivos/main.tex}

% Metodología
\chapter{Metodología}
\input{capitulos/metodologia/main.tex}

% Resultados y discusión (incluyendo la valoración de impactos y de aspectos de responsabilidad legal, ética y profesional relacionados con el trabajo)
\chapter{Resultados y Discusión}
\input{capitulos/resultados_discusion/main.tex}

% Conclusiones
\chapter{Conclusiones}
\input{capitulos/conclusiones/main.tex}

% Planificación temporal y presupuesto
\chapter{Planificación Temporal y Presupuesto}
\input{capitulos/planificacion_presupuesto/main.tex}

% Bibliografía
\newpage
\addcontentsline{toc}{chapter}{Bibliografía}
\printbibliography

\end{document}


% Bibliografía
\newpage
\addcontentsline{toc}{chapter}{Bibliografía}
\printbibliography

\end{document}


% Índice (paginado)
\clearpage
\pagestyle{simple}
% \newpage
\tableofcontents

% Introducción (donde se incluya los antecedentes y justificación)
\clearpage
\pagestyle{myfancy}
\newpage
\chapter{Introducción}
\documentclass[a4paper,11pt,twoside]{report}
\usepackage[left=25mm,right=25mm,top=25mm,bottom=25mm,includehead,includefoot,headsep=15mm,footskip=15mm]{geometry}
\usepackage{graphicx}
\usepackage{fancyhdr}
\usepackage{titlesec}
\usepackage[spanish]{babel}
\usepackage[utf8]{inputenc}
\usepackage{amsmath}
\usepackage{setspace}
\usepackage{svg}
\usepackage{hyperref}
\usepackage[backend=biber,style=numeric]{biblatex}
\addbibresource{references.bib}
\hypersetup{
    colorlinks=true,
    linkcolor=blue,      % color of internal links (sections, etc.)
    urlcolor=blue,       % color of external links
    pdftitle={Optimización energética de sistema híbrido con bomba de calor, suelo radiante, fotovoltaica y almacenamiento para vivienda},    % title
    pdfauthor={Luis D. Aranda Sánchez},     % author
    pdfkeywords={palabra1, palabra2, código1, etc.} % list of keywords
}

% Font change to Arial
\usepackage{helvet}
\renewcommand{\familydefault}{\sfdefault}

% Chapter titles in uppercase and larger font
\titleformat{\chapter}[hang]{\large\bfseries}{\thechapter.}{1em}{\MakeUppercase}
\titleformat{\section}[hang]{\bfseries}{\thesection.}{1em}{}
\titleformat{\subsection}[hang]{\bfseries}{\thesubsection.}{1em}{}

% Fancyhdr setup
\setlength{\headheight}{14.30174pt} % Adjust to recommended value, slightly larger for safety
\fancyhf{} % Clear all headers and footers
\fancyhead[LE]{\nouppercase{\leftmark}}
\fancyhead[RO]{Optimización energética para vivienda}
\fancyfoot[LE]{\thepage}
\fancyfoot[RE]{Escuela Técnica Superior de Ingenieros Industriales (UPM)}
\fancyfoot[LO]{Luis D. Aranda Sánchez}
\fancyfoot[RO]{\thepage}
\renewcommand{\headrulewidth}{0.4pt}
\renewcommand{\footrulewidth}{0.4pt}

\fancypagestyle{myfancy}{
    \fancyhf{} % Clear all headers and footers
    \fancyhead[LE]{\nouppercase{\leftmark}}
    \fancyhead[RO]{Optimización energética para vivienda}
    \fancyfoot[LE]{\thepage}
    \fancyfoot[RE]{Escuela Técnica Superior de Ingenieros Industriales (UPM)}
    \fancyfoot[LO]{Luis D. Aranda Sánchez}
    \fancyfoot[RO]{\thepage}
    \renewcommand{\headrulewidth}{0.4pt}
    \renewcommand{\footrulewidth}{0.4pt}
}

\fancypagestyle{simple}{
    \fancyhf{} % Clear all headers and footers
    \renewcommand{\headrulewidth}{0pt}
    \renewcommand{\footrulewidth}{0pt}
}

% Line spacing
\setstretch{1.2}

% Document starts here
\begin{document}

% Portada
\begin{titlepage}
    \centering
    {\scshape\LARGE Universidad Politécnica de Madrid \par}
    \vspace{1cm}
    {\scshape\Large Escuela Técnica Superior de Ingenieros Industriales\par}
    \vspace{1.5cm}
    {\huge\bfseries Optimización energética de sistema híbrido con bomba de calor, suelo radiante, fotovoltaica y almacenamiento para vivienda \par}
    \vspace{1.5cm}
    {\Large\bfseries Trabajo de Fin de Máster\par}
    \vspace{0.5cm}
    {\large Máster Universitario en Ingeniería de la Energía \par}
    \vspace{2cm}
    {\Large Luis D. Aranda Sánchez\par}
    \vfill
    Director: Javier Rodríguez Martín
    \vfill
    {\large Septiembre 6, 2024\par}
\end{titlepage}

% Resumen (máximo de 5 páginas, incluyendo al final Palabras clave)
\clearpage
\pagestyle{simple}
% \newpage
\chapter*{Resumen}
\addcontentsline{toc}{chapter}{Resumen}
\documentclass[a4paper,11pt,twoside]{report}
\usepackage[left=25mm,right=25mm,top=25mm,bottom=25mm,includehead,includefoot,headsep=15mm,footskip=15mm]{geometry}
\usepackage{graphicx}
\usepackage{fancyhdr}
\usepackage{titlesec}
\usepackage[spanish]{babel}
\usepackage[utf8]{inputenc}
\usepackage{amsmath}
\usepackage{setspace}
\usepackage{svg}
\usepackage{hyperref}
\usepackage[backend=biber,style=numeric]{biblatex}
\addbibresource{references.bib}
\hypersetup{
    colorlinks=true,
    linkcolor=blue,      % color of internal links (sections, etc.)
    urlcolor=blue,       % color of external links
    pdftitle={Optimización energética de sistema híbrido con bomba de calor, suelo radiante, fotovoltaica y almacenamiento para vivienda},    % title
    pdfauthor={Luis D. Aranda Sánchez},     % author
    pdfkeywords={palabra1, palabra2, código1, etc.} % list of keywords
}

% Font change to Arial
\usepackage{helvet}
\renewcommand{\familydefault}{\sfdefault}

% Chapter titles in uppercase and larger font
\titleformat{\chapter}[hang]{\large\bfseries}{\thechapter.}{1em}{\MakeUppercase}
\titleformat{\section}[hang]{\bfseries}{\thesection.}{1em}{}
\titleformat{\subsection}[hang]{\bfseries}{\thesubsection.}{1em}{}

% Fancyhdr setup
\setlength{\headheight}{14.30174pt} % Adjust to recommended value, slightly larger for safety
\fancyhf{} % Clear all headers and footers
\fancyhead[LE]{\nouppercase{\leftmark}}
\fancyhead[RO]{Optimización energética para vivienda}
\fancyfoot[LE]{\thepage}
\fancyfoot[RE]{Escuela Técnica Superior de Ingenieros Industriales (UPM)}
\fancyfoot[LO]{Luis D. Aranda Sánchez}
\fancyfoot[RO]{\thepage}
\renewcommand{\headrulewidth}{0.4pt}
\renewcommand{\footrulewidth}{0.4pt}

\fancypagestyle{myfancy}{
    \fancyhf{} % Clear all headers and footers
    \fancyhead[LE]{\nouppercase{\leftmark}}
    \fancyhead[RO]{Optimización energética para vivienda}
    \fancyfoot[LE]{\thepage}
    \fancyfoot[RE]{Escuela Técnica Superior de Ingenieros Industriales (UPM)}
    \fancyfoot[LO]{Luis D. Aranda Sánchez}
    \fancyfoot[RO]{\thepage}
    \renewcommand{\headrulewidth}{0.4pt}
    \renewcommand{\footrulewidth}{0.4pt}
}

\fancypagestyle{simple}{
    \fancyhf{} % Clear all headers and footers
    \renewcommand{\headrulewidth}{0pt}
    \renewcommand{\footrulewidth}{0pt}
}

% Line spacing
\setstretch{1.2}

% Document starts here
\begin{document}

% Portada
\begin{titlepage}
    \centering
    {\scshape\LARGE Universidad Politécnica de Madrid \par}
    \vspace{1cm}
    {\scshape\Large Escuela Técnica Superior de Ingenieros Industriales\par}
    \vspace{1.5cm}
    {\huge\bfseries Optimización energética de sistema híbrido con bomba de calor, suelo radiante, fotovoltaica y almacenamiento para vivienda \par}
    \vspace{1.5cm}
    {\Large\bfseries Trabajo de Fin de Máster\par}
    \vspace{0.5cm}
    {\large Máster Universitario en Ingeniería de la Energía \par}
    \vspace{2cm}
    {\Large Luis D. Aranda Sánchez\par}
    \vfill
    Director: Javier Rodríguez Martín
    \vfill
    {\large Septiembre 6, 2024\par}
\end{titlepage}

% Resumen (máximo de 5 páginas, incluyendo al final Palabras clave)
\clearpage
\pagestyle{simple}
% \newpage
\chapter*{Resumen}
\addcontentsline{toc}{chapter}{Resumen}
\input{capitulos/resumen/main.tex}

% Índice (paginado)
\clearpage
\pagestyle{simple}
% \newpage
\tableofcontents

% Introducción (donde se incluya los antecedentes y justificación)
\clearpage
\pagestyle{myfancy}
\newpage
\chapter{Introducción}
\input{capitulos/introduccion/main.tex}

% Objetivos
\chapter{Objetivos}
\input{capitulos/objetivos/main.tex}

% Metodología
\chapter{Metodología}
\input{capitulos/metodologia/main.tex}

% Resultados y discusión (incluyendo la valoración de impactos y de aspectos de responsabilidad legal, ética y profesional relacionados con el trabajo)
\chapter{Resultados y Discusión}
\input{capitulos/resultados_discusion/main.tex}

% Conclusiones
\chapter{Conclusiones}
\input{capitulos/conclusiones/main.tex}

% Planificación temporal y presupuesto
\chapter{Planificación Temporal y Presupuesto}
\input{capitulos/planificacion_presupuesto/main.tex}

% Bibliografía
\newpage
\addcontentsline{toc}{chapter}{Bibliografía}
\printbibliography

\end{document}


% Índice (paginado)
\clearpage
\pagestyle{simple}
% \newpage
\tableofcontents

% Introducción (donde se incluya los antecedentes y justificación)
\clearpage
\pagestyle{myfancy}
\newpage
\chapter{Introducción}
\documentclass[a4paper,11pt,twoside]{report}
\usepackage[left=25mm,right=25mm,top=25mm,bottom=25mm,includehead,includefoot,headsep=15mm,footskip=15mm]{geometry}
\usepackage{graphicx}
\usepackage{fancyhdr}
\usepackage{titlesec}
\usepackage[spanish]{babel}
\usepackage[utf8]{inputenc}
\usepackage{amsmath}
\usepackage{setspace}
\usepackage{svg}
\usepackage{hyperref}
\usepackage[backend=biber,style=numeric]{biblatex}
\addbibresource{references.bib}
\hypersetup{
    colorlinks=true,
    linkcolor=blue,      % color of internal links (sections, etc.)
    urlcolor=blue,       % color of external links
    pdftitle={Optimización energética de sistema híbrido con bomba de calor, suelo radiante, fotovoltaica y almacenamiento para vivienda},    % title
    pdfauthor={Luis D. Aranda Sánchez},     % author
    pdfkeywords={palabra1, palabra2, código1, etc.} % list of keywords
}

% Font change to Arial
\usepackage{helvet}
\renewcommand{\familydefault}{\sfdefault}

% Chapter titles in uppercase and larger font
\titleformat{\chapter}[hang]{\large\bfseries}{\thechapter.}{1em}{\MakeUppercase}
\titleformat{\section}[hang]{\bfseries}{\thesection.}{1em}{}
\titleformat{\subsection}[hang]{\bfseries}{\thesubsection.}{1em}{}

% Fancyhdr setup
\setlength{\headheight}{14.30174pt} % Adjust to recommended value, slightly larger for safety
\fancyhf{} % Clear all headers and footers
\fancyhead[LE]{\nouppercase{\leftmark}}
\fancyhead[RO]{Optimización energética para vivienda}
\fancyfoot[LE]{\thepage}
\fancyfoot[RE]{Escuela Técnica Superior de Ingenieros Industriales (UPM)}
\fancyfoot[LO]{Luis D. Aranda Sánchez}
\fancyfoot[RO]{\thepage}
\renewcommand{\headrulewidth}{0.4pt}
\renewcommand{\footrulewidth}{0.4pt}

\fancypagestyle{myfancy}{
    \fancyhf{} % Clear all headers and footers
    \fancyhead[LE]{\nouppercase{\leftmark}}
    \fancyhead[RO]{Optimización energética para vivienda}
    \fancyfoot[LE]{\thepage}
    \fancyfoot[RE]{Escuela Técnica Superior de Ingenieros Industriales (UPM)}
    \fancyfoot[LO]{Luis D. Aranda Sánchez}
    \fancyfoot[RO]{\thepage}
    \renewcommand{\headrulewidth}{0.4pt}
    \renewcommand{\footrulewidth}{0.4pt}
}

\fancypagestyle{simple}{
    \fancyhf{} % Clear all headers and footers
    \renewcommand{\headrulewidth}{0pt}
    \renewcommand{\footrulewidth}{0pt}
}

% Line spacing
\setstretch{1.2}

% Document starts here
\begin{document}

% Portada
\begin{titlepage}
    \centering
    {\scshape\LARGE Universidad Politécnica de Madrid \par}
    \vspace{1cm}
    {\scshape\Large Escuela Técnica Superior de Ingenieros Industriales\par}
    \vspace{1.5cm}
    {\huge\bfseries Optimización energética de sistema híbrido con bomba de calor, suelo radiante, fotovoltaica y almacenamiento para vivienda \par}
    \vspace{1.5cm}
    {\Large\bfseries Trabajo de Fin de Máster\par}
    \vspace{0.5cm}
    {\large Máster Universitario en Ingeniería de la Energía \par}
    \vspace{2cm}
    {\Large Luis D. Aranda Sánchez\par}
    \vfill
    Director: Javier Rodríguez Martín
    \vfill
    {\large Septiembre 6, 2024\par}
\end{titlepage}

% Resumen (máximo de 5 páginas, incluyendo al final Palabras clave)
\clearpage
\pagestyle{simple}
% \newpage
\chapter*{Resumen}
\addcontentsline{toc}{chapter}{Resumen}
\input{capitulos/resumen/main.tex}

% Índice (paginado)
\clearpage
\pagestyle{simple}
% \newpage
\tableofcontents

% Introducción (donde se incluya los antecedentes y justificación)
\clearpage
\pagestyle{myfancy}
\newpage
\chapter{Introducción}
\input{capitulos/introduccion/main.tex}

% Objetivos
\chapter{Objetivos}
\input{capitulos/objetivos/main.tex}

% Metodología
\chapter{Metodología}
\input{capitulos/metodologia/main.tex}

% Resultados y discusión (incluyendo la valoración de impactos y de aspectos de responsabilidad legal, ética y profesional relacionados con el trabajo)
\chapter{Resultados y Discusión}
\input{capitulos/resultados_discusion/main.tex}

% Conclusiones
\chapter{Conclusiones}
\input{capitulos/conclusiones/main.tex}

% Planificación temporal y presupuesto
\chapter{Planificación Temporal y Presupuesto}
\input{capitulos/planificacion_presupuesto/main.tex}

% Bibliografía
\newpage
\addcontentsline{toc}{chapter}{Bibliografía}
\printbibliography

\end{document}


% Objetivos
\chapter{Objetivos}
\documentclass[a4paper,11pt,twoside]{report}
\usepackage[left=25mm,right=25mm,top=25mm,bottom=25mm,includehead,includefoot,headsep=15mm,footskip=15mm]{geometry}
\usepackage{graphicx}
\usepackage{fancyhdr}
\usepackage{titlesec}
\usepackage[spanish]{babel}
\usepackage[utf8]{inputenc}
\usepackage{amsmath}
\usepackage{setspace}
\usepackage{svg}
\usepackage{hyperref}
\usepackage[backend=biber,style=numeric]{biblatex}
\addbibresource{references.bib}
\hypersetup{
    colorlinks=true,
    linkcolor=blue,      % color of internal links (sections, etc.)
    urlcolor=blue,       % color of external links
    pdftitle={Optimización energética de sistema híbrido con bomba de calor, suelo radiante, fotovoltaica y almacenamiento para vivienda},    % title
    pdfauthor={Luis D. Aranda Sánchez},     % author
    pdfkeywords={palabra1, palabra2, código1, etc.} % list of keywords
}

% Font change to Arial
\usepackage{helvet}
\renewcommand{\familydefault}{\sfdefault}

% Chapter titles in uppercase and larger font
\titleformat{\chapter}[hang]{\large\bfseries}{\thechapter.}{1em}{\MakeUppercase}
\titleformat{\section}[hang]{\bfseries}{\thesection.}{1em}{}
\titleformat{\subsection}[hang]{\bfseries}{\thesubsection.}{1em}{}

% Fancyhdr setup
\setlength{\headheight}{14.30174pt} % Adjust to recommended value, slightly larger for safety
\fancyhf{} % Clear all headers and footers
\fancyhead[LE]{\nouppercase{\leftmark}}
\fancyhead[RO]{Optimización energética para vivienda}
\fancyfoot[LE]{\thepage}
\fancyfoot[RE]{Escuela Técnica Superior de Ingenieros Industriales (UPM)}
\fancyfoot[LO]{Luis D. Aranda Sánchez}
\fancyfoot[RO]{\thepage}
\renewcommand{\headrulewidth}{0.4pt}
\renewcommand{\footrulewidth}{0.4pt}

\fancypagestyle{myfancy}{
    \fancyhf{} % Clear all headers and footers
    \fancyhead[LE]{\nouppercase{\leftmark}}
    \fancyhead[RO]{Optimización energética para vivienda}
    \fancyfoot[LE]{\thepage}
    \fancyfoot[RE]{Escuela Técnica Superior de Ingenieros Industriales (UPM)}
    \fancyfoot[LO]{Luis D. Aranda Sánchez}
    \fancyfoot[RO]{\thepage}
    \renewcommand{\headrulewidth}{0.4pt}
    \renewcommand{\footrulewidth}{0.4pt}
}

\fancypagestyle{simple}{
    \fancyhf{} % Clear all headers and footers
    \renewcommand{\headrulewidth}{0pt}
    \renewcommand{\footrulewidth}{0pt}
}

% Line spacing
\setstretch{1.2}

% Document starts here
\begin{document}

% Portada
\begin{titlepage}
    \centering
    {\scshape\LARGE Universidad Politécnica de Madrid \par}
    \vspace{1cm}
    {\scshape\Large Escuela Técnica Superior de Ingenieros Industriales\par}
    \vspace{1.5cm}
    {\huge\bfseries Optimización energética de sistema híbrido con bomba de calor, suelo radiante, fotovoltaica y almacenamiento para vivienda \par}
    \vspace{1.5cm}
    {\Large\bfseries Trabajo de Fin de Máster\par}
    \vspace{0.5cm}
    {\large Máster Universitario en Ingeniería de la Energía \par}
    \vspace{2cm}
    {\Large Luis D. Aranda Sánchez\par}
    \vfill
    Director: Javier Rodríguez Martín
    \vfill
    {\large Septiembre 6, 2024\par}
\end{titlepage}

% Resumen (máximo de 5 páginas, incluyendo al final Palabras clave)
\clearpage
\pagestyle{simple}
% \newpage
\chapter*{Resumen}
\addcontentsline{toc}{chapter}{Resumen}
\input{capitulos/resumen/main.tex}

% Índice (paginado)
\clearpage
\pagestyle{simple}
% \newpage
\tableofcontents

% Introducción (donde se incluya los antecedentes y justificación)
\clearpage
\pagestyle{myfancy}
\newpage
\chapter{Introducción}
\input{capitulos/introduccion/main.tex}

% Objetivos
\chapter{Objetivos}
\input{capitulos/objetivos/main.tex}

% Metodología
\chapter{Metodología}
\input{capitulos/metodologia/main.tex}

% Resultados y discusión (incluyendo la valoración de impactos y de aspectos de responsabilidad legal, ética y profesional relacionados con el trabajo)
\chapter{Resultados y Discusión}
\input{capitulos/resultados_discusion/main.tex}

% Conclusiones
\chapter{Conclusiones}
\input{capitulos/conclusiones/main.tex}

% Planificación temporal y presupuesto
\chapter{Planificación Temporal y Presupuesto}
\input{capitulos/planificacion_presupuesto/main.tex}

% Bibliografía
\newpage
\addcontentsline{toc}{chapter}{Bibliografía}
\printbibliography

\end{document}


% Metodología
\chapter{Metodología}
\documentclass[a4paper,11pt,twoside]{report}
\usepackage[left=25mm,right=25mm,top=25mm,bottom=25mm,includehead,includefoot,headsep=15mm,footskip=15mm]{geometry}
\usepackage{graphicx}
\usepackage{fancyhdr}
\usepackage{titlesec}
\usepackage[spanish]{babel}
\usepackage[utf8]{inputenc}
\usepackage{amsmath}
\usepackage{setspace}
\usepackage{svg}
\usepackage{hyperref}
\usepackage[backend=biber,style=numeric]{biblatex}
\addbibresource{references.bib}
\hypersetup{
    colorlinks=true,
    linkcolor=blue,      % color of internal links (sections, etc.)
    urlcolor=blue,       % color of external links
    pdftitle={Optimización energética de sistema híbrido con bomba de calor, suelo radiante, fotovoltaica y almacenamiento para vivienda},    % title
    pdfauthor={Luis D. Aranda Sánchez},     % author
    pdfkeywords={palabra1, palabra2, código1, etc.} % list of keywords
}

% Font change to Arial
\usepackage{helvet}
\renewcommand{\familydefault}{\sfdefault}

% Chapter titles in uppercase and larger font
\titleformat{\chapter}[hang]{\large\bfseries}{\thechapter.}{1em}{\MakeUppercase}
\titleformat{\section}[hang]{\bfseries}{\thesection.}{1em}{}
\titleformat{\subsection}[hang]{\bfseries}{\thesubsection.}{1em}{}

% Fancyhdr setup
\setlength{\headheight}{14.30174pt} % Adjust to recommended value, slightly larger for safety
\fancyhf{} % Clear all headers and footers
\fancyhead[LE]{\nouppercase{\leftmark}}
\fancyhead[RO]{Optimización energética para vivienda}
\fancyfoot[LE]{\thepage}
\fancyfoot[RE]{Escuela Técnica Superior de Ingenieros Industriales (UPM)}
\fancyfoot[LO]{Luis D. Aranda Sánchez}
\fancyfoot[RO]{\thepage}
\renewcommand{\headrulewidth}{0.4pt}
\renewcommand{\footrulewidth}{0.4pt}

\fancypagestyle{myfancy}{
    \fancyhf{} % Clear all headers and footers
    \fancyhead[LE]{\nouppercase{\leftmark}}
    \fancyhead[RO]{Optimización energética para vivienda}
    \fancyfoot[LE]{\thepage}
    \fancyfoot[RE]{Escuela Técnica Superior de Ingenieros Industriales (UPM)}
    \fancyfoot[LO]{Luis D. Aranda Sánchez}
    \fancyfoot[RO]{\thepage}
    \renewcommand{\headrulewidth}{0.4pt}
    \renewcommand{\footrulewidth}{0.4pt}
}

\fancypagestyle{simple}{
    \fancyhf{} % Clear all headers and footers
    \renewcommand{\headrulewidth}{0pt}
    \renewcommand{\footrulewidth}{0pt}
}

% Line spacing
\setstretch{1.2}

% Document starts here
\begin{document}

% Portada
\begin{titlepage}
    \centering
    {\scshape\LARGE Universidad Politécnica de Madrid \par}
    \vspace{1cm}
    {\scshape\Large Escuela Técnica Superior de Ingenieros Industriales\par}
    \vspace{1.5cm}
    {\huge\bfseries Optimización energética de sistema híbrido con bomba de calor, suelo radiante, fotovoltaica y almacenamiento para vivienda \par}
    \vspace{1.5cm}
    {\Large\bfseries Trabajo de Fin de Máster\par}
    \vspace{0.5cm}
    {\large Máster Universitario en Ingeniería de la Energía \par}
    \vspace{2cm}
    {\Large Luis D. Aranda Sánchez\par}
    \vfill
    Director: Javier Rodríguez Martín
    \vfill
    {\large Septiembre 6, 2024\par}
\end{titlepage}

% Resumen (máximo de 5 páginas, incluyendo al final Palabras clave)
\clearpage
\pagestyle{simple}
% \newpage
\chapter*{Resumen}
\addcontentsline{toc}{chapter}{Resumen}
\input{capitulos/resumen/main.tex}

% Índice (paginado)
\clearpage
\pagestyle{simple}
% \newpage
\tableofcontents

% Introducción (donde se incluya los antecedentes y justificación)
\clearpage
\pagestyle{myfancy}
\newpage
\chapter{Introducción}
\input{capitulos/introduccion/main.tex}

% Objetivos
\chapter{Objetivos}
\input{capitulos/objetivos/main.tex}

% Metodología
\chapter{Metodología}
\input{capitulos/metodologia/main.tex}

% Resultados y discusión (incluyendo la valoración de impactos y de aspectos de responsabilidad legal, ética y profesional relacionados con el trabajo)
\chapter{Resultados y Discusión}
\input{capitulos/resultados_discusion/main.tex}

% Conclusiones
\chapter{Conclusiones}
\input{capitulos/conclusiones/main.tex}

% Planificación temporal y presupuesto
\chapter{Planificación Temporal y Presupuesto}
\input{capitulos/planificacion_presupuesto/main.tex}

% Bibliografía
\newpage
\addcontentsline{toc}{chapter}{Bibliografía}
\printbibliography

\end{document}


% Resultados y discusión (incluyendo la valoración de impactos y de aspectos de responsabilidad legal, ética y profesional relacionados con el trabajo)
\chapter{Resultados y Discusión}
\documentclass[a4paper,11pt,twoside]{report}
\usepackage[left=25mm,right=25mm,top=25mm,bottom=25mm,includehead,includefoot,headsep=15mm,footskip=15mm]{geometry}
\usepackage{graphicx}
\usepackage{fancyhdr}
\usepackage{titlesec}
\usepackage[spanish]{babel}
\usepackage[utf8]{inputenc}
\usepackage{amsmath}
\usepackage{setspace}
\usepackage{svg}
\usepackage{hyperref}
\usepackage[backend=biber,style=numeric]{biblatex}
\addbibresource{references.bib}
\hypersetup{
    colorlinks=true,
    linkcolor=blue,      % color of internal links (sections, etc.)
    urlcolor=blue,       % color of external links
    pdftitle={Optimización energética de sistema híbrido con bomba de calor, suelo radiante, fotovoltaica y almacenamiento para vivienda},    % title
    pdfauthor={Luis D. Aranda Sánchez},     % author
    pdfkeywords={palabra1, palabra2, código1, etc.} % list of keywords
}

% Font change to Arial
\usepackage{helvet}
\renewcommand{\familydefault}{\sfdefault}

% Chapter titles in uppercase and larger font
\titleformat{\chapter}[hang]{\large\bfseries}{\thechapter.}{1em}{\MakeUppercase}
\titleformat{\section}[hang]{\bfseries}{\thesection.}{1em}{}
\titleformat{\subsection}[hang]{\bfseries}{\thesubsection.}{1em}{}

% Fancyhdr setup
\setlength{\headheight}{14.30174pt} % Adjust to recommended value, slightly larger for safety
\fancyhf{} % Clear all headers and footers
\fancyhead[LE]{\nouppercase{\leftmark}}
\fancyhead[RO]{Optimización energética para vivienda}
\fancyfoot[LE]{\thepage}
\fancyfoot[RE]{Escuela Técnica Superior de Ingenieros Industriales (UPM)}
\fancyfoot[LO]{Luis D. Aranda Sánchez}
\fancyfoot[RO]{\thepage}
\renewcommand{\headrulewidth}{0.4pt}
\renewcommand{\footrulewidth}{0.4pt}

\fancypagestyle{myfancy}{
    \fancyhf{} % Clear all headers and footers
    \fancyhead[LE]{\nouppercase{\leftmark}}
    \fancyhead[RO]{Optimización energética para vivienda}
    \fancyfoot[LE]{\thepage}
    \fancyfoot[RE]{Escuela Técnica Superior de Ingenieros Industriales (UPM)}
    \fancyfoot[LO]{Luis D. Aranda Sánchez}
    \fancyfoot[RO]{\thepage}
    \renewcommand{\headrulewidth}{0.4pt}
    \renewcommand{\footrulewidth}{0.4pt}
}

\fancypagestyle{simple}{
    \fancyhf{} % Clear all headers and footers
    \renewcommand{\headrulewidth}{0pt}
    \renewcommand{\footrulewidth}{0pt}
}

% Line spacing
\setstretch{1.2}

% Document starts here
\begin{document}

% Portada
\begin{titlepage}
    \centering
    {\scshape\LARGE Universidad Politécnica de Madrid \par}
    \vspace{1cm}
    {\scshape\Large Escuela Técnica Superior de Ingenieros Industriales\par}
    \vspace{1.5cm}
    {\huge\bfseries Optimización energética de sistema híbrido con bomba de calor, suelo radiante, fotovoltaica y almacenamiento para vivienda \par}
    \vspace{1.5cm}
    {\Large\bfseries Trabajo de Fin de Máster\par}
    \vspace{0.5cm}
    {\large Máster Universitario en Ingeniería de la Energía \par}
    \vspace{2cm}
    {\Large Luis D. Aranda Sánchez\par}
    \vfill
    Director: Javier Rodríguez Martín
    \vfill
    {\large Septiembre 6, 2024\par}
\end{titlepage}

% Resumen (máximo de 5 páginas, incluyendo al final Palabras clave)
\clearpage
\pagestyle{simple}
% \newpage
\chapter*{Resumen}
\addcontentsline{toc}{chapter}{Resumen}
\input{capitulos/resumen/main.tex}

% Índice (paginado)
\clearpage
\pagestyle{simple}
% \newpage
\tableofcontents

% Introducción (donde se incluya los antecedentes y justificación)
\clearpage
\pagestyle{myfancy}
\newpage
\chapter{Introducción}
\input{capitulos/introduccion/main.tex}

% Objetivos
\chapter{Objetivos}
\input{capitulos/objetivos/main.tex}

% Metodología
\chapter{Metodología}
\input{capitulos/metodologia/main.tex}

% Resultados y discusión (incluyendo la valoración de impactos y de aspectos de responsabilidad legal, ética y profesional relacionados con el trabajo)
\chapter{Resultados y Discusión}
\input{capitulos/resultados_discusion/main.tex}

% Conclusiones
\chapter{Conclusiones}
\input{capitulos/conclusiones/main.tex}

% Planificación temporal y presupuesto
\chapter{Planificación Temporal y Presupuesto}
\input{capitulos/planificacion_presupuesto/main.tex}

% Bibliografía
\newpage
\addcontentsline{toc}{chapter}{Bibliografía}
\printbibliography

\end{document}


% Conclusiones
\chapter{Conclusiones}
\documentclass[a4paper,11pt,twoside]{report}
\usepackage[left=25mm,right=25mm,top=25mm,bottom=25mm,includehead,includefoot,headsep=15mm,footskip=15mm]{geometry}
\usepackage{graphicx}
\usepackage{fancyhdr}
\usepackage{titlesec}
\usepackage[spanish]{babel}
\usepackage[utf8]{inputenc}
\usepackage{amsmath}
\usepackage{setspace}
\usepackage{svg}
\usepackage{hyperref}
\usepackage[backend=biber,style=numeric]{biblatex}
\addbibresource{references.bib}
\hypersetup{
    colorlinks=true,
    linkcolor=blue,      % color of internal links (sections, etc.)
    urlcolor=blue,       % color of external links
    pdftitle={Optimización energética de sistema híbrido con bomba de calor, suelo radiante, fotovoltaica y almacenamiento para vivienda},    % title
    pdfauthor={Luis D. Aranda Sánchez},     % author
    pdfkeywords={palabra1, palabra2, código1, etc.} % list of keywords
}

% Font change to Arial
\usepackage{helvet}
\renewcommand{\familydefault}{\sfdefault}

% Chapter titles in uppercase and larger font
\titleformat{\chapter}[hang]{\large\bfseries}{\thechapter.}{1em}{\MakeUppercase}
\titleformat{\section}[hang]{\bfseries}{\thesection.}{1em}{}
\titleformat{\subsection}[hang]{\bfseries}{\thesubsection.}{1em}{}

% Fancyhdr setup
\setlength{\headheight}{14.30174pt} % Adjust to recommended value, slightly larger for safety
\fancyhf{} % Clear all headers and footers
\fancyhead[LE]{\nouppercase{\leftmark}}
\fancyhead[RO]{Optimización energética para vivienda}
\fancyfoot[LE]{\thepage}
\fancyfoot[RE]{Escuela Técnica Superior de Ingenieros Industriales (UPM)}
\fancyfoot[LO]{Luis D. Aranda Sánchez}
\fancyfoot[RO]{\thepage}
\renewcommand{\headrulewidth}{0.4pt}
\renewcommand{\footrulewidth}{0.4pt}

\fancypagestyle{myfancy}{
    \fancyhf{} % Clear all headers and footers
    \fancyhead[LE]{\nouppercase{\leftmark}}
    \fancyhead[RO]{Optimización energética para vivienda}
    \fancyfoot[LE]{\thepage}
    \fancyfoot[RE]{Escuela Técnica Superior de Ingenieros Industriales (UPM)}
    \fancyfoot[LO]{Luis D. Aranda Sánchez}
    \fancyfoot[RO]{\thepage}
    \renewcommand{\headrulewidth}{0.4pt}
    \renewcommand{\footrulewidth}{0.4pt}
}

\fancypagestyle{simple}{
    \fancyhf{} % Clear all headers and footers
    \renewcommand{\headrulewidth}{0pt}
    \renewcommand{\footrulewidth}{0pt}
}

% Line spacing
\setstretch{1.2}

% Document starts here
\begin{document}

% Portada
\begin{titlepage}
    \centering
    {\scshape\LARGE Universidad Politécnica de Madrid \par}
    \vspace{1cm}
    {\scshape\Large Escuela Técnica Superior de Ingenieros Industriales\par}
    \vspace{1.5cm}
    {\huge\bfseries Optimización energética de sistema híbrido con bomba de calor, suelo radiante, fotovoltaica y almacenamiento para vivienda \par}
    \vspace{1.5cm}
    {\Large\bfseries Trabajo de Fin de Máster\par}
    \vspace{0.5cm}
    {\large Máster Universitario en Ingeniería de la Energía \par}
    \vspace{2cm}
    {\Large Luis D. Aranda Sánchez\par}
    \vfill
    Director: Javier Rodríguez Martín
    \vfill
    {\large Septiembre 6, 2024\par}
\end{titlepage}

% Resumen (máximo de 5 páginas, incluyendo al final Palabras clave)
\clearpage
\pagestyle{simple}
% \newpage
\chapter*{Resumen}
\addcontentsline{toc}{chapter}{Resumen}
\input{capitulos/resumen/main.tex}

% Índice (paginado)
\clearpage
\pagestyle{simple}
% \newpage
\tableofcontents

% Introducción (donde se incluya los antecedentes y justificación)
\clearpage
\pagestyle{myfancy}
\newpage
\chapter{Introducción}
\input{capitulos/introduccion/main.tex}

% Objetivos
\chapter{Objetivos}
\input{capitulos/objetivos/main.tex}

% Metodología
\chapter{Metodología}
\input{capitulos/metodologia/main.tex}

% Resultados y discusión (incluyendo la valoración de impactos y de aspectos de responsabilidad legal, ética y profesional relacionados con el trabajo)
\chapter{Resultados y Discusión}
\input{capitulos/resultados_discusion/main.tex}

% Conclusiones
\chapter{Conclusiones}
\input{capitulos/conclusiones/main.tex}

% Planificación temporal y presupuesto
\chapter{Planificación Temporal y Presupuesto}
\input{capitulos/planificacion_presupuesto/main.tex}

% Bibliografía
\newpage
\addcontentsline{toc}{chapter}{Bibliografía}
\printbibliography

\end{document}


% Planificación temporal y presupuesto
\chapter{Planificación Temporal y Presupuesto}
\documentclass[a4paper,11pt,twoside]{report}
\usepackage[left=25mm,right=25mm,top=25mm,bottom=25mm,includehead,includefoot,headsep=15mm,footskip=15mm]{geometry}
\usepackage{graphicx}
\usepackage{fancyhdr}
\usepackage{titlesec}
\usepackage[spanish]{babel}
\usepackage[utf8]{inputenc}
\usepackage{amsmath}
\usepackage{setspace}
\usepackage{svg}
\usepackage{hyperref}
\usepackage[backend=biber,style=numeric]{biblatex}
\addbibresource{references.bib}
\hypersetup{
    colorlinks=true,
    linkcolor=blue,      % color of internal links (sections, etc.)
    urlcolor=blue,       % color of external links
    pdftitle={Optimización energética de sistema híbrido con bomba de calor, suelo radiante, fotovoltaica y almacenamiento para vivienda},    % title
    pdfauthor={Luis D. Aranda Sánchez},     % author
    pdfkeywords={palabra1, palabra2, código1, etc.} % list of keywords
}

% Font change to Arial
\usepackage{helvet}
\renewcommand{\familydefault}{\sfdefault}

% Chapter titles in uppercase and larger font
\titleformat{\chapter}[hang]{\large\bfseries}{\thechapter.}{1em}{\MakeUppercase}
\titleformat{\section}[hang]{\bfseries}{\thesection.}{1em}{}
\titleformat{\subsection}[hang]{\bfseries}{\thesubsection.}{1em}{}

% Fancyhdr setup
\setlength{\headheight}{14.30174pt} % Adjust to recommended value, slightly larger for safety
\fancyhf{} % Clear all headers and footers
\fancyhead[LE]{\nouppercase{\leftmark}}
\fancyhead[RO]{Optimización energética para vivienda}
\fancyfoot[LE]{\thepage}
\fancyfoot[RE]{Escuela Técnica Superior de Ingenieros Industriales (UPM)}
\fancyfoot[LO]{Luis D. Aranda Sánchez}
\fancyfoot[RO]{\thepage}
\renewcommand{\headrulewidth}{0.4pt}
\renewcommand{\footrulewidth}{0.4pt}

\fancypagestyle{myfancy}{
    \fancyhf{} % Clear all headers and footers
    \fancyhead[LE]{\nouppercase{\leftmark}}
    \fancyhead[RO]{Optimización energética para vivienda}
    \fancyfoot[LE]{\thepage}
    \fancyfoot[RE]{Escuela Técnica Superior de Ingenieros Industriales (UPM)}
    \fancyfoot[LO]{Luis D. Aranda Sánchez}
    \fancyfoot[RO]{\thepage}
    \renewcommand{\headrulewidth}{0.4pt}
    \renewcommand{\footrulewidth}{0.4pt}
}

\fancypagestyle{simple}{
    \fancyhf{} % Clear all headers and footers
    \renewcommand{\headrulewidth}{0pt}
    \renewcommand{\footrulewidth}{0pt}
}

% Line spacing
\setstretch{1.2}

% Document starts here
\begin{document}

% Portada
\begin{titlepage}
    \centering
    {\scshape\LARGE Universidad Politécnica de Madrid \par}
    \vspace{1cm}
    {\scshape\Large Escuela Técnica Superior de Ingenieros Industriales\par}
    \vspace{1.5cm}
    {\huge\bfseries Optimización energética de sistema híbrido con bomba de calor, suelo radiante, fotovoltaica y almacenamiento para vivienda \par}
    \vspace{1.5cm}
    {\Large\bfseries Trabajo de Fin de Máster\par}
    \vspace{0.5cm}
    {\large Máster Universitario en Ingeniería de la Energía \par}
    \vspace{2cm}
    {\Large Luis D. Aranda Sánchez\par}
    \vfill
    Director: Javier Rodríguez Martín
    \vfill
    {\large Septiembre 6, 2024\par}
\end{titlepage}

% Resumen (máximo de 5 páginas, incluyendo al final Palabras clave)
\clearpage
\pagestyle{simple}
% \newpage
\chapter*{Resumen}
\addcontentsline{toc}{chapter}{Resumen}
\input{capitulos/resumen/main.tex}

% Índice (paginado)
\clearpage
\pagestyle{simple}
% \newpage
\tableofcontents

% Introducción (donde se incluya los antecedentes y justificación)
\clearpage
\pagestyle{myfancy}
\newpage
\chapter{Introducción}
\input{capitulos/introduccion/main.tex}

% Objetivos
\chapter{Objetivos}
\input{capitulos/objetivos/main.tex}

% Metodología
\chapter{Metodología}
\input{capitulos/metodologia/main.tex}

% Resultados y discusión (incluyendo la valoración de impactos y de aspectos de responsabilidad legal, ética y profesional relacionados con el trabajo)
\chapter{Resultados y Discusión}
\input{capitulos/resultados_discusion/main.tex}

% Conclusiones
\chapter{Conclusiones}
\input{capitulos/conclusiones/main.tex}

% Planificación temporal y presupuesto
\chapter{Planificación Temporal y Presupuesto}
\input{capitulos/planificacion_presupuesto/main.tex}

% Bibliografía
\newpage
\addcontentsline{toc}{chapter}{Bibliografía}
\printbibliography

\end{document}


% Bibliografía
\newpage
\addcontentsline{toc}{chapter}{Bibliografía}
\printbibliography

\end{document}


% Objetivos
\chapter{Objetivos}
\documentclass[a4paper,11pt,twoside]{report}
\usepackage[left=25mm,right=25mm,top=25mm,bottom=25mm,includehead,includefoot,headsep=15mm,footskip=15mm]{geometry}
\usepackage{graphicx}
\usepackage{fancyhdr}
\usepackage{titlesec}
\usepackage[spanish]{babel}
\usepackage[utf8]{inputenc}
\usepackage{amsmath}
\usepackage{setspace}
\usepackage{svg}
\usepackage{hyperref}
\usepackage[backend=biber,style=numeric]{biblatex}
\addbibresource{references.bib}
\hypersetup{
    colorlinks=true,
    linkcolor=blue,      % color of internal links (sections, etc.)
    urlcolor=blue,       % color of external links
    pdftitle={Optimización energética de sistema híbrido con bomba de calor, suelo radiante, fotovoltaica y almacenamiento para vivienda},    % title
    pdfauthor={Luis D. Aranda Sánchez},     % author
    pdfkeywords={palabra1, palabra2, código1, etc.} % list of keywords
}

% Font change to Arial
\usepackage{helvet}
\renewcommand{\familydefault}{\sfdefault}

% Chapter titles in uppercase and larger font
\titleformat{\chapter}[hang]{\large\bfseries}{\thechapter.}{1em}{\MakeUppercase}
\titleformat{\section}[hang]{\bfseries}{\thesection.}{1em}{}
\titleformat{\subsection}[hang]{\bfseries}{\thesubsection.}{1em}{}

% Fancyhdr setup
\setlength{\headheight}{14.30174pt} % Adjust to recommended value, slightly larger for safety
\fancyhf{} % Clear all headers and footers
\fancyhead[LE]{\nouppercase{\leftmark}}
\fancyhead[RO]{Optimización energética para vivienda}
\fancyfoot[LE]{\thepage}
\fancyfoot[RE]{Escuela Técnica Superior de Ingenieros Industriales (UPM)}
\fancyfoot[LO]{Luis D. Aranda Sánchez}
\fancyfoot[RO]{\thepage}
\renewcommand{\headrulewidth}{0.4pt}
\renewcommand{\footrulewidth}{0.4pt}

\fancypagestyle{myfancy}{
    \fancyhf{} % Clear all headers and footers
    \fancyhead[LE]{\nouppercase{\leftmark}}
    \fancyhead[RO]{Optimización energética para vivienda}
    \fancyfoot[LE]{\thepage}
    \fancyfoot[RE]{Escuela Técnica Superior de Ingenieros Industriales (UPM)}
    \fancyfoot[LO]{Luis D. Aranda Sánchez}
    \fancyfoot[RO]{\thepage}
    \renewcommand{\headrulewidth}{0.4pt}
    \renewcommand{\footrulewidth}{0.4pt}
}

\fancypagestyle{simple}{
    \fancyhf{} % Clear all headers and footers
    \renewcommand{\headrulewidth}{0pt}
    \renewcommand{\footrulewidth}{0pt}
}

% Line spacing
\setstretch{1.2}

% Document starts here
\begin{document}

% Portada
\begin{titlepage}
    \centering
    {\scshape\LARGE Universidad Politécnica de Madrid \par}
    \vspace{1cm}
    {\scshape\Large Escuela Técnica Superior de Ingenieros Industriales\par}
    \vspace{1.5cm}
    {\huge\bfseries Optimización energética de sistema híbrido con bomba de calor, suelo radiante, fotovoltaica y almacenamiento para vivienda \par}
    \vspace{1.5cm}
    {\Large\bfseries Trabajo de Fin de Máster\par}
    \vspace{0.5cm}
    {\large Máster Universitario en Ingeniería de la Energía \par}
    \vspace{2cm}
    {\Large Luis D. Aranda Sánchez\par}
    \vfill
    Director: Javier Rodríguez Martín
    \vfill
    {\large Septiembre 6, 2024\par}
\end{titlepage}

% Resumen (máximo de 5 páginas, incluyendo al final Palabras clave)
\clearpage
\pagestyle{simple}
% \newpage
\chapter*{Resumen}
\addcontentsline{toc}{chapter}{Resumen}
\documentclass[a4paper,11pt,twoside]{report}
\usepackage[left=25mm,right=25mm,top=25mm,bottom=25mm,includehead,includefoot,headsep=15mm,footskip=15mm]{geometry}
\usepackage{graphicx}
\usepackage{fancyhdr}
\usepackage{titlesec}
\usepackage[spanish]{babel}
\usepackage[utf8]{inputenc}
\usepackage{amsmath}
\usepackage{setspace}
\usepackage{svg}
\usepackage{hyperref}
\usepackage[backend=biber,style=numeric]{biblatex}
\addbibresource{references.bib}
\hypersetup{
    colorlinks=true,
    linkcolor=blue,      % color of internal links (sections, etc.)
    urlcolor=blue,       % color of external links
    pdftitle={Optimización energética de sistema híbrido con bomba de calor, suelo radiante, fotovoltaica y almacenamiento para vivienda},    % title
    pdfauthor={Luis D. Aranda Sánchez},     % author
    pdfkeywords={palabra1, palabra2, código1, etc.} % list of keywords
}

% Font change to Arial
\usepackage{helvet}
\renewcommand{\familydefault}{\sfdefault}

% Chapter titles in uppercase and larger font
\titleformat{\chapter}[hang]{\large\bfseries}{\thechapter.}{1em}{\MakeUppercase}
\titleformat{\section}[hang]{\bfseries}{\thesection.}{1em}{}
\titleformat{\subsection}[hang]{\bfseries}{\thesubsection.}{1em}{}

% Fancyhdr setup
\setlength{\headheight}{14.30174pt} % Adjust to recommended value, slightly larger for safety
\fancyhf{} % Clear all headers and footers
\fancyhead[LE]{\nouppercase{\leftmark}}
\fancyhead[RO]{Optimización energética para vivienda}
\fancyfoot[LE]{\thepage}
\fancyfoot[RE]{Escuela Técnica Superior de Ingenieros Industriales (UPM)}
\fancyfoot[LO]{Luis D. Aranda Sánchez}
\fancyfoot[RO]{\thepage}
\renewcommand{\headrulewidth}{0.4pt}
\renewcommand{\footrulewidth}{0.4pt}

\fancypagestyle{myfancy}{
    \fancyhf{} % Clear all headers and footers
    \fancyhead[LE]{\nouppercase{\leftmark}}
    \fancyhead[RO]{Optimización energética para vivienda}
    \fancyfoot[LE]{\thepage}
    \fancyfoot[RE]{Escuela Técnica Superior de Ingenieros Industriales (UPM)}
    \fancyfoot[LO]{Luis D. Aranda Sánchez}
    \fancyfoot[RO]{\thepage}
    \renewcommand{\headrulewidth}{0.4pt}
    \renewcommand{\footrulewidth}{0.4pt}
}

\fancypagestyle{simple}{
    \fancyhf{} % Clear all headers and footers
    \renewcommand{\headrulewidth}{0pt}
    \renewcommand{\footrulewidth}{0pt}
}

% Line spacing
\setstretch{1.2}

% Document starts here
\begin{document}

% Portada
\begin{titlepage}
    \centering
    {\scshape\LARGE Universidad Politécnica de Madrid \par}
    \vspace{1cm}
    {\scshape\Large Escuela Técnica Superior de Ingenieros Industriales\par}
    \vspace{1.5cm}
    {\huge\bfseries Optimización energética de sistema híbrido con bomba de calor, suelo radiante, fotovoltaica y almacenamiento para vivienda \par}
    \vspace{1.5cm}
    {\Large\bfseries Trabajo de Fin de Máster\par}
    \vspace{0.5cm}
    {\large Máster Universitario en Ingeniería de la Energía \par}
    \vspace{2cm}
    {\Large Luis D. Aranda Sánchez\par}
    \vfill
    Director: Javier Rodríguez Martín
    \vfill
    {\large Septiembre 6, 2024\par}
\end{titlepage}

% Resumen (máximo de 5 páginas, incluyendo al final Palabras clave)
\clearpage
\pagestyle{simple}
% \newpage
\chapter*{Resumen}
\addcontentsline{toc}{chapter}{Resumen}
\input{capitulos/resumen/main.tex}

% Índice (paginado)
\clearpage
\pagestyle{simple}
% \newpage
\tableofcontents

% Introducción (donde se incluya los antecedentes y justificación)
\clearpage
\pagestyle{myfancy}
\newpage
\chapter{Introducción}
\input{capitulos/introduccion/main.tex}

% Objetivos
\chapter{Objetivos}
\input{capitulos/objetivos/main.tex}

% Metodología
\chapter{Metodología}
\input{capitulos/metodologia/main.tex}

% Resultados y discusión (incluyendo la valoración de impactos y de aspectos de responsabilidad legal, ética y profesional relacionados con el trabajo)
\chapter{Resultados y Discusión}
\input{capitulos/resultados_discusion/main.tex}

% Conclusiones
\chapter{Conclusiones}
\input{capitulos/conclusiones/main.tex}

% Planificación temporal y presupuesto
\chapter{Planificación Temporal y Presupuesto}
\input{capitulos/planificacion_presupuesto/main.tex}

% Bibliografía
\newpage
\addcontentsline{toc}{chapter}{Bibliografía}
\printbibliography

\end{document}


% Índice (paginado)
\clearpage
\pagestyle{simple}
% \newpage
\tableofcontents

% Introducción (donde se incluya los antecedentes y justificación)
\clearpage
\pagestyle{myfancy}
\newpage
\chapter{Introducción}
\documentclass[a4paper,11pt,twoside]{report}
\usepackage[left=25mm,right=25mm,top=25mm,bottom=25mm,includehead,includefoot,headsep=15mm,footskip=15mm]{geometry}
\usepackage{graphicx}
\usepackage{fancyhdr}
\usepackage{titlesec}
\usepackage[spanish]{babel}
\usepackage[utf8]{inputenc}
\usepackage{amsmath}
\usepackage{setspace}
\usepackage{svg}
\usepackage{hyperref}
\usepackage[backend=biber,style=numeric]{biblatex}
\addbibresource{references.bib}
\hypersetup{
    colorlinks=true,
    linkcolor=blue,      % color of internal links (sections, etc.)
    urlcolor=blue,       % color of external links
    pdftitle={Optimización energética de sistema híbrido con bomba de calor, suelo radiante, fotovoltaica y almacenamiento para vivienda},    % title
    pdfauthor={Luis D. Aranda Sánchez},     % author
    pdfkeywords={palabra1, palabra2, código1, etc.} % list of keywords
}

% Font change to Arial
\usepackage{helvet}
\renewcommand{\familydefault}{\sfdefault}

% Chapter titles in uppercase and larger font
\titleformat{\chapter}[hang]{\large\bfseries}{\thechapter.}{1em}{\MakeUppercase}
\titleformat{\section}[hang]{\bfseries}{\thesection.}{1em}{}
\titleformat{\subsection}[hang]{\bfseries}{\thesubsection.}{1em}{}

% Fancyhdr setup
\setlength{\headheight}{14.30174pt} % Adjust to recommended value, slightly larger for safety
\fancyhf{} % Clear all headers and footers
\fancyhead[LE]{\nouppercase{\leftmark}}
\fancyhead[RO]{Optimización energética para vivienda}
\fancyfoot[LE]{\thepage}
\fancyfoot[RE]{Escuela Técnica Superior de Ingenieros Industriales (UPM)}
\fancyfoot[LO]{Luis D. Aranda Sánchez}
\fancyfoot[RO]{\thepage}
\renewcommand{\headrulewidth}{0.4pt}
\renewcommand{\footrulewidth}{0.4pt}

\fancypagestyle{myfancy}{
    \fancyhf{} % Clear all headers and footers
    \fancyhead[LE]{\nouppercase{\leftmark}}
    \fancyhead[RO]{Optimización energética para vivienda}
    \fancyfoot[LE]{\thepage}
    \fancyfoot[RE]{Escuela Técnica Superior de Ingenieros Industriales (UPM)}
    \fancyfoot[LO]{Luis D. Aranda Sánchez}
    \fancyfoot[RO]{\thepage}
    \renewcommand{\headrulewidth}{0.4pt}
    \renewcommand{\footrulewidth}{0.4pt}
}

\fancypagestyle{simple}{
    \fancyhf{} % Clear all headers and footers
    \renewcommand{\headrulewidth}{0pt}
    \renewcommand{\footrulewidth}{0pt}
}

% Line spacing
\setstretch{1.2}

% Document starts here
\begin{document}

% Portada
\begin{titlepage}
    \centering
    {\scshape\LARGE Universidad Politécnica de Madrid \par}
    \vspace{1cm}
    {\scshape\Large Escuela Técnica Superior de Ingenieros Industriales\par}
    \vspace{1.5cm}
    {\huge\bfseries Optimización energética de sistema híbrido con bomba de calor, suelo radiante, fotovoltaica y almacenamiento para vivienda \par}
    \vspace{1.5cm}
    {\Large\bfseries Trabajo de Fin de Máster\par}
    \vspace{0.5cm}
    {\large Máster Universitario en Ingeniería de la Energía \par}
    \vspace{2cm}
    {\Large Luis D. Aranda Sánchez\par}
    \vfill
    Director: Javier Rodríguez Martín
    \vfill
    {\large Septiembre 6, 2024\par}
\end{titlepage}

% Resumen (máximo de 5 páginas, incluyendo al final Palabras clave)
\clearpage
\pagestyle{simple}
% \newpage
\chapter*{Resumen}
\addcontentsline{toc}{chapter}{Resumen}
\input{capitulos/resumen/main.tex}

% Índice (paginado)
\clearpage
\pagestyle{simple}
% \newpage
\tableofcontents

% Introducción (donde se incluya los antecedentes y justificación)
\clearpage
\pagestyle{myfancy}
\newpage
\chapter{Introducción}
\input{capitulos/introduccion/main.tex}

% Objetivos
\chapter{Objetivos}
\input{capitulos/objetivos/main.tex}

% Metodología
\chapter{Metodología}
\input{capitulos/metodologia/main.tex}

% Resultados y discusión (incluyendo la valoración de impactos y de aspectos de responsabilidad legal, ética y profesional relacionados con el trabajo)
\chapter{Resultados y Discusión}
\input{capitulos/resultados_discusion/main.tex}

% Conclusiones
\chapter{Conclusiones}
\input{capitulos/conclusiones/main.tex}

% Planificación temporal y presupuesto
\chapter{Planificación Temporal y Presupuesto}
\input{capitulos/planificacion_presupuesto/main.tex}

% Bibliografía
\newpage
\addcontentsline{toc}{chapter}{Bibliografía}
\printbibliography

\end{document}


% Objetivos
\chapter{Objetivos}
\documentclass[a4paper,11pt,twoside]{report}
\usepackage[left=25mm,right=25mm,top=25mm,bottom=25mm,includehead,includefoot,headsep=15mm,footskip=15mm]{geometry}
\usepackage{graphicx}
\usepackage{fancyhdr}
\usepackage{titlesec}
\usepackage[spanish]{babel}
\usepackage[utf8]{inputenc}
\usepackage{amsmath}
\usepackage{setspace}
\usepackage{svg}
\usepackage{hyperref}
\usepackage[backend=biber,style=numeric]{biblatex}
\addbibresource{references.bib}
\hypersetup{
    colorlinks=true,
    linkcolor=blue,      % color of internal links (sections, etc.)
    urlcolor=blue,       % color of external links
    pdftitle={Optimización energética de sistema híbrido con bomba de calor, suelo radiante, fotovoltaica y almacenamiento para vivienda},    % title
    pdfauthor={Luis D. Aranda Sánchez},     % author
    pdfkeywords={palabra1, palabra2, código1, etc.} % list of keywords
}

% Font change to Arial
\usepackage{helvet}
\renewcommand{\familydefault}{\sfdefault}

% Chapter titles in uppercase and larger font
\titleformat{\chapter}[hang]{\large\bfseries}{\thechapter.}{1em}{\MakeUppercase}
\titleformat{\section}[hang]{\bfseries}{\thesection.}{1em}{}
\titleformat{\subsection}[hang]{\bfseries}{\thesubsection.}{1em}{}

% Fancyhdr setup
\setlength{\headheight}{14.30174pt} % Adjust to recommended value, slightly larger for safety
\fancyhf{} % Clear all headers and footers
\fancyhead[LE]{\nouppercase{\leftmark}}
\fancyhead[RO]{Optimización energética para vivienda}
\fancyfoot[LE]{\thepage}
\fancyfoot[RE]{Escuela Técnica Superior de Ingenieros Industriales (UPM)}
\fancyfoot[LO]{Luis D. Aranda Sánchez}
\fancyfoot[RO]{\thepage}
\renewcommand{\headrulewidth}{0.4pt}
\renewcommand{\footrulewidth}{0.4pt}

\fancypagestyle{myfancy}{
    \fancyhf{} % Clear all headers and footers
    \fancyhead[LE]{\nouppercase{\leftmark}}
    \fancyhead[RO]{Optimización energética para vivienda}
    \fancyfoot[LE]{\thepage}
    \fancyfoot[RE]{Escuela Técnica Superior de Ingenieros Industriales (UPM)}
    \fancyfoot[LO]{Luis D. Aranda Sánchez}
    \fancyfoot[RO]{\thepage}
    \renewcommand{\headrulewidth}{0.4pt}
    \renewcommand{\footrulewidth}{0.4pt}
}

\fancypagestyle{simple}{
    \fancyhf{} % Clear all headers and footers
    \renewcommand{\headrulewidth}{0pt}
    \renewcommand{\footrulewidth}{0pt}
}

% Line spacing
\setstretch{1.2}

% Document starts here
\begin{document}

% Portada
\begin{titlepage}
    \centering
    {\scshape\LARGE Universidad Politécnica de Madrid \par}
    \vspace{1cm}
    {\scshape\Large Escuela Técnica Superior de Ingenieros Industriales\par}
    \vspace{1.5cm}
    {\huge\bfseries Optimización energética de sistema híbrido con bomba de calor, suelo radiante, fotovoltaica y almacenamiento para vivienda \par}
    \vspace{1.5cm}
    {\Large\bfseries Trabajo de Fin de Máster\par}
    \vspace{0.5cm}
    {\large Máster Universitario en Ingeniería de la Energía \par}
    \vspace{2cm}
    {\Large Luis D. Aranda Sánchez\par}
    \vfill
    Director: Javier Rodríguez Martín
    \vfill
    {\large Septiembre 6, 2024\par}
\end{titlepage}

% Resumen (máximo de 5 páginas, incluyendo al final Palabras clave)
\clearpage
\pagestyle{simple}
% \newpage
\chapter*{Resumen}
\addcontentsline{toc}{chapter}{Resumen}
\input{capitulos/resumen/main.tex}

% Índice (paginado)
\clearpage
\pagestyle{simple}
% \newpage
\tableofcontents

% Introducción (donde se incluya los antecedentes y justificación)
\clearpage
\pagestyle{myfancy}
\newpage
\chapter{Introducción}
\input{capitulos/introduccion/main.tex}

% Objetivos
\chapter{Objetivos}
\input{capitulos/objetivos/main.tex}

% Metodología
\chapter{Metodología}
\input{capitulos/metodologia/main.tex}

% Resultados y discusión (incluyendo la valoración de impactos y de aspectos de responsabilidad legal, ética y profesional relacionados con el trabajo)
\chapter{Resultados y Discusión}
\input{capitulos/resultados_discusion/main.tex}

% Conclusiones
\chapter{Conclusiones}
\input{capitulos/conclusiones/main.tex}

% Planificación temporal y presupuesto
\chapter{Planificación Temporal y Presupuesto}
\input{capitulos/planificacion_presupuesto/main.tex}

% Bibliografía
\newpage
\addcontentsline{toc}{chapter}{Bibliografía}
\printbibliography

\end{document}


% Metodología
\chapter{Metodología}
\documentclass[a4paper,11pt,twoside]{report}
\usepackage[left=25mm,right=25mm,top=25mm,bottom=25mm,includehead,includefoot,headsep=15mm,footskip=15mm]{geometry}
\usepackage{graphicx}
\usepackage{fancyhdr}
\usepackage{titlesec}
\usepackage[spanish]{babel}
\usepackage[utf8]{inputenc}
\usepackage{amsmath}
\usepackage{setspace}
\usepackage{svg}
\usepackage{hyperref}
\usepackage[backend=biber,style=numeric]{biblatex}
\addbibresource{references.bib}
\hypersetup{
    colorlinks=true,
    linkcolor=blue,      % color of internal links (sections, etc.)
    urlcolor=blue,       % color of external links
    pdftitle={Optimización energética de sistema híbrido con bomba de calor, suelo radiante, fotovoltaica y almacenamiento para vivienda},    % title
    pdfauthor={Luis D. Aranda Sánchez},     % author
    pdfkeywords={palabra1, palabra2, código1, etc.} % list of keywords
}

% Font change to Arial
\usepackage{helvet}
\renewcommand{\familydefault}{\sfdefault}

% Chapter titles in uppercase and larger font
\titleformat{\chapter}[hang]{\large\bfseries}{\thechapter.}{1em}{\MakeUppercase}
\titleformat{\section}[hang]{\bfseries}{\thesection.}{1em}{}
\titleformat{\subsection}[hang]{\bfseries}{\thesubsection.}{1em}{}

% Fancyhdr setup
\setlength{\headheight}{14.30174pt} % Adjust to recommended value, slightly larger for safety
\fancyhf{} % Clear all headers and footers
\fancyhead[LE]{\nouppercase{\leftmark}}
\fancyhead[RO]{Optimización energética para vivienda}
\fancyfoot[LE]{\thepage}
\fancyfoot[RE]{Escuela Técnica Superior de Ingenieros Industriales (UPM)}
\fancyfoot[LO]{Luis D. Aranda Sánchez}
\fancyfoot[RO]{\thepage}
\renewcommand{\headrulewidth}{0.4pt}
\renewcommand{\footrulewidth}{0.4pt}

\fancypagestyle{myfancy}{
    \fancyhf{} % Clear all headers and footers
    \fancyhead[LE]{\nouppercase{\leftmark}}
    \fancyhead[RO]{Optimización energética para vivienda}
    \fancyfoot[LE]{\thepage}
    \fancyfoot[RE]{Escuela Técnica Superior de Ingenieros Industriales (UPM)}
    \fancyfoot[LO]{Luis D. Aranda Sánchez}
    \fancyfoot[RO]{\thepage}
    \renewcommand{\headrulewidth}{0.4pt}
    \renewcommand{\footrulewidth}{0.4pt}
}

\fancypagestyle{simple}{
    \fancyhf{} % Clear all headers and footers
    \renewcommand{\headrulewidth}{0pt}
    \renewcommand{\footrulewidth}{0pt}
}

% Line spacing
\setstretch{1.2}

% Document starts here
\begin{document}

% Portada
\begin{titlepage}
    \centering
    {\scshape\LARGE Universidad Politécnica de Madrid \par}
    \vspace{1cm}
    {\scshape\Large Escuela Técnica Superior de Ingenieros Industriales\par}
    \vspace{1.5cm}
    {\huge\bfseries Optimización energética de sistema híbrido con bomba de calor, suelo radiante, fotovoltaica y almacenamiento para vivienda \par}
    \vspace{1.5cm}
    {\Large\bfseries Trabajo de Fin de Máster\par}
    \vspace{0.5cm}
    {\large Máster Universitario en Ingeniería de la Energía \par}
    \vspace{2cm}
    {\Large Luis D. Aranda Sánchez\par}
    \vfill
    Director: Javier Rodríguez Martín
    \vfill
    {\large Septiembre 6, 2024\par}
\end{titlepage}

% Resumen (máximo de 5 páginas, incluyendo al final Palabras clave)
\clearpage
\pagestyle{simple}
% \newpage
\chapter*{Resumen}
\addcontentsline{toc}{chapter}{Resumen}
\input{capitulos/resumen/main.tex}

% Índice (paginado)
\clearpage
\pagestyle{simple}
% \newpage
\tableofcontents

% Introducción (donde se incluya los antecedentes y justificación)
\clearpage
\pagestyle{myfancy}
\newpage
\chapter{Introducción}
\input{capitulos/introduccion/main.tex}

% Objetivos
\chapter{Objetivos}
\input{capitulos/objetivos/main.tex}

% Metodología
\chapter{Metodología}
\input{capitulos/metodologia/main.tex}

% Resultados y discusión (incluyendo la valoración de impactos y de aspectos de responsabilidad legal, ética y profesional relacionados con el trabajo)
\chapter{Resultados y Discusión}
\input{capitulos/resultados_discusion/main.tex}

% Conclusiones
\chapter{Conclusiones}
\input{capitulos/conclusiones/main.tex}

% Planificación temporal y presupuesto
\chapter{Planificación Temporal y Presupuesto}
\input{capitulos/planificacion_presupuesto/main.tex}

% Bibliografía
\newpage
\addcontentsline{toc}{chapter}{Bibliografía}
\printbibliography

\end{document}


% Resultados y discusión (incluyendo la valoración de impactos y de aspectos de responsabilidad legal, ética y profesional relacionados con el trabajo)
\chapter{Resultados y Discusión}
\documentclass[a4paper,11pt,twoside]{report}
\usepackage[left=25mm,right=25mm,top=25mm,bottom=25mm,includehead,includefoot,headsep=15mm,footskip=15mm]{geometry}
\usepackage{graphicx}
\usepackage{fancyhdr}
\usepackage{titlesec}
\usepackage[spanish]{babel}
\usepackage[utf8]{inputenc}
\usepackage{amsmath}
\usepackage{setspace}
\usepackage{svg}
\usepackage{hyperref}
\usepackage[backend=biber,style=numeric]{biblatex}
\addbibresource{references.bib}
\hypersetup{
    colorlinks=true,
    linkcolor=blue,      % color of internal links (sections, etc.)
    urlcolor=blue,       % color of external links
    pdftitle={Optimización energética de sistema híbrido con bomba de calor, suelo radiante, fotovoltaica y almacenamiento para vivienda},    % title
    pdfauthor={Luis D. Aranda Sánchez},     % author
    pdfkeywords={palabra1, palabra2, código1, etc.} % list of keywords
}

% Font change to Arial
\usepackage{helvet}
\renewcommand{\familydefault}{\sfdefault}

% Chapter titles in uppercase and larger font
\titleformat{\chapter}[hang]{\large\bfseries}{\thechapter.}{1em}{\MakeUppercase}
\titleformat{\section}[hang]{\bfseries}{\thesection.}{1em}{}
\titleformat{\subsection}[hang]{\bfseries}{\thesubsection.}{1em}{}

% Fancyhdr setup
\setlength{\headheight}{14.30174pt} % Adjust to recommended value, slightly larger for safety
\fancyhf{} % Clear all headers and footers
\fancyhead[LE]{\nouppercase{\leftmark}}
\fancyhead[RO]{Optimización energética para vivienda}
\fancyfoot[LE]{\thepage}
\fancyfoot[RE]{Escuela Técnica Superior de Ingenieros Industriales (UPM)}
\fancyfoot[LO]{Luis D. Aranda Sánchez}
\fancyfoot[RO]{\thepage}
\renewcommand{\headrulewidth}{0.4pt}
\renewcommand{\footrulewidth}{0.4pt}

\fancypagestyle{myfancy}{
    \fancyhf{} % Clear all headers and footers
    \fancyhead[LE]{\nouppercase{\leftmark}}
    \fancyhead[RO]{Optimización energética para vivienda}
    \fancyfoot[LE]{\thepage}
    \fancyfoot[RE]{Escuela Técnica Superior de Ingenieros Industriales (UPM)}
    \fancyfoot[LO]{Luis D. Aranda Sánchez}
    \fancyfoot[RO]{\thepage}
    \renewcommand{\headrulewidth}{0.4pt}
    \renewcommand{\footrulewidth}{0.4pt}
}

\fancypagestyle{simple}{
    \fancyhf{} % Clear all headers and footers
    \renewcommand{\headrulewidth}{0pt}
    \renewcommand{\footrulewidth}{0pt}
}

% Line spacing
\setstretch{1.2}

% Document starts here
\begin{document}

% Portada
\begin{titlepage}
    \centering
    {\scshape\LARGE Universidad Politécnica de Madrid \par}
    \vspace{1cm}
    {\scshape\Large Escuela Técnica Superior de Ingenieros Industriales\par}
    \vspace{1.5cm}
    {\huge\bfseries Optimización energética de sistema híbrido con bomba de calor, suelo radiante, fotovoltaica y almacenamiento para vivienda \par}
    \vspace{1.5cm}
    {\Large\bfseries Trabajo de Fin de Máster\par}
    \vspace{0.5cm}
    {\large Máster Universitario en Ingeniería de la Energía \par}
    \vspace{2cm}
    {\Large Luis D. Aranda Sánchez\par}
    \vfill
    Director: Javier Rodríguez Martín
    \vfill
    {\large Septiembre 6, 2024\par}
\end{titlepage}

% Resumen (máximo de 5 páginas, incluyendo al final Palabras clave)
\clearpage
\pagestyle{simple}
% \newpage
\chapter*{Resumen}
\addcontentsline{toc}{chapter}{Resumen}
\input{capitulos/resumen/main.tex}

% Índice (paginado)
\clearpage
\pagestyle{simple}
% \newpage
\tableofcontents

% Introducción (donde se incluya los antecedentes y justificación)
\clearpage
\pagestyle{myfancy}
\newpage
\chapter{Introducción}
\input{capitulos/introduccion/main.tex}

% Objetivos
\chapter{Objetivos}
\input{capitulos/objetivos/main.tex}

% Metodología
\chapter{Metodología}
\input{capitulos/metodologia/main.tex}

% Resultados y discusión (incluyendo la valoración de impactos y de aspectos de responsabilidad legal, ética y profesional relacionados con el trabajo)
\chapter{Resultados y Discusión}
\input{capitulos/resultados_discusion/main.tex}

% Conclusiones
\chapter{Conclusiones}
\input{capitulos/conclusiones/main.tex}

% Planificación temporal y presupuesto
\chapter{Planificación Temporal y Presupuesto}
\input{capitulos/planificacion_presupuesto/main.tex}

% Bibliografía
\newpage
\addcontentsline{toc}{chapter}{Bibliografía}
\printbibliography

\end{document}


% Conclusiones
\chapter{Conclusiones}
\documentclass[a4paper,11pt,twoside]{report}
\usepackage[left=25mm,right=25mm,top=25mm,bottom=25mm,includehead,includefoot,headsep=15mm,footskip=15mm]{geometry}
\usepackage{graphicx}
\usepackage{fancyhdr}
\usepackage{titlesec}
\usepackage[spanish]{babel}
\usepackage[utf8]{inputenc}
\usepackage{amsmath}
\usepackage{setspace}
\usepackage{svg}
\usepackage{hyperref}
\usepackage[backend=biber,style=numeric]{biblatex}
\addbibresource{references.bib}
\hypersetup{
    colorlinks=true,
    linkcolor=blue,      % color of internal links (sections, etc.)
    urlcolor=blue,       % color of external links
    pdftitle={Optimización energética de sistema híbrido con bomba de calor, suelo radiante, fotovoltaica y almacenamiento para vivienda},    % title
    pdfauthor={Luis D. Aranda Sánchez},     % author
    pdfkeywords={palabra1, palabra2, código1, etc.} % list of keywords
}

% Font change to Arial
\usepackage{helvet}
\renewcommand{\familydefault}{\sfdefault}

% Chapter titles in uppercase and larger font
\titleformat{\chapter}[hang]{\large\bfseries}{\thechapter.}{1em}{\MakeUppercase}
\titleformat{\section}[hang]{\bfseries}{\thesection.}{1em}{}
\titleformat{\subsection}[hang]{\bfseries}{\thesubsection.}{1em}{}

% Fancyhdr setup
\setlength{\headheight}{14.30174pt} % Adjust to recommended value, slightly larger for safety
\fancyhf{} % Clear all headers and footers
\fancyhead[LE]{\nouppercase{\leftmark}}
\fancyhead[RO]{Optimización energética para vivienda}
\fancyfoot[LE]{\thepage}
\fancyfoot[RE]{Escuela Técnica Superior de Ingenieros Industriales (UPM)}
\fancyfoot[LO]{Luis D. Aranda Sánchez}
\fancyfoot[RO]{\thepage}
\renewcommand{\headrulewidth}{0.4pt}
\renewcommand{\footrulewidth}{0.4pt}

\fancypagestyle{myfancy}{
    \fancyhf{} % Clear all headers and footers
    \fancyhead[LE]{\nouppercase{\leftmark}}
    \fancyhead[RO]{Optimización energética para vivienda}
    \fancyfoot[LE]{\thepage}
    \fancyfoot[RE]{Escuela Técnica Superior de Ingenieros Industriales (UPM)}
    \fancyfoot[LO]{Luis D. Aranda Sánchez}
    \fancyfoot[RO]{\thepage}
    \renewcommand{\headrulewidth}{0.4pt}
    \renewcommand{\footrulewidth}{0.4pt}
}

\fancypagestyle{simple}{
    \fancyhf{} % Clear all headers and footers
    \renewcommand{\headrulewidth}{0pt}
    \renewcommand{\footrulewidth}{0pt}
}

% Line spacing
\setstretch{1.2}

% Document starts here
\begin{document}

% Portada
\begin{titlepage}
    \centering
    {\scshape\LARGE Universidad Politécnica de Madrid \par}
    \vspace{1cm}
    {\scshape\Large Escuela Técnica Superior de Ingenieros Industriales\par}
    \vspace{1.5cm}
    {\huge\bfseries Optimización energética de sistema híbrido con bomba de calor, suelo radiante, fotovoltaica y almacenamiento para vivienda \par}
    \vspace{1.5cm}
    {\Large\bfseries Trabajo de Fin de Máster\par}
    \vspace{0.5cm}
    {\large Máster Universitario en Ingeniería de la Energía \par}
    \vspace{2cm}
    {\Large Luis D. Aranda Sánchez\par}
    \vfill
    Director: Javier Rodríguez Martín
    \vfill
    {\large Septiembre 6, 2024\par}
\end{titlepage}

% Resumen (máximo de 5 páginas, incluyendo al final Palabras clave)
\clearpage
\pagestyle{simple}
% \newpage
\chapter*{Resumen}
\addcontentsline{toc}{chapter}{Resumen}
\input{capitulos/resumen/main.tex}

% Índice (paginado)
\clearpage
\pagestyle{simple}
% \newpage
\tableofcontents

% Introducción (donde se incluya los antecedentes y justificación)
\clearpage
\pagestyle{myfancy}
\newpage
\chapter{Introducción}
\input{capitulos/introduccion/main.tex}

% Objetivos
\chapter{Objetivos}
\input{capitulos/objetivos/main.tex}

% Metodología
\chapter{Metodología}
\input{capitulos/metodologia/main.tex}

% Resultados y discusión (incluyendo la valoración de impactos y de aspectos de responsabilidad legal, ética y profesional relacionados con el trabajo)
\chapter{Resultados y Discusión}
\input{capitulos/resultados_discusion/main.tex}

% Conclusiones
\chapter{Conclusiones}
\input{capitulos/conclusiones/main.tex}

% Planificación temporal y presupuesto
\chapter{Planificación Temporal y Presupuesto}
\input{capitulos/planificacion_presupuesto/main.tex}

% Bibliografía
\newpage
\addcontentsline{toc}{chapter}{Bibliografía}
\printbibliography

\end{document}


% Planificación temporal y presupuesto
\chapter{Planificación Temporal y Presupuesto}
\documentclass[a4paper,11pt,twoside]{report}
\usepackage[left=25mm,right=25mm,top=25mm,bottom=25mm,includehead,includefoot,headsep=15mm,footskip=15mm]{geometry}
\usepackage{graphicx}
\usepackage{fancyhdr}
\usepackage{titlesec}
\usepackage[spanish]{babel}
\usepackage[utf8]{inputenc}
\usepackage{amsmath}
\usepackage{setspace}
\usepackage{svg}
\usepackage{hyperref}
\usepackage[backend=biber,style=numeric]{biblatex}
\addbibresource{references.bib}
\hypersetup{
    colorlinks=true,
    linkcolor=blue,      % color of internal links (sections, etc.)
    urlcolor=blue,       % color of external links
    pdftitle={Optimización energética de sistema híbrido con bomba de calor, suelo radiante, fotovoltaica y almacenamiento para vivienda},    % title
    pdfauthor={Luis D. Aranda Sánchez},     % author
    pdfkeywords={palabra1, palabra2, código1, etc.} % list of keywords
}

% Font change to Arial
\usepackage{helvet}
\renewcommand{\familydefault}{\sfdefault}

% Chapter titles in uppercase and larger font
\titleformat{\chapter}[hang]{\large\bfseries}{\thechapter.}{1em}{\MakeUppercase}
\titleformat{\section}[hang]{\bfseries}{\thesection.}{1em}{}
\titleformat{\subsection}[hang]{\bfseries}{\thesubsection.}{1em}{}

% Fancyhdr setup
\setlength{\headheight}{14.30174pt} % Adjust to recommended value, slightly larger for safety
\fancyhf{} % Clear all headers and footers
\fancyhead[LE]{\nouppercase{\leftmark}}
\fancyhead[RO]{Optimización energética para vivienda}
\fancyfoot[LE]{\thepage}
\fancyfoot[RE]{Escuela Técnica Superior de Ingenieros Industriales (UPM)}
\fancyfoot[LO]{Luis D. Aranda Sánchez}
\fancyfoot[RO]{\thepage}
\renewcommand{\headrulewidth}{0.4pt}
\renewcommand{\footrulewidth}{0.4pt}

\fancypagestyle{myfancy}{
    \fancyhf{} % Clear all headers and footers
    \fancyhead[LE]{\nouppercase{\leftmark}}
    \fancyhead[RO]{Optimización energética para vivienda}
    \fancyfoot[LE]{\thepage}
    \fancyfoot[RE]{Escuela Técnica Superior de Ingenieros Industriales (UPM)}
    \fancyfoot[LO]{Luis D. Aranda Sánchez}
    \fancyfoot[RO]{\thepage}
    \renewcommand{\headrulewidth}{0.4pt}
    \renewcommand{\footrulewidth}{0.4pt}
}

\fancypagestyle{simple}{
    \fancyhf{} % Clear all headers and footers
    \renewcommand{\headrulewidth}{0pt}
    \renewcommand{\footrulewidth}{0pt}
}

% Line spacing
\setstretch{1.2}

% Document starts here
\begin{document}

% Portada
\begin{titlepage}
    \centering
    {\scshape\LARGE Universidad Politécnica de Madrid \par}
    \vspace{1cm}
    {\scshape\Large Escuela Técnica Superior de Ingenieros Industriales\par}
    \vspace{1.5cm}
    {\huge\bfseries Optimización energética de sistema híbrido con bomba de calor, suelo radiante, fotovoltaica y almacenamiento para vivienda \par}
    \vspace{1.5cm}
    {\Large\bfseries Trabajo de Fin de Máster\par}
    \vspace{0.5cm}
    {\large Máster Universitario en Ingeniería de la Energía \par}
    \vspace{2cm}
    {\Large Luis D. Aranda Sánchez\par}
    \vfill
    Director: Javier Rodríguez Martín
    \vfill
    {\large Septiembre 6, 2024\par}
\end{titlepage}

% Resumen (máximo de 5 páginas, incluyendo al final Palabras clave)
\clearpage
\pagestyle{simple}
% \newpage
\chapter*{Resumen}
\addcontentsline{toc}{chapter}{Resumen}
\input{capitulos/resumen/main.tex}

% Índice (paginado)
\clearpage
\pagestyle{simple}
% \newpage
\tableofcontents

% Introducción (donde se incluya los antecedentes y justificación)
\clearpage
\pagestyle{myfancy}
\newpage
\chapter{Introducción}
\input{capitulos/introduccion/main.tex}

% Objetivos
\chapter{Objetivos}
\input{capitulos/objetivos/main.tex}

% Metodología
\chapter{Metodología}
\input{capitulos/metodologia/main.tex}

% Resultados y discusión (incluyendo la valoración de impactos y de aspectos de responsabilidad legal, ética y profesional relacionados con el trabajo)
\chapter{Resultados y Discusión}
\input{capitulos/resultados_discusion/main.tex}

% Conclusiones
\chapter{Conclusiones}
\input{capitulos/conclusiones/main.tex}

% Planificación temporal y presupuesto
\chapter{Planificación Temporal y Presupuesto}
\input{capitulos/planificacion_presupuesto/main.tex}

% Bibliografía
\newpage
\addcontentsline{toc}{chapter}{Bibliografía}
\printbibliography

\end{document}


% Bibliografía
\newpage
\addcontentsline{toc}{chapter}{Bibliografía}
\printbibliography

\end{document}


% Metodología
\chapter{Metodología}
\documentclass[a4paper,11pt,twoside]{report}
\usepackage[left=25mm,right=25mm,top=25mm,bottom=25mm,includehead,includefoot,headsep=15mm,footskip=15mm]{geometry}
\usepackage{graphicx}
\usepackage{fancyhdr}
\usepackage{titlesec}
\usepackage[spanish]{babel}
\usepackage[utf8]{inputenc}
\usepackage{amsmath}
\usepackage{setspace}
\usepackage{svg}
\usepackage{hyperref}
\usepackage[backend=biber,style=numeric]{biblatex}
\addbibresource{references.bib}
\hypersetup{
    colorlinks=true,
    linkcolor=blue,      % color of internal links (sections, etc.)
    urlcolor=blue,       % color of external links
    pdftitle={Optimización energética de sistema híbrido con bomba de calor, suelo radiante, fotovoltaica y almacenamiento para vivienda},    % title
    pdfauthor={Luis D. Aranda Sánchez},     % author
    pdfkeywords={palabra1, palabra2, código1, etc.} % list of keywords
}

% Font change to Arial
\usepackage{helvet}
\renewcommand{\familydefault}{\sfdefault}

% Chapter titles in uppercase and larger font
\titleformat{\chapter}[hang]{\large\bfseries}{\thechapter.}{1em}{\MakeUppercase}
\titleformat{\section}[hang]{\bfseries}{\thesection.}{1em}{}
\titleformat{\subsection}[hang]{\bfseries}{\thesubsection.}{1em}{}

% Fancyhdr setup
\setlength{\headheight}{14.30174pt} % Adjust to recommended value, slightly larger for safety
\fancyhf{} % Clear all headers and footers
\fancyhead[LE]{\nouppercase{\leftmark}}
\fancyhead[RO]{Optimización energética para vivienda}
\fancyfoot[LE]{\thepage}
\fancyfoot[RE]{Escuela Técnica Superior de Ingenieros Industriales (UPM)}
\fancyfoot[LO]{Luis D. Aranda Sánchez}
\fancyfoot[RO]{\thepage}
\renewcommand{\headrulewidth}{0.4pt}
\renewcommand{\footrulewidth}{0.4pt}

\fancypagestyle{myfancy}{
    \fancyhf{} % Clear all headers and footers
    \fancyhead[LE]{\nouppercase{\leftmark}}
    \fancyhead[RO]{Optimización energética para vivienda}
    \fancyfoot[LE]{\thepage}
    \fancyfoot[RE]{Escuela Técnica Superior de Ingenieros Industriales (UPM)}
    \fancyfoot[LO]{Luis D. Aranda Sánchez}
    \fancyfoot[RO]{\thepage}
    \renewcommand{\headrulewidth}{0.4pt}
    \renewcommand{\footrulewidth}{0.4pt}
}

\fancypagestyle{simple}{
    \fancyhf{} % Clear all headers and footers
    \renewcommand{\headrulewidth}{0pt}
    \renewcommand{\footrulewidth}{0pt}
}

% Line spacing
\setstretch{1.2}

% Document starts here
\begin{document}

% Portada
\begin{titlepage}
    \centering
    {\scshape\LARGE Universidad Politécnica de Madrid \par}
    \vspace{1cm}
    {\scshape\Large Escuela Técnica Superior de Ingenieros Industriales\par}
    \vspace{1.5cm}
    {\huge\bfseries Optimización energética de sistema híbrido con bomba de calor, suelo radiante, fotovoltaica y almacenamiento para vivienda \par}
    \vspace{1.5cm}
    {\Large\bfseries Trabajo de Fin de Máster\par}
    \vspace{0.5cm}
    {\large Máster Universitario en Ingeniería de la Energía \par}
    \vspace{2cm}
    {\Large Luis D. Aranda Sánchez\par}
    \vfill
    Director: Javier Rodríguez Martín
    \vfill
    {\large Septiembre 6, 2024\par}
\end{titlepage}

% Resumen (máximo de 5 páginas, incluyendo al final Palabras clave)
\clearpage
\pagestyle{simple}
% \newpage
\chapter*{Resumen}
\addcontentsline{toc}{chapter}{Resumen}
\documentclass[a4paper,11pt,twoside]{report}
\usepackage[left=25mm,right=25mm,top=25mm,bottom=25mm,includehead,includefoot,headsep=15mm,footskip=15mm]{geometry}
\usepackage{graphicx}
\usepackage{fancyhdr}
\usepackage{titlesec}
\usepackage[spanish]{babel}
\usepackage[utf8]{inputenc}
\usepackage{amsmath}
\usepackage{setspace}
\usepackage{svg}
\usepackage{hyperref}
\usepackage[backend=biber,style=numeric]{biblatex}
\addbibresource{references.bib}
\hypersetup{
    colorlinks=true,
    linkcolor=blue,      % color of internal links (sections, etc.)
    urlcolor=blue,       % color of external links
    pdftitle={Optimización energética de sistema híbrido con bomba de calor, suelo radiante, fotovoltaica y almacenamiento para vivienda},    % title
    pdfauthor={Luis D. Aranda Sánchez},     % author
    pdfkeywords={palabra1, palabra2, código1, etc.} % list of keywords
}

% Font change to Arial
\usepackage{helvet}
\renewcommand{\familydefault}{\sfdefault}

% Chapter titles in uppercase and larger font
\titleformat{\chapter}[hang]{\large\bfseries}{\thechapter.}{1em}{\MakeUppercase}
\titleformat{\section}[hang]{\bfseries}{\thesection.}{1em}{}
\titleformat{\subsection}[hang]{\bfseries}{\thesubsection.}{1em}{}

% Fancyhdr setup
\setlength{\headheight}{14.30174pt} % Adjust to recommended value, slightly larger for safety
\fancyhf{} % Clear all headers and footers
\fancyhead[LE]{\nouppercase{\leftmark}}
\fancyhead[RO]{Optimización energética para vivienda}
\fancyfoot[LE]{\thepage}
\fancyfoot[RE]{Escuela Técnica Superior de Ingenieros Industriales (UPM)}
\fancyfoot[LO]{Luis D. Aranda Sánchez}
\fancyfoot[RO]{\thepage}
\renewcommand{\headrulewidth}{0.4pt}
\renewcommand{\footrulewidth}{0.4pt}

\fancypagestyle{myfancy}{
    \fancyhf{} % Clear all headers and footers
    \fancyhead[LE]{\nouppercase{\leftmark}}
    \fancyhead[RO]{Optimización energética para vivienda}
    \fancyfoot[LE]{\thepage}
    \fancyfoot[RE]{Escuela Técnica Superior de Ingenieros Industriales (UPM)}
    \fancyfoot[LO]{Luis D. Aranda Sánchez}
    \fancyfoot[RO]{\thepage}
    \renewcommand{\headrulewidth}{0.4pt}
    \renewcommand{\footrulewidth}{0.4pt}
}

\fancypagestyle{simple}{
    \fancyhf{} % Clear all headers and footers
    \renewcommand{\headrulewidth}{0pt}
    \renewcommand{\footrulewidth}{0pt}
}

% Line spacing
\setstretch{1.2}

% Document starts here
\begin{document}

% Portada
\begin{titlepage}
    \centering
    {\scshape\LARGE Universidad Politécnica de Madrid \par}
    \vspace{1cm}
    {\scshape\Large Escuela Técnica Superior de Ingenieros Industriales\par}
    \vspace{1.5cm}
    {\huge\bfseries Optimización energética de sistema híbrido con bomba de calor, suelo radiante, fotovoltaica y almacenamiento para vivienda \par}
    \vspace{1.5cm}
    {\Large\bfseries Trabajo de Fin de Máster\par}
    \vspace{0.5cm}
    {\large Máster Universitario en Ingeniería de la Energía \par}
    \vspace{2cm}
    {\Large Luis D. Aranda Sánchez\par}
    \vfill
    Director: Javier Rodríguez Martín
    \vfill
    {\large Septiembre 6, 2024\par}
\end{titlepage}

% Resumen (máximo de 5 páginas, incluyendo al final Palabras clave)
\clearpage
\pagestyle{simple}
% \newpage
\chapter*{Resumen}
\addcontentsline{toc}{chapter}{Resumen}
\input{capitulos/resumen/main.tex}

% Índice (paginado)
\clearpage
\pagestyle{simple}
% \newpage
\tableofcontents

% Introducción (donde se incluya los antecedentes y justificación)
\clearpage
\pagestyle{myfancy}
\newpage
\chapter{Introducción}
\input{capitulos/introduccion/main.tex}

% Objetivos
\chapter{Objetivos}
\input{capitulos/objetivos/main.tex}

% Metodología
\chapter{Metodología}
\input{capitulos/metodologia/main.tex}

% Resultados y discusión (incluyendo la valoración de impactos y de aspectos de responsabilidad legal, ética y profesional relacionados con el trabajo)
\chapter{Resultados y Discusión}
\input{capitulos/resultados_discusion/main.tex}

% Conclusiones
\chapter{Conclusiones}
\input{capitulos/conclusiones/main.tex}

% Planificación temporal y presupuesto
\chapter{Planificación Temporal y Presupuesto}
\input{capitulos/planificacion_presupuesto/main.tex}

% Bibliografía
\newpage
\addcontentsline{toc}{chapter}{Bibliografía}
\printbibliography

\end{document}


% Índice (paginado)
\clearpage
\pagestyle{simple}
% \newpage
\tableofcontents

% Introducción (donde se incluya los antecedentes y justificación)
\clearpage
\pagestyle{myfancy}
\newpage
\chapter{Introducción}
\documentclass[a4paper,11pt,twoside]{report}
\usepackage[left=25mm,right=25mm,top=25mm,bottom=25mm,includehead,includefoot,headsep=15mm,footskip=15mm]{geometry}
\usepackage{graphicx}
\usepackage{fancyhdr}
\usepackage{titlesec}
\usepackage[spanish]{babel}
\usepackage[utf8]{inputenc}
\usepackage{amsmath}
\usepackage{setspace}
\usepackage{svg}
\usepackage{hyperref}
\usepackage[backend=biber,style=numeric]{biblatex}
\addbibresource{references.bib}
\hypersetup{
    colorlinks=true,
    linkcolor=blue,      % color of internal links (sections, etc.)
    urlcolor=blue,       % color of external links
    pdftitle={Optimización energética de sistema híbrido con bomba de calor, suelo radiante, fotovoltaica y almacenamiento para vivienda},    % title
    pdfauthor={Luis D. Aranda Sánchez},     % author
    pdfkeywords={palabra1, palabra2, código1, etc.} % list of keywords
}

% Font change to Arial
\usepackage{helvet}
\renewcommand{\familydefault}{\sfdefault}

% Chapter titles in uppercase and larger font
\titleformat{\chapter}[hang]{\large\bfseries}{\thechapter.}{1em}{\MakeUppercase}
\titleformat{\section}[hang]{\bfseries}{\thesection.}{1em}{}
\titleformat{\subsection}[hang]{\bfseries}{\thesubsection.}{1em}{}

% Fancyhdr setup
\setlength{\headheight}{14.30174pt} % Adjust to recommended value, slightly larger for safety
\fancyhf{} % Clear all headers and footers
\fancyhead[LE]{\nouppercase{\leftmark}}
\fancyhead[RO]{Optimización energética para vivienda}
\fancyfoot[LE]{\thepage}
\fancyfoot[RE]{Escuela Técnica Superior de Ingenieros Industriales (UPM)}
\fancyfoot[LO]{Luis D. Aranda Sánchez}
\fancyfoot[RO]{\thepage}
\renewcommand{\headrulewidth}{0.4pt}
\renewcommand{\footrulewidth}{0.4pt}

\fancypagestyle{myfancy}{
    \fancyhf{} % Clear all headers and footers
    \fancyhead[LE]{\nouppercase{\leftmark}}
    \fancyhead[RO]{Optimización energética para vivienda}
    \fancyfoot[LE]{\thepage}
    \fancyfoot[RE]{Escuela Técnica Superior de Ingenieros Industriales (UPM)}
    \fancyfoot[LO]{Luis D. Aranda Sánchez}
    \fancyfoot[RO]{\thepage}
    \renewcommand{\headrulewidth}{0.4pt}
    \renewcommand{\footrulewidth}{0.4pt}
}

\fancypagestyle{simple}{
    \fancyhf{} % Clear all headers and footers
    \renewcommand{\headrulewidth}{0pt}
    \renewcommand{\footrulewidth}{0pt}
}

% Line spacing
\setstretch{1.2}

% Document starts here
\begin{document}

% Portada
\begin{titlepage}
    \centering
    {\scshape\LARGE Universidad Politécnica de Madrid \par}
    \vspace{1cm}
    {\scshape\Large Escuela Técnica Superior de Ingenieros Industriales\par}
    \vspace{1.5cm}
    {\huge\bfseries Optimización energética de sistema híbrido con bomba de calor, suelo radiante, fotovoltaica y almacenamiento para vivienda \par}
    \vspace{1.5cm}
    {\Large\bfseries Trabajo de Fin de Máster\par}
    \vspace{0.5cm}
    {\large Máster Universitario en Ingeniería de la Energía \par}
    \vspace{2cm}
    {\Large Luis D. Aranda Sánchez\par}
    \vfill
    Director: Javier Rodríguez Martín
    \vfill
    {\large Septiembre 6, 2024\par}
\end{titlepage}

% Resumen (máximo de 5 páginas, incluyendo al final Palabras clave)
\clearpage
\pagestyle{simple}
% \newpage
\chapter*{Resumen}
\addcontentsline{toc}{chapter}{Resumen}
\input{capitulos/resumen/main.tex}

% Índice (paginado)
\clearpage
\pagestyle{simple}
% \newpage
\tableofcontents

% Introducción (donde se incluya los antecedentes y justificación)
\clearpage
\pagestyle{myfancy}
\newpage
\chapter{Introducción}
\input{capitulos/introduccion/main.tex}

% Objetivos
\chapter{Objetivos}
\input{capitulos/objetivos/main.tex}

% Metodología
\chapter{Metodología}
\input{capitulos/metodologia/main.tex}

% Resultados y discusión (incluyendo la valoración de impactos y de aspectos de responsabilidad legal, ética y profesional relacionados con el trabajo)
\chapter{Resultados y Discusión}
\input{capitulos/resultados_discusion/main.tex}

% Conclusiones
\chapter{Conclusiones}
\input{capitulos/conclusiones/main.tex}

% Planificación temporal y presupuesto
\chapter{Planificación Temporal y Presupuesto}
\input{capitulos/planificacion_presupuesto/main.tex}

% Bibliografía
\newpage
\addcontentsline{toc}{chapter}{Bibliografía}
\printbibliography

\end{document}


% Objetivos
\chapter{Objetivos}
\documentclass[a4paper,11pt,twoside]{report}
\usepackage[left=25mm,right=25mm,top=25mm,bottom=25mm,includehead,includefoot,headsep=15mm,footskip=15mm]{geometry}
\usepackage{graphicx}
\usepackage{fancyhdr}
\usepackage{titlesec}
\usepackage[spanish]{babel}
\usepackage[utf8]{inputenc}
\usepackage{amsmath}
\usepackage{setspace}
\usepackage{svg}
\usepackage{hyperref}
\usepackage[backend=biber,style=numeric]{biblatex}
\addbibresource{references.bib}
\hypersetup{
    colorlinks=true,
    linkcolor=blue,      % color of internal links (sections, etc.)
    urlcolor=blue,       % color of external links
    pdftitle={Optimización energética de sistema híbrido con bomba de calor, suelo radiante, fotovoltaica y almacenamiento para vivienda},    % title
    pdfauthor={Luis D. Aranda Sánchez},     % author
    pdfkeywords={palabra1, palabra2, código1, etc.} % list of keywords
}

% Font change to Arial
\usepackage{helvet}
\renewcommand{\familydefault}{\sfdefault}

% Chapter titles in uppercase and larger font
\titleformat{\chapter}[hang]{\large\bfseries}{\thechapter.}{1em}{\MakeUppercase}
\titleformat{\section}[hang]{\bfseries}{\thesection.}{1em}{}
\titleformat{\subsection}[hang]{\bfseries}{\thesubsection.}{1em}{}

% Fancyhdr setup
\setlength{\headheight}{14.30174pt} % Adjust to recommended value, slightly larger for safety
\fancyhf{} % Clear all headers and footers
\fancyhead[LE]{\nouppercase{\leftmark}}
\fancyhead[RO]{Optimización energética para vivienda}
\fancyfoot[LE]{\thepage}
\fancyfoot[RE]{Escuela Técnica Superior de Ingenieros Industriales (UPM)}
\fancyfoot[LO]{Luis D. Aranda Sánchez}
\fancyfoot[RO]{\thepage}
\renewcommand{\headrulewidth}{0.4pt}
\renewcommand{\footrulewidth}{0.4pt}

\fancypagestyle{myfancy}{
    \fancyhf{} % Clear all headers and footers
    \fancyhead[LE]{\nouppercase{\leftmark}}
    \fancyhead[RO]{Optimización energética para vivienda}
    \fancyfoot[LE]{\thepage}
    \fancyfoot[RE]{Escuela Técnica Superior de Ingenieros Industriales (UPM)}
    \fancyfoot[LO]{Luis D. Aranda Sánchez}
    \fancyfoot[RO]{\thepage}
    \renewcommand{\headrulewidth}{0.4pt}
    \renewcommand{\footrulewidth}{0.4pt}
}

\fancypagestyle{simple}{
    \fancyhf{} % Clear all headers and footers
    \renewcommand{\headrulewidth}{0pt}
    \renewcommand{\footrulewidth}{0pt}
}

% Line spacing
\setstretch{1.2}

% Document starts here
\begin{document}

% Portada
\begin{titlepage}
    \centering
    {\scshape\LARGE Universidad Politécnica de Madrid \par}
    \vspace{1cm}
    {\scshape\Large Escuela Técnica Superior de Ingenieros Industriales\par}
    \vspace{1.5cm}
    {\huge\bfseries Optimización energética de sistema híbrido con bomba de calor, suelo radiante, fotovoltaica y almacenamiento para vivienda \par}
    \vspace{1.5cm}
    {\Large\bfseries Trabajo de Fin de Máster\par}
    \vspace{0.5cm}
    {\large Máster Universitario en Ingeniería de la Energía \par}
    \vspace{2cm}
    {\Large Luis D. Aranda Sánchez\par}
    \vfill
    Director: Javier Rodríguez Martín
    \vfill
    {\large Septiembre 6, 2024\par}
\end{titlepage}

% Resumen (máximo de 5 páginas, incluyendo al final Palabras clave)
\clearpage
\pagestyle{simple}
% \newpage
\chapter*{Resumen}
\addcontentsline{toc}{chapter}{Resumen}
\input{capitulos/resumen/main.tex}

% Índice (paginado)
\clearpage
\pagestyle{simple}
% \newpage
\tableofcontents

% Introducción (donde se incluya los antecedentes y justificación)
\clearpage
\pagestyle{myfancy}
\newpage
\chapter{Introducción}
\input{capitulos/introduccion/main.tex}

% Objetivos
\chapter{Objetivos}
\input{capitulos/objetivos/main.tex}

% Metodología
\chapter{Metodología}
\input{capitulos/metodologia/main.tex}

% Resultados y discusión (incluyendo la valoración de impactos y de aspectos de responsabilidad legal, ética y profesional relacionados con el trabajo)
\chapter{Resultados y Discusión}
\input{capitulos/resultados_discusion/main.tex}

% Conclusiones
\chapter{Conclusiones}
\input{capitulos/conclusiones/main.tex}

% Planificación temporal y presupuesto
\chapter{Planificación Temporal y Presupuesto}
\input{capitulos/planificacion_presupuesto/main.tex}

% Bibliografía
\newpage
\addcontentsline{toc}{chapter}{Bibliografía}
\printbibliography

\end{document}


% Metodología
\chapter{Metodología}
\documentclass[a4paper,11pt,twoside]{report}
\usepackage[left=25mm,right=25mm,top=25mm,bottom=25mm,includehead,includefoot,headsep=15mm,footskip=15mm]{geometry}
\usepackage{graphicx}
\usepackage{fancyhdr}
\usepackage{titlesec}
\usepackage[spanish]{babel}
\usepackage[utf8]{inputenc}
\usepackage{amsmath}
\usepackage{setspace}
\usepackage{svg}
\usepackage{hyperref}
\usepackage[backend=biber,style=numeric]{biblatex}
\addbibresource{references.bib}
\hypersetup{
    colorlinks=true,
    linkcolor=blue,      % color of internal links (sections, etc.)
    urlcolor=blue,       % color of external links
    pdftitle={Optimización energética de sistema híbrido con bomba de calor, suelo radiante, fotovoltaica y almacenamiento para vivienda},    % title
    pdfauthor={Luis D. Aranda Sánchez},     % author
    pdfkeywords={palabra1, palabra2, código1, etc.} % list of keywords
}

% Font change to Arial
\usepackage{helvet}
\renewcommand{\familydefault}{\sfdefault}

% Chapter titles in uppercase and larger font
\titleformat{\chapter}[hang]{\large\bfseries}{\thechapter.}{1em}{\MakeUppercase}
\titleformat{\section}[hang]{\bfseries}{\thesection.}{1em}{}
\titleformat{\subsection}[hang]{\bfseries}{\thesubsection.}{1em}{}

% Fancyhdr setup
\setlength{\headheight}{14.30174pt} % Adjust to recommended value, slightly larger for safety
\fancyhf{} % Clear all headers and footers
\fancyhead[LE]{\nouppercase{\leftmark}}
\fancyhead[RO]{Optimización energética para vivienda}
\fancyfoot[LE]{\thepage}
\fancyfoot[RE]{Escuela Técnica Superior de Ingenieros Industriales (UPM)}
\fancyfoot[LO]{Luis D. Aranda Sánchez}
\fancyfoot[RO]{\thepage}
\renewcommand{\headrulewidth}{0.4pt}
\renewcommand{\footrulewidth}{0.4pt}

\fancypagestyle{myfancy}{
    \fancyhf{} % Clear all headers and footers
    \fancyhead[LE]{\nouppercase{\leftmark}}
    \fancyhead[RO]{Optimización energética para vivienda}
    \fancyfoot[LE]{\thepage}
    \fancyfoot[RE]{Escuela Técnica Superior de Ingenieros Industriales (UPM)}
    \fancyfoot[LO]{Luis D. Aranda Sánchez}
    \fancyfoot[RO]{\thepage}
    \renewcommand{\headrulewidth}{0.4pt}
    \renewcommand{\footrulewidth}{0.4pt}
}

\fancypagestyle{simple}{
    \fancyhf{} % Clear all headers and footers
    \renewcommand{\headrulewidth}{0pt}
    \renewcommand{\footrulewidth}{0pt}
}

% Line spacing
\setstretch{1.2}

% Document starts here
\begin{document}

% Portada
\begin{titlepage}
    \centering
    {\scshape\LARGE Universidad Politécnica de Madrid \par}
    \vspace{1cm}
    {\scshape\Large Escuela Técnica Superior de Ingenieros Industriales\par}
    \vspace{1.5cm}
    {\huge\bfseries Optimización energética de sistema híbrido con bomba de calor, suelo radiante, fotovoltaica y almacenamiento para vivienda \par}
    \vspace{1.5cm}
    {\Large\bfseries Trabajo de Fin de Máster\par}
    \vspace{0.5cm}
    {\large Máster Universitario en Ingeniería de la Energía \par}
    \vspace{2cm}
    {\Large Luis D. Aranda Sánchez\par}
    \vfill
    Director: Javier Rodríguez Martín
    \vfill
    {\large Septiembre 6, 2024\par}
\end{titlepage}

% Resumen (máximo de 5 páginas, incluyendo al final Palabras clave)
\clearpage
\pagestyle{simple}
% \newpage
\chapter*{Resumen}
\addcontentsline{toc}{chapter}{Resumen}
\input{capitulos/resumen/main.tex}

% Índice (paginado)
\clearpage
\pagestyle{simple}
% \newpage
\tableofcontents

% Introducción (donde se incluya los antecedentes y justificación)
\clearpage
\pagestyle{myfancy}
\newpage
\chapter{Introducción}
\input{capitulos/introduccion/main.tex}

% Objetivos
\chapter{Objetivos}
\input{capitulos/objetivos/main.tex}

% Metodología
\chapter{Metodología}
\input{capitulos/metodologia/main.tex}

% Resultados y discusión (incluyendo la valoración de impactos y de aspectos de responsabilidad legal, ética y profesional relacionados con el trabajo)
\chapter{Resultados y Discusión}
\input{capitulos/resultados_discusion/main.tex}

% Conclusiones
\chapter{Conclusiones}
\input{capitulos/conclusiones/main.tex}

% Planificación temporal y presupuesto
\chapter{Planificación Temporal y Presupuesto}
\input{capitulos/planificacion_presupuesto/main.tex}

% Bibliografía
\newpage
\addcontentsline{toc}{chapter}{Bibliografía}
\printbibliography

\end{document}


% Resultados y discusión (incluyendo la valoración de impactos y de aspectos de responsabilidad legal, ética y profesional relacionados con el trabajo)
\chapter{Resultados y Discusión}
\documentclass[a4paper,11pt,twoside]{report}
\usepackage[left=25mm,right=25mm,top=25mm,bottom=25mm,includehead,includefoot,headsep=15mm,footskip=15mm]{geometry}
\usepackage{graphicx}
\usepackage{fancyhdr}
\usepackage{titlesec}
\usepackage[spanish]{babel}
\usepackage[utf8]{inputenc}
\usepackage{amsmath}
\usepackage{setspace}
\usepackage{svg}
\usepackage{hyperref}
\usepackage[backend=biber,style=numeric]{biblatex}
\addbibresource{references.bib}
\hypersetup{
    colorlinks=true,
    linkcolor=blue,      % color of internal links (sections, etc.)
    urlcolor=blue,       % color of external links
    pdftitle={Optimización energética de sistema híbrido con bomba de calor, suelo radiante, fotovoltaica y almacenamiento para vivienda},    % title
    pdfauthor={Luis D. Aranda Sánchez},     % author
    pdfkeywords={palabra1, palabra2, código1, etc.} % list of keywords
}

% Font change to Arial
\usepackage{helvet}
\renewcommand{\familydefault}{\sfdefault}

% Chapter titles in uppercase and larger font
\titleformat{\chapter}[hang]{\large\bfseries}{\thechapter.}{1em}{\MakeUppercase}
\titleformat{\section}[hang]{\bfseries}{\thesection.}{1em}{}
\titleformat{\subsection}[hang]{\bfseries}{\thesubsection.}{1em}{}

% Fancyhdr setup
\setlength{\headheight}{14.30174pt} % Adjust to recommended value, slightly larger for safety
\fancyhf{} % Clear all headers and footers
\fancyhead[LE]{\nouppercase{\leftmark}}
\fancyhead[RO]{Optimización energética para vivienda}
\fancyfoot[LE]{\thepage}
\fancyfoot[RE]{Escuela Técnica Superior de Ingenieros Industriales (UPM)}
\fancyfoot[LO]{Luis D. Aranda Sánchez}
\fancyfoot[RO]{\thepage}
\renewcommand{\headrulewidth}{0.4pt}
\renewcommand{\footrulewidth}{0.4pt}

\fancypagestyle{myfancy}{
    \fancyhf{} % Clear all headers and footers
    \fancyhead[LE]{\nouppercase{\leftmark}}
    \fancyhead[RO]{Optimización energética para vivienda}
    \fancyfoot[LE]{\thepage}
    \fancyfoot[RE]{Escuela Técnica Superior de Ingenieros Industriales (UPM)}
    \fancyfoot[LO]{Luis D. Aranda Sánchez}
    \fancyfoot[RO]{\thepage}
    \renewcommand{\headrulewidth}{0.4pt}
    \renewcommand{\footrulewidth}{0.4pt}
}

\fancypagestyle{simple}{
    \fancyhf{} % Clear all headers and footers
    \renewcommand{\headrulewidth}{0pt}
    \renewcommand{\footrulewidth}{0pt}
}

% Line spacing
\setstretch{1.2}

% Document starts here
\begin{document}

% Portada
\begin{titlepage}
    \centering
    {\scshape\LARGE Universidad Politécnica de Madrid \par}
    \vspace{1cm}
    {\scshape\Large Escuela Técnica Superior de Ingenieros Industriales\par}
    \vspace{1.5cm}
    {\huge\bfseries Optimización energética de sistema híbrido con bomba de calor, suelo radiante, fotovoltaica y almacenamiento para vivienda \par}
    \vspace{1.5cm}
    {\Large\bfseries Trabajo de Fin de Máster\par}
    \vspace{0.5cm}
    {\large Máster Universitario en Ingeniería de la Energía \par}
    \vspace{2cm}
    {\Large Luis D. Aranda Sánchez\par}
    \vfill
    Director: Javier Rodríguez Martín
    \vfill
    {\large Septiembre 6, 2024\par}
\end{titlepage}

% Resumen (máximo de 5 páginas, incluyendo al final Palabras clave)
\clearpage
\pagestyle{simple}
% \newpage
\chapter*{Resumen}
\addcontentsline{toc}{chapter}{Resumen}
\input{capitulos/resumen/main.tex}

% Índice (paginado)
\clearpage
\pagestyle{simple}
% \newpage
\tableofcontents

% Introducción (donde se incluya los antecedentes y justificación)
\clearpage
\pagestyle{myfancy}
\newpage
\chapter{Introducción}
\input{capitulos/introduccion/main.tex}

% Objetivos
\chapter{Objetivos}
\input{capitulos/objetivos/main.tex}

% Metodología
\chapter{Metodología}
\input{capitulos/metodologia/main.tex}

% Resultados y discusión (incluyendo la valoración de impactos y de aspectos de responsabilidad legal, ética y profesional relacionados con el trabajo)
\chapter{Resultados y Discusión}
\input{capitulos/resultados_discusion/main.tex}

% Conclusiones
\chapter{Conclusiones}
\input{capitulos/conclusiones/main.tex}

% Planificación temporal y presupuesto
\chapter{Planificación Temporal y Presupuesto}
\input{capitulos/planificacion_presupuesto/main.tex}

% Bibliografía
\newpage
\addcontentsline{toc}{chapter}{Bibliografía}
\printbibliography

\end{document}


% Conclusiones
\chapter{Conclusiones}
\documentclass[a4paper,11pt,twoside]{report}
\usepackage[left=25mm,right=25mm,top=25mm,bottom=25mm,includehead,includefoot,headsep=15mm,footskip=15mm]{geometry}
\usepackage{graphicx}
\usepackage{fancyhdr}
\usepackage{titlesec}
\usepackage[spanish]{babel}
\usepackage[utf8]{inputenc}
\usepackage{amsmath}
\usepackage{setspace}
\usepackage{svg}
\usepackage{hyperref}
\usepackage[backend=biber,style=numeric]{biblatex}
\addbibresource{references.bib}
\hypersetup{
    colorlinks=true,
    linkcolor=blue,      % color of internal links (sections, etc.)
    urlcolor=blue,       % color of external links
    pdftitle={Optimización energética de sistema híbrido con bomba de calor, suelo radiante, fotovoltaica y almacenamiento para vivienda},    % title
    pdfauthor={Luis D. Aranda Sánchez},     % author
    pdfkeywords={palabra1, palabra2, código1, etc.} % list of keywords
}

% Font change to Arial
\usepackage{helvet}
\renewcommand{\familydefault}{\sfdefault}

% Chapter titles in uppercase and larger font
\titleformat{\chapter}[hang]{\large\bfseries}{\thechapter.}{1em}{\MakeUppercase}
\titleformat{\section}[hang]{\bfseries}{\thesection.}{1em}{}
\titleformat{\subsection}[hang]{\bfseries}{\thesubsection.}{1em}{}

% Fancyhdr setup
\setlength{\headheight}{14.30174pt} % Adjust to recommended value, slightly larger for safety
\fancyhf{} % Clear all headers and footers
\fancyhead[LE]{\nouppercase{\leftmark}}
\fancyhead[RO]{Optimización energética para vivienda}
\fancyfoot[LE]{\thepage}
\fancyfoot[RE]{Escuela Técnica Superior de Ingenieros Industriales (UPM)}
\fancyfoot[LO]{Luis D. Aranda Sánchez}
\fancyfoot[RO]{\thepage}
\renewcommand{\headrulewidth}{0.4pt}
\renewcommand{\footrulewidth}{0.4pt}

\fancypagestyle{myfancy}{
    \fancyhf{} % Clear all headers and footers
    \fancyhead[LE]{\nouppercase{\leftmark}}
    \fancyhead[RO]{Optimización energética para vivienda}
    \fancyfoot[LE]{\thepage}
    \fancyfoot[RE]{Escuela Técnica Superior de Ingenieros Industriales (UPM)}
    \fancyfoot[LO]{Luis D. Aranda Sánchez}
    \fancyfoot[RO]{\thepage}
    \renewcommand{\headrulewidth}{0.4pt}
    \renewcommand{\footrulewidth}{0.4pt}
}

\fancypagestyle{simple}{
    \fancyhf{} % Clear all headers and footers
    \renewcommand{\headrulewidth}{0pt}
    \renewcommand{\footrulewidth}{0pt}
}

% Line spacing
\setstretch{1.2}

% Document starts here
\begin{document}

% Portada
\begin{titlepage}
    \centering
    {\scshape\LARGE Universidad Politécnica de Madrid \par}
    \vspace{1cm}
    {\scshape\Large Escuela Técnica Superior de Ingenieros Industriales\par}
    \vspace{1.5cm}
    {\huge\bfseries Optimización energética de sistema híbrido con bomba de calor, suelo radiante, fotovoltaica y almacenamiento para vivienda \par}
    \vspace{1.5cm}
    {\Large\bfseries Trabajo de Fin de Máster\par}
    \vspace{0.5cm}
    {\large Máster Universitario en Ingeniería de la Energía \par}
    \vspace{2cm}
    {\Large Luis D. Aranda Sánchez\par}
    \vfill
    Director: Javier Rodríguez Martín
    \vfill
    {\large Septiembre 6, 2024\par}
\end{titlepage}

% Resumen (máximo de 5 páginas, incluyendo al final Palabras clave)
\clearpage
\pagestyle{simple}
% \newpage
\chapter*{Resumen}
\addcontentsline{toc}{chapter}{Resumen}
\input{capitulos/resumen/main.tex}

% Índice (paginado)
\clearpage
\pagestyle{simple}
% \newpage
\tableofcontents

% Introducción (donde se incluya los antecedentes y justificación)
\clearpage
\pagestyle{myfancy}
\newpage
\chapter{Introducción}
\input{capitulos/introduccion/main.tex}

% Objetivos
\chapter{Objetivos}
\input{capitulos/objetivos/main.tex}

% Metodología
\chapter{Metodología}
\input{capitulos/metodologia/main.tex}

% Resultados y discusión (incluyendo la valoración de impactos y de aspectos de responsabilidad legal, ética y profesional relacionados con el trabajo)
\chapter{Resultados y Discusión}
\input{capitulos/resultados_discusion/main.tex}

% Conclusiones
\chapter{Conclusiones}
\input{capitulos/conclusiones/main.tex}

% Planificación temporal y presupuesto
\chapter{Planificación Temporal y Presupuesto}
\input{capitulos/planificacion_presupuesto/main.tex}

% Bibliografía
\newpage
\addcontentsline{toc}{chapter}{Bibliografía}
\printbibliography

\end{document}


% Planificación temporal y presupuesto
\chapter{Planificación Temporal y Presupuesto}
\documentclass[a4paper,11pt,twoside]{report}
\usepackage[left=25mm,right=25mm,top=25mm,bottom=25mm,includehead,includefoot,headsep=15mm,footskip=15mm]{geometry}
\usepackage{graphicx}
\usepackage{fancyhdr}
\usepackage{titlesec}
\usepackage[spanish]{babel}
\usepackage[utf8]{inputenc}
\usepackage{amsmath}
\usepackage{setspace}
\usepackage{svg}
\usepackage{hyperref}
\usepackage[backend=biber,style=numeric]{biblatex}
\addbibresource{references.bib}
\hypersetup{
    colorlinks=true,
    linkcolor=blue,      % color of internal links (sections, etc.)
    urlcolor=blue,       % color of external links
    pdftitle={Optimización energética de sistema híbrido con bomba de calor, suelo radiante, fotovoltaica y almacenamiento para vivienda},    % title
    pdfauthor={Luis D. Aranda Sánchez},     % author
    pdfkeywords={palabra1, palabra2, código1, etc.} % list of keywords
}

% Font change to Arial
\usepackage{helvet}
\renewcommand{\familydefault}{\sfdefault}

% Chapter titles in uppercase and larger font
\titleformat{\chapter}[hang]{\large\bfseries}{\thechapter.}{1em}{\MakeUppercase}
\titleformat{\section}[hang]{\bfseries}{\thesection.}{1em}{}
\titleformat{\subsection}[hang]{\bfseries}{\thesubsection.}{1em}{}

% Fancyhdr setup
\setlength{\headheight}{14.30174pt} % Adjust to recommended value, slightly larger for safety
\fancyhf{} % Clear all headers and footers
\fancyhead[LE]{\nouppercase{\leftmark}}
\fancyhead[RO]{Optimización energética para vivienda}
\fancyfoot[LE]{\thepage}
\fancyfoot[RE]{Escuela Técnica Superior de Ingenieros Industriales (UPM)}
\fancyfoot[LO]{Luis D. Aranda Sánchez}
\fancyfoot[RO]{\thepage}
\renewcommand{\headrulewidth}{0.4pt}
\renewcommand{\footrulewidth}{0.4pt}

\fancypagestyle{myfancy}{
    \fancyhf{} % Clear all headers and footers
    \fancyhead[LE]{\nouppercase{\leftmark}}
    \fancyhead[RO]{Optimización energética para vivienda}
    \fancyfoot[LE]{\thepage}
    \fancyfoot[RE]{Escuela Técnica Superior de Ingenieros Industriales (UPM)}
    \fancyfoot[LO]{Luis D. Aranda Sánchez}
    \fancyfoot[RO]{\thepage}
    \renewcommand{\headrulewidth}{0.4pt}
    \renewcommand{\footrulewidth}{0.4pt}
}

\fancypagestyle{simple}{
    \fancyhf{} % Clear all headers and footers
    \renewcommand{\headrulewidth}{0pt}
    \renewcommand{\footrulewidth}{0pt}
}

% Line spacing
\setstretch{1.2}

% Document starts here
\begin{document}

% Portada
\begin{titlepage}
    \centering
    {\scshape\LARGE Universidad Politécnica de Madrid \par}
    \vspace{1cm}
    {\scshape\Large Escuela Técnica Superior de Ingenieros Industriales\par}
    \vspace{1.5cm}
    {\huge\bfseries Optimización energética de sistema híbrido con bomba de calor, suelo radiante, fotovoltaica y almacenamiento para vivienda \par}
    \vspace{1.5cm}
    {\Large\bfseries Trabajo de Fin de Máster\par}
    \vspace{0.5cm}
    {\large Máster Universitario en Ingeniería de la Energía \par}
    \vspace{2cm}
    {\Large Luis D. Aranda Sánchez\par}
    \vfill
    Director: Javier Rodríguez Martín
    \vfill
    {\large Septiembre 6, 2024\par}
\end{titlepage}

% Resumen (máximo de 5 páginas, incluyendo al final Palabras clave)
\clearpage
\pagestyle{simple}
% \newpage
\chapter*{Resumen}
\addcontentsline{toc}{chapter}{Resumen}
\input{capitulos/resumen/main.tex}

% Índice (paginado)
\clearpage
\pagestyle{simple}
% \newpage
\tableofcontents

% Introducción (donde se incluya los antecedentes y justificación)
\clearpage
\pagestyle{myfancy}
\newpage
\chapter{Introducción}
\input{capitulos/introduccion/main.tex}

% Objetivos
\chapter{Objetivos}
\input{capitulos/objetivos/main.tex}

% Metodología
\chapter{Metodología}
\input{capitulos/metodologia/main.tex}

% Resultados y discusión (incluyendo la valoración de impactos y de aspectos de responsabilidad legal, ética y profesional relacionados con el trabajo)
\chapter{Resultados y Discusión}
\input{capitulos/resultados_discusion/main.tex}

% Conclusiones
\chapter{Conclusiones}
\input{capitulos/conclusiones/main.tex}

% Planificación temporal y presupuesto
\chapter{Planificación Temporal y Presupuesto}
\input{capitulos/planificacion_presupuesto/main.tex}

% Bibliografía
\newpage
\addcontentsline{toc}{chapter}{Bibliografía}
\printbibliography

\end{document}


% Bibliografía
\newpage
\addcontentsline{toc}{chapter}{Bibliografía}
\printbibliography

\end{document}


% Resultados y discusión (incluyendo la valoración de impactos y de aspectos de responsabilidad legal, ética y profesional relacionados con el trabajo)
\chapter{Resultados y Discusión}
\documentclass[a4paper,11pt,twoside]{report}
\usepackage[left=25mm,right=25mm,top=25mm,bottom=25mm,includehead,includefoot,headsep=15mm,footskip=15mm]{geometry}
\usepackage{graphicx}
\usepackage{fancyhdr}
\usepackage{titlesec}
\usepackage[spanish]{babel}
\usepackage[utf8]{inputenc}
\usepackage{amsmath}
\usepackage{setspace}
\usepackage{svg}
\usepackage{hyperref}
\usepackage[backend=biber,style=numeric]{biblatex}
\addbibresource{references.bib}
\hypersetup{
    colorlinks=true,
    linkcolor=blue,      % color of internal links (sections, etc.)
    urlcolor=blue,       % color of external links
    pdftitle={Optimización energética de sistema híbrido con bomba de calor, suelo radiante, fotovoltaica y almacenamiento para vivienda},    % title
    pdfauthor={Luis D. Aranda Sánchez},     % author
    pdfkeywords={palabra1, palabra2, código1, etc.} % list of keywords
}

% Font change to Arial
\usepackage{helvet}
\renewcommand{\familydefault}{\sfdefault}

% Chapter titles in uppercase and larger font
\titleformat{\chapter}[hang]{\large\bfseries}{\thechapter.}{1em}{\MakeUppercase}
\titleformat{\section}[hang]{\bfseries}{\thesection.}{1em}{}
\titleformat{\subsection}[hang]{\bfseries}{\thesubsection.}{1em}{}

% Fancyhdr setup
\setlength{\headheight}{14.30174pt} % Adjust to recommended value, slightly larger for safety
\fancyhf{} % Clear all headers and footers
\fancyhead[LE]{\nouppercase{\leftmark}}
\fancyhead[RO]{Optimización energética para vivienda}
\fancyfoot[LE]{\thepage}
\fancyfoot[RE]{Escuela Técnica Superior de Ingenieros Industriales (UPM)}
\fancyfoot[LO]{Luis D. Aranda Sánchez}
\fancyfoot[RO]{\thepage}
\renewcommand{\headrulewidth}{0.4pt}
\renewcommand{\footrulewidth}{0.4pt}

\fancypagestyle{myfancy}{
    \fancyhf{} % Clear all headers and footers
    \fancyhead[LE]{\nouppercase{\leftmark}}
    \fancyhead[RO]{Optimización energética para vivienda}
    \fancyfoot[LE]{\thepage}
    \fancyfoot[RE]{Escuela Técnica Superior de Ingenieros Industriales (UPM)}
    \fancyfoot[LO]{Luis D. Aranda Sánchez}
    \fancyfoot[RO]{\thepage}
    \renewcommand{\headrulewidth}{0.4pt}
    \renewcommand{\footrulewidth}{0.4pt}
}

\fancypagestyle{simple}{
    \fancyhf{} % Clear all headers and footers
    \renewcommand{\headrulewidth}{0pt}
    \renewcommand{\footrulewidth}{0pt}
}

% Line spacing
\setstretch{1.2}

% Document starts here
\begin{document}

% Portada
\begin{titlepage}
    \centering
    {\scshape\LARGE Universidad Politécnica de Madrid \par}
    \vspace{1cm}
    {\scshape\Large Escuela Técnica Superior de Ingenieros Industriales\par}
    \vspace{1.5cm}
    {\huge\bfseries Optimización energética de sistema híbrido con bomba de calor, suelo radiante, fotovoltaica y almacenamiento para vivienda \par}
    \vspace{1.5cm}
    {\Large\bfseries Trabajo de Fin de Máster\par}
    \vspace{0.5cm}
    {\large Máster Universitario en Ingeniería de la Energía \par}
    \vspace{2cm}
    {\Large Luis D. Aranda Sánchez\par}
    \vfill
    Director: Javier Rodríguez Martín
    \vfill
    {\large Septiembre 6, 2024\par}
\end{titlepage}

% Resumen (máximo de 5 páginas, incluyendo al final Palabras clave)
\clearpage
\pagestyle{simple}
% \newpage
\chapter*{Resumen}
\addcontentsline{toc}{chapter}{Resumen}
\documentclass[a4paper,11pt,twoside]{report}
\usepackage[left=25mm,right=25mm,top=25mm,bottom=25mm,includehead,includefoot,headsep=15mm,footskip=15mm]{geometry}
\usepackage{graphicx}
\usepackage{fancyhdr}
\usepackage{titlesec}
\usepackage[spanish]{babel}
\usepackage[utf8]{inputenc}
\usepackage{amsmath}
\usepackage{setspace}
\usepackage{svg}
\usepackage{hyperref}
\usepackage[backend=biber,style=numeric]{biblatex}
\addbibresource{references.bib}
\hypersetup{
    colorlinks=true,
    linkcolor=blue,      % color of internal links (sections, etc.)
    urlcolor=blue,       % color of external links
    pdftitle={Optimización energética de sistema híbrido con bomba de calor, suelo radiante, fotovoltaica y almacenamiento para vivienda},    % title
    pdfauthor={Luis D. Aranda Sánchez},     % author
    pdfkeywords={palabra1, palabra2, código1, etc.} % list of keywords
}

% Font change to Arial
\usepackage{helvet}
\renewcommand{\familydefault}{\sfdefault}

% Chapter titles in uppercase and larger font
\titleformat{\chapter}[hang]{\large\bfseries}{\thechapter.}{1em}{\MakeUppercase}
\titleformat{\section}[hang]{\bfseries}{\thesection.}{1em}{}
\titleformat{\subsection}[hang]{\bfseries}{\thesubsection.}{1em}{}

% Fancyhdr setup
\setlength{\headheight}{14.30174pt} % Adjust to recommended value, slightly larger for safety
\fancyhf{} % Clear all headers and footers
\fancyhead[LE]{\nouppercase{\leftmark}}
\fancyhead[RO]{Optimización energética para vivienda}
\fancyfoot[LE]{\thepage}
\fancyfoot[RE]{Escuela Técnica Superior de Ingenieros Industriales (UPM)}
\fancyfoot[LO]{Luis D. Aranda Sánchez}
\fancyfoot[RO]{\thepage}
\renewcommand{\headrulewidth}{0.4pt}
\renewcommand{\footrulewidth}{0.4pt}

\fancypagestyle{myfancy}{
    \fancyhf{} % Clear all headers and footers
    \fancyhead[LE]{\nouppercase{\leftmark}}
    \fancyhead[RO]{Optimización energética para vivienda}
    \fancyfoot[LE]{\thepage}
    \fancyfoot[RE]{Escuela Técnica Superior de Ingenieros Industriales (UPM)}
    \fancyfoot[LO]{Luis D. Aranda Sánchez}
    \fancyfoot[RO]{\thepage}
    \renewcommand{\headrulewidth}{0.4pt}
    \renewcommand{\footrulewidth}{0.4pt}
}

\fancypagestyle{simple}{
    \fancyhf{} % Clear all headers and footers
    \renewcommand{\headrulewidth}{0pt}
    \renewcommand{\footrulewidth}{0pt}
}

% Line spacing
\setstretch{1.2}

% Document starts here
\begin{document}

% Portada
\begin{titlepage}
    \centering
    {\scshape\LARGE Universidad Politécnica de Madrid \par}
    \vspace{1cm}
    {\scshape\Large Escuela Técnica Superior de Ingenieros Industriales\par}
    \vspace{1.5cm}
    {\huge\bfseries Optimización energética de sistema híbrido con bomba de calor, suelo radiante, fotovoltaica y almacenamiento para vivienda \par}
    \vspace{1.5cm}
    {\Large\bfseries Trabajo de Fin de Máster\par}
    \vspace{0.5cm}
    {\large Máster Universitario en Ingeniería de la Energía \par}
    \vspace{2cm}
    {\Large Luis D. Aranda Sánchez\par}
    \vfill
    Director: Javier Rodríguez Martín
    \vfill
    {\large Septiembre 6, 2024\par}
\end{titlepage}

% Resumen (máximo de 5 páginas, incluyendo al final Palabras clave)
\clearpage
\pagestyle{simple}
% \newpage
\chapter*{Resumen}
\addcontentsline{toc}{chapter}{Resumen}
\input{capitulos/resumen/main.tex}

% Índice (paginado)
\clearpage
\pagestyle{simple}
% \newpage
\tableofcontents

% Introducción (donde se incluya los antecedentes y justificación)
\clearpage
\pagestyle{myfancy}
\newpage
\chapter{Introducción}
\input{capitulos/introduccion/main.tex}

% Objetivos
\chapter{Objetivos}
\input{capitulos/objetivos/main.tex}

% Metodología
\chapter{Metodología}
\input{capitulos/metodologia/main.tex}

% Resultados y discusión (incluyendo la valoración de impactos y de aspectos de responsabilidad legal, ética y profesional relacionados con el trabajo)
\chapter{Resultados y Discusión}
\input{capitulos/resultados_discusion/main.tex}

% Conclusiones
\chapter{Conclusiones}
\input{capitulos/conclusiones/main.tex}

% Planificación temporal y presupuesto
\chapter{Planificación Temporal y Presupuesto}
\input{capitulos/planificacion_presupuesto/main.tex}

% Bibliografía
\newpage
\addcontentsline{toc}{chapter}{Bibliografía}
\printbibliography

\end{document}


% Índice (paginado)
\clearpage
\pagestyle{simple}
% \newpage
\tableofcontents

% Introducción (donde se incluya los antecedentes y justificación)
\clearpage
\pagestyle{myfancy}
\newpage
\chapter{Introducción}
\documentclass[a4paper,11pt,twoside]{report}
\usepackage[left=25mm,right=25mm,top=25mm,bottom=25mm,includehead,includefoot,headsep=15mm,footskip=15mm]{geometry}
\usepackage{graphicx}
\usepackage{fancyhdr}
\usepackage{titlesec}
\usepackage[spanish]{babel}
\usepackage[utf8]{inputenc}
\usepackage{amsmath}
\usepackage{setspace}
\usepackage{svg}
\usepackage{hyperref}
\usepackage[backend=biber,style=numeric]{biblatex}
\addbibresource{references.bib}
\hypersetup{
    colorlinks=true,
    linkcolor=blue,      % color of internal links (sections, etc.)
    urlcolor=blue,       % color of external links
    pdftitle={Optimización energética de sistema híbrido con bomba de calor, suelo radiante, fotovoltaica y almacenamiento para vivienda},    % title
    pdfauthor={Luis D. Aranda Sánchez},     % author
    pdfkeywords={palabra1, palabra2, código1, etc.} % list of keywords
}

% Font change to Arial
\usepackage{helvet}
\renewcommand{\familydefault}{\sfdefault}

% Chapter titles in uppercase and larger font
\titleformat{\chapter}[hang]{\large\bfseries}{\thechapter.}{1em}{\MakeUppercase}
\titleformat{\section}[hang]{\bfseries}{\thesection.}{1em}{}
\titleformat{\subsection}[hang]{\bfseries}{\thesubsection.}{1em}{}

% Fancyhdr setup
\setlength{\headheight}{14.30174pt} % Adjust to recommended value, slightly larger for safety
\fancyhf{} % Clear all headers and footers
\fancyhead[LE]{\nouppercase{\leftmark}}
\fancyhead[RO]{Optimización energética para vivienda}
\fancyfoot[LE]{\thepage}
\fancyfoot[RE]{Escuela Técnica Superior de Ingenieros Industriales (UPM)}
\fancyfoot[LO]{Luis D. Aranda Sánchez}
\fancyfoot[RO]{\thepage}
\renewcommand{\headrulewidth}{0.4pt}
\renewcommand{\footrulewidth}{0.4pt}

\fancypagestyle{myfancy}{
    \fancyhf{} % Clear all headers and footers
    \fancyhead[LE]{\nouppercase{\leftmark}}
    \fancyhead[RO]{Optimización energética para vivienda}
    \fancyfoot[LE]{\thepage}
    \fancyfoot[RE]{Escuela Técnica Superior de Ingenieros Industriales (UPM)}
    \fancyfoot[LO]{Luis D. Aranda Sánchez}
    \fancyfoot[RO]{\thepage}
    \renewcommand{\headrulewidth}{0.4pt}
    \renewcommand{\footrulewidth}{0.4pt}
}

\fancypagestyle{simple}{
    \fancyhf{} % Clear all headers and footers
    \renewcommand{\headrulewidth}{0pt}
    \renewcommand{\footrulewidth}{0pt}
}

% Line spacing
\setstretch{1.2}

% Document starts here
\begin{document}

% Portada
\begin{titlepage}
    \centering
    {\scshape\LARGE Universidad Politécnica de Madrid \par}
    \vspace{1cm}
    {\scshape\Large Escuela Técnica Superior de Ingenieros Industriales\par}
    \vspace{1.5cm}
    {\huge\bfseries Optimización energética de sistema híbrido con bomba de calor, suelo radiante, fotovoltaica y almacenamiento para vivienda \par}
    \vspace{1.5cm}
    {\Large\bfseries Trabajo de Fin de Máster\par}
    \vspace{0.5cm}
    {\large Máster Universitario en Ingeniería de la Energía \par}
    \vspace{2cm}
    {\Large Luis D. Aranda Sánchez\par}
    \vfill
    Director: Javier Rodríguez Martín
    \vfill
    {\large Septiembre 6, 2024\par}
\end{titlepage}

% Resumen (máximo de 5 páginas, incluyendo al final Palabras clave)
\clearpage
\pagestyle{simple}
% \newpage
\chapter*{Resumen}
\addcontentsline{toc}{chapter}{Resumen}
\input{capitulos/resumen/main.tex}

% Índice (paginado)
\clearpage
\pagestyle{simple}
% \newpage
\tableofcontents

% Introducción (donde se incluya los antecedentes y justificación)
\clearpage
\pagestyle{myfancy}
\newpage
\chapter{Introducción}
\input{capitulos/introduccion/main.tex}

% Objetivos
\chapter{Objetivos}
\input{capitulos/objetivos/main.tex}

% Metodología
\chapter{Metodología}
\input{capitulos/metodologia/main.tex}

% Resultados y discusión (incluyendo la valoración de impactos y de aspectos de responsabilidad legal, ética y profesional relacionados con el trabajo)
\chapter{Resultados y Discusión}
\input{capitulos/resultados_discusion/main.tex}

% Conclusiones
\chapter{Conclusiones}
\input{capitulos/conclusiones/main.tex}

% Planificación temporal y presupuesto
\chapter{Planificación Temporal y Presupuesto}
\input{capitulos/planificacion_presupuesto/main.tex}

% Bibliografía
\newpage
\addcontentsline{toc}{chapter}{Bibliografía}
\printbibliography

\end{document}


% Objetivos
\chapter{Objetivos}
\documentclass[a4paper,11pt,twoside]{report}
\usepackage[left=25mm,right=25mm,top=25mm,bottom=25mm,includehead,includefoot,headsep=15mm,footskip=15mm]{geometry}
\usepackage{graphicx}
\usepackage{fancyhdr}
\usepackage{titlesec}
\usepackage[spanish]{babel}
\usepackage[utf8]{inputenc}
\usepackage{amsmath}
\usepackage{setspace}
\usepackage{svg}
\usepackage{hyperref}
\usepackage[backend=biber,style=numeric]{biblatex}
\addbibresource{references.bib}
\hypersetup{
    colorlinks=true,
    linkcolor=blue,      % color of internal links (sections, etc.)
    urlcolor=blue,       % color of external links
    pdftitle={Optimización energética de sistema híbrido con bomba de calor, suelo radiante, fotovoltaica y almacenamiento para vivienda},    % title
    pdfauthor={Luis D. Aranda Sánchez},     % author
    pdfkeywords={palabra1, palabra2, código1, etc.} % list of keywords
}

% Font change to Arial
\usepackage{helvet}
\renewcommand{\familydefault}{\sfdefault}

% Chapter titles in uppercase and larger font
\titleformat{\chapter}[hang]{\large\bfseries}{\thechapter.}{1em}{\MakeUppercase}
\titleformat{\section}[hang]{\bfseries}{\thesection.}{1em}{}
\titleformat{\subsection}[hang]{\bfseries}{\thesubsection.}{1em}{}

% Fancyhdr setup
\setlength{\headheight}{14.30174pt} % Adjust to recommended value, slightly larger for safety
\fancyhf{} % Clear all headers and footers
\fancyhead[LE]{\nouppercase{\leftmark}}
\fancyhead[RO]{Optimización energética para vivienda}
\fancyfoot[LE]{\thepage}
\fancyfoot[RE]{Escuela Técnica Superior de Ingenieros Industriales (UPM)}
\fancyfoot[LO]{Luis D. Aranda Sánchez}
\fancyfoot[RO]{\thepage}
\renewcommand{\headrulewidth}{0.4pt}
\renewcommand{\footrulewidth}{0.4pt}

\fancypagestyle{myfancy}{
    \fancyhf{} % Clear all headers and footers
    \fancyhead[LE]{\nouppercase{\leftmark}}
    \fancyhead[RO]{Optimización energética para vivienda}
    \fancyfoot[LE]{\thepage}
    \fancyfoot[RE]{Escuela Técnica Superior de Ingenieros Industriales (UPM)}
    \fancyfoot[LO]{Luis D. Aranda Sánchez}
    \fancyfoot[RO]{\thepage}
    \renewcommand{\headrulewidth}{0.4pt}
    \renewcommand{\footrulewidth}{0.4pt}
}

\fancypagestyle{simple}{
    \fancyhf{} % Clear all headers and footers
    \renewcommand{\headrulewidth}{0pt}
    \renewcommand{\footrulewidth}{0pt}
}

% Line spacing
\setstretch{1.2}

% Document starts here
\begin{document}

% Portada
\begin{titlepage}
    \centering
    {\scshape\LARGE Universidad Politécnica de Madrid \par}
    \vspace{1cm}
    {\scshape\Large Escuela Técnica Superior de Ingenieros Industriales\par}
    \vspace{1.5cm}
    {\huge\bfseries Optimización energética de sistema híbrido con bomba de calor, suelo radiante, fotovoltaica y almacenamiento para vivienda \par}
    \vspace{1.5cm}
    {\Large\bfseries Trabajo de Fin de Máster\par}
    \vspace{0.5cm}
    {\large Máster Universitario en Ingeniería de la Energía \par}
    \vspace{2cm}
    {\Large Luis D. Aranda Sánchez\par}
    \vfill
    Director: Javier Rodríguez Martín
    \vfill
    {\large Septiembre 6, 2024\par}
\end{titlepage}

% Resumen (máximo de 5 páginas, incluyendo al final Palabras clave)
\clearpage
\pagestyle{simple}
% \newpage
\chapter*{Resumen}
\addcontentsline{toc}{chapter}{Resumen}
\input{capitulos/resumen/main.tex}

% Índice (paginado)
\clearpage
\pagestyle{simple}
% \newpage
\tableofcontents

% Introducción (donde se incluya los antecedentes y justificación)
\clearpage
\pagestyle{myfancy}
\newpage
\chapter{Introducción}
\input{capitulos/introduccion/main.tex}

% Objetivos
\chapter{Objetivos}
\input{capitulos/objetivos/main.tex}

% Metodología
\chapter{Metodología}
\input{capitulos/metodologia/main.tex}

% Resultados y discusión (incluyendo la valoración de impactos y de aspectos de responsabilidad legal, ética y profesional relacionados con el trabajo)
\chapter{Resultados y Discusión}
\input{capitulos/resultados_discusion/main.tex}

% Conclusiones
\chapter{Conclusiones}
\input{capitulos/conclusiones/main.tex}

% Planificación temporal y presupuesto
\chapter{Planificación Temporal y Presupuesto}
\input{capitulos/planificacion_presupuesto/main.tex}

% Bibliografía
\newpage
\addcontentsline{toc}{chapter}{Bibliografía}
\printbibliography

\end{document}


% Metodología
\chapter{Metodología}
\documentclass[a4paper,11pt,twoside]{report}
\usepackage[left=25mm,right=25mm,top=25mm,bottom=25mm,includehead,includefoot,headsep=15mm,footskip=15mm]{geometry}
\usepackage{graphicx}
\usepackage{fancyhdr}
\usepackage{titlesec}
\usepackage[spanish]{babel}
\usepackage[utf8]{inputenc}
\usepackage{amsmath}
\usepackage{setspace}
\usepackage{svg}
\usepackage{hyperref}
\usepackage[backend=biber,style=numeric]{biblatex}
\addbibresource{references.bib}
\hypersetup{
    colorlinks=true,
    linkcolor=blue,      % color of internal links (sections, etc.)
    urlcolor=blue,       % color of external links
    pdftitle={Optimización energética de sistema híbrido con bomba de calor, suelo radiante, fotovoltaica y almacenamiento para vivienda},    % title
    pdfauthor={Luis D. Aranda Sánchez},     % author
    pdfkeywords={palabra1, palabra2, código1, etc.} % list of keywords
}

% Font change to Arial
\usepackage{helvet}
\renewcommand{\familydefault}{\sfdefault}

% Chapter titles in uppercase and larger font
\titleformat{\chapter}[hang]{\large\bfseries}{\thechapter.}{1em}{\MakeUppercase}
\titleformat{\section}[hang]{\bfseries}{\thesection.}{1em}{}
\titleformat{\subsection}[hang]{\bfseries}{\thesubsection.}{1em}{}

% Fancyhdr setup
\setlength{\headheight}{14.30174pt} % Adjust to recommended value, slightly larger for safety
\fancyhf{} % Clear all headers and footers
\fancyhead[LE]{\nouppercase{\leftmark}}
\fancyhead[RO]{Optimización energética para vivienda}
\fancyfoot[LE]{\thepage}
\fancyfoot[RE]{Escuela Técnica Superior de Ingenieros Industriales (UPM)}
\fancyfoot[LO]{Luis D. Aranda Sánchez}
\fancyfoot[RO]{\thepage}
\renewcommand{\headrulewidth}{0.4pt}
\renewcommand{\footrulewidth}{0.4pt}

\fancypagestyle{myfancy}{
    \fancyhf{} % Clear all headers and footers
    \fancyhead[LE]{\nouppercase{\leftmark}}
    \fancyhead[RO]{Optimización energética para vivienda}
    \fancyfoot[LE]{\thepage}
    \fancyfoot[RE]{Escuela Técnica Superior de Ingenieros Industriales (UPM)}
    \fancyfoot[LO]{Luis D. Aranda Sánchez}
    \fancyfoot[RO]{\thepage}
    \renewcommand{\headrulewidth}{0.4pt}
    \renewcommand{\footrulewidth}{0.4pt}
}

\fancypagestyle{simple}{
    \fancyhf{} % Clear all headers and footers
    \renewcommand{\headrulewidth}{0pt}
    \renewcommand{\footrulewidth}{0pt}
}

% Line spacing
\setstretch{1.2}

% Document starts here
\begin{document}

% Portada
\begin{titlepage}
    \centering
    {\scshape\LARGE Universidad Politécnica de Madrid \par}
    \vspace{1cm}
    {\scshape\Large Escuela Técnica Superior de Ingenieros Industriales\par}
    \vspace{1.5cm}
    {\huge\bfseries Optimización energética de sistema híbrido con bomba de calor, suelo radiante, fotovoltaica y almacenamiento para vivienda \par}
    \vspace{1.5cm}
    {\Large\bfseries Trabajo de Fin de Máster\par}
    \vspace{0.5cm}
    {\large Máster Universitario en Ingeniería de la Energía \par}
    \vspace{2cm}
    {\Large Luis D. Aranda Sánchez\par}
    \vfill
    Director: Javier Rodríguez Martín
    \vfill
    {\large Septiembre 6, 2024\par}
\end{titlepage}

% Resumen (máximo de 5 páginas, incluyendo al final Palabras clave)
\clearpage
\pagestyle{simple}
% \newpage
\chapter*{Resumen}
\addcontentsline{toc}{chapter}{Resumen}
\input{capitulos/resumen/main.tex}

% Índice (paginado)
\clearpage
\pagestyle{simple}
% \newpage
\tableofcontents

% Introducción (donde se incluya los antecedentes y justificación)
\clearpage
\pagestyle{myfancy}
\newpage
\chapter{Introducción}
\input{capitulos/introduccion/main.tex}

% Objetivos
\chapter{Objetivos}
\input{capitulos/objetivos/main.tex}

% Metodología
\chapter{Metodología}
\input{capitulos/metodologia/main.tex}

% Resultados y discusión (incluyendo la valoración de impactos y de aspectos de responsabilidad legal, ética y profesional relacionados con el trabajo)
\chapter{Resultados y Discusión}
\input{capitulos/resultados_discusion/main.tex}

% Conclusiones
\chapter{Conclusiones}
\input{capitulos/conclusiones/main.tex}

% Planificación temporal y presupuesto
\chapter{Planificación Temporal y Presupuesto}
\input{capitulos/planificacion_presupuesto/main.tex}

% Bibliografía
\newpage
\addcontentsline{toc}{chapter}{Bibliografía}
\printbibliography

\end{document}


% Resultados y discusión (incluyendo la valoración de impactos y de aspectos de responsabilidad legal, ética y profesional relacionados con el trabajo)
\chapter{Resultados y Discusión}
\documentclass[a4paper,11pt,twoside]{report}
\usepackage[left=25mm,right=25mm,top=25mm,bottom=25mm,includehead,includefoot,headsep=15mm,footskip=15mm]{geometry}
\usepackage{graphicx}
\usepackage{fancyhdr}
\usepackage{titlesec}
\usepackage[spanish]{babel}
\usepackage[utf8]{inputenc}
\usepackage{amsmath}
\usepackage{setspace}
\usepackage{svg}
\usepackage{hyperref}
\usepackage[backend=biber,style=numeric]{biblatex}
\addbibresource{references.bib}
\hypersetup{
    colorlinks=true,
    linkcolor=blue,      % color of internal links (sections, etc.)
    urlcolor=blue,       % color of external links
    pdftitle={Optimización energética de sistema híbrido con bomba de calor, suelo radiante, fotovoltaica y almacenamiento para vivienda},    % title
    pdfauthor={Luis D. Aranda Sánchez},     % author
    pdfkeywords={palabra1, palabra2, código1, etc.} % list of keywords
}

% Font change to Arial
\usepackage{helvet}
\renewcommand{\familydefault}{\sfdefault}

% Chapter titles in uppercase and larger font
\titleformat{\chapter}[hang]{\large\bfseries}{\thechapter.}{1em}{\MakeUppercase}
\titleformat{\section}[hang]{\bfseries}{\thesection.}{1em}{}
\titleformat{\subsection}[hang]{\bfseries}{\thesubsection.}{1em}{}

% Fancyhdr setup
\setlength{\headheight}{14.30174pt} % Adjust to recommended value, slightly larger for safety
\fancyhf{} % Clear all headers and footers
\fancyhead[LE]{\nouppercase{\leftmark}}
\fancyhead[RO]{Optimización energética para vivienda}
\fancyfoot[LE]{\thepage}
\fancyfoot[RE]{Escuela Técnica Superior de Ingenieros Industriales (UPM)}
\fancyfoot[LO]{Luis D. Aranda Sánchez}
\fancyfoot[RO]{\thepage}
\renewcommand{\headrulewidth}{0.4pt}
\renewcommand{\footrulewidth}{0.4pt}

\fancypagestyle{myfancy}{
    \fancyhf{} % Clear all headers and footers
    \fancyhead[LE]{\nouppercase{\leftmark}}
    \fancyhead[RO]{Optimización energética para vivienda}
    \fancyfoot[LE]{\thepage}
    \fancyfoot[RE]{Escuela Técnica Superior de Ingenieros Industriales (UPM)}
    \fancyfoot[LO]{Luis D. Aranda Sánchez}
    \fancyfoot[RO]{\thepage}
    \renewcommand{\headrulewidth}{0.4pt}
    \renewcommand{\footrulewidth}{0.4pt}
}

\fancypagestyle{simple}{
    \fancyhf{} % Clear all headers and footers
    \renewcommand{\headrulewidth}{0pt}
    \renewcommand{\footrulewidth}{0pt}
}

% Line spacing
\setstretch{1.2}

% Document starts here
\begin{document}

% Portada
\begin{titlepage}
    \centering
    {\scshape\LARGE Universidad Politécnica de Madrid \par}
    \vspace{1cm}
    {\scshape\Large Escuela Técnica Superior de Ingenieros Industriales\par}
    \vspace{1.5cm}
    {\huge\bfseries Optimización energética de sistema híbrido con bomba de calor, suelo radiante, fotovoltaica y almacenamiento para vivienda \par}
    \vspace{1.5cm}
    {\Large\bfseries Trabajo de Fin de Máster\par}
    \vspace{0.5cm}
    {\large Máster Universitario en Ingeniería de la Energía \par}
    \vspace{2cm}
    {\Large Luis D. Aranda Sánchez\par}
    \vfill
    Director: Javier Rodríguez Martín
    \vfill
    {\large Septiembre 6, 2024\par}
\end{titlepage}

% Resumen (máximo de 5 páginas, incluyendo al final Palabras clave)
\clearpage
\pagestyle{simple}
% \newpage
\chapter*{Resumen}
\addcontentsline{toc}{chapter}{Resumen}
\input{capitulos/resumen/main.tex}

% Índice (paginado)
\clearpage
\pagestyle{simple}
% \newpage
\tableofcontents

% Introducción (donde se incluya los antecedentes y justificación)
\clearpage
\pagestyle{myfancy}
\newpage
\chapter{Introducción}
\input{capitulos/introduccion/main.tex}

% Objetivos
\chapter{Objetivos}
\input{capitulos/objetivos/main.tex}

% Metodología
\chapter{Metodología}
\input{capitulos/metodologia/main.tex}

% Resultados y discusión (incluyendo la valoración de impactos y de aspectos de responsabilidad legal, ética y profesional relacionados con el trabajo)
\chapter{Resultados y Discusión}
\input{capitulos/resultados_discusion/main.tex}

% Conclusiones
\chapter{Conclusiones}
\input{capitulos/conclusiones/main.tex}

% Planificación temporal y presupuesto
\chapter{Planificación Temporal y Presupuesto}
\input{capitulos/planificacion_presupuesto/main.tex}

% Bibliografía
\newpage
\addcontentsline{toc}{chapter}{Bibliografía}
\printbibliography

\end{document}


% Conclusiones
\chapter{Conclusiones}
\documentclass[a4paper,11pt,twoside]{report}
\usepackage[left=25mm,right=25mm,top=25mm,bottom=25mm,includehead,includefoot,headsep=15mm,footskip=15mm]{geometry}
\usepackage{graphicx}
\usepackage{fancyhdr}
\usepackage{titlesec}
\usepackage[spanish]{babel}
\usepackage[utf8]{inputenc}
\usepackage{amsmath}
\usepackage{setspace}
\usepackage{svg}
\usepackage{hyperref}
\usepackage[backend=biber,style=numeric]{biblatex}
\addbibresource{references.bib}
\hypersetup{
    colorlinks=true,
    linkcolor=blue,      % color of internal links (sections, etc.)
    urlcolor=blue,       % color of external links
    pdftitle={Optimización energética de sistema híbrido con bomba de calor, suelo radiante, fotovoltaica y almacenamiento para vivienda},    % title
    pdfauthor={Luis D. Aranda Sánchez},     % author
    pdfkeywords={palabra1, palabra2, código1, etc.} % list of keywords
}

% Font change to Arial
\usepackage{helvet}
\renewcommand{\familydefault}{\sfdefault}

% Chapter titles in uppercase and larger font
\titleformat{\chapter}[hang]{\large\bfseries}{\thechapter.}{1em}{\MakeUppercase}
\titleformat{\section}[hang]{\bfseries}{\thesection.}{1em}{}
\titleformat{\subsection}[hang]{\bfseries}{\thesubsection.}{1em}{}

% Fancyhdr setup
\setlength{\headheight}{14.30174pt} % Adjust to recommended value, slightly larger for safety
\fancyhf{} % Clear all headers and footers
\fancyhead[LE]{\nouppercase{\leftmark}}
\fancyhead[RO]{Optimización energética para vivienda}
\fancyfoot[LE]{\thepage}
\fancyfoot[RE]{Escuela Técnica Superior de Ingenieros Industriales (UPM)}
\fancyfoot[LO]{Luis D. Aranda Sánchez}
\fancyfoot[RO]{\thepage}
\renewcommand{\headrulewidth}{0.4pt}
\renewcommand{\footrulewidth}{0.4pt}

\fancypagestyle{myfancy}{
    \fancyhf{} % Clear all headers and footers
    \fancyhead[LE]{\nouppercase{\leftmark}}
    \fancyhead[RO]{Optimización energética para vivienda}
    \fancyfoot[LE]{\thepage}
    \fancyfoot[RE]{Escuela Técnica Superior de Ingenieros Industriales (UPM)}
    \fancyfoot[LO]{Luis D. Aranda Sánchez}
    \fancyfoot[RO]{\thepage}
    \renewcommand{\headrulewidth}{0.4pt}
    \renewcommand{\footrulewidth}{0.4pt}
}

\fancypagestyle{simple}{
    \fancyhf{} % Clear all headers and footers
    \renewcommand{\headrulewidth}{0pt}
    \renewcommand{\footrulewidth}{0pt}
}

% Line spacing
\setstretch{1.2}

% Document starts here
\begin{document}

% Portada
\begin{titlepage}
    \centering
    {\scshape\LARGE Universidad Politécnica de Madrid \par}
    \vspace{1cm}
    {\scshape\Large Escuela Técnica Superior de Ingenieros Industriales\par}
    \vspace{1.5cm}
    {\huge\bfseries Optimización energética de sistema híbrido con bomba de calor, suelo radiante, fotovoltaica y almacenamiento para vivienda \par}
    \vspace{1.5cm}
    {\Large\bfseries Trabajo de Fin de Máster\par}
    \vspace{0.5cm}
    {\large Máster Universitario en Ingeniería de la Energía \par}
    \vspace{2cm}
    {\Large Luis D. Aranda Sánchez\par}
    \vfill
    Director: Javier Rodríguez Martín
    \vfill
    {\large Septiembre 6, 2024\par}
\end{titlepage}

% Resumen (máximo de 5 páginas, incluyendo al final Palabras clave)
\clearpage
\pagestyle{simple}
% \newpage
\chapter*{Resumen}
\addcontentsline{toc}{chapter}{Resumen}
\input{capitulos/resumen/main.tex}

% Índice (paginado)
\clearpage
\pagestyle{simple}
% \newpage
\tableofcontents

% Introducción (donde se incluya los antecedentes y justificación)
\clearpage
\pagestyle{myfancy}
\newpage
\chapter{Introducción}
\input{capitulos/introduccion/main.tex}

% Objetivos
\chapter{Objetivos}
\input{capitulos/objetivos/main.tex}

% Metodología
\chapter{Metodología}
\input{capitulos/metodologia/main.tex}

% Resultados y discusión (incluyendo la valoración de impactos y de aspectos de responsabilidad legal, ética y profesional relacionados con el trabajo)
\chapter{Resultados y Discusión}
\input{capitulos/resultados_discusion/main.tex}

% Conclusiones
\chapter{Conclusiones}
\input{capitulos/conclusiones/main.tex}

% Planificación temporal y presupuesto
\chapter{Planificación Temporal y Presupuesto}
\input{capitulos/planificacion_presupuesto/main.tex}

% Bibliografía
\newpage
\addcontentsline{toc}{chapter}{Bibliografía}
\printbibliography

\end{document}


% Planificación temporal y presupuesto
\chapter{Planificación Temporal y Presupuesto}
\documentclass[a4paper,11pt,twoside]{report}
\usepackage[left=25mm,right=25mm,top=25mm,bottom=25mm,includehead,includefoot,headsep=15mm,footskip=15mm]{geometry}
\usepackage{graphicx}
\usepackage{fancyhdr}
\usepackage{titlesec}
\usepackage[spanish]{babel}
\usepackage[utf8]{inputenc}
\usepackage{amsmath}
\usepackage{setspace}
\usepackage{svg}
\usepackage{hyperref}
\usepackage[backend=biber,style=numeric]{biblatex}
\addbibresource{references.bib}
\hypersetup{
    colorlinks=true,
    linkcolor=blue,      % color of internal links (sections, etc.)
    urlcolor=blue,       % color of external links
    pdftitle={Optimización energética de sistema híbrido con bomba de calor, suelo radiante, fotovoltaica y almacenamiento para vivienda},    % title
    pdfauthor={Luis D. Aranda Sánchez},     % author
    pdfkeywords={palabra1, palabra2, código1, etc.} % list of keywords
}

% Font change to Arial
\usepackage{helvet}
\renewcommand{\familydefault}{\sfdefault}

% Chapter titles in uppercase and larger font
\titleformat{\chapter}[hang]{\large\bfseries}{\thechapter.}{1em}{\MakeUppercase}
\titleformat{\section}[hang]{\bfseries}{\thesection.}{1em}{}
\titleformat{\subsection}[hang]{\bfseries}{\thesubsection.}{1em}{}

% Fancyhdr setup
\setlength{\headheight}{14.30174pt} % Adjust to recommended value, slightly larger for safety
\fancyhf{} % Clear all headers and footers
\fancyhead[LE]{\nouppercase{\leftmark}}
\fancyhead[RO]{Optimización energética para vivienda}
\fancyfoot[LE]{\thepage}
\fancyfoot[RE]{Escuela Técnica Superior de Ingenieros Industriales (UPM)}
\fancyfoot[LO]{Luis D. Aranda Sánchez}
\fancyfoot[RO]{\thepage}
\renewcommand{\headrulewidth}{0.4pt}
\renewcommand{\footrulewidth}{0.4pt}

\fancypagestyle{myfancy}{
    \fancyhf{} % Clear all headers and footers
    \fancyhead[LE]{\nouppercase{\leftmark}}
    \fancyhead[RO]{Optimización energética para vivienda}
    \fancyfoot[LE]{\thepage}
    \fancyfoot[RE]{Escuela Técnica Superior de Ingenieros Industriales (UPM)}
    \fancyfoot[LO]{Luis D. Aranda Sánchez}
    \fancyfoot[RO]{\thepage}
    \renewcommand{\headrulewidth}{0.4pt}
    \renewcommand{\footrulewidth}{0.4pt}
}

\fancypagestyle{simple}{
    \fancyhf{} % Clear all headers and footers
    \renewcommand{\headrulewidth}{0pt}
    \renewcommand{\footrulewidth}{0pt}
}

% Line spacing
\setstretch{1.2}

% Document starts here
\begin{document}

% Portada
\begin{titlepage}
    \centering
    {\scshape\LARGE Universidad Politécnica de Madrid \par}
    \vspace{1cm}
    {\scshape\Large Escuela Técnica Superior de Ingenieros Industriales\par}
    \vspace{1.5cm}
    {\huge\bfseries Optimización energética de sistema híbrido con bomba de calor, suelo radiante, fotovoltaica y almacenamiento para vivienda \par}
    \vspace{1.5cm}
    {\Large\bfseries Trabajo de Fin de Máster\par}
    \vspace{0.5cm}
    {\large Máster Universitario en Ingeniería de la Energía \par}
    \vspace{2cm}
    {\Large Luis D. Aranda Sánchez\par}
    \vfill
    Director: Javier Rodríguez Martín
    \vfill
    {\large Septiembre 6, 2024\par}
\end{titlepage}

% Resumen (máximo de 5 páginas, incluyendo al final Palabras clave)
\clearpage
\pagestyle{simple}
% \newpage
\chapter*{Resumen}
\addcontentsline{toc}{chapter}{Resumen}
\input{capitulos/resumen/main.tex}

% Índice (paginado)
\clearpage
\pagestyle{simple}
% \newpage
\tableofcontents

% Introducción (donde se incluya los antecedentes y justificación)
\clearpage
\pagestyle{myfancy}
\newpage
\chapter{Introducción}
\input{capitulos/introduccion/main.tex}

% Objetivos
\chapter{Objetivos}
\input{capitulos/objetivos/main.tex}

% Metodología
\chapter{Metodología}
\input{capitulos/metodologia/main.tex}

% Resultados y discusión (incluyendo la valoración de impactos y de aspectos de responsabilidad legal, ética y profesional relacionados con el trabajo)
\chapter{Resultados y Discusión}
\input{capitulos/resultados_discusion/main.tex}

% Conclusiones
\chapter{Conclusiones}
\input{capitulos/conclusiones/main.tex}

% Planificación temporal y presupuesto
\chapter{Planificación Temporal y Presupuesto}
\input{capitulos/planificacion_presupuesto/main.tex}

% Bibliografía
\newpage
\addcontentsline{toc}{chapter}{Bibliografía}
\printbibliography

\end{document}


% Bibliografía
\newpage
\addcontentsline{toc}{chapter}{Bibliografía}
\printbibliography

\end{document}


% Conclusiones
\chapter{Conclusiones}
\documentclass[a4paper,11pt,twoside]{report}
\usepackage[left=25mm,right=25mm,top=25mm,bottom=25mm,includehead,includefoot,headsep=15mm,footskip=15mm]{geometry}
\usepackage{graphicx}
\usepackage{fancyhdr}
\usepackage{titlesec}
\usepackage[spanish]{babel}
\usepackage[utf8]{inputenc}
\usepackage{amsmath}
\usepackage{setspace}
\usepackage{svg}
\usepackage{hyperref}
\usepackage[backend=biber,style=numeric]{biblatex}
\addbibresource{references.bib}
\hypersetup{
    colorlinks=true,
    linkcolor=blue,      % color of internal links (sections, etc.)
    urlcolor=blue,       % color of external links
    pdftitle={Optimización energética de sistema híbrido con bomba de calor, suelo radiante, fotovoltaica y almacenamiento para vivienda},    % title
    pdfauthor={Luis D. Aranda Sánchez},     % author
    pdfkeywords={palabra1, palabra2, código1, etc.} % list of keywords
}

% Font change to Arial
\usepackage{helvet}
\renewcommand{\familydefault}{\sfdefault}

% Chapter titles in uppercase and larger font
\titleformat{\chapter}[hang]{\large\bfseries}{\thechapter.}{1em}{\MakeUppercase}
\titleformat{\section}[hang]{\bfseries}{\thesection.}{1em}{}
\titleformat{\subsection}[hang]{\bfseries}{\thesubsection.}{1em}{}

% Fancyhdr setup
\setlength{\headheight}{14.30174pt} % Adjust to recommended value, slightly larger for safety
\fancyhf{} % Clear all headers and footers
\fancyhead[LE]{\nouppercase{\leftmark}}
\fancyhead[RO]{Optimización energética para vivienda}
\fancyfoot[LE]{\thepage}
\fancyfoot[RE]{Escuela Técnica Superior de Ingenieros Industriales (UPM)}
\fancyfoot[LO]{Luis D. Aranda Sánchez}
\fancyfoot[RO]{\thepage}
\renewcommand{\headrulewidth}{0.4pt}
\renewcommand{\footrulewidth}{0.4pt}

\fancypagestyle{myfancy}{
    \fancyhf{} % Clear all headers and footers
    \fancyhead[LE]{\nouppercase{\leftmark}}
    \fancyhead[RO]{Optimización energética para vivienda}
    \fancyfoot[LE]{\thepage}
    \fancyfoot[RE]{Escuela Técnica Superior de Ingenieros Industriales (UPM)}
    \fancyfoot[LO]{Luis D. Aranda Sánchez}
    \fancyfoot[RO]{\thepage}
    \renewcommand{\headrulewidth}{0.4pt}
    \renewcommand{\footrulewidth}{0.4pt}
}

\fancypagestyle{simple}{
    \fancyhf{} % Clear all headers and footers
    \renewcommand{\headrulewidth}{0pt}
    \renewcommand{\footrulewidth}{0pt}
}

% Line spacing
\setstretch{1.2}

% Document starts here
\begin{document}

% Portada
\begin{titlepage}
    \centering
    {\scshape\LARGE Universidad Politécnica de Madrid \par}
    \vspace{1cm}
    {\scshape\Large Escuela Técnica Superior de Ingenieros Industriales\par}
    \vspace{1.5cm}
    {\huge\bfseries Optimización energética de sistema híbrido con bomba de calor, suelo radiante, fotovoltaica y almacenamiento para vivienda \par}
    \vspace{1.5cm}
    {\Large\bfseries Trabajo de Fin de Máster\par}
    \vspace{0.5cm}
    {\large Máster Universitario en Ingeniería de la Energía \par}
    \vspace{2cm}
    {\Large Luis D. Aranda Sánchez\par}
    \vfill
    Director: Javier Rodríguez Martín
    \vfill
    {\large Septiembre 6, 2024\par}
\end{titlepage}

% Resumen (máximo de 5 páginas, incluyendo al final Palabras clave)
\clearpage
\pagestyle{simple}
% \newpage
\chapter*{Resumen}
\addcontentsline{toc}{chapter}{Resumen}
\documentclass[a4paper,11pt,twoside]{report}
\usepackage[left=25mm,right=25mm,top=25mm,bottom=25mm,includehead,includefoot,headsep=15mm,footskip=15mm]{geometry}
\usepackage{graphicx}
\usepackage{fancyhdr}
\usepackage{titlesec}
\usepackage[spanish]{babel}
\usepackage[utf8]{inputenc}
\usepackage{amsmath}
\usepackage{setspace}
\usepackage{svg}
\usepackage{hyperref}
\usepackage[backend=biber,style=numeric]{biblatex}
\addbibresource{references.bib}
\hypersetup{
    colorlinks=true,
    linkcolor=blue,      % color of internal links (sections, etc.)
    urlcolor=blue,       % color of external links
    pdftitle={Optimización energética de sistema híbrido con bomba de calor, suelo radiante, fotovoltaica y almacenamiento para vivienda},    % title
    pdfauthor={Luis D. Aranda Sánchez},     % author
    pdfkeywords={palabra1, palabra2, código1, etc.} % list of keywords
}

% Font change to Arial
\usepackage{helvet}
\renewcommand{\familydefault}{\sfdefault}

% Chapter titles in uppercase and larger font
\titleformat{\chapter}[hang]{\large\bfseries}{\thechapter.}{1em}{\MakeUppercase}
\titleformat{\section}[hang]{\bfseries}{\thesection.}{1em}{}
\titleformat{\subsection}[hang]{\bfseries}{\thesubsection.}{1em}{}

% Fancyhdr setup
\setlength{\headheight}{14.30174pt} % Adjust to recommended value, slightly larger for safety
\fancyhf{} % Clear all headers and footers
\fancyhead[LE]{\nouppercase{\leftmark}}
\fancyhead[RO]{Optimización energética para vivienda}
\fancyfoot[LE]{\thepage}
\fancyfoot[RE]{Escuela Técnica Superior de Ingenieros Industriales (UPM)}
\fancyfoot[LO]{Luis D. Aranda Sánchez}
\fancyfoot[RO]{\thepage}
\renewcommand{\headrulewidth}{0.4pt}
\renewcommand{\footrulewidth}{0.4pt}

\fancypagestyle{myfancy}{
    \fancyhf{} % Clear all headers and footers
    \fancyhead[LE]{\nouppercase{\leftmark}}
    \fancyhead[RO]{Optimización energética para vivienda}
    \fancyfoot[LE]{\thepage}
    \fancyfoot[RE]{Escuela Técnica Superior de Ingenieros Industriales (UPM)}
    \fancyfoot[LO]{Luis D. Aranda Sánchez}
    \fancyfoot[RO]{\thepage}
    \renewcommand{\headrulewidth}{0.4pt}
    \renewcommand{\footrulewidth}{0.4pt}
}

\fancypagestyle{simple}{
    \fancyhf{} % Clear all headers and footers
    \renewcommand{\headrulewidth}{0pt}
    \renewcommand{\footrulewidth}{0pt}
}

% Line spacing
\setstretch{1.2}

% Document starts here
\begin{document}

% Portada
\begin{titlepage}
    \centering
    {\scshape\LARGE Universidad Politécnica de Madrid \par}
    \vspace{1cm}
    {\scshape\Large Escuela Técnica Superior de Ingenieros Industriales\par}
    \vspace{1.5cm}
    {\huge\bfseries Optimización energética de sistema híbrido con bomba de calor, suelo radiante, fotovoltaica y almacenamiento para vivienda \par}
    \vspace{1.5cm}
    {\Large\bfseries Trabajo de Fin de Máster\par}
    \vspace{0.5cm}
    {\large Máster Universitario en Ingeniería de la Energía \par}
    \vspace{2cm}
    {\Large Luis D. Aranda Sánchez\par}
    \vfill
    Director: Javier Rodríguez Martín
    \vfill
    {\large Septiembre 6, 2024\par}
\end{titlepage}

% Resumen (máximo de 5 páginas, incluyendo al final Palabras clave)
\clearpage
\pagestyle{simple}
% \newpage
\chapter*{Resumen}
\addcontentsline{toc}{chapter}{Resumen}
\input{capitulos/resumen/main.tex}

% Índice (paginado)
\clearpage
\pagestyle{simple}
% \newpage
\tableofcontents

% Introducción (donde se incluya los antecedentes y justificación)
\clearpage
\pagestyle{myfancy}
\newpage
\chapter{Introducción}
\input{capitulos/introduccion/main.tex}

% Objetivos
\chapter{Objetivos}
\input{capitulos/objetivos/main.tex}

% Metodología
\chapter{Metodología}
\input{capitulos/metodologia/main.tex}

% Resultados y discusión (incluyendo la valoración de impactos y de aspectos de responsabilidad legal, ética y profesional relacionados con el trabajo)
\chapter{Resultados y Discusión}
\input{capitulos/resultados_discusion/main.tex}

% Conclusiones
\chapter{Conclusiones}
\input{capitulos/conclusiones/main.tex}

% Planificación temporal y presupuesto
\chapter{Planificación Temporal y Presupuesto}
\input{capitulos/planificacion_presupuesto/main.tex}

% Bibliografía
\newpage
\addcontentsline{toc}{chapter}{Bibliografía}
\printbibliography

\end{document}


% Índice (paginado)
\clearpage
\pagestyle{simple}
% \newpage
\tableofcontents

% Introducción (donde se incluya los antecedentes y justificación)
\clearpage
\pagestyle{myfancy}
\newpage
\chapter{Introducción}
\documentclass[a4paper,11pt,twoside]{report}
\usepackage[left=25mm,right=25mm,top=25mm,bottom=25mm,includehead,includefoot,headsep=15mm,footskip=15mm]{geometry}
\usepackage{graphicx}
\usepackage{fancyhdr}
\usepackage{titlesec}
\usepackage[spanish]{babel}
\usepackage[utf8]{inputenc}
\usepackage{amsmath}
\usepackage{setspace}
\usepackage{svg}
\usepackage{hyperref}
\usepackage[backend=biber,style=numeric]{biblatex}
\addbibresource{references.bib}
\hypersetup{
    colorlinks=true,
    linkcolor=blue,      % color of internal links (sections, etc.)
    urlcolor=blue,       % color of external links
    pdftitle={Optimización energética de sistema híbrido con bomba de calor, suelo radiante, fotovoltaica y almacenamiento para vivienda},    % title
    pdfauthor={Luis D. Aranda Sánchez},     % author
    pdfkeywords={palabra1, palabra2, código1, etc.} % list of keywords
}

% Font change to Arial
\usepackage{helvet}
\renewcommand{\familydefault}{\sfdefault}

% Chapter titles in uppercase and larger font
\titleformat{\chapter}[hang]{\large\bfseries}{\thechapter.}{1em}{\MakeUppercase}
\titleformat{\section}[hang]{\bfseries}{\thesection.}{1em}{}
\titleformat{\subsection}[hang]{\bfseries}{\thesubsection.}{1em}{}

% Fancyhdr setup
\setlength{\headheight}{14.30174pt} % Adjust to recommended value, slightly larger for safety
\fancyhf{} % Clear all headers and footers
\fancyhead[LE]{\nouppercase{\leftmark}}
\fancyhead[RO]{Optimización energética para vivienda}
\fancyfoot[LE]{\thepage}
\fancyfoot[RE]{Escuela Técnica Superior de Ingenieros Industriales (UPM)}
\fancyfoot[LO]{Luis D. Aranda Sánchez}
\fancyfoot[RO]{\thepage}
\renewcommand{\headrulewidth}{0.4pt}
\renewcommand{\footrulewidth}{0.4pt}

\fancypagestyle{myfancy}{
    \fancyhf{} % Clear all headers and footers
    \fancyhead[LE]{\nouppercase{\leftmark}}
    \fancyhead[RO]{Optimización energética para vivienda}
    \fancyfoot[LE]{\thepage}
    \fancyfoot[RE]{Escuela Técnica Superior de Ingenieros Industriales (UPM)}
    \fancyfoot[LO]{Luis D. Aranda Sánchez}
    \fancyfoot[RO]{\thepage}
    \renewcommand{\headrulewidth}{0.4pt}
    \renewcommand{\footrulewidth}{0.4pt}
}

\fancypagestyle{simple}{
    \fancyhf{} % Clear all headers and footers
    \renewcommand{\headrulewidth}{0pt}
    \renewcommand{\footrulewidth}{0pt}
}

% Line spacing
\setstretch{1.2}

% Document starts here
\begin{document}

% Portada
\begin{titlepage}
    \centering
    {\scshape\LARGE Universidad Politécnica de Madrid \par}
    \vspace{1cm}
    {\scshape\Large Escuela Técnica Superior de Ingenieros Industriales\par}
    \vspace{1.5cm}
    {\huge\bfseries Optimización energética de sistema híbrido con bomba de calor, suelo radiante, fotovoltaica y almacenamiento para vivienda \par}
    \vspace{1.5cm}
    {\Large\bfseries Trabajo de Fin de Máster\par}
    \vspace{0.5cm}
    {\large Máster Universitario en Ingeniería de la Energía \par}
    \vspace{2cm}
    {\Large Luis D. Aranda Sánchez\par}
    \vfill
    Director: Javier Rodríguez Martín
    \vfill
    {\large Septiembre 6, 2024\par}
\end{titlepage}

% Resumen (máximo de 5 páginas, incluyendo al final Palabras clave)
\clearpage
\pagestyle{simple}
% \newpage
\chapter*{Resumen}
\addcontentsline{toc}{chapter}{Resumen}
\input{capitulos/resumen/main.tex}

% Índice (paginado)
\clearpage
\pagestyle{simple}
% \newpage
\tableofcontents

% Introducción (donde se incluya los antecedentes y justificación)
\clearpage
\pagestyle{myfancy}
\newpage
\chapter{Introducción}
\input{capitulos/introduccion/main.tex}

% Objetivos
\chapter{Objetivos}
\input{capitulos/objetivos/main.tex}

% Metodología
\chapter{Metodología}
\input{capitulos/metodologia/main.tex}

% Resultados y discusión (incluyendo la valoración de impactos y de aspectos de responsabilidad legal, ética y profesional relacionados con el trabajo)
\chapter{Resultados y Discusión}
\input{capitulos/resultados_discusion/main.tex}

% Conclusiones
\chapter{Conclusiones}
\input{capitulos/conclusiones/main.tex}

% Planificación temporal y presupuesto
\chapter{Planificación Temporal y Presupuesto}
\input{capitulos/planificacion_presupuesto/main.tex}

% Bibliografía
\newpage
\addcontentsline{toc}{chapter}{Bibliografía}
\printbibliography

\end{document}


% Objetivos
\chapter{Objetivos}
\documentclass[a4paper,11pt,twoside]{report}
\usepackage[left=25mm,right=25mm,top=25mm,bottom=25mm,includehead,includefoot,headsep=15mm,footskip=15mm]{geometry}
\usepackage{graphicx}
\usepackage{fancyhdr}
\usepackage{titlesec}
\usepackage[spanish]{babel}
\usepackage[utf8]{inputenc}
\usepackage{amsmath}
\usepackage{setspace}
\usepackage{svg}
\usepackage{hyperref}
\usepackage[backend=biber,style=numeric]{biblatex}
\addbibresource{references.bib}
\hypersetup{
    colorlinks=true,
    linkcolor=blue,      % color of internal links (sections, etc.)
    urlcolor=blue,       % color of external links
    pdftitle={Optimización energética de sistema híbrido con bomba de calor, suelo radiante, fotovoltaica y almacenamiento para vivienda},    % title
    pdfauthor={Luis D. Aranda Sánchez},     % author
    pdfkeywords={palabra1, palabra2, código1, etc.} % list of keywords
}

% Font change to Arial
\usepackage{helvet}
\renewcommand{\familydefault}{\sfdefault}

% Chapter titles in uppercase and larger font
\titleformat{\chapter}[hang]{\large\bfseries}{\thechapter.}{1em}{\MakeUppercase}
\titleformat{\section}[hang]{\bfseries}{\thesection.}{1em}{}
\titleformat{\subsection}[hang]{\bfseries}{\thesubsection.}{1em}{}

% Fancyhdr setup
\setlength{\headheight}{14.30174pt} % Adjust to recommended value, slightly larger for safety
\fancyhf{} % Clear all headers and footers
\fancyhead[LE]{\nouppercase{\leftmark}}
\fancyhead[RO]{Optimización energética para vivienda}
\fancyfoot[LE]{\thepage}
\fancyfoot[RE]{Escuela Técnica Superior de Ingenieros Industriales (UPM)}
\fancyfoot[LO]{Luis D. Aranda Sánchez}
\fancyfoot[RO]{\thepage}
\renewcommand{\headrulewidth}{0.4pt}
\renewcommand{\footrulewidth}{0.4pt}

\fancypagestyle{myfancy}{
    \fancyhf{} % Clear all headers and footers
    \fancyhead[LE]{\nouppercase{\leftmark}}
    \fancyhead[RO]{Optimización energética para vivienda}
    \fancyfoot[LE]{\thepage}
    \fancyfoot[RE]{Escuela Técnica Superior de Ingenieros Industriales (UPM)}
    \fancyfoot[LO]{Luis D. Aranda Sánchez}
    \fancyfoot[RO]{\thepage}
    \renewcommand{\headrulewidth}{0.4pt}
    \renewcommand{\footrulewidth}{0.4pt}
}

\fancypagestyle{simple}{
    \fancyhf{} % Clear all headers and footers
    \renewcommand{\headrulewidth}{0pt}
    \renewcommand{\footrulewidth}{0pt}
}

% Line spacing
\setstretch{1.2}

% Document starts here
\begin{document}

% Portada
\begin{titlepage}
    \centering
    {\scshape\LARGE Universidad Politécnica de Madrid \par}
    \vspace{1cm}
    {\scshape\Large Escuela Técnica Superior de Ingenieros Industriales\par}
    \vspace{1.5cm}
    {\huge\bfseries Optimización energética de sistema híbrido con bomba de calor, suelo radiante, fotovoltaica y almacenamiento para vivienda \par}
    \vspace{1.5cm}
    {\Large\bfseries Trabajo de Fin de Máster\par}
    \vspace{0.5cm}
    {\large Máster Universitario en Ingeniería de la Energía \par}
    \vspace{2cm}
    {\Large Luis D. Aranda Sánchez\par}
    \vfill
    Director: Javier Rodríguez Martín
    \vfill
    {\large Septiembre 6, 2024\par}
\end{titlepage}

% Resumen (máximo de 5 páginas, incluyendo al final Palabras clave)
\clearpage
\pagestyle{simple}
% \newpage
\chapter*{Resumen}
\addcontentsline{toc}{chapter}{Resumen}
\input{capitulos/resumen/main.tex}

% Índice (paginado)
\clearpage
\pagestyle{simple}
% \newpage
\tableofcontents

% Introducción (donde se incluya los antecedentes y justificación)
\clearpage
\pagestyle{myfancy}
\newpage
\chapter{Introducción}
\input{capitulos/introduccion/main.tex}

% Objetivos
\chapter{Objetivos}
\input{capitulos/objetivos/main.tex}

% Metodología
\chapter{Metodología}
\input{capitulos/metodologia/main.tex}

% Resultados y discusión (incluyendo la valoración de impactos y de aspectos de responsabilidad legal, ética y profesional relacionados con el trabajo)
\chapter{Resultados y Discusión}
\input{capitulos/resultados_discusion/main.tex}

% Conclusiones
\chapter{Conclusiones}
\input{capitulos/conclusiones/main.tex}

% Planificación temporal y presupuesto
\chapter{Planificación Temporal y Presupuesto}
\input{capitulos/planificacion_presupuesto/main.tex}

% Bibliografía
\newpage
\addcontentsline{toc}{chapter}{Bibliografía}
\printbibliography

\end{document}


% Metodología
\chapter{Metodología}
\documentclass[a4paper,11pt,twoside]{report}
\usepackage[left=25mm,right=25mm,top=25mm,bottom=25mm,includehead,includefoot,headsep=15mm,footskip=15mm]{geometry}
\usepackage{graphicx}
\usepackage{fancyhdr}
\usepackage{titlesec}
\usepackage[spanish]{babel}
\usepackage[utf8]{inputenc}
\usepackage{amsmath}
\usepackage{setspace}
\usepackage{svg}
\usepackage{hyperref}
\usepackage[backend=biber,style=numeric]{biblatex}
\addbibresource{references.bib}
\hypersetup{
    colorlinks=true,
    linkcolor=blue,      % color of internal links (sections, etc.)
    urlcolor=blue,       % color of external links
    pdftitle={Optimización energética de sistema híbrido con bomba de calor, suelo radiante, fotovoltaica y almacenamiento para vivienda},    % title
    pdfauthor={Luis D. Aranda Sánchez},     % author
    pdfkeywords={palabra1, palabra2, código1, etc.} % list of keywords
}

% Font change to Arial
\usepackage{helvet}
\renewcommand{\familydefault}{\sfdefault}

% Chapter titles in uppercase and larger font
\titleformat{\chapter}[hang]{\large\bfseries}{\thechapter.}{1em}{\MakeUppercase}
\titleformat{\section}[hang]{\bfseries}{\thesection.}{1em}{}
\titleformat{\subsection}[hang]{\bfseries}{\thesubsection.}{1em}{}

% Fancyhdr setup
\setlength{\headheight}{14.30174pt} % Adjust to recommended value, slightly larger for safety
\fancyhf{} % Clear all headers and footers
\fancyhead[LE]{\nouppercase{\leftmark}}
\fancyhead[RO]{Optimización energética para vivienda}
\fancyfoot[LE]{\thepage}
\fancyfoot[RE]{Escuela Técnica Superior de Ingenieros Industriales (UPM)}
\fancyfoot[LO]{Luis D. Aranda Sánchez}
\fancyfoot[RO]{\thepage}
\renewcommand{\headrulewidth}{0.4pt}
\renewcommand{\footrulewidth}{0.4pt}

\fancypagestyle{myfancy}{
    \fancyhf{} % Clear all headers and footers
    \fancyhead[LE]{\nouppercase{\leftmark}}
    \fancyhead[RO]{Optimización energética para vivienda}
    \fancyfoot[LE]{\thepage}
    \fancyfoot[RE]{Escuela Técnica Superior de Ingenieros Industriales (UPM)}
    \fancyfoot[LO]{Luis D. Aranda Sánchez}
    \fancyfoot[RO]{\thepage}
    \renewcommand{\headrulewidth}{0.4pt}
    \renewcommand{\footrulewidth}{0.4pt}
}

\fancypagestyle{simple}{
    \fancyhf{} % Clear all headers and footers
    \renewcommand{\headrulewidth}{0pt}
    \renewcommand{\footrulewidth}{0pt}
}

% Line spacing
\setstretch{1.2}

% Document starts here
\begin{document}

% Portada
\begin{titlepage}
    \centering
    {\scshape\LARGE Universidad Politécnica de Madrid \par}
    \vspace{1cm}
    {\scshape\Large Escuela Técnica Superior de Ingenieros Industriales\par}
    \vspace{1.5cm}
    {\huge\bfseries Optimización energética de sistema híbrido con bomba de calor, suelo radiante, fotovoltaica y almacenamiento para vivienda \par}
    \vspace{1.5cm}
    {\Large\bfseries Trabajo de Fin de Máster\par}
    \vspace{0.5cm}
    {\large Máster Universitario en Ingeniería de la Energía \par}
    \vspace{2cm}
    {\Large Luis D. Aranda Sánchez\par}
    \vfill
    Director: Javier Rodríguez Martín
    \vfill
    {\large Septiembre 6, 2024\par}
\end{titlepage}

% Resumen (máximo de 5 páginas, incluyendo al final Palabras clave)
\clearpage
\pagestyle{simple}
% \newpage
\chapter*{Resumen}
\addcontentsline{toc}{chapter}{Resumen}
\input{capitulos/resumen/main.tex}

% Índice (paginado)
\clearpage
\pagestyle{simple}
% \newpage
\tableofcontents

% Introducción (donde se incluya los antecedentes y justificación)
\clearpage
\pagestyle{myfancy}
\newpage
\chapter{Introducción}
\input{capitulos/introduccion/main.tex}

% Objetivos
\chapter{Objetivos}
\input{capitulos/objetivos/main.tex}

% Metodología
\chapter{Metodología}
\input{capitulos/metodologia/main.tex}

% Resultados y discusión (incluyendo la valoración de impactos y de aspectos de responsabilidad legal, ética y profesional relacionados con el trabajo)
\chapter{Resultados y Discusión}
\input{capitulos/resultados_discusion/main.tex}

% Conclusiones
\chapter{Conclusiones}
\input{capitulos/conclusiones/main.tex}

% Planificación temporal y presupuesto
\chapter{Planificación Temporal y Presupuesto}
\input{capitulos/planificacion_presupuesto/main.tex}

% Bibliografía
\newpage
\addcontentsline{toc}{chapter}{Bibliografía}
\printbibliography

\end{document}


% Resultados y discusión (incluyendo la valoración de impactos y de aspectos de responsabilidad legal, ética y profesional relacionados con el trabajo)
\chapter{Resultados y Discusión}
\documentclass[a4paper,11pt,twoside]{report}
\usepackage[left=25mm,right=25mm,top=25mm,bottom=25mm,includehead,includefoot,headsep=15mm,footskip=15mm]{geometry}
\usepackage{graphicx}
\usepackage{fancyhdr}
\usepackage{titlesec}
\usepackage[spanish]{babel}
\usepackage[utf8]{inputenc}
\usepackage{amsmath}
\usepackage{setspace}
\usepackage{svg}
\usepackage{hyperref}
\usepackage[backend=biber,style=numeric]{biblatex}
\addbibresource{references.bib}
\hypersetup{
    colorlinks=true,
    linkcolor=blue,      % color of internal links (sections, etc.)
    urlcolor=blue,       % color of external links
    pdftitle={Optimización energética de sistema híbrido con bomba de calor, suelo radiante, fotovoltaica y almacenamiento para vivienda},    % title
    pdfauthor={Luis D. Aranda Sánchez},     % author
    pdfkeywords={palabra1, palabra2, código1, etc.} % list of keywords
}

% Font change to Arial
\usepackage{helvet}
\renewcommand{\familydefault}{\sfdefault}

% Chapter titles in uppercase and larger font
\titleformat{\chapter}[hang]{\large\bfseries}{\thechapter.}{1em}{\MakeUppercase}
\titleformat{\section}[hang]{\bfseries}{\thesection.}{1em}{}
\titleformat{\subsection}[hang]{\bfseries}{\thesubsection.}{1em}{}

% Fancyhdr setup
\setlength{\headheight}{14.30174pt} % Adjust to recommended value, slightly larger for safety
\fancyhf{} % Clear all headers and footers
\fancyhead[LE]{\nouppercase{\leftmark}}
\fancyhead[RO]{Optimización energética para vivienda}
\fancyfoot[LE]{\thepage}
\fancyfoot[RE]{Escuela Técnica Superior de Ingenieros Industriales (UPM)}
\fancyfoot[LO]{Luis D. Aranda Sánchez}
\fancyfoot[RO]{\thepage}
\renewcommand{\headrulewidth}{0.4pt}
\renewcommand{\footrulewidth}{0.4pt}

\fancypagestyle{myfancy}{
    \fancyhf{} % Clear all headers and footers
    \fancyhead[LE]{\nouppercase{\leftmark}}
    \fancyhead[RO]{Optimización energética para vivienda}
    \fancyfoot[LE]{\thepage}
    \fancyfoot[RE]{Escuela Técnica Superior de Ingenieros Industriales (UPM)}
    \fancyfoot[LO]{Luis D. Aranda Sánchez}
    \fancyfoot[RO]{\thepage}
    \renewcommand{\headrulewidth}{0.4pt}
    \renewcommand{\footrulewidth}{0.4pt}
}

\fancypagestyle{simple}{
    \fancyhf{} % Clear all headers and footers
    \renewcommand{\headrulewidth}{0pt}
    \renewcommand{\footrulewidth}{0pt}
}

% Line spacing
\setstretch{1.2}

% Document starts here
\begin{document}

% Portada
\begin{titlepage}
    \centering
    {\scshape\LARGE Universidad Politécnica de Madrid \par}
    \vspace{1cm}
    {\scshape\Large Escuela Técnica Superior de Ingenieros Industriales\par}
    \vspace{1.5cm}
    {\huge\bfseries Optimización energética de sistema híbrido con bomba de calor, suelo radiante, fotovoltaica y almacenamiento para vivienda \par}
    \vspace{1.5cm}
    {\Large\bfseries Trabajo de Fin de Máster\par}
    \vspace{0.5cm}
    {\large Máster Universitario en Ingeniería de la Energía \par}
    \vspace{2cm}
    {\Large Luis D. Aranda Sánchez\par}
    \vfill
    Director: Javier Rodríguez Martín
    \vfill
    {\large Septiembre 6, 2024\par}
\end{titlepage}

% Resumen (máximo de 5 páginas, incluyendo al final Palabras clave)
\clearpage
\pagestyle{simple}
% \newpage
\chapter*{Resumen}
\addcontentsline{toc}{chapter}{Resumen}
\input{capitulos/resumen/main.tex}

% Índice (paginado)
\clearpage
\pagestyle{simple}
% \newpage
\tableofcontents

% Introducción (donde se incluya los antecedentes y justificación)
\clearpage
\pagestyle{myfancy}
\newpage
\chapter{Introducción}
\input{capitulos/introduccion/main.tex}

% Objetivos
\chapter{Objetivos}
\input{capitulos/objetivos/main.tex}

% Metodología
\chapter{Metodología}
\input{capitulos/metodologia/main.tex}

% Resultados y discusión (incluyendo la valoración de impactos y de aspectos de responsabilidad legal, ética y profesional relacionados con el trabajo)
\chapter{Resultados y Discusión}
\input{capitulos/resultados_discusion/main.tex}

% Conclusiones
\chapter{Conclusiones}
\input{capitulos/conclusiones/main.tex}

% Planificación temporal y presupuesto
\chapter{Planificación Temporal y Presupuesto}
\input{capitulos/planificacion_presupuesto/main.tex}

% Bibliografía
\newpage
\addcontentsline{toc}{chapter}{Bibliografía}
\printbibliography

\end{document}


% Conclusiones
\chapter{Conclusiones}
\documentclass[a4paper,11pt,twoside]{report}
\usepackage[left=25mm,right=25mm,top=25mm,bottom=25mm,includehead,includefoot,headsep=15mm,footskip=15mm]{geometry}
\usepackage{graphicx}
\usepackage{fancyhdr}
\usepackage{titlesec}
\usepackage[spanish]{babel}
\usepackage[utf8]{inputenc}
\usepackage{amsmath}
\usepackage{setspace}
\usepackage{svg}
\usepackage{hyperref}
\usepackage[backend=biber,style=numeric]{biblatex}
\addbibresource{references.bib}
\hypersetup{
    colorlinks=true,
    linkcolor=blue,      % color of internal links (sections, etc.)
    urlcolor=blue,       % color of external links
    pdftitle={Optimización energética de sistema híbrido con bomba de calor, suelo radiante, fotovoltaica y almacenamiento para vivienda},    % title
    pdfauthor={Luis D. Aranda Sánchez},     % author
    pdfkeywords={palabra1, palabra2, código1, etc.} % list of keywords
}

% Font change to Arial
\usepackage{helvet}
\renewcommand{\familydefault}{\sfdefault}

% Chapter titles in uppercase and larger font
\titleformat{\chapter}[hang]{\large\bfseries}{\thechapter.}{1em}{\MakeUppercase}
\titleformat{\section}[hang]{\bfseries}{\thesection.}{1em}{}
\titleformat{\subsection}[hang]{\bfseries}{\thesubsection.}{1em}{}

% Fancyhdr setup
\setlength{\headheight}{14.30174pt} % Adjust to recommended value, slightly larger for safety
\fancyhf{} % Clear all headers and footers
\fancyhead[LE]{\nouppercase{\leftmark}}
\fancyhead[RO]{Optimización energética para vivienda}
\fancyfoot[LE]{\thepage}
\fancyfoot[RE]{Escuela Técnica Superior de Ingenieros Industriales (UPM)}
\fancyfoot[LO]{Luis D. Aranda Sánchez}
\fancyfoot[RO]{\thepage}
\renewcommand{\headrulewidth}{0.4pt}
\renewcommand{\footrulewidth}{0.4pt}

\fancypagestyle{myfancy}{
    \fancyhf{} % Clear all headers and footers
    \fancyhead[LE]{\nouppercase{\leftmark}}
    \fancyhead[RO]{Optimización energética para vivienda}
    \fancyfoot[LE]{\thepage}
    \fancyfoot[RE]{Escuela Técnica Superior de Ingenieros Industriales (UPM)}
    \fancyfoot[LO]{Luis D. Aranda Sánchez}
    \fancyfoot[RO]{\thepage}
    \renewcommand{\headrulewidth}{0.4pt}
    \renewcommand{\footrulewidth}{0.4pt}
}

\fancypagestyle{simple}{
    \fancyhf{} % Clear all headers and footers
    \renewcommand{\headrulewidth}{0pt}
    \renewcommand{\footrulewidth}{0pt}
}

% Line spacing
\setstretch{1.2}

% Document starts here
\begin{document}

% Portada
\begin{titlepage}
    \centering
    {\scshape\LARGE Universidad Politécnica de Madrid \par}
    \vspace{1cm}
    {\scshape\Large Escuela Técnica Superior de Ingenieros Industriales\par}
    \vspace{1.5cm}
    {\huge\bfseries Optimización energética de sistema híbrido con bomba de calor, suelo radiante, fotovoltaica y almacenamiento para vivienda \par}
    \vspace{1.5cm}
    {\Large\bfseries Trabajo de Fin de Máster\par}
    \vspace{0.5cm}
    {\large Máster Universitario en Ingeniería de la Energía \par}
    \vspace{2cm}
    {\Large Luis D. Aranda Sánchez\par}
    \vfill
    Director: Javier Rodríguez Martín
    \vfill
    {\large Septiembre 6, 2024\par}
\end{titlepage}

% Resumen (máximo de 5 páginas, incluyendo al final Palabras clave)
\clearpage
\pagestyle{simple}
% \newpage
\chapter*{Resumen}
\addcontentsline{toc}{chapter}{Resumen}
\input{capitulos/resumen/main.tex}

% Índice (paginado)
\clearpage
\pagestyle{simple}
% \newpage
\tableofcontents

% Introducción (donde se incluya los antecedentes y justificación)
\clearpage
\pagestyle{myfancy}
\newpage
\chapter{Introducción}
\input{capitulos/introduccion/main.tex}

% Objetivos
\chapter{Objetivos}
\input{capitulos/objetivos/main.tex}

% Metodología
\chapter{Metodología}
\input{capitulos/metodologia/main.tex}

% Resultados y discusión (incluyendo la valoración de impactos y de aspectos de responsabilidad legal, ética y profesional relacionados con el trabajo)
\chapter{Resultados y Discusión}
\input{capitulos/resultados_discusion/main.tex}

% Conclusiones
\chapter{Conclusiones}
\input{capitulos/conclusiones/main.tex}

% Planificación temporal y presupuesto
\chapter{Planificación Temporal y Presupuesto}
\input{capitulos/planificacion_presupuesto/main.tex}

% Bibliografía
\newpage
\addcontentsline{toc}{chapter}{Bibliografía}
\printbibliography

\end{document}


% Planificación temporal y presupuesto
\chapter{Planificación Temporal y Presupuesto}
\documentclass[a4paper,11pt,twoside]{report}
\usepackage[left=25mm,right=25mm,top=25mm,bottom=25mm,includehead,includefoot,headsep=15mm,footskip=15mm]{geometry}
\usepackage{graphicx}
\usepackage{fancyhdr}
\usepackage{titlesec}
\usepackage[spanish]{babel}
\usepackage[utf8]{inputenc}
\usepackage{amsmath}
\usepackage{setspace}
\usepackage{svg}
\usepackage{hyperref}
\usepackage[backend=biber,style=numeric]{biblatex}
\addbibresource{references.bib}
\hypersetup{
    colorlinks=true,
    linkcolor=blue,      % color of internal links (sections, etc.)
    urlcolor=blue,       % color of external links
    pdftitle={Optimización energética de sistema híbrido con bomba de calor, suelo radiante, fotovoltaica y almacenamiento para vivienda},    % title
    pdfauthor={Luis D. Aranda Sánchez},     % author
    pdfkeywords={palabra1, palabra2, código1, etc.} % list of keywords
}

% Font change to Arial
\usepackage{helvet}
\renewcommand{\familydefault}{\sfdefault}

% Chapter titles in uppercase and larger font
\titleformat{\chapter}[hang]{\large\bfseries}{\thechapter.}{1em}{\MakeUppercase}
\titleformat{\section}[hang]{\bfseries}{\thesection.}{1em}{}
\titleformat{\subsection}[hang]{\bfseries}{\thesubsection.}{1em}{}

% Fancyhdr setup
\setlength{\headheight}{14.30174pt} % Adjust to recommended value, slightly larger for safety
\fancyhf{} % Clear all headers and footers
\fancyhead[LE]{\nouppercase{\leftmark}}
\fancyhead[RO]{Optimización energética para vivienda}
\fancyfoot[LE]{\thepage}
\fancyfoot[RE]{Escuela Técnica Superior de Ingenieros Industriales (UPM)}
\fancyfoot[LO]{Luis D. Aranda Sánchez}
\fancyfoot[RO]{\thepage}
\renewcommand{\headrulewidth}{0.4pt}
\renewcommand{\footrulewidth}{0.4pt}

\fancypagestyle{myfancy}{
    \fancyhf{} % Clear all headers and footers
    \fancyhead[LE]{\nouppercase{\leftmark}}
    \fancyhead[RO]{Optimización energética para vivienda}
    \fancyfoot[LE]{\thepage}
    \fancyfoot[RE]{Escuela Técnica Superior de Ingenieros Industriales (UPM)}
    \fancyfoot[LO]{Luis D. Aranda Sánchez}
    \fancyfoot[RO]{\thepage}
    \renewcommand{\headrulewidth}{0.4pt}
    \renewcommand{\footrulewidth}{0.4pt}
}

\fancypagestyle{simple}{
    \fancyhf{} % Clear all headers and footers
    \renewcommand{\headrulewidth}{0pt}
    \renewcommand{\footrulewidth}{0pt}
}

% Line spacing
\setstretch{1.2}

% Document starts here
\begin{document}

% Portada
\begin{titlepage}
    \centering
    {\scshape\LARGE Universidad Politécnica de Madrid \par}
    \vspace{1cm}
    {\scshape\Large Escuela Técnica Superior de Ingenieros Industriales\par}
    \vspace{1.5cm}
    {\huge\bfseries Optimización energética de sistema híbrido con bomba de calor, suelo radiante, fotovoltaica y almacenamiento para vivienda \par}
    \vspace{1.5cm}
    {\Large\bfseries Trabajo de Fin de Máster\par}
    \vspace{0.5cm}
    {\large Máster Universitario en Ingeniería de la Energía \par}
    \vspace{2cm}
    {\Large Luis D. Aranda Sánchez\par}
    \vfill
    Director: Javier Rodríguez Martín
    \vfill
    {\large Septiembre 6, 2024\par}
\end{titlepage}

% Resumen (máximo de 5 páginas, incluyendo al final Palabras clave)
\clearpage
\pagestyle{simple}
% \newpage
\chapter*{Resumen}
\addcontentsline{toc}{chapter}{Resumen}
\input{capitulos/resumen/main.tex}

% Índice (paginado)
\clearpage
\pagestyle{simple}
% \newpage
\tableofcontents

% Introducción (donde se incluya los antecedentes y justificación)
\clearpage
\pagestyle{myfancy}
\newpage
\chapter{Introducción}
\input{capitulos/introduccion/main.tex}

% Objetivos
\chapter{Objetivos}
\input{capitulos/objetivos/main.tex}

% Metodología
\chapter{Metodología}
\input{capitulos/metodologia/main.tex}

% Resultados y discusión (incluyendo la valoración de impactos y de aspectos de responsabilidad legal, ética y profesional relacionados con el trabajo)
\chapter{Resultados y Discusión}
\input{capitulos/resultados_discusion/main.tex}

% Conclusiones
\chapter{Conclusiones}
\input{capitulos/conclusiones/main.tex}

% Planificación temporal y presupuesto
\chapter{Planificación Temporal y Presupuesto}
\input{capitulos/planificacion_presupuesto/main.tex}

% Bibliografía
\newpage
\addcontentsline{toc}{chapter}{Bibliografía}
\printbibliography

\end{document}


% Bibliografía
\newpage
\addcontentsline{toc}{chapter}{Bibliografía}
\printbibliography

\end{document}


% Planificación temporal y presupuesto
\chapter{Planificación Temporal y Presupuesto}
\documentclass[a4paper,11pt,twoside]{report}
\usepackage[left=25mm,right=25mm,top=25mm,bottom=25mm,includehead,includefoot,headsep=15mm,footskip=15mm]{geometry}
\usepackage{graphicx}
\usepackage{fancyhdr}
\usepackage{titlesec}
\usepackage[spanish]{babel}
\usepackage[utf8]{inputenc}
\usepackage{amsmath}
\usepackage{setspace}
\usepackage{svg}
\usepackage{hyperref}
\usepackage[backend=biber,style=numeric]{biblatex}
\addbibresource{references.bib}
\hypersetup{
    colorlinks=true,
    linkcolor=blue,      % color of internal links (sections, etc.)
    urlcolor=blue,       % color of external links
    pdftitle={Optimización energética de sistema híbrido con bomba de calor, suelo radiante, fotovoltaica y almacenamiento para vivienda},    % title
    pdfauthor={Luis D. Aranda Sánchez},     % author
    pdfkeywords={palabra1, palabra2, código1, etc.} % list of keywords
}

% Font change to Arial
\usepackage{helvet}
\renewcommand{\familydefault}{\sfdefault}

% Chapter titles in uppercase and larger font
\titleformat{\chapter}[hang]{\large\bfseries}{\thechapter.}{1em}{\MakeUppercase}
\titleformat{\section}[hang]{\bfseries}{\thesection.}{1em}{}
\titleformat{\subsection}[hang]{\bfseries}{\thesubsection.}{1em}{}

% Fancyhdr setup
\setlength{\headheight}{14.30174pt} % Adjust to recommended value, slightly larger for safety
\fancyhf{} % Clear all headers and footers
\fancyhead[LE]{\nouppercase{\leftmark}}
\fancyhead[RO]{Optimización energética para vivienda}
\fancyfoot[LE]{\thepage}
\fancyfoot[RE]{Escuela Técnica Superior de Ingenieros Industriales (UPM)}
\fancyfoot[LO]{Luis D. Aranda Sánchez}
\fancyfoot[RO]{\thepage}
\renewcommand{\headrulewidth}{0.4pt}
\renewcommand{\footrulewidth}{0.4pt}

\fancypagestyle{myfancy}{
    \fancyhf{} % Clear all headers and footers
    \fancyhead[LE]{\nouppercase{\leftmark}}
    \fancyhead[RO]{Optimización energética para vivienda}
    \fancyfoot[LE]{\thepage}
    \fancyfoot[RE]{Escuela Técnica Superior de Ingenieros Industriales (UPM)}
    \fancyfoot[LO]{Luis D. Aranda Sánchez}
    \fancyfoot[RO]{\thepage}
    \renewcommand{\headrulewidth}{0.4pt}
    \renewcommand{\footrulewidth}{0.4pt}
}

\fancypagestyle{simple}{
    \fancyhf{} % Clear all headers and footers
    \renewcommand{\headrulewidth}{0pt}
    \renewcommand{\footrulewidth}{0pt}
}

% Line spacing
\setstretch{1.2}

% Document starts here
\begin{document}

% Portada
\begin{titlepage}
    \centering
    {\scshape\LARGE Universidad Politécnica de Madrid \par}
    \vspace{1cm}
    {\scshape\Large Escuela Técnica Superior de Ingenieros Industriales\par}
    \vspace{1.5cm}
    {\huge\bfseries Optimización energética de sistema híbrido con bomba de calor, suelo radiante, fotovoltaica y almacenamiento para vivienda \par}
    \vspace{1.5cm}
    {\Large\bfseries Trabajo de Fin de Máster\par}
    \vspace{0.5cm}
    {\large Máster Universitario en Ingeniería de la Energía \par}
    \vspace{2cm}
    {\Large Luis D. Aranda Sánchez\par}
    \vfill
    Director: Javier Rodríguez Martín
    \vfill
    {\large Septiembre 6, 2024\par}
\end{titlepage}

% Resumen (máximo de 5 páginas, incluyendo al final Palabras clave)
\clearpage
\pagestyle{simple}
% \newpage
\chapter*{Resumen}
\addcontentsline{toc}{chapter}{Resumen}
\documentclass[a4paper,11pt,twoside]{report}
\usepackage[left=25mm,right=25mm,top=25mm,bottom=25mm,includehead,includefoot,headsep=15mm,footskip=15mm]{geometry}
\usepackage{graphicx}
\usepackage{fancyhdr}
\usepackage{titlesec}
\usepackage[spanish]{babel}
\usepackage[utf8]{inputenc}
\usepackage{amsmath}
\usepackage{setspace}
\usepackage{svg}
\usepackage{hyperref}
\usepackage[backend=biber,style=numeric]{biblatex}
\addbibresource{references.bib}
\hypersetup{
    colorlinks=true,
    linkcolor=blue,      % color of internal links (sections, etc.)
    urlcolor=blue,       % color of external links
    pdftitle={Optimización energética de sistema híbrido con bomba de calor, suelo radiante, fotovoltaica y almacenamiento para vivienda},    % title
    pdfauthor={Luis D. Aranda Sánchez},     % author
    pdfkeywords={palabra1, palabra2, código1, etc.} % list of keywords
}

% Font change to Arial
\usepackage{helvet}
\renewcommand{\familydefault}{\sfdefault}

% Chapter titles in uppercase and larger font
\titleformat{\chapter}[hang]{\large\bfseries}{\thechapter.}{1em}{\MakeUppercase}
\titleformat{\section}[hang]{\bfseries}{\thesection.}{1em}{}
\titleformat{\subsection}[hang]{\bfseries}{\thesubsection.}{1em}{}

% Fancyhdr setup
\setlength{\headheight}{14.30174pt} % Adjust to recommended value, slightly larger for safety
\fancyhf{} % Clear all headers and footers
\fancyhead[LE]{\nouppercase{\leftmark}}
\fancyhead[RO]{Optimización energética para vivienda}
\fancyfoot[LE]{\thepage}
\fancyfoot[RE]{Escuela Técnica Superior de Ingenieros Industriales (UPM)}
\fancyfoot[LO]{Luis D. Aranda Sánchez}
\fancyfoot[RO]{\thepage}
\renewcommand{\headrulewidth}{0.4pt}
\renewcommand{\footrulewidth}{0.4pt}

\fancypagestyle{myfancy}{
    \fancyhf{} % Clear all headers and footers
    \fancyhead[LE]{\nouppercase{\leftmark}}
    \fancyhead[RO]{Optimización energética para vivienda}
    \fancyfoot[LE]{\thepage}
    \fancyfoot[RE]{Escuela Técnica Superior de Ingenieros Industriales (UPM)}
    \fancyfoot[LO]{Luis D. Aranda Sánchez}
    \fancyfoot[RO]{\thepage}
    \renewcommand{\headrulewidth}{0.4pt}
    \renewcommand{\footrulewidth}{0.4pt}
}

\fancypagestyle{simple}{
    \fancyhf{} % Clear all headers and footers
    \renewcommand{\headrulewidth}{0pt}
    \renewcommand{\footrulewidth}{0pt}
}

% Line spacing
\setstretch{1.2}

% Document starts here
\begin{document}

% Portada
\begin{titlepage}
    \centering
    {\scshape\LARGE Universidad Politécnica de Madrid \par}
    \vspace{1cm}
    {\scshape\Large Escuela Técnica Superior de Ingenieros Industriales\par}
    \vspace{1.5cm}
    {\huge\bfseries Optimización energética de sistema híbrido con bomba de calor, suelo radiante, fotovoltaica y almacenamiento para vivienda \par}
    \vspace{1.5cm}
    {\Large\bfseries Trabajo de Fin de Máster\par}
    \vspace{0.5cm}
    {\large Máster Universitario en Ingeniería de la Energía \par}
    \vspace{2cm}
    {\Large Luis D. Aranda Sánchez\par}
    \vfill
    Director: Javier Rodríguez Martín
    \vfill
    {\large Septiembre 6, 2024\par}
\end{titlepage}

% Resumen (máximo de 5 páginas, incluyendo al final Palabras clave)
\clearpage
\pagestyle{simple}
% \newpage
\chapter*{Resumen}
\addcontentsline{toc}{chapter}{Resumen}
\input{capitulos/resumen/main.tex}

% Índice (paginado)
\clearpage
\pagestyle{simple}
% \newpage
\tableofcontents

% Introducción (donde se incluya los antecedentes y justificación)
\clearpage
\pagestyle{myfancy}
\newpage
\chapter{Introducción}
\input{capitulos/introduccion/main.tex}

% Objetivos
\chapter{Objetivos}
\input{capitulos/objetivos/main.tex}

% Metodología
\chapter{Metodología}
\input{capitulos/metodologia/main.tex}

% Resultados y discusión (incluyendo la valoración de impactos y de aspectos de responsabilidad legal, ética y profesional relacionados con el trabajo)
\chapter{Resultados y Discusión}
\input{capitulos/resultados_discusion/main.tex}

% Conclusiones
\chapter{Conclusiones}
\input{capitulos/conclusiones/main.tex}

% Planificación temporal y presupuesto
\chapter{Planificación Temporal y Presupuesto}
\input{capitulos/planificacion_presupuesto/main.tex}

% Bibliografía
\newpage
\addcontentsline{toc}{chapter}{Bibliografía}
\printbibliography

\end{document}


% Índice (paginado)
\clearpage
\pagestyle{simple}
% \newpage
\tableofcontents

% Introducción (donde se incluya los antecedentes y justificación)
\clearpage
\pagestyle{myfancy}
\newpage
\chapter{Introducción}
\documentclass[a4paper,11pt,twoside]{report}
\usepackage[left=25mm,right=25mm,top=25mm,bottom=25mm,includehead,includefoot,headsep=15mm,footskip=15mm]{geometry}
\usepackage{graphicx}
\usepackage{fancyhdr}
\usepackage{titlesec}
\usepackage[spanish]{babel}
\usepackage[utf8]{inputenc}
\usepackage{amsmath}
\usepackage{setspace}
\usepackage{svg}
\usepackage{hyperref}
\usepackage[backend=biber,style=numeric]{biblatex}
\addbibresource{references.bib}
\hypersetup{
    colorlinks=true,
    linkcolor=blue,      % color of internal links (sections, etc.)
    urlcolor=blue,       % color of external links
    pdftitle={Optimización energética de sistema híbrido con bomba de calor, suelo radiante, fotovoltaica y almacenamiento para vivienda},    % title
    pdfauthor={Luis D. Aranda Sánchez},     % author
    pdfkeywords={palabra1, palabra2, código1, etc.} % list of keywords
}

% Font change to Arial
\usepackage{helvet}
\renewcommand{\familydefault}{\sfdefault}

% Chapter titles in uppercase and larger font
\titleformat{\chapter}[hang]{\large\bfseries}{\thechapter.}{1em}{\MakeUppercase}
\titleformat{\section}[hang]{\bfseries}{\thesection.}{1em}{}
\titleformat{\subsection}[hang]{\bfseries}{\thesubsection.}{1em}{}

% Fancyhdr setup
\setlength{\headheight}{14.30174pt} % Adjust to recommended value, slightly larger for safety
\fancyhf{} % Clear all headers and footers
\fancyhead[LE]{\nouppercase{\leftmark}}
\fancyhead[RO]{Optimización energética para vivienda}
\fancyfoot[LE]{\thepage}
\fancyfoot[RE]{Escuela Técnica Superior de Ingenieros Industriales (UPM)}
\fancyfoot[LO]{Luis D. Aranda Sánchez}
\fancyfoot[RO]{\thepage}
\renewcommand{\headrulewidth}{0.4pt}
\renewcommand{\footrulewidth}{0.4pt}

\fancypagestyle{myfancy}{
    \fancyhf{} % Clear all headers and footers
    \fancyhead[LE]{\nouppercase{\leftmark}}
    \fancyhead[RO]{Optimización energética para vivienda}
    \fancyfoot[LE]{\thepage}
    \fancyfoot[RE]{Escuela Técnica Superior de Ingenieros Industriales (UPM)}
    \fancyfoot[LO]{Luis D. Aranda Sánchez}
    \fancyfoot[RO]{\thepage}
    \renewcommand{\headrulewidth}{0.4pt}
    \renewcommand{\footrulewidth}{0.4pt}
}

\fancypagestyle{simple}{
    \fancyhf{} % Clear all headers and footers
    \renewcommand{\headrulewidth}{0pt}
    \renewcommand{\footrulewidth}{0pt}
}

% Line spacing
\setstretch{1.2}

% Document starts here
\begin{document}

% Portada
\begin{titlepage}
    \centering
    {\scshape\LARGE Universidad Politécnica de Madrid \par}
    \vspace{1cm}
    {\scshape\Large Escuela Técnica Superior de Ingenieros Industriales\par}
    \vspace{1.5cm}
    {\huge\bfseries Optimización energética de sistema híbrido con bomba de calor, suelo radiante, fotovoltaica y almacenamiento para vivienda \par}
    \vspace{1.5cm}
    {\Large\bfseries Trabajo de Fin de Máster\par}
    \vspace{0.5cm}
    {\large Máster Universitario en Ingeniería de la Energía \par}
    \vspace{2cm}
    {\Large Luis D. Aranda Sánchez\par}
    \vfill
    Director: Javier Rodríguez Martín
    \vfill
    {\large Septiembre 6, 2024\par}
\end{titlepage}

% Resumen (máximo de 5 páginas, incluyendo al final Palabras clave)
\clearpage
\pagestyle{simple}
% \newpage
\chapter*{Resumen}
\addcontentsline{toc}{chapter}{Resumen}
\input{capitulos/resumen/main.tex}

% Índice (paginado)
\clearpage
\pagestyle{simple}
% \newpage
\tableofcontents

% Introducción (donde se incluya los antecedentes y justificación)
\clearpage
\pagestyle{myfancy}
\newpage
\chapter{Introducción}
\input{capitulos/introduccion/main.tex}

% Objetivos
\chapter{Objetivos}
\input{capitulos/objetivos/main.tex}

% Metodología
\chapter{Metodología}
\input{capitulos/metodologia/main.tex}

% Resultados y discusión (incluyendo la valoración de impactos y de aspectos de responsabilidad legal, ética y profesional relacionados con el trabajo)
\chapter{Resultados y Discusión}
\input{capitulos/resultados_discusion/main.tex}

% Conclusiones
\chapter{Conclusiones}
\input{capitulos/conclusiones/main.tex}

% Planificación temporal y presupuesto
\chapter{Planificación Temporal y Presupuesto}
\input{capitulos/planificacion_presupuesto/main.tex}

% Bibliografía
\newpage
\addcontentsline{toc}{chapter}{Bibliografía}
\printbibliography

\end{document}


% Objetivos
\chapter{Objetivos}
\documentclass[a4paper,11pt,twoside]{report}
\usepackage[left=25mm,right=25mm,top=25mm,bottom=25mm,includehead,includefoot,headsep=15mm,footskip=15mm]{geometry}
\usepackage{graphicx}
\usepackage{fancyhdr}
\usepackage{titlesec}
\usepackage[spanish]{babel}
\usepackage[utf8]{inputenc}
\usepackage{amsmath}
\usepackage{setspace}
\usepackage{svg}
\usepackage{hyperref}
\usepackage[backend=biber,style=numeric]{biblatex}
\addbibresource{references.bib}
\hypersetup{
    colorlinks=true,
    linkcolor=blue,      % color of internal links (sections, etc.)
    urlcolor=blue,       % color of external links
    pdftitle={Optimización energética de sistema híbrido con bomba de calor, suelo radiante, fotovoltaica y almacenamiento para vivienda},    % title
    pdfauthor={Luis D. Aranda Sánchez},     % author
    pdfkeywords={palabra1, palabra2, código1, etc.} % list of keywords
}

% Font change to Arial
\usepackage{helvet}
\renewcommand{\familydefault}{\sfdefault}

% Chapter titles in uppercase and larger font
\titleformat{\chapter}[hang]{\large\bfseries}{\thechapter.}{1em}{\MakeUppercase}
\titleformat{\section}[hang]{\bfseries}{\thesection.}{1em}{}
\titleformat{\subsection}[hang]{\bfseries}{\thesubsection.}{1em}{}

% Fancyhdr setup
\setlength{\headheight}{14.30174pt} % Adjust to recommended value, slightly larger for safety
\fancyhf{} % Clear all headers and footers
\fancyhead[LE]{\nouppercase{\leftmark}}
\fancyhead[RO]{Optimización energética para vivienda}
\fancyfoot[LE]{\thepage}
\fancyfoot[RE]{Escuela Técnica Superior de Ingenieros Industriales (UPM)}
\fancyfoot[LO]{Luis D. Aranda Sánchez}
\fancyfoot[RO]{\thepage}
\renewcommand{\headrulewidth}{0.4pt}
\renewcommand{\footrulewidth}{0.4pt}

\fancypagestyle{myfancy}{
    \fancyhf{} % Clear all headers and footers
    \fancyhead[LE]{\nouppercase{\leftmark}}
    \fancyhead[RO]{Optimización energética para vivienda}
    \fancyfoot[LE]{\thepage}
    \fancyfoot[RE]{Escuela Técnica Superior de Ingenieros Industriales (UPM)}
    \fancyfoot[LO]{Luis D. Aranda Sánchez}
    \fancyfoot[RO]{\thepage}
    \renewcommand{\headrulewidth}{0.4pt}
    \renewcommand{\footrulewidth}{0.4pt}
}

\fancypagestyle{simple}{
    \fancyhf{} % Clear all headers and footers
    \renewcommand{\headrulewidth}{0pt}
    \renewcommand{\footrulewidth}{0pt}
}

% Line spacing
\setstretch{1.2}

% Document starts here
\begin{document}

% Portada
\begin{titlepage}
    \centering
    {\scshape\LARGE Universidad Politécnica de Madrid \par}
    \vspace{1cm}
    {\scshape\Large Escuela Técnica Superior de Ingenieros Industriales\par}
    \vspace{1.5cm}
    {\huge\bfseries Optimización energética de sistema híbrido con bomba de calor, suelo radiante, fotovoltaica y almacenamiento para vivienda \par}
    \vspace{1.5cm}
    {\Large\bfseries Trabajo de Fin de Máster\par}
    \vspace{0.5cm}
    {\large Máster Universitario en Ingeniería de la Energía \par}
    \vspace{2cm}
    {\Large Luis D. Aranda Sánchez\par}
    \vfill
    Director: Javier Rodríguez Martín
    \vfill
    {\large Septiembre 6, 2024\par}
\end{titlepage}

% Resumen (máximo de 5 páginas, incluyendo al final Palabras clave)
\clearpage
\pagestyle{simple}
% \newpage
\chapter*{Resumen}
\addcontentsline{toc}{chapter}{Resumen}
\input{capitulos/resumen/main.tex}

% Índice (paginado)
\clearpage
\pagestyle{simple}
% \newpage
\tableofcontents

% Introducción (donde se incluya los antecedentes y justificación)
\clearpage
\pagestyle{myfancy}
\newpage
\chapter{Introducción}
\input{capitulos/introduccion/main.tex}

% Objetivos
\chapter{Objetivos}
\input{capitulos/objetivos/main.tex}

% Metodología
\chapter{Metodología}
\input{capitulos/metodologia/main.tex}

% Resultados y discusión (incluyendo la valoración de impactos y de aspectos de responsabilidad legal, ética y profesional relacionados con el trabajo)
\chapter{Resultados y Discusión}
\input{capitulos/resultados_discusion/main.tex}

% Conclusiones
\chapter{Conclusiones}
\input{capitulos/conclusiones/main.tex}

% Planificación temporal y presupuesto
\chapter{Planificación Temporal y Presupuesto}
\input{capitulos/planificacion_presupuesto/main.tex}

% Bibliografía
\newpage
\addcontentsline{toc}{chapter}{Bibliografía}
\printbibliography

\end{document}


% Metodología
\chapter{Metodología}
\documentclass[a4paper,11pt,twoside]{report}
\usepackage[left=25mm,right=25mm,top=25mm,bottom=25mm,includehead,includefoot,headsep=15mm,footskip=15mm]{geometry}
\usepackage{graphicx}
\usepackage{fancyhdr}
\usepackage{titlesec}
\usepackage[spanish]{babel}
\usepackage[utf8]{inputenc}
\usepackage{amsmath}
\usepackage{setspace}
\usepackage{svg}
\usepackage{hyperref}
\usepackage[backend=biber,style=numeric]{biblatex}
\addbibresource{references.bib}
\hypersetup{
    colorlinks=true,
    linkcolor=blue,      % color of internal links (sections, etc.)
    urlcolor=blue,       % color of external links
    pdftitle={Optimización energética de sistema híbrido con bomba de calor, suelo radiante, fotovoltaica y almacenamiento para vivienda},    % title
    pdfauthor={Luis D. Aranda Sánchez},     % author
    pdfkeywords={palabra1, palabra2, código1, etc.} % list of keywords
}

% Font change to Arial
\usepackage{helvet}
\renewcommand{\familydefault}{\sfdefault}

% Chapter titles in uppercase and larger font
\titleformat{\chapter}[hang]{\large\bfseries}{\thechapter.}{1em}{\MakeUppercase}
\titleformat{\section}[hang]{\bfseries}{\thesection.}{1em}{}
\titleformat{\subsection}[hang]{\bfseries}{\thesubsection.}{1em}{}

% Fancyhdr setup
\setlength{\headheight}{14.30174pt} % Adjust to recommended value, slightly larger for safety
\fancyhf{} % Clear all headers and footers
\fancyhead[LE]{\nouppercase{\leftmark}}
\fancyhead[RO]{Optimización energética para vivienda}
\fancyfoot[LE]{\thepage}
\fancyfoot[RE]{Escuela Técnica Superior de Ingenieros Industriales (UPM)}
\fancyfoot[LO]{Luis D. Aranda Sánchez}
\fancyfoot[RO]{\thepage}
\renewcommand{\headrulewidth}{0.4pt}
\renewcommand{\footrulewidth}{0.4pt}

\fancypagestyle{myfancy}{
    \fancyhf{} % Clear all headers and footers
    \fancyhead[LE]{\nouppercase{\leftmark}}
    \fancyhead[RO]{Optimización energética para vivienda}
    \fancyfoot[LE]{\thepage}
    \fancyfoot[RE]{Escuela Técnica Superior de Ingenieros Industriales (UPM)}
    \fancyfoot[LO]{Luis D. Aranda Sánchez}
    \fancyfoot[RO]{\thepage}
    \renewcommand{\headrulewidth}{0.4pt}
    \renewcommand{\footrulewidth}{0.4pt}
}

\fancypagestyle{simple}{
    \fancyhf{} % Clear all headers and footers
    \renewcommand{\headrulewidth}{0pt}
    \renewcommand{\footrulewidth}{0pt}
}

% Line spacing
\setstretch{1.2}

% Document starts here
\begin{document}

% Portada
\begin{titlepage}
    \centering
    {\scshape\LARGE Universidad Politécnica de Madrid \par}
    \vspace{1cm}
    {\scshape\Large Escuela Técnica Superior de Ingenieros Industriales\par}
    \vspace{1.5cm}
    {\huge\bfseries Optimización energética de sistema híbrido con bomba de calor, suelo radiante, fotovoltaica y almacenamiento para vivienda \par}
    \vspace{1.5cm}
    {\Large\bfseries Trabajo de Fin de Máster\par}
    \vspace{0.5cm}
    {\large Máster Universitario en Ingeniería de la Energía \par}
    \vspace{2cm}
    {\Large Luis D. Aranda Sánchez\par}
    \vfill
    Director: Javier Rodríguez Martín
    \vfill
    {\large Septiembre 6, 2024\par}
\end{titlepage}

% Resumen (máximo de 5 páginas, incluyendo al final Palabras clave)
\clearpage
\pagestyle{simple}
% \newpage
\chapter*{Resumen}
\addcontentsline{toc}{chapter}{Resumen}
\input{capitulos/resumen/main.tex}

% Índice (paginado)
\clearpage
\pagestyle{simple}
% \newpage
\tableofcontents

% Introducción (donde se incluya los antecedentes y justificación)
\clearpage
\pagestyle{myfancy}
\newpage
\chapter{Introducción}
\input{capitulos/introduccion/main.tex}

% Objetivos
\chapter{Objetivos}
\input{capitulos/objetivos/main.tex}

% Metodología
\chapter{Metodología}
\input{capitulos/metodologia/main.tex}

% Resultados y discusión (incluyendo la valoración de impactos y de aspectos de responsabilidad legal, ética y profesional relacionados con el trabajo)
\chapter{Resultados y Discusión}
\input{capitulos/resultados_discusion/main.tex}

% Conclusiones
\chapter{Conclusiones}
\input{capitulos/conclusiones/main.tex}

% Planificación temporal y presupuesto
\chapter{Planificación Temporal y Presupuesto}
\input{capitulos/planificacion_presupuesto/main.tex}

% Bibliografía
\newpage
\addcontentsline{toc}{chapter}{Bibliografía}
\printbibliography

\end{document}


% Resultados y discusión (incluyendo la valoración de impactos y de aspectos de responsabilidad legal, ética y profesional relacionados con el trabajo)
\chapter{Resultados y Discusión}
\documentclass[a4paper,11pt,twoside]{report}
\usepackage[left=25mm,right=25mm,top=25mm,bottom=25mm,includehead,includefoot,headsep=15mm,footskip=15mm]{geometry}
\usepackage{graphicx}
\usepackage{fancyhdr}
\usepackage{titlesec}
\usepackage[spanish]{babel}
\usepackage[utf8]{inputenc}
\usepackage{amsmath}
\usepackage{setspace}
\usepackage{svg}
\usepackage{hyperref}
\usepackage[backend=biber,style=numeric]{biblatex}
\addbibresource{references.bib}
\hypersetup{
    colorlinks=true,
    linkcolor=blue,      % color of internal links (sections, etc.)
    urlcolor=blue,       % color of external links
    pdftitle={Optimización energética de sistema híbrido con bomba de calor, suelo radiante, fotovoltaica y almacenamiento para vivienda},    % title
    pdfauthor={Luis D. Aranda Sánchez},     % author
    pdfkeywords={palabra1, palabra2, código1, etc.} % list of keywords
}

% Font change to Arial
\usepackage{helvet}
\renewcommand{\familydefault}{\sfdefault}

% Chapter titles in uppercase and larger font
\titleformat{\chapter}[hang]{\large\bfseries}{\thechapter.}{1em}{\MakeUppercase}
\titleformat{\section}[hang]{\bfseries}{\thesection.}{1em}{}
\titleformat{\subsection}[hang]{\bfseries}{\thesubsection.}{1em}{}

% Fancyhdr setup
\setlength{\headheight}{14.30174pt} % Adjust to recommended value, slightly larger for safety
\fancyhf{} % Clear all headers and footers
\fancyhead[LE]{\nouppercase{\leftmark}}
\fancyhead[RO]{Optimización energética para vivienda}
\fancyfoot[LE]{\thepage}
\fancyfoot[RE]{Escuela Técnica Superior de Ingenieros Industriales (UPM)}
\fancyfoot[LO]{Luis D. Aranda Sánchez}
\fancyfoot[RO]{\thepage}
\renewcommand{\headrulewidth}{0.4pt}
\renewcommand{\footrulewidth}{0.4pt}

\fancypagestyle{myfancy}{
    \fancyhf{} % Clear all headers and footers
    \fancyhead[LE]{\nouppercase{\leftmark}}
    \fancyhead[RO]{Optimización energética para vivienda}
    \fancyfoot[LE]{\thepage}
    \fancyfoot[RE]{Escuela Técnica Superior de Ingenieros Industriales (UPM)}
    \fancyfoot[LO]{Luis D. Aranda Sánchez}
    \fancyfoot[RO]{\thepage}
    \renewcommand{\headrulewidth}{0.4pt}
    \renewcommand{\footrulewidth}{0.4pt}
}

\fancypagestyle{simple}{
    \fancyhf{} % Clear all headers and footers
    \renewcommand{\headrulewidth}{0pt}
    \renewcommand{\footrulewidth}{0pt}
}

% Line spacing
\setstretch{1.2}

% Document starts here
\begin{document}

% Portada
\begin{titlepage}
    \centering
    {\scshape\LARGE Universidad Politécnica de Madrid \par}
    \vspace{1cm}
    {\scshape\Large Escuela Técnica Superior de Ingenieros Industriales\par}
    \vspace{1.5cm}
    {\huge\bfseries Optimización energética de sistema híbrido con bomba de calor, suelo radiante, fotovoltaica y almacenamiento para vivienda \par}
    \vspace{1.5cm}
    {\Large\bfseries Trabajo de Fin de Máster\par}
    \vspace{0.5cm}
    {\large Máster Universitario en Ingeniería de la Energía \par}
    \vspace{2cm}
    {\Large Luis D. Aranda Sánchez\par}
    \vfill
    Director: Javier Rodríguez Martín
    \vfill
    {\large Septiembre 6, 2024\par}
\end{titlepage}

% Resumen (máximo de 5 páginas, incluyendo al final Palabras clave)
\clearpage
\pagestyle{simple}
% \newpage
\chapter*{Resumen}
\addcontentsline{toc}{chapter}{Resumen}
\input{capitulos/resumen/main.tex}

% Índice (paginado)
\clearpage
\pagestyle{simple}
% \newpage
\tableofcontents

% Introducción (donde se incluya los antecedentes y justificación)
\clearpage
\pagestyle{myfancy}
\newpage
\chapter{Introducción}
\input{capitulos/introduccion/main.tex}

% Objetivos
\chapter{Objetivos}
\input{capitulos/objetivos/main.tex}

% Metodología
\chapter{Metodología}
\input{capitulos/metodologia/main.tex}

% Resultados y discusión (incluyendo la valoración de impactos y de aspectos de responsabilidad legal, ética y profesional relacionados con el trabajo)
\chapter{Resultados y Discusión}
\input{capitulos/resultados_discusion/main.tex}

% Conclusiones
\chapter{Conclusiones}
\input{capitulos/conclusiones/main.tex}

% Planificación temporal y presupuesto
\chapter{Planificación Temporal y Presupuesto}
\input{capitulos/planificacion_presupuesto/main.tex}

% Bibliografía
\newpage
\addcontentsline{toc}{chapter}{Bibliografía}
\printbibliography

\end{document}


% Conclusiones
\chapter{Conclusiones}
\documentclass[a4paper,11pt,twoside]{report}
\usepackage[left=25mm,right=25mm,top=25mm,bottom=25mm,includehead,includefoot,headsep=15mm,footskip=15mm]{geometry}
\usepackage{graphicx}
\usepackage{fancyhdr}
\usepackage{titlesec}
\usepackage[spanish]{babel}
\usepackage[utf8]{inputenc}
\usepackage{amsmath}
\usepackage{setspace}
\usepackage{svg}
\usepackage{hyperref}
\usepackage[backend=biber,style=numeric]{biblatex}
\addbibresource{references.bib}
\hypersetup{
    colorlinks=true,
    linkcolor=blue,      % color of internal links (sections, etc.)
    urlcolor=blue,       % color of external links
    pdftitle={Optimización energética de sistema híbrido con bomba de calor, suelo radiante, fotovoltaica y almacenamiento para vivienda},    % title
    pdfauthor={Luis D. Aranda Sánchez},     % author
    pdfkeywords={palabra1, palabra2, código1, etc.} % list of keywords
}

% Font change to Arial
\usepackage{helvet}
\renewcommand{\familydefault}{\sfdefault}

% Chapter titles in uppercase and larger font
\titleformat{\chapter}[hang]{\large\bfseries}{\thechapter.}{1em}{\MakeUppercase}
\titleformat{\section}[hang]{\bfseries}{\thesection.}{1em}{}
\titleformat{\subsection}[hang]{\bfseries}{\thesubsection.}{1em}{}

% Fancyhdr setup
\setlength{\headheight}{14.30174pt} % Adjust to recommended value, slightly larger for safety
\fancyhf{} % Clear all headers and footers
\fancyhead[LE]{\nouppercase{\leftmark}}
\fancyhead[RO]{Optimización energética para vivienda}
\fancyfoot[LE]{\thepage}
\fancyfoot[RE]{Escuela Técnica Superior de Ingenieros Industriales (UPM)}
\fancyfoot[LO]{Luis D. Aranda Sánchez}
\fancyfoot[RO]{\thepage}
\renewcommand{\headrulewidth}{0.4pt}
\renewcommand{\footrulewidth}{0.4pt}

\fancypagestyle{myfancy}{
    \fancyhf{} % Clear all headers and footers
    \fancyhead[LE]{\nouppercase{\leftmark}}
    \fancyhead[RO]{Optimización energética para vivienda}
    \fancyfoot[LE]{\thepage}
    \fancyfoot[RE]{Escuela Técnica Superior de Ingenieros Industriales (UPM)}
    \fancyfoot[LO]{Luis D. Aranda Sánchez}
    \fancyfoot[RO]{\thepage}
    \renewcommand{\headrulewidth}{0.4pt}
    \renewcommand{\footrulewidth}{0.4pt}
}

\fancypagestyle{simple}{
    \fancyhf{} % Clear all headers and footers
    \renewcommand{\headrulewidth}{0pt}
    \renewcommand{\footrulewidth}{0pt}
}

% Line spacing
\setstretch{1.2}

% Document starts here
\begin{document}

% Portada
\begin{titlepage}
    \centering
    {\scshape\LARGE Universidad Politécnica de Madrid \par}
    \vspace{1cm}
    {\scshape\Large Escuela Técnica Superior de Ingenieros Industriales\par}
    \vspace{1.5cm}
    {\huge\bfseries Optimización energética de sistema híbrido con bomba de calor, suelo radiante, fotovoltaica y almacenamiento para vivienda \par}
    \vspace{1.5cm}
    {\Large\bfseries Trabajo de Fin de Máster\par}
    \vspace{0.5cm}
    {\large Máster Universitario en Ingeniería de la Energía \par}
    \vspace{2cm}
    {\Large Luis D. Aranda Sánchez\par}
    \vfill
    Director: Javier Rodríguez Martín
    \vfill
    {\large Septiembre 6, 2024\par}
\end{titlepage}

% Resumen (máximo de 5 páginas, incluyendo al final Palabras clave)
\clearpage
\pagestyle{simple}
% \newpage
\chapter*{Resumen}
\addcontentsline{toc}{chapter}{Resumen}
\input{capitulos/resumen/main.tex}

% Índice (paginado)
\clearpage
\pagestyle{simple}
% \newpage
\tableofcontents

% Introducción (donde se incluya los antecedentes y justificación)
\clearpage
\pagestyle{myfancy}
\newpage
\chapter{Introducción}
\input{capitulos/introduccion/main.tex}

% Objetivos
\chapter{Objetivos}
\input{capitulos/objetivos/main.tex}

% Metodología
\chapter{Metodología}
\input{capitulos/metodologia/main.tex}

% Resultados y discusión (incluyendo la valoración de impactos y de aspectos de responsabilidad legal, ética y profesional relacionados con el trabajo)
\chapter{Resultados y Discusión}
\input{capitulos/resultados_discusion/main.tex}

% Conclusiones
\chapter{Conclusiones}
\input{capitulos/conclusiones/main.tex}

% Planificación temporal y presupuesto
\chapter{Planificación Temporal y Presupuesto}
\input{capitulos/planificacion_presupuesto/main.tex}

% Bibliografía
\newpage
\addcontentsline{toc}{chapter}{Bibliografía}
\printbibliography

\end{document}


% Planificación temporal y presupuesto
\chapter{Planificación Temporal y Presupuesto}
\documentclass[a4paper,11pt,twoside]{report}
\usepackage[left=25mm,right=25mm,top=25mm,bottom=25mm,includehead,includefoot,headsep=15mm,footskip=15mm]{geometry}
\usepackage{graphicx}
\usepackage{fancyhdr}
\usepackage{titlesec}
\usepackage[spanish]{babel}
\usepackage[utf8]{inputenc}
\usepackage{amsmath}
\usepackage{setspace}
\usepackage{svg}
\usepackage{hyperref}
\usepackage[backend=biber,style=numeric]{biblatex}
\addbibresource{references.bib}
\hypersetup{
    colorlinks=true,
    linkcolor=blue,      % color of internal links (sections, etc.)
    urlcolor=blue,       % color of external links
    pdftitle={Optimización energética de sistema híbrido con bomba de calor, suelo radiante, fotovoltaica y almacenamiento para vivienda},    % title
    pdfauthor={Luis D. Aranda Sánchez},     % author
    pdfkeywords={palabra1, palabra2, código1, etc.} % list of keywords
}

% Font change to Arial
\usepackage{helvet}
\renewcommand{\familydefault}{\sfdefault}

% Chapter titles in uppercase and larger font
\titleformat{\chapter}[hang]{\large\bfseries}{\thechapter.}{1em}{\MakeUppercase}
\titleformat{\section}[hang]{\bfseries}{\thesection.}{1em}{}
\titleformat{\subsection}[hang]{\bfseries}{\thesubsection.}{1em}{}

% Fancyhdr setup
\setlength{\headheight}{14.30174pt} % Adjust to recommended value, slightly larger for safety
\fancyhf{} % Clear all headers and footers
\fancyhead[LE]{\nouppercase{\leftmark}}
\fancyhead[RO]{Optimización energética para vivienda}
\fancyfoot[LE]{\thepage}
\fancyfoot[RE]{Escuela Técnica Superior de Ingenieros Industriales (UPM)}
\fancyfoot[LO]{Luis D. Aranda Sánchez}
\fancyfoot[RO]{\thepage}
\renewcommand{\headrulewidth}{0.4pt}
\renewcommand{\footrulewidth}{0.4pt}

\fancypagestyle{myfancy}{
    \fancyhf{} % Clear all headers and footers
    \fancyhead[LE]{\nouppercase{\leftmark}}
    \fancyhead[RO]{Optimización energética para vivienda}
    \fancyfoot[LE]{\thepage}
    \fancyfoot[RE]{Escuela Técnica Superior de Ingenieros Industriales (UPM)}
    \fancyfoot[LO]{Luis D. Aranda Sánchez}
    \fancyfoot[RO]{\thepage}
    \renewcommand{\headrulewidth}{0.4pt}
    \renewcommand{\footrulewidth}{0.4pt}
}

\fancypagestyle{simple}{
    \fancyhf{} % Clear all headers and footers
    \renewcommand{\headrulewidth}{0pt}
    \renewcommand{\footrulewidth}{0pt}
}

% Line spacing
\setstretch{1.2}

% Document starts here
\begin{document}

% Portada
\begin{titlepage}
    \centering
    {\scshape\LARGE Universidad Politécnica de Madrid \par}
    \vspace{1cm}
    {\scshape\Large Escuela Técnica Superior de Ingenieros Industriales\par}
    \vspace{1.5cm}
    {\huge\bfseries Optimización energética de sistema híbrido con bomba de calor, suelo radiante, fotovoltaica y almacenamiento para vivienda \par}
    \vspace{1.5cm}
    {\Large\bfseries Trabajo de Fin de Máster\par}
    \vspace{0.5cm}
    {\large Máster Universitario en Ingeniería de la Energía \par}
    \vspace{2cm}
    {\Large Luis D. Aranda Sánchez\par}
    \vfill
    Director: Javier Rodríguez Martín
    \vfill
    {\large Septiembre 6, 2024\par}
\end{titlepage}

% Resumen (máximo de 5 páginas, incluyendo al final Palabras clave)
\clearpage
\pagestyle{simple}
% \newpage
\chapter*{Resumen}
\addcontentsline{toc}{chapter}{Resumen}
\input{capitulos/resumen/main.tex}

% Índice (paginado)
\clearpage
\pagestyle{simple}
% \newpage
\tableofcontents

% Introducción (donde se incluya los antecedentes y justificación)
\clearpage
\pagestyle{myfancy}
\newpage
\chapter{Introducción}
\input{capitulos/introduccion/main.tex}

% Objetivos
\chapter{Objetivos}
\input{capitulos/objetivos/main.tex}

% Metodología
\chapter{Metodología}
\input{capitulos/metodologia/main.tex}

% Resultados y discusión (incluyendo la valoración de impactos y de aspectos de responsabilidad legal, ética y profesional relacionados con el trabajo)
\chapter{Resultados y Discusión}
\input{capitulos/resultados_discusion/main.tex}

% Conclusiones
\chapter{Conclusiones}
\input{capitulos/conclusiones/main.tex}

% Planificación temporal y presupuesto
\chapter{Planificación Temporal y Presupuesto}
\input{capitulos/planificacion_presupuesto/main.tex}

% Bibliografía
\newpage
\addcontentsline{toc}{chapter}{Bibliografía}
\printbibliography

\end{document}


% Bibliografía
\newpage
\addcontentsline{toc}{chapter}{Bibliografía}
\printbibliography

\end{document}


% Bibliografía
\newpage
\addcontentsline{toc}{chapter}{Bibliografía}
\printbibliography

\end{document}


% Resultados y discusión (incluyendo la valoración de impactos y de aspectos de responsabilidad legal, ética y profesional relacionados con el trabajo)
\cleardoublepage
\chapter{Resultados y Discusión}
\documentclass[a4paper,11pt,twoside]{report}
\usepackage[left=25mm,right=25mm,top=25mm,bottom=25mm,includehead,includefoot,headsep=15mm,footskip=15mm]{geometry}
\usepackage{graphicx}
\usepackage{fancyhdr}
\usepackage{titlesec}
\usepackage[spanish]{babel}
\usepackage[utf8]{inputenc}
\usepackage{amsmath}
\usepackage{setspace}
\usepackage{svg}
\usepackage{hyperref}
\usepackage[backend=biber,style=numeric]{biblatex}
\addbibresource{references.bib}
\hypersetup{
    colorlinks=true,
    linkcolor=blue,      % color of internal links (sections, etc.)
    urlcolor=blue,       % color of external links
    pdftitle={Optimización energética de sistema híbrido con bomba de calor, suelo radiante, fotovoltaica y almacenamiento para vivienda},    % title
    pdfauthor={Luis D. Aranda Sánchez},     % author
    pdfkeywords={palabra1, palabra2, código1, etc.} % list of keywords
}

% Font change to Arial
\usepackage{helvet}
\renewcommand{\familydefault}{\sfdefault}

% Chapter titles in uppercase and larger font
\titleformat{\chapter}[hang]{\large\bfseries}{\thechapter.}{1em}{\MakeUppercase}
\titleformat{\section}[hang]{\bfseries}{\thesection.}{1em}{}
\titleformat{\subsection}[hang]{\bfseries}{\thesubsection.}{1em}{}

% Fancyhdr setup
\setlength{\headheight}{14.30174pt} % Adjust to recommended value, slightly larger for safety
\fancyhf{} % Clear all headers and footers
\fancyhead[LE]{\nouppercase{\leftmark}}
\fancyhead[RO]{Optimización energética para vivienda}
\fancyfoot[LE]{\thepage}
\fancyfoot[RE]{Escuela Técnica Superior de Ingenieros Industriales (UPM)}
\fancyfoot[LO]{Luis D. Aranda Sánchez}
\fancyfoot[RO]{\thepage}
\renewcommand{\headrulewidth}{0.4pt}
\renewcommand{\footrulewidth}{0.4pt}

\fancypagestyle{myfancy}{
    \fancyhf{} % Clear all headers and footers
    \fancyhead[LE]{\nouppercase{\leftmark}}
    \fancyhead[RO]{Optimización energética para vivienda}
    \fancyfoot[LE]{\thepage}
    \fancyfoot[RE]{Escuela Técnica Superior de Ingenieros Industriales (UPM)}
    \fancyfoot[LO]{Luis D. Aranda Sánchez}
    \fancyfoot[RO]{\thepage}
    \renewcommand{\headrulewidth}{0.4pt}
    \renewcommand{\footrulewidth}{0.4pt}
}

\fancypagestyle{simple}{
    \fancyhf{} % Clear all headers and footers
    \renewcommand{\headrulewidth}{0pt}
    \renewcommand{\footrulewidth}{0pt}
}

% Line spacing
\setstretch{1.2}

% Document starts here
\begin{document}

% Portada
\begin{titlepage}
    \centering
    {\scshape\LARGE Universidad Politécnica de Madrid \par}
    \vspace{1cm}
    {\scshape\Large Escuela Técnica Superior de Ingenieros Industriales\par}
    \vspace{1.5cm}
    {\huge\bfseries Optimización energética de sistema híbrido con bomba de calor, suelo radiante, fotovoltaica y almacenamiento para vivienda \par}
    \vspace{1.5cm}
    {\Large\bfseries Trabajo de Fin de Máster\par}
    \vspace{0.5cm}
    {\large Máster Universitario en Ingeniería de la Energía \par}
    \vspace{2cm}
    {\Large Luis D. Aranda Sánchez\par}
    \vfill
    Director: Javier Rodríguez Martín
    \vfill
    {\large Septiembre 6, 2024\par}
\end{titlepage}

% Resumen (máximo de 5 páginas, incluyendo al final Palabras clave)
\clearpage
\pagestyle{simple}
% \newpage
\chapter*{Resumen}
\addcontentsline{toc}{chapter}{Resumen}
\documentclass[a4paper,11pt,twoside]{report}
\usepackage[left=25mm,right=25mm,top=25mm,bottom=25mm,includehead,includefoot,headsep=15mm,footskip=15mm]{geometry}
\usepackage{graphicx}
\usepackage{fancyhdr}
\usepackage{titlesec}
\usepackage[spanish]{babel}
\usepackage[utf8]{inputenc}
\usepackage{amsmath}
\usepackage{setspace}
\usepackage{svg}
\usepackage{hyperref}
\usepackage[backend=biber,style=numeric]{biblatex}
\addbibresource{references.bib}
\hypersetup{
    colorlinks=true,
    linkcolor=blue,      % color of internal links (sections, etc.)
    urlcolor=blue,       % color of external links
    pdftitle={Optimización energética de sistema híbrido con bomba de calor, suelo radiante, fotovoltaica y almacenamiento para vivienda},    % title
    pdfauthor={Luis D. Aranda Sánchez},     % author
    pdfkeywords={palabra1, palabra2, código1, etc.} % list of keywords
}

% Font change to Arial
\usepackage{helvet}
\renewcommand{\familydefault}{\sfdefault}

% Chapter titles in uppercase and larger font
\titleformat{\chapter}[hang]{\large\bfseries}{\thechapter.}{1em}{\MakeUppercase}
\titleformat{\section}[hang]{\bfseries}{\thesection.}{1em}{}
\titleformat{\subsection}[hang]{\bfseries}{\thesubsection.}{1em}{}

% Fancyhdr setup
\setlength{\headheight}{14.30174pt} % Adjust to recommended value, slightly larger for safety
\fancyhf{} % Clear all headers and footers
\fancyhead[LE]{\nouppercase{\leftmark}}
\fancyhead[RO]{Optimización energética para vivienda}
\fancyfoot[LE]{\thepage}
\fancyfoot[RE]{Escuela Técnica Superior de Ingenieros Industriales (UPM)}
\fancyfoot[LO]{Luis D. Aranda Sánchez}
\fancyfoot[RO]{\thepage}
\renewcommand{\headrulewidth}{0.4pt}
\renewcommand{\footrulewidth}{0.4pt}

\fancypagestyle{myfancy}{
    \fancyhf{} % Clear all headers and footers
    \fancyhead[LE]{\nouppercase{\leftmark}}
    \fancyhead[RO]{Optimización energética para vivienda}
    \fancyfoot[LE]{\thepage}
    \fancyfoot[RE]{Escuela Técnica Superior de Ingenieros Industriales (UPM)}
    \fancyfoot[LO]{Luis D. Aranda Sánchez}
    \fancyfoot[RO]{\thepage}
    \renewcommand{\headrulewidth}{0.4pt}
    \renewcommand{\footrulewidth}{0.4pt}
}

\fancypagestyle{simple}{
    \fancyhf{} % Clear all headers and footers
    \renewcommand{\headrulewidth}{0pt}
    \renewcommand{\footrulewidth}{0pt}
}

% Line spacing
\setstretch{1.2}

% Document starts here
\begin{document}

% Portada
\begin{titlepage}
    \centering
    {\scshape\LARGE Universidad Politécnica de Madrid \par}
    \vspace{1cm}
    {\scshape\Large Escuela Técnica Superior de Ingenieros Industriales\par}
    \vspace{1.5cm}
    {\huge\bfseries Optimización energética de sistema híbrido con bomba de calor, suelo radiante, fotovoltaica y almacenamiento para vivienda \par}
    \vspace{1.5cm}
    {\Large\bfseries Trabajo de Fin de Máster\par}
    \vspace{0.5cm}
    {\large Máster Universitario en Ingeniería de la Energía \par}
    \vspace{2cm}
    {\Large Luis D. Aranda Sánchez\par}
    \vfill
    Director: Javier Rodríguez Martín
    \vfill
    {\large Septiembre 6, 2024\par}
\end{titlepage}

% Resumen (máximo de 5 páginas, incluyendo al final Palabras clave)
\clearpage
\pagestyle{simple}
% \newpage
\chapter*{Resumen}
\addcontentsline{toc}{chapter}{Resumen}
\documentclass[a4paper,11pt,twoside]{report}
\usepackage[left=25mm,right=25mm,top=25mm,bottom=25mm,includehead,includefoot,headsep=15mm,footskip=15mm]{geometry}
\usepackage{graphicx}
\usepackage{fancyhdr}
\usepackage{titlesec}
\usepackage[spanish]{babel}
\usepackage[utf8]{inputenc}
\usepackage{amsmath}
\usepackage{setspace}
\usepackage{svg}
\usepackage{hyperref}
\usepackage[backend=biber,style=numeric]{biblatex}
\addbibresource{references.bib}
\hypersetup{
    colorlinks=true,
    linkcolor=blue,      % color of internal links (sections, etc.)
    urlcolor=blue,       % color of external links
    pdftitle={Optimización energética de sistema híbrido con bomba de calor, suelo radiante, fotovoltaica y almacenamiento para vivienda},    % title
    pdfauthor={Luis D. Aranda Sánchez},     % author
    pdfkeywords={palabra1, palabra2, código1, etc.} % list of keywords
}

% Font change to Arial
\usepackage{helvet}
\renewcommand{\familydefault}{\sfdefault}

% Chapter titles in uppercase and larger font
\titleformat{\chapter}[hang]{\large\bfseries}{\thechapter.}{1em}{\MakeUppercase}
\titleformat{\section}[hang]{\bfseries}{\thesection.}{1em}{}
\titleformat{\subsection}[hang]{\bfseries}{\thesubsection.}{1em}{}

% Fancyhdr setup
\setlength{\headheight}{14.30174pt} % Adjust to recommended value, slightly larger for safety
\fancyhf{} % Clear all headers and footers
\fancyhead[LE]{\nouppercase{\leftmark}}
\fancyhead[RO]{Optimización energética para vivienda}
\fancyfoot[LE]{\thepage}
\fancyfoot[RE]{Escuela Técnica Superior de Ingenieros Industriales (UPM)}
\fancyfoot[LO]{Luis D. Aranda Sánchez}
\fancyfoot[RO]{\thepage}
\renewcommand{\headrulewidth}{0.4pt}
\renewcommand{\footrulewidth}{0.4pt}

\fancypagestyle{myfancy}{
    \fancyhf{} % Clear all headers and footers
    \fancyhead[LE]{\nouppercase{\leftmark}}
    \fancyhead[RO]{Optimización energética para vivienda}
    \fancyfoot[LE]{\thepage}
    \fancyfoot[RE]{Escuela Técnica Superior de Ingenieros Industriales (UPM)}
    \fancyfoot[LO]{Luis D. Aranda Sánchez}
    \fancyfoot[RO]{\thepage}
    \renewcommand{\headrulewidth}{0.4pt}
    \renewcommand{\footrulewidth}{0.4pt}
}

\fancypagestyle{simple}{
    \fancyhf{} % Clear all headers and footers
    \renewcommand{\headrulewidth}{0pt}
    \renewcommand{\footrulewidth}{0pt}
}

% Line spacing
\setstretch{1.2}

% Document starts here
\begin{document}

% Portada
\begin{titlepage}
    \centering
    {\scshape\LARGE Universidad Politécnica de Madrid \par}
    \vspace{1cm}
    {\scshape\Large Escuela Técnica Superior de Ingenieros Industriales\par}
    \vspace{1.5cm}
    {\huge\bfseries Optimización energética de sistema híbrido con bomba de calor, suelo radiante, fotovoltaica y almacenamiento para vivienda \par}
    \vspace{1.5cm}
    {\Large\bfseries Trabajo de Fin de Máster\par}
    \vspace{0.5cm}
    {\large Máster Universitario en Ingeniería de la Energía \par}
    \vspace{2cm}
    {\Large Luis D. Aranda Sánchez\par}
    \vfill
    Director: Javier Rodríguez Martín
    \vfill
    {\large Septiembre 6, 2024\par}
\end{titlepage}

% Resumen (máximo de 5 páginas, incluyendo al final Palabras clave)
\clearpage
\pagestyle{simple}
% \newpage
\chapter*{Resumen}
\addcontentsline{toc}{chapter}{Resumen}
\input{capitulos/resumen/main.tex}

% Índice (paginado)
\clearpage
\pagestyle{simple}
% \newpage
\tableofcontents

% Introducción (donde se incluya los antecedentes y justificación)
\clearpage
\pagestyle{myfancy}
\newpage
\chapter{Introducción}
\input{capitulos/introduccion/main.tex}

% Objetivos
\chapter{Objetivos}
\input{capitulos/objetivos/main.tex}

% Metodología
\chapter{Metodología}
\input{capitulos/metodologia/main.tex}

% Resultados y discusión (incluyendo la valoración de impactos y de aspectos de responsabilidad legal, ética y profesional relacionados con el trabajo)
\chapter{Resultados y Discusión}
\input{capitulos/resultados_discusion/main.tex}

% Conclusiones
\chapter{Conclusiones}
\input{capitulos/conclusiones/main.tex}

% Planificación temporal y presupuesto
\chapter{Planificación Temporal y Presupuesto}
\input{capitulos/planificacion_presupuesto/main.tex}

% Bibliografía
\newpage
\addcontentsline{toc}{chapter}{Bibliografía}
\printbibliography

\end{document}


% Índice (paginado)
\clearpage
\pagestyle{simple}
% \newpage
\tableofcontents

% Introducción (donde se incluya los antecedentes y justificación)
\clearpage
\pagestyle{myfancy}
\newpage
\chapter{Introducción}
\documentclass[a4paper,11pt,twoside]{report}
\usepackage[left=25mm,right=25mm,top=25mm,bottom=25mm,includehead,includefoot,headsep=15mm,footskip=15mm]{geometry}
\usepackage{graphicx}
\usepackage{fancyhdr}
\usepackage{titlesec}
\usepackage[spanish]{babel}
\usepackage[utf8]{inputenc}
\usepackage{amsmath}
\usepackage{setspace}
\usepackage{svg}
\usepackage{hyperref}
\usepackage[backend=biber,style=numeric]{biblatex}
\addbibresource{references.bib}
\hypersetup{
    colorlinks=true,
    linkcolor=blue,      % color of internal links (sections, etc.)
    urlcolor=blue,       % color of external links
    pdftitle={Optimización energética de sistema híbrido con bomba de calor, suelo radiante, fotovoltaica y almacenamiento para vivienda},    % title
    pdfauthor={Luis D. Aranda Sánchez},     % author
    pdfkeywords={palabra1, palabra2, código1, etc.} % list of keywords
}

% Font change to Arial
\usepackage{helvet}
\renewcommand{\familydefault}{\sfdefault}

% Chapter titles in uppercase and larger font
\titleformat{\chapter}[hang]{\large\bfseries}{\thechapter.}{1em}{\MakeUppercase}
\titleformat{\section}[hang]{\bfseries}{\thesection.}{1em}{}
\titleformat{\subsection}[hang]{\bfseries}{\thesubsection.}{1em}{}

% Fancyhdr setup
\setlength{\headheight}{14.30174pt} % Adjust to recommended value, slightly larger for safety
\fancyhf{} % Clear all headers and footers
\fancyhead[LE]{\nouppercase{\leftmark}}
\fancyhead[RO]{Optimización energética para vivienda}
\fancyfoot[LE]{\thepage}
\fancyfoot[RE]{Escuela Técnica Superior de Ingenieros Industriales (UPM)}
\fancyfoot[LO]{Luis D. Aranda Sánchez}
\fancyfoot[RO]{\thepage}
\renewcommand{\headrulewidth}{0.4pt}
\renewcommand{\footrulewidth}{0.4pt}

\fancypagestyle{myfancy}{
    \fancyhf{} % Clear all headers and footers
    \fancyhead[LE]{\nouppercase{\leftmark}}
    \fancyhead[RO]{Optimización energética para vivienda}
    \fancyfoot[LE]{\thepage}
    \fancyfoot[RE]{Escuela Técnica Superior de Ingenieros Industriales (UPM)}
    \fancyfoot[LO]{Luis D. Aranda Sánchez}
    \fancyfoot[RO]{\thepage}
    \renewcommand{\headrulewidth}{0.4pt}
    \renewcommand{\footrulewidth}{0.4pt}
}

\fancypagestyle{simple}{
    \fancyhf{} % Clear all headers and footers
    \renewcommand{\headrulewidth}{0pt}
    \renewcommand{\footrulewidth}{0pt}
}

% Line spacing
\setstretch{1.2}

% Document starts here
\begin{document}

% Portada
\begin{titlepage}
    \centering
    {\scshape\LARGE Universidad Politécnica de Madrid \par}
    \vspace{1cm}
    {\scshape\Large Escuela Técnica Superior de Ingenieros Industriales\par}
    \vspace{1.5cm}
    {\huge\bfseries Optimización energética de sistema híbrido con bomba de calor, suelo radiante, fotovoltaica y almacenamiento para vivienda \par}
    \vspace{1.5cm}
    {\Large\bfseries Trabajo de Fin de Máster\par}
    \vspace{0.5cm}
    {\large Máster Universitario en Ingeniería de la Energía \par}
    \vspace{2cm}
    {\Large Luis D. Aranda Sánchez\par}
    \vfill
    Director: Javier Rodríguez Martín
    \vfill
    {\large Septiembre 6, 2024\par}
\end{titlepage}

% Resumen (máximo de 5 páginas, incluyendo al final Palabras clave)
\clearpage
\pagestyle{simple}
% \newpage
\chapter*{Resumen}
\addcontentsline{toc}{chapter}{Resumen}
\input{capitulos/resumen/main.tex}

% Índice (paginado)
\clearpage
\pagestyle{simple}
% \newpage
\tableofcontents

% Introducción (donde se incluya los antecedentes y justificación)
\clearpage
\pagestyle{myfancy}
\newpage
\chapter{Introducción}
\input{capitulos/introduccion/main.tex}

% Objetivos
\chapter{Objetivos}
\input{capitulos/objetivos/main.tex}

% Metodología
\chapter{Metodología}
\input{capitulos/metodologia/main.tex}

% Resultados y discusión (incluyendo la valoración de impactos y de aspectos de responsabilidad legal, ética y profesional relacionados con el trabajo)
\chapter{Resultados y Discusión}
\input{capitulos/resultados_discusion/main.tex}

% Conclusiones
\chapter{Conclusiones}
\input{capitulos/conclusiones/main.tex}

% Planificación temporal y presupuesto
\chapter{Planificación Temporal y Presupuesto}
\input{capitulos/planificacion_presupuesto/main.tex}

% Bibliografía
\newpage
\addcontentsline{toc}{chapter}{Bibliografía}
\printbibliography

\end{document}


% Objetivos
\chapter{Objetivos}
\documentclass[a4paper,11pt,twoside]{report}
\usepackage[left=25mm,right=25mm,top=25mm,bottom=25mm,includehead,includefoot,headsep=15mm,footskip=15mm]{geometry}
\usepackage{graphicx}
\usepackage{fancyhdr}
\usepackage{titlesec}
\usepackage[spanish]{babel}
\usepackage[utf8]{inputenc}
\usepackage{amsmath}
\usepackage{setspace}
\usepackage{svg}
\usepackage{hyperref}
\usepackage[backend=biber,style=numeric]{biblatex}
\addbibresource{references.bib}
\hypersetup{
    colorlinks=true,
    linkcolor=blue,      % color of internal links (sections, etc.)
    urlcolor=blue,       % color of external links
    pdftitle={Optimización energética de sistema híbrido con bomba de calor, suelo radiante, fotovoltaica y almacenamiento para vivienda},    % title
    pdfauthor={Luis D. Aranda Sánchez},     % author
    pdfkeywords={palabra1, palabra2, código1, etc.} % list of keywords
}

% Font change to Arial
\usepackage{helvet}
\renewcommand{\familydefault}{\sfdefault}

% Chapter titles in uppercase and larger font
\titleformat{\chapter}[hang]{\large\bfseries}{\thechapter.}{1em}{\MakeUppercase}
\titleformat{\section}[hang]{\bfseries}{\thesection.}{1em}{}
\titleformat{\subsection}[hang]{\bfseries}{\thesubsection.}{1em}{}

% Fancyhdr setup
\setlength{\headheight}{14.30174pt} % Adjust to recommended value, slightly larger for safety
\fancyhf{} % Clear all headers and footers
\fancyhead[LE]{\nouppercase{\leftmark}}
\fancyhead[RO]{Optimización energética para vivienda}
\fancyfoot[LE]{\thepage}
\fancyfoot[RE]{Escuela Técnica Superior de Ingenieros Industriales (UPM)}
\fancyfoot[LO]{Luis D. Aranda Sánchez}
\fancyfoot[RO]{\thepage}
\renewcommand{\headrulewidth}{0.4pt}
\renewcommand{\footrulewidth}{0.4pt}

\fancypagestyle{myfancy}{
    \fancyhf{} % Clear all headers and footers
    \fancyhead[LE]{\nouppercase{\leftmark}}
    \fancyhead[RO]{Optimización energética para vivienda}
    \fancyfoot[LE]{\thepage}
    \fancyfoot[RE]{Escuela Técnica Superior de Ingenieros Industriales (UPM)}
    \fancyfoot[LO]{Luis D. Aranda Sánchez}
    \fancyfoot[RO]{\thepage}
    \renewcommand{\headrulewidth}{0.4pt}
    \renewcommand{\footrulewidth}{0.4pt}
}

\fancypagestyle{simple}{
    \fancyhf{} % Clear all headers and footers
    \renewcommand{\headrulewidth}{0pt}
    \renewcommand{\footrulewidth}{0pt}
}

% Line spacing
\setstretch{1.2}

% Document starts here
\begin{document}

% Portada
\begin{titlepage}
    \centering
    {\scshape\LARGE Universidad Politécnica de Madrid \par}
    \vspace{1cm}
    {\scshape\Large Escuela Técnica Superior de Ingenieros Industriales\par}
    \vspace{1.5cm}
    {\huge\bfseries Optimización energética de sistema híbrido con bomba de calor, suelo radiante, fotovoltaica y almacenamiento para vivienda \par}
    \vspace{1.5cm}
    {\Large\bfseries Trabajo de Fin de Máster\par}
    \vspace{0.5cm}
    {\large Máster Universitario en Ingeniería de la Energía \par}
    \vspace{2cm}
    {\Large Luis D. Aranda Sánchez\par}
    \vfill
    Director: Javier Rodríguez Martín
    \vfill
    {\large Septiembre 6, 2024\par}
\end{titlepage}

% Resumen (máximo de 5 páginas, incluyendo al final Palabras clave)
\clearpage
\pagestyle{simple}
% \newpage
\chapter*{Resumen}
\addcontentsline{toc}{chapter}{Resumen}
\input{capitulos/resumen/main.tex}

% Índice (paginado)
\clearpage
\pagestyle{simple}
% \newpage
\tableofcontents

% Introducción (donde se incluya los antecedentes y justificación)
\clearpage
\pagestyle{myfancy}
\newpage
\chapter{Introducción}
\input{capitulos/introduccion/main.tex}

% Objetivos
\chapter{Objetivos}
\input{capitulos/objetivos/main.tex}

% Metodología
\chapter{Metodología}
\input{capitulos/metodologia/main.tex}

% Resultados y discusión (incluyendo la valoración de impactos y de aspectos de responsabilidad legal, ética y profesional relacionados con el trabajo)
\chapter{Resultados y Discusión}
\input{capitulos/resultados_discusion/main.tex}

% Conclusiones
\chapter{Conclusiones}
\input{capitulos/conclusiones/main.tex}

% Planificación temporal y presupuesto
\chapter{Planificación Temporal y Presupuesto}
\input{capitulos/planificacion_presupuesto/main.tex}

% Bibliografía
\newpage
\addcontentsline{toc}{chapter}{Bibliografía}
\printbibliography

\end{document}


% Metodología
\chapter{Metodología}
\documentclass[a4paper,11pt,twoside]{report}
\usepackage[left=25mm,right=25mm,top=25mm,bottom=25mm,includehead,includefoot,headsep=15mm,footskip=15mm]{geometry}
\usepackage{graphicx}
\usepackage{fancyhdr}
\usepackage{titlesec}
\usepackage[spanish]{babel}
\usepackage[utf8]{inputenc}
\usepackage{amsmath}
\usepackage{setspace}
\usepackage{svg}
\usepackage{hyperref}
\usepackage[backend=biber,style=numeric]{biblatex}
\addbibresource{references.bib}
\hypersetup{
    colorlinks=true,
    linkcolor=blue,      % color of internal links (sections, etc.)
    urlcolor=blue,       % color of external links
    pdftitle={Optimización energética de sistema híbrido con bomba de calor, suelo radiante, fotovoltaica y almacenamiento para vivienda},    % title
    pdfauthor={Luis D. Aranda Sánchez},     % author
    pdfkeywords={palabra1, palabra2, código1, etc.} % list of keywords
}

% Font change to Arial
\usepackage{helvet}
\renewcommand{\familydefault}{\sfdefault}

% Chapter titles in uppercase and larger font
\titleformat{\chapter}[hang]{\large\bfseries}{\thechapter.}{1em}{\MakeUppercase}
\titleformat{\section}[hang]{\bfseries}{\thesection.}{1em}{}
\titleformat{\subsection}[hang]{\bfseries}{\thesubsection.}{1em}{}

% Fancyhdr setup
\setlength{\headheight}{14.30174pt} % Adjust to recommended value, slightly larger for safety
\fancyhf{} % Clear all headers and footers
\fancyhead[LE]{\nouppercase{\leftmark}}
\fancyhead[RO]{Optimización energética para vivienda}
\fancyfoot[LE]{\thepage}
\fancyfoot[RE]{Escuela Técnica Superior de Ingenieros Industriales (UPM)}
\fancyfoot[LO]{Luis D. Aranda Sánchez}
\fancyfoot[RO]{\thepage}
\renewcommand{\headrulewidth}{0.4pt}
\renewcommand{\footrulewidth}{0.4pt}

\fancypagestyle{myfancy}{
    \fancyhf{} % Clear all headers and footers
    \fancyhead[LE]{\nouppercase{\leftmark}}
    \fancyhead[RO]{Optimización energética para vivienda}
    \fancyfoot[LE]{\thepage}
    \fancyfoot[RE]{Escuela Técnica Superior de Ingenieros Industriales (UPM)}
    \fancyfoot[LO]{Luis D. Aranda Sánchez}
    \fancyfoot[RO]{\thepage}
    \renewcommand{\headrulewidth}{0.4pt}
    \renewcommand{\footrulewidth}{0.4pt}
}

\fancypagestyle{simple}{
    \fancyhf{} % Clear all headers and footers
    \renewcommand{\headrulewidth}{0pt}
    \renewcommand{\footrulewidth}{0pt}
}

% Line spacing
\setstretch{1.2}

% Document starts here
\begin{document}

% Portada
\begin{titlepage}
    \centering
    {\scshape\LARGE Universidad Politécnica de Madrid \par}
    \vspace{1cm}
    {\scshape\Large Escuela Técnica Superior de Ingenieros Industriales\par}
    \vspace{1.5cm}
    {\huge\bfseries Optimización energética de sistema híbrido con bomba de calor, suelo radiante, fotovoltaica y almacenamiento para vivienda \par}
    \vspace{1.5cm}
    {\Large\bfseries Trabajo de Fin de Máster\par}
    \vspace{0.5cm}
    {\large Máster Universitario en Ingeniería de la Energía \par}
    \vspace{2cm}
    {\Large Luis D. Aranda Sánchez\par}
    \vfill
    Director: Javier Rodríguez Martín
    \vfill
    {\large Septiembre 6, 2024\par}
\end{titlepage}

% Resumen (máximo de 5 páginas, incluyendo al final Palabras clave)
\clearpage
\pagestyle{simple}
% \newpage
\chapter*{Resumen}
\addcontentsline{toc}{chapter}{Resumen}
\input{capitulos/resumen/main.tex}

% Índice (paginado)
\clearpage
\pagestyle{simple}
% \newpage
\tableofcontents

% Introducción (donde se incluya los antecedentes y justificación)
\clearpage
\pagestyle{myfancy}
\newpage
\chapter{Introducción}
\input{capitulos/introduccion/main.tex}

% Objetivos
\chapter{Objetivos}
\input{capitulos/objetivos/main.tex}

% Metodología
\chapter{Metodología}
\input{capitulos/metodologia/main.tex}

% Resultados y discusión (incluyendo la valoración de impactos y de aspectos de responsabilidad legal, ética y profesional relacionados con el trabajo)
\chapter{Resultados y Discusión}
\input{capitulos/resultados_discusion/main.tex}

% Conclusiones
\chapter{Conclusiones}
\input{capitulos/conclusiones/main.tex}

% Planificación temporal y presupuesto
\chapter{Planificación Temporal y Presupuesto}
\input{capitulos/planificacion_presupuesto/main.tex}

% Bibliografía
\newpage
\addcontentsline{toc}{chapter}{Bibliografía}
\printbibliography

\end{document}


% Resultados y discusión (incluyendo la valoración de impactos y de aspectos de responsabilidad legal, ética y profesional relacionados con el trabajo)
\chapter{Resultados y Discusión}
\documentclass[a4paper,11pt,twoside]{report}
\usepackage[left=25mm,right=25mm,top=25mm,bottom=25mm,includehead,includefoot,headsep=15mm,footskip=15mm]{geometry}
\usepackage{graphicx}
\usepackage{fancyhdr}
\usepackage{titlesec}
\usepackage[spanish]{babel}
\usepackage[utf8]{inputenc}
\usepackage{amsmath}
\usepackage{setspace}
\usepackage{svg}
\usepackage{hyperref}
\usepackage[backend=biber,style=numeric]{biblatex}
\addbibresource{references.bib}
\hypersetup{
    colorlinks=true,
    linkcolor=blue,      % color of internal links (sections, etc.)
    urlcolor=blue,       % color of external links
    pdftitle={Optimización energética de sistema híbrido con bomba de calor, suelo radiante, fotovoltaica y almacenamiento para vivienda},    % title
    pdfauthor={Luis D. Aranda Sánchez},     % author
    pdfkeywords={palabra1, palabra2, código1, etc.} % list of keywords
}

% Font change to Arial
\usepackage{helvet}
\renewcommand{\familydefault}{\sfdefault}

% Chapter titles in uppercase and larger font
\titleformat{\chapter}[hang]{\large\bfseries}{\thechapter.}{1em}{\MakeUppercase}
\titleformat{\section}[hang]{\bfseries}{\thesection.}{1em}{}
\titleformat{\subsection}[hang]{\bfseries}{\thesubsection.}{1em}{}

% Fancyhdr setup
\setlength{\headheight}{14.30174pt} % Adjust to recommended value, slightly larger for safety
\fancyhf{} % Clear all headers and footers
\fancyhead[LE]{\nouppercase{\leftmark}}
\fancyhead[RO]{Optimización energética para vivienda}
\fancyfoot[LE]{\thepage}
\fancyfoot[RE]{Escuela Técnica Superior de Ingenieros Industriales (UPM)}
\fancyfoot[LO]{Luis D. Aranda Sánchez}
\fancyfoot[RO]{\thepage}
\renewcommand{\headrulewidth}{0.4pt}
\renewcommand{\footrulewidth}{0.4pt}

\fancypagestyle{myfancy}{
    \fancyhf{} % Clear all headers and footers
    \fancyhead[LE]{\nouppercase{\leftmark}}
    \fancyhead[RO]{Optimización energética para vivienda}
    \fancyfoot[LE]{\thepage}
    \fancyfoot[RE]{Escuela Técnica Superior de Ingenieros Industriales (UPM)}
    \fancyfoot[LO]{Luis D. Aranda Sánchez}
    \fancyfoot[RO]{\thepage}
    \renewcommand{\headrulewidth}{0.4pt}
    \renewcommand{\footrulewidth}{0.4pt}
}

\fancypagestyle{simple}{
    \fancyhf{} % Clear all headers and footers
    \renewcommand{\headrulewidth}{0pt}
    \renewcommand{\footrulewidth}{0pt}
}

% Line spacing
\setstretch{1.2}

% Document starts here
\begin{document}

% Portada
\begin{titlepage}
    \centering
    {\scshape\LARGE Universidad Politécnica de Madrid \par}
    \vspace{1cm}
    {\scshape\Large Escuela Técnica Superior de Ingenieros Industriales\par}
    \vspace{1.5cm}
    {\huge\bfseries Optimización energética de sistema híbrido con bomba de calor, suelo radiante, fotovoltaica y almacenamiento para vivienda \par}
    \vspace{1.5cm}
    {\Large\bfseries Trabajo de Fin de Máster\par}
    \vspace{0.5cm}
    {\large Máster Universitario en Ingeniería de la Energía \par}
    \vspace{2cm}
    {\Large Luis D. Aranda Sánchez\par}
    \vfill
    Director: Javier Rodríguez Martín
    \vfill
    {\large Septiembre 6, 2024\par}
\end{titlepage}

% Resumen (máximo de 5 páginas, incluyendo al final Palabras clave)
\clearpage
\pagestyle{simple}
% \newpage
\chapter*{Resumen}
\addcontentsline{toc}{chapter}{Resumen}
\input{capitulos/resumen/main.tex}

% Índice (paginado)
\clearpage
\pagestyle{simple}
% \newpage
\tableofcontents

% Introducción (donde se incluya los antecedentes y justificación)
\clearpage
\pagestyle{myfancy}
\newpage
\chapter{Introducción}
\input{capitulos/introduccion/main.tex}

% Objetivos
\chapter{Objetivos}
\input{capitulos/objetivos/main.tex}

% Metodología
\chapter{Metodología}
\input{capitulos/metodologia/main.tex}

% Resultados y discusión (incluyendo la valoración de impactos y de aspectos de responsabilidad legal, ética y profesional relacionados con el trabajo)
\chapter{Resultados y Discusión}
\input{capitulos/resultados_discusion/main.tex}

% Conclusiones
\chapter{Conclusiones}
\input{capitulos/conclusiones/main.tex}

% Planificación temporal y presupuesto
\chapter{Planificación Temporal y Presupuesto}
\input{capitulos/planificacion_presupuesto/main.tex}

% Bibliografía
\newpage
\addcontentsline{toc}{chapter}{Bibliografía}
\printbibliography

\end{document}


% Conclusiones
\chapter{Conclusiones}
\documentclass[a4paper,11pt,twoside]{report}
\usepackage[left=25mm,right=25mm,top=25mm,bottom=25mm,includehead,includefoot,headsep=15mm,footskip=15mm]{geometry}
\usepackage{graphicx}
\usepackage{fancyhdr}
\usepackage{titlesec}
\usepackage[spanish]{babel}
\usepackage[utf8]{inputenc}
\usepackage{amsmath}
\usepackage{setspace}
\usepackage{svg}
\usepackage{hyperref}
\usepackage[backend=biber,style=numeric]{biblatex}
\addbibresource{references.bib}
\hypersetup{
    colorlinks=true,
    linkcolor=blue,      % color of internal links (sections, etc.)
    urlcolor=blue,       % color of external links
    pdftitle={Optimización energética de sistema híbrido con bomba de calor, suelo radiante, fotovoltaica y almacenamiento para vivienda},    % title
    pdfauthor={Luis D. Aranda Sánchez},     % author
    pdfkeywords={palabra1, palabra2, código1, etc.} % list of keywords
}

% Font change to Arial
\usepackage{helvet}
\renewcommand{\familydefault}{\sfdefault}

% Chapter titles in uppercase and larger font
\titleformat{\chapter}[hang]{\large\bfseries}{\thechapter.}{1em}{\MakeUppercase}
\titleformat{\section}[hang]{\bfseries}{\thesection.}{1em}{}
\titleformat{\subsection}[hang]{\bfseries}{\thesubsection.}{1em}{}

% Fancyhdr setup
\setlength{\headheight}{14.30174pt} % Adjust to recommended value, slightly larger for safety
\fancyhf{} % Clear all headers and footers
\fancyhead[LE]{\nouppercase{\leftmark}}
\fancyhead[RO]{Optimización energética para vivienda}
\fancyfoot[LE]{\thepage}
\fancyfoot[RE]{Escuela Técnica Superior de Ingenieros Industriales (UPM)}
\fancyfoot[LO]{Luis D. Aranda Sánchez}
\fancyfoot[RO]{\thepage}
\renewcommand{\headrulewidth}{0.4pt}
\renewcommand{\footrulewidth}{0.4pt}

\fancypagestyle{myfancy}{
    \fancyhf{} % Clear all headers and footers
    \fancyhead[LE]{\nouppercase{\leftmark}}
    \fancyhead[RO]{Optimización energética para vivienda}
    \fancyfoot[LE]{\thepage}
    \fancyfoot[RE]{Escuela Técnica Superior de Ingenieros Industriales (UPM)}
    \fancyfoot[LO]{Luis D. Aranda Sánchez}
    \fancyfoot[RO]{\thepage}
    \renewcommand{\headrulewidth}{0.4pt}
    \renewcommand{\footrulewidth}{0.4pt}
}

\fancypagestyle{simple}{
    \fancyhf{} % Clear all headers and footers
    \renewcommand{\headrulewidth}{0pt}
    \renewcommand{\footrulewidth}{0pt}
}

% Line spacing
\setstretch{1.2}

% Document starts here
\begin{document}

% Portada
\begin{titlepage}
    \centering
    {\scshape\LARGE Universidad Politécnica de Madrid \par}
    \vspace{1cm}
    {\scshape\Large Escuela Técnica Superior de Ingenieros Industriales\par}
    \vspace{1.5cm}
    {\huge\bfseries Optimización energética de sistema híbrido con bomba de calor, suelo radiante, fotovoltaica y almacenamiento para vivienda \par}
    \vspace{1.5cm}
    {\Large\bfseries Trabajo de Fin de Máster\par}
    \vspace{0.5cm}
    {\large Máster Universitario en Ingeniería de la Energía \par}
    \vspace{2cm}
    {\Large Luis D. Aranda Sánchez\par}
    \vfill
    Director: Javier Rodríguez Martín
    \vfill
    {\large Septiembre 6, 2024\par}
\end{titlepage}

% Resumen (máximo de 5 páginas, incluyendo al final Palabras clave)
\clearpage
\pagestyle{simple}
% \newpage
\chapter*{Resumen}
\addcontentsline{toc}{chapter}{Resumen}
\input{capitulos/resumen/main.tex}

% Índice (paginado)
\clearpage
\pagestyle{simple}
% \newpage
\tableofcontents

% Introducción (donde se incluya los antecedentes y justificación)
\clearpage
\pagestyle{myfancy}
\newpage
\chapter{Introducción}
\input{capitulos/introduccion/main.tex}

% Objetivos
\chapter{Objetivos}
\input{capitulos/objetivos/main.tex}

% Metodología
\chapter{Metodología}
\input{capitulos/metodologia/main.tex}

% Resultados y discusión (incluyendo la valoración de impactos y de aspectos de responsabilidad legal, ética y profesional relacionados con el trabajo)
\chapter{Resultados y Discusión}
\input{capitulos/resultados_discusion/main.tex}

% Conclusiones
\chapter{Conclusiones}
\input{capitulos/conclusiones/main.tex}

% Planificación temporal y presupuesto
\chapter{Planificación Temporal y Presupuesto}
\input{capitulos/planificacion_presupuesto/main.tex}

% Bibliografía
\newpage
\addcontentsline{toc}{chapter}{Bibliografía}
\printbibliography

\end{document}


% Planificación temporal y presupuesto
\chapter{Planificación Temporal y Presupuesto}
\documentclass[a4paper,11pt,twoside]{report}
\usepackage[left=25mm,right=25mm,top=25mm,bottom=25mm,includehead,includefoot,headsep=15mm,footskip=15mm]{geometry}
\usepackage{graphicx}
\usepackage{fancyhdr}
\usepackage{titlesec}
\usepackage[spanish]{babel}
\usepackage[utf8]{inputenc}
\usepackage{amsmath}
\usepackage{setspace}
\usepackage{svg}
\usepackage{hyperref}
\usepackage[backend=biber,style=numeric]{biblatex}
\addbibresource{references.bib}
\hypersetup{
    colorlinks=true,
    linkcolor=blue,      % color of internal links (sections, etc.)
    urlcolor=blue,       % color of external links
    pdftitle={Optimización energética de sistema híbrido con bomba de calor, suelo radiante, fotovoltaica y almacenamiento para vivienda},    % title
    pdfauthor={Luis D. Aranda Sánchez},     % author
    pdfkeywords={palabra1, palabra2, código1, etc.} % list of keywords
}

% Font change to Arial
\usepackage{helvet}
\renewcommand{\familydefault}{\sfdefault}

% Chapter titles in uppercase and larger font
\titleformat{\chapter}[hang]{\large\bfseries}{\thechapter.}{1em}{\MakeUppercase}
\titleformat{\section}[hang]{\bfseries}{\thesection.}{1em}{}
\titleformat{\subsection}[hang]{\bfseries}{\thesubsection.}{1em}{}

% Fancyhdr setup
\setlength{\headheight}{14.30174pt} % Adjust to recommended value, slightly larger for safety
\fancyhf{} % Clear all headers and footers
\fancyhead[LE]{\nouppercase{\leftmark}}
\fancyhead[RO]{Optimización energética para vivienda}
\fancyfoot[LE]{\thepage}
\fancyfoot[RE]{Escuela Técnica Superior de Ingenieros Industriales (UPM)}
\fancyfoot[LO]{Luis D. Aranda Sánchez}
\fancyfoot[RO]{\thepage}
\renewcommand{\headrulewidth}{0.4pt}
\renewcommand{\footrulewidth}{0.4pt}

\fancypagestyle{myfancy}{
    \fancyhf{} % Clear all headers and footers
    \fancyhead[LE]{\nouppercase{\leftmark}}
    \fancyhead[RO]{Optimización energética para vivienda}
    \fancyfoot[LE]{\thepage}
    \fancyfoot[RE]{Escuela Técnica Superior de Ingenieros Industriales (UPM)}
    \fancyfoot[LO]{Luis D. Aranda Sánchez}
    \fancyfoot[RO]{\thepage}
    \renewcommand{\headrulewidth}{0.4pt}
    \renewcommand{\footrulewidth}{0.4pt}
}

\fancypagestyle{simple}{
    \fancyhf{} % Clear all headers and footers
    \renewcommand{\headrulewidth}{0pt}
    \renewcommand{\footrulewidth}{0pt}
}

% Line spacing
\setstretch{1.2}

% Document starts here
\begin{document}

% Portada
\begin{titlepage}
    \centering
    {\scshape\LARGE Universidad Politécnica de Madrid \par}
    \vspace{1cm}
    {\scshape\Large Escuela Técnica Superior de Ingenieros Industriales\par}
    \vspace{1.5cm}
    {\huge\bfseries Optimización energética de sistema híbrido con bomba de calor, suelo radiante, fotovoltaica y almacenamiento para vivienda \par}
    \vspace{1.5cm}
    {\Large\bfseries Trabajo de Fin de Máster\par}
    \vspace{0.5cm}
    {\large Máster Universitario en Ingeniería de la Energía \par}
    \vspace{2cm}
    {\Large Luis D. Aranda Sánchez\par}
    \vfill
    Director: Javier Rodríguez Martín
    \vfill
    {\large Septiembre 6, 2024\par}
\end{titlepage}

% Resumen (máximo de 5 páginas, incluyendo al final Palabras clave)
\clearpage
\pagestyle{simple}
% \newpage
\chapter*{Resumen}
\addcontentsline{toc}{chapter}{Resumen}
\input{capitulos/resumen/main.tex}

% Índice (paginado)
\clearpage
\pagestyle{simple}
% \newpage
\tableofcontents

% Introducción (donde se incluya los antecedentes y justificación)
\clearpage
\pagestyle{myfancy}
\newpage
\chapter{Introducción}
\input{capitulos/introduccion/main.tex}

% Objetivos
\chapter{Objetivos}
\input{capitulos/objetivos/main.tex}

% Metodología
\chapter{Metodología}
\input{capitulos/metodologia/main.tex}

% Resultados y discusión (incluyendo la valoración de impactos y de aspectos de responsabilidad legal, ética y profesional relacionados con el trabajo)
\chapter{Resultados y Discusión}
\input{capitulos/resultados_discusion/main.tex}

% Conclusiones
\chapter{Conclusiones}
\input{capitulos/conclusiones/main.tex}

% Planificación temporal y presupuesto
\chapter{Planificación Temporal y Presupuesto}
\input{capitulos/planificacion_presupuesto/main.tex}

% Bibliografía
\newpage
\addcontentsline{toc}{chapter}{Bibliografía}
\printbibliography

\end{document}


% Bibliografía
\newpage
\addcontentsline{toc}{chapter}{Bibliografía}
\printbibliography

\end{document}


% Índice (paginado)
\clearpage
\pagestyle{simple}
% \newpage
\tableofcontents

% Introducción (donde se incluya los antecedentes y justificación)
\clearpage
\pagestyle{myfancy}
\newpage
\chapter{Introducción}
\documentclass[a4paper,11pt,twoside]{report}
\usepackage[left=25mm,right=25mm,top=25mm,bottom=25mm,includehead,includefoot,headsep=15mm,footskip=15mm]{geometry}
\usepackage{graphicx}
\usepackage{fancyhdr}
\usepackage{titlesec}
\usepackage[spanish]{babel}
\usepackage[utf8]{inputenc}
\usepackage{amsmath}
\usepackage{setspace}
\usepackage{svg}
\usepackage{hyperref}
\usepackage[backend=biber,style=numeric]{biblatex}
\addbibresource{references.bib}
\hypersetup{
    colorlinks=true,
    linkcolor=blue,      % color of internal links (sections, etc.)
    urlcolor=blue,       % color of external links
    pdftitle={Optimización energética de sistema híbrido con bomba de calor, suelo radiante, fotovoltaica y almacenamiento para vivienda},    % title
    pdfauthor={Luis D. Aranda Sánchez},     % author
    pdfkeywords={palabra1, palabra2, código1, etc.} % list of keywords
}

% Font change to Arial
\usepackage{helvet}
\renewcommand{\familydefault}{\sfdefault}

% Chapter titles in uppercase and larger font
\titleformat{\chapter}[hang]{\large\bfseries}{\thechapter.}{1em}{\MakeUppercase}
\titleformat{\section}[hang]{\bfseries}{\thesection.}{1em}{}
\titleformat{\subsection}[hang]{\bfseries}{\thesubsection.}{1em}{}

% Fancyhdr setup
\setlength{\headheight}{14.30174pt} % Adjust to recommended value, slightly larger for safety
\fancyhf{} % Clear all headers and footers
\fancyhead[LE]{\nouppercase{\leftmark}}
\fancyhead[RO]{Optimización energética para vivienda}
\fancyfoot[LE]{\thepage}
\fancyfoot[RE]{Escuela Técnica Superior de Ingenieros Industriales (UPM)}
\fancyfoot[LO]{Luis D. Aranda Sánchez}
\fancyfoot[RO]{\thepage}
\renewcommand{\headrulewidth}{0.4pt}
\renewcommand{\footrulewidth}{0.4pt}

\fancypagestyle{myfancy}{
    \fancyhf{} % Clear all headers and footers
    \fancyhead[LE]{\nouppercase{\leftmark}}
    \fancyhead[RO]{Optimización energética para vivienda}
    \fancyfoot[LE]{\thepage}
    \fancyfoot[RE]{Escuela Técnica Superior de Ingenieros Industriales (UPM)}
    \fancyfoot[LO]{Luis D. Aranda Sánchez}
    \fancyfoot[RO]{\thepage}
    \renewcommand{\headrulewidth}{0.4pt}
    \renewcommand{\footrulewidth}{0.4pt}
}

\fancypagestyle{simple}{
    \fancyhf{} % Clear all headers and footers
    \renewcommand{\headrulewidth}{0pt}
    \renewcommand{\footrulewidth}{0pt}
}

% Line spacing
\setstretch{1.2}

% Document starts here
\begin{document}

% Portada
\begin{titlepage}
    \centering
    {\scshape\LARGE Universidad Politécnica de Madrid \par}
    \vspace{1cm}
    {\scshape\Large Escuela Técnica Superior de Ingenieros Industriales\par}
    \vspace{1.5cm}
    {\huge\bfseries Optimización energética de sistema híbrido con bomba de calor, suelo radiante, fotovoltaica y almacenamiento para vivienda \par}
    \vspace{1.5cm}
    {\Large\bfseries Trabajo de Fin de Máster\par}
    \vspace{0.5cm}
    {\large Máster Universitario en Ingeniería de la Energía \par}
    \vspace{2cm}
    {\Large Luis D. Aranda Sánchez\par}
    \vfill
    Director: Javier Rodríguez Martín
    \vfill
    {\large Septiembre 6, 2024\par}
\end{titlepage}

% Resumen (máximo de 5 páginas, incluyendo al final Palabras clave)
\clearpage
\pagestyle{simple}
% \newpage
\chapter*{Resumen}
\addcontentsline{toc}{chapter}{Resumen}
\documentclass[a4paper,11pt,twoside]{report}
\usepackage[left=25mm,right=25mm,top=25mm,bottom=25mm,includehead,includefoot,headsep=15mm,footskip=15mm]{geometry}
\usepackage{graphicx}
\usepackage{fancyhdr}
\usepackage{titlesec}
\usepackage[spanish]{babel}
\usepackage[utf8]{inputenc}
\usepackage{amsmath}
\usepackage{setspace}
\usepackage{svg}
\usepackage{hyperref}
\usepackage[backend=biber,style=numeric]{biblatex}
\addbibresource{references.bib}
\hypersetup{
    colorlinks=true,
    linkcolor=blue,      % color of internal links (sections, etc.)
    urlcolor=blue,       % color of external links
    pdftitle={Optimización energética de sistema híbrido con bomba de calor, suelo radiante, fotovoltaica y almacenamiento para vivienda},    % title
    pdfauthor={Luis D. Aranda Sánchez},     % author
    pdfkeywords={palabra1, palabra2, código1, etc.} % list of keywords
}

% Font change to Arial
\usepackage{helvet}
\renewcommand{\familydefault}{\sfdefault}

% Chapter titles in uppercase and larger font
\titleformat{\chapter}[hang]{\large\bfseries}{\thechapter.}{1em}{\MakeUppercase}
\titleformat{\section}[hang]{\bfseries}{\thesection.}{1em}{}
\titleformat{\subsection}[hang]{\bfseries}{\thesubsection.}{1em}{}

% Fancyhdr setup
\setlength{\headheight}{14.30174pt} % Adjust to recommended value, slightly larger for safety
\fancyhf{} % Clear all headers and footers
\fancyhead[LE]{\nouppercase{\leftmark}}
\fancyhead[RO]{Optimización energética para vivienda}
\fancyfoot[LE]{\thepage}
\fancyfoot[RE]{Escuela Técnica Superior de Ingenieros Industriales (UPM)}
\fancyfoot[LO]{Luis D. Aranda Sánchez}
\fancyfoot[RO]{\thepage}
\renewcommand{\headrulewidth}{0.4pt}
\renewcommand{\footrulewidth}{0.4pt}

\fancypagestyle{myfancy}{
    \fancyhf{} % Clear all headers and footers
    \fancyhead[LE]{\nouppercase{\leftmark}}
    \fancyhead[RO]{Optimización energética para vivienda}
    \fancyfoot[LE]{\thepage}
    \fancyfoot[RE]{Escuela Técnica Superior de Ingenieros Industriales (UPM)}
    \fancyfoot[LO]{Luis D. Aranda Sánchez}
    \fancyfoot[RO]{\thepage}
    \renewcommand{\headrulewidth}{0.4pt}
    \renewcommand{\footrulewidth}{0.4pt}
}

\fancypagestyle{simple}{
    \fancyhf{} % Clear all headers and footers
    \renewcommand{\headrulewidth}{0pt}
    \renewcommand{\footrulewidth}{0pt}
}

% Line spacing
\setstretch{1.2}

% Document starts here
\begin{document}

% Portada
\begin{titlepage}
    \centering
    {\scshape\LARGE Universidad Politécnica de Madrid \par}
    \vspace{1cm}
    {\scshape\Large Escuela Técnica Superior de Ingenieros Industriales\par}
    \vspace{1.5cm}
    {\huge\bfseries Optimización energética de sistema híbrido con bomba de calor, suelo radiante, fotovoltaica y almacenamiento para vivienda \par}
    \vspace{1.5cm}
    {\Large\bfseries Trabajo de Fin de Máster\par}
    \vspace{0.5cm}
    {\large Máster Universitario en Ingeniería de la Energía \par}
    \vspace{2cm}
    {\Large Luis D. Aranda Sánchez\par}
    \vfill
    Director: Javier Rodríguez Martín
    \vfill
    {\large Septiembre 6, 2024\par}
\end{titlepage}

% Resumen (máximo de 5 páginas, incluyendo al final Palabras clave)
\clearpage
\pagestyle{simple}
% \newpage
\chapter*{Resumen}
\addcontentsline{toc}{chapter}{Resumen}
\input{capitulos/resumen/main.tex}

% Índice (paginado)
\clearpage
\pagestyle{simple}
% \newpage
\tableofcontents

% Introducción (donde se incluya los antecedentes y justificación)
\clearpage
\pagestyle{myfancy}
\newpage
\chapter{Introducción}
\input{capitulos/introduccion/main.tex}

% Objetivos
\chapter{Objetivos}
\input{capitulos/objetivos/main.tex}

% Metodología
\chapter{Metodología}
\input{capitulos/metodologia/main.tex}

% Resultados y discusión (incluyendo la valoración de impactos y de aspectos de responsabilidad legal, ética y profesional relacionados con el trabajo)
\chapter{Resultados y Discusión}
\input{capitulos/resultados_discusion/main.tex}

% Conclusiones
\chapter{Conclusiones}
\input{capitulos/conclusiones/main.tex}

% Planificación temporal y presupuesto
\chapter{Planificación Temporal y Presupuesto}
\input{capitulos/planificacion_presupuesto/main.tex}

% Bibliografía
\newpage
\addcontentsline{toc}{chapter}{Bibliografía}
\printbibliography

\end{document}


% Índice (paginado)
\clearpage
\pagestyle{simple}
% \newpage
\tableofcontents

% Introducción (donde se incluya los antecedentes y justificación)
\clearpage
\pagestyle{myfancy}
\newpage
\chapter{Introducción}
\documentclass[a4paper,11pt,twoside]{report}
\usepackage[left=25mm,right=25mm,top=25mm,bottom=25mm,includehead,includefoot,headsep=15mm,footskip=15mm]{geometry}
\usepackage{graphicx}
\usepackage{fancyhdr}
\usepackage{titlesec}
\usepackage[spanish]{babel}
\usepackage[utf8]{inputenc}
\usepackage{amsmath}
\usepackage{setspace}
\usepackage{svg}
\usepackage{hyperref}
\usepackage[backend=biber,style=numeric]{biblatex}
\addbibresource{references.bib}
\hypersetup{
    colorlinks=true,
    linkcolor=blue,      % color of internal links (sections, etc.)
    urlcolor=blue,       % color of external links
    pdftitle={Optimización energética de sistema híbrido con bomba de calor, suelo radiante, fotovoltaica y almacenamiento para vivienda},    % title
    pdfauthor={Luis D. Aranda Sánchez},     % author
    pdfkeywords={palabra1, palabra2, código1, etc.} % list of keywords
}

% Font change to Arial
\usepackage{helvet}
\renewcommand{\familydefault}{\sfdefault}

% Chapter titles in uppercase and larger font
\titleformat{\chapter}[hang]{\large\bfseries}{\thechapter.}{1em}{\MakeUppercase}
\titleformat{\section}[hang]{\bfseries}{\thesection.}{1em}{}
\titleformat{\subsection}[hang]{\bfseries}{\thesubsection.}{1em}{}

% Fancyhdr setup
\setlength{\headheight}{14.30174pt} % Adjust to recommended value, slightly larger for safety
\fancyhf{} % Clear all headers and footers
\fancyhead[LE]{\nouppercase{\leftmark}}
\fancyhead[RO]{Optimización energética para vivienda}
\fancyfoot[LE]{\thepage}
\fancyfoot[RE]{Escuela Técnica Superior de Ingenieros Industriales (UPM)}
\fancyfoot[LO]{Luis D. Aranda Sánchez}
\fancyfoot[RO]{\thepage}
\renewcommand{\headrulewidth}{0.4pt}
\renewcommand{\footrulewidth}{0.4pt}

\fancypagestyle{myfancy}{
    \fancyhf{} % Clear all headers and footers
    \fancyhead[LE]{\nouppercase{\leftmark}}
    \fancyhead[RO]{Optimización energética para vivienda}
    \fancyfoot[LE]{\thepage}
    \fancyfoot[RE]{Escuela Técnica Superior de Ingenieros Industriales (UPM)}
    \fancyfoot[LO]{Luis D. Aranda Sánchez}
    \fancyfoot[RO]{\thepage}
    \renewcommand{\headrulewidth}{0.4pt}
    \renewcommand{\footrulewidth}{0.4pt}
}

\fancypagestyle{simple}{
    \fancyhf{} % Clear all headers and footers
    \renewcommand{\headrulewidth}{0pt}
    \renewcommand{\footrulewidth}{0pt}
}

% Line spacing
\setstretch{1.2}

% Document starts here
\begin{document}

% Portada
\begin{titlepage}
    \centering
    {\scshape\LARGE Universidad Politécnica de Madrid \par}
    \vspace{1cm}
    {\scshape\Large Escuela Técnica Superior de Ingenieros Industriales\par}
    \vspace{1.5cm}
    {\huge\bfseries Optimización energética de sistema híbrido con bomba de calor, suelo radiante, fotovoltaica y almacenamiento para vivienda \par}
    \vspace{1.5cm}
    {\Large\bfseries Trabajo de Fin de Máster\par}
    \vspace{0.5cm}
    {\large Máster Universitario en Ingeniería de la Energía \par}
    \vspace{2cm}
    {\Large Luis D. Aranda Sánchez\par}
    \vfill
    Director: Javier Rodríguez Martín
    \vfill
    {\large Septiembre 6, 2024\par}
\end{titlepage}

% Resumen (máximo de 5 páginas, incluyendo al final Palabras clave)
\clearpage
\pagestyle{simple}
% \newpage
\chapter*{Resumen}
\addcontentsline{toc}{chapter}{Resumen}
\input{capitulos/resumen/main.tex}

% Índice (paginado)
\clearpage
\pagestyle{simple}
% \newpage
\tableofcontents

% Introducción (donde se incluya los antecedentes y justificación)
\clearpage
\pagestyle{myfancy}
\newpage
\chapter{Introducción}
\input{capitulos/introduccion/main.tex}

% Objetivos
\chapter{Objetivos}
\input{capitulos/objetivos/main.tex}

% Metodología
\chapter{Metodología}
\input{capitulos/metodologia/main.tex}

% Resultados y discusión (incluyendo la valoración de impactos y de aspectos de responsabilidad legal, ética y profesional relacionados con el trabajo)
\chapter{Resultados y Discusión}
\input{capitulos/resultados_discusion/main.tex}

% Conclusiones
\chapter{Conclusiones}
\input{capitulos/conclusiones/main.tex}

% Planificación temporal y presupuesto
\chapter{Planificación Temporal y Presupuesto}
\input{capitulos/planificacion_presupuesto/main.tex}

% Bibliografía
\newpage
\addcontentsline{toc}{chapter}{Bibliografía}
\printbibliography

\end{document}


% Objetivos
\chapter{Objetivos}
\documentclass[a4paper,11pt,twoside]{report}
\usepackage[left=25mm,right=25mm,top=25mm,bottom=25mm,includehead,includefoot,headsep=15mm,footskip=15mm]{geometry}
\usepackage{graphicx}
\usepackage{fancyhdr}
\usepackage{titlesec}
\usepackage[spanish]{babel}
\usepackage[utf8]{inputenc}
\usepackage{amsmath}
\usepackage{setspace}
\usepackage{svg}
\usepackage{hyperref}
\usepackage[backend=biber,style=numeric]{biblatex}
\addbibresource{references.bib}
\hypersetup{
    colorlinks=true,
    linkcolor=blue,      % color of internal links (sections, etc.)
    urlcolor=blue,       % color of external links
    pdftitle={Optimización energética de sistema híbrido con bomba de calor, suelo radiante, fotovoltaica y almacenamiento para vivienda},    % title
    pdfauthor={Luis D. Aranda Sánchez},     % author
    pdfkeywords={palabra1, palabra2, código1, etc.} % list of keywords
}

% Font change to Arial
\usepackage{helvet}
\renewcommand{\familydefault}{\sfdefault}

% Chapter titles in uppercase and larger font
\titleformat{\chapter}[hang]{\large\bfseries}{\thechapter.}{1em}{\MakeUppercase}
\titleformat{\section}[hang]{\bfseries}{\thesection.}{1em}{}
\titleformat{\subsection}[hang]{\bfseries}{\thesubsection.}{1em}{}

% Fancyhdr setup
\setlength{\headheight}{14.30174pt} % Adjust to recommended value, slightly larger for safety
\fancyhf{} % Clear all headers and footers
\fancyhead[LE]{\nouppercase{\leftmark}}
\fancyhead[RO]{Optimización energética para vivienda}
\fancyfoot[LE]{\thepage}
\fancyfoot[RE]{Escuela Técnica Superior de Ingenieros Industriales (UPM)}
\fancyfoot[LO]{Luis D. Aranda Sánchez}
\fancyfoot[RO]{\thepage}
\renewcommand{\headrulewidth}{0.4pt}
\renewcommand{\footrulewidth}{0.4pt}

\fancypagestyle{myfancy}{
    \fancyhf{} % Clear all headers and footers
    \fancyhead[LE]{\nouppercase{\leftmark}}
    \fancyhead[RO]{Optimización energética para vivienda}
    \fancyfoot[LE]{\thepage}
    \fancyfoot[RE]{Escuela Técnica Superior de Ingenieros Industriales (UPM)}
    \fancyfoot[LO]{Luis D. Aranda Sánchez}
    \fancyfoot[RO]{\thepage}
    \renewcommand{\headrulewidth}{0.4pt}
    \renewcommand{\footrulewidth}{0.4pt}
}

\fancypagestyle{simple}{
    \fancyhf{} % Clear all headers and footers
    \renewcommand{\headrulewidth}{0pt}
    \renewcommand{\footrulewidth}{0pt}
}

% Line spacing
\setstretch{1.2}

% Document starts here
\begin{document}

% Portada
\begin{titlepage}
    \centering
    {\scshape\LARGE Universidad Politécnica de Madrid \par}
    \vspace{1cm}
    {\scshape\Large Escuela Técnica Superior de Ingenieros Industriales\par}
    \vspace{1.5cm}
    {\huge\bfseries Optimización energética de sistema híbrido con bomba de calor, suelo radiante, fotovoltaica y almacenamiento para vivienda \par}
    \vspace{1.5cm}
    {\Large\bfseries Trabajo de Fin de Máster\par}
    \vspace{0.5cm}
    {\large Máster Universitario en Ingeniería de la Energía \par}
    \vspace{2cm}
    {\Large Luis D. Aranda Sánchez\par}
    \vfill
    Director: Javier Rodríguez Martín
    \vfill
    {\large Septiembre 6, 2024\par}
\end{titlepage}

% Resumen (máximo de 5 páginas, incluyendo al final Palabras clave)
\clearpage
\pagestyle{simple}
% \newpage
\chapter*{Resumen}
\addcontentsline{toc}{chapter}{Resumen}
\input{capitulos/resumen/main.tex}

% Índice (paginado)
\clearpage
\pagestyle{simple}
% \newpage
\tableofcontents

% Introducción (donde se incluya los antecedentes y justificación)
\clearpage
\pagestyle{myfancy}
\newpage
\chapter{Introducción}
\input{capitulos/introduccion/main.tex}

% Objetivos
\chapter{Objetivos}
\input{capitulos/objetivos/main.tex}

% Metodología
\chapter{Metodología}
\input{capitulos/metodologia/main.tex}

% Resultados y discusión (incluyendo la valoración de impactos y de aspectos de responsabilidad legal, ética y profesional relacionados con el trabajo)
\chapter{Resultados y Discusión}
\input{capitulos/resultados_discusion/main.tex}

% Conclusiones
\chapter{Conclusiones}
\input{capitulos/conclusiones/main.tex}

% Planificación temporal y presupuesto
\chapter{Planificación Temporal y Presupuesto}
\input{capitulos/planificacion_presupuesto/main.tex}

% Bibliografía
\newpage
\addcontentsline{toc}{chapter}{Bibliografía}
\printbibliography

\end{document}


% Metodología
\chapter{Metodología}
\documentclass[a4paper,11pt,twoside]{report}
\usepackage[left=25mm,right=25mm,top=25mm,bottom=25mm,includehead,includefoot,headsep=15mm,footskip=15mm]{geometry}
\usepackage{graphicx}
\usepackage{fancyhdr}
\usepackage{titlesec}
\usepackage[spanish]{babel}
\usepackage[utf8]{inputenc}
\usepackage{amsmath}
\usepackage{setspace}
\usepackage{svg}
\usepackage{hyperref}
\usepackage[backend=biber,style=numeric]{biblatex}
\addbibresource{references.bib}
\hypersetup{
    colorlinks=true,
    linkcolor=blue,      % color of internal links (sections, etc.)
    urlcolor=blue,       % color of external links
    pdftitle={Optimización energética de sistema híbrido con bomba de calor, suelo radiante, fotovoltaica y almacenamiento para vivienda},    % title
    pdfauthor={Luis D. Aranda Sánchez},     % author
    pdfkeywords={palabra1, palabra2, código1, etc.} % list of keywords
}

% Font change to Arial
\usepackage{helvet}
\renewcommand{\familydefault}{\sfdefault}

% Chapter titles in uppercase and larger font
\titleformat{\chapter}[hang]{\large\bfseries}{\thechapter.}{1em}{\MakeUppercase}
\titleformat{\section}[hang]{\bfseries}{\thesection.}{1em}{}
\titleformat{\subsection}[hang]{\bfseries}{\thesubsection.}{1em}{}

% Fancyhdr setup
\setlength{\headheight}{14.30174pt} % Adjust to recommended value, slightly larger for safety
\fancyhf{} % Clear all headers and footers
\fancyhead[LE]{\nouppercase{\leftmark}}
\fancyhead[RO]{Optimización energética para vivienda}
\fancyfoot[LE]{\thepage}
\fancyfoot[RE]{Escuela Técnica Superior de Ingenieros Industriales (UPM)}
\fancyfoot[LO]{Luis D. Aranda Sánchez}
\fancyfoot[RO]{\thepage}
\renewcommand{\headrulewidth}{0.4pt}
\renewcommand{\footrulewidth}{0.4pt}

\fancypagestyle{myfancy}{
    \fancyhf{} % Clear all headers and footers
    \fancyhead[LE]{\nouppercase{\leftmark}}
    \fancyhead[RO]{Optimización energética para vivienda}
    \fancyfoot[LE]{\thepage}
    \fancyfoot[RE]{Escuela Técnica Superior de Ingenieros Industriales (UPM)}
    \fancyfoot[LO]{Luis D. Aranda Sánchez}
    \fancyfoot[RO]{\thepage}
    \renewcommand{\headrulewidth}{0.4pt}
    \renewcommand{\footrulewidth}{0.4pt}
}

\fancypagestyle{simple}{
    \fancyhf{} % Clear all headers and footers
    \renewcommand{\headrulewidth}{0pt}
    \renewcommand{\footrulewidth}{0pt}
}

% Line spacing
\setstretch{1.2}

% Document starts here
\begin{document}

% Portada
\begin{titlepage}
    \centering
    {\scshape\LARGE Universidad Politécnica de Madrid \par}
    \vspace{1cm}
    {\scshape\Large Escuela Técnica Superior de Ingenieros Industriales\par}
    \vspace{1.5cm}
    {\huge\bfseries Optimización energética de sistema híbrido con bomba de calor, suelo radiante, fotovoltaica y almacenamiento para vivienda \par}
    \vspace{1.5cm}
    {\Large\bfseries Trabajo de Fin de Máster\par}
    \vspace{0.5cm}
    {\large Máster Universitario en Ingeniería de la Energía \par}
    \vspace{2cm}
    {\Large Luis D. Aranda Sánchez\par}
    \vfill
    Director: Javier Rodríguez Martín
    \vfill
    {\large Septiembre 6, 2024\par}
\end{titlepage}

% Resumen (máximo de 5 páginas, incluyendo al final Palabras clave)
\clearpage
\pagestyle{simple}
% \newpage
\chapter*{Resumen}
\addcontentsline{toc}{chapter}{Resumen}
\input{capitulos/resumen/main.tex}

% Índice (paginado)
\clearpage
\pagestyle{simple}
% \newpage
\tableofcontents

% Introducción (donde se incluya los antecedentes y justificación)
\clearpage
\pagestyle{myfancy}
\newpage
\chapter{Introducción}
\input{capitulos/introduccion/main.tex}

% Objetivos
\chapter{Objetivos}
\input{capitulos/objetivos/main.tex}

% Metodología
\chapter{Metodología}
\input{capitulos/metodologia/main.tex}

% Resultados y discusión (incluyendo la valoración de impactos y de aspectos de responsabilidad legal, ética y profesional relacionados con el trabajo)
\chapter{Resultados y Discusión}
\input{capitulos/resultados_discusion/main.tex}

% Conclusiones
\chapter{Conclusiones}
\input{capitulos/conclusiones/main.tex}

% Planificación temporal y presupuesto
\chapter{Planificación Temporal y Presupuesto}
\input{capitulos/planificacion_presupuesto/main.tex}

% Bibliografía
\newpage
\addcontentsline{toc}{chapter}{Bibliografía}
\printbibliography

\end{document}


% Resultados y discusión (incluyendo la valoración de impactos y de aspectos de responsabilidad legal, ética y profesional relacionados con el trabajo)
\chapter{Resultados y Discusión}
\documentclass[a4paper,11pt,twoside]{report}
\usepackage[left=25mm,right=25mm,top=25mm,bottom=25mm,includehead,includefoot,headsep=15mm,footskip=15mm]{geometry}
\usepackage{graphicx}
\usepackage{fancyhdr}
\usepackage{titlesec}
\usepackage[spanish]{babel}
\usepackage[utf8]{inputenc}
\usepackage{amsmath}
\usepackage{setspace}
\usepackage{svg}
\usepackage{hyperref}
\usepackage[backend=biber,style=numeric]{biblatex}
\addbibresource{references.bib}
\hypersetup{
    colorlinks=true,
    linkcolor=blue,      % color of internal links (sections, etc.)
    urlcolor=blue,       % color of external links
    pdftitle={Optimización energética de sistema híbrido con bomba de calor, suelo radiante, fotovoltaica y almacenamiento para vivienda},    % title
    pdfauthor={Luis D. Aranda Sánchez},     % author
    pdfkeywords={palabra1, palabra2, código1, etc.} % list of keywords
}

% Font change to Arial
\usepackage{helvet}
\renewcommand{\familydefault}{\sfdefault}

% Chapter titles in uppercase and larger font
\titleformat{\chapter}[hang]{\large\bfseries}{\thechapter.}{1em}{\MakeUppercase}
\titleformat{\section}[hang]{\bfseries}{\thesection.}{1em}{}
\titleformat{\subsection}[hang]{\bfseries}{\thesubsection.}{1em}{}

% Fancyhdr setup
\setlength{\headheight}{14.30174pt} % Adjust to recommended value, slightly larger for safety
\fancyhf{} % Clear all headers and footers
\fancyhead[LE]{\nouppercase{\leftmark}}
\fancyhead[RO]{Optimización energética para vivienda}
\fancyfoot[LE]{\thepage}
\fancyfoot[RE]{Escuela Técnica Superior de Ingenieros Industriales (UPM)}
\fancyfoot[LO]{Luis D. Aranda Sánchez}
\fancyfoot[RO]{\thepage}
\renewcommand{\headrulewidth}{0.4pt}
\renewcommand{\footrulewidth}{0.4pt}

\fancypagestyle{myfancy}{
    \fancyhf{} % Clear all headers and footers
    \fancyhead[LE]{\nouppercase{\leftmark}}
    \fancyhead[RO]{Optimización energética para vivienda}
    \fancyfoot[LE]{\thepage}
    \fancyfoot[RE]{Escuela Técnica Superior de Ingenieros Industriales (UPM)}
    \fancyfoot[LO]{Luis D. Aranda Sánchez}
    \fancyfoot[RO]{\thepage}
    \renewcommand{\headrulewidth}{0.4pt}
    \renewcommand{\footrulewidth}{0.4pt}
}

\fancypagestyle{simple}{
    \fancyhf{} % Clear all headers and footers
    \renewcommand{\headrulewidth}{0pt}
    \renewcommand{\footrulewidth}{0pt}
}

% Line spacing
\setstretch{1.2}

% Document starts here
\begin{document}

% Portada
\begin{titlepage}
    \centering
    {\scshape\LARGE Universidad Politécnica de Madrid \par}
    \vspace{1cm}
    {\scshape\Large Escuela Técnica Superior de Ingenieros Industriales\par}
    \vspace{1.5cm}
    {\huge\bfseries Optimización energética de sistema híbrido con bomba de calor, suelo radiante, fotovoltaica y almacenamiento para vivienda \par}
    \vspace{1.5cm}
    {\Large\bfseries Trabajo de Fin de Máster\par}
    \vspace{0.5cm}
    {\large Máster Universitario en Ingeniería de la Energía \par}
    \vspace{2cm}
    {\Large Luis D. Aranda Sánchez\par}
    \vfill
    Director: Javier Rodríguez Martín
    \vfill
    {\large Septiembre 6, 2024\par}
\end{titlepage}

% Resumen (máximo de 5 páginas, incluyendo al final Palabras clave)
\clearpage
\pagestyle{simple}
% \newpage
\chapter*{Resumen}
\addcontentsline{toc}{chapter}{Resumen}
\input{capitulos/resumen/main.tex}

% Índice (paginado)
\clearpage
\pagestyle{simple}
% \newpage
\tableofcontents

% Introducción (donde se incluya los antecedentes y justificación)
\clearpage
\pagestyle{myfancy}
\newpage
\chapter{Introducción}
\input{capitulos/introduccion/main.tex}

% Objetivos
\chapter{Objetivos}
\input{capitulos/objetivos/main.tex}

% Metodología
\chapter{Metodología}
\input{capitulos/metodologia/main.tex}

% Resultados y discusión (incluyendo la valoración de impactos y de aspectos de responsabilidad legal, ética y profesional relacionados con el trabajo)
\chapter{Resultados y Discusión}
\input{capitulos/resultados_discusion/main.tex}

% Conclusiones
\chapter{Conclusiones}
\input{capitulos/conclusiones/main.tex}

% Planificación temporal y presupuesto
\chapter{Planificación Temporal y Presupuesto}
\input{capitulos/planificacion_presupuesto/main.tex}

% Bibliografía
\newpage
\addcontentsline{toc}{chapter}{Bibliografía}
\printbibliography

\end{document}


% Conclusiones
\chapter{Conclusiones}
\documentclass[a4paper,11pt,twoside]{report}
\usepackage[left=25mm,right=25mm,top=25mm,bottom=25mm,includehead,includefoot,headsep=15mm,footskip=15mm]{geometry}
\usepackage{graphicx}
\usepackage{fancyhdr}
\usepackage{titlesec}
\usepackage[spanish]{babel}
\usepackage[utf8]{inputenc}
\usepackage{amsmath}
\usepackage{setspace}
\usepackage{svg}
\usepackage{hyperref}
\usepackage[backend=biber,style=numeric]{biblatex}
\addbibresource{references.bib}
\hypersetup{
    colorlinks=true,
    linkcolor=blue,      % color of internal links (sections, etc.)
    urlcolor=blue,       % color of external links
    pdftitle={Optimización energética de sistema híbrido con bomba de calor, suelo radiante, fotovoltaica y almacenamiento para vivienda},    % title
    pdfauthor={Luis D. Aranda Sánchez},     % author
    pdfkeywords={palabra1, palabra2, código1, etc.} % list of keywords
}

% Font change to Arial
\usepackage{helvet}
\renewcommand{\familydefault}{\sfdefault}

% Chapter titles in uppercase and larger font
\titleformat{\chapter}[hang]{\large\bfseries}{\thechapter.}{1em}{\MakeUppercase}
\titleformat{\section}[hang]{\bfseries}{\thesection.}{1em}{}
\titleformat{\subsection}[hang]{\bfseries}{\thesubsection.}{1em}{}

% Fancyhdr setup
\setlength{\headheight}{14.30174pt} % Adjust to recommended value, slightly larger for safety
\fancyhf{} % Clear all headers and footers
\fancyhead[LE]{\nouppercase{\leftmark}}
\fancyhead[RO]{Optimización energética para vivienda}
\fancyfoot[LE]{\thepage}
\fancyfoot[RE]{Escuela Técnica Superior de Ingenieros Industriales (UPM)}
\fancyfoot[LO]{Luis D. Aranda Sánchez}
\fancyfoot[RO]{\thepage}
\renewcommand{\headrulewidth}{0.4pt}
\renewcommand{\footrulewidth}{0.4pt}

\fancypagestyle{myfancy}{
    \fancyhf{} % Clear all headers and footers
    \fancyhead[LE]{\nouppercase{\leftmark}}
    \fancyhead[RO]{Optimización energética para vivienda}
    \fancyfoot[LE]{\thepage}
    \fancyfoot[RE]{Escuela Técnica Superior de Ingenieros Industriales (UPM)}
    \fancyfoot[LO]{Luis D. Aranda Sánchez}
    \fancyfoot[RO]{\thepage}
    \renewcommand{\headrulewidth}{0.4pt}
    \renewcommand{\footrulewidth}{0.4pt}
}

\fancypagestyle{simple}{
    \fancyhf{} % Clear all headers and footers
    \renewcommand{\headrulewidth}{0pt}
    \renewcommand{\footrulewidth}{0pt}
}

% Line spacing
\setstretch{1.2}

% Document starts here
\begin{document}

% Portada
\begin{titlepage}
    \centering
    {\scshape\LARGE Universidad Politécnica de Madrid \par}
    \vspace{1cm}
    {\scshape\Large Escuela Técnica Superior de Ingenieros Industriales\par}
    \vspace{1.5cm}
    {\huge\bfseries Optimización energética de sistema híbrido con bomba de calor, suelo radiante, fotovoltaica y almacenamiento para vivienda \par}
    \vspace{1.5cm}
    {\Large\bfseries Trabajo de Fin de Máster\par}
    \vspace{0.5cm}
    {\large Máster Universitario en Ingeniería de la Energía \par}
    \vspace{2cm}
    {\Large Luis D. Aranda Sánchez\par}
    \vfill
    Director: Javier Rodríguez Martín
    \vfill
    {\large Septiembre 6, 2024\par}
\end{titlepage}

% Resumen (máximo de 5 páginas, incluyendo al final Palabras clave)
\clearpage
\pagestyle{simple}
% \newpage
\chapter*{Resumen}
\addcontentsline{toc}{chapter}{Resumen}
\input{capitulos/resumen/main.tex}

% Índice (paginado)
\clearpage
\pagestyle{simple}
% \newpage
\tableofcontents

% Introducción (donde se incluya los antecedentes y justificación)
\clearpage
\pagestyle{myfancy}
\newpage
\chapter{Introducción}
\input{capitulos/introduccion/main.tex}

% Objetivos
\chapter{Objetivos}
\input{capitulos/objetivos/main.tex}

% Metodología
\chapter{Metodología}
\input{capitulos/metodologia/main.tex}

% Resultados y discusión (incluyendo la valoración de impactos y de aspectos de responsabilidad legal, ética y profesional relacionados con el trabajo)
\chapter{Resultados y Discusión}
\input{capitulos/resultados_discusion/main.tex}

% Conclusiones
\chapter{Conclusiones}
\input{capitulos/conclusiones/main.tex}

% Planificación temporal y presupuesto
\chapter{Planificación Temporal y Presupuesto}
\input{capitulos/planificacion_presupuesto/main.tex}

% Bibliografía
\newpage
\addcontentsline{toc}{chapter}{Bibliografía}
\printbibliography

\end{document}


% Planificación temporal y presupuesto
\chapter{Planificación Temporal y Presupuesto}
\documentclass[a4paper,11pt,twoside]{report}
\usepackage[left=25mm,right=25mm,top=25mm,bottom=25mm,includehead,includefoot,headsep=15mm,footskip=15mm]{geometry}
\usepackage{graphicx}
\usepackage{fancyhdr}
\usepackage{titlesec}
\usepackage[spanish]{babel}
\usepackage[utf8]{inputenc}
\usepackage{amsmath}
\usepackage{setspace}
\usepackage{svg}
\usepackage{hyperref}
\usepackage[backend=biber,style=numeric]{biblatex}
\addbibresource{references.bib}
\hypersetup{
    colorlinks=true,
    linkcolor=blue,      % color of internal links (sections, etc.)
    urlcolor=blue,       % color of external links
    pdftitle={Optimización energética de sistema híbrido con bomba de calor, suelo radiante, fotovoltaica y almacenamiento para vivienda},    % title
    pdfauthor={Luis D. Aranda Sánchez},     % author
    pdfkeywords={palabra1, palabra2, código1, etc.} % list of keywords
}

% Font change to Arial
\usepackage{helvet}
\renewcommand{\familydefault}{\sfdefault}

% Chapter titles in uppercase and larger font
\titleformat{\chapter}[hang]{\large\bfseries}{\thechapter.}{1em}{\MakeUppercase}
\titleformat{\section}[hang]{\bfseries}{\thesection.}{1em}{}
\titleformat{\subsection}[hang]{\bfseries}{\thesubsection.}{1em}{}

% Fancyhdr setup
\setlength{\headheight}{14.30174pt} % Adjust to recommended value, slightly larger for safety
\fancyhf{} % Clear all headers and footers
\fancyhead[LE]{\nouppercase{\leftmark}}
\fancyhead[RO]{Optimización energética para vivienda}
\fancyfoot[LE]{\thepage}
\fancyfoot[RE]{Escuela Técnica Superior de Ingenieros Industriales (UPM)}
\fancyfoot[LO]{Luis D. Aranda Sánchez}
\fancyfoot[RO]{\thepage}
\renewcommand{\headrulewidth}{0.4pt}
\renewcommand{\footrulewidth}{0.4pt}

\fancypagestyle{myfancy}{
    \fancyhf{} % Clear all headers and footers
    \fancyhead[LE]{\nouppercase{\leftmark}}
    \fancyhead[RO]{Optimización energética para vivienda}
    \fancyfoot[LE]{\thepage}
    \fancyfoot[RE]{Escuela Técnica Superior de Ingenieros Industriales (UPM)}
    \fancyfoot[LO]{Luis D. Aranda Sánchez}
    \fancyfoot[RO]{\thepage}
    \renewcommand{\headrulewidth}{0.4pt}
    \renewcommand{\footrulewidth}{0.4pt}
}

\fancypagestyle{simple}{
    \fancyhf{} % Clear all headers and footers
    \renewcommand{\headrulewidth}{0pt}
    \renewcommand{\footrulewidth}{0pt}
}

% Line spacing
\setstretch{1.2}

% Document starts here
\begin{document}

% Portada
\begin{titlepage}
    \centering
    {\scshape\LARGE Universidad Politécnica de Madrid \par}
    \vspace{1cm}
    {\scshape\Large Escuela Técnica Superior de Ingenieros Industriales\par}
    \vspace{1.5cm}
    {\huge\bfseries Optimización energética de sistema híbrido con bomba de calor, suelo radiante, fotovoltaica y almacenamiento para vivienda \par}
    \vspace{1.5cm}
    {\Large\bfseries Trabajo de Fin de Máster\par}
    \vspace{0.5cm}
    {\large Máster Universitario en Ingeniería de la Energía \par}
    \vspace{2cm}
    {\Large Luis D. Aranda Sánchez\par}
    \vfill
    Director: Javier Rodríguez Martín
    \vfill
    {\large Septiembre 6, 2024\par}
\end{titlepage}

% Resumen (máximo de 5 páginas, incluyendo al final Palabras clave)
\clearpage
\pagestyle{simple}
% \newpage
\chapter*{Resumen}
\addcontentsline{toc}{chapter}{Resumen}
\input{capitulos/resumen/main.tex}

% Índice (paginado)
\clearpage
\pagestyle{simple}
% \newpage
\tableofcontents

% Introducción (donde se incluya los antecedentes y justificación)
\clearpage
\pagestyle{myfancy}
\newpage
\chapter{Introducción}
\input{capitulos/introduccion/main.tex}

% Objetivos
\chapter{Objetivos}
\input{capitulos/objetivos/main.tex}

% Metodología
\chapter{Metodología}
\input{capitulos/metodologia/main.tex}

% Resultados y discusión (incluyendo la valoración de impactos y de aspectos de responsabilidad legal, ética y profesional relacionados con el trabajo)
\chapter{Resultados y Discusión}
\input{capitulos/resultados_discusion/main.tex}

% Conclusiones
\chapter{Conclusiones}
\input{capitulos/conclusiones/main.tex}

% Planificación temporal y presupuesto
\chapter{Planificación Temporal y Presupuesto}
\input{capitulos/planificacion_presupuesto/main.tex}

% Bibliografía
\newpage
\addcontentsline{toc}{chapter}{Bibliografía}
\printbibliography

\end{document}


% Bibliografía
\newpage
\addcontentsline{toc}{chapter}{Bibliografía}
\printbibliography

\end{document}


% Objetivos
\chapter{Objetivos}
\documentclass[a4paper,11pt,twoside]{report}
\usepackage[left=25mm,right=25mm,top=25mm,bottom=25mm,includehead,includefoot,headsep=15mm,footskip=15mm]{geometry}
\usepackage{graphicx}
\usepackage{fancyhdr}
\usepackage{titlesec}
\usepackage[spanish]{babel}
\usepackage[utf8]{inputenc}
\usepackage{amsmath}
\usepackage{setspace}
\usepackage{svg}
\usepackage{hyperref}
\usepackage[backend=biber,style=numeric]{biblatex}
\addbibresource{references.bib}
\hypersetup{
    colorlinks=true,
    linkcolor=blue,      % color of internal links (sections, etc.)
    urlcolor=blue,       % color of external links
    pdftitle={Optimización energética de sistema híbrido con bomba de calor, suelo radiante, fotovoltaica y almacenamiento para vivienda},    % title
    pdfauthor={Luis D. Aranda Sánchez},     % author
    pdfkeywords={palabra1, palabra2, código1, etc.} % list of keywords
}

% Font change to Arial
\usepackage{helvet}
\renewcommand{\familydefault}{\sfdefault}

% Chapter titles in uppercase and larger font
\titleformat{\chapter}[hang]{\large\bfseries}{\thechapter.}{1em}{\MakeUppercase}
\titleformat{\section}[hang]{\bfseries}{\thesection.}{1em}{}
\titleformat{\subsection}[hang]{\bfseries}{\thesubsection.}{1em}{}

% Fancyhdr setup
\setlength{\headheight}{14.30174pt} % Adjust to recommended value, slightly larger for safety
\fancyhf{} % Clear all headers and footers
\fancyhead[LE]{\nouppercase{\leftmark}}
\fancyhead[RO]{Optimización energética para vivienda}
\fancyfoot[LE]{\thepage}
\fancyfoot[RE]{Escuela Técnica Superior de Ingenieros Industriales (UPM)}
\fancyfoot[LO]{Luis D. Aranda Sánchez}
\fancyfoot[RO]{\thepage}
\renewcommand{\headrulewidth}{0.4pt}
\renewcommand{\footrulewidth}{0.4pt}

\fancypagestyle{myfancy}{
    \fancyhf{} % Clear all headers and footers
    \fancyhead[LE]{\nouppercase{\leftmark}}
    \fancyhead[RO]{Optimización energética para vivienda}
    \fancyfoot[LE]{\thepage}
    \fancyfoot[RE]{Escuela Técnica Superior de Ingenieros Industriales (UPM)}
    \fancyfoot[LO]{Luis D. Aranda Sánchez}
    \fancyfoot[RO]{\thepage}
    \renewcommand{\headrulewidth}{0.4pt}
    \renewcommand{\footrulewidth}{0.4pt}
}

\fancypagestyle{simple}{
    \fancyhf{} % Clear all headers and footers
    \renewcommand{\headrulewidth}{0pt}
    \renewcommand{\footrulewidth}{0pt}
}

% Line spacing
\setstretch{1.2}

% Document starts here
\begin{document}

% Portada
\begin{titlepage}
    \centering
    {\scshape\LARGE Universidad Politécnica de Madrid \par}
    \vspace{1cm}
    {\scshape\Large Escuela Técnica Superior de Ingenieros Industriales\par}
    \vspace{1.5cm}
    {\huge\bfseries Optimización energética de sistema híbrido con bomba de calor, suelo radiante, fotovoltaica y almacenamiento para vivienda \par}
    \vspace{1.5cm}
    {\Large\bfseries Trabajo de Fin de Máster\par}
    \vspace{0.5cm}
    {\large Máster Universitario en Ingeniería de la Energía \par}
    \vspace{2cm}
    {\Large Luis D. Aranda Sánchez\par}
    \vfill
    Director: Javier Rodríguez Martín
    \vfill
    {\large Septiembre 6, 2024\par}
\end{titlepage}

% Resumen (máximo de 5 páginas, incluyendo al final Palabras clave)
\clearpage
\pagestyle{simple}
% \newpage
\chapter*{Resumen}
\addcontentsline{toc}{chapter}{Resumen}
\documentclass[a4paper,11pt,twoside]{report}
\usepackage[left=25mm,right=25mm,top=25mm,bottom=25mm,includehead,includefoot,headsep=15mm,footskip=15mm]{geometry}
\usepackage{graphicx}
\usepackage{fancyhdr}
\usepackage{titlesec}
\usepackage[spanish]{babel}
\usepackage[utf8]{inputenc}
\usepackage{amsmath}
\usepackage{setspace}
\usepackage{svg}
\usepackage{hyperref}
\usepackage[backend=biber,style=numeric]{biblatex}
\addbibresource{references.bib}
\hypersetup{
    colorlinks=true,
    linkcolor=blue,      % color of internal links (sections, etc.)
    urlcolor=blue,       % color of external links
    pdftitle={Optimización energética de sistema híbrido con bomba de calor, suelo radiante, fotovoltaica y almacenamiento para vivienda},    % title
    pdfauthor={Luis D. Aranda Sánchez},     % author
    pdfkeywords={palabra1, palabra2, código1, etc.} % list of keywords
}

% Font change to Arial
\usepackage{helvet}
\renewcommand{\familydefault}{\sfdefault}

% Chapter titles in uppercase and larger font
\titleformat{\chapter}[hang]{\large\bfseries}{\thechapter.}{1em}{\MakeUppercase}
\titleformat{\section}[hang]{\bfseries}{\thesection.}{1em}{}
\titleformat{\subsection}[hang]{\bfseries}{\thesubsection.}{1em}{}

% Fancyhdr setup
\setlength{\headheight}{14.30174pt} % Adjust to recommended value, slightly larger for safety
\fancyhf{} % Clear all headers and footers
\fancyhead[LE]{\nouppercase{\leftmark}}
\fancyhead[RO]{Optimización energética para vivienda}
\fancyfoot[LE]{\thepage}
\fancyfoot[RE]{Escuela Técnica Superior de Ingenieros Industriales (UPM)}
\fancyfoot[LO]{Luis D. Aranda Sánchez}
\fancyfoot[RO]{\thepage}
\renewcommand{\headrulewidth}{0.4pt}
\renewcommand{\footrulewidth}{0.4pt}

\fancypagestyle{myfancy}{
    \fancyhf{} % Clear all headers and footers
    \fancyhead[LE]{\nouppercase{\leftmark}}
    \fancyhead[RO]{Optimización energética para vivienda}
    \fancyfoot[LE]{\thepage}
    \fancyfoot[RE]{Escuela Técnica Superior de Ingenieros Industriales (UPM)}
    \fancyfoot[LO]{Luis D. Aranda Sánchez}
    \fancyfoot[RO]{\thepage}
    \renewcommand{\headrulewidth}{0.4pt}
    \renewcommand{\footrulewidth}{0.4pt}
}

\fancypagestyle{simple}{
    \fancyhf{} % Clear all headers and footers
    \renewcommand{\headrulewidth}{0pt}
    \renewcommand{\footrulewidth}{0pt}
}

% Line spacing
\setstretch{1.2}

% Document starts here
\begin{document}

% Portada
\begin{titlepage}
    \centering
    {\scshape\LARGE Universidad Politécnica de Madrid \par}
    \vspace{1cm}
    {\scshape\Large Escuela Técnica Superior de Ingenieros Industriales\par}
    \vspace{1.5cm}
    {\huge\bfseries Optimización energética de sistema híbrido con bomba de calor, suelo radiante, fotovoltaica y almacenamiento para vivienda \par}
    \vspace{1.5cm}
    {\Large\bfseries Trabajo de Fin de Máster\par}
    \vspace{0.5cm}
    {\large Máster Universitario en Ingeniería de la Energía \par}
    \vspace{2cm}
    {\Large Luis D. Aranda Sánchez\par}
    \vfill
    Director: Javier Rodríguez Martín
    \vfill
    {\large Septiembre 6, 2024\par}
\end{titlepage}

% Resumen (máximo de 5 páginas, incluyendo al final Palabras clave)
\clearpage
\pagestyle{simple}
% \newpage
\chapter*{Resumen}
\addcontentsline{toc}{chapter}{Resumen}
\input{capitulos/resumen/main.tex}

% Índice (paginado)
\clearpage
\pagestyle{simple}
% \newpage
\tableofcontents

% Introducción (donde se incluya los antecedentes y justificación)
\clearpage
\pagestyle{myfancy}
\newpage
\chapter{Introducción}
\input{capitulos/introduccion/main.tex}

% Objetivos
\chapter{Objetivos}
\input{capitulos/objetivos/main.tex}

% Metodología
\chapter{Metodología}
\input{capitulos/metodologia/main.tex}

% Resultados y discusión (incluyendo la valoración de impactos y de aspectos de responsabilidad legal, ética y profesional relacionados con el trabajo)
\chapter{Resultados y Discusión}
\input{capitulos/resultados_discusion/main.tex}

% Conclusiones
\chapter{Conclusiones}
\input{capitulos/conclusiones/main.tex}

% Planificación temporal y presupuesto
\chapter{Planificación Temporal y Presupuesto}
\input{capitulos/planificacion_presupuesto/main.tex}

% Bibliografía
\newpage
\addcontentsline{toc}{chapter}{Bibliografía}
\printbibliography

\end{document}


% Índice (paginado)
\clearpage
\pagestyle{simple}
% \newpage
\tableofcontents

% Introducción (donde se incluya los antecedentes y justificación)
\clearpage
\pagestyle{myfancy}
\newpage
\chapter{Introducción}
\documentclass[a4paper,11pt,twoside]{report}
\usepackage[left=25mm,right=25mm,top=25mm,bottom=25mm,includehead,includefoot,headsep=15mm,footskip=15mm]{geometry}
\usepackage{graphicx}
\usepackage{fancyhdr}
\usepackage{titlesec}
\usepackage[spanish]{babel}
\usepackage[utf8]{inputenc}
\usepackage{amsmath}
\usepackage{setspace}
\usepackage{svg}
\usepackage{hyperref}
\usepackage[backend=biber,style=numeric]{biblatex}
\addbibresource{references.bib}
\hypersetup{
    colorlinks=true,
    linkcolor=blue,      % color of internal links (sections, etc.)
    urlcolor=blue,       % color of external links
    pdftitle={Optimización energética de sistema híbrido con bomba de calor, suelo radiante, fotovoltaica y almacenamiento para vivienda},    % title
    pdfauthor={Luis D. Aranda Sánchez},     % author
    pdfkeywords={palabra1, palabra2, código1, etc.} % list of keywords
}

% Font change to Arial
\usepackage{helvet}
\renewcommand{\familydefault}{\sfdefault}

% Chapter titles in uppercase and larger font
\titleformat{\chapter}[hang]{\large\bfseries}{\thechapter.}{1em}{\MakeUppercase}
\titleformat{\section}[hang]{\bfseries}{\thesection.}{1em}{}
\titleformat{\subsection}[hang]{\bfseries}{\thesubsection.}{1em}{}

% Fancyhdr setup
\setlength{\headheight}{14.30174pt} % Adjust to recommended value, slightly larger for safety
\fancyhf{} % Clear all headers and footers
\fancyhead[LE]{\nouppercase{\leftmark}}
\fancyhead[RO]{Optimización energética para vivienda}
\fancyfoot[LE]{\thepage}
\fancyfoot[RE]{Escuela Técnica Superior de Ingenieros Industriales (UPM)}
\fancyfoot[LO]{Luis D. Aranda Sánchez}
\fancyfoot[RO]{\thepage}
\renewcommand{\headrulewidth}{0.4pt}
\renewcommand{\footrulewidth}{0.4pt}

\fancypagestyle{myfancy}{
    \fancyhf{} % Clear all headers and footers
    \fancyhead[LE]{\nouppercase{\leftmark}}
    \fancyhead[RO]{Optimización energética para vivienda}
    \fancyfoot[LE]{\thepage}
    \fancyfoot[RE]{Escuela Técnica Superior de Ingenieros Industriales (UPM)}
    \fancyfoot[LO]{Luis D. Aranda Sánchez}
    \fancyfoot[RO]{\thepage}
    \renewcommand{\headrulewidth}{0.4pt}
    \renewcommand{\footrulewidth}{0.4pt}
}

\fancypagestyle{simple}{
    \fancyhf{} % Clear all headers and footers
    \renewcommand{\headrulewidth}{0pt}
    \renewcommand{\footrulewidth}{0pt}
}

% Line spacing
\setstretch{1.2}

% Document starts here
\begin{document}

% Portada
\begin{titlepage}
    \centering
    {\scshape\LARGE Universidad Politécnica de Madrid \par}
    \vspace{1cm}
    {\scshape\Large Escuela Técnica Superior de Ingenieros Industriales\par}
    \vspace{1.5cm}
    {\huge\bfseries Optimización energética de sistema híbrido con bomba de calor, suelo radiante, fotovoltaica y almacenamiento para vivienda \par}
    \vspace{1.5cm}
    {\Large\bfseries Trabajo de Fin de Máster\par}
    \vspace{0.5cm}
    {\large Máster Universitario en Ingeniería de la Energía \par}
    \vspace{2cm}
    {\Large Luis D. Aranda Sánchez\par}
    \vfill
    Director: Javier Rodríguez Martín
    \vfill
    {\large Septiembre 6, 2024\par}
\end{titlepage}

% Resumen (máximo de 5 páginas, incluyendo al final Palabras clave)
\clearpage
\pagestyle{simple}
% \newpage
\chapter*{Resumen}
\addcontentsline{toc}{chapter}{Resumen}
\input{capitulos/resumen/main.tex}

% Índice (paginado)
\clearpage
\pagestyle{simple}
% \newpage
\tableofcontents

% Introducción (donde se incluya los antecedentes y justificación)
\clearpage
\pagestyle{myfancy}
\newpage
\chapter{Introducción}
\input{capitulos/introduccion/main.tex}

% Objetivos
\chapter{Objetivos}
\input{capitulos/objetivos/main.tex}

% Metodología
\chapter{Metodología}
\input{capitulos/metodologia/main.tex}

% Resultados y discusión (incluyendo la valoración de impactos y de aspectos de responsabilidad legal, ética y profesional relacionados con el trabajo)
\chapter{Resultados y Discusión}
\input{capitulos/resultados_discusion/main.tex}

% Conclusiones
\chapter{Conclusiones}
\input{capitulos/conclusiones/main.tex}

% Planificación temporal y presupuesto
\chapter{Planificación Temporal y Presupuesto}
\input{capitulos/planificacion_presupuesto/main.tex}

% Bibliografía
\newpage
\addcontentsline{toc}{chapter}{Bibliografía}
\printbibliography

\end{document}


% Objetivos
\chapter{Objetivos}
\documentclass[a4paper,11pt,twoside]{report}
\usepackage[left=25mm,right=25mm,top=25mm,bottom=25mm,includehead,includefoot,headsep=15mm,footskip=15mm]{geometry}
\usepackage{graphicx}
\usepackage{fancyhdr}
\usepackage{titlesec}
\usepackage[spanish]{babel}
\usepackage[utf8]{inputenc}
\usepackage{amsmath}
\usepackage{setspace}
\usepackage{svg}
\usepackage{hyperref}
\usepackage[backend=biber,style=numeric]{biblatex}
\addbibresource{references.bib}
\hypersetup{
    colorlinks=true,
    linkcolor=blue,      % color of internal links (sections, etc.)
    urlcolor=blue,       % color of external links
    pdftitle={Optimización energética de sistema híbrido con bomba de calor, suelo radiante, fotovoltaica y almacenamiento para vivienda},    % title
    pdfauthor={Luis D. Aranda Sánchez},     % author
    pdfkeywords={palabra1, palabra2, código1, etc.} % list of keywords
}

% Font change to Arial
\usepackage{helvet}
\renewcommand{\familydefault}{\sfdefault}

% Chapter titles in uppercase and larger font
\titleformat{\chapter}[hang]{\large\bfseries}{\thechapter.}{1em}{\MakeUppercase}
\titleformat{\section}[hang]{\bfseries}{\thesection.}{1em}{}
\titleformat{\subsection}[hang]{\bfseries}{\thesubsection.}{1em}{}

% Fancyhdr setup
\setlength{\headheight}{14.30174pt} % Adjust to recommended value, slightly larger for safety
\fancyhf{} % Clear all headers and footers
\fancyhead[LE]{\nouppercase{\leftmark}}
\fancyhead[RO]{Optimización energética para vivienda}
\fancyfoot[LE]{\thepage}
\fancyfoot[RE]{Escuela Técnica Superior de Ingenieros Industriales (UPM)}
\fancyfoot[LO]{Luis D. Aranda Sánchez}
\fancyfoot[RO]{\thepage}
\renewcommand{\headrulewidth}{0.4pt}
\renewcommand{\footrulewidth}{0.4pt}

\fancypagestyle{myfancy}{
    \fancyhf{} % Clear all headers and footers
    \fancyhead[LE]{\nouppercase{\leftmark}}
    \fancyhead[RO]{Optimización energética para vivienda}
    \fancyfoot[LE]{\thepage}
    \fancyfoot[RE]{Escuela Técnica Superior de Ingenieros Industriales (UPM)}
    \fancyfoot[LO]{Luis D. Aranda Sánchez}
    \fancyfoot[RO]{\thepage}
    \renewcommand{\headrulewidth}{0.4pt}
    \renewcommand{\footrulewidth}{0.4pt}
}

\fancypagestyle{simple}{
    \fancyhf{} % Clear all headers and footers
    \renewcommand{\headrulewidth}{0pt}
    \renewcommand{\footrulewidth}{0pt}
}

% Line spacing
\setstretch{1.2}

% Document starts here
\begin{document}

% Portada
\begin{titlepage}
    \centering
    {\scshape\LARGE Universidad Politécnica de Madrid \par}
    \vspace{1cm}
    {\scshape\Large Escuela Técnica Superior de Ingenieros Industriales\par}
    \vspace{1.5cm}
    {\huge\bfseries Optimización energética de sistema híbrido con bomba de calor, suelo radiante, fotovoltaica y almacenamiento para vivienda \par}
    \vspace{1.5cm}
    {\Large\bfseries Trabajo de Fin de Máster\par}
    \vspace{0.5cm}
    {\large Máster Universitario en Ingeniería de la Energía \par}
    \vspace{2cm}
    {\Large Luis D. Aranda Sánchez\par}
    \vfill
    Director: Javier Rodríguez Martín
    \vfill
    {\large Septiembre 6, 2024\par}
\end{titlepage}

% Resumen (máximo de 5 páginas, incluyendo al final Palabras clave)
\clearpage
\pagestyle{simple}
% \newpage
\chapter*{Resumen}
\addcontentsline{toc}{chapter}{Resumen}
\input{capitulos/resumen/main.tex}

% Índice (paginado)
\clearpage
\pagestyle{simple}
% \newpage
\tableofcontents

% Introducción (donde se incluya los antecedentes y justificación)
\clearpage
\pagestyle{myfancy}
\newpage
\chapter{Introducción}
\input{capitulos/introduccion/main.tex}

% Objetivos
\chapter{Objetivos}
\input{capitulos/objetivos/main.tex}

% Metodología
\chapter{Metodología}
\input{capitulos/metodologia/main.tex}

% Resultados y discusión (incluyendo la valoración de impactos y de aspectos de responsabilidad legal, ética y profesional relacionados con el trabajo)
\chapter{Resultados y Discusión}
\input{capitulos/resultados_discusion/main.tex}

% Conclusiones
\chapter{Conclusiones}
\input{capitulos/conclusiones/main.tex}

% Planificación temporal y presupuesto
\chapter{Planificación Temporal y Presupuesto}
\input{capitulos/planificacion_presupuesto/main.tex}

% Bibliografía
\newpage
\addcontentsline{toc}{chapter}{Bibliografía}
\printbibliography

\end{document}


% Metodología
\chapter{Metodología}
\documentclass[a4paper,11pt,twoside]{report}
\usepackage[left=25mm,right=25mm,top=25mm,bottom=25mm,includehead,includefoot,headsep=15mm,footskip=15mm]{geometry}
\usepackage{graphicx}
\usepackage{fancyhdr}
\usepackage{titlesec}
\usepackage[spanish]{babel}
\usepackage[utf8]{inputenc}
\usepackage{amsmath}
\usepackage{setspace}
\usepackage{svg}
\usepackage{hyperref}
\usepackage[backend=biber,style=numeric]{biblatex}
\addbibresource{references.bib}
\hypersetup{
    colorlinks=true,
    linkcolor=blue,      % color of internal links (sections, etc.)
    urlcolor=blue,       % color of external links
    pdftitle={Optimización energética de sistema híbrido con bomba de calor, suelo radiante, fotovoltaica y almacenamiento para vivienda},    % title
    pdfauthor={Luis D. Aranda Sánchez},     % author
    pdfkeywords={palabra1, palabra2, código1, etc.} % list of keywords
}

% Font change to Arial
\usepackage{helvet}
\renewcommand{\familydefault}{\sfdefault}

% Chapter titles in uppercase and larger font
\titleformat{\chapter}[hang]{\large\bfseries}{\thechapter.}{1em}{\MakeUppercase}
\titleformat{\section}[hang]{\bfseries}{\thesection.}{1em}{}
\titleformat{\subsection}[hang]{\bfseries}{\thesubsection.}{1em}{}

% Fancyhdr setup
\setlength{\headheight}{14.30174pt} % Adjust to recommended value, slightly larger for safety
\fancyhf{} % Clear all headers and footers
\fancyhead[LE]{\nouppercase{\leftmark}}
\fancyhead[RO]{Optimización energética para vivienda}
\fancyfoot[LE]{\thepage}
\fancyfoot[RE]{Escuela Técnica Superior de Ingenieros Industriales (UPM)}
\fancyfoot[LO]{Luis D. Aranda Sánchez}
\fancyfoot[RO]{\thepage}
\renewcommand{\headrulewidth}{0.4pt}
\renewcommand{\footrulewidth}{0.4pt}

\fancypagestyle{myfancy}{
    \fancyhf{} % Clear all headers and footers
    \fancyhead[LE]{\nouppercase{\leftmark}}
    \fancyhead[RO]{Optimización energética para vivienda}
    \fancyfoot[LE]{\thepage}
    \fancyfoot[RE]{Escuela Técnica Superior de Ingenieros Industriales (UPM)}
    \fancyfoot[LO]{Luis D. Aranda Sánchez}
    \fancyfoot[RO]{\thepage}
    \renewcommand{\headrulewidth}{0.4pt}
    \renewcommand{\footrulewidth}{0.4pt}
}

\fancypagestyle{simple}{
    \fancyhf{} % Clear all headers and footers
    \renewcommand{\headrulewidth}{0pt}
    \renewcommand{\footrulewidth}{0pt}
}

% Line spacing
\setstretch{1.2}

% Document starts here
\begin{document}

% Portada
\begin{titlepage}
    \centering
    {\scshape\LARGE Universidad Politécnica de Madrid \par}
    \vspace{1cm}
    {\scshape\Large Escuela Técnica Superior de Ingenieros Industriales\par}
    \vspace{1.5cm}
    {\huge\bfseries Optimización energética de sistema híbrido con bomba de calor, suelo radiante, fotovoltaica y almacenamiento para vivienda \par}
    \vspace{1.5cm}
    {\Large\bfseries Trabajo de Fin de Máster\par}
    \vspace{0.5cm}
    {\large Máster Universitario en Ingeniería de la Energía \par}
    \vspace{2cm}
    {\Large Luis D. Aranda Sánchez\par}
    \vfill
    Director: Javier Rodríguez Martín
    \vfill
    {\large Septiembre 6, 2024\par}
\end{titlepage}

% Resumen (máximo de 5 páginas, incluyendo al final Palabras clave)
\clearpage
\pagestyle{simple}
% \newpage
\chapter*{Resumen}
\addcontentsline{toc}{chapter}{Resumen}
\input{capitulos/resumen/main.tex}

% Índice (paginado)
\clearpage
\pagestyle{simple}
% \newpage
\tableofcontents

% Introducción (donde se incluya los antecedentes y justificación)
\clearpage
\pagestyle{myfancy}
\newpage
\chapter{Introducción}
\input{capitulos/introduccion/main.tex}

% Objetivos
\chapter{Objetivos}
\input{capitulos/objetivos/main.tex}

% Metodología
\chapter{Metodología}
\input{capitulos/metodologia/main.tex}

% Resultados y discusión (incluyendo la valoración de impactos y de aspectos de responsabilidad legal, ética y profesional relacionados con el trabajo)
\chapter{Resultados y Discusión}
\input{capitulos/resultados_discusion/main.tex}

% Conclusiones
\chapter{Conclusiones}
\input{capitulos/conclusiones/main.tex}

% Planificación temporal y presupuesto
\chapter{Planificación Temporal y Presupuesto}
\input{capitulos/planificacion_presupuesto/main.tex}

% Bibliografía
\newpage
\addcontentsline{toc}{chapter}{Bibliografía}
\printbibliography

\end{document}


% Resultados y discusión (incluyendo la valoración de impactos y de aspectos de responsabilidad legal, ética y profesional relacionados con el trabajo)
\chapter{Resultados y Discusión}
\documentclass[a4paper,11pt,twoside]{report}
\usepackage[left=25mm,right=25mm,top=25mm,bottom=25mm,includehead,includefoot,headsep=15mm,footskip=15mm]{geometry}
\usepackage{graphicx}
\usepackage{fancyhdr}
\usepackage{titlesec}
\usepackage[spanish]{babel}
\usepackage[utf8]{inputenc}
\usepackage{amsmath}
\usepackage{setspace}
\usepackage{svg}
\usepackage{hyperref}
\usepackage[backend=biber,style=numeric]{biblatex}
\addbibresource{references.bib}
\hypersetup{
    colorlinks=true,
    linkcolor=blue,      % color of internal links (sections, etc.)
    urlcolor=blue,       % color of external links
    pdftitle={Optimización energética de sistema híbrido con bomba de calor, suelo radiante, fotovoltaica y almacenamiento para vivienda},    % title
    pdfauthor={Luis D. Aranda Sánchez},     % author
    pdfkeywords={palabra1, palabra2, código1, etc.} % list of keywords
}

% Font change to Arial
\usepackage{helvet}
\renewcommand{\familydefault}{\sfdefault}

% Chapter titles in uppercase and larger font
\titleformat{\chapter}[hang]{\large\bfseries}{\thechapter.}{1em}{\MakeUppercase}
\titleformat{\section}[hang]{\bfseries}{\thesection.}{1em}{}
\titleformat{\subsection}[hang]{\bfseries}{\thesubsection.}{1em}{}

% Fancyhdr setup
\setlength{\headheight}{14.30174pt} % Adjust to recommended value, slightly larger for safety
\fancyhf{} % Clear all headers and footers
\fancyhead[LE]{\nouppercase{\leftmark}}
\fancyhead[RO]{Optimización energética para vivienda}
\fancyfoot[LE]{\thepage}
\fancyfoot[RE]{Escuela Técnica Superior de Ingenieros Industriales (UPM)}
\fancyfoot[LO]{Luis D. Aranda Sánchez}
\fancyfoot[RO]{\thepage}
\renewcommand{\headrulewidth}{0.4pt}
\renewcommand{\footrulewidth}{0.4pt}

\fancypagestyle{myfancy}{
    \fancyhf{} % Clear all headers and footers
    \fancyhead[LE]{\nouppercase{\leftmark}}
    \fancyhead[RO]{Optimización energética para vivienda}
    \fancyfoot[LE]{\thepage}
    \fancyfoot[RE]{Escuela Técnica Superior de Ingenieros Industriales (UPM)}
    \fancyfoot[LO]{Luis D. Aranda Sánchez}
    \fancyfoot[RO]{\thepage}
    \renewcommand{\headrulewidth}{0.4pt}
    \renewcommand{\footrulewidth}{0.4pt}
}

\fancypagestyle{simple}{
    \fancyhf{} % Clear all headers and footers
    \renewcommand{\headrulewidth}{0pt}
    \renewcommand{\footrulewidth}{0pt}
}

% Line spacing
\setstretch{1.2}

% Document starts here
\begin{document}

% Portada
\begin{titlepage}
    \centering
    {\scshape\LARGE Universidad Politécnica de Madrid \par}
    \vspace{1cm}
    {\scshape\Large Escuela Técnica Superior de Ingenieros Industriales\par}
    \vspace{1.5cm}
    {\huge\bfseries Optimización energética de sistema híbrido con bomba de calor, suelo radiante, fotovoltaica y almacenamiento para vivienda \par}
    \vspace{1.5cm}
    {\Large\bfseries Trabajo de Fin de Máster\par}
    \vspace{0.5cm}
    {\large Máster Universitario en Ingeniería de la Energía \par}
    \vspace{2cm}
    {\Large Luis D. Aranda Sánchez\par}
    \vfill
    Director: Javier Rodríguez Martín
    \vfill
    {\large Septiembre 6, 2024\par}
\end{titlepage}

% Resumen (máximo de 5 páginas, incluyendo al final Palabras clave)
\clearpage
\pagestyle{simple}
% \newpage
\chapter*{Resumen}
\addcontentsline{toc}{chapter}{Resumen}
\input{capitulos/resumen/main.tex}

% Índice (paginado)
\clearpage
\pagestyle{simple}
% \newpage
\tableofcontents

% Introducción (donde se incluya los antecedentes y justificación)
\clearpage
\pagestyle{myfancy}
\newpage
\chapter{Introducción}
\input{capitulos/introduccion/main.tex}

% Objetivos
\chapter{Objetivos}
\input{capitulos/objetivos/main.tex}

% Metodología
\chapter{Metodología}
\input{capitulos/metodologia/main.tex}

% Resultados y discusión (incluyendo la valoración de impactos y de aspectos de responsabilidad legal, ética y profesional relacionados con el trabajo)
\chapter{Resultados y Discusión}
\input{capitulos/resultados_discusion/main.tex}

% Conclusiones
\chapter{Conclusiones}
\input{capitulos/conclusiones/main.tex}

% Planificación temporal y presupuesto
\chapter{Planificación Temporal y Presupuesto}
\input{capitulos/planificacion_presupuesto/main.tex}

% Bibliografía
\newpage
\addcontentsline{toc}{chapter}{Bibliografía}
\printbibliography

\end{document}


% Conclusiones
\chapter{Conclusiones}
\documentclass[a4paper,11pt,twoside]{report}
\usepackage[left=25mm,right=25mm,top=25mm,bottom=25mm,includehead,includefoot,headsep=15mm,footskip=15mm]{geometry}
\usepackage{graphicx}
\usepackage{fancyhdr}
\usepackage{titlesec}
\usepackage[spanish]{babel}
\usepackage[utf8]{inputenc}
\usepackage{amsmath}
\usepackage{setspace}
\usepackage{svg}
\usepackage{hyperref}
\usepackage[backend=biber,style=numeric]{biblatex}
\addbibresource{references.bib}
\hypersetup{
    colorlinks=true,
    linkcolor=blue,      % color of internal links (sections, etc.)
    urlcolor=blue,       % color of external links
    pdftitle={Optimización energética de sistema híbrido con bomba de calor, suelo radiante, fotovoltaica y almacenamiento para vivienda},    % title
    pdfauthor={Luis D. Aranda Sánchez},     % author
    pdfkeywords={palabra1, palabra2, código1, etc.} % list of keywords
}

% Font change to Arial
\usepackage{helvet}
\renewcommand{\familydefault}{\sfdefault}

% Chapter titles in uppercase and larger font
\titleformat{\chapter}[hang]{\large\bfseries}{\thechapter.}{1em}{\MakeUppercase}
\titleformat{\section}[hang]{\bfseries}{\thesection.}{1em}{}
\titleformat{\subsection}[hang]{\bfseries}{\thesubsection.}{1em}{}

% Fancyhdr setup
\setlength{\headheight}{14.30174pt} % Adjust to recommended value, slightly larger for safety
\fancyhf{} % Clear all headers and footers
\fancyhead[LE]{\nouppercase{\leftmark}}
\fancyhead[RO]{Optimización energética para vivienda}
\fancyfoot[LE]{\thepage}
\fancyfoot[RE]{Escuela Técnica Superior de Ingenieros Industriales (UPM)}
\fancyfoot[LO]{Luis D. Aranda Sánchez}
\fancyfoot[RO]{\thepage}
\renewcommand{\headrulewidth}{0.4pt}
\renewcommand{\footrulewidth}{0.4pt}

\fancypagestyle{myfancy}{
    \fancyhf{} % Clear all headers and footers
    \fancyhead[LE]{\nouppercase{\leftmark}}
    \fancyhead[RO]{Optimización energética para vivienda}
    \fancyfoot[LE]{\thepage}
    \fancyfoot[RE]{Escuela Técnica Superior de Ingenieros Industriales (UPM)}
    \fancyfoot[LO]{Luis D. Aranda Sánchez}
    \fancyfoot[RO]{\thepage}
    \renewcommand{\headrulewidth}{0.4pt}
    \renewcommand{\footrulewidth}{0.4pt}
}

\fancypagestyle{simple}{
    \fancyhf{} % Clear all headers and footers
    \renewcommand{\headrulewidth}{0pt}
    \renewcommand{\footrulewidth}{0pt}
}

% Line spacing
\setstretch{1.2}

% Document starts here
\begin{document}

% Portada
\begin{titlepage}
    \centering
    {\scshape\LARGE Universidad Politécnica de Madrid \par}
    \vspace{1cm}
    {\scshape\Large Escuela Técnica Superior de Ingenieros Industriales\par}
    \vspace{1.5cm}
    {\huge\bfseries Optimización energética de sistema híbrido con bomba de calor, suelo radiante, fotovoltaica y almacenamiento para vivienda \par}
    \vspace{1.5cm}
    {\Large\bfseries Trabajo de Fin de Máster\par}
    \vspace{0.5cm}
    {\large Máster Universitario en Ingeniería de la Energía \par}
    \vspace{2cm}
    {\Large Luis D. Aranda Sánchez\par}
    \vfill
    Director: Javier Rodríguez Martín
    \vfill
    {\large Septiembre 6, 2024\par}
\end{titlepage}

% Resumen (máximo de 5 páginas, incluyendo al final Palabras clave)
\clearpage
\pagestyle{simple}
% \newpage
\chapter*{Resumen}
\addcontentsline{toc}{chapter}{Resumen}
\input{capitulos/resumen/main.tex}

% Índice (paginado)
\clearpage
\pagestyle{simple}
% \newpage
\tableofcontents

% Introducción (donde se incluya los antecedentes y justificación)
\clearpage
\pagestyle{myfancy}
\newpage
\chapter{Introducción}
\input{capitulos/introduccion/main.tex}

% Objetivos
\chapter{Objetivos}
\input{capitulos/objetivos/main.tex}

% Metodología
\chapter{Metodología}
\input{capitulos/metodologia/main.tex}

% Resultados y discusión (incluyendo la valoración de impactos y de aspectos de responsabilidad legal, ética y profesional relacionados con el trabajo)
\chapter{Resultados y Discusión}
\input{capitulos/resultados_discusion/main.tex}

% Conclusiones
\chapter{Conclusiones}
\input{capitulos/conclusiones/main.tex}

% Planificación temporal y presupuesto
\chapter{Planificación Temporal y Presupuesto}
\input{capitulos/planificacion_presupuesto/main.tex}

% Bibliografía
\newpage
\addcontentsline{toc}{chapter}{Bibliografía}
\printbibliography

\end{document}


% Planificación temporal y presupuesto
\chapter{Planificación Temporal y Presupuesto}
\documentclass[a4paper,11pt,twoside]{report}
\usepackage[left=25mm,right=25mm,top=25mm,bottom=25mm,includehead,includefoot,headsep=15mm,footskip=15mm]{geometry}
\usepackage{graphicx}
\usepackage{fancyhdr}
\usepackage{titlesec}
\usepackage[spanish]{babel}
\usepackage[utf8]{inputenc}
\usepackage{amsmath}
\usepackage{setspace}
\usepackage{svg}
\usepackage{hyperref}
\usepackage[backend=biber,style=numeric]{biblatex}
\addbibresource{references.bib}
\hypersetup{
    colorlinks=true,
    linkcolor=blue,      % color of internal links (sections, etc.)
    urlcolor=blue,       % color of external links
    pdftitle={Optimización energética de sistema híbrido con bomba de calor, suelo radiante, fotovoltaica y almacenamiento para vivienda},    % title
    pdfauthor={Luis D. Aranda Sánchez},     % author
    pdfkeywords={palabra1, palabra2, código1, etc.} % list of keywords
}

% Font change to Arial
\usepackage{helvet}
\renewcommand{\familydefault}{\sfdefault}

% Chapter titles in uppercase and larger font
\titleformat{\chapter}[hang]{\large\bfseries}{\thechapter.}{1em}{\MakeUppercase}
\titleformat{\section}[hang]{\bfseries}{\thesection.}{1em}{}
\titleformat{\subsection}[hang]{\bfseries}{\thesubsection.}{1em}{}

% Fancyhdr setup
\setlength{\headheight}{14.30174pt} % Adjust to recommended value, slightly larger for safety
\fancyhf{} % Clear all headers and footers
\fancyhead[LE]{\nouppercase{\leftmark}}
\fancyhead[RO]{Optimización energética para vivienda}
\fancyfoot[LE]{\thepage}
\fancyfoot[RE]{Escuela Técnica Superior de Ingenieros Industriales (UPM)}
\fancyfoot[LO]{Luis D. Aranda Sánchez}
\fancyfoot[RO]{\thepage}
\renewcommand{\headrulewidth}{0.4pt}
\renewcommand{\footrulewidth}{0.4pt}

\fancypagestyle{myfancy}{
    \fancyhf{} % Clear all headers and footers
    \fancyhead[LE]{\nouppercase{\leftmark}}
    \fancyhead[RO]{Optimización energética para vivienda}
    \fancyfoot[LE]{\thepage}
    \fancyfoot[RE]{Escuela Técnica Superior de Ingenieros Industriales (UPM)}
    \fancyfoot[LO]{Luis D. Aranda Sánchez}
    \fancyfoot[RO]{\thepage}
    \renewcommand{\headrulewidth}{0.4pt}
    \renewcommand{\footrulewidth}{0.4pt}
}

\fancypagestyle{simple}{
    \fancyhf{} % Clear all headers and footers
    \renewcommand{\headrulewidth}{0pt}
    \renewcommand{\footrulewidth}{0pt}
}

% Line spacing
\setstretch{1.2}

% Document starts here
\begin{document}

% Portada
\begin{titlepage}
    \centering
    {\scshape\LARGE Universidad Politécnica de Madrid \par}
    \vspace{1cm}
    {\scshape\Large Escuela Técnica Superior de Ingenieros Industriales\par}
    \vspace{1.5cm}
    {\huge\bfseries Optimización energética de sistema híbrido con bomba de calor, suelo radiante, fotovoltaica y almacenamiento para vivienda \par}
    \vspace{1.5cm}
    {\Large\bfseries Trabajo de Fin de Máster\par}
    \vspace{0.5cm}
    {\large Máster Universitario en Ingeniería de la Energía \par}
    \vspace{2cm}
    {\Large Luis D. Aranda Sánchez\par}
    \vfill
    Director: Javier Rodríguez Martín
    \vfill
    {\large Septiembre 6, 2024\par}
\end{titlepage}

% Resumen (máximo de 5 páginas, incluyendo al final Palabras clave)
\clearpage
\pagestyle{simple}
% \newpage
\chapter*{Resumen}
\addcontentsline{toc}{chapter}{Resumen}
\input{capitulos/resumen/main.tex}

% Índice (paginado)
\clearpage
\pagestyle{simple}
% \newpage
\tableofcontents

% Introducción (donde se incluya los antecedentes y justificación)
\clearpage
\pagestyle{myfancy}
\newpage
\chapter{Introducción}
\input{capitulos/introduccion/main.tex}

% Objetivos
\chapter{Objetivos}
\input{capitulos/objetivos/main.tex}

% Metodología
\chapter{Metodología}
\input{capitulos/metodologia/main.tex}

% Resultados y discusión (incluyendo la valoración de impactos y de aspectos de responsabilidad legal, ética y profesional relacionados con el trabajo)
\chapter{Resultados y Discusión}
\input{capitulos/resultados_discusion/main.tex}

% Conclusiones
\chapter{Conclusiones}
\input{capitulos/conclusiones/main.tex}

% Planificación temporal y presupuesto
\chapter{Planificación Temporal y Presupuesto}
\input{capitulos/planificacion_presupuesto/main.tex}

% Bibliografía
\newpage
\addcontentsline{toc}{chapter}{Bibliografía}
\printbibliography

\end{document}


% Bibliografía
\newpage
\addcontentsline{toc}{chapter}{Bibliografía}
\printbibliography

\end{document}


% Metodología
\chapter{Metodología}
\documentclass[a4paper,11pt,twoside]{report}
\usepackage[left=25mm,right=25mm,top=25mm,bottom=25mm,includehead,includefoot,headsep=15mm,footskip=15mm]{geometry}
\usepackage{graphicx}
\usepackage{fancyhdr}
\usepackage{titlesec}
\usepackage[spanish]{babel}
\usepackage[utf8]{inputenc}
\usepackage{amsmath}
\usepackage{setspace}
\usepackage{svg}
\usepackage{hyperref}
\usepackage[backend=biber,style=numeric]{biblatex}
\addbibresource{references.bib}
\hypersetup{
    colorlinks=true,
    linkcolor=blue,      % color of internal links (sections, etc.)
    urlcolor=blue,       % color of external links
    pdftitle={Optimización energética de sistema híbrido con bomba de calor, suelo radiante, fotovoltaica y almacenamiento para vivienda},    % title
    pdfauthor={Luis D. Aranda Sánchez},     % author
    pdfkeywords={palabra1, palabra2, código1, etc.} % list of keywords
}

% Font change to Arial
\usepackage{helvet}
\renewcommand{\familydefault}{\sfdefault}

% Chapter titles in uppercase and larger font
\titleformat{\chapter}[hang]{\large\bfseries}{\thechapter.}{1em}{\MakeUppercase}
\titleformat{\section}[hang]{\bfseries}{\thesection.}{1em}{}
\titleformat{\subsection}[hang]{\bfseries}{\thesubsection.}{1em}{}

% Fancyhdr setup
\setlength{\headheight}{14.30174pt} % Adjust to recommended value, slightly larger for safety
\fancyhf{} % Clear all headers and footers
\fancyhead[LE]{\nouppercase{\leftmark}}
\fancyhead[RO]{Optimización energética para vivienda}
\fancyfoot[LE]{\thepage}
\fancyfoot[RE]{Escuela Técnica Superior de Ingenieros Industriales (UPM)}
\fancyfoot[LO]{Luis D. Aranda Sánchez}
\fancyfoot[RO]{\thepage}
\renewcommand{\headrulewidth}{0.4pt}
\renewcommand{\footrulewidth}{0.4pt}

\fancypagestyle{myfancy}{
    \fancyhf{} % Clear all headers and footers
    \fancyhead[LE]{\nouppercase{\leftmark}}
    \fancyhead[RO]{Optimización energética para vivienda}
    \fancyfoot[LE]{\thepage}
    \fancyfoot[RE]{Escuela Técnica Superior de Ingenieros Industriales (UPM)}
    \fancyfoot[LO]{Luis D. Aranda Sánchez}
    \fancyfoot[RO]{\thepage}
    \renewcommand{\headrulewidth}{0.4pt}
    \renewcommand{\footrulewidth}{0.4pt}
}

\fancypagestyle{simple}{
    \fancyhf{} % Clear all headers and footers
    \renewcommand{\headrulewidth}{0pt}
    \renewcommand{\footrulewidth}{0pt}
}

% Line spacing
\setstretch{1.2}

% Document starts here
\begin{document}

% Portada
\begin{titlepage}
    \centering
    {\scshape\LARGE Universidad Politécnica de Madrid \par}
    \vspace{1cm}
    {\scshape\Large Escuela Técnica Superior de Ingenieros Industriales\par}
    \vspace{1.5cm}
    {\huge\bfseries Optimización energética de sistema híbrido con bomba de calor, suelo radiante, fotovoltaica y almacenamiento para vivienda \par}
    \vspace{1.5cm}
    {\Large\bfseries Trabajo de Fin de Máster\par}
    \vspace{0.5cm}
    {\large Máster Universitario en Ingeniería de la Energía \par}
    \vspace{2cm}
    {\Large Luis D. Aranda Sánchez\par}
    \vfill
    Director: Javier Rodríguez Martín
    \vfill
    {\large Septiembre 6, 2024\par}
\end{titlepage}

% Resumen (máximo de 5 páginas, incluyendo al final Palabras clave)
\clearpage
\pagestyle{simple}
% \newpage
\chapter*{Resumen}
\addcontentsline{toc}{chapter}{Resumen}
\documentclass[a4paper,11pt,twoside]{report}
\usepackage[left=25mm,right=25mm,top=25mm,bottom=25mm,includehead,includefoot,headsep=15mm,footskip=15mm]{geometry}
\usepackage{graphicx}
\usepackage{fancyhdr}
\usepackage{titlesec}
\usepackage[spanish]{babel}
\usepackage[utf8]{inputenc}
\usepackage{amsmath}
\usepackage{setspace}
\usepackage{svg}
\usepackage{hyperref}
\usepackage[backend=biber,style=numeric]{biblatex}
\addbibresource{references.bib}
\hypersetup{
    colorlinks=true,
    linkcolor=blue,      % color of internal links (sections, etc.)
    urlcolor=blue,       % color of external links
    pdftitle={Optimización energética de sistema híbrido con bomba de calor, suelo radiante, fotovoltaica y almacenamiento para vivienda},    % title
    pdfauthor={Luis D. Aranda Sánchez},     % author
    pdfkeywords={palabra1, palabra2, código1, etc.} % list of keywords
}

% Font change to Arial
\usepackage{helvet}
\renewcommand{\familydefault}{\sfdefault}

% Chapter titles in uppercase and larger font
\titleformat{\chapter}[hang]{\large\bfseries}{\thechapter.}{1em}{\MakeUppercase}
\titleformat{\section}[hang]{\bfseries}{\thesection.}{1em}{}
\titleformat{\subsection}[hang]{\bfseries}{\thesubsection.}{1em}{}

% Fancyhdr setup
\setlength{\headheight}{14.30174pt} % Adjust to recommended value, slightly larger for safety
\fancyhf{} % Clear all headers and footers
\fancyhead[LE]{\nouppercase{\leftmark}}
\fancyhead[RO]{Optimización energética para vivienda}
\fancyfoot[LE]{\thepage}
\fancyfoot[RE]{Escuela Técnica Superior de Ingenieros Industriales (UPM)}
\fancyfoot[LO]{Luis D. Aranda Sánchez}
\fancyfoot[RO]{\thepage}
\renewcommand{\headrulewidth}{0.4pt}
\renewcommand{\footrulewidth}{0.4pt}

\fancypagestyle{myfancy}{
    \fancyhf{} % Clear all headers and footers
    \fancyhead[LE]{\nouppercase{\leftmark}}
    \fancyhead[RO]{Optimización energética para vivienda}
    \fancyfoot[LE]{\thepage}
    \fancyfoot[RE]{Escuela Técnica Superior de Ingenieros Industriales (UPM)}
    \fancyfoot[LO]{Luis D. Aranda Sánchez}
    \fancyfoot[RO]{\thepage}
    \renewcommand{\headrulewidth}{0.4pt}
    \renewcommand{\footrulewidth}{0.4pt}
}

\fancypagestyle{simple}{
    \fancyhf{} % Clear all headers and footers
    \renewcommand{\headrulewidth}{0pt}
    \renewcommand{\footrulewidth}{0pt}
}

% Line spacing
\setstretch{1.2}

% Document starts here
\begin{document}

% Portada
\begin{titlepage}
    \centering
    {\scshape\LARGE Universidad Politécnica de Madrid \par}
    \vspace{1cm}
    {\scshape\Large Escuela Técnica Superior de Ingenieros Industriales\par}
    \vspace{1.5cm}
    {\huge\bfseries Optimización energética de sistema híbrido con bomba de calor, suelo radiante, fotovoltaica y almacenamiento para vivienda \par}
    \vspace{1.5cm}
    {\Large\bfseries Trabajo de Fin de Máster\par}
    \vspace{0.5cm}
    {\large Máster Universitario en Ingeniería de la Energía \par}
    \vspace{2cm}
    {\Large Luis D. Aranda Sánchez\par}
    \vfill
    Director: Javier Rodríguez Martín
    \vfill
    {\large Septiembre 6, 2024\par}
\end{titlepage}

% Resumen (máximo de 5 páginas, incluyendo al final Palabras clave)
\clearpage
\pagestyle{simple}
% \newpage
\chapter*{Resumen}
\addcontentsline{toc}{chapter}{Resumen}
\input{capitulos/resumen/main.tex}

% Índice (paginado)
\clearpage
\pagestyle{simple}
% \newpage
\tableofcontents

% Introducción (donde se incluya los antecedentes y justificación)
\clearpage
\pagestyle{myfancy}
\newpage
\chapter{Introducción}
\input{capitulos/introduccion/main.tex}

% Objetivos
\chapter{Objetivos}
\input{capitulos/objetivos/main.tex}

% Metodología
\chapter{Metodología}
\input{capitulos/metodologia/main.tex}

% Resultados y discusión (incluyendo la valoración de impactos y de aspectos de responsabilidad legal, ética y profesional relacionados con el trabajo)
\chapter{Resultados y Discusión}
\input{capitulos/resultados_discusion/main.tex}

% Conclusiones
\chapter{Conclusiones}
\input{capitulos/conclusiones/main.tex}

% Planificación temporal y presupuesto
\chapter{Planificación Temporal y Presupuesto}
\input{capitulos/planificacion_presupuesto/main.tex}

% Bibliografía
\newpage
\addcontentsline{toc}{chapter}{Bibliografía}
\printbibliography

\end{document}


% Índice (paginado)
\clearpage
\pagestyle{simple}
% \newpage
\tableofcontents

% Introducción (donde se incluya los antecedentes y justificación)
\clearpage
\pagestyle{myfancy}
\newpage
\chapter{Introducción}
\documentclass[a4paper,11pt,twoside]{report}
\usepackage[left=25mm,right=25mm,top=25mm,bottom=25mm,includehead,includefoot,headsep=15mm,footskip=15mm]{geometry}
\usepackage{graphicx}
\usepackage{fancyhdr}
\usepackage{titlesec}
\usepackage[spanish]{babel}
\usepackage[utf8]{inputenc}
\usepackage{amsmath}
\usepackage{setspace}
\usepackage{svg}
\usepackage{hyperref}
\usepackage[backend=biber,style=numeric]{biblatex}
\addbibresource{references.bib}
\hypersetup{
    colorlinks=true,
    linkcolor=blue,      % color of internal links (sections, etc.)
    urlcolor=blue,       % color of external links
    pdftitle={Optimización energética de sistema híbrido con bomba de calor, suelo radiante, fotovoltaica y almacenamiento para vivienda},    % title
    pdfauthor={Luis D. Aranda Sánchez},     % author
    pdfkeywords={palabra1, palabra2, código1, etc.} % list of keywords
}

% Font change to Arial
\usepackage{helvet}
\renewcommand{\familydefault}{\sfdefault}

% Chapter titles in uppercase and larger font
\titleformat{\chapter}[hang]{\large\bfseries}{\thechapter.}{1em}{\MakeUppercase}
\titleformat{\section}[hang]{\bfseries}{\thesection.}{1em}{}
\titleformat{\subsection}[hang]{\bfseries}{\thesubsection.}{1em}{}

% Fancyhdr setup
\setlength{\headheight}{14.30174pt} % Adjust to recommended value, slightly larger for safety
\fancyhf{} % Clear all headers and footers
\fancyhead[LE]{\nouppercase{\leftmark}}
\fancyhead[RO]{Optimización energética para vivienda}
\fancyfoot[LE]{\thepage}
\fancyfoot[RE]{Escuela Técnica Superior de Ingenieros Industriales (UPM)}
\fancyfoot[LO]{Luis D. Aranda Sánchez}
\fancyfoot[RO]{\thepage}
\renewcommand{\headrulewidth}{0.4pt}
\renewcommand{\footrulewidth}{0.4pt}

\fancypagestyle{myfancy}{
    \fancyhf{} % Clear all headers and footers
    \fancyhead[LE]{\nouppercase{\leftmark}}
    \fancyhead[RO]{Optimización energética para vivienda}
    \fancyfoot[LE]{\thepage}
    \fancyfoot[RE]{Escuela Técnica Superior de Ingenieros Industriales (UPM)}
    \fancyfoot[LO]{Luis D. Aranda Sánchez}
    \fancyfoot[RO]{\thepage}
    \renewcommand{\headrulewidth}{0.4pt}
    \renewcommand{\footrulewidth}{0.4pt}
}

\fancypagestyle{simple}{
    \fancyhf{} % Clear all headers and footers
    \renewcommand{\headrulewidth}{0pt}
    \renewcommand{\footrulewidth}{0pt}
}

% Line spacing
\setstretch{1.2}

% Document starts here
\begin{document}

% Portada
\begin{titlepage}
    \centering
    {\scshape\LARGE Universidad Politécnica de Madrid \par}
    \vspace{1cm}
    {\scshape\Large Escuela Técnica Superior de Ingenieros Industriales\par}
    \vspace{1.5cm}
    {\huge\bfseries Optimización energética de sistema híbrido con bomba de calor, suelo radiante, fotovoltaica y almacenamiento para vivienda \par}
    \vspace{1.5cm}
    {\Large\bfseries Trabajo de Fin de Máster\par}
    \vspace{0.5cm}
    {\large Máster Universitario en Ingeniería de la Energía \par}
    \vspace{2cm}
    {\Large Luis D. Aranda Sánchez\par}
    \vfill
    Director: Javier Rodríguez Martín
    \vfill
    {\large Septiembre 6, 2024\par}
\end{titlepage}

% Resumen (máximo de 5 páginas, incluyendo al final Palabras clave)
\clearpage
\pagestyle{simple}
% \newpage
\chapter*{Resumen}
\addcontentsline{toc}{chapter}{Resumen}
\input{capitulos/resumen/main.tex}

% Índice (paginado)
\clearpage
\pagestyle{simple}
% \newpage
\tableofcontents

% Introducción (donde se incluya los antecedentes y justificación)
\clearpage
\pagestyle{myfancy}
\newpage
\chapter{Introducción}
\input{capitulos/introduccion/main.tex}

% Objetivos
\chapter{Objetivos}
\input{capitulos/objetivos/main.tex}

% Metodología
\chapter{Metodología}
\input{capitulos/metodologia/main.tex}

% Resultados y discusión (incluyendo la valoración de impactos y de aspectos de responsabilidad legal, ética y profesional relacionados con el trabajo)
\chapter{Resultados y Discusión}
\input{capitulos/resultados_discusion/main.tex}

% Conclusiones
\chapter{Conclusiones}
\input{capitulos/conclusiones/main.tex}

% Planificación temporal y presupuesto
\chapter{Planificación Temporal y Presupuesto}
\input{capitulos/planificacion_presupuesto/main.tex}

% Bibliografía
\newpage
\addcontentsline{toc}{chapter}{Bibliografía}
\printbibliography

\end{document}


% Objetivos
\chapter{Objetivos}
\documentclass[a4paper,11pt,twoside]{report}
\usepackage[left=25mm,right=25mm,top=25mm,bottom=25mm,includehead,includefoot,headsep=15mm,footskip=15mm]{geometry}
\usepackage{graphicx}
\usepackage{fancyhdr}
\usepackage{titlesec}
\usepackage[spanish]{babel}
\usepackage[utf8]{inputenc}
\usepackage{amsmath}
\usepackage{setspace}
\usepackage{svg}
\usepackage{hyperref}
\usepackage[backend=biber,style=numeric]{biblatex}
\addbibresource{references.bib}
\hypersetup{
    colorlinks=true,
    linkcolor=blue,      % color of internal links (sections, etc.)
    urlcolor=blue,       % color of external links
    pdftitle={Optimización energética de sistema híbrido con bomba de calor, suelo radiante, fotovoltaica y almacenamiento para vivienda},    % title
    pdfauthor={Luis D. Aranda Sánchez},     % author
    pdfkeywords={palabra1, palabra2, código1, etc.} % list of keywords
}

% Font change to Arial
\usepackage{helvet}
\renewcommand{\familydefault}{\sfdefault}

% Chapter titles in uppercase and larger font
\titleformat{\chapter}[hang]{\large\bfseries}{\thechapter.}{1em}{\MakeUppercase}
\titleformat{\section}[hang]{\bfseries}{\thesection.}{1em}{}
\titleformat{\subsection}[hang]{\bfseries}{\thesubsection.}{1em}{}

% Fancyhdr setup
\setlength{\headheight}{14.30174pt} % Adjust to recommended value, slightly larger for safety
\fancyhf{} % Clear all headers and footers
\fancyhead[LE]{\nouppercase{\leftmark}}
\fancyhead[RO]{Optimización energética para vivienda}
\fancyfoot[LE]{\thepage}
\fancyfoot[RE]{Escuela Técnica Superior de Ingenieros Industriales (UPM)}
\fancyfoot[LO]{Luis D. Aranda Sánchez}
\fancyfoot[RO]{\thepage}
\renewcommand{\headrulewidth}{0.4pt}
\renewcommand{\footrulewidth}{0.4pt}

\fancypagestyle{myfancy}{
    \fancyhf{} % Clear all headers and footers
    \fancyhead[LE]{\nouppercase{\leftmark}}
    \fancyhead[RO]{Optimización energética para vivienda}
    \fancyfoot[LE]{\thepage}
    \fancyfoot[RE]{Escuela Técnica Superior de Ingenieros Industriales (UPM)}
    \fancyfoot[LO]{Luis D. Aranda Sánchez}
    \fancyfoot[RO]{\thepage}
    \renewcommand{\headrulewidth}{0.4pt}
    \renewcommand{\footrulewidth}{0.4pt}
}

\fancypagestyle{simple}{
    \fancyhf{} % Clear all headers and footers
    \renewcommand{\headrulewidth}{0pt}
    \renewcommand{\footrulewidth}{0pt}
}

% Line spacing
\setstretch{1.2}

% Document starts here
\begin{document}

% Portada
\begin{titlepage}
    \centering
    {\scshape\LARGE Universidad Politécnica de Madrid \par}
    \vspace{1cm}
    {\scshape\Large Escuela Técnica Superior de Ingenieros Industriales\par}
    \vspace{1.5cm}
    {\huge\bfseries Optimización energética de sistema híbrido con bomba de calor, suelo radiante, fotovoltaica y almacenamiento para vivienda \par}
    \vspace{1.5cm}
    {\Large\bfseries Trabajo de Fin de Máster\par}
    \vspace{0.5cm}
    {\large Máster Universitario en Ingeniería de la Energía \par}
    \vspace{2cm}
    {\Large Luis D. Aranda Sánchez\par}
    \vfill
    Director: Javier Rodríguez Martín
    \vfill
    {\large Septiembre 6, 2024\par}
\end{titlepage}

% Resumen (máximo de 5 páginas, incluyendo al final Palabras clave)
\clearpage
\pagestyle{simple}
% \newpage
\chapter*{Resumen}
\addcontentsline{toc}{chapter}{Resumen}
\input{capitulos/resumen/main.tex}

% Índice (paginado)
\clearpage
\pagestyle{simple}
% \newpage
\tableofcontents

% Introducción (donde se incluya los antecedentes y justificación)
\clearpage
\pagestyle{myfancy}
\newpage
\chapter{Introducción}
\input{capitulos/introduccion/main.tex}

% Objetivos
\chapter{Objetivos}
\input{capitulos/objetivos/main.tex}

% Metodología
\chapter{Metodología}
\input{capitulos/metodologia/main.tex}

% Resultados y discusión (incluyendo la valoración de impactos y de aspectos de responsabilidad legal, ética y profesional relacionados con el trabajo)
\chapter{Resultados y Discusión}
\input{capitulos/resultados_discusion/main.tex}

% Conclusiones
\chapter{Conclusiones}
\input{capitulos/conclusiones/main.tex}

% Planificación temporal y presupuesto
\chapter{Planificación Temporal y Presupuesto}
\input{capitulos/planificacion_presupuesto/main.tex}

% Bibliografía
\newpage
\addcontentsline{toc}{chapter}{Bibliografía}
\printbibliography

\end{document}


% Metodología
\chapter{Metodología}
\documentclass[a4paper,11pt,twoside]{report}
\usepackage[left=25mm,right=25mm,top=25mm,bottom=25mm,includehead,includefoot,headsep=15mm,footskip=15mm]{geometry}
\usepackage{graphicx}
\usepackage{fancyhdr}
\usepackage{titlesec}
\usepackage[spanish]{babel}
\usepackage[utf8]{inputenc}
\usepackage{amsmath}
\usepackage{setspace}
\usepackage{svg}
\usepackage{hyperref}
\usepackage[backend=biber,style=numeric]{biblatex}
\addbibresource{references.bib}
\hypersetup{
    colorlinks=true,
    linkcolor=blue,      % color of internal links (sections, etc.)
    urlcolor=blue,       % color of external links
    pdftitle={Optimización energética de sistema híbrido con bomba de calor, suelo radiante, fotovoltaica y almacenamiento para vivienda},    % title
    pdfauthor={Luis D. Aranda Sánchez},     % author
    pdfkeywords={palabra1, palabra2, código1, etc.} % list of keywords
}

% Font change to Arial
\usepackage{helvet}
\renewcommand{\familydefault}{\sfdefault}

% Chapter titles in uppercase and larger font
\titleformat{\chapter}[hang]{\large\bfseries}{\thechapter.}{1em}{\MakeUppercase}
\titleformat{\section}[hang]{\bfseries}{\thesection.}{1em}{}
\titleformat{\subsection}[hang]{\bfseries}{\thesubsection.}{1em}{}

% Fancyhdr setup
\setlength{\headheight}{14.30174pt} % Adjust to recommended value, slightly larger for safety
\fancyhf{} % Clear all headers and footers
\fancyhead[LE]{\nouppercase{\leftmark}}
\fancyhead[RO]{Optimización energética para vivienda}
\fancyfoot[LE]{\thepage}
\fancyfoot[RE]{Escuela Técnica Superior de Ingenieros Industriales (UPM)}
\fancyfoot[LO]{Luis D. Aranda Sánchez}
\fancyfoot[RO]{\thepage}
\renewcommand{\headrulewidth}{0.4pt}
\renewcommand{\footrulewidth}{0.4pt}

\fancypagestyle{myfancy}{
    \fancyhf{} % Clear all headers and footers
    \fancyhead[LE]{\nouppercase{\leftmark}}
    \fancyhead[RO]{Optimización energética para vivienda}
    \fancyfoot[LE]{\thepage}
    \fancyfoot[RE]{Escuela Técnica Superior de Ingenieros Industriales (UPM)}
    \fancyfoot[LO]{Luis D. Aranda Sánchez}
    \fancyfoot[RO]{\thepage}
    \renewcommand{\headrulewidth}{0.4pt}
    \renewcommand{\footrulewidth}{0.4pt}
}

\fancypagestyle{simple}{
    \fancyhf{} % Clear all headers and footers
    \renewcommand{\headrulewidth}{0pt}
    \renewcommand{\footrulewidth}{0pt}
}

% Line spacing
\setstretch{1.2}

% Document starts here
\begin{document}

% Portada
\begin{titlepage}
    \centering
    {\scshape\LARGE Universidad Politécnica de Madrid \par}
    \vspace{1cm}
    {\scshape\Large Escuela Técnica Superior de Ingenieros Industriales\par}
    \vspace{1.5cm}
    {\huge\bfseries Optimización energética de sistema híbrido con bomba de calor, suelo radiante, fotovoltaica y almacenamiento para vivienda \par}
    \vspace{1.5cm}
    {\Large\bfseries Trabajo de Fin de Máster\par}
    \vspace{0.5cm}
    {\large Máster Universitario en Ingeniería de la Energía \par}
    \vspace{2cm}
    {\Large Luis D. Aranda Sánchez\par}
    \vfill
    Director: Javier Rodríguez Martín
    \vfill
    {\large Septiembre 6, 2024\par}
\end{titlepage}

% Resumen (máximo de 5 páginas, incluyendo al final Palabras clave)
\clearpage
\pagestyle{simple}
% \newpage
\chapter*{Resumen}
\addcontentsline{toc}{chapter}{Resumen}
\input{capitulos/resumen/main.tex}

% Índice (paginado)
\clearpage
\pagestyle{simple}
% \newpage
\tableofcontents

% Introducción (donde se incluya los antecedentes y justificación)
\clearpage
\pagestyle{myfancy}
\newpage
\chapter{Introducción}
\input{capitulos/introduccion/main.tex}

% Objetivos
\chapter{Objetivos}
\input{capitulos/objetivos/main.tex}

% Metodología
\chapter{Metodología}
\input{capitulos/metodologia/main.tex}

% Resultados y discusión (incluyendo la valoración de impactos y de aspectos de responsabilidad legal, ética y profesional relacionados con el trabajo)
\chapter{Resultados y Discusión}
\input{capitulos/resultados_discusion/main.tex}

% Conclusiones
\chapter{Conclusiones}
\input{capitulos/conclusiones/main.tex}

% Planificación temporal y presupuesto
\chapter{Planificación Temporal y Presupuesto}
\input{capitulos/planificacion_presupuesto/main.tex}

% Bibliografía
\newpage
\addcontentsline{toc}{chapter}{Bibliografía}
\printbibliography

\end{document}


% Resultados y discusión (incluyendo la valoración de impactos y de aspectos de responsabilidad legal, ética y profesional relacionados con el trabajo)
\chapter{Resultados y Discusión}
\documentclass[a4paper,11pt,twoside]{report}
\usepackage[left=25mm,right=25mm,top=25mm,bottom=25mm,includehead,includefoot,headsep=15mm,footskip=15mm]{geometry}
\usepackage{graphicx}
\usepackage{fancyhdr}
\usepackage{titlesec}
\usepackage[spanish]{babel}
\usepackage[utf8]{inputenc}
\usepackage{amsmath}
\usepackage{setspace}
\usepackage{svg}
\usepackage{hyperref}
\usepackage[backend=biber,style=numeric]{biblatex}
\addbibresource{references.bib}
\hypersetup{
    colorlinks=true,
    linkcolor=blue,      % color of internal links (sections, etc.)
    urlcolor=blue,       % color of external links
    pdftitle={Optimización energética de sistema híbrido con bomba de calor, suelo radiante, fotovoltaica y almacenamiento para vivienda},    % title
    pdfauthor={Luis D. Aranda Sánchez},     % author
    pdfkeywords={palabra1, palabra2, código1, etc.} % list of keywords
}

% Font change to Arial
\usepackage{helvet}
\renewcommand{\familydefault}{\sfdefault}

% Chapter titles in uppercase and larger font
\titleformat{\chapter}[hang]{\large\bfseries}{\thechapter.}{1em}{\MakeUppercase}
\titleformat{\section}[hang]{\bfseries}{\thesection.}{1em}{}
\titleformat{\subsection}[hang]{\bfseries}{\thesubsection.}{1em}{}

% Fancyhdr setup
\setlength{\headheight}{14.30174pt} % Adjust to recommended value, slightly larger for safety
\fancyhf{} % Clear all headers and footers
\fancyhead[LE]{\nouppercase{\leftmark}}
\fancyhead[RO]{Optimización energética para vivienda}
\fancyfoot[LE]{\thepage}
\fancyfoot[RE]{Escuela Técnica Superior de Ingenieros Industriales (UPM)}
\fancyfoot[LO]{Luis D. Aranda Sánchez}
\fancyfoot[RO]{\thepage}
\renewcommand{\headrulewidth}{0.4pt}
\renewcommand{\footrulewidth}{0.4pt}

\fancypagestyle{myfancy}{
    \fancyhf{} % Clear all headers and footers
    \fancyhead[LE]{\nouppercase{\leftmark}}
    \fancyhead[RO]{Optimización energética para vivienda}
    \fancyfoot[LE]{\thepage}
    \fancyfoot[RE]{Escuela Técnica Superior de Ingenieros Industriales (UPM)}
    \fancyfoot[LO]{Luis D. Aranda Sánchez}
    \fancyfoot[RO]{\thepage}
    \renewcommand{\headrulewidth}{0.4pt}
    \renewcommand{\footrulewidth}{0.4pt}
}

\fancypagestyle{simple}{
    \fancyhf{} % Clear all headers and footers
    \renewcommand{\headrulewidth}{0pt}
    \renewcommand{\footrulewidth}{0pt}
}

% Line spacing
\setstretch{1.2}

% Document starts here
\begin{document}

% Portada
\begin{titlepage}
    \centering
    {\scshape\LARGE Universidad Politécnica de Madrid \par}
    \vspace{1cm}
    {\scshape\Large Escuela Técnica Superior de Ingenieros Industriales\par}
    \vspace{1.5cm}
    {\huge\bfseries Optimización energética de sistema híbrido con bomba de calor, suelo radiante, fotovoltaica y almacenamiento para vivienda \par}
    \vspace{1.5cm}
    {\Large\bfseries Trabajo de Fin de Máster\par}
    \vspace{0.5cm}
    {\large Máster Universitario en Ingeniería de la Energía \par}
    \vspace{2cm}
    {\Large Luis D. Aranda Sánchez\par}
    \vfill
    Director: Javier Rodríguez Martín
    \vfill
    {\large Septiembre 6, 2024\par}
\end{titlepage}

% Resumen (máximo de 5 páginas, incluyendo al final Palabras clave)
\clearpage
\pagestyle{simple}
% \newpage
\chapter*{Resumen}
\addcontentsline{toc}{chapter}{Resumen}
\input{capitulos/resumen/main.tex}

% Índice (paginado)
\clearpage
\pagestyle{simple}
% \newpage
\tableofcontents

% Introducción (donde se incluya los antecedentes y justificación)
\clearpage
\pagestyle{myfancy}
\newpage
\chapter{Introducción}
\input{capitulos/introduccion/main.tex}

% Objetivos
\chapter{Objetivos}
\input{capitulos/objetivos/main.tex}

% Metodología
\chapter{Metodología}
\input{capitulos/metodologia/main.tex}

% Resultados y discusión (incluyendo la valoración de impactos y de aspectos de responsabilidad legal, ética y profesional relacionados con el trabajo)
\chapter{Resultados y Discusión}
\input{capitulos/resultados_discusion/main.tex}

% Conclusiones
\chapter{Conclusiones}
\input{capitulos/conclusiones/main.tex}

% Planificación temporal y presupuesto
\chapter{Planificación Temporal y Presupuesto}
\input{capitulos/planificacion_presupuesto/main.tex}

% Bibliografía
\newpage
\addcontentsline{toc}{chapter}{Bibliografía}
\printbibliography

\end{document}


% Conclusiones
\chapter{Conclusiones}
\documentclass[a4paper,11pt,twoside]{report}
\usepackage[left=25mm,right=25mm,top=25mm,bottom=25mm,includehead,includefoot,headsep=15mm,footskip=15mm]{geometry}
\usepackage{graphicx}
\usepackage{fancyhdr}
\usepackage{titlesec}
\usepackage[spanish]{babel}
\usepackage[utf8]{inputenc}
\usepackage{amsmath}
\usepackage{setspace}
\usepackage{svg}
\usepackage{hyperref}
\usepackage[backend=biber,style=numeric]{biblatex}
\addbibresource{references.bib}
\hypersetup{
    colorlinks=true,
    linkcolor=blue,      % color of internal links (sections, etc.)
    urlcolor=blue,       % color of external links
    pdftitle={Optimización energética de sistema híbrido con bomba de calor, suelo radiante, fotovoltaica y almacenamiento para vivienda},    % title
    pdfauthor={Luis D. Aranda Sánchez},     % author
    pdfkeywords={palabra1, palabra2, código1, etc.} % list of keywords
}

% Font change to Arial
\usepackage{helvet}
\renewcommand{\familydefault}{\sfdefault}

% Chapter titles in uppercase and larger font
\titleformat{\chapter}[hang]{\large\bfseries}{\thechapter.}{1em}{\MakeUppercase}
\titleformat{\section}[hang]{\bfseries}{\thesection.}{1em}{}
\titleformat{\subsection}[hang]{\bfseries}{\thesubsection.}{1em}{}

% Fancyhdr setup
\setlength{\headheight}{14.30174pt} % Adjust to recommended value, slightly larger for safety
\fancyhf{} % Clear all headers and footers
\fancyhead[LE]{\nouppercase{\leftmark}}
\fancyhead[RO]{Optimización energética para vivienda}
\fancyfoot[LE]{\thepage}
\fancyfoot[RE]{Escuela Técnica Superior de Ingenieros Industriales (UPM)}
\fancyfoot[LO]{Luis D. Aranda Sánchez}
\fancyfoot[RO]{\thepage}
\renewcommand{\headrulewidth}{0.4pt}
\renewcommand{\footrulewidth}{0.4pt}

\fancypagestyle{myfancy}{
    \fancyhf{} % Clear all headers and footers
    \fancyhead[LE]{\nouppercase{\leftmark}}
    \fancyhead[RO]{Optimización energética para vivienda}
    \fancyfoot[LE]{\thepage}
    \fancyfoot[RE]{Escuela Técnica Superior de Ingenieros Industriales (UPM)}
    \fancyfoot[LO]{Luis D. Aranda Sánchez}
    \fancyfoot[RO]{\thepage}
    \renewcommand{\headrulewidth}{0.4pt}
    \renewcommand{\footrulewidth}{0.4pt}
}

\fancypagestyle{simple}{
    \fancyhf{} % Clear all headers and footers
    \renewcommand{\headrulewidth}{0pt}
    \renewcommand{\footrulewidth}{0pt}
}

% Line spacing
\setstretch{1.2}

% Document starts here
\begin{document}

% Portada
\begin{titlepage}
    \centering
    {\scshape\LARGE Universidad Politécnica de Madrid \par}
    \vspace{1cm}
    {\scshape\Large Escuela Técnica Superior de Ingenieros Industriales\par}
    \vspace{1.5cm}
    {\huge\bfseries Optimización energética de sistema híbrido con bomba de calor, suelo radiante, fotovoltaica y almacenamiento para vivienda \par}
    \vspace{1.5cm}
    {\Large\bfseries Trabajo de Fin de Máster\par}
    \vspace{0.5cm}
    {\large Máster Universitario en Ingeniería de la Energía \par}
    \vspace{2cm}
    {\Large Luis D. Aranda Sánchez\par}
    \vfill
    Director: Javier Rodríguez Martín
    \vfill
    {\large Septiembre 6, 2024\par}
\end{titlepage}

% Resumen (máximo de 5 páginas, incluyendo al final Palabras clave)
\clearpage
\pagestyle{simple}
% \newpage
\chapter*{Resumen}
\addcontentsline{toc}{chapter}{Resumen}
\input{capitulos/resumen/main.tex}

% Índice (paginado)
\clearpage
\pagestyle{simple}
% \newpage
\tableofcontents

% Introducción (donde se incluya los antecedentes y justificación)
\clearpage
\pagestyle{myfancy}
\newpage
\chapter{Introducción}
\input{capitulos/introduccion/main.tex}

% Objetivos
\chapter{Objetivos}
\input{capitulos/objetivos/main.tex}

% Metodología
\chapter{Metodología}
\input{capitulos/metodologia/main.tex}

% Resultados y discusión (incluyendo la valoración de impactos y de aspectos de responsabilidad legal, ética y profesional relacionados con el trabajo)
\chapter{Resultados y Discusión}
\input{capitulos/resultados_discusion/main.tex}

% Conclusiones
\chapter{Conclusiones}
\input{capitulos/conclusiones/main.tex}

% Planificación temporal y presupuesto
\chapter{Planificación Temporal y Presupuesto}
\input{capitulos/planificacion_presupuesto/main.tex}

% Bibliografía
\newpage
\addcontentsline{toc}{chapter}{Bibliografía}
\printbibliography

\end{document}


% Planificación temporal y presupuesto
\chapter{Planificación Temporal y Presupuesto}
\documentclass[a4paper,11pt,twoside]{report}
\usepackage[left=25mm,right=25mm,top=25mm,bottom=25mm,includehead,includefoot,headsep=15mm,footskip=15mm]{geometry}
\usepackage{graphicx}
\usepackage{fancyhdr}
\usepackage{titlesec}
\usepackage[spanish]{babel}
\usepackage[utf8]{inputenc}
\usepackage{amsmath}
\usepackage{setspace}
\usepackage{svg}
\usepackage{hyperref}
\usepackage[backend=biber,style=numeric]{biblatex}
\addbibresource{references.bib}
\hypersetup{
    colorlinks=true,
    linkcolor=blue,      % color of internal links (sections, etc.)
    urlcolor=blue,       % color of external links
    pdftitle={Optimización energética de sistema híbrido con bomba de calor, suelo radiante, fotovoltaica y almacenamiento para vivienda},    % title
    pdfauthor={Luis D. Aranda Sánchez},     % author
    pdfkeywords={palabra1, palabra2, código1, etc.} % list of keywords
}

% Font change to Arial
\usepackage{helvet}
\renewcommand{\familydefault}{\sfdefault}

% Chapter titles in uppercase and larger font
\titleformat{\chapter}[hang]{\large\bfseries}{\thechapter.}{1em}{\MakeUppercase}
\titleformat{\section}[hang]{\bfseries}{\thesection.}{1em}{}
\titleformat{\subsection}[hang]{\bfseries}{\thesubsection.}{1em}{}

% Fancyhdr setup
\setlength{\headheight}{14.30174pt} % Adjust to recommended value, slightly larger for safety
\fancyhf{} % Clear all headers and footers
\fancyhead[LE]{\nouppercase{\leftmark}}
\fancyhead[RO]{Optimización energética para vivienda}
\fancyfoot[LE]{\thepage}
\fancyfoot[RE]{Escuela Técnica Superior de Ingenieros Industriales (UPM)}
\fancyfoot[LO]{Luis D. Aranda Sánchez}
\fancyfoot[RO]{\thepage}
\renewcommand{\headrulewidth}{0.4pt}
\renewcommand{\footrulewidth}{0.4pt}

\fancypagestyle{myfancy}{
    \fancyhf{} % Clear all headers and footers
    \fancyhead[LE]{\nouppercase{\leftmark}}
    \fancyhead[RO]{Optimización energética para vivienda}
    \fancyfoot[LE]{\thepage}
    \fancyfoot[RE]{Escuela Técnica Superior de Ingenieros Industriales (UPM)}
    \fancyfoot[LO]{Luis D. Aranda Sánchez}
    \fancyfoot[RO]{\thepage}
    \renewcommand{\headrulewidth}{0.4pt}
    \renewcommand{\footrulewidth}{0.4pt}
}

\fancypagestyle{simple}{
    \fancyhf{} % Clear all headers and footers
    \renewcommand{\headrulewidth}{0pt}
    \renewcommand{\footrulewidth}{0pt}
}

% Line spacing
\setstretch{1.2}

% Document starts here
\begin{document}

% Portada
\begin{titlepage}
    \centering
    {\scshape\LARGE Universidad Politécnica de Madrid \par}
    \vspace{1cm}
    {\scshape\Large Escuela Técnica Superior de Ingenieros Industriales\par}
    \vspace{1.5cm}
    {\huge\bfseries Optimización energética de sistema híbrido con bomba de calor, suelo radiante, fotovoltaica y almacenamiento para vivienda \par}
    \vspace{1.5cm}
    {\Large\bfseries Trabajo de Fin de Máster\par}
    \vspace{0.5cm}
    {\large Máster Universitario en Ingeniería de la Energía \par}
    \vspace{2cm}
    {\Large Luis D. Aranda Sánchez\par}
    \vfill
    Director: Javier Rodríguez Martín
    \vfill
    {\large Septiembre 6, 2024\par}
\end{titlepage}

% Resumen (máximo de 5 páginas, incluyendo al final Palabras clave)
\clearpage
\pagestyle{simple}
% \newpage
\chapter*{Resumen}
\addcontentsline{toc}{chapter}{Resumen}
\input{capitulos/resumen/main.tex}

% Índice (paginado)
\clearpage
\pagestyle{simple}
% \newpage
\tableofcontents

% Introducción (donde se incluya los antecedentes y justificación)
\clearpage
\pagestyle{myfancy}
\newpage
\chapter{Introducción}
\input{capitulos/introduccion/main.tex}

% Objetivos
\chapter{Objetivos}
\input{capitulos/objetivos/main.tex}

% Metodología
\chapter{Metodología}
\input{capitulos/metodologia/main.tex}

% Resultados y discusión (incluyendo la valoración de impactos y de aspectos de responsabilidad legal, ética y profesional relacionados con el trabajo)
\chapter{Resultados y Discusión}
\input{capitulos/resultados_discusion/main.tex}

% Conclusiones
\chapter{Conclusiones}
\input{capitulos/conclusiones/main.tex}

% Planificación temporal y presupuesto
\chapter{Planificación Temporal y Presupuesto}
\input{capitulos/planificacion_presupuesto/main.tex}

% Bibliografía
\newpage
\addcontentsline{toc}{chapter}{Bibliografía}
\printbibliography

\end{document}


% Bibliografía
\newpage
\addcontentsline{toc}{chapter}{Bibliografía}
\printbibliography

\end{document}


% Resultados y discusión (incluyendo la valoración de impactos y de aspectos de responsabilidad legal, ética y profesional relacionados con el trabajo)
\chapter{Resultados y Discusión}
\documentclass[a4paper,11pt,twoside]{report}
\usepackage[left=25mm,right=25mm,top=25mm,bottom=25mm,includehead,includefoot,headsep=15mm,footskip=15mm]{geometry}
\usepackage{graphicx}
\usepackage{fancyhdr}
\usepackage{titlesec}
\usepackage[spanish]{babel}
\usepackage[utf8]{inputenc}
\usepackage{amsmath}
\usepackage{setspace}
\usepackage{svg}
\usepackage{hyperref}
\usepackage[backend=biber,style=numeric]{biblatex}
\addbibresource{references.bib}
\hypersetup{
    colorlinks=true,
    linkcolor=blue,      % color of internal links (sections, etc.)
    urlcolor=blue,       % color of external links
    pdftitle={Optimización energética de sistema híbrido con bomba de calor, suelo radiante, fotovoltaica y almacenamiento para vivienda},    % title
    pdfauthor={Luis D. Aranda Sánchez},     % author
    pdfkeywords={palabra1, palabra2, código1, etc.} % list of keywords
}

% Font change to Arial
\usepackage{helvet}
\renewcommand{\familydefault}{\sfdefault}

% Chapter titles in uppercase and larger font
\titleformat{\chapter}[hang]{\large\bfseries}{\thechapter.}{1em}{\MakeUppercase}
\titleformat{\section}[hang]{\bfseries}{\thesection.}{1em}{}
\titleformat{\subsection}[hang]{\bfseries}{\thesubsection.}{1em}{}

% Fancyhdr setup
\setlength{\headheight}{14.30174pt} % Adjust to recommended value, slightly larger for safety
\fancyhf{} % Clear all headers and footers
\fancyhead[LE]{\nouppercase{\leftmark}}
\fancyhead[RO]{Optimización energética para vivienda}
\fancyfoot[LE]{\thepage}
\fancyfoot[RE]{Escuela Técnica Superior de Ingenieros Industriales (UPM)}
\fancyfoot[LO]{Luis D. Aranda Sánchez}
\fancyfoot[RO]{\thepage}
\renewcommand{\headrulewidth}{0.4pt}
\renewcommand{\footrulewidth}{0.4pt}

\fancypagestyle{myfancy}{
    \fancyhf{} % Clear all headers and footers
    \fancyhead[LE]{\nouppercase{\leftmark}}
    \fancyhead[RO]{Optimización energética para vivienda}
    \fancyfoot[LE]{\thepage}
    \fancyfoot[RE]{Escuela Técnica Superior de Ingenieros Industriales (UPM)}
    \fancyfoot[LO]{Luis D. Aranda Sánchez}
    \fancyfoot[RO]{\thepage}
    \renewcommand{\headrulewidth}{0.4pt}
    \renewcommand{\footrulewidth}{0.4pt}
}

\fancypagestyle{simple}{
    \fancyhf{} % Clear all headers and footers
    \renewcommand{\headrulewidth}{0pt}
    \renewcommand{\footrulewidth}{0pt}
}

% Line spacing
\setstretch{1.2}

% Document starts here
\begin{document}

% Portada
\begin{titlepage}
    \centering
    {\scshape\LARGE Universidad Politécnica de Madrid \par}
    \vspace{1cm}
    {\scshape\Large Escuela Técnica Superior de Ingenieros Industriales\par}
    \vspace{1.5cm}
    {\huge\bfseries Optimización energética de sistema híbrido con bomba de calor, suelo radiante, fotovoltaica y almacenamiento para vivienda \par}
    \vspace{1.5cm}
    {\Large\bfseries Trabajo de Fin de Máster\par}
    \vspace{0.5cm}
    {\large Máster Universitario en Ingeniería de la Energía \par}
    \vspace{2cm}
    {\Large Luis D. Aranda Sánchez\par}
    \vfill
    Director: Javier Rodríguez Martín
    \vfill
    {\large Septiembre 6, 2024\par}
\end{titlepage}

% Resumen (máximo de 5 páginas, incluyendo al final Palabras clave)
\clearpage
\pagestyle{simple}
% \newpage
\chapter*{Resumen}
\addcontentsline{toc}{chapter}{Resumen}
\documentclass[a4paper,11pt,twoside]{report}
\usepackage[left=25mm,right=25mm,top=25mm,bottom=25mm,includehead,includefoot,headsep=15mm,footskip=15mm]{geometry}
\usepackage{graphicx}
\usepackage{fancyhdr}
\usepackage{titlesec}
\usepackage[spanish]{babel}
\usepackage[utf8]{inputenc}
\usepackage{amsmath}
\usepackage{setspace}
\usepackage{svg}
\usepackage{hyperref}
\usepackage[backend=biber,style=numeric]{biblatex}
\addbibresource{references.bib}
\hypersetup{
    colorlinks=true,
    linkcolor=blue,      % color of internal links (sections, etc.)
    urlcolor=blue,       % color of external links
    pdftitle={Optimización energética de sistema híbrido con bomba de calor, suelo radiante, fotovoltaica y almacenamiento para vivienda},    % title
    pdfauthor={Luis D. Aranda Sánchez},     % author
    pdfkeywords={palabra1, palabra2, código1, etc.} % list of keywords
}

% Font change to Arial
\usepackage{helvet}
\renewcommand{\familydefault}{\sfdefault}

% Chapter titles in uppercase and larger font
\titleformat{\chapter}[hang]{\large\bfseries}{\thechapter.}{1em}{\MakeUppercase}
\titleformat{\section}[hang]{\bfseries}{\thesection.}{1em}{}
\titleformat{\subsection}[hang]{\bfseries}{\thesubsection.}{1em}{}

% Fancyhdr setup
\setlength{\headheight}{14.30174pt} % Adjust to recommended value, slightly larger for safety
\fancyhf{} % Clear all headers and footers
\fancyhead[LE]{\nouppercase{\leftmark}}
\fancyhead[RO]{Optimización energética para vivienda}
\fancyfoot[LE]{\thepage}
\fancyfoot[RE]{Escuela Técnica Superior de Ingenieros Industriales (UPM)}
\fancyfoot[LO]{Luis D. Aranda Sánchez}
\fancyfoot[RO]{\thepage}
\renewcommand{\headrulewidth}{0.4pt}
\renewcommand{\footrulewidth}{0.4pt}

\fancypagestyle{myfancy}{
    \fancyhf{} % Clear all headers and footers
    \fancyhead[LE]{\nouppercase{\leftmark}}
    \fancyhead[RO]{Optimización energética para vivienda}
    \fancyfoot[LE]{\thepage}
    \fancyfoot[RE]{Escuela Técnica Superior de Ingenieros Industriales (UPM)}
    \fancyfoot[LO]{Luis D. Aranda Sánchez}
    \fancyfoot[RO]{\thepage}
    \renewcommand{\headrulewidth}{0.4pt}
    \renewcommand{\footrulewidth}{0.4pt}
}

\fancypagestyle{simple}{
    \fancyhf{} % Clear all headers and footers
    \renewcommand{\headrulewidth}{0pt}
    \renewcommand{\footrulewidth}{0pt}
}

% Line spacing
\setstretch{1.2}

% Document starts here
\begin{document}

% Portada
\begin{titlepage}
    \centering
    {\scshape\LARGE Universidad Politécnica de Madrid \par}
    \vspace{1cm}
    {\scshape\Large Escuela Técnica Superior de Ingenieros Industriales\par}
    \vspace{1.5cm}
    {\huge\bfseries Optimización energética de sistema híbrido con bomba de calor, suelo radiante, fotovoltaica y almacenamiento para vivienda \par}
    \vspace{1.5cm}
    {\Large\bfseries Trabajo de Fin de Máster\par}
    \vspace{0.5cm}
    {\large Máster Universitario en Ingeniería de la Energía \par}
    \vspace{2cm}
    {\Large Luis D. Aranda Sánchez\par}
    \vfill
    Director: Javier Rodríguez Martín
    \vfill
    {\large Septiembre 6, 2024\par}
\end{titlepage}

% Resumen (máximo de 5 páginas, incluyendo al final Palabras clave)
\clearpage
\pagestyle{simple}
% \newpage
\chapter*{Resumen}
\addcontentsline{toc}{chapter}{Resumen}
\input{capitulos/resumen/main.tex}

% Índice (paginado)
\clearpage
\pagestyle{simple}
% \newpage
\tableofcontents

% Introducción (donde se incluya los antecedentes y justificación)
\clearpage
\pagestyle{myfancy}
\newpage
\chapter{Introducción}
\input{capitulos/introduccion/main.tex}

% Objetivos
\chapter{Objetivos}
\input{capitulos/objetivos/main.tex}

% Metodología
\chapter{Metodología}
\input{capitulos/metodologia/main.tex}

% Resultados y discusión (incluyendo la valoración de impactos y de aspectos de responsabilidad legal, ética y profesional relacionados con el trabajo)
\chapter{Resultados y Discusión}
\input{capitulos/resultados_discusion/main.tex}

% Conclusiones
\chapter{Conclusiones}
\input{capitulos/conclusiones/main.tex}

% Planificación temporal y presupuesto
\chapter{Planificación Temporal y Presupuesto}
\input{capitulos/planificacion_presupuesto/main.tex}

% Bibliografía
\newpage
\addcontentsline{toc}{chapter}{Bibliografía}
\printbibliography

\end{document}


% Índice (paginado)
\clearpage
\pagestyle{simple}
% \newpage
\tableofcontents

% Introducción (donde se incluya los antecedentes y justificación)
\clearpage
\pagestyle{myfancy}
\newpage
\chapter{Introducción}
\documentclass[a4paper,11pt,twoside]{report}
\usepackage[left=25mm,right=25mm,top=25mm,bottom=25mm,includehead,includefoot,headsep=15mm,footskip=15mm]{geometry}
\usepackage{graphicx}
\usepackage{fancyhdr}
\usepackage{titlesec}
\usepackage[spanish]{babel}
\usepackage[utf8]{inputenc}
\usepackage{amsmath}
\usepackage{setspace}
\usepackage{svg}
\usepackage{hyperref}
\usepackage[backend=biber,style=numeric]{biblatex}
\addbibresource{references.bib}
\hypersetup{
    colorlinks=true,
    linkcolor=blue,      % color of internal links (sections, etc.)
    urlcolor=blue,       % color of external links
    pdftitle={Optimización energética de sistema híbrido con bomba de calor, suelo radiante, fotovoltaica y almacenamiento para vivienda},    % title
    pdfauthor={Luis D. Aranda Sánchez},     % author
    pdfkeywords={palabra1, palabra2, código1, etc.} % list of keywords
}

% Font change to Arial
\usepackage{helvet}
\renewcommand{\familydefault}{\sfdefault}

% Chapter titles in uppercase and larger font
\titleformat{\chapter}[hang]{\large\bfseries}{\thechapter.}{1em}{\MakeUppercase}
\titleformat{\section}[hang]{\bfseries}{\thesection.}{1em}{}
\titleformat{\subsection}[hang]{\bfseries}{\thesubsection.}{1em}{}

% Fancyhdr setup
\setlength{\headheight}{14.30174pt} % Adjust to recommended value, slightly larger for safety
\fancyhf{} % Clear all headers and footers
\fancyhead[LE]{\nouppercase{\leftmark}}
\fancyhead[RO]{Optimización energética para vivienda}
\fancyfoot[LE]{\thepage}
\fancyfoot[RE]{Escuela Técnica Superior de Ingenieros Industriales (UPM)}
\fancyfoot[LO]{Luis D. Aranda Sánchez}
\fancyfoot[RO]{\thepage}
\renewcommand{\headrulewidth}{0.4pt}
\renewcommand{\footrulewidth}{0.4pt}

\fancypagestyle{myfancy}{
    \fancyhf{} % Clear all headers and footers
    \fancyhead[LE]{\nouppercase{\leftmark}}
    \fancyhead[RO]{Optimización energética para vivienda}
    \fancyfoot[LE]{\thepage}
    \fancyfoot[RE]{Escuela Técnica Superior de Ingenieros Industriales (UPM)}
    \fancyfoot[LO]{Luis D. Aranda Sánchez}
    \fancyfoot[RO]{\thepage}
    \renewcommand{\headrulewidth}{0.4pt}
    \renewcommand{\footrulewidth}{0.4pt}
}

\fancypagestyle{simple}{
    \fancyhf{} % Clear all headers and footers
    \renewcommand{\headrulewidth}{0pt}
    \renewcommand{\footrulewidth}{0pt}
}

% Line spacing
\setstretch{1.2}

% Document starts here
\begin{document}

% Portada
\begin{titlepage}
    \centering
    {\scshape\LARGE Universidad Politécnica de Madrid \par}
    \vspace{1cm}
    {\scshape\Large Escuela Técnica Superior de Ingenieros Industriales\par}
    \vspace{1.5cm}
    {\huge\bfseries Optimización energética de sistema híbrido con bomba de calor, suelo radiante, fotovoltaica y almacenamiento para vivienda \par}
    \vspace{1.5cm}
    {\Large\bfseries Trabajo de Fin de Máster\par}
    \vspace{0.5cm}
    {\large Máster Universitario en Ingeniería de la Energía \par}
    \vspace{2cm}
    {\Large Luis D. Aranda Sánchez\par}
    \vfill
    Director: Javier Rodríguez Martín
    \vfill
    {\large Septiembre 6, 2024\par}
\end{titlepage}

% Resumen (máximo de 5 páginas, incluyendo al final Palabras clave)
\clearpage
\pagestyle{simple}
% \newpage
\chapter*{Resumen}
\addcontentsline{toc}{chapter}{Resumen}
\input{capitulos/resumen/main.tex}

% Índice (paginado)
\clearpage
\pagestyle{simple}
% \newpage
\tableofcontents

% Introducción (donde se incluya los antecedentes y justificación)
\clearpage
\pagestyle{myfancy}
\newpage
\chapter{Introducción}
\input{capitulos/introduccion/main.tex}

% Objetivos
\chapter{Objetivos}
\input{capitulos/objetivos/main.tex}

% Metodología
\chapter{Metodología}
\input{capitulos/metodologia/main.tex}

% Resultados y discusión (incluyendo la valoración de impactos y de aspectos de responsabilidad legal, ética y profesional relacionados con el trabajo)
\chapter{Resultados y Discusión}
\input{capitulos/resultados_discusion/main.tex}

% Conclusiones
\chapter{Conclusiones}
\input{capitulos/conclusiones/main.tex}

% Planificación temporal y presupuesto
\chapter{Planificación Temporal y Presupuesto}
\input{capitulos/planificacion_presupuesto/main.tex}

% Bibliografía
\newpage
\addcontentsline{toc}{chapter}{Bibliografía}
\printbibliography

\end{document}


% Objetivos
\chapter{Objetivos}
\documentclass[a4paper,11pt,twoside]{report}
\usepackage[left=25mm,right=25mm,top=25mm,bottom=25mm,includehead,includefoot,headsep=15mm,footskip=15mm]{geometry}
\usepackage{graphicx}
\usepackage{fancyhdr}
\usepackage{titlesec}
\usepackage[spanish]{babel}
\usepackage[utf8]{inputenc}
\usepackage{amsmath}
\usepackage{setspace}
\usepackage{svg}
\usepackage{hyperref}
\usepackage[backend=biber,style=numeric]{biblatex}
\addbibresource{references.bib}
\hypersetup{
    colorlinks=true,
    linkcolor=blue,      % color of internal links (sections, etc.)
    urlcolor=blue,       % color of external links
    pdftitle={Optimización energética de sistema híbrido con bomba de calor, suelo radiante, fotovoltaica y almacenamiento para vivienda},    % title
    pdfauthor={Luis D. Aranda Sánchez},     % author
    pdfkeywords={palabra1, palabra2, código1, etc.} % list of keywords
}

% Font change to Arial
\usepackage{helvet}
\renewcommand{\familydefault}{\sfdefault}

% Chapter titles in uppercase and larger font
\titleformat{\chapter}[hang]{\large\bfseries}{\thechapter.}{1em}{\MakeUppercase}
\titleformat{\section}[hang]{\bfseries}{\thesection.}{1em}{}
\titleformat{\subsection}[hang]{\bfseries}{\thesubsection.}{1em}{}

% Fancyhdr setup
\setlength{\headheight}{14.30174pt} % Adjust to recommended value, slightly larger for safety
\fancyhf{} % Clear all headers and footers
\fancyhead[LE]{\nouppercase{\leftmark}}
\fancyhead[RO]{Optimización energética para vivienda}
\fancyfoot[LE]{\thepage}
\fancyfoot[RE]{Escuela Técnica Superior de Ingenieros Industriales (UPM)}
\fancyfoot[LO]{Luis D. Aranda Sánchez}
\fancyfoot[RO]{\thepage}
\renewcommand{\headrulewidth}{0.4pt}
\renewcommand{\footrulewidth}{0.4pt}

\fancypagestyle{myfancy}{
    \fancyhf{} % Clear all headers and footers
    \fancyhead[LE]{\nouppercase{\leftmark}}
    \fancyhead[RO]{Optimización energética para vivienda}
    \fancyfoot[LE]{\thepage}
    \fancyfoot[RE]{Escuela Técnica Superior de Ingenieros Industriales (UPM)}
    \fancyfoot[LO]{Luis D. Aranda Sánchez}
    \fancyfoot[RO]{\thepage}
    \renewcommand{\headrulewidth}{0.4pt}
    \renewcommand{\footrulewidth}{0.4pt}
}

\fancypagestyle{simple}{
    \fancyhf{} % Clear all headers and footers
    \renewcommand{\headrulewidth}{0pt}
    \renewcommand{\footrulewidth}{0pt}
}

% Line spacing
\setstretch{1.2}

% Document starts here
\begin{document}

% Portada
\begin{titlepage}
    \centering
    {\scshape\LARGE Universidad Politécnica de Madrid \par}
    \vspace{1cm}
    {\scshape\Large Escuela Técnica Superior de Ingenieros Industriales\par}
    \vspace{1.5cm}
    {\huge\bfseries Optimización energética de sistema híbrido con bomba de calor, suelo radiante, fotovoltaica y almacenamiento para vivienda \par}
    \vspace{1.5cm}
    {\Large\bfseries Trabajo de Fin de Máster\par}
    \vspace{0.5cm}
    {\large Máster Universitario en Ingeniería de la Energía \par}
    \vspace{2cm}
    {\Large Luis D. Aranda Sánchez\par}
    \vfill
    Director: Javier Rodríguez Martín
    \vfill
    {\large Septiembre 6, 2024\par}
\end{titlepage}

% Resumen (máximo de 5 páginas, incluyendo al final Palabras clave)
\clearpage
\pagestyle{simple}
% \newpage
\chapter*{Resumen}
\addcontentsline{toc}{chapter}{Resumen}
\input{capitulos/resumen/main.tex}

% Índice (paginado)
\clearpage
\pagestyle{simple}
% \newpage
\tableofcontents

% Introducción (donde se incluya los antecedentes y justificación)
\clearpage
\pagestyle{myfancy}
\newpage
\chapter{Introducción}
\input{capitulos/introduccion/main.tex}

% Objetivos
\chapter{Objetivos}
\input{capitulos/objetivos/main.tex}

% Metodología
\chapter{Metodología}
\input{capitulos/metodologia/main.tex}

% Resultados y discusión (incluyendo la valoración de impactos y de aspectos de responsabilidad legal, ética y profesional relacionados con el trabajo)
\chapter{Resultados y Discusión}
\input{capitulos/resultados_discusion/main.tex}

% Conclusiones
\chapter{Conclusiones}
\input{capitulos/conclusiones/main.tex}

% Planificación temporal y presupuesto
\chapter{Planificación Temporal y Presupuesto}
\input{capitulos/planificacion_presupuesto/main.tex}

% Bibliografía
\newpage
\addcontentsline{toc}{chapter}{Bibliografía}
\printbibliography

\end{document}


% Metodología
\chapter{Metodología}
\documentclass[a4paper,11pt,twoside]{report}
\usepackage[left=25mm,right=25mm,top=25mm,bottom=25mm,includehead,includefoot,headsep=15mm,footskip=15mm]{geometry}
\usepackage{graphicx}
\usepackage{fancyhdr}
\usepackage{titlesec}
\usepackage[spanish]{babel}
\usepackage[utf8]{inputenc}
\usepackage{amsmath}
\usepackage{setspace}
\usepackage{svg}
\usepackage{hyperref}
\usepackage[backend=biber,style=numeric]{biblatex}
\addbibresource{references.bib}
\hypersetup{
    colorlinks=true,
    linkcolor=blue,      % color of internal links (sections, etc.)
    urlcolor=blue,       % color of external links
    pdftitle={Optimización energética de sistema híbrido con bomba de calor, suelo radiante, fotovoltaica y almacenamiento para vivienda},    % title
    pdfauthor={Luis D. Aranda Sánchez},     % author
    pdfkeywords={palabra1, palabra2, código1, etc.} % list of keywords
}

% Font change to Arial
\usepackage{helvet}
\renewcommand{\familydefault}{\sfdefault}

% Chapter titles in uppercase and larger font
\titleformat{\chapter}[hang]{\large\bfseries}{\thechapter.}{1em}{\MakeUppercase}
\titleformat{\section}[hang]{\bfseries}{\thesection.}{1em}{}
\titleformat{\subsection}[hang]{\bfseries}{\thesubsection.}{1em}{}

% Fancyhdr setup
\setlength{\headheight}{14.30174pt} % Adjust to recommended value, slightly larger for safety
\fancyhf{} % Clear all headers and footers
\fancyhead[LE]{\nouppercase{\leftmark}}
\fancyhead[RO]{Optimización energética para vivienda}
\fancyfoot[LE]{\thepage}
\fancyfoot[RE]{Escuela Técnica Superior de Ingenieros Industriales (UPM)}
\fancyfoot[LO]{Luis D. Aranda Sánchez}
\fancyfoot[RO]{\thepage}
\renewcommand{\headrulewidth}{0.4pt}
\renewcommand{\footrulewidth}{0.4pt}

\fancypagestyle{myfancy}{
    \fancyhf{} % Clear all headers and footers
    \fancyhead[LE]{\nouppercase{\leftmark}}
    \fancyhead[RO]{Optimización energética para vivienda}
    \fancyfoot[LE]{\thepage}
    \fancyfoot[RE]{Escuela Técnica Superior de Ingenieros Industriales (UPM)}
    \fancyfoot[LO]{Luis D. Aranda Sánchez}
    \fancyfoot[RO]{\thepage}
    \renewcommand{\headrulewidth}{0.4pt}
    \renewcommand{\footrulewidth}{0.4pt}
}

\fancypagestyle{simple}{
    \fancyhf{} % Clear all headers and footers
    \renewcommand{\headrulewidth}{0pt}
    \renewcommand{\footrulewidth}{0pt}
}

% Line spacing
\setstretch{1.2}

% Document starts here
\begin{document}

% Portada
\begin{titlepage}
    \centering
    {\scshape\LARGE Universidad Politécnica de Madrid \par}
    \vspace{1cm}
    {\scshape\Large Escuela Técnica Superior de Ingenieros Industriales\par}
    \vspace{1.5cm}
    {\huge\bfseries Optimización energética de sistema híbrido con bomba de calor, suelo radiante, fotovoltaica y almacenamiento para vivienda \par}
    \vspace{1.5cm}
    {\Large\bfseries Trabajo de Fin de Máster\par}
    \vspace{0.5cm}
    {\large Máster Universitario en Ingeniería de la Energía \par}
    \vspace{2cm}
    {\Large Luis D. Aranda Sánchez\par}
    \vfill
    Director: Javier Rodríguez Martín
    \vfill
    {\large Septiembre 6, 2024\par}
\end{titlepage}

% Resumen (máximo de 5 páginas, incluyendo al final Palabras clave)
\clearpage
\pagestyle{simple}
% \newpage
\chapter*{Resumen}
\addcontentsline{toc}{chapter}{Resumen}
\input{capitulos/resumen/main.tex}

% Índice (paginado)
\clearpage
\pagestyle{simple}
% \newpage
\tableofcontents

% Introducción (donde se incluya los antecedentes y justificación)
\clearpage
\pagestyle{myfancy}
\newpage
\chapter{Introducción}
\input{capitulos/introduccion/main.tex}

% Objetivos
\chapter{Objetivos}
\input{capitulos/objetivos/main.tex}

% Metodología
\chapter{Metodología}
\input{capitulos/metodologia/main.tex}

% Resultados y discusión (incluyendo la valoración de impactos y de aspectos de responsabilidad legal, ética y profesional relacionados con el trabajo)
\chapter{Resultados y Discusión}
\input{capitulos/resultados_discusion/main.tex}

% Conclusiones
\chapter{Conclusiones}
\input{capitulos/conclusiones/main.tex}

% Planificación temporal y presupuesto
\chapter{Planificación Temporal y Presupuesto}
\input{capitulos/planificacion_presupuesto/main.tex}

% Bibliografía
\newpage
\addcontentsline{toc}{chapter}{Bibliografía}
\printbibliography

\end{document}


% Resultados y discusión (incluyendo la valoración de impactos y de aspectos de responsabilidad legal, ética y profesional relacionados con el trabajo)
\chapter{Resultados y Discusión}
\documentclass[a4paper,11pt,twoside]{report}
\usepackage[left=25mm,right=25mm,top=25mm,bottom=25mm,includehead,includefoot,headsep=15mm,footskip=15mm]{geometry}
\usepackage{graphicx}
\usepackage{fancyhdr}
\usepackage{titlesec}
\usepackage[spanish]{babel}
\usepackage[utf8]{inputenc}
\usepackage{amsmath}
\usepackage{setspace}
\usepackage{svg}
\usepackage{hyperref}
\usepackage[backend=biber,style=numeric]{biblatex}
\addbibresource{references.bib}
\hypersetup{
    colorlinks=true,
    linkcolor=blue,      % color of internal links (sections, etc.)
    urlcolor=blue,       % color of external links
    pdftitle={Optimización energética de sistema híbrido con bomba de calor, suelo radiante, fotovoltaica y almacenamiento para vivienda},    % title
    pdfauthor={Luis D. Aranda Sánchez},     % author
    pdfkeywords={palabra1, palabra2, código1, etc.} % list of keywords
}

% Font change to Arial
\usepackage{helvet}
\renewcommand{\familydefault}{\sfdefault}

% Chapter titles in uppercase and larger font
\titleformat{\chapter}[hang]{\large\bfseries}{\thechapter.}{1em}{\MakeUppercase}
\titleformat{\section}[hang]{\bfseries}{\thesection.}{1em}{}
\titleformat{\subsection}[hang]{\bfseries}{\thesubsection.}{1em}{}

% Fancyhdr setup
\setlength{\headheight}{14.30174pt} % Adjust to recommended value, slightly larger for safety
\fancyhf{} % Clear all headers and footers
\fancyhead[LE]{\nouppercase{\leftmark}}
\fancyhead[RO]{Optimización energética para vivienda}
\fancyfoot[LE]{\thepage}
\fancyfoot[RE]{Escuela Técnica Superior de Ingenieros Industriales (UPM)}
\fancyfoot[LO]{Luis D. Aranda Sánchez}
\fancyfoot[RO]{\thepage}
\renewcommand{\headrulewidth}{0.4pt}
\renewcommand{\footrulewidth}{0.4pt}

\fancypagestyle{myfancy}{
    \fancyhf{} % Clear all headers and footers
    \fancyhead[LE]{\nouppercase{\leftmark}}
    \fancyhead[RO]{Optimización energética para vivienda}
    \fancyfoot[LE]{\thepage}
    \fancyfoot[RE]{Escuela Técnica Superior de Ingenieros Industriales (UPM)}
    \fancyfoot[LO]{Luis D. Aranda Sánchez}
    \fancyfoot[RO]{\thepage}
    \renewcommand{\headrulewidth}{0.4pt}
    \renewcommand{\footrulewidth}{0.4pt}
}

\fancypagestyle{simple}{
    \fancyhf{} % Clear all headers and footers
    \renewcommand{\headrulewidth}{0pt}
    \renewcommand{\footrulewidth}{0pt}
}

% Line spacing
\setstretch{1.2}

% Document starts here
\begin{document}

% Portada
\begin{titlepage}
    \centering
    {\scshape\LARGE Universidad Politécnica de Madrid \par}
    \vspace{1cm}
    {\scshape\Large Escuela Técnica Superior de Ingenieros Industriales\par}
    \vspace{1.5cm}
    {\huge\bfseries Optimización energética de sistema híbrido con bomba de calor, suelo radiante, fotovoltaica y almacenamiento para vivienda \par}
    \vspace{1.5cm}
    {\Large\bfseries Trabajo de Fin de Máster\par}
    \vspace{0.5cm}
    {\large Máster Universitario en Ingeniería de la Energía \par}
    \vspace{2cm}
    {\Large Luis D. Aranda Sánchez\par}
    \vfill
    Director: Javier Rodríguez Martín
    \vfill
    {\large Septiembre 6, 2024\par}
\end{titlepage}

% Resumen (máximo de 5 páginas, incluyendo al final Palabras clave)
\clearpage
\pagestyle{simple}
% \newpage
\chapter*{Resumen}
\addcontentsline{toc}{chapter}{Resumen}
\input{capitulos/resumen/main.tex}

% Índice (paginado)
\clearpage
\pagestyle{simple}
% \newpage
\tableofcontents

% Introducción (donde se incluya los antecedentes y justificación)
\clearpage
\pagestyle{myfancy}
\newpage
\chapter{Introducción}
\input{capitulos/introduccion/main.tex}

% Objetivos
\chapter{Objetivos}
\input{capitulos/objetivos/main.tex}

% Metodología
\chapter{Metodología}
\input{capitulos/metodologia/main.tex}

% Resultados y discusión (incluyendo la valoración de impactos y de aspectos de responsabilidad legal, ética y profesional relacionados con el trabajo)
\chapter{Resultados y Discusión}
\input{capitulos/resultados_discusion/main.tex}

% Conclusiones
\chapter{Conclusiones}
\input{capitulos/conclusiones/main.tex}

% Planificación temporal y presupuesto
\chapter{Planificación Temporal y Presupuesto}
\input{capitulos/planificacion_presupuesto/main.tex}

% Bibliografía
\newpage
\addcontentsline{toc}{chapter}{Bibliografía}
\printbibliography

\end{document}


% Conclusiones
\chapter{Conclusiones}
\documentclass[a4paper,11pt,twoside]{report}
\usepackage[left=25mm,right=25mm,top=25mm,bottom=25mm,includehead,includefoot,headsep=15mm,footskip=15mm]{geometry}
\usepackage{graphicx}
\usepackage{fancyhdr}
\usepackage{titlesec}
\usepackage[spanish]{babel}
\usepackage[utf8]{inputenc}
\usepackage{amsmath}
\usepackage{setspace}
\usepackage{svg}
\usepackage{hyperref}
\usepackage[backend=biber,style=numeric]{biblatex}
\addbibresource{references.bib}
\hypersetup{
    colorlinks=true,
    linkcolor=blue,      % color of internal links (sections, etc.)
    urlcolor=blue,       % color of external links
    pdftitle={Optimización energética de sistema híbrido con bomba de calor, suelo radiante, fotovoltaica y almacenamiento para vivienda},    % title
    pdfauthor={Luis D. Aranda Sánchez},     % author
    pdfkeywords={palabra1, palabra2, código1, etc.} % list of keywords
}

% Font change to Arial
\usepackage{helvet}
\renewcommand{\familydefault}{\sfdefault}

% Chapter titles in uppercase and larger font
\titleformat{\chapter}[hang]{\large\bfseries}{\thechapter.}{1em}{\MakeUppercase}
\titleformat{\section}[hang]{\bfseries}{\thesection.}{1em}{}
\titleformat{\subsection}[hang]{\bfseries}{\thesubsection.}{1em}{}

% Fancyhdr setup
\setlength{\headheight}{14.30174pt} % Adjust to recommended value, slightly larger for safety
\fancyhf{} % Clear all headers and footers
\fancyhead[LE]{\nouppercase{\leftmark}}
\fancyhead[RO]{Optimización energética para vivienda}
\fancyfoot[LE]{\thepage}
\fancyfoot[RE]{Escuela Técnica Superior de Ingenieros Industriales (UPM)}
\fancyfoot[LO]{Luis D. Aranda Sánchez}
\fancyfoot[RO]{\thepage}
\renewcommand{\headrulewidth}{0.4pt}
\renewcommand{\footrulewidth}{0.4pt}

\fancypagestyle{myfancy}{
    \fancyhf{} % Clear all headers and footers
    \fancyhead[LE]{\nouppercase{\leftmark}}
    \fancyhead[RO]{Optimización energética para vivienda}
    \fancyfoot[LE]{\thepage}
    \fancyfoot[RE]{Escuela Técnica Superior de Ingenieros Industriales (UPM)}
    \fancyfoot[LO]{Luis D. Aranda Sánchez}
    \fancyfoot[RO]{\thepage}
    \renewcommand{\headrulewidth}{0.4pt}
    \renewcommand{\footrulewidth}{0.4pt}
}

\fancypagestyle{simple}{
    \fancyhf{} % Clear all headers and footers
    \renewcommand{\headrulewidth}{0pt}
    \renewcommand{\footrulewidth}{0pt}
}

% Line spacing
\setstretch{1.2}

% Document starts here
\begin{document}

% Portada
\begin{titlepage}
    \centering
    {\scshape\LARGE Universidad Politécnica de Madrid \par}
    \vspace{1cm}
    {\scshape\Large Escuela Técnica Superior de Ingenieros Industriales\par}
    \vspace{1.5cm}
    {\huge\bfseries Optimización energética de sistema híbrido con bomba de calor, suelo radiante, fotovoltaica y almacenamiento para vivienda \par}
    \vspace{1.5cm}
    {\Large\bfseries Trabajo de Fin de Máster\par}
    \vspace{0.5cm}
    {\large Máster Universitario en Ingeniería de la Energía \par}
    \vspace{2cm}
    {\Large Luis D. Aranda Sánchez\par}
    \vfill
    Director: Javier Rodríguez Martín
    \vfill
    {\large Septiembre 6, 2024\par}
\end{titlepage}

% Resumen (máximo de 5 páginas, incluyendo al final Palabras clave)
\clearpage
\pagestyle{simple}
% \newpage
\chapter*{Resumen}
\addcontentsline{toc}{chapter}{Resumen}
\input{capitulos/resumen/main.tex}

% Índice (paginado)
\clearpage
\pagestyle{simple}
% \newpage
\tableofcontents

% Introducción (donde se incluya los antecedentes y justificación)
\clearpage
\pagestyle{myfancy}
\newpage
\chapter{Introducción}
\input{capitulos/introduccion/main.tex}

% Objetivos
\chapter{Objetivos}
\input{capitulos/objetivos/main.tex}

% Metodología
\chapter{Metodología}
\input{capitulos/metodologia/main.tex}

% Resultados y discusión (incluyendo la valoración de impactos y de aspectos de responsabilidad legal, ética y profesional relacionados con el trabajo)
\chapter{Resultados y Discusión}
\input{capitulos/resultados_discusion/main.tex}

% Conclusiones
\chapter{Conclusiones}
\input{capitulos/conclusiones/main.tex}

% Planificación temporal y presupuesto
\chapter{Planificación Temporal y Presupuesto}
\input{capitulos/planificacion_presupuesto/main.tex}

% Bibliografía
\newpage
\addcontentsline{toc}{chapter}{Bibliografía}
\printbibliography

\end{document}


% Planificación temporal y presupuesto
\chapter{Planificación Temporal y Presupuesto}
\documentclass[a4paper,11pt,twoside]{report}
\usepackage[left=25mm,right=25mm,top=25mm,bottom=25mm,includehead,includefoot,headsep=15mm,footskip=15mm]{geometry}
\usepackage{graphicx}
\usepackage{fancyhdr}
\usepackage{titlesec}
\usepackage[spanish]{babel}
\usepackage[utf8]{inputenc}
\usepackage{amsmath}
\usepackage{setspace}
\usepackage{svg}
\usepackage{hyperref}
\usepackage[backend=biber,style=numeric]{biblatex}
\addbibresource{references.bib}
\hypersetup{
    colorlinks=true,
    linkcolor=blue,      % color of internal links (sections, etc.)
    urlcolor=blue,       % color of external links
    pdftitle={Optimización energética de sistema híbrido con bomba de calor, suelo radiante, fotovoltaica y almacenamiento para vivienda},    % title
    pdfauthor={Luis D. Aranda Sánchez},     % author
    pdfkeywords={palabra1, palabra2, código1, etc.} % list of keywords
}

% Font change to Arial
\usepackage{helvet}
\renewcommand{\familydefault}{\sfdefault}

% Chapter titles in uppercase and larger font
\titleformat{\chapter}[hang]{\large\bfseries}{\thechapter.}{1em}{\MakeUppercase}
\titleformat{\section}[hang]{\bfseries}{\thesection.}{1em}{}
\titleformat{\subsection}[hang]{\bfseries}{\thesubsection.}{1em}{}

% Fancyhdr setup
\setlength{\headheight}{14.30174pt} % Adjust to recommended value, slightly larger for safety
\fancyhf{} % Clear all headers and footers
\fancyhead[LE]{\nouppercase{\leftmark}}
\fancyhead[RO]{Optimización energética para vivienda}
\fancyfoot[LE]{\thepage}
\fancyfoot[RE]{Escuela Técnica Superior de Ingenieros Industriales (UPM)}
\fancyfoot[LO]{Luis D. Aranda Sánchez}
\fancyfoot[RO]{\thepage}
\renewcommand{\headrulewidth}{0.4pt}
\renewcommand{\footrulewidth}{0.4pt}

\fancypagestyle{myfancy}{
    \fancyhf{} % Clear all headers and footers
    \fancyhead[LE]{\nouppercase{\leftmark}}
    \fancyhead[RO]{Optimización energética para vivienda}
    \fancyfoot[LE]{\thepage}
    \fancyfoot[RE]{Escuela Técnica Superior de Ingenieros Industriales (UPM)}
    \fancyfoot[LO]{Luis D. Aranda Sánchez}
    \fancyfoot[RO]{\thepage}
    \renewcommand{\headrulewidth}{0.4pt}
    \renewcommand{\footrulewidth}{0.4pt}
}

\fancypagestyle{simple}{
    \fancyhf{} % Clear all headers and footers
    \renewcommand{\headrulewidth}{0pt}
    \renewcommand{\footrulewidth}{0pt}
}

% Line spacing
\setstretch{1.2}

% Document starts here
\begin{document}

% Portada
\begin{titlepage}
    \centering
    {\scshape\LARGE Universidad Politécnica de Madrid \par}
    \vspace{1cm}
    {\scshape\Large Escuela Técnica Superior de Ingenieros Industriales\par}
    \vspace{1.5cm}
    {\huge\bfseries Optimización energética de sistema híbrido con bomba de calor, suelo radiante, fotovoltaica y almacenamiento para vivienda \par}
    \vspace{1.5cm}
    {\Large\bfseries Trabajo de Fin de Máster\par}
    \vspace{0.5cm}
    {\large Máster Universitario en Ingeniería de la Energía \par}
    \vspace{2cm}
    {\Large Luis D. Aranda Sánchez\par}
    \vfill
    Director: Javier Rodríguez Martín
    \vfill
    {\large Septiembre 6, 2024\par}
\end{titlepage}

% Resumen (máximo de 5 páginas, incluyendo al final Palabras clave)
\clearpage
\pagestyle{simple}
% \newpage
\chapter*{Resumen}
\addcontentsline{toc}{chapter}{Resumen}
\input{capitulos/resumen/main.tex}

% Índice (paginado)
\clearpage
\pagestyle{simple}
% \newpage
\tableofcontents

% Introducción (donde se incluya los antecedentes y justificación)
\clearpage
\pagestyle{myfancy}
\newpage
\chapter{Introducción}
\input{capitulos/introduccion/main.tex}

% Objetivos
\chapter{Objetivos}
\input{capitulos/objetivos/main.tex}

% Metodología
\chapter{Metodología}
\input{capitulos/metodologia/main.tex}

% Resultados y discusión (incluyendo la valoración de impactos y de aspectos de responsabilidad legal, ética y profesional relacionados con el trabajo)
\chapter{Resultados y Discusión}
\input{capitulos/resultados_discusion/main.tex}

% Conclusiones
\chapter{Conclusiones}
\input{capitulos/conclusiones/main.tex}

% Planificación temporal y presupuesto
\chapter{Planificación Temporal y Presupuesto}
\input{capitulos/planificacion_presupuesto/main.tex}

% Bibliografía
\newpage
\addcontentsline{toc}{chapter}{Bibliografía}
\printbibliography

\end{document}


% Bibliografía
\newpage
\addcontentsline{toc}{chapter}{Bibliografía}
\printbibliography

\end{document}


% Conclusiones
\chapter{Conclusiones}
\documentclass[a4paper,11pt,twoside]{report}
\usepackage[left=25mm,right=25mm,top=25mm,bottom=25mm,includehead,includefoot,headsep=15mm,footskip=15mm]{geometry}
\usepackage{graphicx}
\usepackage{fancyhdr}
\usepackage{titlesec}
\usepackage[spanish]{babel}
\usepackage[utf8]{inputenc}
\usepackage{amsmath}
\usepackage{setspace}
\usepackage{svg}
\usepackage{hyperref}
\usepackage[backend=biber,style=numeric]{biblatex}
\addbibresource{references.bib}
\hypersetup{
    colorlinks=true,
    linkcolor=blue,      % color of internal links (sections, etc.)
    urlcolor=blue,       % color of external links
    pdftitle={Optimización energética de sistema híbrido con bomba de calor, suelo radiante, fotovoltaica y almacenamiento para vivienda},    % title
    pdfauthor={Luis D. Aranda Sánchez},     % author
    pdfkeywords={palabra1, palabra2, código1, etc.} % list of keywords
}

% Font change to Arial
\usepackage{helvet}
\renewcommand{\familydefault}{\sfdefault}

% Chapter titles in uppercase and larger font
\titleformat{\chapter}[hang]{\large\bfseries}{\thechapter.}{1em}{\MakeUppercase}
\titleformat{\section}[hang]{\bfseries}{\thesection.}{1em}{}
\titleformat{\subsection}[hang]{\bfseries}{\thesubsection.}{1em}{}

% Fancyhdr setup
\setlength{\headheight}{14.30174pt} % Adjust to recommended value, slightly larger for safety
\fancyhf{} % Clear all headers and footers
\fancyhead[LE]{\nouppercase{\leftmark}}
\fancyhead[RO]{Optimización energética para vivienda}
\fancyfoot[LE]{\thepage}
\fancyfoot[RE]{Escuela Técnica Superior de Ingenieros Industriales (UPM)}
\fancyfoot[LO]{Luis D. Aranda Sánchez}
\fancyfoot[RO]{\thepage}
\renewcommand{\headrulewidth}{0.4pt}
\renewcommand{\footrulewidth}{0.4pt}

\fancypagestyle{myfancy}{
    \fancyhf{} % Clear all headers and footers
    \fancyhead[LE]{\nouppercase{\leftmark}}
    \fancyhead[RO]{Optimización energética para vivienda}
    \fancyfoot[LE]{\thepage}
    \fancyfoot[RE]{Escuela Técnica Superior de Ingenieros Industriales (UPM)}
    \fancyfoot[LO]{Luis D. Aranda Sánchez}
    \fancyfoot[RO]{\thepage}
    \renewcommand{\headrulewidth}{0.4pt}
    \renewcommand{\footrulewidth}{0.4pt}
}

\fancypagestyle{simple}{
    \fancyhf{} % Clear all headers and footers
    \renewcommand{\headrulewidth}{0pt}
    \renewcommand{\footrulewidth}{0pt}
}

% Line spacing
\setstretch{1.2}

% Document starts here
\begin{document}

% Portada
\begin{titlepage}
    \centering
    {\scshape\LARGE Universidad Politécnica de Madrid \par}
    \vspace{1cm}
    {\scshape\Large Escuela Técnica Superior de Ingenieros Industriales\par}
    \vspace{1.5cm}
    {\huge\bfseries Optimización energética de sistema híbrido con bomba de calor, suelo radiante, fotovoltaica y almacenamiento para vivienda \par}
    \vspace{1.5cm}
    {\Large\bfseries Trabajo de Fin de Máster\par}
    \vspace{0.5cm}
    {\large Máster Universitario en Ingeniería de la Energía \par}
    \vspace{2cm}
    {\Large Luis D. Aranda Sánchez\par}
    \vfill
    Director: Javier Rodríguez Martín
    \vfill
    {\large Septiembre 6, 2024\par}
\end{titlepage}

% Resumen (máximo de 5 páginas, incluyendo al final Palabras clave)
\clearpage
\pagestyle{simple}
% \newpage
\chapter*{Resumen}
\addcontentsline{toc}{chapter}{Resumen}
\documentclass[a4paper,11pt,twoside]{report}
\usepackage[left=25mm,right=25mm,top=25mm,bottom=25mm,includehead,includefoot,headsep=15mm,footskip=15mm]{geometry}
\usepackage{graphicx}
\usepackage{fancyhdr}
\usepackage{titlesec}
\usepackage[spanish]{babel}
\usepackage[utf8]{inputenc}
\usepackage{amsmath}
\usepackage{setspace}
\usepackage{svg}
\usepackage{hyperref}
\usepackage[backend=biber,style=numeric]{biblatex}
\addbibresource{references.bib}
\hypersetup{
    colorlinks=true,
    linkcolor=blue,      % color of internal links (sections, etc.)
    urlcolor=blue,       % color of external links
    pdftitle={Optimización energética de sistema híbrido con bomba de calor, suelo radiante, fotovoltaica y almacenamiento para vivienda},    % title
    pdfauthor={Luis D. Aranda Sánchez},     % author
    pdfkeywords={palabra1, palabra2, código1, etc.} % list of keywords
}

% Font change to Arial
\usepackage{helvet}
\renewcommand{\familydefault}{\sfdefault}

% Chapter titles in uppercase and larger font
\titleformat{\chapter}[hang]{\large\bfseries}{\thechapter.}{1em}{\MakeUppercase}
\titleformat{\section}[hang]{\bfseries}{\thesection.}{1em}{}
\titleformat{\subsection}[hang]{\bfseries}{\thesubsection.}{1em}{}

% Fancyhdr setup
\setlength{\headheight}{14.30174pt} % Adjust to recommended value, slightly larger for safety
\fancyhf{} % Clear all headers and footers
\fancyhead[LE]{\nouppercase{\leftmark}}
\fancyhead[RO]{Optimización energética para vivienda}
\fancyfoot[LE]{\thepage}
\fancyfoot[RE]{Escuela Técnica Superior de Ingenieros Industriales (UPM)}
\fancyfoot[LO]{Luis D. Aranda Sánchez}
\fancyfoot[RO]{\thepage}
\renewcommand{\headrulewidth}{0.4pt}
\renewcommand{\footrulewidth}{0.4pt}

\fancypagestyle{myfancy}{
    \fancyhf{} % Clear all headers and footers
    \fancyhead[LE]{\nouppercase{\leftmark}}
    \fancyhead[RO]{Optimización energética para vivienda}
    \fancyfoot[LE]{\thepage}
    \fancyfoot[RE]{Escuela Técnica Superior de Ingenieros Industriales (UPM)}
    \fancyfoot[LO]{Luis D. Aranda Sánchez}
    \fancyfoot[RO]{\thepage}
    \renewcommand{\headrulewidth}{0.4pt}
    \renewcommand{\footrulewidth}{0.4pt}
}

\fancypagestyle{simple}{
    \fancyhf{} % Clear all headers and footers
    \renewcommand{\headrulewidth}{0pt}
    \renewcommand{\footrulewidth}{0pt}
}

% Line spacing
\setstretch{1.2}

% Document starts here
\begin{document}

% Portada
\begin{titlepage}
    \centering
    {\scshape\LARGE Universidad Politécnica de Madrid \par}
    \vspace{1cm}
    {\scshape\Large Escuela Técnica Superior de Ingenieros Industriales\par}
    \vspace{1.5cm}
    {\huge\bfseries Optimización energética de sistema híbrido con bomba de calor, suelo radiante, fotovoltaica y almacenamiento para vivienda \par}
    \vspace{1.5cm}
    {\Large\bfseries Trabajo de Fin de Máster\par}
    \vspace{0.5cm}
    {\large Máster Universitario en Ingeniería de la Energía \par}
    \vspace{2cm}
    {\Large Luis D. Aranda Sánchez\par}
    \vfill
    Director: Javier Rodríguez Martín
    \vfill
    {\large Septiembre 6, 2024\par}
\end{titlepage}

% Resumen (máximo de 5 páginas, incluyendo al final Palabras clave)
\clearpage
\pagestyle{simple}
% \newpage
\chapter*{Resumen}
\addcontentsline{toc}{chapter}{Resumen}
\input{capitulos/resumen/main.tex}

% Índice (paginado)
\clearpage
\pagestyle{simple}
% \newpage
\tableofcontents

% Introducción (donde se incluya los antecedentes y justificación)
\clearpage
\pagestyle{myfancy}
\newpage
\chapter{Introducción}
\input{capitulos/introduccion/main.tex}

% Objetivos
\chapter{Objetivos}
\input{capitulos/objetivos/main.tex}

% Metodología
\chapter{Metodología}
\input{capitulos/metodologia/main.tex}

% Resultados y discusión (incluyendo la valoración de impactos y de aspectos de responsabilidad legal, ética y profesional relacionados con el trabajo)
\chapter{Resultados y Discusión}
\input{capitulos/resultados_discusion/main.tex}

% Conclusiones
\chapter{Conclusiones}
\input{capitulos/conclusiones/main.tex}

% Planificación temporal y presupuesto
\chapter{Planificación Temporal y Presupuesto}
\input{capitulos/planificacion_presupuesto/main.tex}

% Bibliografía
\newpage
\addcontentsline{toc}{chapter}{Bibliografía}
\printbibliography

\end{document}


% Índice (paginado)
\clearpage
\pagestyle{simple}
% \newpage
\tableofcontents

% Introducción (donde se incluya los antecedentes y justificación)
\clearpage
\pagestyle{myfancy}
\newpage
\chapter{Introducción}
\documentclass[a4paper,11pt,twoside]{report}
\usepackage[left=25mm,right=25mm,top=25mm,bottom=25mm,includehead,includefoot,headsep=15mm,footskip=15mm]{geometry}
\usepackage{graphicx}
\usepackage{fancyhdr}
\usepackage{titlesec}
\usepackage[spanish]{babel}
\usepackage[utf8]{inputenc}
\usepackage{amsmath}
\usepackage{setspace}
\usepackage{svg}
\usepackage{hyperref}
\usepackage[backend=biber,style=numeric]{biblatex}
\addbibresource{references.bib}
\hypersetup{
    colorlinks=true,
    linkcolor=blue,      % color of internal links (sections, etc.)
    urlcolor=blue,       % color of external links
    pdftitle={Optimización energética de sistema híbrido con bomba de calor, suelo radiante, fotovoltaica y almacenamiento para vivienda},    % title
    pdfauthor={Luis D. Aranda Sánchez},     % author
    pdfkeywords={palabra1, palabra2, código1, etc.} % list of keywords
}

% Font change to Arial
\usepackage{helvet}
\renewcommand{\familydefault}{\sfdefault}

% Chapter titles in uppercase and larger font
\titleformat{\chapter}[hang]{\large\bfseries}{\thechapter.}{1em}{\MakeUppercase}
\titleformat{\section}[hang]{\bfseries}{\thesection.}{1em}{}
\titleformat{\subsection}[hang]{\bfseries}{\thesubsection.}{1em}{}

% Fancyhdr setup
\setlength{\headheight}{14.30174pt} % Adjust to recommended value, slightly larger for safety
\fancyhf{} % Clear all headers and footers
\fancyhead[LE]{\nouppercase{\leftmark}}
\fancyhead[RO]{Optimización energética para vivienda}
\fancyfoot[LE]{\thepage}
\fancyfoot[RE]{Escuela Técnica Superior de Ingenieros Industriales (UPM)}
\fancyfoot[LO]{Luis D. Aranda Sánchez}
\fancyfoot[RO]{\thepage}
\renewcommand{\headrulewidth}{0.4pt}
\renewcommand{\footrulewidth}{0.4pt}

\fancypagestyle{myfancy}{
    \fancyhf{} % Clear all headers and footers
    \fancyhead[LE]{\nouppercase{\leftmark}}
    \fancyhead[RO]{Optimización energética para vivienda}
    \fancyfoot[LE]{\thepage}
    \fancyfoot[RE]{Escuela Técnica Superior de Ingenieros Industriales (UPM)}
    \fancyfoot[LO]{Luis D. Aranda Sánchez}
    \fancyfoot[RO]{\thepage}
    \renewcommand{\headrulewidth}{0.4pt}
    \renewcommand{\footrulewidth}{0.4pt}
}

\fancypagestyle{simple}{
    \fancyhf{} % Clear all headers and footers
    \renewcommand{\headrulewidth}{0pt}
    \renewcommand{\footrulewidth}{0pt}
}

% Line spacing
\setstretch{1.2}

% Document starts here
\begin{document}

% Portada
\begin{titlepage}
    \centering
    {\scshape\LARGE Universidad Politécnica de Madrid \par}
    \vspace{1cm}
    {\scshape\Large Escuela Técnica Superior de Ingenieros Industriales\par}
    \vspace{1.5cm}
    {\huge\bfseries Optimización energética de sistema híbrido con bomba de calor, suelo radiante, fotovoltaica y almacenamiento para vivienda \par}
    \vspace{1.5cm}
    {\Large\bfseries Trabajo de Fin de Máster\par}
    \vspace{0.5cm}
    {\large Máster Universitario en Ingeniería de la Energía \par}
    \vspace{2cm}
    {\Large Luis D. Aranda Sánchez\par}
    \vfill
    Director: Javier Rodríguez Martín
    \vfill
    {\large Septiembre 6, 2024\par}
\end{titlepage}

% Resumen (máximo de 5 páginas, incluyendo al final Palabras clave)
\clearpage
\pagestyle{simple}
% \newpage
\chapter*{Resumen}
\addcontentsline{toc}{chapter}{Resumen}
\input{capitulos/resumen/main.tex}

% Índice (paginado)
\clearpage
\pagestyle{simple}
% \newpage
\tableofcontents

% Introducción (donde se incluya los antecedentes y justificación)
\clearpage
\pagestyle{myfancy}
\newpage
\chapter{Introducción}
\input{capitulos/introduccion/main.tex}

% Objetivos
\chapter{Objetivos}
\input{capitulos/objetivos/main.tex}

% Metodología
\chapter{Metodología}
\input{capitulos/metodologia/main.tex}

% Resultados y discusión (incluyendo la valoración de impactos y de aspectos de responsabilidad legal, ética y profesional relacionados con el trabajo)
\chapter{Resultados y Discusión}
\input{capitulos/resultados_discusion/main.tex}

% Conclusiones
\chapter{Conclusiones}
\input{capitulos/conclusiones/main.tex}

% Planificación temporal y presupuesto
\chapter{Planificación Temporal y Presupuesto}
\input{capitulos/planificacion_presupuesto/main.tex}

% Bibliografía
\newpage
\addcontentsline{toc}{chapter}{Bibliografía}
\printbibliography

\end{document}


% Objetivos
\chapter{Objetivos}
\documentclass[a4paper,11pt,twoside]{report}
\usepackage[left=25mm,right=25mm,top=25mm,bottom=25mm,includehead,includefoot,headsep=15mm,footskip=15mm]{geometry}
\usepackage{graphicx}
\usepackage{fancyhdr}
\usepackage{titlesec}
\usepackage[spanish]{babel}
\usepackage[utf8]{inputenc}
\usepackage{amsmath}
\usepackage{setspace}
\usepackage{svg}
\usepackage{hyperref}
\usepackage[backend=biber,style=numeric]{biblatex}
\addbibresource{references.bib}
\hypersetup{
    colorlinks=true,
    linkcolor=blue,      % color of internal links (sections, etc.)
    urlcolor=blue,       % color of external links
    pdftitle={Optimización energética de sistema híbrido con bomba de calor, suelo radiante, fotovoltaica y almacenamiento para vivienda},    % title
    pdfauthor={Luis D. Aranda Sánchez},     % author
    pdfkeywords={palabra1, palabra2, código1, etc.} % list of keywords
}

% Font change to Arial
\usepackage{helvet}
\renewcommand{\familydefault}{\sfdefault}

% Chapter titles in uppercase and larger font
\titleformat{\chapter}[hang]{\large\bfseries}{\thechapter.}{1em}{\MakeUppercase}
\titleformat{\section}[hang]{\bfseries}{\thesection.}{1em}{}
\titleformat{\subsection}[hang]{\bfseries}{\thesubsection.}{1em}{}

% Fancyhdr setup
\setlength{\headheight}{14.30174pt} % Adjust to recommended value, slightly larger for safety
\fancyhf{} % Clear all headers and footers
\fancyhead[LE]{\nouppercase{\leftmark}}
\fancyhead[RO]{Optimización energética para vivienda}
\fancyfoot[LE]{\thepage}
\fancyfoot[RE]{Escuela Técnica Superior de Ingenieros Industriales (UPM)}
\fancyfoot[LO]{Luis D. Aranda Sánchez}
\fancyfoot[RO]{\thepage}
\renewcommand{\headrulewidth}{0.4pt}
\renewcommand{\footrulewidth}{0.4pt}

\fancypagestyle{myfancy}{
    \fancyhf{} % Clear all headers and footers
    \fancyhead[LE]{\nouppercase{\leftmark}}
    \fancyhead[RO]{Optimización energética para vivienda}
    \fancyfoot[LE]{\thepage}
    \fancyfoot[RE]{Escuela Técnica Superior de Ingenieros Industriales (UPM)}
    \fancyfoot[LO]{Luis D. Aranda Sánchez}
    \fancyfoot[RO]{\thepage}
    \renewcommand{\headrulewidth}{0.4pt}
    \renewcommand{\footrulewidth}{0.4pt}
}

\fancypagestyle{simple}{
    \fancyhf{} % Clear all headers and footers
    \renewcommand{\headrulewidth}{0pt}
    \renewcommand{\footrulewidth}{0pt}
}

% Line spacing
\setstretch{1.2}

% Document starts here
\begin{document}

% Portada
\begin{titlepage}
    \centering
    {\scshape\LARGE Universidad Politécnica de Madrid \par}
    \vspace{1cm}
    {\scshape\Large Escuela Técnica Superior de Ingenieros Industriales\par}
    \vspace{1.5cm}
    {\huge\bfseries Optimización energética de sistema híbrido con bomba de calor, suelo radiante, fotovoltaica y almacenamiento para vivienda \par}
    \vspace{1.5cm}
    {\Large\bfseries Trabajo de Fin de Máster\par}
    \vspace{0.5cm}
    {\large Máster Universitario en Ingeniería de la Energía \par}
    \vspace{2cm}
    {\Large Luis D. Aranda Sánchez\par}
    \vfill
    Director: Javier Rodríguez Martín
    \vfill
    {\large Septiembre 6, 2024\par}
\end{titlepage}

% Resumen (máximo de 5 páginas, incluyendo al final Palabras clave)
\clearpage
\pagestyle{simple}
% \newpage
\chapter*{Resumen}
\addcontentsline{toc}{chapter}{Resumen}
\input{capitulos/resumen/main.tex}

% Índice (paginado)
\clearpage
\pagestyle{simple}
% \newpage
\tableofcontents

% Introducción (donde se incluya los antecedentes y justificación)
\clearpage
\pagestyle{myfancy}
\newpage
\chapter{Introducción}
\input{capitulos/introduccion/main.tex}

% Objetivos
\chapter{Objetivos}
\input{capitulos/objetivos/main.tex}

% Metodología
\chapter{Metodología}
\input{capitulos/metodologia/main.tex}

% Resultados y discusión (incluyendo la valoración de impactos y de aspectos de responsabilidad legal, ética y profesional relacionados con el trabajo)
\chapter{Resultados y Discusión}
\input{capitulos/resultados_discusion/main.tex}

% Conclusiones
\chapter{Conclusiones}
\input{capitulos/conclusiones/main.tex}

% Planificación temporal y presupuesto
\chapter{Planificación Temporal y Presupuesto}
\input{capitulos/planificacion_presupuesto/main.tex}

% Bibliografía
\newpage
\addcontentsline{toc}{chapter}{Bibliografía}
\printbibliography

\end{document}


% Metodología
\chapter{Metodología}
\documentclass[a4paper,11pt,twoside]{report}
\usepackage[left=25mm,right=25mm,top=25mm,bottom=25mm,includehead,includefoot,headsep=15mm,footskip=15mm]{geometry}
\usepackage{graphicx}
\usepackage{fancyhdr}
\usepackage{titlesec}
\usepackage[spanish]{babel}
\usepackage[utf8]{inputenc}
\usepackage{amsmath}
\usepackage{setspace}
\usepackage{svg}
\usepackage{hyperref}
\usepackage[backend=biber,style=numeric]{biblatex}
\addbibresource{references.bib}
\hypersetup{
    colorlinks=true,
    linkcolor=blue,      % color of internal links (sections, etc.)
    urlcolor=blue,       % color of external links
    pdftitle={Optimización energética de sistema híbrido con bomba de calor, suelo radiante, fotovoltaica y almacenamiento para vivienda},    % title
    pdfauthor={Luis D. Aranda Sánchez},     % author
    pdfkeywords={palabra1, palabra2, código1, etc.} % list of keywords
}

% Font change to Arial
\usepackage{helvet}
\renewcommand{\familydefault}{\sfdefault}

% Chapter titles in uppercase and larger font
\titleformat{\chapter}[hang]{\large\bfseries}{\thechapter.}{1em}{\MakeUppercase}
\titleformat{\section}[hang]{\bfseries}{\thesection.}{1em}{}
\titleformat{\subsection}[hang]{\bfseries}{\thesubsection.}{1em}{}

% Fancyhdr setup
\setlength{\headheight}{14.30174pt} % Adjust to recommended value, slightly larger for safety
\fancyhf{} % Clear all headers and footers
\fancyhead[LE]{\nouppercase{\leftmark}}
\fancyhead[RO]{Optimización energética para vivienda}
\fancyfoot[LE]{\thepage}
\fancyfoot[RE]{Escuela Técnica Superior de Ingenieros Industriales (UPM)}
\fancyfoot[LO]{Luis D. Aranda Sánchez}
\fancyfoot[RO]{\thepage}
\renewcommand{\headrulewidth}{0.4pt}
\renewcommand{\footrulewidth}{0.4pt}

\fancypagestyle{myfancy}{
    \fancyhf{} % Clear all headers and footers
    \fancyhead[LE]{\nouppercase{\leftmark}}
    \fancyhead[RO]{Optimización energética para vivienda}
    \fancyfoot[LE]{\thepage}
    \fancyfoot[RE]{Escuela Técnica Superior de Ingenieros Industriales (UPM)}
    \fancyfoot[LO]{Luis D. Aranda Sánchez}
    \fancyfoot[RO]{\thepage}
    \renewcommand{\headrulewidth}{0.4pt}
    \renewcommand{\footrulewidth}{0.4pt}
}

\fancypagestyle{simple}{
    \fancyhf{} % Clear all headers and footers
    \renewcommand{\headrulewidth}{0pt}
    \renewcommand{\footrulewidth}{0pt}
}

% Line spacing
\setstretch{1.2}

% Document starts here
\begin{document}

% Portada
\begin{titlepage}
    \centering
    {\scshape\LARGE Universidad Politécnica de Madrid \par}
    \vspace{1cm}
    {\scshape\Large Escuela Técnica Superior de Ingenieros Industriales\par}
    \vspace{1.5cm}
    {\huge\bfseries Optimización energética de sistema híbrido con bomba de calor, suelo radiante, fotovoltaica y almacenamiento para vivienda \par}
    \vspace{1.5cm}
    {\Large\bfseries Trabajo de Fin de Máster\par}
    \vspace{0.5cm}
    {\large Máster Universitario en Ingeniería de la Energía \par}
    \vspace{2cm}
    {\Large Luis D. Aranda Sánchez\par}
    \vfill
    Director: Javier Rodríguez Martín
    \vfill
    {\large Septiembre 6, 2024\par}
\end{titlepage}

% Resumen (máximo de 5 páginas, incluyendo al final Palabras clave)
\clearpage
\pagestyle{simple}
% \newpage
\chapter*{Resumen}
\addcontentsline{toc}{chapter}{Resumen}
\input{capitulos/resumen/main.tex}

% Índice (paginado)
\clearpage
\pagestyle{simple}
% \newpage
\tableofcontents

% Introducción (donde se incluya los antecedentes y justificación)
\clearpage
\pagestyle{myfancy}
\newpage
\chapter{Introducción}
\input{capitulos/introduccion/main.tex}

% Objetivos
\chapter{Objetivos}
\input{capitulos/objetivos/main.tex}

% Metodología
\chapter{Metodología}
\input{capitulos/metodologia/main.tex}

% Resultados y discusión (incluyendo la valoración de impactos y de aspectos de responsabilidad legal, ética y profesional relacionados con el trabajo)
\chapter{Resultados y Discusión}
\input{capitulos/resultados_discusion/main.tex}

% Conclusiones
\chapter{Conclusiones}
\input{capitulos/conclusiones/main.tex}

% Planificación temporal y presupuesto
\chapter{Planificación Temporal y Presupuesto}
\input{capitulos/planificacion_presupuesto/main.tex}

% Bibliografía
\newpage
\addcontentsline{toc}{chapter}{Bibliografía}
\printbibliography

\end{document}


% Resultados y discusión (incluyendo la valoración de impactos y de aspectos de responsabilidad legal, ética y profesional relacionados con el trabajo)
\chapter{Resultados y Discusión}
\documentclass[a4paper,11pt,twoside]{report}
\usepackage[left=25mm,right=25mm,top=25mm,bottom=25mm,includehead,includefoot,headsep=15mm,footskip=15mm]{geometry}
\usepackage{graphicx}
\usepackage{fancyhdr}
\usepackage{titlesec}
\usepackage[spanish]{babel}
\usepackage[utf8]{inputenc}
\usepackage{amsmath}
\usepackage{setspace}
\usepackage{svg}
\usepackage{hyperref}
\usepackage[backend=biber,style=numeric]{biblatex}
\addbibresource{references.bib}
\hypersetup{
    colorlinks=true,
    linkcolor=blue,      % color of internal links (sections, etc.)
    urlcolor=blue,       % color of external links
    pdftitle={Optimización energética de sistema híbrido con bomba de calor, suelo radiante, fotovoltaica y almacenamiento para vivienda},    % title
    pdfauthor={Luis D. Aranda Sánchez},     % author
    pdfkeywords={palabra1, palabra2, código1, etc.} % list of keywords
}

% Font change to Arial
\usepackage{helvet}
\renewcommand{\familydefault}{\sfdefault}

% Chapter titles in uppercase and larger font
\titleformat{\chapter}[hang]{\large\bfseries}{\thechapter.}{1em}{\MakeUppercase}
\titleformat{\section}[hang]{\bfseries}{\thesection.}{1em}{}
\titleformat{\subsection}[hang]{\bfseries}{\thesubsection.}{1em}{}

% Fancyhdr setup
\setlength{\headheight}{14.30174pt} % Adjust to recommended value, slightly larger for safety
\fancyhf{} % Clear all headers and footers
\fancyhead[LE]{\nouppercase{\leftmark}}
\fancyhead[RO]{Optimización energética para vivienda}
\fancyfoot[LE]{\thepage}
\fancyfoot[RE]{Escuela Técnica Superior de Ingenieros Industriales (UPM)}
\fancyfoot[LO]{Luis D. Aranda Sánchez}
\fancyfoot[RO]{\thepage}
\renewcommand{\headrulewidth}{0.4pt}
\renewcommand{\footrulewidth}{0.4pt}

\fancypagestyle{myfancy}{
    \fancyhf{} % Clear all headers and footers
    \fancyhead[LE]{\nouppercase{\leftmark}}
    \fancyhead[RO]{Optimización energética para vivienda}
    \fancyfoot[LE]{\thepage}
    \fancyfoot[RE]{Escuela Técnica Superior de Ingenieros Industriales (UPM)}
    \fancyfoot[LO]{Luis D. Aranda Sánchez}
    \fancyfoot[RO]{\thepage}
    \renewcommand{\headrulewidth}{0.4pt}
    \renewcommand{\footrulewidth}{0.4pt}
}

\fancypagestyle{simple}{
    \fancyhf{} % Clear all headers and footers
    \renewcommand{\headrulewidth}{0pt}
    \renewcommand{\footrulewidth}{0pt}
}

% Line spacing
\setstretch{1.2}

% Document starts here
\begin{document}

% Portada
\begin{titlepage}
    \centering
    {\scshape\LARGE Universidad Politécnica de Madrid \par}
    \vspace{1cm}
    {\scshape\Large Escuela Técnica Superior de Ingenieros Industriales\par}
    \vspace{1.5cm}
    {\huge\bfseries Optimización energética de sistema híbrido con bomba de calor, suelo radiante, fotovoltaica y almacenamiento para vivienda \par}
    \vspace{1.5cm}
    {\Large\bfseries Trabajo de Fin de Máster\par}
    \vspace{0.5cm}
    {\large Máster Universitario en Ingeniería de la Energía \par}
    \vspace{2cm}
    {\Large Luis D. Aranda Sánchez\par}
    \vfill
    Director: Javier Rodríguez Martín
    \vfill
    {\large Septiembre 6, 2024\par}
\end{titlepage}

% Resumen (máximo de 5 páginas, incluyendo al final Palabras clave)
\clearpage
\pagestyle{simple}
% \newpage
\chapter*{Resumen}
\addcontentsline{toc}{chapter}{Resumen}
\input{capitulos/resumen/main.tex}

% Índice (paginado)
\clearpage
\pagestyle{simple}
% \newpage
\tableofcontents

% Introducción (donde se incluya los antecedentes y justificación)
\clearpage
\pagestyle{myfancy}
\newpage
\chapter{Introducción}
\input{capitulos/introduccion/main.tex}

% Objetivos
\chapter{Objetivos}
\input{capitulos/objetivos/main.tex}

% Metodología
\chapter{Metodología}
\input{capitulos/metodologia/main.tex}

% Resultados y discusión (incluyendo la valoración de impactos y de aspectos de responsabilidad legal, ética y profesional relacionados con el trabajo)
\chapter{Resultados y Discusión}
\input{capitulos/resultados_discusion/main.tex}

% Conclusiones
\chapter{Conclusiones}
\input{capitulos/conclusiones/main.tex}

% Planificación temporal y presupuesto
\chapter{Planificación Temporal y Presupuesto}
\input{capitulos/planificacion_presupuesto/main.tex}

% Bibliografía
\newpage
\addcontentsline{toc}{chapter}{Bibliografía}
\printbibliography

\end{document}


% Conclusiones
\chapter{Conclusiones}
\documentclass[a4paper,11pt,twoside]{report}
\usepackage[left=25mm,right=25mm,top=25mm,bottom=25mm,includehead,includefoot,headsep=15mm,footskip=15mm]{geometry}
\usepackage{graphicx}
\usepackage{fancyhdr}
\usepackage{titlesec}
\usepackage[spanish]{babel}
\usepackage[utf8]{inputenc}
\usepackage{amsmath}
\usepackage{setspace}
\usepackage{svg}
\usepackage{hyperref}
\usepackage[backend=biber,style=numeric]{biblatex}
\addbibresource{references.bib}
\hypersetup{
    colorlinks=true,
    linkcolor=blue,      % color of internal links (sections, etc.)
    urlcolor=blue,       % color of external links
    pdftitle={Optimización energética de sistema híbrido con bomba de calor, suelo radiante, fotovoltaica y almacenamiento para vivienda},    % title
    pdfauthor={Luis D. Aranda Sánchez},     % author
    pdfkeywords={palabra1, palabra2, código1, etc.} % list of keywords
}

% Font change to Arial
\usepackage{helvet}
\renewcommand{\familydefault}{\sfdefault}

% Chapter titles in uppercase and larger font
\titleformat{\chapter}[hang]{\large\bfseries}{\thechapter.}{1em}{\MakeUppercase}
\titleformat{\section}[hang]{\bfseries}{\thesection.}{1em}{}
\titleformat{\subsection}[hang]{\bfseries}{\thesubsection.}{1em}{}

% Fancyhdr setup
\setlength{\headheight}{14.30174pt} % Adjust to recommended value, slightly larger for safety
\fancyhf{} % Clear all headers and footers
\fancyhead[LE]{\nouppercase{\leftmark}}
\fancyhead[RO]{Optimización energética para vivienda}
\fancyfoot[LE]{\thepage}
\fancyfoot[RE]{Escuela Técnica Superior de Ingenieros Industriales (UPM)}
\fancyfoot[LO]{Luis D. Aranda Sánchez}
\fancyfoot[RO]{\thepage}
\renewcommand{\headrulewidth}{0.4pt}
\renewcommand{\footrulewidth}{0.4pt}

\fancypagestyle{myfancy}{
    \fancyhf{} % Clear all headers and footers
    \fancyhead[LE]{\nouppercase{\leftmark}}
    \fancyhead[RO]{Optimización energética para vivienda}
    \fancyfoot[LE]{\thepage}
    \fancyfoot[RE]{Escuela Técnica Superior de Ingenieros Industriales (UPM)}
    \fancyfoot[LO]{Luis D. Aranda Sánchez}
    \fancyfoot[RO]{\thepage}
    \renewcommand{\headrulewidth}{0.4pt}
    \renewcommand{\footrulewidth}{0.4pt}
}

\fancypagestyle{simple}{
    \fancyhf{} % Clear all headers and footers
    \renewcommand{\headrulewidth}{0pt}
    \renewcommand{\footrulewidth}{0pt}
}

% Line spacing
\setstretch{1.2}

% Document starts here
\begin{document}

% Portada
\begin{titlepage}
    \centering
    {\scshape\LARGE Universidad Politécnica de Madrid \par}
    \vspace{1cm}
    {\scshape\Large Escuela Técnica Superior de Ingenieros Industriales\par}
    \vspace{1.5cm}
    {\huge\bfseries Optimización energética de sistema híbrido con bomba de calor, suelo radiante, fotovoltaica y almacenamiento para vivienda \par}
    \vspace{1.5cm}
    {\Large\bfseries Trabajo de Fin de Máster\par}
    \vspace{0.5cm}
    {\large Máster Universitario en Ingeniería de la Energía \par}
    \vspace{2cm}
    {\Large Luis D. Aranda Sánchez\par}
    \vfill
    Director: Javier Rodríguez Martín
    \vfill
    {\large Septiembre 6, 2024\par}
\end{titlepage}

% Resumen (máximo de 5 páginas, incluyendo al final Palabras clave)
\clearpage
\pagestyle{simple}
% \newpage
\chapter*{Resumen}
\addcontentsline{toc}{chapter}{Resumen}
\input{capitulos/resumen/main.tex}

% Índice (paginado)
\clearpage
\pagestyle{simple}
% \newpage
\tableofcontents

% Introducción (donde se incluya los antecedentes y justificación)
\clearpage
\pagestyle{myfancy}
\newpage
\chapter{Introducción}
\input{capitulos/introduccion/main.tex}

% Objetivos
\chapter{Objetivos}
\input{capitulos/objetivos/main.tex}

% Metodología
\chapter{Metodología}
\input{capitulos/metodologia/main.tex}

% Resultados y discusión (incluyendo la valoración de impactos y de aspectos de responsabilidad legal, ética y profesional relacionados con el trabajo)
\chapter{Resultados y Discusión}
\input{capitulos/resultados_discusion/main.tex}

% Conclusiones
\chapter{Conclusiones}
\input{capitulos/conclusiones/main.tex}

% Planificación temporal y presupuesto
\chapter{Planificación Temporal y Presupuesto}
\input{capitulos/planificacion_presupuesto/main.tex}

% Bibliografía
\newpage
\addcontentsline{toc}{chapter}{Bibliografía}
\printbibliography

\end{document}


% Planificación temporal y presupuesto
\chapter{Planificación Temporal y Presupuesto}
\documentclass[a4paper,11pt,twoside]{report}
\usepackage[left=25mm,right=25mm,top=25mm,bottom=25mm,includehead,includefoot,headsep=15mm,footskip=15mm]{geometry}
\usepackage{graphicx}
\usepackage{fancyhdr}
\usepackage{titlesec}
\usepackage[spanish]{babel}
\usepackage[utf8]{inputenc}
\usepackage{amsmath}
\usepackage{setspace}
\usepackage{svg}
\usepackage{hyperref}
\usepackage[backend=biber,style=numeric]{biblatex}
\addbibresource{references.bib}
\hypersetup{
    colorlinks=true,
    linkcolor=blue,      % color of internal links (sections, etc.)
    urlcolor=blue,       % color of external links
    pdftitle={Optimización energética de sistema híbrido con bomba de calor, suelo radiante, fotovoltaica y almacenamiento para vivienda},    % title
    pdfauthor={Luis D. Aranda Sánchez},     % author
    pdfkeywords={palabra1, palabra2, código1, etc.} % list of keywords
}

% Font change to Arial
\usepackage{helvet}
\renewcommand{\familydefault}{\sfdefault}

% Chapter titles in uppercase and larger font
\titleformat{\chapter}[hang]{\large\bfseries}{\thechapter.}{1em}{\MakeUppercase}
\titleformat{\section}[hang]{\bfseries}{\thesection.}{1em}{}
\titleformat{\subsection}[hang]{\bfseries}{\thesubsection.}{1em}{}

% Fancyhdr setup
\setlength{\headheight}{14.30174pt} % Adjust to recommended value, slightly larger for safety
\fancyhf{} % Clear all headers and footers
\fancyhead[LE]{\nouppercase{\leftmark}}
\fancyhead[RO]{Optimización energética para vivienda}
\fancyfoot[LE]{\thepage}
\fancyfoot[RE]{Escuela Técnica Superior de Ingenieros Industriales (UPM)}
\fancyfoot[LO]{Luis D. Aranda Sánchez}
\fancyfoot[RO]{\thepage}
\renewcommand{\headrulewidth}{0.4pt}
\renewcommand{\footrulewidth}{0.4pt}

\fancypagestyle{myfancy}{
    \fancyhf{} % Clear all headers and footers
    \fancyhead[LE]{\nouppercase{\leftmark}}
    \fancyhead[RO]{Optimización energética para vivienda}
    \fancyfoot[LE]{\thepage}
    \fancyfoot[RE]{Escuela Técnica Superior de Ingenieros Industriales (UPM)}
    \fancyfoot[LO]{Luis D. Aranda Sánchez}
    \fancyfoot[RO]{\thepage}
    \renewcommand{\headrulewidth}{0.4pt}
    \renewcommand{\footrulewidth}{0.4pt}
}

\fancypagestyle{simple}{
    \fancyhf{} % Clear all headers and footers
    \renewcommand{\headrulewidth}{0pt}
    \renewcommand{\footrulewidth}{0pt}
}

% Line spacing
\setstretch{1.2}

% Document starts here
\begin{document}

% Portada
\begin{titlepage}
    \centering
    {\scshape\LARGE Universidad Politécnica de Madrid \par}
    \vspace{1cm}
    {\scshape\Large Escuela Técnica Superior de Ingenieros Industriales\par}
    \vspace{1.5cm}
    {\huge\bfseries Optimización energética de sistema híbrido con bomba de calor, suelo radiante, fotovoltaica y almacenamiento para vivienda \par}
    \vspace{1.5cm}
    {\Large\bfseries Trabajo de Fin de Máster\par}
    \vspace{0.5cm}
    {\large Máster Universitario en Ingeniería de la Energía \par}
    \vspace{2cm}
    {\Large Luis D. Aranda Sánchez\par}
    \vfill
    Director: Javier Rodríguez Martín
    \vfill
    {\large Septiembre 6, 2024\par}
\end{titlepage}

% Resumen (máximo de 5 páginas, incluyendo al final Palabras clave)
\clearpage
\pagestyle{simple}
% \newpage
\chapter*{Resumen}
\addcontentsline{toc}{chapter}{Resumen}
\input{capitulos/resumen/main.tex}

% Índice (paginado)
\clearpage
\pagestyle{simple}
% \newpage
\tableofcontents

% Introducción (donde se incluya los antecedentes y justificación)
\clearpage
\pagestyle{myfancy}
\newpage
\chapter{Introducción}
\input{capitulos/introduccion/main.tex}

% Objetivos
\chapter{Objetivos}
\input{capitulos/objetivos/main.tex}

% Metodología
\chapter{Metodología}
\input{capitulos/metodologia/main.tex}

% Resultados y discusión (incluyendo la valoración de impactos y de aspectos de responsabilidad legal, ética y profesional relacionados con el trabajo)
\chapter{Resultados y Discusión}
\input{capitulos/resultados_discusion/main.tex}

% Conclusiones
\chapter{Conclusiones}
\input{capitulos/conclusiones/main.tex}

% Planificación temporal y presupuesto
\chapter{Planificación Temporal y Presupuesto}
\input{capitulos/planificacion_presupuesto/main.tex}

% Bibliografía
\newpage
\addcontentsline{toc}{chapter}{Bibliografía}
\printbibliography

\end{document}


% Bibliografía
\newpage
\addcontentsline{toc}{chapter}{Bibliografía}
\printbibliography

\end{document}


% Planificación temporal y presupuesto
\chapter{Planificación Temporal y Presupuesto}
\documentclass[a4paper,11pt,twoside]{report}
\usepackage[left=25mm,right=25mm,top=25mm,bottom=25mm,includehead,includefoot,headsep=15mm,footskip=15mm]{geometry}
\usepackage{graphicx}
\usepackage{fancyhdr}
\usepackage{titlesec}
\usepackage[spanish]{babel}
\usepackage[utf8]{inputenc}
\usepackage{amsmath}
\usepackage{setspace}
\usepackage{svg}
\usepackage{hyperref}
\usepackage[backend=biber,style=numeric]{biblatex}
\addbibresource{references.bib}
\hypersetup{
    colorlinks=true,
    linkcolor=blue,      % color of internal links (sections, etc.)
    urlcolor=blue,       % color of external links
    pdftitle={Optimización energética de sistema híbrido con bomba de calor, suelo radiante, fotovoltaica y almacenamiento para vivienda},    % title
    pdfauthor={Luis D. Aranda Sánchez},     % author
    pdfkeywords={palabra1, palabra2, código1, etc.} % list of keywords
}

% Font change to Arial
\usepackage{helvet}
\renewcommand{\familydefault}{\sfdefault}

% Chapter titles in uppercase and larger font
\titleformat{\chapter}[hang]{\large\bfseries}{\thechapter.}{1em}{\MakeUppercase}
\titleformat{\section}[hang]{\bfseries}{\thesection.}{1em}{}
\titleformat{\subsection}[hang]{\bfseries}{\thesubsection.}{1em}{}

% Fancyhdr setup
\setlength{\headheight}{14.30174pt} % Adjust to recommended value, slightly larger for safety
\fancyhf{} % Clear all headers and footers
\fancyhead[LE]{\nouppercase{\leftmark}}
\fancyhead[RO]{Optimización energética para vivienda}
\fancyfoot[LE]{\thepage}
\fancyfoot[RE]{Escuela Técnica Superior de Ingenieros Industriales (UPM)}
\fancyfoot[LO]{Luis D. Aranda Sánchez}
\fancyfoot[RO]{\thepage}
\renewcommand{\headrulewidth}{0.4pt}
\renewcommand{\footrulewidth}{0.4pt}

\fancypagestyle{myfancy}{
    \fancyhf{} % Clear all headers and footers
    \fancyhead[LE]{\nouppercase{\leftmark}}
    \fancyhead[RO]{Optimización energética para vivienda}
    \fancyfoot[LE]{\thepage}
    \fancyfoot[RE]{Escuela Técnica Superior de Ingenieros Industriales (UPM)}
    \fancyfoot[LO]{Luis D. Aranda Sánchez}
    \fancyfoot[RO]{\thepage}
    \renewcommand{\headrulewidth}{0.4pt}
    \renewcommand{\footrulewidth}{0.4pt}
}

\fancypagestyle{simple}{
    \fancyhf{} % Clear all headers and footers
    \renewcommand{\headrulewidth}{0pt}
    \renewcommand{\footrulewidth}{0pt}
}

% Line spacing
\setstretch{1.2}

% Document starts here
\begin{document}

% Portada
\begin{titlepage}
    \centering
    {\scshape\LARGE Universidad Politécnica de Madrid \par}
    \vspace{1cm}
    {\scshape\Large Escuela Técnica Superior de Ingenieros Industriales\par}
    \vspace{1.5cm}
    {\huge\bfseries Optimización energética de sistema híbrido con bomba de calor, suelo radiante, fotovoltaica y almacenamiento para vivienda \par}
    \vspace{1.5cm}
    {\Large\bfseries Trabajo de Fin de Máster\par}
    \vspace{0.5cm}
    {\large Máster Universitario en Ingeniería de la Energía \par}
    \vspace{2cm}
    {\Large Luis D. Aranda Sánchez\par}
    \vfill
    Director: Javier Rodríguez Martín
    \vfill
    {\large Septiembre 6, 2024\par}
\end{titlepage}

% Resumen (máximo de 5 páginas, incluyendo al final Palabras clave)
\clearpage
\pagestyle{simple}
% \newpage
\chapter*{Resumen}
\addcontentsline{toc}{chapter}{Resumen}
\documentclass[a4paper,11pt,twoside]{report}
\usepackage[left=25mm,right=25mm,top=25mm,bottom=25mm,includehead,includefoot,headsep=15mm,footskip=15mm]{geometry}
\usepackage{graphicx}
\usepackage{fancyhdr}
\usepackage{titlesec}
\usepackage[spanish]{babel}
\usepackage[utf8]{inputenc}
\usepackage{amsmath}
\usepackage{setspace}
\usepackage{svg}
\usepackage{hyperref}
\usepackage[backend=biber,style=numeric]{biblatex}
\addbibresource{references.bib}
\hypersetup{
    colorlinks=true,
    linkcolor=blue,      % color of internal links (sections, etc.)
    urlcolor=blue,       % color of external links
    pdftitle={Optimización energética de sistema híbrido con bomba de calor, suelo radiante, fotovoltaica y almacenamiento para vivienda},    % title
    pdfauthor={Luis D. Aranda Sánchez},     % author
    pdfkeywords={palabra1, palabra2, código1, etc.} % list of keywords
}

% Font change to Arial
\usepackage{helvet}
\renewcommand{\familydefault}{\sfdefault}

% Chapter titles in uppercase and larger font
\titleformat{\chapter}[hang]{\large\bfseries}{\thechapter.}{1em}{\MakeUppercase}
\titleformat{\section}[hang]{\bfseries}{\thesection.}{1em}{}
\titleformat{\subsection}[hang]{\bfseries}{\thesubsection.}{1em}{}

% Fancyhdr setup
\setlength{\headheight}{14.30174pt} % Adjust to recommended value, slightly larger for safety
\fancyhf{} % Clear all headers and footers
\fancyhead[LE]{\nouppercase{\leftmark}}
\fancyhead[RO]{Optimización energética para vivienda}
\fancyfoot[LE]{\thepage}
\fancyfoot[RE]{Escuela Técnica Superior de Ingenieros Industriales (UPM)}
\fancyfoot[LO]{Luis D. Aranda Sánchez}
\fancyfoot[RO]{\thepage}
\renewcommand{\headrulewidth}{0.4pt}
\renewcommand{\footrulewidth}{0.4pt}

\fancypagestyle{myfancy}{
    \fancyhf{} % Clear all headers and footers
    \fancyhead[LE]{\nouppercase{\leftmark}}
    \fancyhead[RO]{Optimización energética para vivienda}
    \fancyfoot[LE]{\thepage}
    \fancyfoot[RE]{Escuela Técnica Superior de Ingenieros Industriales (UPM)}
    \fancyfoot[LO]{Luis D. Aranda Sánchez}
    \fancyfoot[RO]{\thepage}
    \renewcommand{\headrulewidth}{0.4pt}
    \renewcommand{\footrulewidth}{0.4pt}
}

\fancypagestyle{simple}{
    \fancyhf{} % Clear all headers and footers
    \renewcommand{\headrulewidth}{0pt}
    \renewcommand{\footrulewidth}{0pt}
}

% Line spacing
\setstretch{1.2}

% Document starts here
\begin{document}

% Portada
\begin{titlepage}
    \centering
    {\scshape\LARGE Universidad Politécnica de Madrid \par}
    \vspace{1cm}
    {\scshape\Large Escuela Técnica Superior de Ingenieros Industriales\par}
    \vspace{1.5cm}
    {\huge\bfseries Optimización energética de sistema híbrido con bomba de calor, suelo radiante, fotovoltaica y almacenamiento para vivienda \par}
    \vspace{1.5cm}
    {\Large\bfseries Trabajo de Fin de Máster\par}
    \vspace{0.5cm}
    {\large Máster Universitario en Ingeniería de la Energía \par}
    \vspace{2cm}
    {\Large Luis D. Aranda Sánchez\par}
    \vfill
    Director: Javier Rodríguez Martín
    \vfill
    {\large Septiembre 6, 2024\par}
\end{titlepage}

% Resumen (máximo de 5 páginas, incluyendo al final Palabras clave)
\clearpage
\pagestyle{simple}
% \newpage
\chapter*{Resumen}
\addcontentsline{toc}{chapter}{Resumen}
\input{capitulos/resumen/main.tex}

% Índice (paginado)
\clearpage
\pagestyle{simple}
% \newpage
\tableofcontents

% Introducción (donde se incluya los antecedentes y justificación)
\clearpage
\pagestyle{myfancy}
\newpage
\chapter{Introducción}
\input{capitulos/introduccion/main.tex}

% Objetivos
\chapter{Objetivos}
\input{capitulos/objetivos/main.tex}

% Metodología
\chapter{Metodología}
\input{capitulos/metodologia/main.tex}

% Resultados y discusión (incluyendo la valoración de impactos y de aspectos de responsabilidad legal, ética y profesional relacionados con el trabajo)
\chapter{Resultados y Discusión}
\input{capitulos/resultados_discusion/main.tex}

% Conclusiones
\chapter{Conclusiones}
\input{capitulos/conclusiones/main.tex}

% Planificación temporal y presupuesto
\chapter{Planificación Temporal y Presupuesto}
\input{capitulos/planificacion_presupuesto/main.tex}

% Bibliografía
\newpage
\addcontentsline{toc}{chapter}{Bibliografía}
\printbibliography

\end{document}


% Índice (paginado)
\clearpage
\pagestyle{simple}
% \newpage
\tableofcontents

% Introducción (donde se incluya los antecedentes y justificación)
\clearpage
\pagestyle{myfancy}
\newpage
\chapter{Introducción}
\documentclass[a4paper,11pt,twoside]{report}
\usepackage[left=25mm,right=25mm,top=25mm,bottom=25mm,includehead,includefoot,headsep=15mm,footskip=15mm]{geometry}
\usepackage{graphicx}
\usepackage{fancyhdr}
\usepackage{titlesec}
\usepackage[spanish]{babel}
\usepackage[utf8]{inputenc}
\usepackage{amsmath}
\usepackage{setspace}
\usepackage{svg}
\usepackage{hyperref}
\usepackage[backend=biber,style=numeric]{biblatex}
\addbibresource{references.bib}
\hypersetup{
    colorlinks=true,
    linkcolor=blue,      % color of internal links (sections, etc.)
    urlcolor=blue,       % color of external links
    pdftitle={Optimización energética de sistema híbrido con bomba de calor, suelo radiante, fotovoltaica y almacenamiento para vivienda},    % title
    pdfauthor={Luis D. Aranda Sánchez},     % author
    pdfkeywords={palabra1, palabra2, código1, etc.} % list of keywords
}

% Font change to Arial
\usepackage{helvet}
\renewcommand{\familydefault}{\sfdefault}

% Chapter titles in uppercase and larger font
\titleformat{\chapter}[hang]{\large\bfseries}{\thechapter.}{1em}{\MakeUppercase}
\titleformat{\section}[hang]{\bfseries}{\thesection.}{1em}{}
\titleformat{\subsection}[hang]{\bfseries}{\thesubsection.}{1em}{}

% Fancyhdr setup
\setlength{\headheight}{14.30174pt} % Adjust to recommended value, slightly larger for safety
\fancyhf{} % Clear all headers and footers
\fancyhead[LE]{\nouppercase{\leftmark}}
\fancyhead[RO]{Optimización energética para vivienda}
\fancyfoot[LE]{\thepage}
\fancyfoot[RE]{Escuela Técnica Superior de Ingenieros Industriales (UPM)}
\fancyfoot[LO]{Luis D. Aranda Sánchez}
\fancyfoot[RO]{\thepage}
\renewcommand{\headrulewidth}{0.4pt}
\renewcommand{\footrulewidth}{0.4pt}

\fancypagestyle{myfancy}{
    \fancyhf{} % Clear all headers and footers
    \fancyhead[LE]{\nouppercase{\leftmark}}
    \fancyhead[RO]{Optimización energética para vivienda}
    \fancyfoot[LE]{\thepage}
    \fancyfoot[RE]{Escuela Técnica Superior de Ingenieros Industriales (UPM)}
    \fancyfoot[LO]{Luis D. Aranda Sánchez}
    \fancyfoot[RO]{\thepage}
    \renewcommand{\headrulewidth}{0.4pt}
    \renewcommand{\footrulewidth}{0.4pt}
}

\fancypagestyle{simple}{
    \fancyhf{} % Clear all headers and footers
    \renewcommand{\headrulewidth}{0pt}
    \renewcommand{\footrulewidth}{0pt}
}

% Line spacing
\setstretch{1.2}

% Document starts here
\begin{document}

% Portada
\begin{titlepage}
    \centering
    {\scshape\LARGE Universidad Politécnica de Madrid \par}
    \vspace{1cm}
    {\scshape\Large Escuela Técnica Superior de Ingenieros Industriales\par}
    \vspace{1.5cm}
    {\huge\bfseries Optimización energética de sistema híbrido con bomba de calor, suelo radiante, fotovoltaica y almacenamiento para vivienda \par}
    \vspace{1.5cm}
    {\Large\bfseries Trabajo de Fin de Máster\par}
    \vspace{0.5cm}
    {\large Máster Universitario en Ingeniería de la Energía \par}
    \vspace{2cm}
    {\Large Luis D. Aranda Sánchez\par}
    \vfill
    Director: Javier Rodríguez Martín
    \vfill
    {\large Septiembre 6, 2024\par}
\end{titlepage}

% Resumen (máximo de 5 páginas, incluyendo al final Palabras clave)
\clearpage
\pagestyle{simple}
% \newpage
\chapter*{Resumen}
\addcontentsline{toc}{chapter}{Resumen}
\input{capitulos/resumen/main.tex}

% Índice (paginado)
\clearpage
\pagestyle{simple}
% \newpage
\tableofcontents

% Introducción (donde se incluya los antecedentes y justificación)
\clearpage
\pagestyle{myfancy}
\newpage
\chapter{Introducción}
\input{capitulos/introduccion/main.tex}

% Objetivos
\chapter{Objetivos}
\input{capitulos/objetivos/main.tex}

% Metodología
\chapter{Metodología}
\input{capitulos/metodologia/main.tex}

% Resultados y discusión (incluyendo la valoración de impactos y de aspectos de responsabilidad legal, ética y profesional relacionados con el trabajo)
\chapter{Resultados y Discusión}
\input{capitulos/resultados_discusion/main.tex}

% Conclusiones
\chapter{Conclusiones}
\input{capitulos/conclusiones/main.tex}

% Planificación temporal y presupuesto
\chapter{Planificación Temporal y Presupuesto}
\input{capitulos/planificacion_presupuesto/main.tex}

% Bibliografía
\newpage
\addcontentsline{toc}{chapter}{Bibliografía}
\printbibliography

\end{document}


% Objetivos
\chapter{Objetivos}
\documentclass[a4paper,11pt,twoside]{report}
\usepackage[left=25mm,right=25mm,top=25mm,bottom=25mm,includehead,includefoot,headsep=15mm,footskip=15mm]{geometry}
\usepackage{graphicx}
\usepackage{fancyhdr}
\usepackage{titlesec}
\usepackage[spanish]{babel}
\usepackage[utf8]{inputenc}
\usepackage{amsmath}
\usepackage{setspace}
\usepackage{svg}
\usepackage{hyperref}
\usepackage[backend=biber,style=numeric]{biblatex}
\addbibresource{references.bib}
\hypersetup{
    colorlinks=true,
    linkcolor=blue,      % color of internal links (sections, etc.)
    urlcolor=blue,       % color of external links
    pdftitle={Optimización energética de sistema híbrido con bomba de calor, suelo radiante, fotovoltaica y almacenamiento para vivienda},    % title
    pdfauthor={Luis D. Aranda Sánchez},     % author
    pdfkeywords={palabra1, palabra2, código1, etc.} % list of keywords
}

% Font change to Arial
\usepackage{helvet}
\renewcommand{\familydefault}{\sfdefault}

% Chapter titles in uppercase and larger font
\titleformat{\chapter}[hang]{\large\bfseries}{\thechapter.}{1em}{\MakeUppercase}
\titleformat{\section}[hang]{\bfseries}{\thesection.}{1em}{}
\titleformat{\subsection}[hang]{\bfseries}{\thesubsection.}{1em}{}

% Fancyhdr setup
\setlength{\headheight}{14.30174pt} % Adjust to recommended value, slightly larger for safety
\fancyhf{} % Clear all headers and footers
\fancyhead[LE]{\nouppercase{\leftmark}}
\fancyhead[RO]{Optimización energética para vivienda}
\fancyfoot[LE]{\thepage}
\fancyfoot[RE]{Escuela Técnica Superior de Ingenieros Industriales (UPM)}
\fancyfoot[LO]{Luis D. Aranda Sánchez}
\fancyfoot[RO]{\thepage}
\renewcommand{\headrulewidth}{0.4pt}
\renewcommand{\footrulewidth}{0.4pt}

\fancypagestyle{myfancy}{
    \fancyhf{} % Clear all headers and footers
    \fancyhead[LE]{\nouppercase{\leftmark}}
    \fancyhead[RO]{Optimización energética para vivienda}
    \fancyfoot[LE]{\thepage}
    \fancyfoot[RE]{Escuela Técnica Superior de Ingenieros Industriales (UPM)}
    \fancyfoot[LO]{Luis D. Aranda Sánchez}
    \fancyfoot[RO]{\thepage}
    \renewcommand{\headrulewidth}{0.4pt}
    \renewcommand{\footrulewidth}{0.4pt}
}

\fancypagestyle{simple}{
    \fancyhf{} % Clear all headers and footers
    \renewcommand{\headrulewidth}{0pt}
    \renewcommand{\footrulewidth}{0pt}
}

% Line spacing
\setstretch{1.2}

% Document starts here
\begin{document}

% Portada
\begin{titlepage}
    \centering
    {\scshape\LARGE Universidad Politécnica de Madrid \par}
    \vspace{1cm}
    {\scshape\Large Escuela Técnica Superior de Ingenieros Industriales\par}
    \vspace{1.5cm}
    {\huge\bfseries Optimización energética de sistema híbrido con bomba de calor, suelo radiante, fotovoltaica y almacenamiento para vivienda \par}
    \vspace{1.5cm}
    {\Large\bfseries Trabajo de Fin de Máster\par}
    \vspace{0.5cm}
    {\large Máster Universitario en Ingeniería de la Energía \par}
    \vspace{2cm}
    {\Large Luis D. Aranda Sánchez\par}
    \vfill
    Director: Javier Rodríguez Martín
    \vfill
    {\large Septiembre 6, 2024\par}
\end{titlepage}

% Resumen (máximo de 5 páginas, incluyendo al final Palabras clave)
\clearpage
\pagestyle{simple}
% \newpage
\chapter*{Resumen}
\addcontentsline{toc}{chapter}{Resumen}
\input{capitulos/resumen/main.tex}

% Índice (paginado)
\clearpage
\pagestyle{simple}
% \newpage
\tableofcontents

% Introducción (donde se incluya los antecedentes y justificación)
\clearpage
\pagestyle{myfancy}
\newpage
\chapter{Introducción}
\input{capitulos/introduccion/main.tex}

% Objetivos
\chapter{Objetivos}
\input{capitulos/objetivos/main.tex}

% Metodología
\chapter{Metodología}
\input{capitulos/metodologia/main.tex}

% Resultados y discusión (incluyendo la valoración de impactos y de aspectos de responsabilidad legal, ética y profesional relacionados con el trabajo)
\chapter{Resultados y Discusión}
\input{capitulos/resultados_discusion/main.tex}

% Conclusiones
\chapter{Conclusiones}
\input{capitulos/conclusiones/main.tex}

% Planificación temporal y presupuesto
\chapter{Planificación Temporal y Presupuesto}
\input{capitulos/planificacion_presupuesto/main.tex}

% Bibliografía
\newpage
\addcontentsline{toc}{chapter}{Bibliografía}
\printbibliography

\end{document}


% Metodología
\chapter{Metodología}
\documentclass[a4paper,11pt,twoside]{report}
\usepackage[left=25mm,right=25mm,top=25mm,bottom=25mm,includehead,includefoot,headsep=15mm,footskip=15mm]{geometry}
\usepackage{graphicx}
\usepackage{fancyhdr}
\usepackage{titlesec}
\usepackage[spanish]{babel}
\usepackage[utf8]{inputenc}
\usepackage{amsmath}
\usepackage{setspace}
\usepackage{svg}
\usepackage{hyperref}
\usepackage[backend=biber,style=numeric]{biblatex}
\addbibresource{references.bib}
\hypersetup{
    colorlinks=true,
    linkcolor=blue,      % color of internal links (sections, etc.)
    urlcolor=blue,       % color of external links
    pdftitle={Optimización energética de sistema híbrido con bomba de calor, suelo radiante, fotovoltaica y almacenamiento para vivienda},    % title
    pdfauthor={Luis D. Aranda Sánchez},     % author
    pdfkeywords={palabra1, palabra2, código1, etc.} % list of keywords
}

% Font change to Arial
\usepackage{helvet}
\renewcommand{\familydefault}{\sfdefault}

% Chapter titles in uppercase and larger font
\titleformat{\chapter}[hang]{\large\bfseries}{\thechapter.}{1em}{\MakeUppercase}
\titleformat{\section}[hang]{\bfseries}{\thesection.}{1em}{}
\titleformat{\subsection}[hang]{\bfseries}{\thesubsection.}{1em}{}

% Fancyhdr setup
\setlength{\headheight}{14.30174pt} % Adjust to recommended value, slightly larger for safety
\fancyhf{} % Clear all headers and footers
\fancyhead[LE]{\nouppercase{\leftmark}}
\fancyhead[RO]{Optimización energética para vivienda}
\fancyfoot[LE]{\thepage}
\fancyfoot[RE]{Escuela Técnica Superior de Ingenieros Industriales (UPM)}
\fancyfoot[LO]{Luis D. Aranda Sánchez}
\fancyfoot[RO]{\thepage}
\renewcommand{\headrulewidth}{0.4pt}
\renewcommand{\footrulewidth}{0.4pt}

\fancypagestyle{myfancy}{
    \fancyhf{} % Clear all headers and footers
    \fancyhead[LE]{\nouppercase{\leftmark}}
    \fancyhead[RO]{Optimización energética para vivienda}
    \fancyfoot[LE]{\thepage}
    \fancyfoot[RE]{Escuela Técnica Superior de Ingenieros Industriales (UPM)}
    \fancyfoot[LO]{Luis D. Aranda Sánchez}
    \fancyfoot[RO]{\thepage}
    \renewcommand{\headrulewidth}{0.4pt}
    \renewcommand{\footrulewidth}{0.4pt}
}

\fancypagestyle{simple}{
    \fancyhf{} % Clear all headers and footers
    \renewcommand{\headrulewidth}{0pt}
    \renewcommand{\footrulewidth}{0pt}
}

% Line spacing
\setstretch{1.2}

% Document starts here
\begin{document}

% Portada
\begin{titlepage}
    \centering
    {\scshape\LARGE Universidad Politécnica de Madrid \par}
    \vspace{1cm}
    {\scshape\Large Escuela Técnica Superior de Ingenieros Industriales\par}
    \vspace{1.5cm}
    {\huge\bfseries Optimización energética de sistema híbrido con bomba de calor, suelo radiante, fotovoltaica y almacenamiento para vivienda \par}
    \vspace{1.5cm}
    {\Large\bfseries Trabajo de Fin de Máster\par}
    \vspace{0.5cm}
    {\large Máster Universitario en Ingeniería de la Energía \par}
    \vspace{2cm}
    {\Large Luis D. Aranda Sánchez\par}
    \vfill
    Director: Javier Rodríguez Martín
    \vfill
    {\large Septiembre 6, 2024\par}
\end{titlepage}

% Resumen (máximo de 5 páginas, incluyendo al final Palabras clave)
\clearpage
\pagestyle{simple}
% \newpage
\chapter*{Resumen}
\addcontentsline{toc}{chapter}{Resumen}
\input{capitulos/resumen/main.tex}

% Índice (paginado)
\clearpage
\pagestyle{simple}
% \newpage
\tableofcontents

% Introducción (donde se incluya los antecedentes y justificación)
\clearpage
\pagestyle{myfancy}
\newpage
\chapter{Introducción}
\input{capitulos/introduccion/main.tex}

% Objetivos
\chapter{Objetivos}
\input{capitulos/objetivos/main.tex}

% Metodología
\chapter{Metodología}
\input{capitulos/metodologia/main.tex}

% Resultados y discusión (incluyendo la valoración de impactos y de aspectos de responsabilidad legal, ética y profesional relacionados con el trabajo)
\chapter{Resultados y Discusión}
\input{capitulos/resultados_discusion/main.tex}

% Conclusiones
\chapter{Conclusiones}
\input{capitulos/conclusiones/main.tex}

% Planificación temporal y presupuesto
\chapter{Planificación Temporal y Presupuesto}
\input{capitulos/planificacion_presupuesto/main.tex}

% Bibliografía
\newpage
\addcontentsline{toc}{chapter}{Bibliografía}
\printbibliography

\end{document}


% Resultados y discusión (incluyendo la valoración de impactos y de aspectos de responsabilidad legal, ética y profesional relacionados con el trabajo)
\chapter{Resultados y Discusión}
\documentclass[a4paper,11pt,twoside]{report}
\usepackage[left=25mm,right=25mm,top=25mm,bottom=25mm,includehead,includefoot,headsep=15mm,footskip=15mm]{geometry}
\usepackage{graphicx}
\usepackage{fancyhdr}
\usepackage{titlesec}
\usepackage[spanish]{babel}
\usepackage[utf8]{inputenc}
\usepackage{amsmath}
\usepackage{setspace}
\usepackage{svg}
\usepackage{hyperref}
\usepackage[backend=biber,style=numeric]{biblatex}
\addbibresource{references.bib}
\hypersetup{
    colorlinks=true,
    linkcolor=blue,      % color of internal links (sections, etc.)
    urlcolor=blue,       % color of external links
    pdftitle={Optimización energética de sistema híbrido con bomba de calor, suelo radiante, fotovoltaica y almacenamiento para vivienda},    % title
    pdfauthor={Luis D. Aranda Sánchez},     % author
    pdfkeywords={palabra1, palabra2, código1, etc.} % list of keywords
}

% Font change to Arial
\usepackage{helvet}
\renewcommand{\familydefault}{\sfdefault}

% Chapter titles in uppercase and larger font
\titleformat{\chapter}[hang]{\large\bfseries}{\thechapter.}{1em}{\MakeUppercase}
\titleformat{\section}[hang]{\bfseries}{\thesection.}{1em}{}
\titleformat{\subsection}[hang]{\bfseries}{\thesubsection.}{1em}{}

% Fancyhdr setup
\setlength{\headheight}{14.30174pt} % Adjust to recommended value, slightly larger for safety
\fancyhf{} % Clear all headers and footers
\fancyhead[LE]{\nouppercase{\leftmark}}
\fancyhead[RO]{Optimización energética para vivienda}
\fancyfoot[LE]{\thepage}
\fancyfoot[RE]{Escuela Técnica Superior de Ingenieros Industriales (UPM)}
\fancyfoot[LO]{Luis D. Aranda Sánchez}
\fancyfoot[RO]{\thepage}
\renewcommand{\headrulewidth}{0.4pt}
\renewcommand{\footrulewidth}{0.4pt}

\fancypagestyle{myfancy}{
    \fancyhf{} % Clear all headers and footers
    \fancyhead[LE]{\nouppercase{\leftmark}}
    \fancyhead[RO]{Optimización energética para vivienda}
    \fancyfoot[LE]{\thepage}
    \fancyfoot[RE]{Escuela Técnica Superior de Ingenieros Industriales (UPM)}
    \fancyfoot[LO]{Luis D. Aranda Sánchez}
    \fancyfoot[RO]{\thepage}
    \renewcommand{\headrulewidth}{0.4pt}
    \renewcommand{\footrulewidth}{0.4pt}
}

\fancypagestyle{simple}{
    \fancyhf{} % Clear all headers and footers
    \renewcommand{\headrulewidth}{0pt}
    \renewcommand{\footrulewidth}{0pt}
}

% Line spacing
\setstretch{1.2}

% Document starts here
\begin{document}

% Portada
\begin{titlepage}
    \centering
    {\scshape\LARGE Universidad Politécnica de Madrid \par}
    \vspace{1cm}
    {\scshape\Large Escuela Técnica Superior de Ingenieros Industriales\par}
    \vspace{1.5cm}
    {\huge\bfseries Optimización energética de sistema híbrido con bomba de calor, suelo radiante, fotovoltaica y almacenamiento para vivienda \par}
    \vspace{1.5cm}
    {\Large\bfseries Trabajo de Fin de Máster\par}
    \vspace{0.5cm}
    {\large Máster Universitario en Ingeniería de la Energía \par}
    \vspace{2cm}
    {\Large Luis D. Aranda Sánchez\par}
    \vfill
    Director: Javier Rodríguez Martín
    \vfill
    {\large Septiembre 6, 2024\par}
\end{titlepage}

% Resumen (máximo de 5 páginas, incluyendo al final Palabras clave)
\clearpage
\pagestyle{simple}
% \newpage
\chapter*{Resumen}
\addcontentsline{toc}{chapter}{Resumen}
\input{capitulos/resumen/main.tex}

% Índice (paginado)
\clearpage
\pagestyle{simple}
% \newpage
\tableofcontents

% Introducción (donde se incluya los antecedentes y justificación)
\clearpage
\pagestyle{myfancy}
\newpage
\chapter{Introducción}
\input{capitulos/introduccion/main.tex}

% Objetivos
\chapter{Objetivos}
\input{capitulos/objetivos/main.tex}

% Metodología
\chapter{Metodología}
\input{capitulos/metodologia/main.tex}

% Resultados y discusión (incluyendo la valoración de impactos y de aspectos de responsabilidad legal, ética y profesional relacionados con el trabajo)
\chapter{Resultados y Discusión}
\input{capitulos/resultados_discusion/main.tex}

% Conclusiones
\chapter{Conclusiones}
\input{capitulos/conclusiones/main.tex}

% Planificación temporal y presupuesto
\chapter{Planificación Temporal y Presupuesto}
\input{capitulos/planificacion_presupuesto/main.tex}

% Bibliografía
\newpage
\addcontentsline{toc}{chapter}{Bibliografía}
\printbibliography

\end{document}


% Conclusiones
\chapter{Conclusiones}
\documentclass[a4paper,11pt,twoside]{report}
\usepackage[left=25mm,right=25mm,top=25mm,bottom=25mm,includehead,includefoot,headsep=15mm,footskip=15mm]{geometry}
\usepackage{graphicx}
\usepackage{fancyhdr}
\usepackage{titlesec}
\usepackage[spanish]{babel}
\usepackage[utf8]{inputenc}
\usepackage{amsmath}
\usepackage{setspace}
\usepackage{svg}
\usepackage{hyperref}
\usepackage[backend=biber,style=numeric]{biblatex}
\addbibresource{references.bib}
\hypersetup{
    colorlinks=true,
    linkcolor=blue,      % color of internal links (sections, etc.)
    urlcolor=blue,       % color of external links
    pdftitle={Optimización energética de sistema híbrido con bomba de calor, suelo radiante, fotovoltaica y almacenamiento para vivienda},    % title
    pdfauthor={Luis D. Aranda Sánchez},     % author
    pdfkeywords={palabra1, palabra2, código1, etc.} % list of keywords
}

% Font change to Arial
\usepackage{helvet}
\renewcommand{\familydefault}{\sfdefault}

% Chapter titles in uppercase and larger font
\titleformat{\chapter}[hang]{\large\bfseries}{\thechapter.}{1em}{\MakeUppercase}
\titleformat{\section}[hang]{\bfseries}{\thesection.}{1em}{}
\titleformat{\subsection}[hang]{\bfseries}{\thesubsection.}{1em}{}

% Fancyhdr setup
\setlength{\headheight}{14.30174pt} % Adjust to recommended value, slightly larger for safety
\fancyhf{} % Clear all headers and footers
\fancyhead[LE]{\nouppercase{\leftmark}}
\fancyhead[RO]{Optimización energética para vivienda}
\fancyfoot[LE]{\thepage}
\fancyfoot[RE]{Escuela Técnica Superior de Ingenieros Industriales (UPM)}
\fancyfoot[LO]{Luis D. Aranda Sánchez}
\fancyfoot[RO]{\thepage}
\renewcommand{\headrulewidth}{0.4pt}
\renewcommand{\footrulewidth}{0.4pt}

\fancypagestyle{myfancy}{
    \fancyhf{} % Clear all headers and footers
    \fancyhead[LE]{\nouppercase{\leftmark}}
    \fancyhead[RO]{Optimización energética para vivienda}
    \fancyfoot[LE]{\thepage}
    \fancyfoot[RE]{Escuela Técnica Superior de Ingenieros Industriales (UPM)}
    \fancyfoot[LO]{Luis D. Aranda Sánchez}
    \fancyfoot[RO]{\thepage}
    \renewcommand{\headrulewidth}{0.4pt}
    \renewcommand{\footrulewidth}{0.4pt}
}

\fancypagestyle{simple}{
    \fancyhf{} % Clear all headers and footers
    \renewcommand{\headrulewidth}{0pt}
    \renewcommand{\footrulewidth}{0pt}
}

% Line spacing
\setstretch{1.2}

% Document starts here
\begin{document}

% Portada
\begin{titlepage}
    \centering
    {\scshape\LARGE Universidad Politécnica de Madrid \par}
    \vspace{1cm}
    {\scshape\Large Escuela Técnica Superior de Ingenieros Industriales\par}
    \vspace{1.5cm}
    {\huge\bfseries Optimización energética de sistema híbrido con bomba de calor, suelo radiante, fotovoltaica y almacenamiento para vivienda \par}
    \vspace{1.5cm}
    {\Large\bfseries Trabajo de Fin de Máster\par}
    \vspace{0.5cm}
    {\large Máster Universitario en Ingeniería de la Energía \par}
    \vspace{2cm}
    {\Large Luis D. Aranda Sánchez\par}
    \vfill
    Director: Javier Rodríguez Martín
    \vfill
    {\large Septiembre 6, 2024\par}
\end{titlepage}

% Resumen (máximo de 5 páginas, incluyendo al final Palabras clave)
\clearpage
\pagestyle{simple}
% \newpage
\chapter*{Resumen}
\addcontentsline{toc}{chapter}{Resumen}
\input{capitulos/resumen/main.tex}

% Índice (paginado)
\clearpage
\pagestyle{simple}
% \newpage
\tableofcontents

% Introducción (donde se incluya los antecedentes y justificación)
\clearpage
\pagestyle{myfancy}
\newpage
\chapter{Introducción}
\input{capitulos/introduccion/main.tex}

% Objetivos
\chapter{Objetivos}
\input{capitulos/objetivos/main.tex}

% Metodología
\chapter{Metodología}
\input{capitulos/metodologia/main.tex}

% Resultados y discusión (incluyendo la valoración de impactos y de aspectos de responsabilidad legal, ética y profesional relacionados con el trabajo)
\chapter{Resultados y Discusión}
\input{capitulos/resultados_discusion/main.tex}

% Conclusiones
\chapter{Conclusiones}
\input{capitulos/conclusiones/main.tex}

% Planificación temporal y presupuesto
\chapter{Planificación Temporal y Presupuesto}
\input{capitulos/planificacion_presupuesto/main.tex}

% Bibliografía
\newpage
\addcontentsline{toc}{chapter}{Bibliografía}
\printbibliography

\end{document}


% Planificación temporal y presupuesto
\chapter{Planificación Temporal y Presupuesto}
\documentclass[a4paper,11pt,twoside]{report}
\usepackage[left=25mm,right=25mm,top=25mm,bottom=25mm,includehead,includefoot,headsep=15mm,footskip=15mm]{geometry}
\usepackage{graphicx}
\usepackage{fancyhdr}
\usepackage{titlesec}
\usepackage[spanish]{babel}
\usepackage[utf8]{inputenc}
\usepackage{amsmath}
\usepackage{setspace}
\usepackage{svg}
\usepackage{hyperref}
\usepackage[backend=biber,style=numeric]{biblatex}
\addbibresource{references.bib}
\hypersetup{
    colorlinks=true,
    linkcolor=blue,      % color of internal links (sections, etc.)
    urlcolor=blue,       % color of external links
    pdftitle={Optimización energética de sistema híbrido con bomba de calor, suelo radiante, fotovoltaica y almacenamiento para vivienda},    % title
    pdfauthor={Luis D. Aranda Sánchez},     % author
    pdfkeywords={palabra1, palabra2, código1, etc.} % list of keywords
}

% Font change to Arial
\usepackage{helvet}
\renewcommand{\familydefault}{\sfdefault}

% Chapter titles in uppercase and larger font
\titleformat{\chapter}[hang]{\large\bfseries}{\thechapter.}{1em}{\MakeUppercase}
\titleformat{\section}[hang]{\bfseries}{\thesection.}{1em}{}
\titleformat{\subsection}[hang]{\bfseries}{\thesubsection.}{1em}{}

% Fancyhdr setup
\setlength{\headheight}{14.30174pt} % Adjust to recommended value, slightly larger for safety
\fancyhf{} % Clear all headers and footers
\fancyhead[LE]{\nouppercase{\leftmark}}
\fancyhead[RO]{Optimización energética para vivienda}
\fancyfoot[LE]{\thepage}
\fancyfoot[RE]{Escuela Técnica Superior de Ingenieros Industriales (UPM)}
\fancyfoot[LO]{Luis D. Aranda Sánchez}
\fancyfoot[RO]{\thepage}
\renewcommand{\headrulewidth}{0.4pt}
\renewcommand{\footrulewidth}{0.4pt}

\fancypagestyle{myfancy}{
    \fancyhf{} % Clear all headers and footers
    \fancyhead[LE]{\nouppercase{\leftmark}}
    \fancyhead[RO]{Optimización energética para vivienda}
    \fancyfoot[LE]{\thepage}
    \fancyfoot[RE]{Escuela Técnica Superior de Ingenieros Industriales (UPM)}
    \fancyfoot[LO]{Luis D. Aranda Sánchez}
    \fancyfoot[RO]{\thepage}
    \renewcommand{\headrulewidth}{0.4pt}
    \renewcommand{\footrulewidth}{0.4pt}
}

\fancypagestyle{simple}{
    \fancyhf{} % Clear all headers and footers
    \renewcommand{\headrulewidth}{0pt}
    \renewcommand{\footrulewidth}{0pt}
}

% Line spacing
\setstretch{1.2}

% Document starts here
\begin{document}

% Portada
\begin{titlepage}
    \centering
    {\scshape\LARGE Universidad Politécnica de Madrid \par}
    \vspace{1cm}
    {\scshape\Large Escuela Técnica Superior de Ingenieros Industriales\par}
    \vspace{1.5cm}
    {\huge\bfseries Optimización energética de sistema híbrido con bomba de calor, suelo radiante, fotovoltaica y almacenamiento para vivienda \par}
    \vspace{1.5cm}
    {\Large\bfseries Trabajo de Fin de Máster\par}
    \vspace{0.5cm}
    {\large Máster Universitario en Ingeniería de la Energía \par}
    \vspace{2cm}
    {\Large Luis D. Aranda Sánchez\par}
    \vfill
    Director: Javier Rodríguez Martín
    \vfill
    {\large Septiembre 6, 2024\par}
\end{titlepage}

% Resumen (máximo de 5 páginas, incluyendo al final Palabras clave)
\clearpage
\pagestyle{simple}
% \newpage
\chapter*{Resumen}
\addcontentsline{toc}{chapter}{Resumen}
\input{capitulos/resumen/main.tex}

% Índice (paginado)
\clearpage
\pagestyle{simple}
% \newpage
\tableofcontents

% Introducción (donde se incluya los antecedentes y justificación)
\clearpage
\pagestyle{myfancy}
\newpage
\chapter{Introducción}
\input{capitulos/introduccion/main.tex}

% Objetivos
\chapter{Objetivos}
\input{capitulos/objetivos/main.tex}

% Metodología
\chapter{Metodología}
\input{capitulos/metodologia/main.tex}

% Resultados y discusión (incluyendo la valoración de impactos y de aspectos de responsabilidad legal, ética y profesional relacionados con el trabajo)
\chapter{Resultados y Discusión}
\input{capitulos/resultados_discusion/main.tex}

% Conclusiones
\chapter{Conclusiones}
\input{capitulos/conclusiones/main.tex}

% Planificación temporal y presupuesto
\chapter{Planificación Temporal y Presupuesto}
\input{capitulos/planificacion_presupuesto/main.tex}

% Bibliografía
\newpage
\addcontentsline{toc}{chapter}{Bibliografía}
\printbibliography

\end{document}


% Bibliografía
\newpage
\addcontentsline{toc}{chapter}{Bibliografía}
\printbibliography

\end{document}


% Bibliografía
\newpage
\addcontentsline{toc}{chapter}{Bibliografía}
\printbibliography

\end{document}


% Conclusiones
\cleardoublepage
\chapter{Conclusiones}
\documentclass[a4paper,11pt,twoside]{report}
\usepackage[left=25mm,right=25mm,top=25mm,bottom=25mm,includehead,includefoot,headsep=15mm,footskip=15mm]{geometry}
\usepackage{graphicx}
\usepackage{fancyhdr}
\usepackage{titlesec}
\usepackage[spanish]{babel}
\usepackage[utf8]{inputenc}
\usepackage{amsmath}
\usepackage{setspace}
\usepackage{svg}
\usepackage{hyperref}
\usepackage[backend=biber,style=numeric]{biblatex}
\addbibresource{references.bib}
\hypersetup{
    colorlinks=true,
    linkcolor=blue,      % color of internal links (sections, etc.)
    urlcolor=blue,       % color of external links
    pdftitle={Optimización energética de sistema híbrido con bomba de calor, suelo radiante, fotovoltaica y almacenamiento para vivienda},    % title
    pdfauthor={Luis D. Aranda Sánchez},     % author
    pdfkeywords={palabra1, palabra2, código1, etc.} % list of keywords
}

% Font change to Arial
\usepackage{helvet}
\renewcommand{\familydefault}{\sfdefault}

% Chapter titles in uppercase and larger font
\titleformat{\chapter}[hang]{\large\bfseries}{\thechapter.}{1em}{\MakeUppercase}
\titleformat{\section}[hang]{\bfseries}{\thesection.}{1em}{}
\titleformat{\subsection}[hang]{\bfseries}{\thesubsection.}{1em}{}

% Fancyhdr setup
\setlength{\headheight}{14.30174pt} % Adjust to recommended value, slightly larger for safety
\fancyhf{} % Clear all headers and footers
\fancyhead[LE]{\nouppercase{\leftmark}}
\fancyhead[RO]{Optimización energética para vivienda}
\fancyfoot[LE]{\thepage}
\fancyfoot[RE]{Escuela Técnica Superior de Ingenieros Industriales (UPM)}
\fancyfoot[LO]{Luis D. Aranda Sánchez}
\fancyfoot[RO]{\thepage}
\renewcommand{\headrulewidth}{0.4pt}
\renewcommand{\footrulewidth}{0.4pt}

\fancypagestyle{myfancy}{
    \fancyhf{} % Clear all headers and footers
    \fancyhead[LE]{\nouppercase{\leftmark}}
    \fancyhead[RO]{Optimización energética para vivienda}
    \fancyfoot[LE]{\thepage}
    \fancyfoot[RE]{Escuela Técnica Superior de Ingenieros Industriales (UPM)}
    \fancyfoot[LO]{Luis D. Aranda Sánchez}
    \fancyfoot[RO]{\thepage}
    \renewcommand{\headrulewidth}{0.4pt}
    \renewcommand{\footrulewidth}{0.4pt}
}

\fancypagestyle{simple}{
    \fancyhf{} % Clear all headers and footers
    \renewcommand{\headrulewidth}{0pt}
    \renewcommand{\footrulewidth}{0pt}
}

% Line spacing
\setstretch{1.2}

% Document starts here
\begin{document}

% Portada
\begin{titlepage}
    \centering
    {\scshape\LARGE Universidad Politécnica de Madrid \par}
    \vspace{1cm}
    {\scshape\Large Escuela Técnica Superior de Ingenieros Industriales\par}
    \vspace{1.5cm}
    {\huge\bfseries Optimización energética de sistema híbrido con bomba de calor, suelo radiante, fotovoltaica y almacenamiento para vivienda \par}
    \vspace{1.5cm}
    {\Large\bfseries Trabajo de Fin de Máster\par}
    \vspace{0.5cm}
    {\large Máster Universitario en Ingeniería de la Energía \par}
    \vspace{2cm}
    {\Large Luis D. Aranda Sánchez\par}
    \vfill
    Director: Javier Rodríguez Martín
    \vfill
    {\large Septiembre 6, 2024\par}
\end{titlepage}

% Resumen (máximo de 5 páginas, incluyendo al final Palabras clave)
\clearpage
\pagestyle{simple}
% \newpage
\chapter*{Resumen}
\addcontentsline{toc}{chapter}{Resumen}
\documentclass[a4paper,11pt,twoside]{report}
\usepackage[left=25mm,right=25mm,top=25mm,bottom=25mm,includehead,includefoot,headsep=15mm,footskip=15mm]{geometry}
\usepackage{graphicx}
\usepackage{fancyhdr}
\usepackage{titlesec}
\usepackage[spanish]{babel}
\usepackage[utf8]{inputenc}
\usepackage{amsmath}
\usepackage{setspace}
\usepackage{svg}
\usepackage{hyperref}
\usepackage[backend=biber,style=numeric]{biblatex}
\addbibresource{references.bib}
\hypersetup{
    colorlinks=true,
    linkcolor=blue,      % color of internal links (sections, etc.)
    urlcolor=blue,       % color of external links
    pdftitle={Optimización energética de sistema híbrido con bomba de calor, suelo radiante, fotovoltaica y almacenamiento para vivienda},    % title
    pdfauthor={Luis D. Aranda Sánchez},     % author
    pdfkeywords={palabra1, palabra2, código1, etc.} % list of keywords
}

% Font change to Arial
\usepackage{helvet}
\renewcommand{\familydefault}{\sfdefault}

% Chapter titles in uppercase and larger font
\titleformat{\chapter}[hang]{\large\bfseries}{\thechapter.}{1em}{\MakeUppercase}
\titleformat{\section}[hang]{\bfseries}{\thesection.}{1em}{}
\titleformat{\subsection}[hang]{\bfseries}{\thesubsection.}{1em}{}

% Fancyhdr setup
\setlength{\headheight}{14.30174pt} % Adjust to recommended value, slightly larger for safety
\fancyhf{} % Clear all headers and footers
\fancyhead[LE]{\nouppercase{\leftmark}}
\fancyhead[RO]{Optimización energética para vivienda}
\fancyfoot[LE]{\thepage}
\fancyfoot[RE]{Escuela Técnica Superior de Ingenieros Industriales (UPM)}
\fancyfoot[LO]{Luis D. Aranda Sánchez}
\fancyfoot[RO]{\thepage}
\renewcommand{\headrulewidth}{0.4pt}
\renewcommand{\footrulewidth}{0.4pt}

\fancypagestyle{myfancy}{
    \fancyhf{} % Clear all headers and footers
    \fancyhead[LE]{\nouppercase{\leftmark}}
    \fancyhead[RO]{Optimización energética para vivienda}
    \fancyfoot[LE]{\thepage}
    \fancyfoot[RE]{Escuela Técnica Superior de Ingenieros Industriales (UPM)}
    \fancyfoot[LO]{Luis D. Aranda Sánchez}
    \fancyfoot[RO]{\thepage}
    \renewcommand{\headrulewidth}{0.4pt}
    \renewcommand{\footrulewidth}{0.4pt}
}

\fancypagestyle{simple}{
    \fancyhf{} % Clear all headers and footers
    \renewcommand{\headrulewidth}{0pt}
    \renewcommand{\footrulewidth}{0pt}
}

% Line spacing
\setstretch{1.2}

% Document starts here
\begin{document}

% Portada
\begin{titlepage}
    \centering
    {\scshape\LARGE Universidad Politécnica de Madrid \par}
    \vspace{1cm}
    {\scshape\Large Escuela Técnica Superior de Ingenieros Industriales\par}
    \vspace{1.5cm}
    {\huge\bfseries Optimización energética de sistema híbrido con bomba de calor, suelo radiante, fotovoltaica y almacenamiento para vivienda \par}
    \vspace{1.5cm}
    {\Large\bfseries Trabajo de Fin de Máster\par}
    \vspace{0.5cm}
    {\large Máster Universitario en Ingeniería de la Energía \par}
    \vspace{2cm}
    {\Large Luis D. Aranda Sánchez\par}
    \vfill
    Director: Javier Rodríguez Martín
    \vfill
    {\large Septiembre 6, 2024\par}
\end{titlepage}

% Resumen (máximo de 5 páginas, incluyendo al final Palabras clave)
\clearpage
\pagestyle{simple}
% \newpage
\chapter*{Resumen}
\addcontentsline{toc}{chapter}{Resumen}
\documentclass[a4paper,11pt,twoside]{report}
\usepackage[left=25mm,right=25mm,top=25mm,bottom=25mm,includehead,includefoot,headsep=15mm,footskip=15mm]{geometry}
\usepackage{graphicx}
\usepackage{fancyhdr}
\usepackage{titlesec}
\usepackage[spanish]{babel}
\usepackage[utf8]{inputenc}
\usepackage{amsmath}
\usepackage{setspace}
\usepackage{svg}
\usepackage{hyperref}
\usepackage[backend=biber,style=numeric]{biblatex}
\addbibresource{references.bib}
\hypersetup{
    colorlinks=true,
    linkcolor=blue,      % color of internal links (sections, etc.)
    urlcolor=blue,       % color of external links
    pdftitle={Optimización energética de sistema híbrido con bomba de calor, suelo radiante, fotovoltaica y almacenamiento para vivienda},    % title
    pdfauthor={Luis D. Aranda Sánchez},     % author
    pdfkeywords={palabra1, palabra2, código1, etc.} % list of keywords
}

% Font change to Arial
\usepackage{helvet}
\renewcommand{\familydefault}{\sfdefault}

% Chapter titles in uppercase and larger font
\titleformat{\chapter}[hang]{\large\bfseries}{\thechapter.}{1em}{\MakeUppercase}
\titleformat{\section}[hang]{\bfseries}{\thesection.}{1em}{}
\titleformat{\subsection}[hang]{\bfseries}{\thesubsection.}{1em}{}

% Fancyhdr setup
\setlength{\headheight}{14.30174pt} % Adjust to recommended value, slightly larger for safety
\fancyhf{} % Clear all headers and footers
\fancyhead[LE]{\nouppercase{\leftmark}}
\fancyhead[RO]{Optimización energética para vivienda}
\fancyfoot[LE]{\thepage}
\fancyfoot[RE]{Escuela Técnica Superior de Ingenieros Industriales (UPM)}
\fancyfoot[LO]{Luis D. Aranda Sánchez}
\fancyfoot[RO]{\thepage}
\renewcommand{\headrulewidth}{0.4pt}
\renewcommand{\footrulewidth}{0.4pt}

\fancypagestyle{myfancy}{
    \fancyhf{} % Clear all headers and footers
    \fancyhead[LE]{\nouppercase{\leftmark}}
    \fancyhead[RO]{Optimización energética para vivienda}
    \fancyfoot[LE]{\thepage}
    \fancyfoot[RE]{Escuela Técnica Superior de Ingenieros Industriales (UPM)}
    \fancyfoot[LO]{Luis D. Aranda Sánchez}
    \fancyfoot[RO]{\thepage}
    \renewcommand{\headrulewidth}{0.4pt}
    \renewcommand{\footrulewidth}{0.4pt}
}

\fancypagestyle{simple}{
    \fancyhf{} % Clear all headers and footers
    \renewcommand{\headrulewidth}{0pt}
    \renewcommand{\footrulewidth}{0pt}
}

% Line spacing
\setstretch{1.2}

% Document starts here
\begin{document}

% Portada
\begin{titlepage}
    \centering
    {\scshape\LARGE Universidad Politécnica de Madrid \par}
    \vspace{1cm}
    {\scshape\Large Escuela Técnica Superior de Ingenieros Industriales\par}
    \vspace{1.5cm}
    {\huge\bfseries Optimización energética de sistema híbrido con bomba de calor, suelo radiante, fotovoltaica y almacenamiento para vivienda \par}
    \vspace{1.5cm}
    {\Large\bfseries Trabajo de Fin de Máster\par}
    \vspace{0.5cm}
    {\large Máster Universitario en Ingeniería de la Energía \par}
    \vspace{2cm}
    {\Large Luis D. Aranda Sánchez\par}
    \vfill
    Director: Javier Rodríguez Martín
    \vfill
    {\large Septiembre 6, 2024\par}
\end{titlepage}

% Resumen (máximo de 5 páginas, incluyendo al final Palabras clave)
\clearpage
\pagestyle{simple}
% \newpage
\chapter*{Resumen}
\addcontentsline{toc}{chapter}{Resumen}
\input{capitulos/resumen/main.tex}

% Índice (paginado)
\clearpage
\pagestyle{simple}
% \newpage
\tableofcontents

% Introducción (donde se incluya los antecedentes y justificación)
\clearpage
\pagestyle{myfancy}
\newpage
\chapter{Introducción}
\input{capitulos/introduccion/main.tex}

% Objetivos
\chapter{Objetivos}
\input{capitulos/objetivos/main.tex}

% Metodología
\chapter{Metodología}
\input{capitulos/metodologia/main.tex}

% Resultados y discusión (incluyendo la valoración de impactos y de aspectos de responsabilidad legal, ética y profesional relacionados con el trabajo)
\chapter{Resultados y Discusión}
\input{capitulos/resultados_discusion/main.tex}

% Conclusiones
\chapter{Conclusiones}
\input{capitulos/conclusiones/main.tex}

% Planificación temporal y presupuesto
\chapter{Planificación Temporal y Presupuesto}
\input{capitulos/planificacion_presupuesto/main.tex}

% Bibliografía
\newpage
\addcontentsline{toc}{chapter}{Bibliografía}
\printbibliography

\end{document}


% Índice (paginado)
\clearpage
\pagestyle{simple}
% \newpage
\tableofcontents

% Introducción (donde se incluya los antecedentes y justificación)
\clearpage
\pagestyle{myfancy}
\newpage
\chapter{Introducción}
\documentclass[a4paper,11pt,twoside]{report}
\usepackage[left=25mm,right=25mm,top=25mm,bottom=25mm,includehead,includefoot,headsep=15mm,footskip=15mm]{geometry}
\usepackage{graphicx}
\usepackage{fancyhdr}
\usepackage{titlesec}
\usepackage[spanish]{babel}
\usepackage[utf8]{inputenc}
\usepackage{amsmath}
\usepackage{setspace}
\usepackage{svg}
\usepackage{hyperref}
\usepackage[backend=biber,style=numeric]{biblatex}
\addbibresource{references.bib}
\hypersetup{
    colorlinks=true,
    linkcolor=blue,      % color of internal links (sections, etc.)
    urlcolor=blue,       % color of external links
    pdftitle={Optimización energética de sistema híbrido con bomba de calor, suelo radiante, fotovoltaica y almacenamiento para vivienda},    % title
    pdfauthor={Luis D. Aranda Sánchez},     % author
    pdfkeywords={palabra1, palabra2, código1, etc.} % list of keywords
}

% Font change to Arial
\usepackage{helvet}
\renewcommand{\familydefault}{\sfdefault}

% Chapter titles in uppercase and larger font
\titleformat{\chapter}[hang]{\large\bfseries}{\thechapter.}{1em}{\MakeUppercase}
\titleformat{\section}[hang]{\bfseries}{\thesection.}{1em}{}
\titleformat{\subsection}[hang]{\bfseries}{\thesubsection.}{1em}{}

% Fancyhdr setup
\setlength{\headheight}{14.30174pt} % Adjust to recommended value, slightly larger for safety
\fancyhf{} % Clear all headers and footers
\fancyhead[LE]{\nouppercase{\leftmark}}
\fancyhead[RO]{Optimización energética para vivienda}
\fancyfoot[LE]{\thepage}
\fancyfoot[RE]{Escuela Técnica Superior de Ingenieros Industriales (UPM)}
\fancyfoot[LO]{Luis D. Aranda Sánchez}
\fancyfoot[RO]{\thepage}
\renewcommand{\headrulewidth}{0.4pt}
\renewcommand{\footrulewidth}{0.4pt}

\fancypagestyle{myfancy}{
    \fancyhf{} % Clear all headers and footers
    \fancyhead[LE]{\nouppercase{\leftmark}}
    \fancyhead[RO]{Optimización energética para vivienda}
    \fancyfoot[LE]{\thepage}
    \fancyfoot[RE]{Escuela Técnica Superior de Ingenieros Industriales (UPM)}
    \fancyfoot[LO]{Luis D. Aranda Sánchez}
    \fancyfoot[RO]{\thepage}
    \renewcommand{\headrulewidth}{0.4pt}
    \renewcommand{\footrulewidth}{0.4pt}
}

\fancypagestyle{simple}{
    \fancyhf{} % Clear all headers and footers
    \renewcommand{\headrulewidth}{0pt}
    \renewcommand{\footrulewidth}{0pt}
}

% Line spacing
\setstretch{1.2}

% Document starts here
\begin{document}

% Portada
\begin{titlepage}
    \centering
    {\scshape\LARGE Universidad Politécnica de Madrid \par}
    \vspace{1cm}
    {\scshape\Large Escuela Técnica Superior de Ingenieros Industriales\par}
    \vspace{1.5cm}
    {\huge\bfseries Optimización energética de sistema híbrido con bomba de calor, suelo radiante, fotovoltaica y almacenamiento para vivienda \par}
    \vspace{1.5cm}
    {\Large\bfseries Trabajo de Fin de Máster\par}
    \vspace{0.5cm}
    {\large Máster Universitario en Ingeniería de la Energía \par}
    \vspace{2cm}
    {\Large Luis D. Aranda Sánchez\par}
    \vfill
    Director: Javier Rodríguez Martín
    \vfill
    {\large Septiembre 6, 2024\par}
\end{titlepage}

% Resumen (máximo de 5 páginas, incluyendo al final Palabras clave)
\clearpage
\pagestyle{simple}
% \newpage
\chapter*{Resumen}
\addcontentsline{toc}{chapter}{Resumen}
\input{capitulos/resumen/main.tex}

% Índice (paginado)
\clearpage
\pagestyle{simple}
% \newpage
\tableofcontents

% Introducción (donde se incluya los antecedentes y justificación)
\clearpage
\pagestyle{myfancy}
\newpage
\chapter{Introducción}
\input{capitulos/introduccion/main.tex}

% Objetivos
\chapter{Objetivos}
\input{capitulos/objetivos/main.tex}

% Metodología
\chapter{Metodología}
\input{capitulos/metodologia/main.tex}

% Resultados y discusión (incluyendo la valoración de impactos y de aspectos de responsabilidad legal, ética y profesional relacionados con el trabajo)
\chapter{Resultados y Discusión}
\input{capitulos/resultados_discusion/main.tex}

% Conclusiones
\chapter{Conclusiones}
\input{capitulos/conclusiones/main.tex}

% Planificación temporal y presupuesto
\chapter{Planificación Temporal y Presupuesto}
\input{capitulos/planificacion_presupuesto/main.tex}

% Bibliografía
\newpage
\addcontentsline{toc}{chapter}{Bibliografía}
\printbibliography

\end{document}


% Objetivos
\chapter{Objetivos}
\documentclass[a4paper,11pt,twoside]{report}
\usepackage[left=25mm,right=25mm,top=25mm,bottom=25mm,includehead,includefoot,headsep=15mm,footskip=15mm]{geometry}
\usepackage{graphicx}
\usepackage{fancyhdr}
\usepackage{titlesec}
\usepackage[spanish]{babel}
\usepackage[utf8]{inputenc}
\usepackage{amsmath}
\usepackage{setspace}
\usepackage{svg}
\usepackage{hyperref}
\usepackage[backend=biber,style=numeric]{biblatex}
\addbibresource{references.bib}
\hypersetup{
    colorlinks=true,
    linkcolor=blue,      % color of internal links (sections, etc.)
    urlcolor=blue,       % color of external links
    pdftitle={Optimización energética de sistema híbrido con bomba de calor, suelo radiante, fotovoltaica y almacenamiento para vivienda},    % title
    pdfauthor={Luis D. Aranda Sánchez},     % author
    pdfkeywords={palabra1, palabra2, código1, etc.} % list of keywords
}

% Font change to Arial
\usepackage{helvet}
\renewcommand{\familydefault}{\sfdefault}

% Chapter titles in uppercase and larger font
\titleformat{\chapter}[hang]{\large\bfseries}{\thechapter.}{1em}{\MakeUppercase}
\titleformat{\section}[hang]{\bfseries}{\thesection.}{1em}{}
\titleformat{\subsection}[hang]{\bfseries}{\thesubsection.}{1em}{}

% Fancyhdr setup
\setlength{\headheight}{14.30174pt} % Adjust to recommended value, slightly larger for safety
\fancyhf{} % Clear all headers and footers
\fancyhead[LE]{\nouppercase{\leftmark}}
\fancyhead[RO]{Optimización energética para vivienda}
\fancyfoot[LE]{\thepage}
\fancyfoot[RE]{Escuela Técnica Superior de Ingenieros Industriales (UPM)}
\fancyfoot[LO]{Luis D. Aranda Sánchez}
\fancyfoot[RO]{\thepage}
\renewcommand{\headrulewidth}{0.4pt}
\renewcommand{\footrulewidth}{0.4pt}

\fancypagestyle{myfancy}{
    \fancyhf{} % Clear all headers and footers
    \fancyhead[LE]{\nouppercase{\leftmark}}
    \fancyhead[RO]{Optimización energética para vivienda}
    \fancyfoot[LE]{\thepage}
    \fancyfoot[RE]{Escuela Técnica Superior de Ingenieros Industriales (UPM)}
    \fancyfoot[LO]{Luis D. Aranda Sánchez}
    \fancyfoot[RO]{\thepage}
    \renewcommand{\headrulewidth}{0.4pt}
    \renewcommand{\footrulewidth}{0.4pt}
}

\fancypagestyle{simple}{
    \fancyhf{} % Clear all headers and footers
    \renewcommand{\headrulewidth}{0pt}
    \renewcommand{\footrulewidth}{0pt}
}

% Line spacing
\setstretch{1.2}

% Document starts here
\begin{document}

% Portada
\begin{titlepage}
    \centering
    {\scshape\LARGE Universidad Politécnica de Madrid \par}
    \vspace{1cm}
    {\scshape\Large Escuela Técnica Superior de Ingenieros Industriales\par}
    \vspace{1.5cm}
    {\huge\bfseries Optimización energética de sistema híbrido con bomba de calor, suelo radiante, fotovoltaica y almacenamiento para vivienda \par}
    \vspace{1.5cm}
    {\Large\bfseries Trabajo de Fin de Máster\par}
    \vspace{0.5cm}
    {\large Máster Universitario en Ingeniería de la Energía \par}
    \vspace{2cm}
    {\Large Luis D. Aranda Sánchez\par}
    \vfill
    Director: Javier Rodríguez Martín
    \vfill
    {\large Septiembre 6, 2024\par}
\end{titlepage}

% Resumen (máximo de 5 páginas, incluyendo al final Palabras clave)
\clearpage
\pagestyle{simple}
% \newpage
\chapter*{Resumen}
\addcontentsline{toc}{chapter}{Resumen}
\input{capitulos/resumen/main.tex}

% Índice (paginado)
\clearpage
\pagestyle{simple}
% \newpage
\tableofcontents

% Introducción (donde se incluya los antecedentes y justificación)
\clearpage
\pagestyle{myfancy}
\newpage
\chapter{Introducción}
\input{capitulos/introduccion/main.tex}

% Objetivos
\chapter{Objetivos}
\input{capitulos/objetivos/main.tex}

% Metodología
\chapter{Metodología}
\input{capitulos/metodologia/main.tex}

% Resultados y discusión (incluyendo la valoración de impactos y de aspectos de responsabilidad legal, ética y profesional relacionados con el trabajo)
\chapter{Resultados y Discusión}
\input{capitulos/resultados_discusion/main.tex}

% Conclusiones
\chapter{Conclusiones}
\input{capitulos/conclusiones/main.tex}

% Planificación temporal y presupuesto
\chapter{Planificación Temporal y Presupuesto}
\input{capitulos/planificacion_presupuesto/main.tex}

% Bibliografía
\newpage
\addcontentsline{toc}{chapter}{Bibliografía}
\printbibliography

\end{document}


% Metodología
\chapter{Metodología}
\documentclass[a4paper,11pt,twoside]{report}
\usepackage[left=25mm,right=25mm,top=25mm,bottom=25mm,includehead,includefoot,headsep=15mm,footskip=15mm]{geometry}
\usepackage{graphicx}
\usepackage{fancyhdr}
\usepackage{titlesec}
\usepackage[spanish]{babel}
\usepackage[utf8]{inputenc}
\usepackage{amsmath}
\usepackage{setspace}
\usepackage{svg}
\usepackage{hyperref}
\usepackage[backend=biber,style=numeric]{biblatex}
\addbibresource{references.bib}
\hypersetup{
    colorlinks=true,
    linkcolor=blue,      % color of internal links (sections, etc.)
    urlcolor=blue,       % color of external links
    pdftitle={Optimización energética de sistema híbrido con bomba de calor, suelo radiante, fotovoltaica y almacenamiento para vivienda},    % title
    pdfauthor={Luis D. Aranda Sánchez},     % author
    pdfkeywords={palabra1, palabra2, código1, etc.} % list of keywords
}

% Font change to Arial
\usepackage{helvet}
\renewcommand{\familydefault}{\sfdefault}

% Chapter titles in uppercase and larger font
\titleformat{\chapter}[hang]{\large\bfseries}{\thechapter.}{1em}{\MakeUppercase}
\titleformat{\section}[hang]{\bfseries}{\thesection.}{1em}{}
\titleformat{\subsection}[hang]{\bfseries}{\thesubsection.}{1em}{}

% Fancyhdr setup
\setlength{\headheight}{14.30174pt} % Adjust to recommended value, slightly larger for safety
\fancyhf{} % Clear all headers and footers
\fancyhead[LE]{\nouppercase{\leftmark}}
\fancyhead[RO]{Optimización energética para vivienda}
\fancyfoot[LE]{\thepage}
\fancyfoot[RE]{Escuela Técnica Superior de Ingenieros Industriales (UPM)}
\fancyfoot[LO]{Luis D. Aranda Sánchez}
\fancyfoot[RO]{\thepage}
\renewcommand{\headrulewidth}{0.4pt}
\renewcommand{\footrulewidth}{0.4pt}

\fancypagestyle{myfancy}{
    \fancyhf{} % Clear all headers and footers
    \fancyhead[LE]{\nouppercase{\leftmark}}
    \fancyhead[RO]{Optimización energética para vivienda}
    \fancyfoot[LE]{\thepage}
    \fancyfoot[RE]{Escuela Técnica Superior de Ingenieros Industriales (UPM)}
    \fancyfoot[LO]{Luis D. Aranda Sánchez}
    \fancyfoot[RO]{\thepage}
    \renewcommand{\headrulewidth}{0.4pt}
    \renewcommand{\footrulewidth}{0.4pt}
}

\fancypagestyle{simple}{
    \fancyhf{} % Clear all headers and footers
    \renewcommand{\headrulewidth}{0pt}
    \renewcommand{\footrulewidth}{0pt}
}

% Line spacing
\setstretch{1.2}

% Document starts here
\begin{document}

% Portada
\begin{titlepage}
    \centering
    {\scshape\LARGE Universidad Politécnica de Madrid \par}
    \vspace{1cm}
    {\scshape\Large Escuela Técnica Superior de Ingenieros Industriales\par}
    \vspace{1.5cm}
    {\huge\bfseries Optimización energética de sistema híbrido con bomba de calor, suelo radiante, fotovoltaica y almacenamiento para vivienda \par}
    \vspace{1.5cm}
    {\Large\bfseries Trabajo de Fin de Máster\par}
    \vspace{0.5cm}
    {\large Máster Universitario en Ingeniería de la Energía \par}
    \vspace{2cm}
    {\Large Luis D. Aranda Sánchez\par}
    \vfill
    Director: Javier Rodríguez Martín
    \vfill
    {\large Septiembre 6, 2024\par}
\end{titlepage}

% Resumen (máximo de 5 páginas, incluyendo al final Palabras clave)
\clearpage
\pagestyle{simple}
% \newpage
\chapter*{Resumen}
\addcontentsline{toc}{chapter}{Resumen}
\input{capitulos/resumen/main.tex}

% Índice (paginado)
\clearpage
\pagestyle{simple}
% \newpage
\tableofcontents

% Introducción (donde se incluya los antecedentes y justificación)
\clearpage
\pagestyle{myfancy}
\newpage
\chapter{Introducción}
\input{capitulos/introduccion/main.tex}

% Objetivos
\chapter{Objetivos}
\input{capitulos/objetivos/main.tex}

% Metodología
\chapter{Metodología}
\input{capitulos/metodologia/main.tex}

% Resultados y discusión (incluyendo la valoración de impactos y de aspectos de responsabilidad legal, ética y profesional relacionados con el trabajo)
\chapter{Resultados y Discusión}
\input{capitulos/resultados_discusion/main.tex}

% Conclusiones
\chapter{Conclusiones}
\input{capitulos/conclusiones/main.tex}

% Planificación temporal y presupuesto
\chapter{Planificación Temporal y Presupuesto}
\input{capitulos/planificacion_presupuesto/main.tex}

% Bibliografía
\newpage
\addcontentsline{toc}{chapter}{Bibliografía}
\printbibliography

\end{document}


% Resultados y discusión (incluyendo la valoración de impactos y de aspectos de responsabilidad legal, ética y profesional relacionados con el trabajo)
\chapter{Resultados y Discusión}
\documentclass[a4paper,11pt,twoside]{report}
\usepackage[left=25mm,right=25mm,top=25mm,bottom=25mm,includehead,includefoot,headsep=15mm,footskip=15mm]{geometry}
\usepackage{graphicx}
\usepackage{fancyhdr}
\usepackage{titlesec}
\usepackage[spanish]{babel}
\usepackage[utf8]{inputenc}
\usepackage{amsmath}
\usepackage{setspace}
\usepackage{svg}
\usepackage{hyperref}
\usepackage[backend=biber,style=numeric]{biblatex}
\addbibresource{references.bib}
\hypersetup{
    colorlinks=true,
    linkcolor=blue,      % color of internal links (sections, etc.)
    urlcolor=blue,       % color of external links
    pdftitle={Optimización energética de sistema híbrido con bomba de calor, suelo radiante, fotovoltaica y almacenamiento para vivienda},    % title
    pdfauthor={Luis D. Aranda Sánchez},     % author
    pdfkeywords={palabra1, palabra2, código1, etc.} % list of keywords
}

% Font change to Arial
\usepackage{helvet}
\renewcommand{\familydefault}{\sfdefault}

% Chapter titles in uppercase and larger font
\titleformat{\chapter}[hang]{\large\bfseries}{\thechapter.}{1em}{\MakeUppercase}
\titleformat{\section}[hang]{\bfseries}{\thesection.}{1em}{}
\titleformat{\subsection}[hang]{\bfseries}{\thesubsection.}{1em}{}

% Fancyhdr setup
\setlength{\headheight}{14.30174pt} % Adjust to recommended value, slightly larger for safety
\fancyhf{} % Clear all headers and footers
\fancyhead[LE]{\nouppercase{\leftmark}}
\fancyhead[RO]{Optimización energética para vivienda}
\fancyfoot[LE]{\thepage}
\fancyfoot[RE]{Escuela Técnica Superior de Ingenieros Industriales (UPM)}
\fancyfoot[LO]{Luis D. Aranda Sánchez}
\fancyfoot[RO]{\thepage}
\renewcommand{\headrulewidth}{0.4pt}
\renewcommand{\footrulewidth}{0.4pt}

\fancypagestyle{myfancy}{
    \fancyhf{} % Clear all headers and footers
    \fancyhead[LE]{\nouppercase{\leftmark}}
    \fancyhead[RO]{Optimización energética para vivienda}
    \fancyfoot[LE]{\thepage}
    \fancyfoot[RE]{Escuela Técnica Superior de Ingenieros Industriales (UPM)}
    \fancyfoot[LO]{Luis D. Aranda Sánchez}
    \fancyfoot[RO]{\thepage}
    \renewcommand{\headrulewidth}{0.4pt}
    \renewcommand{\footrulewidth}{0.4pt}
}

\fancypagestyle{simple}{
    \fancyhf{} % Clear all headers and footers
    \renewcommand{\headrulewidth}{0pt}
    \renewcommand{\footrulewidth}{0pt}
}

% Line spacing
\setstretch{1.2}

% Document starts here
\begin{document}

% Portada
\begin{titlepage}
    \centering
    {\scshape\LARGE Universidad Politécnica de Madrid \par}
    \vspace{1cm}
    {\scshape\Large Escuela Técnica Superior de Ingenieros Industriales\par}
    \vspace{1.5cm}
    {\huge\bfseries Optimización energética de sistema híbrido con bomba de calor, suelo radiante, fotovoltaica y almacenamiento para vivienda \par}
    \vspace{1.5cm}
    {\Large\bfseries Trabajo de Fin de Máster\par}
    \vspace{0.5cm}
    {\large Máster Universitario en Ingeniería de la Energía \par}
    \vspace{2cm}
    {\Large Luis D. Aranda Sánchez\par}
    \vfill
    Director: Javier Rodríguez Martín
    \vfill
    {\large Septiembre 6, 2024\par}
\end{titlepage}

% Resumen (máximo de 5 páginas, incluyendo al final Palabras clave)
\clearpage
\pagestyle{simple}
% \newpage
\chapter*{Resumen}
\addcontentsline{toc}{chapter}{Resumen}
\input{capitulos/resumen/main.tex}

% Índice (paginado)
\clearpage
\pagestyle{simple}
% \newpage
\tableofcontents

% Introducción (donde se incluya los antecedentes y justificación)
\clearpage
\pagestyle{myfancy}
\newpage
\chapter{Introducción}
\input{capitulos/introduccion/main.tex}

% Objetivos
\chapter{Objetivos}
\input{capitulos/objetivos/main.tex}

% Metodología
\chapter{Metodología}
\input{capitulos/metodologia/main.tex}

% Resultados y discusión (incluyendo la valoración de impactos y de aspectos de responsabilidad legal, ética y profesional relacionados con el trabajo)
\chapter{Resultados y Discusión}
\input{capitulos/resultados_discusion/main.tex}

% Conclusiones
\chapter{Conclusiones}
\input{capitulos/conclusiones/main.tex}

% Planificación temporal y presupuesto
\chapter{Planificación Temporal y Presupuesto}
\input{capitulos/planificacion_presupuesto/main.tex}

% Bibliografía
\newpage
\addcontentsline{toc}{chapter}{Bibliografía}
\printbibliography

\end{document}


% Conclusiones
\chapter{Conclusiones}
\documentclass[a4paper,11pt,twoside]{report}
\usepackage[left=25mm,right=25mm,top=25mm,bottom=25mm,includehead,includefoot,headsep=15mm,footskip=15mm]{geometry}
\usepackage{graphicx}
\usepackage{fancyhdr}
\usepackage{titlesec}
\usepackage[spanish]{babel}
\usepackage[utf8]{inputenc}
\usepackage{amsmath}
\usepackage{setspace}
\usepackage{svg}
\usepackage{hyperref}
\usepackage[backend=biber,style=numeric]{biblatex}
\addbibresource{references.bib}
\hypersetup{
    colorlinks=true,
    linkcolor=blue,      % color of internal links (sections, etc.)
    urlcolor=blue,       % color of external links
    pdftitle={Optimización energética de sistema híbrido con bomba de calor, suelo radiante, fotovoltaica y almacenamiento para vivienda},    % title
    pdfauthor={Luis D. Aranda Sánchez},     % author
    pdfkeywords={palabra1, palabra2, código1, etc.} % list of keywords
}

% Font change to Arial
\usepackage{helvet}
\renewcommand{\familydefault}{\sfdefault}

% Chapter titles in uppercase and larger font
\titleformat{\chapter}[hang]{\large\bfseries}{\thechapter.}{1em}{\MakeUppercase}
\titleformat{\section}[hang]{\bfseries}{\thesection.}{1em}{}
\titleformat{\subsection}[hang]{\bfseries}{\thesubsection.}{1em}{}

% Fancyhdr setup
\setlength{\headheight}{14.30174pt} % Adjust to recommended value, slightly larger for safety
\fancyhf{} % Clear all headers and footers
\fancyhead[LE]{\nouppercase{\leftmark}}
\fancyhead[RO]{Optimización energética para vivienda}
\fancyfoot[LE]{\thepage}
\fancyfoot[RE]{Escuela Técnica Superior de Ingenieros Industriales (UPM)}
\fancyfoot[LO]{Luis D. Aranda Sánchez}
\fancyfoot[RO]{\thepage}
\renewcommand{\headrulewidth}{0.4pt}
\renewcommand{\footrulewidth}{0.4pt}

\fancypagestyle{myfancy}{
    \fancyhf{} % Clear all headers and footers
    \fancyhead[LE]{\nouppercase{\leftmark}}
    \fancyhead[RO]{Optimización energética para vivienda}
    \fancyfoot[LE]{\thepage}
    \fancyfoot[RE]{Escuela Técnica Superior de Ingenieros Industriales (UPM)}
    \fancyfoot[LO]{Luis D. Aranda Sánchez}
    \fancyfoot[RO]{\thepage}
    \renewcommand{\headrulewidth}{0.4pt}
    \renewcommand{\footrulewidth}{0.4pt}
}

\fancypagestyle{simple}{
    \fancyhf{} % Clear all headers and footers
    \renewcommand{\headrulewidth}{0pt}
    \renewcommand{\footrulewidth}{0pt}
}

% Line spacing
\setstretch{1.2}

% Document starts here
\begin{document}

% Portada
\begin{titlepage}
    \centering
    {\scshape\LARGE Universidad Politécnica de Madrid \par}
    \vspace{1cm}
    {\scshape\Large Escuela Técnica Superior de Ingenieros Industriales\par}
    \vspace{1.5cm}
    {\huge\bfseries Optimización energética de sistema híbrido con bomba de calor, suelo radiante, fotovoltaica y almacenamiento para vivienda \par}
    \vspace{1.5cm}
    {\Large\bfseries Trabajo de Fin de Máster\par}
    \vspace{0.5cm}
    {\large Máster Universitario en Ingeniería de la Energía \par}
    \vspace{2cm}
    {\Large Luis D. Aranda Sánchez\par}
    \vfill
    Director: Javier Rodríguez Martín
    \vfill
    {\large Septiembre 6, 2024\par}
\end{titlepage}

% Resumen (máximo de 5 páginas, incluyendo al final Palabras clave)
\clearpage
\pagestyle{simple}
% \newpage
\chapter*{Resumen}
\addcontentsline{toc}{chapter}{Resumen}
\input{capitulos/resumen/main.tex}

% Índice (paginado)
\clearpage
\pagestyle{simple}
% \newpage
\tableofcontents

% Introducción (donde se incluya los antecedentes y justificación)
\clearpage
\pagestyle{myfancy}
\newpage
\chapter{Introducción}
\input{capitulos/introduccion/main.tex}

% Objetivos
\chapter{Objetivos}
\input{capitulos/objetivos/main.tex}

% Metodología
\chapter{Metodología}
\input{capitulos/metodologia/main.tex}

% Resultados y discusión (incluyendo la valoración de impactos y de aspectos de responsabilidad legal, ética y profesional relacionados con el trabajo)
\chapter{Resultados y Discusión}
\input{capitulos/resultados_discusion/main.tex}

% Conclusiones
\chapter{Conclusiones}
\input{capitulos/conclusiones/main.tex}

% Planificación temporal y presupuesto
\chapter{Planificación Temporal y Presupuesto}
\input{capitulos/planificacion_presupuesto/main.tex}

% Bibliografía
\newpage
\addcontentsline{toc}{chapter}{Bibliografía}
\printbibliography

\end{document}


% Planificación temporal y presupuesto
\chapter{Planificación Temporal y Presupuesto}
\documentclass[a4paper,11pt,twoside]{report}
\usepackage[left=25mm,right=25mm,top=25mm,bottom=25mm,includehead,includefoot,headsep=15mm,footskip=15mm]{geometry}
\usepackage{graphicx}
\usepackage{fancyhdr}
\usepackage{titlesec}
\usepackage[spanish]{babel}
\usepackage[utf8]{inputenc}
\usepackage{amsmath}
\usepackage{setspace}
\usepackage{svg}
\usepackage{hyperref}
\usepackage[backend=biber,style=numeric]{biblatex}
\addbibresource{references.bib}
\hypersetup{
    colorlinks=true,
    linkcolor=blue,      % color of internal links (sections, etc.)
    urlcolor=blue,       % color of external links
    pdftitle={Optimización energética de sistema híbrido con bomba de calor, suelo radiante, fotovoltaica y almacenamiento para vivienda},    % title
    pdfauthor={Luis D. Aranda Sánchez},     % author
    pdfkeywords={palabra1, palabra2, código1, etc.} % list of keywords
}

% Font change to Arial
\usepackage{helvet}
\renewcommand{\familydefault}{\sfdefault}

% Chapter titles in uppercase and larger font
\titleformat{\chapter}[hang]{\large\bfseries}{\thechapter.}{1em}{\MakeUppercase}
\titleformat{\section}[hang]{\bfseries}{\thesection.}{1em}{}
\titleformat{\subsection}[hang]{\bfseries}{\thesubsection.}{1em}{}

% Fancyhdr setup
\setlength{\headheight}{14.30174pt} % Adjust to recommended value, slightly larger for safety
\fancyhf{} % Clear all headers and footers
\fancyhead[LE]{\nouppercase{\leftmark}}
\fancyhead[RO]{Optimización energética para vivienda}
\fancyfoot[LE]{\thepage}
\fancyfoot[RE]{Escuela Técnica Superior de Ingenieros Industriales (UPM)}
\fancyfoot[LO]{Luis D. Aranda Sánchez}
\fancyfoot[RO]{\thepage}
\renewcommand{\headrulewidth}{0.4pt}
\renewcommand{\footrulewidth}{0.4pt}

\fancypagestyle{myfancy}{
    \fancyhf{} % Clear all headers and footers
    \fancyhead[LE]{\nouppercase{\leftmark}}
    \fancyhead[RO]{Optimización energética para vivienda}
    \fancyfoot[LE]{\thepage}
    \fancyfoot[RE]{Escuela Técnica Superior de Ingenieros Industriales (UPM)}
    \fancyfoot[LO]{Luis D. Aranda Sánchez}
    \fancyfoot[RO]{\thepage}
    \renewcommand{\headrulewidth}{0.4pt}
    \renewcommand{\footrulewidth}{0.4pt}
}

\fancypagestyle{simple}{
    \fancyhf{} % Clear all headers and footers
    \renewcommand{\headrulewidth}{0pt}
    \renewcommand{\footrulewidth}{0pt}
}

% Line spacing
\setstretch{1.2}

% Document starts here
\begin{document}

% Portada
\begin{titlepage}
    \centering
    {\scshape\LARGE Universidad Politécnica de Madrid \par}
    \vspace{1cm}
    {\scshape\Large Escuela Técnica Superior de Ingenieros Industriales\par}
    \vspace{1.5cm}
    {\huge\bfseries Optimización energética de sistema híbrido con bomba de calor, suelo radiante, fotovoltaica y almacenamiento para vivienda \par}
    \vspace{1.5cm}
    {\Large\bfseries Trabajo de Fin de Máster\par}
    \vspace{0.5cm}
    {\large Máster Universitario en Ingeniería de la Energía \par}
    \vspace{2cm}
    {\Large Luis D. Aranda Sánchez\par}
    \vfill
    Director: Javier Rodríguez Martín
    \vfill
    {\large Septiembre 6, 2024\par}
\end{titlepage}

% Resumen (máximo de 5 páginas, incluyendo al final Palabras clave)
\clearpage
\pagestyle{simple}
% \newpage
\chapter*{Resumen}
\addcontentsline{toc}{chapter}{Resumen}
\input{capitulos/resumen/main.tex}

% Índice (paginado)
\clearpage
\pagestyle{simple}
% \newpage
\tableofcontents

% Introducción (donde se incluya los antecedentes y justificación)
\clearpage
\pagestyle{myfancy}
\newpage
\chapter{Introducción}
\input{capitulos/introduccion/main.tex}

% Objetivos
\chapter{Objetivos}
\input{capitulos/objetivos/main.tex}

% Metodología
\chapter{Metodología}
\input{capitulos/metodologia/main.tex}

% Resultados y discusión (incluyendo la valoración de impactos y de aspectos de responsabilidad legal, ética y profesional relacionados con el trabajo)
\chapter{Resultados y Discusión}
\input{capitulos/resultados_discusion/main.tex}

% Conclusiones
\chapter{Conclusiones}
\input{capitulos/conclusiones/main.tex}

% Planificación temporal y presupuesto
\chapter{Planificación Temporal y Presupuesto}
\input{capitulos/planificacion_presupuesto/main.tex}

% Bibliografía
\newpage
\addcontentsline{toc}{chapter}{Bibliografía}
\printbibliography

\end{document}


% Bibliografía
\newpage
\addcontentsline{toc}{chapter}{Bibliografía}
\printbibliography

\end{document}


% Índice (paginado)
\clearpage
\pagestyle{simple}
% \newpage
\tableofcontents

% Introducción (donde se incluya los antecedentes y justificación)
\clearpage
\pagestyle{myfancy}
\newpage
\chapter{Introducción}
\documentclass[a4paper,11pt,twoside]{report}
\usepackage[left=25mm,right=25mm,top=25mm,bottom=25mm,includehead,includefoot,headsep=15mm,footskip=15mm]{geometry}
\usepackage{graphicx}
\usepackage{fancyhdr}
\usepackage{titlesec}
\usepackage[spanish]{babel}
\usepackage[utf8]{inputenc}
\usepackage{amsmath}
\usepackage{setspace}
\usepackage{svg}
\usepackage{hyperref}
\usepackage[backend=biber,style=numeric]{biblatex}
\addbibresource{references.bib}
\hypersetup{
    colorlinks=true,
    linkcolor=blue,      % color of internal links (sections, etc.)
    urlcolor=blue,       % color of external links
    pdftitle={Optimización energética de sistema híbrido con bomba de calor, suelo radiante, fotovoltaica y almacenamiento para vivienda},    % title
    pdfauthor={Luis D. Aranda Sánchez},     % author
    pdfkeywords={palabra1, palabra2, código1, etc.} % list of keywords
}

% Font change to Arial
\usepackage{helvet}
\renewcommand{\familydefault}{\sfdefault}

% Chapter titles in uppercase and larger font
\titleformat{\chapter}[hang]{\large\bfseries}{\thechapter.}{1em}{\MakeUppercase}
\titleformat{\section}[hang]{\bfseries}{\thesection.}{1em}{}
\titleformat{\subsection}[hang]{\bfseries}{\thesubsection.}{1em}{}

% Fancyhdr setup
\setlength{\headheight}{14.30174pt} % Adjust to recommended value, slightly larger for safety
\fancyhf{} % Clear all headers and footers
\fancyhead[LE]{\nouppercase{\leftmark}}
\fancyhead[RO]{Optimización energética para vivienda}
\fancyfoot[LE]{\thepage}
\fancyfoot[RE]{Escuela Técnica Superior de Ingenieros Industriales (UPM)}
\fancyfoot[LO]{Luis D. Aranda Sánchez}
\fancyfoot[RO]{\thepage}
\renewcommand{\headrulewidth}{0.4pt}
\renewcommand{\footrulewidth}{0.4pt}

\fancypagestyle{myfancy}{
    \fancyhf{} % Clear all headers and footers
    \fancyhead[LE]{\nouppercase{\leftmark}}
    \fancyhead[RO]{Optimización energética para vivienda}
    \fancyfoot[LE]{\thepage}
    \fancyfoot[RE]{Escuela Técnica Superior de Ingenieros Industriales (UPM)}
    \fancyfoot[LO]{Luis D. Aranda Sánchez}
    \fancyfoot[RO]{\thepage}
    \renewcommand{\headrulewidth}{0.4pt}
    \renewcommand{\footrulewidth}{0.4pt}
}

\fancypagestyle{simple}{
    \fancyhf{} % Clear all headers and footers
    \renewcommand{\headrulewidth}{0pt}
    \renewcommand{\footrulewidth}{0pt}
}

% Line spacing
\setstretch{1.2}

% Document starts here
\begin{document}

% Portada
\begin{titlepage}
    \centering
    {\scshape\LARGE Universidad Politécnica de Madrid \par}
    \vspace{1cm}
    {\scshape\Large Escuela Técnica Superior de Ingenieros Industriales\par}
    \vspace{1.5cm}
    {\huge\bfseries Optimización energética de sistema híbrido con bomba de calor, suelo radiante, fotovoltaica y almacenamiento para vivienda \par}
    \vspace{1.5cm}
    {\Large\bfseries Trabajo de Fin de Máster\par}
    \vspace{0.5cm}
    {\large Máster Universitario en Ingeniería de la Energía \par}
    \vspace{2cm}
    {\Large Luis D. Aranda Sánchez\par}
    \vfill
    Director: Javier Rodríguez Martín
    \vfill
    {\large Septiembre 6, 2024\par}
\end{titlepage}

% Resumen (máximo de 5 páginas, incluyendo al final Palabras clave)
\clearpage
\pagestyle{simple}
% \newpage
\chapter*{Resumen}
\addcontentsline{toc}{chapter}{Resumen}
\documentclass[a4paper,11pt,twoside]{report}
\usepackage[left=25mm,right=25mm,top=25mm,bottom=25mm,includehead,includefoot,headsep=15mm,footskip=15mm]{geometry}
\usepackage{graphicx}
\usepackage{fancyhdr}
\usepackage{titlesec}
\usepackage[spanish]{babel}
\usepackage[utf8]{inputenc}
\usepackage{amsmath}
\usepackage{setspace}
\usepackage{svg}
\usepackage{hyperref}
\usepackage[backend=biber,style=numeric]{biblatex}
\addbibresource{references.bib}
\hypersetup{
    colorlinks=true,
    linkcolor=blue,      % color of internal links (sections, etc.)
    urlcolor=blue,       % color of external links
    pdftitle={Optimización energética de sistema híbrido con bomba de calor, suelo radiante, fotovoltaica y almacenamiento para vivienda},    % title
    pdfauthor={Luis D. Aranda Sánchez},     % author
    pdfkeywords={palabra1, palabra2, código1, etc.} % list of keywords
}

% Font change to Arial
\usepackage{helvet}
\renewcommand{\familydefault}{\sfdefault}

% Chapter titles in uppercase and larger font
\titleformat{\chapter}[hang]{\large\bfseries}{\thechapter.}{1em}{\MakeUppercase}
\titleformat{\section}[hang]{\bfseries}{\thesection.}{1em}{}
\titleformat{\subsection}[hang]{\bfseries}{\thesubsection.}{1em}{}

% Fancyhdr setup
\setlength{\headheight}{14.30174pt} % Adjust to recommended value, slightly larger for safety
\fancyhf{} % Clear all headers and footers
\fancyhead[LE]{\nouppercase{\leftmark}}
\fancyhead[RO]{Optimización energética para vivienda}
\fancyfoot[LE]{\thepage}
\fancyfoot[RE]{Escuela Técnica Superior de Ingenieros Industriales (UPM)}
\fancyfoot[LO]{Luis D. Aranda Sánchez}
\fancyfoot[RO]{\thepage}
\renewcommand{\headrulewidth}{0.4pt}
\renewcommand{\footrulewidth}{0.4pt}

\fancypagestyle{myfancy}{
    \fancyhf{} % Clear all headers and footers
    \fancyhead[LE]{\nouppercase{\leftmark}}
    \fancyhead[RO]{Optimización energética para vivienda}
    \fancyfoot[LE]{\thepage}
    \fancyfoot[RE]{Escuela Técnica Superior de Ingenieros Industriales (UPM)}
    \fancyfoot[LO]{Luis D. Aranda Sánchez}
    \fancyfoot[RO]{\thepage}
    \renewcommand{\headrulewidth}{0.4pt}
    \renewcommand{\footrulewidth}{0.4pt}
}

\fancypagestyle{simple}{
    \fancyhf{} % Clear all headers and footers
    \renewcommand{\headrulewidth}{0pt}
    \renewcommand{\footrulewidth}{0pt}
}

% Line spacing
\setstretch{1.2}

% Document starts here
\begin{document}

% Portada
\begin{titlepage}
    \centering
    {\scshape\LARGE Universidad Politécnica de Madrid \par}
    \vspace{1cm}
    {\scshape\Large Escuela Técnica Superior de Ingenieros Industriales\par}
    \vspace{1.5cm}
    {\huge\bfseries Optimización energética de sistema híbrido con bomba de calor, suelo radiante, fotovoltaica y almacenamiento para vivienda \par}
    \vspace{1.5cm}
    {\Large\bfseries Trabajo de Fin de Máster\par}
    \vspace{0.5cm}
    {\large Máster Universitario en Ingeniería de la Energía \par}
    \vspace{2cm}
    {\Large Luis D. Aranda Sánchez\par}
    \vfill
    Director: Javier Rodríguez Martín
    \vfill
    {\large Septiembre 6, 2024\par}
\end{titlepage}

% Resumen (máximo de 5 páginas, incluyendo al final Palabras clave)
\clearpage
\pagestyle{simple}
% \newpage
\chapter*{Resumen}
\addcontentsline{toc}{chapter}{Resumen}
\input{capitulos/resumen/main.tex}

% Índice (paginado)
\clearpage
\pagestyle{simple}
% \newpage
\tableofcontents

% Introducción (donde se incluya los antecedentes y justificación)
\clearpage
\pagestyle{myfancy}
\newpage
\chapter{Introducción}
\input{capitulos/introduccion/main.tex}

% Objetivos
\chapter{Objetivos}
\input{capitulos/objetivos/main.tex}

% Metodología
\chapter{Metodología}
\input{capitulos/metodologia/main.tex}

% Resultados y discusión (incluyendo la valoración de impactos y de aspectos de responsabilidad legal, ética y profesional relacionados con el trabajo)
\chapter{Resultados y Discusión}
\input{capitulos/resultados_discusion/main.tex}

% Conclusiones
\chapter{Conclusiones}
\input{capitulos/conclusiones/main.tex}

% Planificación temporal y presupuesto
\chapter{Planificación Temporal y Presupuesto}
\input{capitulos/planificacion_presupuesto/main.tex}

% Bibliografía
\newpage
\addcontentsline{toc}{chapter}{Bibliografía}
\printbibliography

\end{document}


% Índice (paginado)
\clearpage
\pagestyle{simple}
% \newpage
\tableofcontents

% Introducción (donde se incluya los antecedentes y justificación)
\clearpage
\pagestyle{myfancy}
\newpage
\chapter{Introducción}
\documentclass[a4paper,11pt,twoside]{report}
\usepackage[left=25mm,right=25mm,top=25mm,bottom=25mm,includehead,includefoot,headsep=15mm,footskip=15mm]{geometry}
\usepackage{graphicx}
\usepackage{fancyhdr}
\usepackage{titlesec}
\usepackage[spanish]{babel}
\usepackage[utf8]{inputenc}
\usepackage{amsmath}
\usepackage{setspace}
\usepackage{svg}
\usepackage{hyperref}
\usepackage[backend=biber,style=numeric]{biblatex}
\addbibresource{references.bib}
\hypersetup{
    colorlinks=true,
    linkcolor=blue,      % color of internal links (sections, etc.)
    urlcolor=blue,       % color of external links
    pdftitle={Optimización energética de sistema híbrido con bomba de calor, suelo radiante, fotovoltaica y almacenamiento para vivienda},    % title
    pdfauthor={Luis D. Aranda Sánchez},     % author
    pdfkeywords={palabra1, palabra2, código1, etc.} % list of keywords
}

% Font change to Arial
\usepackage{helvet}
\renewcommand{\familydefault}{\sfdefault}

% Chapter titles in uppercase and larger font
\titleformat{\chapter}[hang]{\large\bfseries}{\thechapter.}{1em}{\MakeUppercase}
\titleformat{\section}[hang]{\bfseries}{\thesection.}{1em}{}
\titleformat{\subsection}[hang]{\bfseries}{\thesubsection.}{1em}{}

% Fancyhdr setup
\setlength{\headheight}{14.30174pt} % Adjust to recommended value, slightly larger for safety
\fancyhf{} % Clear all headers and footers
\fancyhead[LE]{\nouppercase{\leftmark}}
\fancyhead[RO]{Optimización energética para vivienda}
\fancyfoot[LE]{\thepage}
\fancyfoot[RE]{Escuela Técnica Superior de Ingenieros Industriales (UPM)}
\fancyfoot[LO]{Luis D. Aranda Sánchez}
\fancyfoot[RO]{\thepage}
\renewcommand{\headrulewidth}{0.4pt}
\renewcommand{\footrulewidth}{0.4pt}

\fancypagestyle{myfancy}{
    \fancyhf{} % Clear all headers and footers
    \fancyhead[LE]{\nouppercase{\leftmark}}
    \fancyhead[RO]{Optimización energética para vivienda}
    \fancyfoot[LE]{\thepage}
    \fancyfoot[RE]{Escuela Técnica Superior de Ingenieros Industriales (UPM)}
    \fancyfoot[LO]{Luis D. Aranda Sánchez}
    \fancyfoot[RO]{\thepage}
    \renewcommand{\headrulewidth}{0.4pt}
    \renewcommand{\footrulewidth}{0.4pt}
}

\fancypagestyle{simple}{
    \fancyhf{} % Clear all headers and footers
    \renewcommand{\headrulewidth}{0pt}
    \renewcommand{\footrulewidth}{0pt}
}

% Line spacing
\setstretch{1.2}

% Document starts here
\begin{document}

% Portada
\begin{titlepage}
    \centering
    {\scshape\LARGE Universidad Politécnica de Madrid \par}
    \vspace{1cm}
    {\scshape\Large Escuela Técnica Superior de Ingenieros Industriales\par}
    \vspace{1.5cm}
    {\huge\bfseries Optimización energética de sistema híbrido con bomba de calor, suelo radiante, fotovoltaica y almacenamiento para vivienda \par}
    \vspace{1.5cm}
    {\Large\bfseries Trabajo de Fin de Máster\par}
    \vspace{0.5cm}
    {\large Máster Universitario en Ingeniería de la Energía \par}
    \vspace{2cm}
    {\Large Luis D. Aranda Sánchez\par}
    \vfill
    Director: Javier Rodríguez Martín
    \vfill
    {\large Septiembre 6, 2024\par}
\end{titlepage}

% Resumen (máximo de 5 páginas, incluyendo al final Palabras clave)
\clearpage
\pagestyle{simple}
% \newpage
\chapter*{Resumen}
\addcontentsline{toc}{chapter}{Resumen}
\input{capitulos/resumen/main.tex}

% Índice (paginado)
\clearpage
\pagestyle{simple}
% \newpage
\tableofcontents

% Introducción (donde se incluya los antecedentes y justificación)
\clearpage
\pagestyle{myfancy}
\newpage
\chapter{Introducción}
\input{capitulos/introduccion/main.tex}

% Objetivos
\chapter{Objetivos}
\input{capitulos/objetivos/main.tex}

% Metodología
\chapter{Metodología}
\input{capitulos/metodologia/main.tex}

% Resultados y discusión (incluyendo la valoración de impactos y de aspectos de responsabilidad legal, ética y profesional relacionados con el trabajo)
\chapter{Resultados y Discusión}
\input{capitulos/resultados_discusion/main.tex}

% Conclusiones
\chapter{Conclusiones}
\input{capitulos/conclusiones/main.tex}

% Planificación temporal y presupuesto
\chapter{Planificación Temporal y Presupuesto}
\input{capitulos/planificacion_presupuesto/main.tex}

% Bibliografía
\newpage
\addcontentsline{toc}{chapter}{Bibliografía}
\printbibliography

\end{document}


% Objetivos
\chapter{Objetivos}
\documentclass[a4paper,11pt,twoside]{report}
\usepackage[left=25mm,right=25mm,top=25mm,bottom=25mm,includehead,includefoot,headsep=15mm,footskip=15mm]{geometry}
\usepackage{graphicx}
\usepackage{fancyhdr}
\usepackage{titlesec}
\usepackage[spanish]{babel}
\usepackage[utf8]{inputenc}
\usepackage{amsmath}
\usepackage{setspace}
\usepackage{svg}
\usepackage{hyperref}
\usepackage[backend=biber,style=numeric]{biblatex}
\addbibresource{references.bib}
\hypersetup{
    colorlinks=true,
    linkcolor=blue,      % color of internal links (sections, etc.)
    urlcolor=blue,       % color of external links
    pdftitle={Optimización energética de sistema híbrido con bomba de calor, suelo radiante, fotovoltaica y almacenamiento para vivienda},    % title
    pdfauthor={Luis D. Aranda Sánchez},     % author
    pdfkeywords={palabra1, palabra2, código1, etc.} % list of keywords
}

% Font change to Arial
\usepackage{helvet}
\renewcommand{\familydefault}{\sfdefault}

% Chapter titles in uppercase and larger font
\titleformat{\chapter}[hang]{\large\bfseries}{\thechapter.}{1em}{\MakeUppercase}
\titleformat{\section}[hang]{\bfseries}{\thesection.}{1em}{}
\titleformat{\subsection}[hang]{\bfseries}{\thesubsection.}{1em}{}

% Fancyhdr setup
\setlength{\headheight}{14.30174pt} % Adjust to recommended value, slightly larger for safety
\fancyhf{} % Clear all headers and footers
\fancyhead[LE]{\nouppercase{\leftmark}}
\fancyhead[RO]{Optimización energética para vivienda}
\fancyfoot[LE]{\thepage}
\fancyfoot[RE]{Escuela Técnica Superior de Ingenieros Industriales (UPM)}
\fancyfoot[LO]{Luis D. Aranda Sánchez}
\fancyfoot[RO]{\thepage}
\renewcommand{\headrulewidth}{0.4pt}
\renewcommand{\footrulewidth}{0.4pt}

\fancypagestyle{myfancy}{
    \fancyhf{} % Clear all headers and footers
    \fancyhead[LE]{\nouppercase{\leftmark}}
    \fancyhead[RO]{Optimización energética para vivienda}
    \fancyfoot[LE]{\thepage}
    \fancyfoot[RE]{Escuela Técnica Superior de Ingenieros Industriales (UPM)}
    \fancyfoot[LO]{Luis D. Aranda Sánchez}
    \fancyfoot[RO]{\thepage}
    \renewcommand{\headrulewidth}{0.4pt}
    \renewcommand{\footrulewidth}{0.4pt}
}

\fancypagestyle{simple}{
    \fancyhf{} % Clear all headers and footers
    \renewcommand{\headrulewidth}{0pt}
    \renewcommand{\footrulewidth}{0pt}
}

% Line spacing
\setstretch{1.2}

% Document starts here
\begin{document}

% Portada
\begin{titlepage}
    \centering
    {\scshape\LARGE Universidad Politécnica de Madrid \par}
    \vspace{1cm}
    {\scshape\Large Escuela Técnica Superior de Ingenieros Industriales\par}
    \vspace{1.5cm}
    {\huge\bfseries Optimización energética de sistema híbrido con bomba de calor, suelo radiante, fotovoltaica y almacenamiento para vivienda \par}
    \vspace{1.5cm}
    {\Large\bfseries Trabajo de Fin de Máster\par}
    \vspace{0.5cm}
    {\large Máster Universitario en Ingeniería de la Energía \par}
    \vspace{2cm}
    {\Large Luis D. Aranda Sánchez\par}
    \vfill
    Director: Javier Rodríguez Martín
    \vfill
    {\large Septiembre 6, 2024\par}
\end{titlepage}

% Resumen (máximo de 5 páginas, incluyendo al final Palabras clave)
\clearpage
\pagestyle{simple}
% \newpage
\chapter*{Resumen}
\addcontentsline{toc}{chapter}{Resumen}
\input{capitulos/resumen/main.tex}

% Índice (paginado)
\clearpage
\pagestyle{simple}
% \newpage
\tableofcontents

% Introducción (donde se incluya los antecedentes y justificación)
\clearpage
\pagestyle{myfancy}
\newpage
\chapter{Introducción}
\input{capitulos/introduccion/main.tex}

% Objetivos
\chapter{Objetivos}
\input{capitulos/objetivos/main.tex}

% Metodología
\chapter{Metodología}
\input{capitulos/metodologia/main.tex}

% Resultados y discusión (incluyendo la valoración de impactos y de aspectos de responsabilidad legal, ética y profesional relacionados con el trabajo)
\chapter{Resultados y Discusión}
\input{capitulos/resultados_discusion/main.tex}

% Conclusiones
\chapter{Conclusiones}
\input{capitulos/conclusiones/main.tex}

% Planificación temporal y presupuesto
\chapter{Planificación Temporal y Presupuesto}
\input{capitulos/planificacion_presupuesto/main.tex}

% Bibliografía
\newpage
\addcontentsline{toc}{chapter}{Bibliografía}
\printbibliography

\end{document}


% Metodología
\chapter{Metodología}
\documentclass[a4paper,11pt,twoside]{report}
\usepackage[left=25mm,right=25mm,top=25mm,bottom=25mm,includehead,includefoot,headsep=15mm,footskip=15mm]{geometry}
\usepackage{graphicx}
\usepackage{fancyhdr}
\usepackage{titlesec}
\usepackage[spanish]{babel}
\usepackage[utf8]{inputenc}
\usepackage{amsmath}
\usepackage{setspace}
\usepackage{svg}
\usepackage{hyperref}
\usepackage[backend=biber,style=numeric]{biblatex}
\addbibresource{references.bib}
\hypersetup{
    colorlinks=true,
    linkcolor=blue,      % color of internal links (sections, etc.)
    urlcolor=blue,       % color of external links
    pdftitle={Optimización energética de sistema híbrido con bomba de calor, suelo radiante, fotovoltaica y almacenamiento para vivienda},    % title
    pdfauthor={Luis D. Aranda Sánchez},     % author
    pdfkeywords={palabra1, palabra2, código1, etc.} % list of keywords
}

% Font change to Arial
\usepackage{helvet}
\renewcommand{\familydefault}{\sfdefault}

% Chapter titles in uppercase and larger font
\titleformat{\chapter}[hang]{\large\bfseries}{\thechapter.}{1em}{\MakeUppercase}
\titleformat{\section}[hang]{\bfseries}{\thesection.}{1em}{}
\titleformat{\subsection}[hang]{\bfseries}{\thesubsection.}{1em}{}

% Fancyhdr setup
\setlength{\headheight}{14.30174pt} % Adjust to recommended value, slightly larger for safety
\fancyhf{} % Clear all headers and footers
\fancyhead[LE]{\nouppercase{\leftmark}}
\fancyhead[RO]{Optimización energética para vivienda}
\fancyfoot[LE]{\thepage}
\fancyfoot[RE]{Escuela Técnica Superior de Ingenieros Industriales (UPM)}
\fancyfoot[LO]{Luis D. Aranda Sánchez}
\fancyfoot[RO]{\thepage}
\renewcommand{\headrulewidth}{0.4pt}
\renewcommand{\footrulewidth}{0.4pt}

\fancypagestyle{myfancy}{
    \fancyhf{} % Clear all headers and footers
    \fancyhead[LE]{\nouppercase{\leftmark}}
    \fancyhead[RO]{Optimización energética para vivienda}
    \fancyfoot[LE]{\thepage}
    \fancyfoot[RE]{Escuela Técnica Superior de Ingenieros Industriales (UPM)}
    \fancyfoot[LO]{Luis D. Aranda Sánchez}
    \fancyfoot[RO]{\thepage}
    \renewcommand{\headrulewidth}{0.4pt}
    \renewcommand{\footrulewidth}{0.4pt}
}

\fancypagestyle{simple}{
    \fancyhf{} % Clear all headers and footers
    \renewcommand{\headrulewidth}{0pt}
    \renewcommand{\footrulewidth}{0pt}
}

% Line spacing
\setstretch{1.2}

% Document starts here
\begin{document}

% Portada
\begin{titlepage}
    \centering
    {\scshape\LARGE Universidad Politécnica de Madrid \par}
    \vspace{1cm}
    {\scshape\Large Escuela Técnica Superior de Ingenieros Industriales\par}
    \vspace{1.5cm}
    {\huge\bfseries Optimización energética de sistema híbrido con bomba de calor, suelo radiante, fotovoltaica y almacenamiento para vivienda \par}
    \vspace{1.5cm}
    {\Large\bfseries Trabajo de Fin de Máster\par}
    \vspace{0.5cm}
    {\large Máster Universitario en Ingeniería de la Energía \par}
    \vspace{2cm}
    {\Large Luis D. Aranda Sánchez\par}
    \vfill
    Director: Javier Rodríguez Martín
    \vfill
    {\large Septiembre 6, 2024\par}
\end{titlepage}

% Resumen (máximo de 5 páginas, incluyendo al final Palabras clave)
\clearpage
\pagestyle{simple}
% \newpage
\chapter*{Resumen}
\addcontentsline{toc}{chapter}{Resumen}
\input{capitulos/resumen/main.tex}

% Índice (paginado)
\clearpage
\pagestyle{simple}
% \newpage
\tableofcontents

% Introducción (donde se incluya los antecedentes y justificación)
\clearpage
\pagestyle{myfancy}
\newpage
\chapter{Introducción}
\input{capitulos/introduccion/main.tex}

% Objetivos
\chapter{Objetivos}
\input{capitulos/objetivos/main.tex}

% Metodología
\chapter{Metodología}
\input{capitulos/metodologia/main.tex}

% Resultados y discusión (incluyendo la valoración de impactos y de aspectos de responsabilidad legal, ética y profesional relacionados con el trabajo)
\chapter{Resultados y Discusión}
\input{capitulos/resultados_discusion/main.tex}

% Conclusiones
\chapter{Conclusiones}
\input{capitulos/conclusiones/main.tex}

% Planificación temporal y presupuesto
\chapter{Planificación Temporal y Presupuesto}
\input{capitulos/planificacion_presupuesto/main.tex}

% Bibliografía
\newpage
\addcontentsline{toc}{chapter}{Bibliografía}
\printbibliography

\end{document}


% Resultados y discusión (incluyendo la valoración de impactos y de aspectos de responsabilidad legal, ética y profesional relacionados con el trabajo)
\chapter{Resultados y Discusión}
\documentclass[a4paper,11pt,twoside]{report}
\usepackage[left=25mm,right=25mm,top=25mm,bottom=25mm,includehead,includefoot,headsep=15mm,footskip=15mm]{geometry}
\usepackage{graphicx}
\usepackage{fancyhdr}
\usepackage{titlesec}
\usepackage[spanish]{babel}
\usepackage[utf8]{inputenc}
\usepackage{amsmath}
\usepackage{setspace}
\usepackage{svg}
\usepackage{hyperref}
\usepackage[backend=biber,style=numeric]{biblatex}
\addbibresource{references.bib}
\hypersetup{
    colorlinks=true,
    linkcolor=blue,      % color of internal links (sections, etc.)
    urlcolor=blue,       % color of external links
    pdftitle={Optimización energética de sistema híbrido con bomba de calor, suelo radiante, fotovoltaica y almacenamiento para vivienda},    % title
    pdfauthor={Luis D. Aranda Sánchez},     % author
    pdfkeywords={palabra1, palabra2, código1, etc.} % list of keywords
}

% Font change to Arial
\usepackage{helvet}
\renewcommand{\familydefault}{\sfdefault}

% Chapter titles in uppercase and larger font
\titleformat{\chapter}[hang]{\large\bfseries}{\thechapter.}{1em}{\MakeUppercase}
\titleformat{\section}[hang]{\bfseries}{\thesection.}{1em}{}
\titleformat{\subsection}[hang]{\bfseries}{\thesubsection.}{1em}{}

% Fancyhdr setup
\setlength{\headheight}{14.30174pt} % Adjust to recommended value, slightly larger for safety
\fancyhf{} % Clear all headers and footers
\fancyhead[LE]{\nouppercase{\leftmark}}
\fancyhead[RO]{Optimización energética para vivienda}
\fancyfoot[LE]{\thepage}
\fancyfoot[RE]{Escuela Técnica Superior de Ingenieros Industriales (UPM)}
\fancyfoot[LO]{Luis D. Aranda Sánchez}
\fancyfoot[RO]{\thepage}
\renewcommand{\headrulewidth}{0.4pt}
\renewcommand{\footrulewidth}{0.4pt}

\fancypagestyle{myfancy}{
    \fancyhf{} % Clear all headers and footers
    \fancyhead[LE]{\nouppercase{\leftmark}}
    \fancyhead[RO]{Optimización energética para vivienda}
    \fancyfoot[LE]{\thepage}
    \fancyfoot[RE]{Escuela Técnica Superior de Ingenieros Industriales (UPM)}
    \fancyfoot[LO]{Luis D. Aranda Sánchez}
    \fancyfoot[RO]{\thepage}
    \renewcommand{\headrulewidth}{0.4pt}
    \renewcommand{\footrulewidth}{0.4pt}
}

\fancypagestyle{simple}{
    \fancyhf{} % Clear all headers and footers
    \renewcommand{\headrulewidth}{0pt}
    \renewcommand{\footrulewidth}{0pt}
}

% Line spacing
\setstretch{1.2}

% Document starts here
\begin{document}

% Portada
\begin{titlepage}
    \centering
    {\scshape\LARGE Universidad Politécnica de Madrid \par}
    \vspace{1cm}
    {\scshape\Large Escuela Técnica Superior de Ingenieros Industriales\par}
    \vspace{1.5cm}
    {\huge\bfseries Optimización energética de sistema híbrido con bomba de calor, suelo radiante, fotovoltaica y almacenamiento para vivienda \par}
    \vspace{1.5cm}
    {\Large\bfseries Trabajo de Fin de Máster\par}
    \vspace{0.5cm}
    {\large Máster Universitario en Ingeniería de la Energía \par}
    \vspace{2cm}
    {\Large Luis D. Aranda Sánchez\par}
    \vfill
    Director: Javier Rodríguez Martín
    \vfill
    {\large Septiembre 6, 2024\par}
\end{titlepage}

% Resumen (máximo de 5 páginas, incluyendo al final Palabras clave)
\clearpage
\pagestyle{simple}
% \newpage
\chapter*{Resumen}
\addcontentsline{toc}{chapter}{Resumen}
\input{capitulos/resumen/main.tex}

% Índice (paginado)
\clearpage
\pagestyle{simple}
% \newpage
\tableofcontents

% Introducción (donde se incluya los antecedentes y justificación)
\clearpage
\pagestyle{myfancy}
\newpage
\chapter{Introducción}
\input{capitulos/introduccion/main.tex}

% Objetivos
\chapter{Objetivos}
\input{capitulos/objetivos/main.tex}

% Metodología
\chapter{Metodología}
\input{capitulos/metodologia/main.tex}

% Resultados y discusión (incluyendo la valoración de impactos y de aspectos de responsabilidad legal, ética y profesional relacionados con el trabajo)
\chapter{Resultados y Discusión}
\input{capitulos/resultados_discusion/main.tex}

% Conclusiones
\chapter{Conclusiones}
\input{capitulos/conclusiones/main.tex}

% Planificación temporal y presupuesto
\chapter{Planificación Temporal y Presupuesto}
\input{capitulos/planificacion_presupuesto/main.tex}

% Bibliografía
\newpage
\addcontentsline{toc}{chapter}{Bibliografía}
\printbibliography

\end{document}


% Conclusiones
\chapter{Conclusiones}
\documentclass[a4paper,11pt,twoside]{report}
\usepackage[left=25mm,right=25mm,top=25mm,bottom=25mm,includehead,includefoot,headsep=15mm,footskip=15mm]{geometry}
\usepackage{graphicx}
\usepackage{fancyhdr}
\usepackage{titlesec}
\usepackage[spanish]{babel}
\usepackage[utf8]{inputenc}
\usepackage{amsmath}
\usepackage{setspace}
\usepackage{svg}
\usepackage{hyperref}
\usepackage[backend=biber,style=numeric]{biblatex}
\addbibresource{references.bib}
\hypersetup{
    colorlinks=true,
    linkcolor=blue,      % color of internal links (sections, etc.)
    urlcolor=blue,       % color of external links
    pdftitle={Optimización energética de sistema híbrido con bomba de calor, suelo radiante, fotovoltaica y almacenamiento para vivienda},    % title
    pdfauthor={Luis D. Aranda Sánchez},     % author
    pdfkeywords={palabra1, palabra2, código1, etc.} % list of keywords
}

% Font change to Arial
\usepackage{helvet}
\renewcommand{\familydefault}{\sfdefault}

% Chapter titles in uppercase and larger font
\titleformat{\chapter}[hang]{\large\bfseries}{\thechapter.}{1em}{\MakeUppercase}
\titleformat{\section}[hang]{\bfseries}{\thesection.}{1em}{}
\titleformat{\subsection}[hang]{\bfseries}{\thesubsection.}{1em}{}

% Fancyhdr setup
\setlength{\headheight}{14.30174pt} % Adjust to recommended value, slightly larger for safety
\fancyhf{} % Clear all headers and footers
\fancyhead[LE]{\nouppercase{\leftmark}}
\fancyhead[RO]{Optimización energética para vivienda}
\fancyfoot[LE]{\thepage}
\fancyfoot[RE]{Escuela Técnica Superior de Ingenieros Industriales (UPM)}
\fancyfoot[LO]{Luis D. Aranda Sánchez}
\fancyfoot[RO]{\thepage}
\renewcommand{\headrulewidth}{0.4pt}
\renewcommand{\footrulewidth}{0.4pt}

\fancypagestyle{myfancy}{
    \fancyhf{} % Clear all headers and footers
    \fancyhead[LE]{\nouppercase{\leftmark}}
    \fancyhead[RO]{Optimización energética para vivienda}
    \fancyfoot[LE]{\thepage}
    \fancyfoot[RE]{Escuela Técnica Superior de Ingenieros Industriales (UPM)}
    \fancyfoot[LO]{Luis D. Aranda Sánchez}
    \fancyfoot[RO]{\thepage}
    \renewcommand{\headrulewidth}{0.4pt}
    \renewcommand{\footrulewidth}{0.4pt}
}

\fancypagestyle{simple}{
    \fancyhf{} % Clear all headers and footers
    \renewcommand{\headrulewidth}{0pt}
    \renewcommand{\footrulewidth}{0pt}
}

% Line spacing
\setstretch{1.2}

% Document starts here
\begin{document}

% Portada
\begin{titlepage}
    \centering
    {\scshape\LARGE Universidad Politécnica de Madrid \par}
    \vspace{1cm}
    {\scshape\Large Escuela Técnica Superior de Ingenieros Industriales\par}
    \vspace{1.5cm}
    {\huge\bfseries Optimización energética de sistema híbrido con bomba de calor, suelo radiante, fotovoltaica y almacenamiento para vivienda \par}
    \vspace{1.5cm}
    {\Large\bfseries Trabajo de Fin de Máster\par}
    \vspace{0.5cm}
    {\large Máster Universitario en Ingeniería de la Energía \par}
    \vspace{2cm}
    {\Large Luis D. Aranda Sánchez\par}
    \vfill
    Director: Javier Rodríguez Martín
    \vfill
    {\large Septiembre 6, 2024\par}
\end{titlepage}

% Resumen (máximo de 5 páginas, incluyendo al final Palabras clave)
\clearpage
\pagestyle{simple}
% \newpage
\chapter*{Resumen}
\addcontentsline{toc}{chapter}{Resumen}
\input{capitulos/resumen/main.tex}

% Índice (paginado)
\clearpage
\pagestyle{simple}
% \newpage
\tableofcontents

% Introducción (donde se incluya los antecedentes y justificación)
\clearpage
\pagestyle{myfancy}
\newpage
\chapter{Introducción}
\input{capitulos/introduccion/main.tex}

% Objetivos
\chapter{Objetivos}
\input{capitulos/objetivos/main.tex}

% Metodología
\chapter{Metodología}
\input{capitulos/metodologia/main.tex}

% Resultados y discusión (incluyendo la valoración de impactos y de aspectos de responsabilidad legal, ética y profesional relacionados con el trabajo)
\chapter{Resultados y Discusión}
\input{capitulos/resultados_discusion/main.tex}

% Conclusiones
\chapter{Conclusiones}
\input{capitulos/conclusiones/main.tex}

% Planificación temporal y presupuesto
\chapter{Planificación Temporal y Presupuesto}
\input{capitulos/planificacion_presupuesto/main.tex}

% Bibliografía
\newpage
\addcontentsline{toc}{chapter}{Bibliografía}
\printbibliography

\end{document}


% Planificación temporal y presupuesto
\chapter{Planificación Temporal y Presupuesto}
\documentclass[a4paper,11pt,twoside]{report}
\usepackage[left=25mm,right=25mm,top=25mm,bottom=25mm,includehead,includefoot,headsep=15mm,footskip=15mm]{geometry}
\usepackage{graphicx}
\usepackage{fancyhdr}
\usepackage{titlesec}
\usepackage[spanish]{babel}
\usepackage[utf8]{inputenc}
\usepackage{amsmath}
\usepackage{setspace}
\usepackage{svg}
\usepackage{hyperref}
\usepackage[backend=biber,style=numeric]{biblatex}
\addbibresource{references.bib}
\hypersetup{
    colorlinks=true,
    linkcolor=blue,      % color of internal links (sections, etc.)
    urlcolor=blue,       % color of external links
    pdftitle={Optimización energética de sistema híbrido con bomba de calor, suelo radiante, fotovoltaica y almacenamiento para vivienda},    % title
    pdfauthor={Luis D. Aranda Sánchez},     % author
    pdfkeywords={palabra1, palabra2, código1, etc.} % list of keywords
}

% Font change to Arial
\usepackage{helvet}
\renewcommand{\familydefault}{\sfdefault}

% Chapter titles in uppercase and larger font
\titleformat{\chapter}[hang]{\large\bfseries}{\thechapter.}{1em}{\MakeUppercase}
\titleformat{\section}[hang]{\bfseries}{\thesection.}{1em}{}
\titleformat{\subsection}[hang]{\bfseries}{\thesubsection.}{1em}{}

% Fancyhdr setup
\setlength{\headheight}{14.30174pt} % Adjust to recommended value, slightly larger for safety
\fancyhf{} % Clear all headers and footers
\fancyhead[LE]{\nouppercase{\leftmark}}
\fancyhead[RO]{Optimización energética para vivienda}
\fancyfoot[LE]{\thepage}
\fancyfoot[RE]{Escuela Técnica Superior de Ingenieros Industriales (UPM)}
\fancyfoot[LO]{Luis D. Aranda Sánchez}
\fancyfoot[RO]{\thepage}
\renewcommand{\headrulewidth}{0.4pt}
\renewcommand{\footrulewidth}{0.4pt}

\fancypagestyle{myfancy}{
    \fancyhf{} % Clear all headers and footers
    \fancyhead[LE]{\nouppercase{\leftmark}}
    \fancyhead[RO]{Optimización energética para vivienda}
    \fancyfoot[LE]{\thepage}
    \fancyfoot[RE]{Escuela Técnica Superior de Ingenieros Industriales (UPM)}
    \fancyfoot[LO]{Luis D. Aranda Sánchez}
    \fancyfoot[RO]{\thepage}
    \renewcommand{\headrulewidth}{0.4pt}
    \renewcommand{\footrulewidth}{0.4pt}
}

\fancypagestyle{simple}{
    \fancyhf{} % Clear all headers and footers
    \renewcommand{\headrulewidth}{0pt}
    \renewcommand{\footrulewidth}{0pt}
}

% Line spacing
\setstretch{1.2}

% Document starts here
\begin{document}

% Portada
\begin{titlepage}
    \centering
    {\scshape\LARGE Universidad Politécnica de Madrid \par}
    \vspace{1cm}
    {\scshape\Large Escuela Técnica Superior de Ingenieros Industriales\par}
    \vspace{1.5cm}
    {\huge\bfseries Optimización energética de sistema híbrido con bomba de calor, suelo radiante, fotovoltaica y almacenamiento para vivienda \par}
    \vspace{1.5cm}
    {\Large\bfseries Trabajo de Fin de Máster\par}
    \vspace{0.5cm}
    {\large Máster Universitario en Ingeniería de la Energía \par}
    \vspace{2cm}
    {\Large Luis D. Aranda Sánchez\par}
    \vfill
    Director: Javier Rodríguez Martín
    \vfill
    {\large Septiembre 6, 2024\par}
\end{titlepage}

% Resumen (máximo de 5 páginas, incluyendo al final Palabras clave)
\clearpage
\pagestyle{simple}
% \newpage
\chapter*{Resumen}
\addcontentsline{toc}{chapter}{Resumen}
\input{capitulos/resumen/main.tex}

% Índice (paginado)
\clearpage
\pagestyle{simple}
% \newpage
\tableofcontents

% Introducción (donde se incluya los antecedentes y justificación)
\clearpage
\pagestyle{myfancy}
\newpage
\chapter{Introducción}
\input{capitulos/introduccion/main.tex}

% Objetivos
\chapter{Objetivos}
\input{capitulos/objetivos/main.tex}

% Metodología
\chapter{Metodología}
\input{capitulos/metodologia/main.tex}

% Resultados y discusión (incluyendo la valoración de impactos y de aspectos de responsabilidad legal, ética y profesional relacionados con el trabajo)
\chapter{Resultados y Discusión}
\input{capitulos/resultados_discusion/main.tex}

% Conclusiones
\chapter{Conclusiones}
\input{capitulos/conclusiones/main.tex}

% Planificación temporal y presupuesto
\chapter{Planificación Temporal y Presupuesto}
\input{capitulos/planificacion_presupuesto/main.tex}

% Bibliografía
\newpage
\addcontentsline{toc}{chapter}{Bibliografía}
\printbibliography

\end{document}


% Bibliografía
\newpage
\addcontentsline{toc}{chapter}{Bibliografía}
\printbibliography

\end{document}


% Objetivos
\chapter{Objetivos}
\documentclass[a4paper,11pt,twoside]{report}
\usepackage[left=25mm,right=25mm,top=25mm,bottom=25mm,includehead,includefoot,headsep=15mm,footskip=15mm]{geometry}
\usepackage{graphicx}
\usepackage{fancyhdr}
\usepackage{titlesec}
\usepackage[spanish]{babel}
\usepackage[utf8]{inputenc}
\usepackage{amsmath}
\usepackage{setspace}
\usepackage{svg}
\usepackage{hyperref}
\usepackage[backend=biber,style=numeric]{biblatex}
\addbibresource{references.bib}
\hypersetup{
    colorlinks=true,
    linkcolor=blue,      % color of internal links (sections, etc.)
    urlcolor=blue,       % color of external links
    pdftitle={Optimización energética de sistema híbrido con bomba de calor, suelo radiante, fotovoltaica y almacenamiento para vivienda},    % title
    pdfauthor={Luis D. Aranda Sánchez},     % author
    pdfkeywords={palabra1, palabra2, código1, etc.} % list of keywords
}

% Font change to Arial
\usepackage{helvet}
\renewcommand{\familydefault}{\sfdefault}

% Chapter titles in uppercase and larger font
\titleformat{\chapter}[hang]{\large\bfseries}{\thechapter.}{1em}{\MakeUppercase}
\titleformat{\section}[hang]{\bfseries}{\thesection.}{1em}{}
\titleformat{\subsection}[hang]{\bfseries}{\thesubsection.}{1em}{}

% Fancyhdr setup
\setlength{\headheight}{14.30174pt} % Adjust to recommended value, slightly larger for safety
\fancyhf{} % Clear all headers and footers
\fancyhead[LE]{\nouppercase{\leftmark}}
\fancyhead[RO]{Optimización energética para vivienda}
\fancyfoot[LE]{\thepage}
\fancyfoot[RE]{Escuela Técnica Superior de Ingenieros Industriales (UPM)}
\fancyfoot[LO]{Luis D. Aranda Sánchez}
\fancyfoot[RO]{\thepage}
\renewcommand{\headrulewidth}{0.4pt}
\renewcommand{\footrulewidth}{0.4pt}

\fancypagestyle{myfancy}{
    \fancyhf{} % Clear all headers and footers
    \fancyhead[LE]{\nouppercase{\leftmark}}
    \fancyhead[RO]{Optimización energética para vivienda}
    \fancyfoot[LE]{\thepage}
    \fancyfoot[RE]{Escuela Técnica Superior de Ingenieros Industriales (UPM)}
    \fancyfoot[LO]{Luis D. Aranda Sánchez}
    \fancyfoot[RO]{\thepage}
    \renewcommand{\headrulewidth}{0.4pt}
    \renewcommand{\footrulewidth}{0.4pt}
}

\fancypagestyle{simple}{
    \fancyhf{} % Clear all headers and footers
    \renewcommand{\headrulewidth}{0pt}
    \renewcommand{\footrulewidth}{0pt}
}

% Line spacing
\setstretch{1.2}

% Document starts here
\begin{document}

% Portada
\begin{titlepage}
    \centering
    {\scshape\LARGE Universidad Politécnica de Madrid \par}
    \vspace{1cm}
    {\scshape\Large Escuela Técnica Superior de Ingenieros Industriales\par}
    \vspace{1.5cm}
    {\huge\bfseries Optimización energética de sistema híbrido con bomba de calor, suelo radiante, fotovoltaica y almacenamiento para vivienda \par}
    \vspace{1.5cm}
    {\Large\bfseries Trabajo de Fin de Máster\par}
    \vspace{0.5cm}
    {\large Máster Universitario en Ingeniería de la Energía \par}
    \vspace{2cm}
    {\Large Luis D. Aranda Sánchez\par}
    \vfill
    Director: Javier Rodríguez Martín
    \vfill
    {\large Septiembre 6, 2024\par}
\end{titlepage}

% Resumen (máximo de 5 páginas, incluyendo al final Palabras clave)
\clearpage
\pagestyle{simple}
% \newpage
\chapter*{Resumen}
\addcontentsline{toc}{chapter}{Resumen}
\documentclass[a4paper,11pt,twoside]{report}
\usepackage[left=25mm,right=25mm,top=25mm,bottom=25mm,includehead,includefoot,headsep=15mm,footskip=15mm]{geometry}
\usepackage{graphicx}
\usepackage{fancyhdr}
\usepackage{titlesec}
\usepackage[spanish]{babel}
\usepackage[utf8]{inputenc}
\usepackage{amsmath}
\usepackage{setspace}
\usepackage{svg}
\usepackage{hyperref}
\usepackage[backend=biber,style=numeric]{biblatex}
\addbibresource{references.bib}
\hypersetup{
    colorlinks=true,
    linkcolor=blue,      % color of internal links (sections, etc.)
    urlcolor=blue,       % color of external links
    pdftitle={Optimización energética de sistema híbrido con bomba de calor, suelo radiante, fotovoltaica y almacenamiento para vivienda},    % title
    pdfauthor={Luis D. Aranda Sánchez},     % author
    pdfkeywords={palabra1, palabra2, código1, etc.} % list of keywords
}

% Font change to Arial
\usepackage{helvet}
\renewcommand{\familydefault}{\sfdefault}

% Chapter titles in uppercase and larger font
\titleformat{\chapter}[hang]{\large\bfseries}{\thechapter.}{1em}{\MakeUppercase}
\titleformat{\section}[hang]{\bfseries}{\thesection.}{1em}{}
\titleformat{\subsection}[hang]{\bfseries}{\thesubsection.}{1em}{}

% Fancyhdr setup
\setlength{\headheight}{14.30174pt} % Adjust to recommended value, slightly larger for safety
\fancyhf{} % Clear all headers and footers
\fancyhead[LE]{\nouppercase{\leftmark}}
\fancyhead[RO]{Optimización energética para vivienda}
\fancyfoot[LE]{\thepage}
\fancyfoot[RE]{Escuela Técnica Superior de Ingenieros Industriales (UPM)}
\fancyfoot[LO]{Luis D. Aranda Sánchez}
\fancyfoot[RO]{\thepage}
\renewcommand{\headrulewidth}{0.4pt}
\renewcommand{\footrulewidth}{0.4pt}

\fancypagestyle{myfancy}{
    \fancyhf{} % Clear all headers and footers
    \fancyhead[LE]{\nouppercase{\leftmark}}
    \fancyhead[RO]{Optimización energética para vivienda}
    \fancyfoot[LE]{\thepage}
    \fancyfoot[RE]{Escuela Técnica Superior de Ingenieros Industriales (UPM)}
    \fancyfoot[LO]{Luis D. Aranda Sánchez}
    \fancyfoot[RO]{\thepage}
    \renewcommand{\headrulewidth}{0.4pt}
    \renewcommand{\footrulewidth}{0.4pt}
}

\fancypagestyle{simple}{
    \fancyhf{} % Clear all headers and footers
    \renewcommand{\headrulewidth}{0pt}
    \renewcommand{\footrulewidth}{0pt}
}

% Line spacing
\setstretch{1.2}

% Document starts here
\begin{document}

% Portada
\begin{titlepage}
    \centering
    {\scshape\LARGE Universidad Politécnica de Madrid \par}
    \vspace{1cm}
    {\scshape\Large Escuela Técnica Superior de Ingenieros Industriales\par}
    \vspace{1.5cm}
    {\huge\bfseries Optimización energética de sistema híbrido con bomba de calor, suelo radiante, fotovoltaica y almacenamiento para vivienda \par}
    \vspace{1.5cm}
    {\Large\bfseries Trabajo de Fin de Máster\par}
    \vspace{0.5cm}
    {\large Máster Universitario en Ingeniería de la Energía \par}
    \vspace{2cm}
    {\Large Luis D. Aranda Sánchez\par}
    \vfill
    Director: Javier Rodríguez Martín
    \vfill
    {\large Septiembre 6, 2024\par}
\end{titlepage}

% Resumen (máximo de 5 páginas, incluyendo al final Palabras clave)
\clearpage
\pagestyle{simple}
% \newpage
\chapter*{Resumen}
\addcontentsline{toc}{chapter}{Resumen}
\input{capitulos/resumen/main.tex}

% Índice (paginado)
\clearpage
\pagestyle{simple}
% \newpage
\tableofcontents

% Introducción (donde se incluya los antecedentes y justificación)
\clearpage
\pagestyle{myfancy}
\newpage
\chapter{Introducción}
\input{capitulos/introduccion/main.tex}

% Objetivos
\chapter{Objetivos}
\input{capitulos/objetivos/main.tex}

% Metodología
\chapter{Metodología}
\input{capitulos/metodologia/main.tex}

% Resultados y discusión (incluyendo la valoración de impactos y de aspectos de responsabilidad legal, ética y profesional relacionados con el trabajo)
\chapter{Resultados y Discusión}
\input{capitulos/resultados_discusion/main.tex}

% Conclusiones
\chapter{Conclusiones}
\input{capitulos/conclusiones/main.tex}

% Planificación temporal y presupuesto
\chapter{Planificación Temporal y Presupuesto}
\input{capitulos/planificacion_presupuesto/main.tex}

% Bibliografía
\newpage
\addcontentsline{toc}{chapter}{Bibliografía}
\printbibliography

\end{document}


% Índice (paginado)
\clearpage
\pagestyle{simple}
% \newpage
\tableofcontents

% Introducción (donde se incluya los antecedentes y justificación)
\clearpage
\pagestyle{myfancy}
\newpage
\chapter{Introducción}
\documentclass[a4paper,11pt,twoside]{report}
\usepackage[left=25mm,right=25mm,top=25mm,bottom=25mm,includehead,includefoot,headsep=15mm,footskip=15mm]{geometry}
\usepackage{graphicx}
\usepackage{fancyhdr}
\usepackage{titlesec}
\usepackage[spanish]{babel}
\usepackage[utf8]{inputenc}
\usepackage{amsmath}
\usepackage{setspace}
\usepackage{svg}
\usepackage{hyperref}
\usepackage[backend=biber,style=numeric]{biblatex}
\addbibresource{references.bib}
\hypersetup{
    colorlinks=true,
    linkcolor=blue,      % color of internal links (sections, etc.)
    urlcolor=blue,       % color of external links
    pdftitle={Optimización energética de sistema híbrido con bomba de calor, suelo radiante, fotovoltaica y almacenamiento para vivienda},    % title
    pdfauthor={Luis D. Aranda Sánchez},     % author
    pdfkeywords={palabra1, palabra2, código1, etc.} % list of keywords
}

% Font change to Arial
\usepackage{helvet}
\renewcommand{\familydefault}{\sfdefault}

% Chapter titles in uppercase and larger font
\titleformat{\chapter}[hang]{\large\bfseries}{\thechapter.}{1em}{\MakeUppercase}
\titleformat{\section}[hang]{\bfseries}{\thesection.}{1em}{}
\titleformat{\subsection}[hang]{\bfseries}{\thesubsection.}{1em}{}

% Fancyhdr setup
\setlength{\headheight}{14.30174pt} % Adjust to recommended value, slightly larger for safety
\fancyhf{} % Clear all headers and footers
\fancyhead[LE]{\nouppercase{\leftmark}}
\fancyhead[RO]{Optimización energética para vivienda}
\fancyfoot[LE]{\thepage}
\fancyfoot[RE]{Escuela Técnica Superior de Ingenieros Industriales (UPM)}
\fancyfoot[LO]{Luis D. Aranda Sánchez}
\fancyfoot[RO]{\thepage}
\renewcommand{\headrulewidth}{0.4pt}
\renewcommand{\footrulewidth}{0.4pt}

\fancypagestyle{myfancy}{
    \fancyhf{} % Clear all headers and footers
    \fancyhead[LE]{\nouppercase{\leftmark}}
    \fancyhead[RO]{Optimización energética para vivienda}
    \fancyfoot[LE]{\thepage}
    \fancyfoot[RE]{Escuela Técnica Superior de Ingenieros Industriales (UPM)}
    \fancyfoot[LO]{Luis D. Aranda Sánchez}
    \fancyfoot[RO]{\thepage}
    \renewcommand{\headrulewidth}{0.4pt}
    \renewcommand{\footrulewidth}{0.4pt}
}

\fancypagestyle{simple}{
    \fancyhf{} % Clear all headers and footers
    \renewcommand{\headrulewidth}{0pt}
    \renewcommand{\footrulewidth}{0pt}
}

% Line spacing
\setstretch{1.2}

% Document starts here
\begin{document}

% Portada
\begin{titlepage}
    \centering
    {\scshape\LARGE Universidad Politécnica de Madrid \par}
    \vspace{1cm}
    {\scshape\Large Escuela Técnica Superior de Ingenieros Industriales\par}
    \vspace{1.5cm}
    {\huge\bfseries Optimización energética de sistema híbrido con bomba de calor, suelo radiante, fotovoltaica y almacenamiento para vivienda \par}
    \vspace{1.5cm}
    {\Large\bfseries Trabajo de Fin de Máster\par}
    \vspace{0.5cm}
    {\large Máster Universitario en Ingeniería de la Energía \par}
    \vspace{2cm}
    {\Large Luis D. Aranda Sánchez\par}
    \vfill
    Director: Javier Rodríguez Martín
    \vfill
    {\large Septiembre 6, 2024\par}
\end{titlepage}

% Resumen (máximo de 5 páginas, incluyendo al final Palabras clave)
\clearpage
\pagestyle{simple}
% \newpage
\chapter*{Resumen}
\addcontentsline{toc}{chapter}{Resumen}
\input{capitulos/resumen/main.tex}

% Índice (paginado)
\clearpage
\pagestyle{simple}
% \newpage
\tableofcontents

% Introducción (donde se incluya los antecedentes y justificación)
\clearpage
\pagestyle{myfancy}
\newpage
\chapter{Introducción}
\input{capitulos/introduccion/main.tex}

% Objetivos
\chapter{Objetivos}
\input{capitulos/objetivos/main.tex}

% Metodología
\chapter{Metodología}
\input{capitulos/metodologia/main.tex}

% Resultados y discusión (incluyendo la valoración de impactos y de aspectos de responsabilidad legal, ética y profesional relacionados con el trabajo)
\chapter{Resultados y Discusión}
\input{capitulos/resultados_discusion/main.tex}

% Conclusiones
\chapter{Conclusiones}
\input{capitulos/conclusiones/main.tex}

% Planificación temporal y presupuesto
\chapter{Planificación Temporal y Presupuesto}
\input{capitulos/planificacion_presupuesto/main.tex}

% Bibliografía
\newpage
\addcontentsline{toc}{chapter}{Bibliografía}
\printbibliography

\end{document}


% Objetivos
\chapter{Objetivos}
\documentclass[a4paper,11pt,twoside]{report}
\usepackage[left=25mm,right=25mm,top=25mm,bottom=25mm,includehead,includefoot,headsep=15mm,footskip=15mm]{geometry}
\usepackage{graphicx}
\usepackage{fancyhdr}
\usepackage{titlesec}
\usepackage[spanish]{babel}
\usepackage[utf8]{inputenc}
\usepackage{amsmath}
\usepackage{setspace}
\usepackage{svg}
\usepackage{hyperref}
\usepackage[backend=biber,style=numeric]{biblatex}
\addbibresource{references.bib}
\hypersetup{
    colorlinks=true,
    linkcolor=blue,      % color of internal links (sections, etc.)
    urlcolor=blue,       % color of external links
    pdftitle={Optimización energética de sistema híbrido con bomba de calor, suelo radiante, fotovoltaica y almacenamiento para vivienda},    % title
    pdfauthor={Luis D. Aranda Sánchez},     % author
    pdfkeywords={palabra1, palabra2, código1, etc.} % list of keywords
}

% Font change to Arial
\usepackage{helvet}
\renewcommand{\familydefault}{\sfdefault}

% Chapter titles in uppercase and larger font
\titleformat{\chapter}[hang]{\large\bfseries}{\thechapter.}{1em}{\MakeUppercase}
\titleformat{\section}[hang]{\bfseries}{\thesection.}{1em}{}
\titleformat{\subsection}[hang]{\bfseries}{\thesubsection.}{1em}{}

% Fancyhdr setup
\setlength{\headheight}{14.30174pt} % Adjust to recommended value, slightly larger for safety
\fancyhf{} % Clear all headers and footers
\fancyhead[LE]{\nouppercase{\leftmark}}
\fancyhead[RO]{Optimización energética para vivienda}
\fancyfoot[LE]{\thepage}
\fancyfoot[RE]{Escuela Técnica Superior de Ingenieros Industriales (UPM)}
\fancyfoot[LO]{Luis D. Aranda Sánchez}
\fancyfoot[RO]{\thepage}
\renewcommand{\headrulewidth}{0.4pt}
\renewcommand{\footrulewidth}{0.4pt}

\fancypagestyle{myfancy}{
    \fancyhf{} % Clear all headers and footers
    \fancyhead[LE]{\nouppercase{\leftmark}}
    \fancyhead[RO]{Optimización energética para vivienda}
    \fancyfoot[LE]{\thepage}
    \fancyfoot[RE]{Escuela Técnica Superior de Ingenieros Industriales (UPM)}
    \fancyfoot[LO]{Luis D. Aranda Sánchez}
    \fancyfoot[RO]{\thepage}
    \renewcommand{\headrulewidth}{0.4pt}
    \renewcommand{\footrulewidth}{0.4pt}
}

\fancypagestyle{simple}{
    \fancyhf{} % Clear all headers and footers
    \renewcommand{\headrulewidth}{0pt}
    \renewcommand{\footrulewidth}{0pt}
}

% Line spacing
\setstretch{1.2}

% Document starts here
\begin{document}

% Portada
\begin{titlepage}
    \centering
    {\scshape\LARGE Universidad Politécnica de Madrid \par}
    \vspace{1cm}
    {\scshape\Large Escuela Técnica Superior de Ingenieros Industriales\par}
    \vspace{1.5cm}
    {\huge\bfseries Optimización energética de sistema híbrido con bomba de calor, suelo radiante, fotovoltaica y almacenamiento para vivienda \par}
    \vspace{1.5cm}
    {\Large\bfseries Trabajo de Fin de Máster\par}
    \vspace{0.5cm}
    {\large Máster Universitario en Ingeniería de la Energía \par}
    \vspace{2cm}
    {\Large Luis D. Aranda Sánchez\par}
    \vfill
    Director: Javier Rodríguez Martín
    \vfill
    {\large Septiembre 6, 2024\par}
\end{titlepage}

% Resumen (máximo de 5 páginas, incluyendo al final Palabras clave)
\clearpage
\pagestyle{simple}
% \newpage
\chapter*{Resumen}
\addcontentsline{toc}{chapter}{Resumen}
\input{capitulos/resumen/main.tex}

% Índice (paginado)
\clearpage
\pagestyle{simple}
% \newpage
\tableofcontents

% Introducción (donde se incluya los antecedentes y justificación)
\clearpage
\pagestyle{myfancy}
\newpage
\chapter{Introducción}
\input{capitulos/introduccion/main.tex}

% Objetivos
\chapter{Objetivos}
\input{capitulos/objetivos/main.tex}

% Metodología
\chapter{Metodología}
\input{capitulos/metodologia/main.tex}

% Resultados y discusión (incluyendo la valoración de impactos y de aspectos de responsabilidad legal, ética y profesional relacionados con el trabajo)
\chapter{Resultados y Discusión}
\input{capitulos/resultados_discusion/main.tex}

% Conclusiones
\chapter{Conclusiones}
\input{capitulos/conclusiones/main.tex}

% Planificación temporal y presupuesto
\chapter{Planificación Temporal y Presupuesto}
\input{capitulos/planificacion_presupuesto/main.tex}

% Bibliografía
\newpage
\addcontentsline{toc}{chapter}{Bibliografía}
\printbibliography

\end{document}


% Metodología
\chapter{Metodología}
\documentclass[a4paper,11pt,twoside]{report}
\usepackage[left=25mm,right=25mm,top=25mm,bottom=25mm,includehead,includefoot,headsep=15mm,footskip=15mm]{geometry}
\usepackage{graphicx}
\usepackage{fancyhdr}
\usepackage{titlesec}
\usepackage[spanish]{babel}
\usepackage[utf8]{inputenc}
\usepackage{amsmath}
\usepackage{setspace}
\usepackage{svg}
\usepackage{hyperref}
\usepackage[backend=biber,style=numeric]{biblatex}
\addbibresource{references.bib}
\hypersetup{
    colorlinks=true,
    linkcolor=blue,      % color of internal links (sections, etc.)
    urlcolor=blue,       % color of external links
    pdftitle={Optimización energética de sistema híbrido con bomba de calor, suelo radiante, fotovoltaica y almacenamiento para vivienda},    % title
    pdfauthor={Luis D. Aranda Sánchez},     % author
    pdfkeywords={palabra1, palabra2, código1, etc.} % list of keywords
}

% Font change to Arial
\usepackage{helvet}
\renewcommand{\familydefault}{\sfdefault}

% Chapter titles in uppercase and larger font
\titleformat{\chapter}[hang]{\large\bfseries}{\thechapter.}{1em}{\MakeUppercase}
\titleformat{\section}[hang]{\bfseries}{\thesection.}{1em}{}
\titleformat{\subsection}[hang]{\bfseries}{\thesubsection.}{1em}{}

% Fancyhdr setup
\setlength{\headheight}{14.30174pt} % Adjust to recommended value, slightly larger for safety
\fancyhf{} % Clear all headers and footers
\fancyhead[LE]{\nouppercase{\leftmark}}
\fancyhead[RO]{Optimización energética para vivienda}
\fancyfoot[LE]{\thepage}
\fancyfoot[RE]{Escuela Técnica Superior de Ingenieros Industriales (UPM)}
\fancyfoot[LO]{Luis D. Aranda Sánchez}
\fancyfoot[RO]{\thepage}
\renewcommand{\headrulewidth}{0.4pt}
\renewcommand{\footrulewidth}{0.4pt}

\fancypagestyle{myfancy}{
    \fancyhf{} % Clear all headers and footers
    \fancyhead[LE]{\nouppercase{\leftmark}}
    \fancyhead[RO]{Optimización energética para vivienda}
    \fancyfoot[LE]{\thepage}
    \fancyfoot[RE]{Escuela Técnica Superior de Ingenieros Industriales (UPM)}
    \fancyfoot[LO]{Luis D. Aranda Sánchez}
    \fancyfoot[RO]{\thepage}
    \renewcommand{\headrulewidth}{0.4pt}
    \renewcommand{\footrulewidth}{0.4pt}
}

\fancypagestyle{simple}{
    \fancyhf{} % Clear all headers and footers
    \renewcommand{\headrulewidth}{0pt}
    \renewcommand{\footrulewidth}{0pt}
}

% Line spacing
\setstretch{1.2}

% Document starts here
\begin{document}

% Portada
\begin{titlepage}
    \centering
    {\scshape\LARGE Universidad Politécnica de Madrid \par}
    \vspace{1cm}
    {\scshape\Large Escuela Técnica Superior de Ingenieros Industriales\par}
    \vspace{1.5cm}
    {\huge\bfseries Optimización energética de sistema híbrido con bomba de calor, suelo radiante, fotovoltaica y almacenamiento para vivienda \par}
    \vspace{1.5cm}
    {\Large\bfseries Trabajo de Fin de Máster\par}
    \vspace{0.5cm}
    {\large Máster Universitario en Ingeniería de la Energía \par}
    \vspace{2cm}
    {\Large Luis D. Aranda Sánchez\par}
    \vfill
    Director: Javier Rodríguez Martín
    \vfill
    {\large Septiembre 6, 2024\par}
\end{titlepage}

% Resumen (máximo de 5 páginas, incluyendo al final Palabras clave)
\clearpage
\pagestyle{simple}
% \newpage
\chapter*{Resumen}
\addcontentsline{toc}{chapter}{Resumen}
\input{capitulos/resumen/main.tex}

% Índice (paginado)
\clearpage
\pagestyle{simple}
% \newpage
\tableofcontents

% Introducción (donde se incluya los antecedentes y justificación)
\clearpage
\pagestyle{myfancy}
\newpage
\chapter{Introducción}
\input{capitulos/introduccion/main.tex}

% Objetivos
\chapter{Objetivos}
\input{capitulos/objetivos/main.tex}

% Metodología
\chapter{Metodología}
\input{capitulos/metodologia/main.tex}

% Resultados y discusión (incluyendo la valoración de impactos y de aspectos de responsabilidad legal, ética y profesional relacionados con el trabajo)
\chapter{Resultados y Discusión}
\input{capitulos/resultados_discusion/main.tex}

% Conclusiones
\chapter{Conclusiones}
\input{capitulos/conclusiones/main.tex}

% Planificación temporal y presupuesto
\chapter{Planificación Temporal y Presupuesto}
\input{capitulos/planificacion_presupuesto/main.tex}

% Bibliografía
\newpage
\addcontentsline{toc}{chapter}{Bibliografía}
\printbibliography

\end{document}


% Resultados y discusión (incluyendo la valoración de impactos y de aspectos de responsabilidad legal, ética y profesional relacionados con el trabajo)
\chapter{Resultados y Discusión}
\documentclass[a4paper,11pt,twoside]{report}
\usepackage[left=25mm,right=25mm,top=25mm,bottom=25mm,includehead,includefoot,headsep=15mm,footskip=15mm]{geometry}
\usepackage{graphicx}
\usepackage{fancyhdr}
\usepackage{titlesec}
\usepackage[spanish]{babel}
\usepackage[utf8]{inputenc}
\usepackage{amsmath}
\usepackage{setspace}
\usepackage{svg}
\usepackage{hyperref}
\usepackage[backend=biber,style=numeric]{biblatex}
\addbibresource{references.bib}
\hypersetup{
    colorlinks=true,
    linkcolor=blue,      % color of internal links (sections, etc.)
    urlcolor=blue,       % color of external links
    pdftitle={Optimización energética de sistema híbrido con bomba de calor, suelo radiante, fotovoltaica y almacenamiento para vivienda},    % title
    pdfauthor={Luis D. Aranda Sánchez},     % author
    pdfkeywords={palabra1, palabra2, código1, etc.} % list of keywords
}

% Font change to Arial
\usepackage{helvet}
\renewcommand{\familydefault}{\sfdefault}

% Chapter titles in uppercase and larger font
\titleformat{\chapter}[hang]{\large\bfseries}{\thechapter.}{1em}{\MakeUppercase}
\titleformat{\section}[hang]{\bfseries}{\thesection.}{1em}{}
\titleformat{\subsection}[hang]{\bfseries}{\thesubsection.}{1em}{}

% Fancyhdr setup
\setlength{\headheight}{14.30174pt} % Adjust to recommended value, slightly larger for safety
\fancyhf{} % Clear all headers and footers
\fancyhead[LE]{\nouppercase{\leftmark}}
\fancyhead[RO]{Optimización energética para vivienda}
\fancyfoot[LE]{\thepage}
\fancyfoot[RE]{Escuela Técnica Superior de Ingenieros Industriales (UPM)}
\fancyfoot[LO]{Luis D. Aranda Sánchez}
\fancyfoot[RO]{\thepage}
\renewcommand{\headrulewidth}{0.4pt}
\renewcommand{\footrulewidth}{0.4pt}

\fancypagestyle{myfancy}{
    \fancyhf{} % Clear all headers and footers
    \fancyhead[LE]{\nouppercase{\leftmark}}
    \fancyhead[RO]{Optimización energética para vivienda}
    \fancyfoot[LE]{\thepage}
    \fancyfoot[RE]{Escuela Técnica Superior de Ingenieros Industriales (UPM)}
    \fancyfoot[LO]{Luis D. Aranda Sánchez}
    \fancyfoot[RO]{\thepage}
    \renewcommand{\headrulewidth}{0.4pt}
    \renewcommand{\footrulewidth}{0.4pt}
}

\fancypagestyle{simple}{
    \fancyhf{} % Clear all headers and footers
    \renewcommand{\headrulewidth}{0pt}
    \renewcommand{\footrulewidth}{0pt}
}

% Line spacing
\setstretch{1.2}

% Document starts here
\begin{document}

% Portada
\begin{titlepage}
    \centering
    {\scshape\LARGE Universidad Politécnica de Madrid \par}
    \vspace{1cm}
    {\scshape\Large Escuela Técnica Superior de Ingenieros Industriales\par}
    \vspace{1.5cm}
    {\huge\bfseries Optimización energética de sistema híbrido con bomba de calor, suelo radiante, fotovoltaica y almacenamiento para vivienda \par}
    \vspace{1.5cm}
    {\Large\bfseries Trabajo de Fin de Máster\par}
    \vspace{0.5cm}
    {\large Máster Universitario en Ingeniería de la Energía \par}
    \vspace{2cm}
    {\Large Luis D. Aranda Sánchez\par}
    \vfill
    Director: Javier Rodríguez Martín
    \vfill
    {\large Septiembre 6, 2024\par}
\end{titlepage}

% Resumen (máximo de 5 páginas, incluyendo al final Palabras clave)
\clearpage
\pagestyle{simple}
% \newpage
\chapter*{Resumen}
\addcontentsline{toc}{chapter}{Resumen}
\input{capitulos/resumen/main.tex}

% Índice (paginado)
\clearpage
\pagestyle{simple}
% \newpage
\tableofcontents

% Introducción (donde se incluya los antecedentes y justificación)
\clearpage
\pagestyle{myfancy}
\newpage
\chapter{Introducción}
\input{capitulos/introduccion/main.tex}

% Objetivos
\chapter{Objetivos}
\input{capitulos/objetivos/main.tex}

% Metodología
\chapter{Metodología}
\input{capitulos/metodologia/main.tex}

% Resultados y discusión (incluyendo la valoración de impactos y de aspectos de responsabilidad legal, ética y profesional relacionados con el trabajo)
\chapter{Resultados y Discusión}
\input{capitulos/resultados_discusion/main.tex}

% Conclusiones
\chapter{Conclusiones}
\input{capitulos/conclusiones/main.tex}

% Planificación temporal y presupuesto
\chapter{Planificación Temporal y Presupuesto}
\input{capitulos/planificacion_presupuesto/main.tex}

% Bibliografía
\newpage
\addcontentsline{toc}{chapter}{Bibliografía}
\printbibliography

\end{document}


% Conclusiones
\chapter{Conclusiones}
\documentclass[a4paper,11pt,twoside]{report}
\usepackage[left=25mm,right=25mm,top=25mm,bottom=25mm,includehead,includefoot,headsep=15mm,footskip=15mm]{geometry}
\usepackage{graphicx}
\usepackage{fancyhdr}
\usepackage{titlesec}
\usepackage[spanish]{babel}
\usepackage[utf8]{inputenc}
\usepackage{amsmath}
\usepackage{setspace}
\usepackage{svg}
\usepackage{hyperref}
\usepackage[backend=biber,style=numeric]{biblatex}
\addbibresource{references.bib}
\hypersetup{
    colorlinks=true,
    linkcolor=blue,      % color of internal links (sections, etc.)
    urlcolor=blue,       % color of external links
    pdftitle={Optimización energética de sistema híbrido con bomba de calor, suelo radiante, fotovoltaica y almacenamiento para vivienda},    % title
    pdfauthor={Luis D. Aranda Sánchez},     % author
    pdfkeywords={palabra1, palabra2, código1, etc.} % list of keywords
}

% Font change to Arial
\usepackage{helvet}
\renewcommand{\familydefault}{\sfdefault}

% Chapter titles in uppercase and larger font
\titleformat{\chapter}[hang]{\large\bfseries}{\thechapter.}{1em}{\MakeUppercase}
\titleformat{\section}[hang]{\bfseries}{\thesection.}{1em}{}
\titleformat{\subsection}[hang]{\bfseries}{\thesubsection.}{1em}{}

% Fancyhdr setup
\setlength{\headheight}{14.30174pt} % Adjust to recommended value, slightly larger for safety
\fancyhf{} % Clear all headers and footers
\fancyhead[LE]{\nouppercase{\leftmark}}
\fancyhead[RO]{Optimización energética para vivienda}
\fancyfoot[LE]{\thepage}
\fancyfoot[RE]{Escuela Técnica Superior de Ingenieros Industriales (UPM)}
\fancyfoot[LO]{Luis D. Aranda Sánchez}
\fancyfoot[RO]{\thepage}
\renewcommand{\headrulewidth}{0.4pt}
\renewcommand{\footrulewidth}{0.4pt}

\fancypagestyle{myfancy}{
    \fancyhf{} % Clear all headers and footers
    \fancyhead[LE]{\nouppercase{\leftmark}}
    \fancyhead[RO]{Optimización energética para vivienda}
    \fancyfoot[LE]{\thepage}
    \fancyfoot[RE]{Escuela Técnica Superior de Ingenieros Industriales (UPM)}
    \fancyfoot[LO]{Luis D. Aranda Sánchez}
    \fancyfoot[RO]{\thepage}
    \renewcommand{\headrulewidth}{0.4pt}
    \renewcommand{\footrulewidth}{0.4pt}
}

\fancypagestyle{simple}{
    \fancyhf{} % Clear all headers and footers
    \renewcommand{\headrulewidth}{0pt}
    \renewcommand{\footrulewidth}{0pt}
}

% Line spacing
\setstretch{1.2}

% Document starts here
\begin{document}

% Portada
\begin{titlepage}
    \centering
    {\scshape\LARGE Universidad Politécnica de Madrid \par}
    \vspace{1cm}
    {\scshape\Large Escuela Técnica Superior de Ingenieros Industriales\par}
    \vspace{1.5cm}
    {\huge\bfseries Optimización energética de sistema híbrido con bomba de calor, suelo radiante, fotovoltaica y almacenamiento para vivienda \par}
    \vspace{1.5cm}
    {\Large\bfseries Trabajo de Fin de Máster\par}
    \vspace{0.5cm}
    {\large Máster Universitario en Ingeniería de la Energía \par}
    \vspace{2cm}
    {\Large Luis D. Aranda Sánchez\par}
    \vfill
    Director: Javier Rodríguez Martín
    \vfill
    {\large Septiembre 6, 2024\par}
\end{titlepage}

% Resumen (máximo de 5 páginas, incluyendo al final Palabras clave)
\clearpage
\pagestyle{simple}
% \newpage
\chapter*{Resumen}
\addcontentsline{toc}{chapter}{Resumen}
\input{capitulos/resumen/main.tex}

% Índice (paginado)
\clearpage
\pagestyle{simple}
% \newpage
\tableofcontents

% Introducción (donde se incluya los antecedentes y justificación)
\clearpage
\pagestyle{myfancy}
\newpage
\chapter{Introducción}
\input{capitulos/introduccion/main.tex}

% Objetivos
\chapter{Objetivos}
\input{capitulos/objetivos/main.tex}

% Metodología
\chapter{Metodología}
\input{capitulos/metodologia/main.tex}

% Resultados y discusión (incluyendo la valoración de impactos y de aspectos de responsabilidad legal, ética y profesional relacionados con el trabajo)
\chapter{Resultados y Discusión}
\input{capitulos/resultados_discusion/main.tex}

% Conclusiones
\chapter{Conclusiones}
\input{capitulos/conclusiones/main.tex}

% Planificación temporal y presupuesto
\chapter{Planificación Temporal y Presupuesto}
\input{capitulos/planificacion_presupuesto/main.tex}

% Bibliografía
\newpage
\addcontentsline{toc}{chapter}{Bibliografía}
\printbibliography

\end{document}


% Planificación temporal y presupuesto
\chapter{Planificación Temporal y Presupuesto}
\documentclass[a4paper,11pt,twoside]{report}
\usepackage[left=25mm,right=25mm,top=25mm,bottom=25mm,includehead,includefoot,headsep=15mm,footskip=15mm]{geometry}
\usepackage{graphicx}
\usepackage{fancyhdr}
\usepackage{titlesec}
\usepackage[spanish]{babel}
\usepackage[utf8]{inputenc}
\usepackage{amsmath}
\usepackage{setspace}
\usepackage{svg}
\usepackage{hyperref}
\usepackage[backend=biber,style=numeric]{biblatex}
\addbibresource{references.bib}
\hypersetup{
    colorlinks=true,
    linkcolor=blue,      % color of internal links (sections, etc.)
    urlcolor=blue,       % color of external links
    pdftitle={Optimización energética de sistema híbrido con bomba de calor, suelo radiante, fotovoltaica y almacenamiento para vivienda},    % title
    pdfauthor={Luis D. Aranda Sánchez},     % author
    pdfkeywords={palabra1, palabra2, código1, etc.} % list of keywords
}

% Font change to Arial
\usepackage{helvet}
\renewcommand{\familydefault}{\sfdefault}

% Chapter titles in uppercase and larger font
\titleformat{\chapter}[hang]{\large\bfseries}{\thechapter.}{1em}{\MakeUppercase}
\titleformat{\section}[hang]{\bfseries}{\thesection.}{1em}{}
\titleformat{\subsection}[hang]{\bfseries}{\thesubsection.}{1em}{}

% Fancyhdr setup
\setlength{\headheight}{14.30174pt} % Adjust to recommended value, slightly larger for safety
\fancyhf{} % Clear all headers and footers
\fancyhead[LE]{\nouppercase{\leftmark}}
\fancyhead[RO]{Optimización energética para vivienda}
\fancyfoot[LE]{\thepage}
\fancyfoot[RE]{Escuela Técnica Superior de Ingenieros Industriales (UPM)}
\fancyfoot[LO]{Luis D. Aranda Sánchez}
\fancyfoot[RO]{\thepage}
\renewcommand{\headrulewidth}{0.4pt}
\renewcommand{\footrulewidth}{0.4pt}

\fancypagestyle{myfancy}{
    \fancyhf{} % Clear all headers and footers
    \fancyhead[LE]{\nouppercase{\leftmark}}
    \fancyhead[RO]{Optimización energética para vivienda}
    \fancyfoot[LE]{\thepage}
    \fancyfoot[RE]{Escuela Técnica Superior de Ingenieros Industriales (UPM)}
    \fancyfoot[LO]{Luis D. Aranda Sánchez}
    \fancyfoot[RO]{\thepage}
    \renewcommand{\headrulewidth}{0.4pt}
    \renewcommand{\footrulewidth}{0.4pt}
}

\fancypagestyle{simple}{
    \fancyhf{} % Clear all headers and footers
    \renewcommand{\headrulewidth}{0pt}
    \renewcommand{\footrulewidth}{0pt}
}

% Line spacing
\setstretch{1.2}

% Document starts here
\begin{document}

% Portada
\begin{titlepage}
    \centering
    {\scshape\LARGE Universidad Politécnica de Madrid \par}
    \vspace{1cm}
    {\scshape\Large Escuela Técnica Superior de Ingenieros Industriales\par}
    \vspace{1.5cm}
    {\huge\bfseries Optimización energética de sistema híbrido con bomba de calor, suelo radiante, fotovoltaica y almacenamiento para vivienda \par}
    \vspace{1.5cm}
    {\Large\bfseries Trabajo de Fin de Máster\par}
    \vspace{0.5cm}
    {\large Máster Universitario en Ingeniería de la Energía \par}
    \vspace{2cm}
    {\Large Luis D. Aranda Sánchez\par}
    \vfill
    Director: Javier Rodríguez Martín
    \vfill
    {\large Septiembre 6, 2024\par}
\end{titlepage}

% Resumen (máximo de 5 páginas, incluyendo al final Palabras clave)
\clearpage
\pagestyle{simple}
% \newpage
\chapter*{Resumen}
\addcontentsline{toc}{chapter}{Resumen}
\input{capitulos/resumen/main.tex}

% Índice (paginado)
\clearpage
\pagestyle{simple}
% \newpage
\tableofcontents

% Introducción (donde se incluya los antecedentes y justificación)
\clearpage
\pagestyle{myfancy}
\newpage
\chapter{Introducción}
\input{capitulos/introduccion/main.tex}

% Objetivos
\chapter{Objetivos}
\input{capitulos/objetivos/main.tex}

% Metodología
\chapter{Metodología}
\input{capitulos/metodologia/main.tex}

% Resultados y discusión (incluyendo la valoración de impactos y de aspectos de responsabilidad legal, ética y profesional relacionados con el trabajo)
\chapter{Resultados y Discusión}
\input{capitulos/resultados_discusion/main.tex}

% Conclusiones
\chapter{Conclusiones}
\input{capitulos/conclusiones/main.tex}

% Planificación temporal y presupuesto
\chapter{Planificación Temporal y Presupuesto}
\input{capitulos/planificacion_presupuesto/main.tex}

% Bibliografía
\newpage
\addcontentsline{toc}{chapter}{Bibliografía}
\printbibliography

\end{document}


% Bibliografía
\newpage
\addcontentsline{toc}{chapter}{Bibliografía}
\printbibliography

\end{document}


% Metodología
\chapter{Metodología}
\documentclass[a4paper,11pt,twoside]{report}
\usepackage[left=25mm,right=25mm,top=25mm,bottom=25mm,includehead,includefoot,headsep=15mm,footskip=15mm]{geometry}
\usepackage{graphicx}
\usepackage{fancyhdr}
\usepackage{titlesec}
\usepackage[spanish]{babel}
\usepackage[utf8]{inputenc}
\usepackage{amsmath}
\usepackage{setspace}
\usepackage{svg}
\usepackage{hyperref}
\usepackage[backend=biber,style=numeric]{biblatex}
\addbibresource{references.bib}
\hypersetup{
    colorlinks=true,
    linkcolor=blue,      % color of internal links (sections, etc.)
    urlcolor=blue,       % color of external links
    pdftitle={Optimización energética de sistema híbrido con bomba de calor, suelo radiante, fotovoltaica y almacenamiento para vivienda},    % title
    pdfauthor={Luis D. Aranda Sánchez},     % author
    pdfkeywords={palabra1, palabra2, código1, etc.} % list of keywords
}

% Font change to Arial
\usepackage{helvet}
\renewcommand{\familydefault}{\sfdefault}

% Chapter titles in uppercase and larger font
\titleformat{\chapter}[hang]{\large\bfseries}{\thechapter.}{1em}{\MakeUppercase}
\titleformat{\section}[hang]{\bfseries}{\thesection.}{1em}{}
\titleformat{\subsection}[hang]{\bfseries}{\thesubsection.}{1em}{}

% Fancyhdr setup
\setlength{\headheight}{14.30174pt} % Adjust to recommended value, slightly larger for safety
\fancyhf{} % Clear all headers and footers
\fancyhead[LE]{\nouppercase{\leftmark}}
\fancyhead[RO]{Optimización energética para vivienda}
\fancyfoot[LE]{\thepage}
\fancyfoot[RE]{Escuela Técnica Superior de Ingenieros Industriales (UPM)}
\fancyfoot[LO]{Luis D. Aranda Sánchez}
\fancyfoot[RO]{\thepage}
\renewcommand{\headrulewidth}{0.4pt}
\renewcommand{\footrulewidth}{0.4pt}

\fancypagestyle{myfancy}{
    \fancyhf{} % Clear all headers and footers
    \fancyhead[LE]{\nouppercase{\leftmark}}
    \fancyhead[RO]{Optimización energética para vivienda}
    \fancyfoot[LE]{\thepage}
    \fancyfoot[RE]{Escuela Técnica Superior de Ingenieros Industriales (UPM)}
    \fancyfoot[LO]{Luis D. Aranda Sánchez}
    \fancyfoot[RO]{\thepage}
    \renewcommand{\headrulewidth}{0.4pt}
    \renewcommand{\footrulewidth}{0.4pt}
}

\fancypagestyle{simple}{
    \fancyhf{} % Clear all headers and footers
    \renewcommand{\headrulewidth}{0pt}
    \renewcommand{\footrulewidth}{0pt}
}

% Line spacing
\setstretch{1.2}

% Document starts here
\begin{document}

% Portada
\begin{titlepage}
    \centering
    {\scshape\LARGE Universidad Politécnica de Madrid \par}
    \vspace{1cm}
    {\scshape\Large Escuela Técnica Superior de Ingenieros Industriales\par}
    \vspace{1.5cm}
    {\huge\bfseries Optimización energética de sistema híbrido con bomba de calor, suelo radiante, fotovoltaica y almacenamiento para vivienda \par}
    \vspace{1.5cm}
    {\Large\bfseries Trabajo de Fin de Máster\par}
    \vspace{0.5cm}
    {\large Máster Universitario en Ingeniería de la Energía \par}
    \vspace{2cm}
    {\Large Luis D. Aranda Sánchez\par}
    \vfill
    Director: Javier Rodríguez Martín
    \vfill
    {\large Septiembre 6, 2024\par}
\end{titlepage}

% Resumen (máximo de 5 páginas, incluyendo al final Palabras clave)
\clearpage
\pagestyle{simple}
% \newpage
\chapter*{Resumen}
\addcontentsline{toc}{chapter}{Resumen}
\documentclass[a4paper,11pt,twoside]{report}
\usepackage[left=25mm,right=25mm,top=25mm,bottom=25mm,includehead,includefoot,headsep=15mm,footskip=15mm]{geometry}
\usepackage{graphicx}
\usepackage{fancyhdr}
\usepackage{titlesec}
\usepackage[spanish]{babel}
\usepackage[utf8]{inputenc}
\usepackage{amsmath}
\usepackage{setspace}
\usepackage{svg}
\usepackage{hyperref}
\usepackage[backend=biber,style=numeric]{biblatex}
\addbibresource{references.bib}
\hypersetup{
    colorlinks=true,
    linkcolor=blue,      % color of internal links (sections, etc.)
    urlcolor=blue,       % color of external links
    pdftitle={Optimización energética de sistema híbrido con bomba de calor, suelo radiante, fotovoltaica y almacenamiento para vivienda},    % title
    pdfauthor={Luis D. Aranda Sánchez},     % author
    pdfkeywords={palabra1, palabra2, código1, etc.} % list of keywords
}

% Font change to Arial
\usepackage{helvet}
\renewcommand{\familydefault}{\sfdefault}

% Chapter titles in uppercase and larger font
\titleformat{\chapter}[hang]{\large\bfseries}{\thechapter.}{1em}{\MakeUppercase}
\titleformat{\section}[hang]{\bfseries}{\thesection.}{1em}{}
\titleformat{\subsection}[hang]{\bfseries}{\thesubsection.}{1em}{}

% Fancyhdr setup
\setlength{\headheight}{14.30174pt} % Adjust to recommended value, slightly larger for safety
\fancyhf{} % Clear all headers and footers
\fancyhead[LE]{\nouppercase{\leftmark}}
\fancyhead[RO]{Optimización energética para vivienda}
\fancyfoot[LE]{\thepage}
\fancyfoot[RE]{Escuela Técnica Superior de Ingenieros Industriales (UPM)}
\fancyfoot[LO]{Luis D. Aranda Sánchez}
\fancyfoot[RO]{\thepage}
\renewcommand{\headrulewidth}{0.4pt}
\renewcommand{\footrulewidth}{0.4pt}

\fancypagestyle{myfancy}{
    \fancyhf{} % Clear all headers and footers
    \fancyhead[LE]{\nouppercase{\leftmark}}
    \fancyhead[RO]{Optimización energética para vivienda}
    \fancyfoot[LE]{\thepage}
    \fancyfoot[RE]{Escuela Técnica Superior de Ingenieros Industriales (UPM)}
    \fancyfoot[LO]{Luis D. Aranda Sánchez}
    \fancyfoot[RO]{\thepage}
    \renewcommand{\headrulewidth}{0.4pt}
    \renewcommand{\footrulewidth}{0.4pt}
}

\fancypagestyle{simple}{
    \fancyhf{} % Clear all headers and footers
    \renewcommand{\headrulewidth}{0pt}
    \renewcommand{\footrulewidth}{0pt}
}

% Line spacing
\setstretch{1.2}

% Document starts here
\begin{document}

% Portada
\begin{titlepage}
    \centering
    {\scshape\LARGE Universidad Politécnica de Madrid \par}
    \vspace{1cm}
    {\scshape\Large Escuela Técnica Superior de Ingenieros Industriales\par}
    \vspace{1.5cm}
    {\huge\bfseries Optimización energética de sistema híbrido con bomba de calor, suelo radiante, fotovoltaica y almacenamiento para vivienda \par}
    \vspace{1.5cm}
    {\Large\bfseries Trabajo de Fin de Máster\par}
    \vspace{0.5cm}
    {\large Máster Universitario en Ingeniería de la Energía \par}
    \vspace{2cm}
    {\Large Luis D. Aranda Sánchez\par}
    \vfill
    Director: Javier Rodríguez Martín
    \vfill
    {\large Septiembre 6, 2024\par}
\end{titlepage}

% Resumen (máximo de 5 páginas, incluyendo al final Palabras clave)
\clearpage
\pagestyle{simple}
% \newpage
\chapter*{Resumen}
\addcontentsline{toc}{chapter}{Resumen}
\input{capitulos/resumen/main.tex}

% Índice (paginado)
\clearpage
\pagestyle{simple}
% \newpage
\tableofcontents

% Introducción (donde se incluya los antecedentes y justificación)
\clearpage
\pagestyle{myfancy}
\newpage
\chapter{Introducción}
\input{capitulos/introduccion/main.tex}

% Objetivos
\chapter{Objetivos}
\input{capitulos/objetivos/main.tex}

% Metodología
\chapter{Metodología}
\input{capitulos/metodologia/main.tex}

% Resultados y discusión (incluyendo la valoración de impactos y de aspectos de responsabilidad legal, ética y profesional relacionados con el trabajo)
\chapter{Resultados y Discusión}
\input{capitulos/resultados_discusion/main.tex}

% Conclusiones
\chapter{Conclusiones}
\input{capitulos/conclusiones/main.tex}

% Planificación temporal y presupuesto
\chapter{Planificación Temporal y Presupuesto}
\input{capitulos/planificacion_presupuesto/main.tex}

% Bibliografía
\newpage
\addcontentsline{toc}{chapter}{Bibliografía}
\printbibliography

\end{document}


% Índice (paginado)
\clearpage
\pagestyle{simple}
% \newpage
\tableofcontents

% Introducción (donde se incluya los antecedentes y justificación)
\clearpage
\pagestyle{myfancy}
\newpage
\chapter{Introducción}
\documentclass[a4paper,11pt,twoside]{report}
\usepackage[left=25mm,right=25mm,top=25mm,bottom=25mm,includehead,includefoot,headsep=15mm,footskip=15mm]{geometry}
\usepackage{graphicx}
\usepackage{fancyhdr}
\usepackage{titlesec}
\usepackage[spanish]{babel}
\usepackage[utf8]{inputenc}
\usepackage{amsmath}
\usepackage{setspace}
\usepackage{svg}
\usepackage{hyperref}
\usepackage[backend=biber,style=numeric]{biblatex}
\addbibresource{references.bib}
\hypersetup{
    colorlinks=true,
    linkcolor=blue,      % color of internal links (sections, etc.)
    urlcolor=blue,       % color of external links
    pdftitle={Optimización energética de sistema híbrido con bomba de calor, suelo radiante, fotovoltaica y almacenamiento para vivienda},    % title
    pdfauthor={Luis D. Aranda Sánchez},     % author
    pdfkeywords={palabra1, palabra2, código1, etc.} % list of keywords
}

% Font change to Arial
\usepackage{helvet}
\renewcommand{\familydefault}{\sfdefault}

% Chapter titles in uppercase and larger font
\titleformat{\chapter}[hang]{\large\bfseries}{\thechapter.}{1em}{\MakeUppercase}
\titleformat{\section}[hang]{\bfseries}{\thesection.}{1em}{}
\titleformat{\subsection}[hang]{\bfseries}{\thesubsection.}{1em}{}

% Fancyhdr setup
\setlength{\headheight}{14.30174pt} % Adjust to recommended value, slightly larger for safety
\fancyhf{} % Clear all headers and footers
\fancyhead[LE]{\nouppercase{\leftmark}}
\fancyhead[RO]{Optimización energética para vivienda}
\fancyfoot[LE]{\thepage}
\fancyfoot[RE]{Escuela Técnica Superior de Ingenieros Industriales (UPM)}
\fancyfoot[LO]{Luis D. Aranda Sánchez}
\fancyfoot[RO]{\thepage}
\renewcommand{\headrulewidth}{0.4pt}
\renewcommand{\footrulewidth}{0.4pt}

\fancypagestyle{myfancy}{
    \fancyhf{} % Clear all headers and footers
    \fancyhead[LE]{\nouppercase{\leftmark}}
    \fancyhead[RO]{Optimización energética para vivienda}
    \fancyfoot[LE]{\thepage}
    \fancyfoot[RE]{Escuela Técnica Superior de Ingenieros Industriales (UPM)}
    \fancyfoot[LO]{Luis D. Aranda Sánchez}
    \fancyfoot[RO]{\thepage}
    \renewcommand{\headrulewidth}{0.4pt}
    \renewcommand{\footrulewidth}{0.4pt}
}

\fancypagestyle{simple}{
    \fancyhf{} % Clear all headers and footers
    \renewcommand{\headrulewidth}{0pt}
    \renewcommand{\footrulewidth}{0pt}
}

% Line spacing
\setstretch{1.2}

% Document starts here
\begin{document}

% Portada
\begin{titlepage}
    \centering
    {\scshape\LARGE Universidad Politécnica de Madrid \par}
    \vspace{1cm}
    {\scshape\Large Escuela Técnica Superior de Ingenieros Industriales\par}
    \vspace{1.5cm}
    {\huge\bfseries Optimización energética de sistema híbrido con bomba de calor, suelo radiante, fotovoltaica y almacenamiento para vivienda \par}
    \vspace{1.5cm}
    {\Large\bfseries Trabajo de Fin de Máster\par}
    \vspace{0.5cm}
    {\large Máster Universitario en Ingeniería de la Energía \par}
    \vspace{2cm}
    {\Large Luis D. Aranda Sánchez\par}
    \vfill
    Director: Javier Rodríguez Martín
    \vfill
    {\large Septiembre 6, 2024\par}
\end{titlepage}

% Resumen (máximo de 5 páginas, incluyendo al final Palabras clave)
\clearpage
\pagestyle{simple}
% \newpage
\chapter*{Resumen}
\addcontentsline{toc}{chapter}{Resumen}
\input{capitulos/resumen/main.tex}

% Índice (paginado)
\clearpage
\pagestyle{simple}
% \newpage
\tableofcontents

% Introducción (donde se incluya los antecedentes y justificación)
\clearpage
\pagestyle{myfancy}
\newpage
\chapter{Introducción}
\input{capitulos/introduccion/main.tex}

% Objetivos
\chapter{Objetivos}
\input{capitulos/objetivos/main.tex}

% Metodología
\chapter{Metodología}
\input{capitulos/metodologia/main.tex}

% Resultados y discusión (incluyendo la valoración de impactos y de aspectos de responsabilidad legal, ética y profesional relacionados con el trabajo)
\chapter{Resultados y Discusión}
\input{capitulos/resultados_discusion/main.tex}

% Conclusiones
\chapter{Conclusiones}
\input{capitulos/conclusiones/main.tex}

% Planificación temporal y presupuesto
\chapter{Planificación Temporal y Presupuesto}
\input{capitulos/planificacion_presupuesto/main.tex}

% Bibliografía
\newpage
\addcontentsline{toc}{chapter}{Bibliografía}
\printbibliography

\end{document}


% Objetivos
\chapter{Objetivos}
\documentclass[a4paper,11pt,twoside]{report}
\usepackage[left=25mm,right=25mm,top=25mm,bottom=25mm,includehead,includefoot,headsep=15mm,footskip=15mm]{geometry}
\usepackage{graphicx}
\usepackage{fancyhdr}
\usepackage{titlesec}
\usepackage[spanish]{babel}
\usepackage[utf8]{inputenc}
\usepackage{amsmath}
\usepackage{setspace}
\usepackage{svg}
\usepackage{hyperref}
\usepackage[backend=biber,style=numeric]{biblatex}
\addbibresource{references.bib}
\hypersetup{
    colorlinks=true,
    linkcolor=blue,      % color of internal links (sections, etc.)
    urlcolor=blue,       % color of external links
    pdftitle={Optimización energética de sistema híbrido con bomba de calor, suelo radiante, fotovoltaica y almacenamiento para vivienda},    % title
    pdfauthor={Luis D. Aranda Sánchez},     % author
    pdfkeywords={palabra1, palabra2, código1, etc.} % list of keywords
}

% Font change to Arial
\usepackage{helvet}
\renewcommand{\familydefault}{\sfdefault}

% Chapter titles in uppercase and larger font
\titleformat{\chapter}[hang]{\large\bfseries}{\thechapter.}{1em}{\MakeUppercase}
\titleformat{\section}[hang]{\bfseries}{\thesection.}{1em}{}
\titleformat{\subsection}[hang]{\bfseries}{\thesubsection.}{1em}{}

% Fancyhdr setup
\setlength{\headheight}{14.30174pt} % Adjust to recommended value, slightly larger for safety
\fancyhf{} % Clear all headers and footers
\fancyhead[LE]{\nouppercase{\leftmark}}
\fancyhead[RO]{Optimización energética para vivienda}
\fancyfoot[LE]{\thepage}
\fancyfoot[RE]{Escuela Técnica Superior de Ingenieros Industriales (UPM)}
\fancyfoot[LO]{Luis D. Aranda Sánchez}
\fancyfoot[RO]{\thepage}
\renewcommand{\headrulewidth}{0.4pt}
\renewcommand{\footrulewidth}{0.4pt}

\fancypagestyle{myfancy}{
    \fancyhf{} % Clear all headers and footers
    \fancyhead[LE]{\nouppercase{\leftmark}}
    \fancyhead[RO]{Optimización energética para vivienda}
    \fancyfoot[LE]{\thepage}
    \fancyfoot[RE]{Escuela Técnica Superior de Ingenieros Industriales (UPM)}
    \fancyfoot[LO]{Luis D. Aranda Sánchez}
    \fancyfoot[RO]{\thepage}
    \renewcommand{\headrulewidth}{0.4pt}
    \renewcommand{\footrulewidth}{0.4pt}
}

\fancypagestyle{simple}{
    \fancyhf{} % Clear all headers and footers
    \renewcommand{\headrulewidth}{0pt}
    \renewcommand{\footrulewidth}{0pt}
}

% Line spacing
\setstretch{1.2}

% Document starts here
\begin{document}

% Portada
\begin{titlepage}
    \centering
    {\scshape\LARGE Universidad Politécnica de Madrid \par}
    \vspace{1cm}
    {\scshape\Large Escuela Técnica Superior de Ingenieros Industriales\par}
    \vspace{1.5cm}
    {\huge\bfseries Optimización energética de sistema híbrido con bomba de calor, suelo radiante, fotovoltaica y almacenamiento para vivienda \par}
    \vspace{1.5cm}
    {\Large\bfseries Trabajo de Fin de Máster\par}
    \vspace{0.5cm}
    {\large Máster Universitario en Ingeniería de la Energía \par}
    \vspace{2cm}
    {\Large Luis D. Aranda Sánchez\par}
    \vfill
    Director: Javier Rodríguez Martín
    \vfill
    {\large Septiembre 6, 2024\par}
\end{titlepage}

% Resumen (máximo de 5 páginas, incluyendo al final Palabras clave)
\clearpage
\pagestyle{simple}
% \newpage
\chapter*{Resumen}
\addcontentsline{toc}{chapter}{Resumen}
\input{capitulos/resumen/main.tex}

% Índice (paginado)
\clearpage
\pagestyle{simple}
% \newpage
\tableofcontents

% Introducción (donde se incluya los antecedentes y justificación)
\clearpage
\pagestyle{myfancy}
\newpage
\chapter{Introducción}
\input{capitulos/introduccion/main.tex}

% Objetivos
\chapter{Objetivos}
\input{capitulos/objetivos/main.tex}

% Metodología
\chapter{Metodología}
\input{capitulos/metodologia/main.tex}

% Resultados y discusión (incluyendo la valoración de impactos y de aspectos de responsabilidad legal, ética y profesional relacionados con el trabajo)
\chapter{Resultados y Discusión}
\input{capitulos/resultados_discusion/main.tex}

% Conclusiones
\chapter{Conclusiones}
\input{capitulos/conclusiones/main.tex}

% Planificación temporal y presupuesto
\chapter{Planificación Temporal y Presupuesto}
\input{capitulos/planificacion_presupuesto/main.tex}

% Bibliografía
\newpage
\addcontentsline{toc}{chapter}{Bibliografía}
\printbibliography

\end{document}


% Metodología
\chapter{Metodología}
\documentclass[a4paper,11pt,twoside]{report}
\usepackage[left=25mm,right=25mm,top=25mm,bottom=25mm,includehead,includefoot,headsep=15mm,footskip=15mm]{geometry}
\usepackage{graphicx}
\usepackage{fancyhdr}
\usepackage{titlesec}
\usepackage[spanish]{babel}
\usepackage[utf8]{inputenc}
\usepackage{amsmath}
\usepackage{setspace}
\usepackage{svg}
\usepackage{hyperref}
\usepackage[backend=biber,style=numeric]{biblatex}
\addbibresource{references.bib}
\hypersetup{
    colorlinks=true,
    linkcolor=blue,      % color of internal links (sections, etc.)
    urlcolor=blue,       % color of external links
    pdftitle={Optimización energética de sistema híbrido con bomba de calor, suelo radiante, fotovoltaica y almacenamiento para vivienda},    % title
    pdfauthor={Luis D. Aranda Sánchez},     % author
    pdfkeywords={palabra1, palabra2, código1, etc.} % list of keywords
}

% Font change to Arial
\usepackage{helvet}
\renewcommand{\familydefault}{\sfdefault}

% Chapter titles in uppercase and larger font
\titleformat{\chapter}[hang]{\large\bfseries}{\thechapter.}{1em}{\MakeUppercase}
\titleformat{\section}[hang]{\bfseries}{\thesection.}{1em}{}
\titleformat{\subsection}[hang]{\bfseries}{\thesubsection.}{1em}{}

% Fancyhdr setup
\setlength{\headheight}{14.30174pt} % Adjust to recommended value, slightly larger for safety
\fancyhf{} % Clear all headers and footers
\fancyhead[LE]{\nouppercase{\leftmark}}
\fancyhead[RO]{Optimización energética para vivienda}
\fancyfoot[LE]{\thepage}
\fancyfoot[RE]{Escuela Técnica Superior de Ingenieros Industriales (UPM)}
\fancyfoot[LO]{Luis D. Aranda Sánchez}
\fancyfoot[RO]{\thepage}
\renewcommand{\headrulewidth}{0.4pt}
\renewcommand{\footrulewidth}{0.4pt}

\fancypagestyle{myfancy}{
    \fancyhf{} % Clear all headers and footers
    \fancyhead[LE]{\nouppercase{\leftmark}}
    \fancyhead[RO]{Optimización energética para vivienda}
    \fancyfoot[LE]{\thepage}
    \fancyfoot[RE]{Escuela Técnica Superior de Ingenieros Industriales (UPM)}
    \fancyfoot[LO]{Luis D. Aranda Sánchez}
    \fancyfoot[RO]{\thepage}
    \renewcommand{\headrulewidth}{0.4pt}
    \renewcommand{\footrulewidth}{0.4pt}
}

\fancypagestyle{simple}{
    \fancyhf{} % Clear all headers and footers
    \renewcommand{\headrulewidth}{0pt}
    \renewcommand{\footrulewidth}{0pt}
}

% Line spacing
\setstretch{1.2}

% Document starts here
\begin{document}

% Portada
\begin{titlepage}
    \centering
    {\scshape\LARGE Universidad Politécnica de Madrid \par}
    \vspace{1cm}
    {\scshape\Large Escuela Técnica Superior de Ingenieros Industriales\par}
    \vspace{1.5cm}
    {\huge\bfseries Optimización energética de sistema híbrido con bomba de calor, suelo radiante, fotovoltaica y almacenamiento para vivienda \par}
    \vspace{1.5cm}
    {\Large\bfseries Trabajo de Fin de Máster\par}
    \vspace{0.5cm}
    {\large Máster Universitario en Ingeniería de la Energía \par}
    \vspace{2cm}
    {\Large Luis D. Aranda Sánchez\par}
    \vfill
    Director: Javier Rodríguez Martín
    \vfill
    {\large Septiembre 6, 2024\par}
\end{titlepage}

% Resumen (máximo de 5 páginas, incluyendo al final Palabras clave)
\clearpage
\pagestyle{simple}
% \newpage
\chapter*{Resumen}
\addcontentsline{toc}{chapter}{Resumen}
\input{capitulos/resumen/main.tex}

% Índice (paginado)
\clearpage
\pagestyle{simple}
% \newpage
\tableofcontents

% Introducción (donde se incluya los antecedentes y justificación)
\clearpage
\pagestyle{myfancy}
\newpage
\chapter{Introducción}
\input{capitulos/introduccion/main.tex}

% Objetivos
\chapter{Objetivos}
\input{capitulos/objetivos/main.tex}

% Metodología
\chapter{Metodología}
\input{capitulos/metodologia/main.tex}

% Resultados y discusión (incluyendo la valoración de impactos y de aspectos de responsabilidad legal, ética y profesional relacionados con el trabajo)
\chapter{Resultados y Discusión}
\input{capitulos/resultados_discusion/main.tex}

% Conclusiones
\chapter{Conclusiones}
\input{capitulos/conclusiones/main.tex}

% Planificación temporal y presupuesto
\chapter{Planificación Temporal y Presupuesto}
\input{capitulos/planificacion_presupuesto/main.tex}

% Bibliografía
\newpage
\addcontentsline{toc}{chapter}{Bibliografía}
\printbibliography

\end{document}


% Resultados y discusión (incluyendo la valoración de impactos y de aspectos de responsabilidad legal, ética y profesional relacionados con el trabajo)
\chapter{Resultados y Discusión}
\documentclass[a4paper,11pt,twoside]{report}
\usepackage[left=25mm,right=25mm,top=25mm,bottom=25mm,includehead,includefoot,headsep=15mm,footskip=15mm]{geometry}
\usepackage{graphicx}
\usepackage{fancyhdr}
\usepackage{titlesec}
\usepackage[spanish]{babel}
\usepackage[utf8]{inputenc}
\usepackage{amsmath}
\usepackage{setspace}
\usepackage{svg}
\usepackage{hyperref}
\usepackage[backend=biber,style=numeric]{biblatex}
\addbibresource{references.bib}
\hypersetup{
    colorlinks=true,
    linkcolor=blue,      % color of internal links (sections, etc.)
    urlcolor=blue,       % color of external links
    pdftitle={Optimización energética de sistema híbrido con bomba de calor, suelo radiante, fotovoltaica y almacenamiento para vivienda},    % title
    pdfauthor={Luis D. Aranda Sánchez},     % author
    pdfkeywords={palabra1, palabra2, código1, etc.} % list of keywords
}

% Font change to Arial
\usepackage{helvet}
\renewcommand{\familydefault}{\sfdefault}

% Chapter titles in uppercase and larger font
\titleformat{\chapter}[hang]{\large\bfseries}{\thechapter.}{1em}{\MakeUppercase}
\titleformat{\section}[hang]{\bfseries}{\thesection.}{1em}{}
\titleformat{\subsection}[hang]{\bfseries}{\thesubsection.}{1em}{}

% Fancyhdr setup
\setlength{\headheight}{14.30174pt} % Adjust to recommended value, slightly larger for safety
\fancyhf{} % Clear all headers and footers
\fancyhead[LE]{\nouppercase{\leftmark}}
\fancyhead[RO]{Optimización energética para vivienda}
\fancyfoot[LE]{\thepage}
\fancyfoot[RE]{Escuela Técnica Superior de Ingenieros Industriales (UPM)}
\fancyfoot[LO]{Luis D. Aranda Sánchez}
\fancyfoot[RO]{\thepage}
\renewcommand{\headrulewidth}{0.4pt}
\renewcommand{\footrulewidth}{0.4pt}

\fancypagestyle{myfancy}{
    \fancyhf{} % Clear all headers and footers
    \fancyhead[LE]{\nouppercase{\leftmark}}
    \fancyhead[RO]{Optimización energética para vivienda}
    \fancyfoot[LE]{\thepage}
    \fancyfoot[RE]{Escuela Técnica Superior de Ingenieros Industriales (UPM)}
    \fancyfoot[LO]{Luis D. Aranda Sánchez}
    \fancyfoot[RO]{\thepage}
    \renewcommand{\headrulewidth}{0.4pt}
    \renewcommand{\footrulewidth}{0.4pt}
}

\fancypagestyle{simple}{
    \fancyhf{} % Clear all headers and footers
    \renewcommand{\headrulewidth}{0pt}
    \renewcommand{\footrulewidth}{0pt}
}

% Line spacing
\setstretch{1.2}

% Document starts here
\begin{document}

% Portada
\begin{titlepage}
    \centering
    {\scshape\LARGE Universidad Politécnica de Madrid \par}
    \vspace{1cm}
    {\scshape\Large Escuela Técnica Superior de Ingenieros Industriales\par}
    \vspace{1.5cm}
    {\huge\bfseries Optimización energética de sistema híbrido con bomba de calor, suelo radiante, fotovoltaica y almacenamiento para vivienda \par}
    \vspace{1.5cm}
    {\Large\bfseries Trabajo de Fin de Máster\par}
    \vspace{0.5cm}
    {\large Máster Universitario en Ingeniería de la Energía \par}
    \vspace{2cm}
    {\Large Luis D. Aranda Sánchez\par}
    \vfill
    Director: Javier Rodríguez Martín
    \vfill
    {\large Septiembre 6, 2024\par}
\end{titlepage}

% Resumen (máximo de 5 páginas, incluyendo al final Palabras clave)
\clearpage
\pagestyle{simple}
% \newpage
\chapter*{Resumen}
\addcontentsline{toc}{chapter}{Resumen}
\input{capitulos/resumen/main.tex}

% Índice (paginado)
\clearpage
\pagestyle{simple}
% \newpage
\tableofcontents

% Introducción (donde se incluya los antecedentes y justificación)
\clearpage
\pagestyle{myfancy}
\newpage
\chapter{Introducción}
\input{capitulos/introduccion/main.tex}

% Objetivos
\chapter{Objetivos}
\input{capitulos/objetivos/main.tex}

% Metodología
\chapter{Metodología}
\input{capitulos/metodologia/main.tex}

% Resultados y discusión (incluyendo la valoración de impactos y de aspectos de responsabilidad legal, ética y profesional relacionados con el trabajo)
\chapter{Resultados y Discusión}
\input{capitulos/resultados_discusion/main.tex}

% Conclusiones
\chapter{Conclusiones}
\input{capitulos/conclusiones/main.tex}

% Planificación temporal y presupuesto
\chapter{Planificación Temporal y Presupuesto}
\input{capitulos/planificacion_presupuesto/main.tex}

% Bibliografía
\newpage
\addcontentsline{toc}{chapter}{Bibliografía}
\printbibliography

\end{document}


% Conclusiones
\chapter{Conclusiones}
\documentclass[a4paper,11pt,twoside]{report}
\usepackage[left=25mm,right=25mm,top=25mm,bottom=25mm,includehead,includefoot,headsep=15mm,footskip=15mm]{geometry}
\usepackage{graphicx}
\usepackage{fancyhdr}
\usepackage{titlesec}
\usepackage[spanish]{babel}
\usepackage[utf8]{inputenc}
\usepackage{amsmath}
\usepackage{setspace}
\usepackage{svg}
\usepackage{hyperref}
\usepackage[backend=biber,style=numeric]{biblatex}
\addbibresource{references.bib}
\hypersetup{
    colorlinks=true,
    linkcolor=blue,      % color of internal links (sections, etc.)
    urlcolor=blue,       % color of external links
    pdftitle={Optimización energética de sistema híbrido con bomba de calor, suelo radiante, fotovoltaica y almacenamiento para vivienda},    % title
    pdfauthor={Luis D. Aranda Sánchez},     % author
    pdfkeywords={palabra1, palabra2, código1, etc.} % list of keywords
}

% Font change to Arial
\usepackage{helvet}
\renewcommand{\familydefault}{\sfdefault}

% Chapter titles in uppercase and larger font
\titleformat{\chapter}[hang]{\large\bfseries}{\thechapter.}{1em}{\MakeUppercase}
\titleformat{\section}[hang]{\bfseries}{\thesection.}{1em}{}
\titleformat{\subsection}[hang]{\bfseries}{\thesubsection.}{1em}{}

% Fancyhdr setup
\setlength{\headheight}{14.30174pt} % Adjust to recommended value, slightly larger for safety
\fancyhf{} % Clear all headers and footers
\fancyhead[LE]{\nouppercase{\leftmark}}
\fancyhead[RO]{Optimización energética para vivienda}
\fancyfoot[LE]{\thepage}
\fancyfoot[RE]{Escuela Técnica Superior de Ingenieros Industriales (UPM)}
\fancyfoot[LO]{Luis D. Aranda Sánchez}
\fancyfoot[RO]{\thepage}
\renewcommand{\headrulewidth}{0.4pt}
\renewcommand{\footrulewidth}{0.4pt}

\fancypagestyle{myfancy}{
    \fancyhf{} % Clear all headers and footers
    \fancyhead[LE]{\nouppercase{\leftmark}}
    \fancyhead[RO]{Optimización energética para vivienda}
    \fancyfoot[LE]{\thepage}
    \fancyfoot[RE]{Escuela Técnica Superior de Ingenieros Industriales (UPM)}
    \fancyfoot[LO]{Luis D. Aranda Sánchez}
    \fancyfoot[RO]{\thepage}
    \renewcommand{\headrulewidth}{0.4pt}
    \renewcommand{\footrulewidth}{0.4pt}
}

\fancypagestyle{simple}{
    \fancyhf{} % Clear all headers and footers
    \renewcommand{\headrulewidth}{0pt}
    \renewcommand{\footrulewidth}{0pt}
}

% Line spacing
\setstretch{1.2}

% Document starts here
\begin{document}

% Portada
\begin{titlepage}
    \centering
    {\scshape\LARGE Universidad Politécnica de Madrid \par}
    \vspace{1cm}
    {\scshape\Large Escuela Técnica Superior de Ingenieros Industriales\par}
    \vspace{1.5cm}
    {\huge\bfseries Optimización energética de sistema híbrido con bomba de calor, suelo radiante, fotovoltaica y almacenamiento para vivienda \par}
    \vspace{1.5cm}
    {\Large\bfseries Trabajo de Fin de Máster\par}
    \vspace{0.5cm}
    {\large Máster Universitario en Ingeniería de la Energía \par}
    \vspace{2cm}
    {\Large Luis D. Aranda Sánchez\par}
    \vfill
    Director: Javier Rodríguez Martín
    \vfill
    {\large Septiembre 6, 2024\par}
\end{titlepage}

% Resumen (máximo de 5 páginas, incluyendo al final Palabras clave)
\clearpage
\pagestyle{simple}
% \newpage
\chapter*{Resumen}
\addcontentsline{toc}{chapter}{Resumen}
\input{capitulos/resumen/main.tex}

% Índice (paginado)
\clearpage
\pagestyle{simple}
% \newpage
\tableofcontents

% Introducción (donde se incluya los antecedentes y justificación)
\clearpage
\pagestyle{myfancy}
\newpage
\chapter{Introducción}
\input{capitulos/introduccion/main.tex}

% Objetivos
\chapter{Objetivos}
\input{capitulos/objetivos/main.tex}

% Metodología
\chapter{Metodología}
\input{capitulos/metodologia/main.tex}

% Resultados y discusión (incluyendo la valoración de impactos y de aspectos de responsabilidad legal, ética y profesional relacionados con el trabajo)
\chapter{Resultados y Discusión}
\input{capitulos/resultados_discusion/main.tex}

% Conclusiones
\chapter{Conclusiones}
\input{capitulos/conclusiones/main.tex}

% Planificación temporal y presupuesto
\chapter{Planificación Temporal y Presupuesto}
\input{capitulos/planificacion_presupuesto/main.tex}

% Bibliografía
\newpage
\addcontentsline{toc}{chapter}{Bibliografía}
\printbibliography

\end{document}


% Planificación temporal y presupuesto
\chapter{Planificación Temporal y Presupuesto}
\documentclass[a4paper,11pt,twoside]{report}
\usepackage[left=25mm,right=25mm,top=25mm,bottom=25mm,includehead,includefoot,headsep=15mm,footskip=15mm]{geometry}
\usepackage{graphicx}
\usepackage{fancyhdr}
\usepackage{titlesec}
\usepackage[spanish]{babel}
\usepackage[utf8]{inputenc}
\usepackage{amsmath}
\usepackage{setspace}
\usepackage{svg}
\usepackage{hyperref}
\usepackage[backend=biber,style=numeric]{biblatex}
\addbibresource{references.bib}
\hypersetup{
    colorlinks=true,
    linkcolor=blue,      % color of internal links (sections, etc.)
    urlcolor=blue,       % color of external links
    pdftitle={Optimización energética de sistema híbrido con bomba de calor, suelo radiante, fotovoltaica y almacenamiento para vivienda},    % title
    pdfauthor={Luis D. Aranda Sánchez},     % author
    pdfkeywords={palabra1, palabra2, código1, etc.} % list of keywords
}

% Font change to Arial
\usepackage{helvet}
\renewcommand{\familydefault}{\sfdefault}

% Chapter titles in uppercase and larger font
\titleformat{\chapter}[hang]{\large\bfseries}{\thechapter.}{1em}{\MakeUppercase}
\titleformat{\section}[hang]{\bfseries}{\thesection.}{1em}{}
\titleformat{\subsection}[hang]{\bfseries}{\thesubsection.}{1em}{}

% Fancyhdr setup
\setlength{\headheight}{14.30174pt} % Adjust to recommended value, slightly larger for safety
\fancyhf{} % Clear all headers and footers
\fancyhead[LE]{\nouppercase{\leftmark}}
\fancyhead[RO]{Optimización energética para vivienda}
\fancyfoot[LE]{\thepage}
\fancyfoot[RE]{Escuela Técnica Superior de Ingenieros Industriales (UPM)}
\fancyfoot[LO]{Luis D. Aranda Sánchez}
\fancyfoot[RO]{\thepage}
\renewcommand{\headrulewidth}{0.4pt}
\renewcommand{\footrulewidth}{0.4pt}

\fancypagestyle{myfancy}{
    \fancyhf{} % Clear all headers and footers
    \fancyhead[LE]{\nouppercase{\leftmark}}
    \fancyhead[RO]{Optimización energética para vivienda}
    \fancyfoot[LE]{\thepage}
    \fancyfoot[RE]{Escuela Técnica Superior de Ingenieros Industriales (UPM)}
    \fancyfoot[LO]{Luis D. Aranda Sánchez}
    \fancyfoot[RO]{\thepage}
    \renewcommand{\headrulewidth}{0.4pt}
    \renewcommand{\footrulewidth}{0.4pt}
}

\fancypagestyle{simple}{
    \fancyhf{} % Clear all headers and footers
    \renewcommand{\headrulewidth}{0pt}
    \renewcommand{\footrulewidth}{0pt}
}

% Line spacing
\setstretch{1.2}

% Document starts here
\begin{document}

% Portada
\begin{titlepage}
    \centering
    {\scshape\LARGE Universidad Politécnica de Madrid \par}
    \vspace{1cm}
    {\scshape\Large Escuela Técnica Superior de Ingenieros Industriales\par}
    \vspace{1.5cm}
    {\huge\bfseries Optimización energética de sistema híbrido con bomba de calor, suelo radiante, fotovoltaica y almacenamiento para vivienda \par}
    \vspace{1.5cm}
    {\Large\bfseries Trabajo de Fin de Máster\par}
    \vspace{0.5cm}
    {\large Máster Universitario en Ingeniería de la Energía \par}
    \vspace{2cm}
    {\Large Luis D. Aranda Sánchez\par}
    \vfill
    Director: Javier Rodríguez Martín
    \vfill
    {\large Septiembre 6, 2024\par}
\end{titlepage}

% Resumen (máximo de 5 páginas, incluyendo al final Palabras clave)
\clearpage
\pagestyle{simple}
% \newpage
\chapter*{Resumen}
\addcontentsline{toc}{chapter}{Resumen}
\input{capitulos/resumen/main.tex}

% Índice (paginado)
\clearpage
\pagestyle{simple}
% \newpage
\tableofcontents

% Introducción (donde se incluya los antecedentes y justificación)
\clearpage
\pagestyle{myfancy}
\newpage
\chapter{Introducción}
\input{capitulos/introduccion/main.tex}

% Objetivos
\chapter{Objetivos}
\input{capitulos/objetivos/main.tex}

% Metodología
\chapter{Metodología}
\input{capitulos/metodologia/main.tex}

% Resultados y discusión (incluyendo la valoración de impactos y de aspectos de responsabilidad legal, ética y profesional relacionados con el trabajo)
\chapter{Resultados y Discusión}
\input{capitulos/resultados_discusion/main.tex}

% Conclusiones
\chapter{Conclusiones}
\input{capitulos/conclusiones/main.tex}

% Planificación temporal y presupuesto
\chapter{Planificación Temporal y Presupuesto}
\input{capitulos/planificacion_presupuesto/main.tex}

% Bibliografía
\newpage
\addcontentsline{toc}{chapter}{Bibliografía}
\printbibliography

\end{document}


% Bibliografía
\newpage
\addcontentsline{toc}{chapter}{Bibliografía}
\printbibliography

\end{document}


% Resultados y discusión (incluyendo la valoración de impactos y de aspectos de responsabilidad legal, ética y profesional relacionados con el trabajo)
\chapter{Resultados y Discusión}
\documentclass[a4paper,11pt,twoside]{report}
\usepackage[left=25mm,right=25mm,top=25mm,bottom=25mm,includehead,includefoot,headsep=15mm,footskip=15mm]{geometry}
\usepackage{graphicx}
\usepackage{fancyhdr}
\usepackage{titlesec}
\usepackage[spanish]{babel}
\usepackage[utf8]{inputenc}
\usepackage{amsmath}
\usepackage{setspace}
\usepackage{svg}
\usepackage{hyperref}
\usepackage[backend=biber,style=numeric]{biblatex}
\addbibresource{references.bib}
\hypersetup{
    colorlinks=true,
    linkcolor=blue,      % color of internal links (sections, etc.)
    urlcolor=blue,       % color of external links
    pdftitle={Optimización energética de sistema híbrido con bomba de calor, suelo radiante, fotovoltaica y almacenamiento para vivienda},    % title
    pdfauthor={Luis D. Aranda Sánchez},     % author
    pdfkeywords={palabra1, palabra2, código1, etc.} % list of keywords
}

% Font change to Arial
\usepackage{helvet}
\renewcommand{\familydefault}{\sfdefault}

% Chapter titles in uppercase and larger font
\titleformat{\chapter}[hang]{\large\bfseries}{\thechapter.}{1em}{\MakeUppercase}
\titleformat{\section}[hang]{\bfseries}{\thesection.}{1em}{}
\titleformat{\subsection}[hang]{\bfseries}{\thesubsection.}{1em}{}

% Fancyhdr setup
\setlength{\headheight}{14.30174pt} % Adjust to recommended value, slightly larger for safety
\fancyhf{} % Clear all headers and footers
\fancyhead[LE]{\nouppercase{\leftmark}}
\fancyhead[RO]{Optimización energética para vivienda}
\fancyfoot[LE]{\thepage}
\fancyfoot[RE]{Escuela Técnica Superior de Ingenieros Industriales (UPM)}
\fancyfoot[LO]{Luis D. Aranda Sánchez}
\fancyfoot[RO]{\thepage}
\renewcommand{\headrulewidth}{0.4pt}
\renewcommand{\footrulewidth}{0.4pt}

\fancypagestyle{myfancy}{
    \fancyhf{} % Clear all headers and footers
    \fancyhead[LE]{\nouppercase{\leftmark}}
    \fancyhead[RO]{Optimización energética para vivienda}
    \fancyfoot[LE]{\thepage}
    \fancyfoot[RE]{Escuela Técnica Superior de Ingenieros Industriales (UPM)}
    \fancyfoot[LO]{Luis D. Aranda Sánchez}
    \fancyfoot[RO]{\thepage}
    \renewcommand{\headrulewidth}{0.4pt}
    \renewcommand{\footrulewidth}{0.4pt}
}

\fancypagestyle{simple}{
    \fancyhf{} % Clear all headers and footers
    \renewcommand{\headrulewidth}{0pt}
    \renewcommand{\footrulewidth}{0pt}
}

% Line spacing
\setstretch{1.2}

% Document starts here
\begin{document}

% Portada
\begin{titlepage}
    \centering
    {\scshape\LARGE Universidad Politécnica de Madrid \par}
    \vspace{1cm}
    {\scshape\Large Escuela Técnica Superior de Ingenieros Industriales\par}
    \vspace{1.5cm}
    {\huge\bfseries Optimización energética de sistema híbrido con bomba de calor, suelo radiante, fotovoltaica y almacenamiento para vivienda \par}
    \vspace{1.5cm}
    {\Large\bfseries Trabajo de Fin de Máster\par}
    \vspace{0.5cm}
    {\large Máster Universitario en Ingeniería de la Energía \par}
    \vspace{2cm}
    {\Large Luis D. Aranda Sánchez\par}
    \vfill
    Director: Javier Rodríguez Martín
    \vfill
    {\large Septiembre 6, 2024\par}
\end{titlepage}

% Resumen (máximo de 5 páginas, incluyendo al final Palabras clave)
\clearpage
\pagestyle{simple}
% \newpage
\chapter*{Resumen}
\addcontentsline{toc}{chapter}{Resumen}
\documentclass[a4paper,11pt,twoside]{report}
\usepackage[left=25mm,right=25mm,top=25mm,bottom=25mm,includehead,includefoot,headsep=15mm,footskip=15mm]{geometry}
\usepackage{graphicx}
\usepackage{fancyhdr}
\usepackage{titlesec}
\usepackage[spanish]{babel}
\usepackage[utf8]{inputenc}
\usepackage{amsmath}
\usepackage{setspace}
\usepackage{svg}
\usepackage{hyperref}
\usepackage[backend=biber,style=numeric]{biblatex}
\addbibresource{references.bib}
\hypersetup{
    colorlinks=true,
    linkcolor=blue,      % color of internal links (sections, etc.)
    urlcolor=blue,       % color of external links
    pdftitle={Optimización energética de sistema híbrido con bomba de calor, suelo radiante, fotovoltaica y almacenamiento para vivienda},    % title
    pdfauthor={Luis D. Aranda Sánchez},     % author
    pdfkeywords={palabra1, palabra2, código1, etc.} % list of keywords
}

% Font change to Arial
\usepackage{helvet}
\renewcommand{\familydefault}{\sfdefault}

% Chapter titles in uppercase and larger font
\titleformat{\chapter}[hang]{\large\bfseries}{\thechapter.}{1em}{\MakeUppercase}
\titleformat{\section}[hang]{\bfseries}{\thesection.}{1em}{}
\titleformat{\subsection}[hang]{\bfseries}{\thesubsection.}{1em}{}

% Fancyhdr setup
\setlength{\headheight}{14.30174pt} % Adjust to recommended value, slightly larger for safety
\fancyhf{} % Clear all headers and footers
\fancyhead[LE]{\nouppercase{\leftmark}}
\fancyhead[RO]{Optimización energética para vivienda}
\fancyfoot[LE]{\thepage}
\fancyfoot[RE]{Escuela Técnica Superior de Ingenieros Industriales (UPM)}
\fancyfoot[LO]{Luis D. Aranda Sánchez}
\fancyfoot[RO]{\thepage}
\renewcommand{\headrulewidth}{0.4pt}
\renewcommand{\footrulewidth}{0.4pt}

\fancypagestyle{myfancy}{
    \fancyhf{} % Clear all headers and footers
    \fancyhead[LE]{\nouppercase{\leftmark}}
    \fancyhead[RO]{Optimización energética para vivienda}
    \fancyfoot[LE]{\thepage}
    \fancyfoot[RE]{Escuela Técnica Superior de Ingenieros Industriales (UPM)}
    \fancyfoot[LO]{Luis D. Aranda Sánchez}
    \fancyfoot[RO]{\thepage}
    \renewcommand{\headrulewidth}{0.4pt}
    \renewcommand{\footrulewidth}{0.4pt}
}

\fancypagestyle{simple}{
    \fancyhf{} % Clear all headers and footers
    \renewcommand{\headrulewidth}{0pt}
    \renewcommand{\footrulewidth}{0pt}
}

% Line spacing
\setstretch{1.2}

% Document starts here
\begin{document}

% Portada
\begin{titlepage}
    \centering
    {\scshape\LARGE Universidad Politécnica de Madrid \par}
    \vspace{1cm}
    {\scshape\Large Escuela Técnica Superior de Ingenieros Industriales\par}
    \vspace{1.5cm}
    {\huge\bfseries Optimización energética de sistema híbrido con bomba de calor, suelo radiante, fotovoltaica y almacenamiento para vivienda \par}
    \vspace{1.5cm}
    {\Large\bfseries Trabajo de Fin de Máster\par}
    \vspace{0.5cm}
    {\large Máster Universitario en Ingeniería de la Energía \par}
    \vspace{2cm}
    {\Large Luis D. Aranda Sánchez\par}
    \vfill
    Director: Javier Rodríguez Martín
    \vfill
    {\large Septiembre 6, 2024\par}
\end{titlepage}

% Resumen (máximo de 5 páginas, incluyendo al final Palabras clave)
\clearpage
\pagestyle{simple}
% \newpage
\chapter*{Resumen}
\addcontentsline{toc}{chapter}{Resumen}
\input{capitulos/resumen/main.tex}

% Índice (paginado)
\clearpage
\pagestyle{simple}
% \newpage
\tableofcontents

% Introducción (donde se incluya los antecedentes y justificación)
\clearpage
\pagestyle{myfancy}
\newpage
\chapter{Introducción}
\input{capitulos/introduccion/main.tex}

% Objetivos
\chapter{Objetivos}
\input{capitulos/objetivos/main.tex}

% Metodología
\chapter{Metodología}
\input{capitulos/metodologia/main.tex}

% Resultados y discusión (incluyendo la valoración de impactos y de aspectos de responsabilidad legal, ética y profesional relacionados con el trabajo)
\chapter{Resultados y Discusión}
\input{capitulos/resultados_discusion/main.tex}

% Conclusiones
\chapter{Conclusiones}
\input{capitulos/conclusiones/main.tex}

% Planificación temporal y presupuesto
\chapter{Planificación Temporal y Presupuesto}
\input{capitulos/planificacion_presupuesto/main.tex}

% Bibliografía
\newpage
\addcontentsline{toc}{chapter}{Bibliografía}
\printbibliography

\end{document}


% Índice (paginado)
\clearpage
\pagestyle{simple}
% \newpage
\tableofcontents

% Introducción (donde se incluya los antecedentes y justificación)
\clearpage
\pagestyle{myfancy}
\newpage
\chapter{Introducción}
\documentclass[a4paper,11pt,twoside]{report}
\usepackage[left=25mm,right=25mm,top=25mm,bottom=25mm,includehead,includefoot,headsep=15mm,footskip=15mm]{geometry}
\usepackage{graphicx}
\usepackage{fancyhdr}
\usepackage{titlesec}
\usepackage[spanish]{babel}
\usepackage[utf8]{inputenc}
\usepackage{amsmath}
\usepackage{setspace}
\usepackage{svg}
\usepackage{hyperref}
\usepackage[backend=biber,style=numeric]{biblatex}
\addbibresource{references.bib}
\hypersetup{
    colorlinks=true,
    linkcolor=blue,      % color of internal links (sections, etc.)
    urlcolor=blue,       % color of external links
    pdftitle={Optimización energética de sistema híbrido con bomba de calor, suelo radiante, fotovoltaica y almacenamiento para vivienda},    % title
    pdfauthor={Luis D. Aranda Sánchez},     % author
    pdfkeywords={palabra1, palabra2, código1, etc.} % list of keywords
}

% Font change to Arial
\usepackage{helvet}
\renewcommand{\familydefault}{\sfdefault}

% Chapter titles in uppercase and larger font
\titleformat{\chapter}[hang]{\large\bfseries}{\thechapter.}{1em}{\MakeUppercase}
\titleformat{\section}[hang]{\bfseries}{\thesection.}{1em}{}
\titleformat{\subsection}[hang]{\bfseries}{\thesubsection.}{1em}{}

% Fancyhdr setup
\setlength{\headheight}{14.30174pt} % Adjust to recommended value, slightly larger for safety
\fancyhf{} % Clear all headers and footers
\fancyhead[LE]{\nouppercase{\leftmark}}
\fancyhead[RO]{Optimización energética para vivienda}
\fancyfoot[LE]{\thepage}
\fancyfoot[RE]{Escuela Técnica Superior de Ingenieros Industriales (UPM)}
\fancyfoot[LO]{Luis D. Aranda Sánchez}
\fancyfoot[RO]{\thepage}
\renewcommand{\headrulewidth}{0.4pt}
\renewcommand{\footrulewidth}{0.4pt}

\fancypagestyle{myfancy}{
    \fancyhf{} % Clear all headers and footers
    \fancyhead[LE]{\nouppercase{\leftmark}}
    \fancyhead[RO]{Optimización energética para vivienda}
    \fancyfoot[LE]{\thepage}
    \fancyfoot[RE]{Escuela Técnica Superior de Ingenieros Industriales (UPM)}
    \fancyfoot[LO]{Luis D. Aranda Sánchez}
    \fancyfoot[RO]{\thepage}
    \renewcommand{\headrulewidth}{0.4pt}
    \renewcommand{\footrulewidth}{0.4pt}
}

\fancypagestyle{simple}{
    \fancyhf{} % Clear all headers and footers
    \renewcommand{\headrulewidth}{0pt}
    \renewcommand{\footrulewidth}{0pt}
}

% Line spacing
\setstretch{1.2}

% Document starts here
\begin{document}

% Portada
\begin{titlepage}
    \centering
    {\scshape\LARGE Universidad Politécnica de Madrid \par}
    \vspace{1cm}
    {\scshape\Large Escuela Técnica Superior de Ingenieros Industriales\par}
    \vspace{1.5cm}
    {\huge\bfseries Optimización energética de sistema híbrido con bomba de calor, suelo radiante, fotovoltaica y almacenamiento para vivienda \par}
    \vspace{1.5cm}
    {\Large\bfseries Trabajo de Fin de Máster\par}
    \vspace{0.5cm}
    {\large Máster Universitario en Ingeniería de la Energía \par}
    \vspace{2cm}
    {\Large Luis D. Aranda Sánchez\par}
    \vfill
    Director: Javier Rodríguez Martín
    \vfill
    {\large Septiembre 6, 2024\par}
\end{titlepage}

% Resumen (máximo de 5 páginas, incluyendo al final Palabras clave)
\clearpage
\pagestyle{simple}
% \newpage
\chapter*{Resumen}
\addcontentsline{toc}{chapter}{Resumen}
\input{capitulos/resumen/main.tex}

% Índice (paginado)
\clearpage
\pagestyle{simple}
% \newpage
\tableofcontents

% Introducción (donde se incluya los antecedentes y justificación)
\clearpage
\pagestyle{myfancy}
\newpage
\chapter{Introducción}
\input{capitulos/introduccion/main.tex}

% Objetivos
\chapter{Objetivos}
\input{capitulos/objetivos/main.tex}

% Metodología
\chapter{Metodología}
\input{capitulos/metodologia/main.tex}

% Resultados y discusión (incluyendo la valoración de impactos y de aspectos de responsabilidad legal, ética y profesional relacionados con el trabajo)
\chapter{Resultados y Discusión}
\input{capitulos/resultados_discusion/main.tex}

% Conclusiones
\chapter{Conclusiones}
\input{capitulos/conclusiones/main.tex}

% Planificación temporal y presupuesto
\chapter{Planificación Temporal y Presupuesto}
\input{capitulos/planificacion_presupuesto/main.tex}

% Bibliografía
\newpage
\addcontentsline{toc}{chapter}{Bibliografía}
\printbibliography

\end{document}


% Objetivos
\chapter{Objetivos}
\documentclass[a4paper,11pt,twoside]{report}
\usepackage[left=25mm,right=25mm,top=25mm,bottom=25mm,includehead,includefoot,headsep=15mm,footskip=15mm]{geometry}
\usepackage{graphicx}
\usepackage{fancyhdr}
\usepackage{titlesec}
\usepackage[spanish]{babel}
\usepackage[utf8]{inputenc}
\usepackage{amsmath}
\usepackage{setspace}
\usepackage{svg}
\usepackage{hyperref}
\usepackage[backend=biber,style=numeric]{biblatex}
\addbibresource{references.bib}
\hypersetup{
    colorlinks=true,
    linkcolor=blue,      % color of internal links (sections, etc.)
    urlcolor=blue,       % color of external links
    pdftitle={Optimización energética de sistema híbrido con bomba de calor, suelo radiante, fotovoltaica y almacenamiento para vivienda},    % title
    pdfauthor={Luis D. Aranda Sánchez},     % author
    pdfkeywords={palabra1, palabra2, código1, etc.} % list of keywords
}

% Font change to Arial
\usepackage{helvet}
\renewcommand{\familydefault}{\sfdefault}

% Chapter titles in uppercase and larger font
\titleformat{\chapter}[hang]{\large\bfseries}{\thechapter.}{1em}{\MakeUppercase}
\titleformat{\section}[hang]{\bfseries}{\thesection.}{1em}{}
\titleformat{\subsection}[hang]{\bfseries}{\thesubsection.}{1em}{}

% Fancyhdr setup
\setlength{\headheight}{14.30174pt} % Adjust to recommended value, slightly larger for safety
\fancyhf{} % Clear all headers and footers
\fancyhead[LE]{\nouppercase{\leftmark}}
\fancyhead[RO]{Optimización energética para vivienda}
\fancyfoot[LE]{\thepage}
\fancyfoot[RE]{Escuela Técnica Superior de Ingenieros Industriales (UPM)}
\fancyfoot[LO]{Luis D. Aranda Sánchez}
\fancyfoot[RO]{\thepage}
\renewcommand{\headrulewidth}{0.4pt}
\renewcommand{\footrulewidth}{0.4pt}

\fancypagestyle{myfancy}{
    \fancyhf{} % Clear all headers and footers
    \fancyhead[LE]{\nouppercase{\leftmark}}
    \fancyhead[RO]{Optimización energética para vivienda}
    \fancyfoot[LE]{\thepage}
    \fancyfoot[RE]{Escuela Técnica Superior de Ingenieros Industriales (UPM)}
    \fancyfoot[LO]{Luis D. Aranda Sánchez}
    \fancyfoot[RO]{\thepage}
    \renewcommand{\headrulewidth}{0.4pt}
    \renewcommand{\footrulewidth}{0.4pt}
}

\fancypagestyle{simple}{
    \fancyhf{} % Clear all headers and footers
    \renewcommand{\headrulewidth}{0pt}
    \renewcommand{\footrulewidth}{0pt}
}

% Line spacing
\setstretch{1.2}

% Document starts here
\begin{document}

% Portada
\begin{titlepage}
    \centering
    {\scshape\LARGE Universidad Politécnica de Madrid \par}
    \vspace{1cm}
    {\scshape\Large Escuela Técnica Superior de Ingenieros Industriales\par}
    \vspace{1.5cm}
    {\huge\bfseries Optimización energética de sistema híbrido con bomba de calor, suelo radiante, fotovoltaica y almacenamiento para vivienda \par}
    \vspace{1.5cm}
    {\Large\bfseries Trabajo de Fin de Máster\par}
    \vspace{0.5cm}
    {\large Máster Universitario en Ingeniería de la Energía \par}
    \vspace{2cm}
    {\Large Luis D. Aranda Sánchez\par}
    \vfill
    Director: Javier Rodríguez Martín
    \vfill
    {\large Septiembre 6, 2024\par}
\end{titlepage}

% Resumen (máximo de 5 páginas, incluyendo al final Palabras clave)
\clearpage
\pagestyle{simple}
% \newpage
\chapter*{Resumen}
\addcontentsline{toc}{chapter}{Resumen}
\input{capitulos/resumen/main.tex}

% Índice (paginado)
\clearpage
\pagestyle{simple}
% \newpage
\tableofcontents

% Introducción (donde se incluya los antecedentes y justificación)
\clearpage
\pagestyle{myfancy}
\newpage
\chapter{Introducción}
\input{capitulos/introduccion/main.tex}

% Objetivos
\chapter{Objetivos}
\input{capitulos/objetivos/main.tex}

% Metodología
\chapter{Metodología}
\input{capitulos/metodologia/main.tex}

% Resultados y discusión (incluyendo la valoración de impactos y de aspectos de responsabilidad legal, ética y profesional relacionados con el trabajo)
\chapter{Resultados y Discusión}
\input{capitulos/resultados_discusion/main.tex}

% Conclusiones
\chapter{Conclusiones}
\input{capitulos/conclusiones/main.tex}

% Planificación temporal y presupuesto
\chapter{Planificación Temporal y Presupuesto}
\input{capitulos/planificacion_presupuesto/main.tex}

% Bibliografía
\newpage
\addcontentsline{toc}{chapter}{Bibliografía}
\printbibliography

\end{document}


% Metodología
\chapter{Metodología}
\documentclass[a4paper,11pt,twoside]{report}
\usepackage[left=25mm,right=25mm,top=25mm,bottom=25mm,includehead,includefoot,headsep=15mm,footskip=15mm]{geometry}
\usepackage{graphicx}
\usepackage{fancyhdr}
\usepackage{titlesec}
\usepackage[spanish]{babel}
\usepackage[utf8]{inputenc}
\usepackage{amsmath}
\usepackage{setspace}
\usepackage{svg}
\usepackage{hyperref}
\usepackage[backend=biber,style=numeric]{biblatex}
\addbibresource{references.bib}
\hypersetup{
    colorlinks=true,
    linkcolor=blue,      % color of internal links (sections, etc.)
    urlcolor=blue,       % color of external links
    pdftitle={Optimización energética de sistema híbrido con bomba de calor, suelo radiante, fotovoltaica y almacenamiento para vivienda},    % title
    pdfauthor={Luis D. Aranda Sánchez},     % author
    pdfkeywords={palabra1, palabra2, código1, etc.} % list of keywords
}

% Font change to Arial
\usepackage{helvet}
\renewcommand{\familydefault}{\sfdefault}

% Chapter titles in uppercase and larger font
\titleformat{\chapter}[hang]{\large\bfseries}{\thechapter.}{1em}{\MakeUppercase}
\titleformat{\section}[hang]{\bfseries}{\thesection.}{1em}{}
\titleformat{\subsection}[hang]{\bfseries}{\thesubsection.}{1em}{}

% Fancyhdr setup
\setlength{\headheight}{14.30174pt} % Adjust to recommended value, slightly larger for safety
\fancyhf{} % Clear all headers and footers
\fancyhead[LE]{\nouppercase{\leftmark}}
\fancyhead[RO]{Optimización energética para vivienda}
\fancyfoot[LE]{\thepage}
\fancyfoot[RE]{Escuela Técnica Superior de Ingenieros Industriales (UPM)}
\fancyfoot[LO]{Luis D. Aranda Sánchez}
\fancyfoot[RO]{\thepage}
\renewcommand{\headrulewidth}{0.4pt}
\renewcommand{\footrulewidth}{0.4pt}

\fancypagestyle{myfancy}{
    \fancyhf{} % Clear all headers and footers
    \fancyhead[LE]{\nouppercase{\leftmark}}
    \fancyhead[RO]{Optimización energética para vivienda}
    \fancyfoot[LE]{\thepage}
    \fancyfoot[RE]{Escuela Técnica Superior de Ingenieros Industriales (UPM)}
    \fancyfoot[LO]{Luis D. Aranda Sánchez}
    \fancyfoot[RO]{\thepage}
    \renewcommand{\headrulewidth}{0.4pt}
    \renewcommand{\footrulewidth}{0.4pt}
}

\fancypagestyle{simple}{
    \fancyhf{} % Clear all headers and footers
    \renewcommand{\headrulewidth}{0pt}
    \renewcommand{\footrulewidth}{0pt}
}

% Line spacing
\setstretch{1.2}

% Document starts here
\begin{document}

% Portada
\begin{titlepage}
    \centering
    {\scshape\LARGE Universidad Politécnica de Madrid \par}
    \vspace{1cm}
    {\scshape\Large Escuela Técnica Superior de Ingenieros Industriales\par}
    \vspace{1.5cm}
    {\huge\bfseries Optimización energética de sistema híbrido con bomba de calor, suelo radiante, fotovoltaica y almacenamiento para vivienda \par}
    \vspace{1.5cm}
    {\Large\bfseries Trabajo de Fin de Máster\par}
    \vspace{0.5cm}
    {\large Máster Universitario en Ingeniería de la Energía \par}
    \vspace{2cm}
    {\Large Luis D. Aranda Sánchez\par}
    \vfill
    Director: Javier Rodríguez Martín
    \vfill
    {\large Septiembre 6, 2024\par}
\end{titlepage}

% Resumen (máximo de 5 páginas, incluyendo al final Palabras clave)
\clearpage
\pagestyle{simple}
% \newpage
\chapter*{Resumen}
\addcontentsline{toc}{chapter}{Resumen}
\input{capitulos/resumen/main.tex}

% Índice (paginado)
\clearpage
\pagestyle{simple}
% \newpage
\tableofcontents

% Introducción (donde se incluya los antecedentes y justificación)
\clearpage
\pagestyle{myfancy}
\newpage
\chapter{Introducción}
\input{capitulos/introduccion/main.tex}

% Objetivos
\chapter{Objetivos}
\input{capitulos/objetivos/main.tex}

% Metodología
\chapter{Metodología}
\input{capitulos/metodologia/main.tex}

% Resultados y discusión (incluyendo la valoración de impactos y de aspectos de responsabilidad legal, ética y profesional relacionados con el trabajo)
\chapter{Resultados y Discusión}
\input{capitulos/resultados_discusion/main.tex}

% Conclusiones
\chapter{Conclusiones}
\input{capitulos/conclusiones/main.tex}

% Planificación temporal y presupuesto
\chapter{Planificación Temporal y Presupuesto}
\input{capitulos/planificacion_presupuesto/main.tex}

% Bibliografía
\newpage
\addcontentsline{toc}{chapter}{Bibliografía}
\printbibliography

\end{document}


% Resultados y discusión (incluyendo la valoración de impactos y de aspectos de responsabilidad legal, ética y profesional relacionados con el trabajo)
\chapter{Resultados y Discusión}
\documentclass[a4paper,11pt,twoside]{report}
\usepackage[left=25mm,right=25mm,top=25mm,bottom=25mm,includehead,includefoot,headsep=15mm,footskip=15mm]{geometry}
\usepackage{graphicx}
\usepackage{fancyhdr}
\usepackage{titlesec}
\usepackage[spanish]{babel}
\usepackage[utf8]{inputenc}
\usepackage{amsmath}
\usepackage{setspace}
\usepackage{svg}
\usepackage{hyperref}
\usepackage[backend=biber,style=numeric]{biblatex}
\addbibresource{references.bib}
\hypersetup{
    colorlinks=true,
    linkcolor=blue,      % color of internal links (sections, etc.)
    urlcolor=blue,       % color of external links
    pdftitle={Optimización energética de sistema híbrido con bomba de calor, suelo radiante, fotovoltaica y almacenamiento para vivienda},    % title
    pdfauthor={Luis D. Aranda Sánchez},     % author
    pdfkeywords={palabra1, palabra2, código1, etc.} % list of keywords
}

% Font change to Arial
\usepackage{helvet}
\renewcommand{\familydefault}{\sfdefault}

% Chapter titles in uppercase and larger font
\titleformat{\chapter}[hang]{\large\bfseries}{\thechapter.}{1em}{\MakeUppercase}
\titleformat{\section}[hang]{\bfseries}{\thesection.}{1em}{}
\titleformat{\subsection}[hang]{\bfseries}{\thesubsection.}{1em}{}

% Fancyhdr setup
\setlength{\headheight}{14.30174pt} % Adjust to recommended value, slightly larger for safety
\fancyhf{} % Clear all headers and footers
\fancyhead[LE]{\nouppercase{\leftmark}}
\fancyhead[RO]{Optimización energética para vivienda}
\fancyfoot[LE]{\thepage}
\fancyfoot[RE]{Escuela Técnica Superior de Ingenieros Industriales (UPM)}
\fancyfoot[LO]{Luis D. Aranda Sánchez}
\fancyfoot[RO]{\thepage}
\renewcommand{\headrulewidth}{0.4pt}
\renewcommand{\footrulewidth}{0.4pt}

\fancypagestyle{myfancy}{
    \fancyhf{} % Clear all headers and footers
    \fancyhead[LE]{\nouppercase{\leftmark}}
    \fancyhead[RO]{Optimización energética para vivienda}
    \fancyfoot[LE]{\thepage}
    \fancyfoot[RE]{Escuela Técnica Superior de Ingenieros Industriales (UPM)}
    \fancyfoot[LO]{Luis D. Aranda Sánchez}
    \fancyfoot[RO]{\thepage}
    \renewcommand{\headrulewidth}{0.4pt}
    \renewcommand{\footrulewidth}{0.4pt}
}

\fancypagestyle{simple}{
    \fancyhf{} % Clear all headers and footers
    \renewcommand{\headrulewidth}{0pt}
    \renewcommand{\footrulewidth}{0pt}
}

% Line spacing
\setstretch{1.2}

% Document starts here
\begin{document}

% Portada
\begin{titlepage}
    \centering
    {\scshape\LARGE Universidad Politécnica de Madrid \par}
    \vspace{1cm}
    {\scshape\Large Escuela Técnica Superior de Ingenieros Industriales\par}
    \vspace{1.5cm}
    {\huge\bfseries Optimización energética de sistema híbrido con bomba de calor, suelo radiante, fotovoltaica y almacenamiento para vivienda \par}
    \vspace{1.5cm}
    {\Large\bfseries Trabajo de Fin de Máster\par}
    \vspace{0.5cm}
    {\large Máster Universitario en Ingeniería de la Energía \par}
    \vspace{2cm}
    {\Large Luis D. Aranda Sánchez\par}
    \vfill
    Director: Javier Rodríguez Martín
    \vfill
    {\large Septiembre 6, 2024\par}
\end{titlepage}

% Resumen (máximo de 5 páginas, incluyendo al final Palabras clave)
\clearpage
\pagestyle{simple}
% \newpage
\chapter*{Resumen}
\addcontentsline{toc}{chapter}{Resumen}
\input{capitulos/resumen/main.tex}

% Índice (paginado)
\clearpage
\pagestyle{simple}
% \newpage
\tableofcontents

% Introducción (donde se incluya los antecedentes y justificación)
\clearpage
\pagestyle{myfancy}
\newpage
\chapter{Introducción}
\input{capitulos/introduccion/main.tex}

% Objetivos
\chapter{Objetivos}
\input{capitulos/objetivos/main.tex}

% Metodología
\chapter{Metodología}
\input{capitulos/metodologia/main.tex}

% Resultados y discusión (incluyendo la valoración de impactos y de aspectos de responsabilidad legal, ética y profesional relacionados con el trabajo)
\chapter{Resultados y Discusión}
\input{capitulos/resultados_discusion/main.tex}

% Conclusiones
\chapter{Conclusiones}
\input{capitulos/conclusiones/main.tex}

% Planificación temporal y presupuesto
\chapter{Planificación Temporal y Presupuesto}
\input{capitulos/planificacion_presupuesto/main.tex}

% Bibliografía
\newpage
\addcontentsline{toc}{chapter}{Bibliografía}
\printbibliography

\end{document}


% Conclusiones
\chapter{Conclusiones}
\documentclass[a4paper,11pt,twoside]{report}
\usepackage[left=25mm,right=25mm,top=25mm,bottom=25mm,includehead,includefoot,headsep=15mm,footskip=15mm]{geometry}
\usepackage{graphicx}
\usepackage{fancyhdr}
\usepackage{titlesec}
\usepackage[spanish]{babel}
\usepackage[utf8]{inputenc}
\usepackage{amsmath}
\usepackage{setspace}
\usepackage{svg}
\usepackage{hyperref}
\usepackage[backend=biber,style=numeric]{biblatex}
\addbibresource{references.bib}
\hypersetup{
    colorlinks=true,
    linkcolor=blue,      % color of internal links (sections, etc.)
    urlcolor=blue,       % color of external links
    pdftitle={Optimización energética de sistema híbrido con bomba de calor, suelo radiante, fotovoltaica y almacenamiento para vivienda},    % title
    pdfauthor={Luis D. Aranda Sánchez},     % author
    pdfkeywords={palabra1, palabra2, código1, etc.} % list of keywords
}

% Font change to Arial
\usepackage{helvet}
\renewcommand{\familydefault}{\sfdefault}

% Chapter titles in uppercase and larger font
\titleformat{\chapter}[hang]{\large\bfseries}{\thechapter.}{1em}{\MakeUppercase}
\titleformat{\section}[hang]{\bfseries}{\thesection.}{1em}{}
\titleformat{\subsection}[hang]{\bfseries}{\thesubsection.}{1em}{}

% Fancyhdr setup
\setlength{\headheight}{14.30174pt} % Adjust to recommended value, slightly larger for safety
\fancyhf{} % Clear all headers and footers
\fancyhead[LE]{\nouppercase{\leftmark}}
\fancyhead[RO]{Optimización energética para vivienda}
\fancyfoot[LE]{\thepage}
\fancyfoot[RE]{Escuela Técnica Superior de Ingenieros Industriales (UPM)}
\fancyfoot[LO]{Luis D. Aranda Sánchez}
\fancyfoot[RO]{\thepage}
\renewcommand{\headrulewidth}{0.4pt}
\renewcommand{\footrulewidth}{0.4pt}

\fancypagestyle{myfancy}{
    \fancyhf{} % Clear all headers and footers
    \fancyhead[LE]{\nouppercase{\leftmark}}
    \fancyhead[RO]{Optimización energética para vivienda}
    \fancyfoot[LE]{\thepage}
    \fancyfoot[RE]{Escuela Técnica Superior de Ingenieros Industriales (UPM)}
    \fancyfoot[LO]{Luis D. Aranda Sánchez}
    \fancyfoot[RO]{\thepage}
    \renewcommand{\headrulewidth}{0.4pt}
    \renewcommand{\footrulewidth}{0.4pt}
}

\fancypagestyle{simple}{
    \fancyhf{} % Clear all headers and footers
    \renewcommand{\headrulewidth}{0pt}
    \renewcommand{\footrulewidth}{0pt}
}

% Line spacing
\setstretch{1.2}

% Document starts here
\begin{document}

% Portada
\begin{titlepage}
    \centering
    {\scshape\LARGE Universidad Politécnica de Madrid \par}
    \vspace{1cm}
    {\scshape\Large Escuela Técnica Superior de Ingenieros Industriales\par}
    \vspace{1.5cm}
    {\huge\bfseries Optimización energética de sistema híbrido con bomba de calor, suelo radiante, fotovoltaica y almacenamiento para vivienda \par}
    \vspace{1.5cm}
    {\Large\bfseries Trabajo de Fin de Máster\par}
    \vspace{0.5cm}
    {\large Máster Universitario en Ingeniería de la Energía \par}
    \vspace{2cm}
    {\Large Luis D. Aranda Sánchez\par}
    \vfill
    Director: Javier Rodríguez Martín
    \vfill
    {\large Septiembre 6, 2024\par}
\end{titlepage}

% Resumen (máximo de 5 páginas, incluyendo al final Palabras clave)
\clearpage
\pagestyle{simple}
% \newpage
\chapter*{Resumen}
\addcontentsline{toc}{chapter}{Resumen}
\input{capitulos/resumen/main.tex}

% Índice (paginado)
\clearpage
\pagestyle{simple}
% \newpage
\tableofcontents

% Introducción (donde se incluya los antecedentes y justificación)
\clearpage
\pagestyle{myfancy}
\newpage
\chapter{Introducción}
\input{capitulos/introduccion/main.tex}

% Objetivos
\chapter{Objetivos}
\input{capitulos/objetivos/main.tex}

% Metodología
\chapter{Metodología}
\input{capitulos/metodologia/main.tex}

% Resultados y discusión (incluyendo la valoración de impactos y de aspectos de responsabilidad legal, ética y profesional relacionados con el trabajo)
\chapter{Resultados y Discusión}
\input{capitulos/resultados_discusion/main.tex}

% Conclusiones
\chapter{Conclusiones}
\input{capitulos/conclusiones/main.tex}

% Planificación temporal y presupuesto
\chapter{Planificación Temporal y Presupuesto}
\input{capitulos/planificacion_presupuesto/main.tex}

% Bibliografía
\newpage
\addcontentsline{toc}{chapter}{Bibliografía}
\printbibliography

\end{document}


% Planificación temporal y presupuesto
\chapter{Planificación Temporal y Presupuesto}
\documentclass[a4paper,11pt,twoside]{report}
\usepackage[left=25mm,right=25mm,top=25mm,bottom=25mm,includehead,includefoot,headsep=15mm,footskip=15mm]{geometry}
\usepackage{graphicx}
\usepackage{fancyhdr}
\usepackage{titlesec}
\usepackage[spanish]{babel}
\usepackage[utf8]{inputenc}
\usepackage{amsmath}
\usepackage{setspace}
\usepackage{svg}
\usepackage{hyperref}
\usepackage[backend=biber,style=numeric]{biblatex}
\addbibresource{references.bib}
\hypersetup{
    colorlinks=true,
    linkcolor=blue,      % color of internal links (sections, etc.)
    urlcolor=blue,       % color of external links
    pdftitle={Optimización energética de sistema híbrido con bomba de calor, suelo radiante, fotovoltaica y almacenamiento para vivienda},    % title
    pdfauthor={Luis D. Aranda Sánchez},     % author
    pdfkeywords={palabra1, palabra2, código1, etc.} % list of keywords
}

% Font change to Arial
\usepackage{helvet}
\renewcommand{\familydefault}{\sfdefault}

% Chapter titles in uppercase and larger font
\titleformat{\chapter}[hang]{\large\bfseries}{\thechapter.}{1em}{\MakeUppercase}
\titleformat{\section}[hang]{\bfseries}{\thesection.}{1em}{}
\titleformat{\subsection}[hang]{\bfseries}{\thesubsection.}{1em}{}

% Fancyhdr setup
\setlength{\headheight}{14.30174pt} % Adjust to recommended value, slightly larger for safety
\fancyhf{} % Clear all headers and footers
\fancyhead[LE]{\nouppercase{\leftmark}}
\fancyhead[RO]{Optimización energética para vivienda}
\fancyfoot[LE]{\thepage}
\fancyfoot[RE]{Escuela Técnica Superior de Ingenieros Industriales (UPM)}
\fancyfoot[LO]{Luis D. Aranda Sánchez}
\fancyfoot[RO]{\thepage}
\renewcommand{\headrulewidth}{0.4pt}
\renewcommand{\footrulewidth}{0.4pt}

\fancypagestyle{myfancy}{
    \fancyhf{} % Clear all headers and footers
    \fancyhead[LE]{\nouppercase{\leftmark}}
    \fancyhead[RO]{Optimización energética para vivienda}
    \fancyfoot[LE]{\thepage}
    \fancyfoot[RE]{Escuela Técnica Superior de Ingenieros Industriales (UPM)}
    \fancyfoot[LO]{Luis D. Aranda Sánchez}
    \fancyfoot[RO]{\thepage}
    \renewcommand{\headrulewidth}{0.4pt}
    \renewcommand{\footrulewidth}{0.4pt}
}

\fancypagestyle{simple}{
    \fancyhf{} % Clear all headers and footers
    \renewcommand{\headrulewidth}{0pt}
    \renewcommand{\footrulewidth}{0pt}
}

% Line spacing
\setstretch{1.2}

% Document starts here
\begin{document}

% Portada
\begin{titlepage}
    \centering
    {\scshape\LARGE Universidad Politécnica de Madrid \par}
    \vspace{1cm}
    {\scshape\Large Escuela Técnica Superior de Ingenieros Industriales\par}
    \vspace{1.5cm}
    {\huge\bfseries Optimización energética de sistema híbrido con bomba de calor, suelo radiante, fotovoltaica y almacenamiento para vivienda \par}
    \vspace{1.5cm}
    {\Large\bfseries Trabajo de Fin de Máster\par}
    \vspace{0.5cm}
    {\large Máster Universitario en Ingeniería de la Energía \par}
    \vspace{2cm}
    {\Large Luis D. Aranda Sánchez\par}
    \vfill
    Director: Javier Rodríguez Martín
    \vfill
    {\large Septiembre 6, 2024\par}
\end{titlepage}

% Resumen (máximo de 5 páginas, incluyendo al final Palabras clave)
\clearpage
\pagestyle{simple}
% \newpage
\chapter*{Resumen}
\addcontentsline{toc}{chapter}{Resumen}
\input{capitulos/resumen/main.tex}

% Índice (paginado)
\clearpage
\pagestyle{simple}
% \newpage
\tableofcontents

% Introducción (donde se incluya los antecedentes y justificación)
\clearpage
\pagestyle{myfancy}
\newpage
\chapter{Introducción}
\input{capitulos/introduccion/main.tex}

% Objetivos
\chapter{Objetivos}
\input{capitulos/objetivos/main.tex}

% Metodología
\chapter{Metodología}
\input{capitulos/metodologia/main.tex}

% Resultados y discusión (incluyendo la valoración de impactos y de aspectos de responsabilidad legal, ética y profesional relacionados con el trabajo)
\chapter{Resultados y Discusión}
\input{capitulos/resultados_discusion/main.tex}

% Conclusiones
\chapter{Conclusiones}
\input{capitulos/conclusiones/main.tex}

% Planificación temporal y presupuesto
\chapter{Planificación Temporal y Presupuesto}
\input{capitulos/planificacion_presupuesto/main.tex}

% Bibliografía
\newpage
\addcontentsline{toc}{chapter}{Bibliografía}
\printbibliography

\end{document}


% Bibliografía
\newpage
\addcontentsline{toc}{chapter}{Bibliografía}
\printbibliography

\end{document}


% Conclusiones
\chapter{Conclusiones}
\documentclass[a4paper,11pt,twoside]{report}
\usepackage[left=25mm,right=25mm,top=25mm,bottom=25mm,includehead,includefoot,headsep=15mm,footskip=15mm]{geometry}
\usepackage{graphicx}
\usepackage{fancyhdr}
\usepackage{titlesec}
\usepackage[spanish]{babel}
\usepackage[utf8]{inputenc}
\usepackage{amsmath}
\usepackage{setspace}
\usepackage{svg}
\usepackage{hyperref}
\usepackage[backend=biber,style=numeric]{biblatex}
\addbibresource{references.bib}
\hypersetup{
    colorlinks=true,
    linkcolor=blue,      % color of internal links (sections, etc.)
    urlcolor=blue,       % color of external links
    pdftitle={Optimización energética de sistema híbrido con bomba de calor, suelo radiante, fotovoltaica y almacenamiento para vivienda},    % title
    pdfauthor={Luis D. Aranda Sánchez},     % author
    pdfkeywords={palabra1, palabra2, código1, etc.} % list of keywords
}

% Font change to Arial
\usepackage{helvet}
\renewcommand{\familydefault}{\sfdefault}

% Chapter titles in uppercase and larger font
\titleformat{\chapter}[hang]{\large\bfseries}{\thechapter.}{1em}{\MakeUppercase}
\titleformat{\section}[hang]{\bfseries}{\thesection.}{1em}{}
\titleformat{\subsection}[hang]{\bfseries}{\thesubsection.}{1em}{}

% Fancyhdr setup
\setlength{\headheight}{14.30174pt} % Adjust to recommended value, slightly larger for safety
\fancyhf{} % Clear all headers and footers
\fancyhead[LE]{\nouppercase{\leftmark}}
\fancyhead[RO]{Optimización energética para vivienda}
\fancyfoot[LE]{\thepage}
\fancyfoot[RE]{Escuela Técnica Superior de Ingenieros Industriales (UPM)}
\fancyfoot[LO]{Luis D. Aranda Sánchez}
\fancyfoot[RO]{\thepage}
\renewcommand{\headrulewidth}{0.4pt}
\renewcommand{\footrulewidth}{0.4pt}

\fancypagestyle{myfancy}{
    \fancyhf{} % Clear all headers and footers
    \fancyhead[LE]{\nouppercase{\leftmark}}
    \fancyhead[RO]{Optimización energética para vivienda}
    \fancyfoot[LE]{\thepage}
    \fancyfoot[RE]{Escuela Técnica Superior de Ingenieros Industriales (UPM)}
    \fancyfoot[LO]{Luis D. Aranda Sánchez}
    \fancyfoot[RO]{\thepage}
    \renewcommand{\headrulewidth}{0.4pt}
    \renewcommand{\footrulewidth}{0.4pt}
}

\fancypagestyle{simple}{
    \fancyhf{} % Clear all headers and footers
    \renewcommand{\headrulewidth}{0pt}
    \renewcommand{\footrulewidth}{0pt}
}

% Line spacing
\setstretch{1.2}

% Document starts here
\begin{document}

% Portada
\begin{titlepage}
    \centering
    {\scshape\LARGE Universidad Politécnica de Madrid \par}
    \vspace{1cm}
    {\scshape\Large Escuela Técnica Superior de Ingenieros Industriales\par}
    \vspace{1.5cm}
    {\huge\bfseries Optimización energética de sistema híbrido con bomba de calor, suelo radiante, fotovoltaica y almacenamiento para vivienda \par}
    \vspace{1.5cm}
    {\Large\bfseries Trabajo de Fin de Máster\par}
    \vspace{0.5cm}
    {\large Máster Universitario en Ingeniería de la Energía \par}
    \vspace{2cm}
    {\Large Luis D. Aranda Sánchez\par}
    \vfill
    Director: Javier Rodríguez Martín
    \vfill
    {\large Septiembre 6, 2024\par}
\end{titlepage}

% Resumen (máximo de 5 páginas, incluyendo al final Palabras clave)
\clearpage
\pagestyle{simple}
% \newpage
\chapter*{Resumen}
\addcontentsline{toc}{chapter}{Resumen}
\documentclass[a4paper,11pt,twoside]{report}
\usepackage[left=25mm,right=25mm,top=25mm,bottom=25mm,includehead,includefoot,headsep=15mm,footskip=15mm]{geometry}
\usepackage{graphicx}
\usepackage{fancyhdr}
\usepackage{titlesec}
\usepackage[spanish]{babel}
\usepackage[utf8]{inputenc}
\usepackage{amsmath}
\usepackage{setspace}
\usepackage{svg}
\usepackage{hyperref}
\usepackage[backend=biber,style=numeric]{biblatex}
\addbibresource{references.bib}
\hypersetup{
    colorlinks=true,
    linkcolor=blue,      % color of internal links (sections, etc.)
    urlcolor=blue,       % color of external links
    pdftitle={Optimización energética de sistema híbrido con bomba de calor, suelo radiante, fotovoltaica y almacenamiento para vivienda},    % title
    pdfauthor={Luis D. Aranda Sánchez},     % author
    pdfkeywords={palabra1, palabra2, código1, etc.} % list of keywords
}

% Font change to Arial
\usepackage{helvet}
\renewcommand{\familydefault}{\sfdefault}

% Chapter titles in uppercase and larger font
\titleformat{\chapter}[hang]{\large\bfseries}{\thechapter.}{1em}{\MakeUppercase}
\titleformat{\section}[hang]{\bfseries}{\thesection.}{1em}{}
\titleformat{\subsection}[hang]{\bfseries}{\thesubsection.}{1em}{}

% Fancyhdr setup
\setlength{\headheight}{14.30174pt} % Adjust to recommended value, slightly larger for safety
\fancyhf{} % Clear all headers and footers
\fancyhead[LE]{\nouppercase{\leftmark}}
\fancyhead[RO]{Optimización energética para vivienda}
\fancyfoot[LE]{\thepage}
\fancyfoot[RE]{Escuela Técnica Superior de Ingenieros Industriales (UPM)}
\fancyfoot[LO]{Luis D. Aranda Sánchez}
\fancyfoot[RO]{\thepage}
\renewcommand{\headrulewidth}{0.4pt}
\renewcommand{\footrulewidth}{0.4pt}

\fancypagestyle{myfancy}{
    \fancyhf{} % Clear all headers and footers
    \fancyhead[LE]{\nouppercase{\leftmark}}
    \fancyhead[RO]{Optimización energética para vivienda}
    \fancyfoot[LE]{\thepage}
    \fancyfoot[RE]{Escuela Técnica Superior de Ingenieros Industriales (UPM)}
    \fancyfoot[LO]{Luis D. Aranda Sánchez}
    \fancyfoot[RO]{\thepage}
    \renewcommand{\headrulewidth}{0.4pt}
    \renewcommand{\footrulewidth}{0.4pt}
}

\fancypagestyle{simple}{
    \fancyhf{} % Clear all headers and footers
    \renewcommand{\headrulewidth}{0pt}
    \renewcommand{\footrulewidth}{0pt}
}

% Line spacing
\setstretch{1.2}

% Document starts here
\begin{document}

% Portada
\begin{titlepage}
    \centering
    {\scshape\LARGE Universidad Politécnica de Madrid \par}
    \vspace{1cm}
    {\scshape\Large Escuela Técnica Superior de Ingenieros Industriales\par}
    \vspace{1.5cm}
    {\huge\bfseries Optimización energética de sistema híbrido con bomba de calor, suelo radiante, fotovoltaica y almacenamiento para vivienda \par}
    \vspace{1.5cm}
    {\Large\bfseries Trabajo de Fin de Máster\par}
    \vspace{0.5cm}
    {\large Máster Universitario en Ingeniería de la Energía \par}
    \vspace{2cm}
    {\Large Luis D. Aranda Sánchez\par}
    \vfill
    Director: Javier Rodríguez Martín
    \vfill
    {\large Septiembre 6, 2024\par}
\end{titlepage}

% Resumen (máximo de 5 páginas, incluyendo al final Palabras clave)
\clearpage
\pagestyle{simple}
% \newpage
\chapter*{Resumen}
\addcontentsline{toc}{chapter}{Resumen}
\input{capitulos/resumen/main.tex}

% Índice (paginado)
\clearpage
\pagestyle{simple}
% \newpage
\tableofcontents

% Introducción (donde se incluya los antecedentes y justificación)
\clearpage
\pagestyle{myfancy}
\newpage
\chapter{Introducción}
\input{capitulos/introduccion/main.tex}

% Objetivos
\chapter{Objetivos}
\input{capitulos/objetivos/main.tex}

% Metodología
\chapter{Metodología}
\input{capitulos/metodologia/main.tex}

% Resultados y discusión (incluyendo la valoración de impactos y de aspectos de responsabilidad legal, ética y profesional relacionados con el trabajo)
\chapter{Resultados y Discusión}
\input{capitulos/resultados_discusion/main.tex}

% Conclusiones
\chapter{Conclusiones}
\input{capitulos/conclusiones/main.tex}

% Planificación temporal y presupuesto
\chapter{Planificación Temporal y Presupuesto}
\input{capitulos/planificacion_presupuesto/main.tex}

% Bibliografía
\newpage
\addcontentsline{toc}{chapter}{Bibliografía}
\printbibliography

\end{document}


% Índice (paginado)
\clearpage
\pagestyle{simple}
% \newpage
\tableofcontents

% Introducción (donde se incluya los antecedentes y justificación)
\clearpage
\pagestyle{myfancy}
\newpage
\chapter{Introducción}
\documentclass[a4paper,11pt,twoside]{report}
\usepackage[left=25mm,right=25mm,top=25mm,bottom=25mm,includehead,includefoot,headsep=15mm,footskip=15mm]{geometry}
\usepackage{graphicx}
\usepackage{fancyhdr}
\usepackage{titlesec}
\usepackage[spanish]{babel}
\usepackage[utf8]{inputenc}
\usepackage{amsmath}
\usepackage{setspace}
\usepackage{svg}
\usepackage{hyperref}
\usepackage[backend=biber,style=numeric]{biblatex}
\addbibresource{references.bib}
\hypersetup{
    colorlinks=true,
    linkcolor=blue,      % color of internal links (sections, etc.)
    urlcolor=blue,       % color of external links
    pdftitle={Optimización energética de sistema híbrido con bomba de calor, suelo radiante, fotovoltaica y almacenamiento para vivienda},    % title
    pdfauthor={Luis D. Aranda Sánchez},     % author
    pdfkeywords={palabra1, palabra2, código1, etc.} % list of keywords
}

% Font change to Arial
\usepackage{helvet}
\renewcommand{\familydefault}{\sfdefault}

% Chapter titles in uppercase and larger font
\titleformat{\chapter}[hang]{\large\bfseries}{\thechapter.}{1em}{\MakeUppercase}
\titleformat{\section}[hang]{\bfseries}{\thesection.}{1em}{}
\titleformat{\subsection}[hang]{\bfseries}{\thesubsection.}{1em}{}

% Fancyhdr setup
\setlength{\headheight}{14.30174pt} % Adjust to recommended value, slightly larger for safety
\fancyhf{} % Clear all headers and footers
\fancyhead[LE]{\nouppercase{\leftmark}}
\fancyhead[RO]{Optimización energética para vivienda}
\fancyfoot[LE]{\thepage}
\fancyfoot[RE]{Escuela Técnica Superior de Ingenieros Industriales (UPM)}
\fancyfoot[LO]{Luis D. Aranda Sánchez}
\fancyfoot[RO]{\thepage}
\renewcommand{\headrulewidth}{0.4pt}
\renewcommand{\footrulewidth}{0.4pt}

\fancypagestyle{myfancy}{
    \fancyhf{} % Clear all headers and footers
    \fancyhead[LE]{\nouppercase{\leftmark}}
    \fancyhead[RO]{Optimización energética para vivienda}
    \fancyfoot[LE]{\thepage}
    \fancyfoot[RE]{Escuela Técnica Superior de Ingenieros Industriales (UPM)}
    \fancyfoot[LO]{Luis D. Aranda Sánchez}
    \fancyfoot[RO]{\thepage}
    \renewcommand{\headrulewidth}{0.4pt}
    \renewcommand{\footrulewidth}{0.4pt}
}

\fancypagestyle{simple}{
    \fancyhf{} % Clear all headers and footers
    \renewcommand{\headrulewidth}{0pt}
    \renewcommand{\footrulewidth}{0pt}
}

% Line spacing
\setstretch{1.2}

% Document starts here
\begin{document}

% Portada
\begin{titlepage}
    \centering
    {\scshape\LARGE Universidad Politécnica de Madrid \par}
    \vspace{1cm}
    {\scshape\Large Escuela Técnica Superior de Ingenieros Industriales\par}
    \vspace{1.5cm}
    {\huge\bfseries Optimización energética de sistema híbrido con bomba de calor, suelo radiante, fotovoltaica y almacenamiento para vivienda \par}
    \vspace{1.5cm}
    {\Large\bfseries Trabajo de Fin de Máster\par}
    \vspace{0.5cm}
    {\large Máster Universitario en Ingeniería de la Energía \par}
    \vspace{2cm}
    {\Large Luis D. Aranda Sánchez\par}
    \vfill
    Director: Javier Rodríguez Martín
    \vfill
    {\large Septiembre 6, 2024\par}
\end{titlepage}

% Resumen (máximo de 5 páginas, incluyendo al final Palabras clave)
\clearpage
\pagestyle{simple}
% \newpage
\chapter*{Resumen}
\addcontentsline{toc}{chapter}{Resumen}
\input{capitulos/resumen/main.tex}

% Índice (paginado)
\clearpage
\pagestyle{simple}
% \newpage
\tableofcontents

% Introducción (donde se incluya los antecedentes y justificación)
\clearpage
\pagestyle{myfancy}
\newpage
\chapter{Introducción}
\input{capitulos/introduccion/main.tex}

% Objetivos
\chapter{Objetivos}
\input{capitulos/objetivos/main.tex}

% Metodología
\chapter{Metodología}
\input{capitulos/metodologia/main.tex}

% Resultados y discusión (incluyendo la valoración de impactos y de aspectos de responsabilidad legal, ética y profesional relacionados con el trabajo)
\chapter{Resultados y Discusión}
\input{capitulos/resultados_discusion/main.tex}

% Conclusiones
\chapter{Conclusiones}
\input{capitulos/conclusiones/main.tex}

% Planificación temporal y presupuesto
\chapter{Planificación Temporal y Presupuesto}
\input{capitulos/planificacion_presupuesto/main.tex}

% Bibliografía
\newpage
\addcontentsline{toc}{chapter}{Bibliografía}
\printbibliography

\end{document}


% Objetivos
\chapter{Objetivos}
\documentclass[a4paper,11pt,twoside]{report}
\usepackage[left=25mm,right=25mm,top=25mm,bottom=25mm,includehead,includefoot,headsep=15mm,footskip=15mm]{geometry}
\usepackage{graphicx}
\usepackage{fancyhdr}
\usepackage{titlesec}
\usepackage[spanish]{babel}
\usepackage[utf8]{inputenc}
\usepackage{amsmath}
\usepackage{setspace}
\usepackage{svg}
\usepackage{hyperref}
\usepackage[backend=biber,style=numeric]{biblatex}
\addbibresource{references.bib}
\hypersetup{
    colorlinks=true,
    linkcolor=blue,      % color of internal links (sections, etc.)
    urlcolor=blue,       % color of external links
    pdftitle={Optimización energética de sistema híbrido con bomba de calor, suelo radiante, fotovoltaica y almacenamiento para vivienda},    % title
    pdfauthor={Luis D. Aranda Sánchez},     % author
    pdfkeywords={palabra1, palabra2, código1, etc.} % list of keywords
}

% Font change to Arial
\usepackage{helvet}
\renewcommand{\familydefault}{\sfdefault}

% Chapter titles in uppercase and larger font
\titleformat{\chapter}[hang]{\large\bfseries}{\thechapter.}{1em}{\MakeUppercase}
\titleformat{\section}[hang]{\bfseries}{\thesection.}{1em}{}
\titleformat{\subsection}[hang]{\bfseries}{\thesubsection.}{1em}{}

% Fancyhdr setup
\setlength{\headheight}{14.30174pt} % Adjust to recommended value, slightly larger for safety
\fancyhf{} % Clear all headers and footers
\fancyhead[LE]{\nouppercase{\leftmark}}
\fancyhead[RO]{Optimización energética para vivienda}
\fancyfoot[LE]{\thepage}
\fancyfoot[RE]{Escuela Técnica Superior de Ingenieros Industriales (UPM)}
\fancyfoot[LO]{Luis D. Aranda Sánchez}
\fancyfoot[RO]{\thepage}
\renewcommand{\headrulewidth}{0.4pt}
\renewcommand{\footrulewidth}{0.4pt}

\fancypagestyle{myfancy}{
    \fancyhf{} % Clear all headers and footers
    \fancyhead[LE]{\nouppercase{\leftmark}}
    \fancyhead[RO]{Optimización energética para vivienda}
    \fancyfoot[LE]{\thepage}
    \fancyfoot[RE]{Escuela Técnica Superior de Ingenieros Industriales (UPM)}
    \fancyfoot[LO]{Luis D. Aranda Sánchez}
    \fancyfoot[RO]{\thepage}
    \renewcommand{\headrulewidth}{0.4pt}
    \renewcommand{\footrulewidth}{0.4pt}
}

\fancypagestyle{simple}{
    \fancyhf{} % Clear all headers and footers
    \renewcommand{\headrulewidth}{0pt}
    \renewcommand{\footrulewidth}{0pt}
}

% Line spacing
\setstretch{1.2}

% Document starts here
\begin{document}

% Portada
\begin{titlepage}
    \centering
    {\scshape\LARGE Universidad Politécnica de Madrid \par}
    \vspace{1cm}
    {\scshape\Large Escuela Técnica Superior de Ingenieros Industriales\par}
    \vspace{1.5cm}
    {\huge\bfseries Optimización energética de sistema híbrido con bomba de calor, suelo radiante, fotovoltaica y almacenamiento para vivienda \par}
    \vspace{1.5cm}
    {\Large\bfseries Trabajo de Fin de Máster\par}
    \vspace{0.5cm}
    {\large Máster Universitario en Ingeniería de la Energía \par}
    \vspace{2cm}
    {\Large Luis D. Aranda Sánchez\par}
    \vfill
    Director: Javier Rodríguez Martín
    \vfill
    {\large Septiembre 6, 2024\par}
\end{titlepage}

% Resumen (máximo de 5 páginas, incluyendo al final Palabras clave)
\clearpage
\pagestyle{simple}
% \newpage
\chapter*{Resumen}
\addcontentsline{toc}{chapter}{Resumen}
\input{capitulos/resumen/main.tex}

% Índice (paginado)
\clearpage
\pagestyle{simple}
% \newpage
\tableofcontents

% Introducción (donde se incluya los antecedentes y justificación)
\clearpage
\pagestyle{myfancy}
\newpage
\chapter{Introducción}
\input{capitulos/introduccion/main.tex}

% Objetivos
\chapter{Objetivos}
\input{capitulos/objetivos/main.tex}

% Metodología
\chapter{Metodología}
\input{capitulos/metodologia/main.tex}

% Resultados y discusión (incluyendo la valoración de impactos y de aspectos de responsabilidad legal, ética y profesional relacionados con el trabajo)
\chapter{Resultados y Discusión}
\input{capitulos/resultados_discusion/main.tex}

% Conclusiones
\chapter{Conclusiones}
\input{capitulos/conclusiones/main.tex}

% Planificación temporal y presupuesto
\chapter{Planificación Temporal y Presupuesto}
\input{capitulos/planificacion_presupuesto/main.tex}

% Bibliografía
\newpage
\addcontentsline{toc}{chapter}{Bibliografía}
\printbibliography

\end{document}


% Metodología
\chapter{Metodología}
\documentclass[a4paper,11pt,twoside]{report}
\usepackage[left=25mm,right=25mm,top=25mm,bottom=25mm,includehead,includefoot,headsep=15mm,footskip=15mm]{geometry}
\usepackage{graphicx}
\usepackage{fancyhdr}
\usepackage{titlesec}
\usepackage[spanish]{babel}
\usepackage[utf8]{inputenc}
\usepackage{amsmath}
\usepackage{setspace}
\usepackage{svg}
\usepackage{hyperref}
\usepackage[backend=biber,style=numeric]{biblatex}
\addbibresource{references.bib}
\hypersetup{
    colorlinks=true,
    linkcolor=blue,      % color of internal links (sections, etc.)
    urlcolor=blue,       % color of external links
    pdftitle={Optimización energética de sistema híbrido con bomba de calor, suelo radiante, fotovoltaica y almacenamiento para vivienda},    % title
    pdfauthor={Luis D. Aranda Sánchez},     % author
    pdfkeywords={palabra1, palabra2, código1, etc.} % list of keywords
}

% Font change to Arial
\usepackage{helvet}
\renewcommand{\familydefault}{\sfdefault}

% Chapter titles in uppercase and larger font
\titleformat{\chapter}[hang]{\large\bfseries}{\thechapter.}{1em}{\MakeUppercase}
\titleformat{\section}[hang]{\bfseries}{\thesection.}{1em}{}
\titleformat{\subsection}[hang]{\bfseries}{\thesubsection.}{1em}{}

% Fancyhdr setup
\setlength{\headheight}{14.30174pt} % Adjust to recommended value, slightly larger for safety
\fancyhf{} % Clear all headers and footers
\fancyhead[LE]{\nouppercase{\leftmark}}
\fancyhead[RO]{Optimización energética para vivienda}
\fancyfoot[LE]{\thepage}
\fancyfoot[RE]{Escuela Técnica Superior de Ingenieros Industriales (UPM)}
\fancyfoot[LO]{Luis D. Aranda Sánchez}
\fancyfoot[RO]{\thepage}
\renewcommand{\headrulewidth}{0.4pt}
\renewcommand{\footrulewidth}{0.4pt}

\fancypagestyle{myfancy}{
    \fancyhf{} % Clear all headers and footers
    \fancyhead[LE]{\nouppercase{\leftmark}}
    \fancyhead[RO]{Optimización energética para vivienda}
    \fancyfoot[LE]{\thepage}
    \fancyfoot[RE]{Escuela Técnica Superior de Ingenieros Industriales (UPM)}
    \fancyfoot[LO]{Luis D. Aranda Sánchez}
    \fancyfoot[RO]{\thepage}
    \renewcommand{\headrulewidth}{0.4pt}
    \renewcommand{\footrulewidth}{0.4pt}
}

\fancypagestyle{simple}{
    \fancyhf{} % Clear all headers and footers
    \renewcommand{\headrulewidth}{0pt}
    \renewcommand{\footrulewidth}{0pt}
}

% Line spacing
\setstretch{1.2}

% Document starts here
\begin{document}

% Portada
\begin{titlepage}
    \centering
    {\scshape\LARGE Universidad Politécnica de Madrid \par}
    \vspace{1cm}
    {\scshape\Large Escuela Técnica Superior de Ingenieros Industriales\par}
    \vspace{1.5cm}
    {\huge\bfseries Optimización energética de sistema híbrido con bomba de calor, suelo radiante, fotovoltaica y almacenamiento para vivienda \par}
    \vspace{1.5cm}
    {\Large\bfseries Trabajo de Fin de Máster\par}
    \vspace{0.5cm}
    {\large Máster Universitario en Ingeniería de la Energía \par}
    \vspace{2cm}
    {\Large Luis D. Aranda Sánchez\par}
    \vfill
    Director: Javier Rodríguez Martín
    \vfill
    {\large Septiembre 6, 2024\par}
\end{titlepage}

% Resumen (máximo de 5 páginas, incluyendo al final Palabras clave)
\clearpage
\pagestyle{simple}
% \newpage
\chapter*{Resumen}
\addcontentsline{toc}{chapter}{Resumen}
\input{capitulos/resumen/main.tex}

% Índice (paginado)
\clearpage
\pagestyle{simple}
% \newpage
\tableofcontents

% Introducción (donde se incluya los antecedentes y justificación)
\clearpage
\pagestyle{myfancy}
\newpage
\chapter{Introducción}
\input{capitulos/introduccion/main.tex}

% Objetivos
\chapter{Objetivos}
\input{capitulos/objetivos/main.tex}

% Metodología
\chapter{Metodología}
\input{capitulos/metodologia/main.tex}

% Resultados y discusión (incluyendo la valoración de impactos y de aspectos de responsabilidad legal, ética y profesional relacionados con el trabajo)
\chapter{Resultados y Discusión}
\input{capitulos/resultados_discusion/main.tex}

% Conclusiones
\chapter{Conclusiones}
\input{capitulos/conclusiones/main.tex}

% Planificación temporal y presupuesto
\chapter{Planificación Temporal y Presupuesto}
\input{capitulos/planificacion_presupuesto/main.tex}

% Bibliografía
\newpage
\addcontentsline{toc}{chapter}{Bibliografía}
\printbibliography

\end{document}


% Resultados y discusión (incluyendo la valoración de impactos y de aspectos de responsabilidad legal, ética y profesional relacionados con el trabajo)
\chapter{Resultados y Discusión}
\documentclass[a4paper,11pt,twoside]{report}
\usepackage[left=25mm,right=25mm,top=25mm,bottom=25mm,includehead,includefoot,headsep=15mm,footskip=15mm]{geometry}
\usepackage{graphicx}
\usepackage{fancyhdr}
\usepackage{titlesec}
\usepackage[spanish]{babel}
\usepackage[utf8]{inputenc}
\usepackage{amsmath}
\usepackage{setspace}
\usepackage{svg}
\usepackage{hyperref}
\usepackage[backend=biber,style=numeric]{biblatex}
\addbibresource{references.bib}
\hypersetup{
    colorlinks=true,
    linkcolor=blue,      % color of internal links (sections, etc.)
    urlcolor=blue,       % color of external links
    pdftitle={Optimización energética de sistema híbrido con bomba de calor, suelo radiante, fotovoltaica y almacenamiento para vivienda},    % title
    pdfauthor={Luis D. Aranda Sánchez},     % author
    pdfkeywords={palabra1, palabra2, código1, etc.} % list of keywords
}

% Font change to Arial
\usepackage{helvet}
\renewcommand{\familydefault}{\sfdefault}

% Chapter titles in uppercase and larger font
\titleformat{\chapter}[hang]{\large\bfseries}{\thechapter.}{1em}{\MakeUppercase}
\titleformat{\section}[hang]{\bfseries}{\thesection.}{1em}{}
\titleformat{\subsection}[hang]{\bfseries}{\thesubsection.}{1em}{}

% Fancyhdr setup
\setlength{\headheight}{14.30174pt} % Adjust to recommended value, slightly larger for safety
\fancyhf{} % Clear all headers and footers
\fancyhead[LE]{\nouppercase{\leftmark}}
\fancyhead[RO]{Optimización energética para vivienda}
\fancyfoot[LE]{\thepage}
\fancyfoot[RE]{Escuela Técnica Superior de Ingenieros Industriales (UPM)}
\fancyfoot[LO]{Luis D. Aranda Sánchez}
\fancyfoot[RO]{\thepage}
\renewcommand{\headrulewidth}{0.4pt}
\renewcommand{\footrulewidth}{0.4pt}

\fancypagestyle{myfancy}{
    \fancyhf{} % Clear all headers and footers
    \fancyhead[LE]{\nouppercase{\leftmark}}
    \fancyhead[RO]{Optimización energética para vivienda}
    \fancyfoot[LE]{\thepage}
    \fancyfoot[RE]{Escuela Técnica Superior de Ingenieros Industriales (UPM)}
    \fancyfoot[LO]{Luis D. Aranda Sánchez}
    \fancyfoot[RO]{\thepage}
    \renewcommand{\headrulewidth}{0.4pt}
    \renewcommand{\footrulewidth}{0.4pt}
}

\fancypagestyle{simple}{
    \fancyhf{} % Clear all headers and footers
    \renewcommand{\headrulewidth}{0pt}
    \renewcommand{\footrulewidth}{0pt}
}

% Line spacing
\setstretch{1.2}

% Document starts here
\begin{document}

% Portada
\begin{titlepage}
    \centering
    {\scshape\LARGE Universidad Politécnica de Madrid \par}
    \vspace{1cm}
    {\scshape\Large Escuela Técnica Superior de Ingenieros Industriales\par}
    \vspace{1.5cm}
    {\huge\bfseries Optimización energética de sistema híbrido con bomba de calor, suelo radiante, fotovoltaica y almacenamiento para vivienda \par}
    \vspace{1.5cm}
    {\Large\bfseries Trabajo de Fin de Máster\par}
    \vspace{0.5cm}
    {\large Máster Universitario en Ingeniería de la Energía \par}
    \vspace{2cm}
    {\Large Luis D. Aranda Sánchez\par}
    \vfill
    Director: Javier Rodríguez Martín
    \vfill
    {\large Septiembre 6, 2024\par}
\end{titlepage}

% Resumen (máximo de 5 páginas, incluyendo al final Palabras clave)
\clearpage
\pagestyle{simple}
% \newpage
\chapter*{Resumen}
\addcontentsline{toc}{chapter}{Resumen}
\input{capitulos/resumen/main.tex}

% Índice (paginado)
\clearpage
\pagestyle{simple}
% \newpage
\tableofcontents

% Introducción (donde se incluya los antecedentes y justificación)
\clearpage
\pagestyle{myfancy}
\newpage
\chapter{Introducción}
\input{capitulos/introduccion/main.tex}

% Objetivos
\chapter{Objetivos}
\input{capitulos/objetivos/main.tex}

% Metodología
\chapter{Metodología}
\input{capitulos/metodologia/main.tex}

% Resultados y discusión (incluyendo la valoración de impactos y de aspectos de responsabilidad legal, ética y profesional relacionados con el trabajo)
\chapter{Resultados y Discusión}
\input{capitulos/resultados_discusion/main.tex}

% Conclusiones
\chapter{Conclusiones}
\input{capitulos/conclusiones/main.tex}

% Planificación temporal y presupuesto
\chapter{Planificación Temporal y Presupuesto}
\input{capitulos/planificacion_presupuesto/main.tex}

% Bibliografía
\newpage
\addcontentsline{toc}{chapter}{Bibliografía}
\printbibliography

\end{document}


% Conclusiones
\chapter{Conclusiones}
\documentclass[a4paper,11pt,twoside]{report}
\usepackage[left=25mm,right=25mm,top=25mm,bottom=25mm,includehead,includefoot,headsep=15mm,footskip=15mm]{geometry}
\usepackage{graphicx}
\usepackage{fancyhdr}
\usepackage{titlesec}
\usepackage[spanish]{babel}
\usepackage[utf8]{inputenc}
\usepackage{amsmath}
\usepackage{setspace}
\usepackage{svg}
\usepackage{hyperref}
\usepackage[backend=biber,style=numeric]{biblatex}
\addbibresource{references.bib}
\hypersetup{
    colorlinks=true,
    linkcolor=blue,      % color of internal links (sections, etc.)
    urlcolor=blue,       % color of external links
    pdftitle={Optimización energética de sistema híbrido con bomba de calor, suelo radiante, fotovoltaica y almacenamiento para vivienda},    % title
    pdfauthor={Luis D. Aranda Sánchez},     % author
    pdfkeywords={palabra1, palabra2, código1, etc.} % list of keywords
}

% Font change to Arial
\usepackage{helvet}
\renewcommand{\familydefault}{\sfdefault}

% Chapter titles in uppercase and larger font
\titleformat{\chapter}[hang]{\large\bfseries}{\thechapter.}{1em}{\MakeUppercase}
\titleformat{\section}[hang]{\bfseries}{\thesection.}{1em}{}
\titleformat{\subsection}[hang]{\bfseries}{\thesubsection.}{1em}{}

% Fancyhdr setup
\setlength{\headheight}{14.30174pt} % Adjust to recommended value, slightly larger for safety
\fancyhf{} % Clear all headers and footers
\fancyhead[LE]{\nouppercase{\leftmark}}
\fancyhead[RO]{Optimización energética para vivienda}
\fancyfoot[LE]{\thepage}
\fancyfoot[RE]{Escuela Técnica Superior de Ingenieros Industriales (UPM)}
\fancyfoot[LO]{Luis D. Aranda Sánchez}
\fancyfoot[RO]{\thepage}
\renewcommand{\headrulewidth}{0.4pt}
\renewcommand{\footrulewidth}{0.4pt}

\fancypagestyle{myfancy}{
    \fancyhf{} % Clear all headers and footers
    \fancyhead[LE]{\nouppercase{\leftmark}}
    \fancyhead[RO]{Optimización energética para vivienda}
    \fancyfoot[LE]{\thepage}
    \fancyfoot[RE]{Escuela Técnica Superior de Ingenieros Industriales (UPM)}
    \fancyfoot[LO]{Luis D. Aranda Sánchez}
    \fancyfoot[RO]{\thepage}
    \renewcommand{\headrulewidth}{0.4pt}
    \renewcommand{\footrulewidth}{0.4pt}
}

\fancypagestyle{simple}{
    \fancyhf{} % Clear all headers and footers
    \renewcommand{\headrulewidth}{0pt}
    \renewcommand{\footrulewidth}{0pt}
}

% Line spacing
\setstretch{1.2}

% Document starts here
\begin{document}

% Portada
\begin{titlepage}
    \centering
    {\scshape\LARGE Universidad Politécnica de Madrid \par}
    \vspace{1cm}
    {\scshape\Large Escuela Técnica Superior de Ingenieros Industriales\par}
    \vspace{1.5cm}
    {\huge\bfseries Optimización energética de sistema híbrido con bomba de calor, suelo radiante, fotovoltaica y almacenamiento para vivienda \par}
    \vspace{1.5cm}
    {\Large\bfseries Trabajo de Fin de Máster\par}
    \vspace{0.5cm}
    {\large Máster Universitario en Ingeniería de la Energía \par}
    \vspace{2cm}
    {\Large Luis D. Aranda Sánchez\par}
    \vfill
    Director: Javier Rodríguez Martín
    \vfill
    {\large Septiembre 6, 2024\par}
\end{titlepage}

% Resumen (máximo de 5 páginas, incluyendo al final Palabras clave)
\clearpage
\pagestyle{simple}
% \newpage
\chapter*{Resumen}
\addcontentsline{toc}{chapter}{Resumen}
\input{capitulos/resumen/main.tex}

% Índice (paginado)
\clearpage
\pagestyle{simple}
% \newpage
\tableofcontents

% Introducción (donde se incluya los antecedentes y justificación)
\clearpage
\pagestyle{myfancy}
\newpage
\chapter{Introducción}
\input{capitulos/introduccion/main.tex}

% Objetivos
\chapter{Objetivos}
\input{capitulos/objetivos/main.tex}

% Metodología
\chapter{Metodología}
\input{capitulos/metodologia/main.tex}

% Resultados y discusión (incluyendo la valoración de impactos y de aspectos de responsabilidad legal, ética y profesional relacionados con el trabajo)
\chapter{Resultados y Discusión}
\input{capitulos/resultados_discusion/main.tex}

% Conclusiones
\chapter{Conclusiones}
\input{capitulos/conclusiones/main.tex}

% Planificación temporal y presupuesto
\chapter{Planificación Temporal y Presupuesto}
\input{capitulos/planificacion_presupuesto/main.tex}

% Bibliografía
\newpage
\addcontentsline{toc}{chapter}{Bibliografía}
\printbibliography

\end{document}


% Planificación temporal y presupuesto
\chapter{Planificación Temporal y Presupuesto}
\documentclass[a4paper,11pt,twoside]{report}
\usepackage[left=25mm,right=25mm,top=25mm,bottom=25mm,includehead,includefoot,headsep=15mm,footskip=15mm]{geometry}
\usepackage{graphicx}
\usepackage{fancyhdr}
\usepackage{titlesec}
\usepackage[spanish]{babel}
\usepackage[utf8]{inputenc}
\usepackage{amsmath}
\usepackage{setspace}
\usepackage{svg}
\usepackage{hyperref}
\usepackage[backend=biber,style=numeric]{biblatex}
\addbibresource{references.bib}
\hypersetup{
    colorlinks=true,
    linkcolor=blue,      % color of internal links (sections, etc.)
    urlcolor=blue,       % color of external links
    pdftitle={Optimización energética de sistema híbrido con bomba de calor, suelo radiante, fotovoltaica y almacenamiento para vivienda},    % title
    pdfauthor={Luis D. Aranda Sánchez},     % author
    pdfkeywords={palabra1, palabra2, código1, etc.} % list of keywords
}

% Font change to Arial
\usepackage{helvet}
\renewcommand{\familydefault}{\sfdefault}

% Chapter titles in uppercase and larger font
\titleformat{\chapter}[hang]{\large\bfseries}{\thechapter.}{1em}{\MakeUppercase}
\titleformat{\section}[hang]{\bfseries}{\thesection.}{1em}{}
\titleformat{\subsection}[hang]{\bfseries}{\thesubsection.}{1em}{}

% Fancyhdr setup
\setlength{\headheight}{14.30174pt} % Adjust to recommended value, slightly larger for safety
\fancyhf{} % Clear all headers and footers
\fancyhead[LE]{\nouppercase{\leftmark}}
\fancyhead[RO]{Optimización energética para vivienda}
\fancyfoot[LE]{\thepage}
\fancyfoot[RE]{Escuela Técnica Superior de Ingenieros Industriales (UPM)}
\fancyfoot[LO]{Luis D. Aranda Sánchez}
\fancyfoot[RO]{\thepage}
\renewcommand{\headrulewidth}{0.4pt}
\renewcommand{\footrulewidth}{0.4pt}

\fancypagestyle{myfancy}{
    \fancyhf{} % Clear all headers and footers
    \fancyhead[LE]{\nouppercase{\leftmark}}
    \fancyhead[RO]{Optimización energética para vivienda}
    \fancyfoot[LE]{\thepage}
    \fancyfoot[RE]{Escuela Técnica Superior de Ingenieros Industriales (UPM)}
    \fancyfoot[LO]{Luis D. Aranda Sánchez}
    \fancyfoot[RO]{\thepage}
    \renewcommand{\headrulewidth}{0.4pt}
    \renewcommand{\footrulewidth}{0.4pt}
}

\fancypagestyle{simple}{
    \fancyhf{} % Clear all headers and footers
    \renewcommand{\headrulewidth}{0pt}
    \renewcommand{\footrulewidth}{0pt}
}

% Line spacing
\setstretch{1.2}

% Document starts here
\begin{document}

% Portada
\begin{titlepage}
    \centering
    {\scshape\LARGE Universidad Politécnica de Madrid \par}
    \vspace{1cm}
    {\scshape\Large Escuela Técnica Superior de Ingenieros Industriales\par}
    \vspace{1.5cm}
    {\huge\bfseries Optimización energética de sistema híbrido con bomba de calor, suelo radiante, fotovoltaica y almacenamiento para vivienda \par}
    \vspace{1.5cm}
    {\Large\bfseries Trabajo de Fin de Máster\par}
    \vspace{0.5cm}
    {\large Máster Universitario en Ingeniería de la Energía \par}
    \vspace{2cm}
    {\Large Luis D. Aranda Sánchez\par}
    \vfill
    Director: Javier Rodríguez Martín
    \vfill
    {\large Septiembre 6, 2024\par}
\end{titlepage}

% Resumen (máximo de 5 páginas, incluyendo al final Palabras clave)
\clearpage
\pagestyle{simple}
% \newpage
\chapter*{Resumen}
\addcontentsline{toc}{chapter}{Resumen}
\input{capitulos/resumen/main.tex}

% Índice (paginado)
\clearpage
\pagestyle{simple}
% \newpage
\tableofcontents

% Introducción (donde se incluya los antecedentes y justificación)
\clearpage
\pagestyle{myfancy}
\newpage
\chapter{Introducción}
\input{capitulos/introduccion/main.tex}

% Objetivos
\chapter{Objetivos}
\input{capitulos/objetivos/main.tex}

% Metodología
\chapter{Metodología}
\input{capitulos/metodologia/main.tex}

% Resultados y discusión (incluyendo la valoración de impactos y de aspectos de responsabilidad legal, ética y profesional relacionados con el trabajo)
\chapter{Resultados y Discusión}
\input{capitulos/resultados_discusion/main.tex}

% Conclusiones
\chapter{Conclusiones}
\input{capitulos/conclusiones/main.tex}

% Planificación temporal y presupuesto
\chapter{Planificación Temporal y Presupuesto}
\input{capitulos/planificacion_presupuesto/main.tex}

% Bibliografía
\newpage
\addcontentsline{toc}{chapter}{Bibliografía}
\printbibliography

\end{document}


% Bibliografía
\newpage
\addcontentsline{toc}{chapter}{Bibliografía}
\printbibliography

\end{document}


% Planificación temporal y presupuesto
\chapter{Planificación Temporal y Presupuesto}
\documentclass[a4paper,11pt,twoside]{report}
\usepackage[left=25mm,right=25mm,top=25mm,bottom=25mm,includehead,includefoot,headsep=15mm,footskip=15mm]{geometry}
\usepackage{graphicx}
\usepackage{fancyhdr}
\usepackage{titlesec}
\usepackage[spanish]{babel}
\usepackage[utf8]{inputenc}
\usepackage{amsmath}
\usepackage{setspace}
\usepackage{svg}
\usepackage{hyperref}
\usepackage[backend=biber,style=numeric]{biblatex}
\addbibresource{references.bib}
\hypersetup{
    colorlinks=true,
    linkcolor=blue,      % color of internal links (sections, etc.)
    urlcolor=blue,       % color of external links
    pdftitle={Optimización energética de sistema híbrido con bomba de calor, suelo radiante, fotovoltaica y almacenamiento para vivienda},    % title
    pdfauthor={Luis D. Aranda Sánchez},     % author
    pdfkeywords={palabra1, palabra2, código1, etc.} % list of keywords
}

% Font change to Arial
\usepackage{helvet}
\renewcommand{\familydefault}{\sfdefault}

% Chapter titles in uppercase and larger font
\titleformat{\chapter}[hang]{\large\bfseries}{\thechapter.}{1em}{\MakeUppercase}
\titleformat{\section}[hang]{\bfseries}{\thesection.}{1em}{}
\titleformat{\subsection}[hang]{\bfseries}{\thesubsection.}{1em}{}

% Fancyhdr setup
\setlength{\headheight}{14.30174pt} % Adjust to recommended value, slightly larger for safety
\fancyhf{} % Clear all headers and footers
\fancyhead[LE]{\nouppercase{\leftmark}}
\fancyhead[RO]{Optimización energética para vivienda}
\fancyfoot[LE]{\thepage}
\fancyfoot[RE]{Escuela Técnica Superior de Ingenieros Industriales (UPM)}
\fancyfoot[LO]{Luis D. Aranda Sánchez}
\fancyfoot[RO]{\thepage}
\renewcommand{\headrulewidth}{0.4pt}
\renewcommand{\footrulewidth}{0.4pt}

\fancypagestyle{myfancy}{
    \fancyhf{} % Clear all headers and footers
    \fancyhead[LE]{\nouppercase{\leftmark}}
    \fancyhead[RO]{Optimización energética para vivienda}
    \fancyfoot[LE]{\thepage}
    \fancyfoot[RE]{Escuela Técnica Superior de Ingenieros Industriales (UPM)}
    \fancyfoot[LO]{Luis D. Aranda Sánchez}
    \fancyfoot[RO]{\thepage}
    \renewcommand{\headrulewidth}{0.4pt}
    \renewcommand{\footrulewidth}{0.4pt}
}

\fancypagestyle{simple}{
    \fancyhf{} % Clear all headers and footers
    \renewcommand{\headrulewidth}{0pt}
    \renewcommand{\footrulewidth}{0pt}
}

% Line spacing
\setstretch{1.2}

% Document starts here
\begin{document}

% Portada
\begin{titlepage}
    \centering
    {\scshape\LARGE Universidad Politécnica de Madrid \par}
    \vspace{1cm}
    {\scshape\Large Escuela Técnica Superior de Ingenieros Industriales\par}
    \vspace{1.5cm}
    {\huge\bfseries Optimización energética de sistema híbrido con bomba de calor, suelo radiante, fotovoltaica y almacenamiento para vivienda \par}
    \vspace{1.5cm}
    {\Large\bfseries Trabajo de Fin de Máster\par}
    \vspace{0.5cm}
    {\large Máster Universitario en Ingeniería de la Energía \par}
    \vspace{2cm}
    {\Large Luis D. Aranda Sánchez\par}
    \vfill
    Director: Javier Rodríguez Martín
    \vfill
    {\large Septiembre 6, 2024\par}
\end{titlepage}

% Resumen (máximo de 5 páginas, incluyendo al final Palabras clave)
\clearpage
\pagestyle{simple}
% \newpage
\chapter*{Resumen}
\addcontentsline{toc}{chapter}{Resumen}
\documentclass[a4paper,11pt,twoside]{report}
\usepackage[left=25mm,right=25mm,top=25mm,bottom=25mm,includehead,includefoot,headsep=15mm,footskip=15mm]{geometry}
\usepackage{graphicx}
\usepackage{fancyhdr}
\usepackage{titlesec}
\usepackage[spanish]{babel}
\usepackage[utf8]{inputenc}
\usepackage{amsmath}
\usepackage{setspace}
\usepackage{svg}
\usepackage{hyperref}
\usepackage[backend=biber,style=numeric]{biblatex}
\addbibresource{references.bib}
\hypersetup{
    colorlinks=true,
    linkcolor=blue,      % color of internal links (sections, etc.)
    urlcolor=blue,       % color of external links
    pdftitle={Optimización energética de sistema híbrido con bomba de calor, suelo radiante, fotovoltaica y almacenamiento para vivienda},    % title
    pdfauthor={Luis D. Aranda Sánchez},     % author
    pdfkeywords={palabra1, palabra2, código1, etc.} % list of keywords
}

% Font change to Arial
\usepackage{helvet}
\renewcommand{\familydefault}{\sfdefault}

% Chapter titles in uppercase and larger font
\titleformat{\chapter}[hang]{\large\bfseries}{\thechapter.}{1em}{\MakeUppercase}
\titleformat{\section}[hang]{\bfseries}{\thesection.}{1em}{}
\titleformat{\subsection}[hang]{\bfseries}{\thesubsection.}{1em}{}

% Fancyhdr setup
\setlength{\headheight}{14.30174pt} % Adjust to recommended value, slightly larger for safety
\fancyhf{} % Clear all headers and footers
\fancyhead[LE]{\nouppercase{\leftmark}}
\fancyhead[RO]{Optimización energética para vivienda}
\fancyfoot[LE]{\thepage}
\fancyfoot[RE]{Escuela Técnica Superior de Ingenieros Industriales (UPM)}
\fancyfoot[LO]{Luis D. Aranda Sánchez}
\fancyfoot[RO]{\thepage}
\renewcommand{\headrulewidth}{0.4pt}
\renewcommand{\footrulewidth}{0.4pt}

\fancypagestyle{myfancy}{
    \fancyhf{} % Clear all headers and footers
    \fancyhead[LE]{\nouppercase{\leftmark}}
    \fancyhead[RO]{Optimización energética para vivienda}
    \fancyfoot[LE]{\thepage}
    \fancyfoot[RE]{Escuela Técnica Superior de Ingenieros Industriales (UPM)}
    \fancyfoot[LO]{Luis D. Aranda Sánchez}
    \fancyfoot[RO]{\thepage}
    \renewcommand{\headrulewidth}{0.4pt}
    \renewcommand{\footrulewidth}{0.4pt}
}

\fancypagestyle{simple}{
    \fancyhf{} % Clear all headers and footers
    \renewcommand{\headrulewidth}{0pt}
    \renewcommand{\footrulewidth}{0pt}
}

% Line spacing
\setstretch{1.2}

% Document starts here
\begin{document}

% Portada
\begin{titlepage}
    \centering
    {\scshape\LARGE Universidad Politécnica de Madrid \par}
    \vspace{1cm}
    {\scshape\Large Escuela Técnica Superior de Ingenieros Industriales\par}
    \vspace{1.5cm}
    {\huge\bfseries Optimización energética de sistema híbrido con bomba de calor, suelo radiante, fotovoltaica y almacenamiento para vivienda \par}
    \vspace{1.5cm}
    {\Large\bfseries Trabajo de Fin de Máster\par}
    \vspace{0.5cm}
    {\large Máster Universitario en Ingeniería de la Energía \par}
    \vspace{2cm}
    {\Large Luis D. Aranda Sánchez\par}
    \vfill
    Director: Javier Rodríguez Martín
    \vfill
    {\large Septiembre 6, 2024\par}
\end{titlepage}

% Resumen (máximo de 5 páginas, incluyendo al final Palabras clave)
\clearpage
\pagestyle{simple}
% \newpage
\chapter*{Resumen}
\addcontentsline{toc}{chapter}{Resumen}
\input{capitulos/resumen/main.tex}

% Índice (paginado)
\clearpage
\pagestyle{simple}
% \newpage
\tableofcontents

% Introducción (donde se incluya los antecedentes y justificación)
\clearpage
\pagestyle{myfancy}
\newpage
\chapter{Introducción}
\input{capitulos/introduccion/main.tex}

% Objetivos
\chapter{Objetivos}
\input{capitulos/objetivos/main.tex}

% Metodología
\chapter{Metodología}
\input{capitulos/metodologia/main.tex}

% Resultados y discusión (incluyendo la valoración de impactos y de aspectos de responsabilidad legal, ética y profesional relacionados con el trabajo)
\chapter{Resultados y Discusión}
\input{capitulos/resultados_discusion/main.tex}

% Conclusiones
\chapter{Conclusiones}
\input{capitulos/conclusiones/main.tex}

% Planificación temporal y presupuesto
\chapter{Planificación Temporal y Presupuesto}
\input{capitulos/planificacion_presupuesto/main.tex}

% Bibliografía
\newpage
\addcontentsline{toc}{chapter}{Bibliografía}
\printbibliography

\end{document}


% Índice (paginado)
\clearpage
\pagestyle{simple}
% \newpage
\tableofcontents

% Introducción (donde se incluya los antecedentes y justificación)
\clearpage
\pagestyle{myfancy}
\newpage
\chapter{Introducción}
\documentclass[a4paper,11pt,twoside]{report}
\usepackage[left=25mm,right=25mm,top=25mm,bottom=25mm,includehead,includefoot,headsep=15mm,footskip=15mm]{geometry}
\usepackage{graphicx}
\usepackage{fancyhdr}
\usepackage{titlesec}
\usepackage[spanish]{babel}
\usepackage[utf8]{inputenc}
\usepackage{amsmath}
\usepackage{setspace}
\usepackage{svg}
\usepackage{hyperref}
\usepackage[backend=biber,style=numeric]{biblatex}
\addbibresource{references.bib}
\hypersetup{
    colorlinks=true,
    linkcolor=blue,      % color of internal links (sections, etc.)
    urlcolor=blue,       % color of external links
    pdftitle={Optimización energética de sistema híbrido con bomba de calor, suelo radiante, fotovoltaica y almacenamiento para vivienda},    % title
    pdfauthor={Luis D. Aranda Sánchez},     % author
    pdfkeywords={palabra1, palabra2, código1, etc.} % list of keywords
}

% Font change to Arial
\usepackage{helvet}
\renewcommand{\familydefault}{\sfdefault}

% Chapter titles in uppercase and larger font
\titleformat{\chapter}[hang]{\large\bfseries}{\thechapter.}{1em}{\MakeUppercase}
\titleformat{\section}[hang]{\bfseries}{\thesection.}{1em}{}
\titleformat{\subsection}[hang]{\bfseries}{\thesubsection.}{1em}{}

% Fancyhdr setup
\setlength{\headheight}{14.30174pt} % Adjust to recommended value, slightly larger for safety
\fancyhf{} % Clear all headers and footers
\fancyhead[LE]{\nouppercase{\leftmark}}
\fancyhead[RO]{Optimización energética para vivienda}
\fancyfoot[LE]{\thepage}
\fancyfoot[RE]{Escuela Técnica Superior de Ingenieros Industriales (UPM)}
\fancyfoot[LO]{Luis D. Aranda Sánchez}
\fancyfoot[RO]{\thepage}
\renewcommand{\headrulewidth}{0.4pt}
\renewcommand{\footrulewidth}{0.4pt}

\fancypagestyle{myfancy}{
    \fancyhf{} % Clear all headers and footers
    \fancyhead[LE]{\nouppercase{\leftmark}}
    \fancyhead[RO]{Optimización energética para vivienda}
    \fancyfoot[LE]{\thepage}
    \fancyfoot[RE]{Escuela Técnica Superior de Ingenieros Industriales (UPM)}
    \fancyfoot[LO]{Luis D. Aranda Sánchez}
    \fancyfoot[RO]{\thepage}
    \renewcommand{\headrulewidth}{0.4pt}
    \renewcommand{\footrulewidth}{0.4pt}
}

\fancypagestyle{simple}{
    \fancyhf{} % Clear all headers and footers
    \renewcommand{\headrulewidth}{0pt}
    \renewcommand{\footrulewidth}{0pt}
}

% Line spacing
\setstretch{1.2}

% Document starts here
\begin{document}

% Portada
\begin{titlepage}
    \centering
    {\scshape\LARGE Universidad Politécnica de Madrid \par}
    \vspace{1cm}
    {\scshape\Large Escuela Técnica Superior de Ingenieros Industriales\par}
    \vspace{1.5cm}
    {\huge\bfseries Optimización energética de sistema híbrido con bomba de calor, suelo radiante, fotovoltaica y almacenamiento para vivienda \par}
    \vspace{1.5cm}
    {\Large\bfseries Trabajo de Fin de Máster\par}
    \vspace{0.5cm}
    {\large Máster Universitario en Ingeniería de la Energía \par}
    \vspace{2cm}
    {\Large Luis D. Aranda Sánchez\par}
    \vfill
    Director: Javier Rodríguez Martín
    \vfill
    {\large Septiembre 6, 2024\par}
\end{titlepage}

% Resumen (máximo de 5 páginas, incluyendo al final Palabras clave)
\clearpage
\pagestyle{simple}
% \newpage
\chapter*{Resumen}
\addcontentsline{toc}{chapter}{Resumen}
\input{capitulos/resumen/main.tex}

% Índice (paginado)
\clearpage
\pagestyle{simple}
% \newpage
\tableofcontents

% Introducción (donde se incluya los antecedentes y justificación)
\clearpage
\pagestyle{myfancy}
\newpage
\chapter{Introducción}
\input{capitulos/introduccion/main.tex}

% Objetivos
\chapter{Objetivos}
\input{capitulos/objetivos/main.tex}

% Metodología
\chapter{Metodología}
\input{capitulos/metodologia/main.tex}

% Resultados y discusión (incluyendo la valoración de impactos y de aspectos de responsabilidad legal, ética y profesional relacionados con el trabajo)
\chapter{Resultados y Discusión}
\input{capitulos/resultados_discusion/main.tex}

% Conclusiones
\chapter{Conclusiones}
\input{capitulos/conclusiones/main.tex}

% Planificación temporal y presupuesto
\chapter{Planificación Temporal y Presupuesto}
\input{capitulos/planificacion_presupuesto/main.tex}

% Bibliografía
\newpage
\addcontentsline{toc}{chapter}{Bibliografía}
\printbibliography

\end{document}


% Objetivos
\chapter{Objetivos}
\documentclass[a4paper,11pt,twoside]{report}
\usepackage[left=25mm,right=25mm,top=25mm,bottom=25mm,includehead,includefoot,headsep=15mm,footskip=15mm]{geometry}
\usepackage{graphicx}
\usepackage{fancyhdr}
\usepackage{titlesec}
\usepackage[spanish]{babel}
\usepackage[utf8]{inputenc}
\usepackage{amsmath}
\usepackage{setspace}
\usepackage{svg}
\usepackage{hyperref}
\usepackage[backend=biber,style=numeric]{biblatex}
\addbibresource{references.bib}
\hypersetup{
    colorlinks=true,
    linkcolor=blue,      % color of internal links (sections, etc.)
    urlcolor=blue,       % color of external links
    pdftitle={Optimización energética de sistema híbrido con bomba de calor, suelo radiante, fotovoltaica y almacenamiento para vivienda},    % title
    pdfauthor={Luis D. Aranda Sánchez},     % author
    pdfkeywords={palabra1, palabra2, código1, etc.} % list of keywords
}

% Font change to Arial
\usepackage{helvet}
\renewcommand{\familydefault}{\sfdefault}

% Chapter titles in uppercase and larger font
\titleformat{\chapter}[hang]{\large\bfseries}{\thechapter.}{1em}{\MakeUppercase}
\titleformat{\section}[hang]{\bfseries}{\thesection.}{1em}{}
\titleformat{\subsection}[hang]{\bfseries}{\thesubsection.}{1em}{}

% Fancyhdr setup
\setlength{\headheight}{14.30174pt} % Adjust to recommended value, slightly larger for safety
\fancyhf{} % Clear all headers and footers
\fancyhead[LE]{\nouppercase{\leftmark}}
\fancyhead[RO]{Optimización energética para vivienda}
\fancyfoot[LE]{\thepage}
\fancyfoot[RE]{Escuela Técnica Superior de Ingenieros Industriales (UPM)}
\fancyfoot[LO]{Luis D. Aranda Sánchez}
\fancyfoot[RO]{\thepage}
\renewcommand{\headrulewidth}{0.4pt}
\renewcommand{\footrulewidth}{0.4pt}

\fancypagestyle{myfancy}{
    \fancyhf{} % Clear all headers and footers
    \fancyhead[LE]{\nouppercase{\leftmark}}
    \fancyhead[RO]{Optimización energética para vivienda}
    \fancyfoot[LE]{\thepage}
    \fancyfoot[RE]{Escuela Técnica Superior de Ingenieros Industriales (UPM)}
    \fancyfoot[LO]{Luis D. Aranda Sánchez}
    \fancyfoot[RO]{\thepage}
    \renewcommand{\headrulewidth}{0.4pt}
    \renewcommand{\footrulewidth}{0.4pt}
}

\fancypagestyle{simple}{
    \fancyhf{} % Clear all headers and footers
    \renewcommand{\headrulewidth}{0pt}
    \renewcommand{\footrulewidth}{0pt}
}

% Line spacing
\setstretch{1.2}

% Document starts here
\begin{document}

% Portada
\begin{titlepage}
    \centering
    {\scshape\LARGE Universidad Politécnica de Madrid \par}
    \vspace{1cm}
    {\scshape\Large Escuela Técnica Superior de Ingenieros Industriales\par}
    \vspace{1.5cm}
    {\huge\bfseries Optimización energética de sistema híbrido con bomba de calor, suelo radiante, fotovoltaica y almacenamiento para vivienda \par}
    \vspace{1.5cm}
    {\Large\bfseries Trabajo de Fin de Máster\par}
    \vspace{0.5cm}
    {\large Máster Universitario en Ingeniería de la Energía \par}
    \vspace{2cm}
    {\Large Luis D. Aranda Sánchez\par}
    \vfill
    Director: Javier Rodríguez Martín
    \vfill
    {\large Septiembre 6, 2024\par}
\end{titlepage}

% Resumen (máximo de 5 páginas, incluyendo al final Palabras clave)
\clearpage
\pagestyle{simple}
% \newpage
\chapter*{Resumen}
\addcontentsline{toc}{chapter}{Resumen}
\input{capitulos/resumen/main.tex}

% Índice (paginado)
\clearpage
\pagestyle{simple}
% \newpage
\tableofcontents

% Introducción (donde se incluya los antecedentes y justificación)
\clearpage
\pagestyle{myfancy}
\newpage
\chapter{Introducción}
\input{capitulos/introduccion/main.tex}

% Objetivos
\chapter{Objetivos}
\input{capitulos/objetivos/main.tex}

% Metodología
\chapter{Metodología}
\input{capitulos/metodologia/main.tex}

% Resultados y discusión (incluyendo la valoración de impactos y de aspectos de responsabilidad legal, ética y profesional relacionados con el trabajo)
\chapter{Resultados y Discusión}
\input{capitulos/resultados_discusion/main.tex}

% Conclusiones
\chapter{Conclusiones}
\input{capitulos/conclusiones/main.tex}

% Planificación temporal y presupuesto
\chapter{Planificación Temporal y Presupuesto}
\input{capitulos/planificacion_presupuesto/main.tex}

% Bibliografía
\newpage
\addcontentsline{toc}{chapter}{Bibliografía}
\printbibliography

\end{document}


% Metodología
\chapter{Metodología}
\documentclass[a4paper,11pt,twoside]{report}
\usepackage[left=25mm,right=25mm,top=25mm,bottom=25mm,includehead,includefoot,headsep=15mm,footskip=15mm]{geometry}
\usepackage{graphicx}
\usepackage{fancyhdr}
\usepackage{titlesec}
\usepackage[spanish]{babel}
\usepackage[utf8]{inputenc}
\usepackage{amsmath}
\usepackage{setspace}
\usepackage{svg}
\usepackage{hyperref}
\usepackage[backend=biber,style=numeric]{biblatex}
\addbibresource{references.bib}
\hypersetup{
    colorlinks=true,
    linkcolor=blue,      % color of internal links (sections, etc.)
    urlcolor=blue,       % color of external links
    pdftitle={Optimización energética de sistema híbrido con bomba de calor, suelo radiante, fotovoltaica y almacenamiento para vivienda},    % title
    pdfauthor={Luis D. Aranda Sánchez},     % author
    pdfkeywords={palabra1, palabra2, código1, etc.} % list of keywords
}

% Font change to Arial
\usepackage{helvet}
\renewcommand{\familydefault}{\sfdefault}

% Chapter titles in uppercase and larger font
\titleformat{\chapter}[hang]{\large\bfseries}{\thechapter.}{1em}{\MakeUppercase}
\titleformat{\section}[hang]{\bfseries}{\thesection.}{1em}{}
\titleformat{\subsection}[hang]{\bfseries}{\thesubsection.}{1em}{}

% Fancyhdr setup
\setlength{\headheight}{14.30174pt} % Adjust to recommended value, slightly larger for safety
\fancyhf{} % Clear all headers and footers
\fancyhead[LE]{\nouppercase{\leftmark}}
\fancyhead[RO]{Optimización energética para vivienda}
\fancyfoot[LE]{\thepage}
\fancyfoot[RE]{Escuela Técnica Superior de Ingenieros Industriales (UPM)}
\fancyfoot[LO]{Luis D. Aranda Sánchez}
\fancyfoot[RO]{\thepage}
\renewcommand{\headrulewidth}{0.4pt}
\renewcommand{\footrulewidth}{0.4pt}

\fancypagestyle{myfancy}{
    \fancyhf{} % Clear all headers and footers
    \fancyhead[LE]{\nouppercase{\leftmark}}
    \fancyhead[RO]{Optimización energética para vivienda}
    \fancyfoot[LE]{\thepage}
    \fancyfoot[RE]{Escuela Técnica Superior de Ingenieros Industriales (UPM)}
    \fancyfoot[LO]{Luis D. Aranda Sánchez}
    \fancyfoot[RO]{\thepage}
    \renewcommand{\headrulewidth}{0.4pt}
    \renewcommand{\footrulewidth}{0.4pt}
}

\fancypagestyle{simple}{
    \fancyhf{} % Clear all headers and footers
    \renewcommand{\headrulewidth}{0pt}
    \renewcommand{\footrulewidth}{0pt}
}

% Line spacing
\setstretch{1.2}

% Document starts here
\begin{document}

% Portada
\begin{titlepage}
    \centering
    {\scshape\LARGE Universidad Politécnica de Madrid \par}
    \vspace{1cm}
    {\scshape\Large Escuela Técnica Superior de Ingenieros Industriales\par}
    \vspace{1.5cm}
    {\huge\bfseries Optimización energética de sistema híbrido con bomba de calor, suelo radiante, fotovoltaica y almacenamiento para vivienda \par}
    \vspace{1.5cm}
    {\Large\bfseries Trabajo de Fin de Máster\par}
    \vspace{0.5cm}
    {\large Máster Universitario en Ingeniería de la Energía \par}
    \vspace{2cm}
    {\Large Luis D. Aranda Sánchez\par}
    \vfill
    Director: Javier Rodríguez Martín
    \vfill
    {\large Septiembre 6, 2024\par}
\end{titlepage}

% Resumen (máximo de 5 páginas, incluyendo al final Palabras clave)
\clearpage
\pagestyle{simple}
% \newpage
\chapter*{Resumen}
\addcontentsline{toc}{chapter}{Resumen}
\input{capitulos/resumen/main.tex}

% Índice (paginado)
\clearpage
\pagestyle{simple}
% \newpage
\tableofcontents

% Introducción (donde se incluya los antecedentes y justificación)
\clearpage
\pagestyle{myfancy}
\newpage
\chapter{Introducción}
\input{capitulos/introduccion/main.tex}

% Objetivos
\chapter{Objetivos}
\input{capitulos/objetivos/main.tex}

% Metodología
\chapter{Metodología}
\input{capitulos/metodologia/main.tex}

% Resultados y discusión (incluyendo la valoración de impactos y de aspectos de responsabilidad legal, ética y profesional relacionados con el trabajo)
\chapter{Resultados y Discusión}
\input{capitulos/resultados_discusion/main.tex}

% Conclusiones
\chapter{Conclusiones}
\input{capitulos/conclusiones/main.tex}

% Planificación temporal y presupuesto
\chapter{Planificación Temporal y Presupuesto}
\input{capitulos/planificacion_presupuesto/main.tex}

% Bibliografía
\newpage
\addcontentsline{toc}{chapter}{Bibliografía}
\printbibliography

\end{document}


% Resultados y discusión (incluyendo la valoración de impactos y de aspectos de responsabilidad legal, ética y profesional relacionados con el trabajo)
\chapter{Resultados y Discusión}
\documentclass[a4paper,11pt,twoside]{report}
\usepackage[left=25mm,right=25mm,top=25mm,bottom=25mm,includehead,includefoot,headsep=15mm,footskip=15mm]{geometry}
\usepackage{graphicx}
\usepackage{fancyhdr}
\usepackage{titlesec}
\usepackage[spanish]{babel}
\usepackage[utf8]{inputenc}
\usepackage{amsmath}
\usepackage{setspace}
\usepackage{svg}
\usepackage{hyperref}
\usepackage[backend=biber,style=numeric]{biblatex}
\addbibresource{references.bib}
\hypersetup{
    colorlinks=true,
    linkcolor=blue,      % color of internal links (sections, etc.)
    urlcolor=blue,       % color of external links
    pdftitle={Optimización energética de sistema híbrido con bomba de calor, suelo radiante, fotovoltaica y almacenamiento para vivienda},    % title
    pdfauthor={Luis D. Aranda Sánchez},     % author
    pdfkeywords={palabra1, palabra2, código1, etc.} % list of keywords
}

% Font change to Arial
\usepackage{helvet}
\renewcommand{\familydefault}{\sfdefault}

% Chapter titles in uppercase and larger font
\titleformat{\chapter}[hang]{\large\bfseries}{\thechapter.}{1em}{\MakeUppercase}
\titleformat{\section}[hang]{\bfseries}{\thesection.}{1em}{}
\titleformat{\subsection}[hang]{\bfseries}{\thesubsection.}{1em}{}

% Fancyhdr setup
\setlength{\headheight}{14.30174pt} % Adjust to recommended value, slightly larger for safety
\fancyhf{} % Clear all headers and footers
\fancyhead[LE]{\nouppercase{\leftmark}}
\fancyhead[RO]{Optimización energética para vivienda}
\fancyfoot[LE]{\thepage}
\fancyfoot[RE]{Escuela Técnica Superior de Ingenieros Industriales (UPM)}
\fancyfoot[LO]{Luis D. Aranda Sánchez}
\fancyfoot[RO]{\thepage}
\renewcommand{\headrulewidth}{0.4pt}
\renewcommand{\footrulewidth}{0.4pt}

\fancypagestyle{myfancy}{
    \fancyhf{} % Clear all headers and footers
    \fancyhead[LE]{\nouppercase{\leftmark}}
    \fancyhead[RO]{Optimización energética para vivienda}
    \fancyfoot[LE]{\thepage}
    \fancyfoot[RE]{Escuela Técnica Superior de Ingenieros Industriales (UPM)}
    \fancyfoot[LO]{Luis D. Aranda Sánchez}
    \fancyfoot[RO]{\thepage}
    \renewcommand{\headrulewidth}{0.4pt}
    \renewcommand{\footrulewidth}{0.4pt}
}

\fancypagestyle{simple}{
    \fancyhf{} % Clear all headers and footers
    \renewcommand{\headrulewidth}{0pt}
    \renewcommand{\footrulewidth}{0pt}
}

% Line spacing
\setstretch{1.2}

% Document starts here
\begin{document}

% Portada
\begin{titlepage}
    \centering
    {\scshape\LARGE Universidad Politécnica de Madrid \par}
    \vspace{1cm}
    {\scshape\Large Escuela Técnica Superior de Ingenieros Industriales\par}
    \vspace{1.5cm}
    {\huge\bfseries Optimización energética de sistema híbrido con bomba de calor, suelo radiante, fotovoltaica y almacenamiento para vivienda \par}
    \vspace{1.5cm}
    {\Large\bfseries Trabajo de Fin de Máster\par}
    \vspace{0.5cm}
    {\large Máster Universitario en Ingeniería de la Energía \par}
    \vspace{2cm}
    {\Large Luis D. Aranda Sánchez\par}
    \vfill
    Director: Javier Rodríguez Martín
    \vfill
    {\large Septiembre 6, 2024\par}
\end{titlepage}

% Resumen (máximo de 5 páginas, incluyendo al final Palabras clave)
\clearpage
\pagestyle{simple}
% \newpage
\chapter*{Resumen}
\addcontentsline{toc}{chapter}{Resumen}
\input{capitulos/resumen/main.tex}

% Índice (paginado)
\clearpage
\pagestyle{simple}
% \newpage
\tableofcontents

% Introducción (donde se incluya los antecedentes y justificación)
\clearpage
\pagestyle{myfancy}
\newpage
\chapter{Introducción}
\input{capitulos/introduccion/main.tex}

% Objetivos
\chapter{Objetivos}
\input{capitulos/objetivos/main.tex}

% Metodología
\chapter{Metodología}
\input{capitulos/metodologia/main.tex}

% Resultados y discusión (incluyendo la valoración de impactos y de aspectos de responsabilidad legal, ética y profesional relacionados con el trabajo)
\chapter{Resultados y Discusión}
\input{capitulos/resultados_discusion/main.tex}

% Conclusiones
\chapter{Conclusiones}
\input{capitulos/conclusiones/main.tex}

% Planificación temporal y presupuesto
\chapter{Planificación Temporal y Presupuesto}
\input{capitulos/planificacion_presupuesto/main.tex}

% Bibliografía
\newpage
\addcontentsline{toc}{chapter}{Bibliografía}
\printbibliography

\end{document}


% Conclusiones
\chapter{Conclusiones}
\documentclass[a4paper,11pt,twoside]{report}
\usepackage[left=25mm,right=25mm,top=25mm,bottom=25mm,includehead,includefoot,headsep=15mm,footskip=15mm]{geometry}
\usepackage{graphicx}
\usepackage{fancyhdr}
\usepackage{titlesec}
\usepackage[spanish]{babel}
\usepackage[utf8]{inputenc}
\usepackage{amsmath}
\usepackage{setspace}
\usepackage{svg}
\usepackage{hyperref}
\usepackage[backend=biber,style=numeric]{biblatex}
\addbibresource{references.bib}
\hypersetup{
    colorlinks=true,
    linkcolor=blue,      % color of internal links (sections, etc.)
    urlcolor=blue,       % color of external links
    pdftitle={Optimización energética de sistema híbrido con bomba de calor, suelo radiante, fotovoltaica y almacenamiento para vivienda},    % title
    pdfauthor={Luis D. Aranda Sánchez},     % author
    pdfkeywords={palabra1, palabra2, código1, etc.} % list of keywords
}

% Font change to Arial
\usepackage{helvet}
\renewcommand{\familydefault}{\sfdefault}

% Chapter titles in uppercase and larger font
\titleformat{\chapter}[hang]{\large\bfseries}{\thechapter.}{1em}{\MakeUppercase}
\titleformat{\section}[hang]{\bfseries}{\thesection.}{1em}{}
\titleformat{\subsection}[hang]{\bfseries}{\thesubsection.}{1em}{}

% Fancyhdr setup
\setlength{\headheight}{14.30174pt} % Adjust to recommended value, slightly larger for safety
\fancyhf{} % Clear all headers and footers
\fancyhead[LE]{\nouppercase{\leftmark}}
\fancyhead[RO]{Optimización energética para vivienda}
\fancyfoot[LE]{\thepage}
\fancyfoot[RE]{Escuela Técnica Superior de Ingenieros Industriales (UPM)}
\fancyfoot[LO]{Luis D. Aranda Sánchez}
\fancyfoot[RO]{\thepage}
\renewcommand{\headrulewidth}{0.4pt}
\renewcommand{\footrulewidth}{0.4pt}

\fancypagestyle{myfancy}{
    \fancyhf{} % Clear all headers and footers
    \fancyhead[LE]{\nouppercase{\leftmark}}
    \fancyhead[RO]{Optimización energética para vivienda}
    \fancyfoot[LE]{\thepage}
    \fancyfoot[RE]{Escuela Técnica Superior de Ingenieros Industriales (UPM)}
    \fancyfoot[LO]{Luis D. Aranda Sánchez}
    \fancyfoot[RO]{\thepage}
    \renewcommand{\headrulewidth}{0.4pt}
    \renewcommand{\footrulewidth}{0.4pt}
}

\fancypagestyle{simple}{
    \fancyhf{} % Clear all headers and footers
    \renewcommand{\headrulewidth}{0pt}
    \renewcommand{\footrulewidth}{0pt}
}

% Line spacing
\setstretch{1.2}

% Document starts here
\begin{document}

% Portada
\begin{titlepage}
    \centering
    {\scshape\LARGE Universidad Politécnica de Madrid \par}
    \vspace{1cm}
    {\scshape\Large Escuela Técnica Superior de Ingenieros Industriales\par}
    \vspace{1.5cm}
    {\huge\bfseries Optimización energética de sistema híbrido con bomba de calor, suelo radiante, fotovoltaica y almacenamiento para vivienda \par}
    \vspace{1.5cm}
    {\Large\bfseries Trabajo de Fin de Máster\par}
    \vspace{0.5cm}
    {\large Máster Universitario en Ingeniería de la Energía \par}
    \vspace{2cm}
    {\Large Luis D. Aranda Sánchez\par}
    \vfill
    Director: Javier Rodríguez Martín
    \vfill
    {\large Septiembre 6, 2024\par}
\end{titlepage}

% Resumen (máximo de 5 páginas, incluyendo al final Palabras clave)
\clearpage
\pagestyle{simple}
% \newpage
\chapter*{Resumen}
\addcontentsline{toc}{chapter}{Resumen}
\input{capitulos/resumen/main.tex}

% Índice (paginado)
\clearpage
\pagestyle{simple}
% \newpage
\tableofcontents

% Introducción (donde se incluya los antecedentes y justificación)
\clearpage
\pagestyle{myfancy}
\newpage
\chapter{Introducción}
\input{capitulos/introduccion/main.tex}

% Objetivos
\chapter{Objetivos}
\input{capitulos/objetivos/main.tex}

% Metodología
\chapter{Metodología}
\input{capitulos/metodologia/main.tex}

% Resultados y discusión (incluyendo la valoración de impactos y de aspectos de responsabilidad legal, ética y profesional relacionados con el trabajo)
\chapter{Resultados y Discusión}
\input{capitulos/resultados_discusion/main.tex}

% Conclusiones
\chapter{Conclusiones}
\input{capitulos/conclusiones/main.tex}

% Planificación temporal y presupuesto
\chapter{Planificación Temporal y Presupuesto}
\input{capitulos/planificacion_presupuesto/main.tex}

% Bibliografía
\newpage
\addcontentsline{toc}{chapter}{Bibliografía}
\printbibliography

\end{document}


% Planificación temporal y presupuesto
\chapter{Planificación Temporal y Presupuesto}
\documentclass[a4paper,11pt,twoside]{report}
\usepackage[left=25mm,right=25mm,top=25mm,bottom=25mm,includehead,includefoot,headsep=15mm,footskip=15mm]{geometry}
\usepackage{graphicx}
\usepackage{fancyhdr}
\usepackage{titlesec}
\usepackage[spanish]{babel}
\usepackage[utf8]{inputenc}
\usepackage{amsmath}
\usepackage{setspace}
\usepackage{svg}
\usepackage{hyperref}
\usepackage[backend=biber,style=numeric]{biblatex}
\addbibresource{references.bib}
\hypersetup{
    colorlinks=true,
    linkcolor=blue,      % color of internal links (sections, etc.)
    urlcolor=blue,       % color of external links
    pdftitle={Optimización energética de sistema híbrido con bomba de calor, suelo radiante, fotovoltaica y almacenamiento para vivienda},    % title
    pdfauthor={Luis D. Aranda Sánchez},     % author
    pdfkeywords={palabra1, palabra2, código1, etc.} % list of keywords
}

% Font change to Arial
\usepackage{helvet}
\renewcommand{\familydefault}{\sfdefault}

% Chapter titles in uppercase and larger font
\titleformat{\chapter}[hang]{\large\bfseries}{\thechapter.}{1em}{\MakeUppercase}
\titleformat{\section}[hang]{\bfseries}{\thesection.}{1em}{}
\titleformat{\subsection}[hang]{\bfseries}{\thesubsection.}{1em}{}

% Fancyhdr setup
\setlength{\headheight}{14.30174pt} % Adjust to recommended value, slightly larger for safety
\fancyhf{} % Clear all headers and footers
\fancyhead[LE]{\nouppercase{\leftmark}}
\fancyhead[RO]{Optimización energética para vivienda}
\fancyfoot[LE]{\thepage}
\fancyfoot[RE]{Escuela Técnica Superior de Ingenieros Industriales (UPM)}
\fancyfoot[LO]{Luis D. Aranda Sánchez}
\fancyfoot[RO]{\thepage}
\renewcommand{\headrulewidth}{0.4pt}
\renewcommand{\footrulewidth}{0.4pt}

\fancypagestyle{myfancy}{
    \fancyhf{} % Clear all headers and footers
    \fancyhead[LE]{\nouppercase{\leftmark}}
    \fancyhead[RO]{Optimización energética para vivienda}
    \fancyfoot[LE]{\thepage}
    \fancyfoot[RE]{Escuela Técnica Superior de Ingenieros Industriales (UPM)}
    \fancyfoot[LO]{Luis D. Aranda Sánchez}
    \fancyfoot[RO]{\thepage}
    \renewcommand{\headrulewidth}{0.4pt}
    \renewcommand{\footrulewidth}{0.4pt}
}

\fancypagestyle{simple}{
    \fancyhf{} % Clear all headers and footers
    \renewcommand{\headrulewidth}{0pt}
    \renewcommand{\footrulewidth}{0pt}
}

% Line spacing
\setstretch{1.2}

% Document starts here
\begin{document}

% Portada
\begin{titlepage}
    \centering
    {\scshape\LARGE Universidad Politécnica de Madrid \par}
    \vspace{1cm}
    {\scshape\Large Escuela Técnica Superior de Ingenieros Industriales\par}
    \vspace{1.5cm}
    {\huge\bfseries Optimización energética de sistema híbrido con bomba de calor, suelo radiante, fotovoltaica y almacenamiento para vivienda \par}
    \vspace{1.5cm}
    {\Large\bfseries Trabajo de Fin de Máster\par}
    \vspace{0.5cm}
    {\large Máster Universitario en Ingeniería de la Energía \par}
    \vspace{2cm}
    {\Large Luis D. Aranda Sánchez\par}
    \vfill
    Director: Javier Rodríguez Martín
    \vfill
    {\large Septiembre 6, 2024\par}
\end{titlepage}

% Resumen (máximo de 5 páginas, incluyendo al final Palabras clave)
\clearpage
\pagestyle{simple}
% \newpage
\chapter*{Resumen}
\addcontentsline{toc}{chapter}{Resumen}
\input{capitulos/resumen/main.tex}

% Índice (paginado)
\clearpage
\pagestyle{simple}
% \newpage
\tableofcontents

% Introducción (donde se incluya los antecedentes y justificación)
\clearpage
\pagestyle{myfancy}
\newpage
\chapter{Introducción}
\input{capitulos/introduccion/main.tex}

% Objetivos
\chapter{Objetivos}
\input{capitulos/objetivos/main.tex}

% Metodología
\chapter{Metodología}
\input{capitulos/metodologia/main.tex}

% Resultados y discusión (incluyendo la valoración de impactos y de aspectos de responsabilidad legal, ética y profesional relacionados con el trabajo)
\chapter{Resultados y Discusión}
\input{capitulos/resultados_discusion/main.tex}

% Conclusiones
\chapter{Conclusiones}
\input{capitulos/conclusiones/main.tex}

% Planificación temporal y presupuesto
\chapter{Planificación Temporal y Presupuesto}
\input{capitulos/planificacion_presupuesto/main.tex}

% Bibliografía
\newpage
\addcontentsline{toc}{chapter}{Bibliografía}
\printbibliography

\end{document}


% Bibliografía
\newpage
\addcontentsline{toc}{chapter}{Bibliografía}
\printbibliography

\end{document}


% Bibliografía
\newpage
\addcontentsline{toc}{chapter}{Bibliografía}
\printbibliography

\end{document}


% Planificación temporal y presupuesto
\cleardoublepage
\chapter{Planificación Temporal y Presupuesto}
\documentclass[a4paper,11pt,twoside]{report}
\usepackage[left=25mm,right=25mm,top=25mm,bottom=25mm,includehead,includefoot,headsep=15mm,footskip=15mm]{geometry}
\usepackage{graphicx}
\usepackage{fancyhdr}
\usepackage{titlesec}
\usepackage[spanish]{babel}
\usepackage[utf8]{inputenc}
\usepackage{amsmath}
\usepackage{setspace}
\usepackage{svg}
\usepackage{hyperref}
\usepackage[backend=biber,style=numeric]{biblatex}
\addbibresource{references.bib}
\hypersetup{
    colorlinks=true,
    linkcolor=blue,      % color of internal links (sections, etc.)
    urlcolor=blue,       % color of external links
    pdftitle={Optimización energética de sistema híbrido con bomba de calor, suelo radiante, fotovoltaica y almacenamiento para vivienda},    % title
    pdfauthor={Luis D. Aranda Sánchez},     % author
    pdfkeywords={palabra1, palabra2, código1, etc.} % list of keywords
}

% Font change to Arial
\usepackage{helvet}
\renewcommand{\familydefault}{\sfdefault}

% Chapter titles in uppercase and larger font
\titleformat{\chapter}[hang]{\large\bfseries}{\thechapter.}{1em}{\MakeUppercase}
\titleformat{\section}[hang]{\bfseries}{\thesection.}{1em}{}
\titleformat{\subsection}[hang]{\bfseries}{\thesubsection.}{1em}{}

% Fancyhdr setup
\setlength{\headheight}{14.30174pt} % Adjust to recommended value, slightly larger for safety
\fancyhf{} % Clear all headers and footers
\fancyhead[LE]{\nouppercase{\leftmark}}
\fancyhead[RO]{Optimización energética para vivienda}
\fancyfoot[LE]{\thepage}
\fancyfoot[RE]{Escuela Técnica Superior de Ingenieros Industriales (UPM)}
\fancyfoot[LO]{Luis D. Aranda Sánchez}
\fancyfoot[RO]{\thepage}
\renewcommand{\headrulewidth}{0.4pt}
\renewcommand{\footrulewidth}{0.4pt}

\fancypagestyle{myfancy}{
    \fancyhf{} % Clear all headers and footers
    \fancyhead[LE]{\nouppercase{\leftmark}}
    \fancyhead[RO]{Optimización energética para vivienda}
    \fancyfoot[LE]{\thepage}
    \fancyfoot[RE]{Escuela Técnica Superior de Ingenieros Industriales (UPM)}
    \fancyfoot[LO]{Luis D. Aranda Sánchez}
    \fancyfoot[RO]{\thepage}
    \renewcommand{\headrulewidth}{0.4pt}
    \renewcommand{\footrulewidth}{0.4pt}
}

\fancypagestyle{simple}{
    \fancyhf{} % Clear all headers and footers
    \renewcommand{\headrulewidth}{0pt}
    \renewcommand{\footrulewidth}{0pt}
}

% Line spacing
\setstretch{1.2}

% Document starts here
\begin{document}

% Portada
\begin{titlepage}
    \centering
    {\scshape\LARGE Universidad Politécnica de Madrid \par}
    \vspace{1cm}
    {\scshape\Large Escuela Técnica Superior de Ingenieros Industriales\par}
    \vspace{1.5cm}
    {\huge\bfseries Optimización energética de sistema híbrido con bomba de calor, suelo radiante, fotovoltaica y almacenamiento para vivienda \par}
    \vspace{1.5cm}
    {\Large\bfseries Trabajo de Fin de Máster\par}
    \vspace{0.5cm}
    {\large Máster Universitario en Ingeniería de la Energía \par}
    \vspace{2cm}
    {\Large Luis D. Aranda Sánchez\par}
    \vfill
    Director: Javier Rodríguez Martín
    \vfill
    {\large Septiembre 6, 2024\par}
\end{titlepage}

% Resumen (máximo de 5 páginas, incluyendo al final Palabras clave)
\clearpage
\pagestyle{simple}
% \newpage
\chapter*{Resumen}
\addcontentsline{toc}{chapter}{Resumen}
\documentclass[a4paper,11pt,twoside]{report}
\usepackage[left=25mm,right=25mm,top=25mm,bottom=25mm,includehead,includefoot,headsep=15mm,footskip=15mm]{geometry}
\usepackage{graphicx}
\usepackage{fancyhdr}
\usepackage{titlesec}
\usepackage[spanish]{babel}
\usepackage[utf8]{inputenc}
\usepackage{amsmath}
\usepackage{setspace}
\usepackage{svg}
\usepackage{hyperref}
\usepackage[backend=biber,style=numeric]{biblatex}
\addbibresource{references.bib}
\hypersetup{
    colorlinks=true,
    linkcolor=blue,      % color of internal links (sections, etc.)
    urlcolor=blue,       % color of external links
    pdftitle={Optimización energética de sistema híbrido con bomba de calor, suelo radiante, fotovoltaica y almacenamiento para vivienda},    % title
    pdfauthor={Luis D. Aranda Sánchez},     % author
    pdfkeywords={palabra1, palabra2, código1, etc.} % list of keywords
}

% Font change to Arial
\usepackage{helvet}
\renewcommand{\familydefault}{\sfdefault}

% Chapter titles in uppercase and larger font
\titleformat{\chapter}[hang]{\large\bfseries}{\thechapter.}{1em}{\MakeUppercase}
\titleformat{\section}[hang]{\bfseries}{\thesection.}{1em}{}
\titleformat{\subsection}[hang]{\bfseries}{\thesubsection.}{1em}{}

% Fancyhdr setup
\setlength{\headheight}{14.30174pt} % Adjust to recommended value, slightly larger for safety
\fancyhf{} % Clear all headers and footers
\fancyhead[LE]{\nouppercase{\leftmark}}
\fancyhead[RO]{Optimización energética para vivienda}
\fancyfoot[LE]{\thepage}
\fancyfoot[RE]{Escuela Técnica Superior de Ingenieros Industriales (UPM)}
\fancyfoot[LO]{Luis D. Aranda Sánchez}
\fancyfoot[RO]{\thepage}
\renewcommand{\headrulewidth}{0.4pt}
\renewcommand{\footrulewidth}{0.4pt}

\fancypagestyle{myfancy}{
    \fancyhf{} % Clear all headers and footers
    \fancyhead[LE]{\nouppercase{\leftmark}}
    \fancyhead[RO]{Optimización energética para vivienda}
    \fancyfoot[LE]{\thepage}
    \fancyfoot[RE]{Escuela Técnica Superior de Ingenieros Industriales (UPM)}
    \fancyfoot[LO]{Luis D. Aranda Sánchez}
    \fancyfoot[RO]{\thepage}
    \renewcommand{\headrulewidth}{0.4pt}
    \renewcommand{\footrulewidth}{0.4pt}
}

\fancypagestyle{simple}{
    \fancyhf{} % Clear all headers and footers
    \renewcommand{\headrulewidth}{0pt}
    \renewcommand{\footrulewidth}{0pt}
}

% Line spacing
\setstretch{1.2}

% Document starts here
\begin{document}

% Portada
\begin{titlepage}
    \centering
    {\scshape\LARGE Universidad Politécnica de Madrid \par}
    \vspace{1cm}
    {\scshape\Large Escuela Técnica Superior de Ingenieros Industriales\par}
    \vspace{1.5cm}
    {\huge\bfseries Optimización energética de sistema híbrido con bomba de calor, suelo radiante, fotovoltaica y almacenamiento para vivienda \par}
    \vspace{1.5cm}
    {\Large\bfseries Trabajo de Fin de Máster\par}
    \vspace{0.5cm}
    {\large Máster Universitario en Ingeniería de la Energía \par}
    \vspace{2cm}
    {\Large Luis D. Aranda Sánchez\par}
    \vfill
    Director: Javier Rodríguez Martín
    \vfill
    {\large Septiembre 6, 2024\par}
\end{titlepage}

% Resumen (máximo de 5 páginas, incluyendo al final Palabras clave)
\clearpage
\pagestyle{simple}
% \newpage
\chapter*{Resumen}
\addcontentsline{toc}{chapter}{Resumen}
\documentclass[a4paper,11pt,twoside]{report}
\usepackage[left=25mm,right=25mm,top=25mm,bottom=25mm,includehead,includefoot,headsep=15mm,footskip=15mm]{geometry}
\usepackage{graphicx}
\usepackage{fancyhdr}
\usepackage{titlesec}
\usepackage[spanish]{babel}
\usepackage[utf8]{inputenc}
\usepackage{amsmath}
\usepackage{setspace}
\usepackage{svg}
\usepackage{hyperref}
\usepackage[backend=biber,style=numeric]{biblatex}
\addbibresource{references.bib}
\hypersetup{
    colorlinks=true,
    linkcolor=blue,      % color of internal links (sections, etc.)
    urlcolor=blue,       % color of external links
    pdftitle={Optimización energética de sistema híbrido con bomba de calor, suelo radiante, fotovoltaica y almacenamiento para vivienda},    % title
    pdfauthor={Luis D. Aranda Sánchez},     % author
    pdfkeywords={palabra1, palabra2, código1, etc.} % list of keywords
}

% Font change to Arial
\usepackage{helvet}
\renewcommand{\familydefault}{\sfdefault}

% Chapter titles in uppercase and larger font
\titleformat{\chapter}[hang]{\large\bfseries}{\thechapter.}{1em}{\MakeUppercase}
\titleformat{\section}[hang]{\bfseries}{\thesection.}{1em}{}
\titleformat{\subsection}[hang]{\bfseries}{\thesubsection.}{1em}{}

% Fancyhdr setup
\setlength{\headheight}{14.30174pt} % Adjust to recommended value, slightly larger for safety
\fancyhf{} % Clear all headers and footers
\fancyhead[LE]{\nouppercase{\leftmark}}
\fancyhead[RO]{Optimización energética para vivienda}
\fancyfoot[LE]{\thepage}
\fancyfoot[RE]{Escuela Técnica Superior de Ingenieros Industriales (UPM)}
\fancyfoot[LO]{Luis D. Aranda Sánchez}
\fancyfoot[RO]{\thepage}
\renewcommand{\headrulewidth}{0.4pt}
\renewcommand{\footrulewidth}{0.4pt}

\fancypagestyle{myfancy}{
    \fancyhf{} % Clear all headers and footers
    \fancyhead[LE]{\nouppercase{\leftmark}}
    \fancyhead[RO]{Optimización energética para vivienda}
    \fancyfoot[LE]{\thepage}
    \fancyfoot[RE]{Escuela Técnica Superior de Ingenieros Industriales (UPM)}
    \fancyfoot[LO]{Luis D. Aranda Sánchez}
    \fancyfoot[RO]{\thepage}
    \renewcommand{\headrulewidth}{0.4pt}
    \renewcommand{\footrulewidth}{0.4pt}
}

\fancypagestyle{simple}{
    \fancyhf{} % Clear all headers and footers
    \renewcommand{\headrulewidth}{0pt}
    \renewcommand{\footrulewidth}{0pt}
}

% Line spacing
\setstretch{1.2}

% Document starts here
\begin{document}

% Portada
\begin{titlepage}
    \centering
    {\scshape\LARGE Universidad Politécnica de Madrid \par}
    \vspace{1cm}
    {\scshape\Large Escuela Técnica Superior de Ingenieros Industriales\par}
    \vspace{1.5cm}
    {\huge\bfseries Optimización energética de sistema híbrido con bomba de calor, suelo radiante, fotovoltaica y almacenamiento para vivienda \par}
    \vspace{1.5cm}
    {\Large\bfseries Trabajo de Fin de Máster\par}
    \vspace{0.5cm}
    {\large Máster Universitario en Ingeniería de la Energía \par}
    \vspace{2cm}
    {\Large Luis D. Aranda Sánchez\par}
    \vfill
    Director: Javier Rodríguez Martín
    \vfill
    {\large Septiembre 6, 2024\par}
\end{titlepage}

% Resumen (máximo de 5 páginas, incluyendo al final Palabras clave)
\clearpage
\pagestyle{simple}
% \newpage
\chapter*{Resumen}
\addcontentsline{toc}{chapter}{Resumen}
\input{capitulos/resumen/main.tex}

% Índice (paginado)
\clearpage
\pagestyle{simple}
% \newpage
\tableofcontents

% Introducción (donde se incluya los antecedentes y justificación)
\clearpage
\pagestyle{myfancy}
\newpage
\chapter{Introducción}
\input{capitulos/introduccion/main.tex}

% Objetivos
\chapter{Objetivos}
\input{capitulos/objetivos/main.tex}

% Metodología
\chapter{Metodología}
\input{capitulos/metodologia/main.tex}

% Resultados y discusión (incluyendo la valoración de impactos y de aspectos de responsabilidad legal, ética y profesional relacionados con el trabajo)
\chapter{Resultados y Discusión}
\input{capitulos/resultados_discusion/main.tex}

% Conclusiones
\chapter{Conclusiones}
\input{capitulos/conclusiones/main.tex}

% Planificación temporal y presupuesto
\chapter{Planificación Temporal y Presupuesto}
\input{capitulos/planificacion_presupuesto/main.tex}

% Bibliografía
\newpage
\addcontentsline{toc}{chapter}{Bibliografía}
\printbibliography

\end{document}


% Índice (paginado)
\clearpage
\pagestyle{simple}
% \newpage
\tableofcontents

% Introducción (donde se incluya los antecedentes y justificación)
\clearpage
\pagestyle{myfancy}
\newpage
\chapter{Introducción}
\documentclass[a4paper,11pt,twoside]{report}
\usepackage[left=25mm,right=25mm,top=25mm,bottom=25mm,includehead,includefoot,headsep=15mm,footskip=15mm]{geometry}
\usepackage{graphicx}
\usepackage{fancyhdr}
\usepackage{titlesec}
\usepackage[spanish]{babel}
\usepackage[utf8]{inputenc}
\usepackage{amsmath}
\usepackage{setspace}
\usepackage{svg}
\usepackage{hyperref}
\usepackage[backend=biber,style=numeric]{biblatex}
\addbibresource{references.bib}
\hypersetup{
    colorlinks=true,
    linkcolor=blue,      % color of internal links (sections, etc.)
    urlcolor=blue,       % color of external links
    pdftitle={Optimización energética de sistema híbrido con bomba de calor, suelo radiante, fotovoltaica y almacenamiento para vivienda},    % title
    pdfauthor={Luis D. Aranda Sánchez},     % author
    pdfkeywords={palabra1, palabra2, código1, etc.} % list of keywords
}

% Font change to Arial
\usepackage{helvet}
\renewcommand{\familydefault}{\sfdefault}

% Chapter titles in uppercase and larger font
\titleformat{\chapter}[hang]{\large\bfseries}{\thechapter.}{1em}{\MakeUppercase}
\titleformat{\section}[hang]{\bfseries}{\thesection.}{1em}{}
\titleformat{\subsection}[hang]{\bfseries}{\thesubsection.}{1em}{}

% Fancyhdr setup
\setlength{\headheight}{14.30174pt} % Adjust to recommended value, slightly larger for safety
\fancyhf{} % Clear all headers and footers
\fancyhead[LE]{\nouppercase{\leftmark}}
\fancyhead[RO]{Optimización energética para vivienda}
\fancyfoot[LE]{\thepage}
\fancyfoot[RE]{Escuela Técnica Superior de Ingenieros Industriales (UPM)}
\fancyfoot[LO]{Luis D. Aranda Sánchez}
\fancyfoot[RO]{\thepage}
\renewcommand{\headrulewidth}{0.4pt}
\renewcommand{\footrulewidth}{0.4pt}

\fancypagestyle{myfancy}{
    \fancyhf{} % Clear all headers and footers
    \fancyhead[LE]{\nouppercase{\leftmark}}
    \fancyhead[RO]{Optimización energética para vivienda}
    \fancyfoot[LE]{\thepage}
    \fancyfoot[RE]{Escuela Técnica Superior de Ingenieros Industriales (UPM)}
    \fancyfoot[LO]{Luis D. Aranda Sánchez}
    \fancyfoot[RO]{\thepage}
    \renewcommand{\headrulewidth}{0.4pt}
    \renewcommand{\footrulewidth}{0.4pt}
}

\fancypagestyle{simple}{
    \fancyhf{} % Clear all headers and footers
    \renewcommand{\headrulewidth}{0pt}
    \renewcommand{\footrulewidth}{0pt}
}

% Line spacing
\setstretch{1.2}

% Document starts here
\begin{document}

% Portada
\begin{titlepage}
    \centering
    {\scshape\LARGE Universidad Politécnica de Madrid \par}
    \vspace{1cm}
    {\scshape\Large Escuela Técnica Superior de Ingenieros Industriales\par}
    \vspace{1.5cm}
    {\huge\bfseries Optimización energética de sistema híbrido con bomba de calor, suelo radiante, fotovoltaica y almacenamiento para vivienda \par}
    \vspace{1.5cm}
    {\Large\bfseries Trabajo de Fin de Máster\par}
    \vspace{0.5cm}
    {\large Máster Universitario en Ingeniería de la Energía \par}
    \vspace{2cm}
    {\Large Luis D. Aranda Sánchez\par}
    \vfill
    Director: Javier Rodríguez Martín
    \vfill
    {\large Septiembre 6, 2024\par}
\end{titlepage}

% Resumen (máximo de 5 páginas, incluyendo al final Palabras clave)
\clearpage
\pagestyle{simple}
% \newpage
\chapter*{Resumen}
\addcontentsline{toc}{chapter}{Resumen}
\input{capitulos/resumen/main.tex}

% Índice (paginado)
\clearpage
\pagestyle{simple}
% \newpage
\tableofcontents

% Introducción (donde se incluya los antecedentes y justificación)
\clearpage
\pagestyle{myfancy}
\newpage
\chapter{Introducción}
\input{capitulos/introduccion/main.tex}

% Objetivos
\chapter{Objetivos}
\input{capitulos/objetivos/main.tex}

% Metodología
\chapter{Metodología}
\input{capitulos/metodologia/main.tex}

% Resultados y discusión (incluyendo la valoración de impactos y de aspectos de responsabilidad legal, ética y profesional relacionados con el trabajo)
\chapter{Resultados y Discusión}
\input{capitulos/resultados_discusion/main.tex}

% Conclusiones
\chapter{Conclusiones}
\input{capitulos/conclusiones/main.tex}

% Planificación temporal y presupuesto
\chapter{Planificación Temporal y Presupuesto}
\input{capitulos/planificacion_presupuesto/main.tex}

% Bibliografía
\newpage
\addcontentsline{toc}{chapter}{Bibliografía}
\printbibliography

\end{document}


% Objetivos
\chapter{Objetivos}
\documentclass[a4paper,11pt,twoside]{report}
\usepackage[left=25mm,right=25mm,top=25mm,bottom=25mm,includehead,includefoot,headsep=15mm,footskip=15mm]{geometry}
\usepackage{graphicx}
\usepackage{fancyhdr}
\usepackage{titlesec}
\usepackage[spanish]{babel}
\usepackage[utf8]{inputenc}
\usepackage{amsmath}
\usepackage{setspace}
\usepackage{svg}
\usepackage{hyperref}
\usepackage[backend=biber,style=numeric]{biblatex}
\addbibresource{references.bib}
\hypersetup{
    colorlinks=true,
    linkcolor=blue,      % color of internal links (sections, etc.)
    urlcolor=blue,       % color of external links
    pdftitle={Optimización energética de sistema híbrido con bomba de calor, suelo radiante, fotovoltaica y almacenamiento para vivienda},    % title
    pdfauthor={Luis D. Aranda Sánchez},     % author
    pdfkeywords={palabra1, palabra2, código1, etc.} % list of keywords
}

% Font change to Arial
\usepackage{helvet}
\renewcommand{\familydefault}{\sfdefault}

% Chapter titles in uppercase and larger font
\titleformat{\chapter}[hang]{\large\bfseries}{\thechapter.}{1em}{\MakeUppercase}
\titleformat{\section}[hang]{\bfseries}{\thesection.}{1em}{}
\titleformat{\subsection}[hang]{\bfseries}{\thesubsection.}{1em}{}

% Fancyhdr setup
\setlength{\headheight}{14.30174pt} % Adjust to recommended value, slightly larger for safety
\fancyhf{} % Clear all headers and footers
\fancyhead[LE]{\nouppercase{\leftmark}}
\fancyhead[RO]{Optimización energética para vivienda}
\fancyfoot[LE]{\thepage}
\fancyfoot[RE]{Escuela Técnica Superior de Ingenieros Industriales (UPM)}
\fancyfoot[LO]{Luis D. Aranda Sánchez}
\fancyfoot[RO]{\thepage}
\renewcommand{\headrulewidth}{0.4pt}
\renewcommand{\footrulewidth}{0.4pt}

\fancypagestyle{myfancy}{
    \fancyhf{} % Clear all headers and footers
    \fancyhead[LE]{\nouppercase{\leftmark}}
    \fancyhead[RO]{Optimización energética para vivienda}
    \fancyfoot[LE]{\thepage}
    \fancyfoot[RE]{Escuela Técnica Superior de Ingenieros Industriales (UPM)}
    \fancyfoot[LO]{Luis D. Aranda Sánchez}
    \fancyfoot[RO]{\thepage}
    \renewcommand{\headrulewidth}{0.4pt}
    \renewcommand{\footrulewidth}{0.4pt}
}

\fancypagestyle{simple}{
    \fancyhf{} % Clear all headers and footers
    \renewcommand{\headrulewidth}{0pt}
    \renewcommand{\footrulewidth}{0pt}
}

% Line spacing
\setstretch{1.2}

% Document starts here
\begin{document}

% Portada
\begin{titlepage}
    \centering
    {\scshape\LARGE Universidad Politécnica de Madrid \par}
    \vspace{1cm}
    {\scshape\Large Escuela Técnica Superior de Ingenieros Industriales\par}
    \vspace{1.5cm}
    {\huge\bfseries Optimización energética de sistema híbrido con bomba de calor, suelo radiante, fotovoltaica y almacenamiento para vivienda \par}
    \vspace{1.5cm}
    {\Large\bfseries Trabajo de Fin de Máster\par}
    \vspace{0.5cm}
    {\large Máster Universitario en Ingeniería de la Energía \par}
    \vspace{2cm}
    {\Large Luis D. Aranda Sánchez\par}
    \vfill
    Director: Javier Rodríguez Martín
    \vfill
    {\large Septiembre 6, 2024\par}
\end{titlepage}

% Resumen (máximo de 5 páginas, incluyendo al final Palabras clave)
\clearpage
\pagestyle{simple}
% \newpage
\chapter*{Resumen}
\addcontentsline{toc}{chapter}{Resumen}
\input{capitulos/resumen/main.tex}

% Índice (paginado)
\clearpage
\pagestyle{simple}
% \newpage
\tableofcontents

% Introducción (donde se incluya los antecedentes y justificación)
\clearpage
\pagestyle{myfancy}
\newpage
\chapter{Introducción}
\input{capitulos/introduccion/main.tex}

% Objetivos
\chapter{Objetivos}
\input{capitulos/objetivos/main.tex}

% Metodología
\chapter{Metodología}
\input{capitulos/metodologia/main.tex}

% Resultados y discusión (incluyendo la valoración de impactos y de aspectos de responsabilidad legal, ética y profesional relacionados con el trabajo)
\chapter{Resultados y Discusión}
\input{capitulos/resultados_discusion/main.tex}

% Conclusiones
\chapter{Conclusiones}
\input{capitulos/conclusiones/main.tex}

% Planificación temporal y presupuesto
\chapter{Planificación Temporal y Presupuesto}
\input{capitulos/planificacion_presupuesto/main.tex}

% Bibliografía
\newpage
\addcontentsline{toc}{chapter}{Bibliografía}
\printbibliography

\end{document}


% Metodología
\chapter{Metodología}
\documentclass[a4paper,11pt,twoside]{report}
\usepackage[left=25mm,right=25mm,top=25mm,bottom=25mm,includehead,includefoot,headsep=15mm,footskip=15mm]{geometry}
\usepackage{graphicx}
\usepackage{fancyhdr}
\usepackage{titlesec}
\usepackage[spanish]{babel}
\usepackage[utf8]{inputenc}
\usepackage{amsmath}
\usepackage{setspace}
\usepackage{svg}
\usepackage{hyperref}
\usepackage[backend=biber,style=numeric]{biblatex}
\addbibresource{references.bib}
\hypersetup{
    colorlinks=true,
    linkcolor=blue,      % color of internal links (sections, etc.)
    urlcolor=blue,       % color of external links
    pdftitle={Optimización energética de sistema híbrido con bomba de calor, suelo radiante, fotovoltaica y almacenamiento para vivienda},    % title
    pdfauthor={Luis D. Aranda Sánchez},     % author
    pdfkeywords={palabra1, palabra2, código1, etc.} % list of keywords
}

% Font change to Arial
\usepackage{helvet}
\renewcommand{\familydefault}{\sfdefault}

% Chapter titles in uppercase and larger font
\titleformat{\chapter}[hang]{\large\bfseries}{\thechapter.}{1em}{\MakeUppercase}
\titleformat{\section}[hang]{\bfseries}{\thesection.}{1em}{}
\titleformat{\subsection}[hang]{\bfseries}{\thesubsection.}{1em}{}

% Fancyhdr setup
\setlength{\headheight}{14.30174pt} % Adjust to recommended value, slightly larger for safety
\fancyhf{} % Clear all headers and footers
\fancyhead[LE]{\nouppercase{\leftmark}}
\fancyhead[RO]{Optimización energética para vivienda}
\fancyfoot[LE]{\thepage}
\fancyfoot[RE]{Escuela Técnica Superior de Ingenieros Industriales (UPM)}
\fancyfoot[LO]{Luis D. Aranda Sánchez}
\fancyfoot[RO]{\thepage}
\renewcommand{\headrulewidth}{0.4pt}
\renewcommand{\footrulewidth}{0.4pt}

\fancypagestyle{myfancy}{
    \fancyhf{} % Clear all headers and footers
    \fancyhead[LE]{\nouppercase{\leftmark}}
    \fancyhead[RO]{Optimización energética para vivienda}
    \fancyfoot[LE]{\thepage}
    \fancyfoot[RE]{Escuela Técnica Superior de Ingenieros Industriales (UPM)}
    \fancyfoot[LO]{Luis D. Aranda Sánchez}
    \fancyfoot[RO]{\thepage}
    \renewcommand{\headrulewidth}{0.4pt}
    \renewcommand{\footrulewidth}{0.4pt}
}

\fancypagestyle{simple}{
    \fancyhf{} % Clear all headers and footers
    \renewcommand{\headrulewidth}{0pt}
    \renewcommand{\footrulewidth}{0pt}
}

% Line spacing
\setstretch{1.2}

% Document starts here
\begin{document}

% Portada
\begin{titlepage}
    \centering
    {\scshape\LARGE Universidad Politécnica de Madrid \par}
    \vspace{1cm}
    {\scshape\Large Escuela Técnica Superior de Ingenieros Industriales\par}
    \vspace{1.5cm}
    {\huge\bfseries Optimización energética de sistema híbrido con bomba de calor, suelo radiante, fotovoltaica y almacenamiento para vivienda \par}
    \vspace{1.5cm}
    {\Large\bfseries Trabajo de Fin de Máster\par}
    \vspace{0.5cm}
    {\large Máster Universitario en Ingeniería de la Energía \par}
    \vspace{2cm}
    {\Large Luis D. Aranda Sánchez\par}
    \vfill
    Director: Javier Rodríguez Martín
    \vfill
    {\large Septiembre 6, 2024\par}
\end{titlepage}

% Resumen (máximo de 5 páginas, incluyendo al final Palabras clave)
\clearpage
\pagestyle{simple}
% \newpage
\chapter*{Resumen}
\addcontentsline{toc}{chapter}{Resumen}
\input{capitulos/resumen/main.tex}

% Índice (paginado)
\clearpage
\pagestyle{simple}
% \newpage
\tableofcontents

% Introducción (donde se incluya los antecedentes y justificación)
\clearpage
\pagestyle{myfancy}
\newpage
\chapter{Introducción}
\input{capitulos/introduccion/main.tex}

% Objetivos
\chapter{Objetivos}
\input{capitulos/objetivos/main.tex}

% Metodología
\chapter{Metodología}
\input{capitulos/metodologia/main.tex}

% Resultados y discusión (incluyendo la valoración de impactos y de aspectos de responsabilidad legal, ética y profesional relacionados con el trabajo)
\chapter{Resultados y Discusión}
\input{capitulos/resultados_discusion/main.tex}

% Conclusiones
\chapter{Conclusiones}
\input{capitulos/conclusiones/main.tex}

% Planificación temporal y presupuesto
\chapter{Planificación Temporal y Presupuesto}
\input{capitulos/planificacion_presupuesto/main.tex}

% Bibliografía
\newpage
\addcontentsline{toc}{chapter}{Bibliografía}
\printbibliography

\end{document}


% Resultados y discusión (incluyendo la valoración de impactos y de aspectos de responsabilidad legal, ética y profesional relacionados con el trabajo)
\chapter{Resultados y Discusión}
\documentclass[a4paper,11pt,twoside]{report}
\usepackage[left=25mm,right=25mm,top=25mm,bottom=25mm,includehead,includefoot,headsep=15mm,footskip=15mm]{geometry}
\usepackage{graphicx}
\usepackage{fancyhdr}
\usepackage{titlesec}
\usepackage[spanish]{babel}
\usepackage[utf8]{inputenc}
\usepackage{amsmath}
\usepackage{setspace}
\usepackage{svg}
\usepackage{hyperref}
\usepackage[backend=biber,style=numeric]{biblatex}
\addbibresource{references.bib}
\hypersetup{
    colorlinks=true,
    linkcolor=blue,      % color of internal links (sections, etc.)
    urlcolor=blue,       % color of external links
    pdftitle={Optimización energética de sistema híbrido con bomba de calor, suelo radiante, fotovoltaica y almacenamiento para vivienda},    % title
    pdfauthor={Luis D. Aranda Sánchez},     % author
    pdfkeywords={palabra1, palabra2, código1, etc.} % list of keywords
}

% Font change to Arial
\usepackage{helvet}
\renewcommand{\familydefault}{\sfdefault}

% Chapter titles in uppercase and larger font
\titleformat{\chapter}[hang]{\large\bfseries}{\thechapter.}{1em}{\MakeUppercase}
\titleformat{\section}[hang]{\bfseries}{\thesection.}{1em}{}
\titleformat{\subsection}[hang]{\bfseries}{\thesubsection.}{1em}{}

% Fancyhdr setup
\setlength{\headheight}{14.30174pt} % Adjust to recommended value, slightly larger for safety
\fancyhf{} % Clear all headers and footers
\fancyhead[LE]{\nouppercase{\leftmark}}
\fancyhead[RO]{Optimización energética para vivienda}
\fancyfoot[LE]{\thepage}
\fancyfoot[RE]{Escuela Técnica Superior de Ingenieros Industriales (UPM)}
\fancyfoot[LO]{Luis D. Aranda Sánchez}
\fancyfoot[RO]{\thepage}
\renewcommand{\headrulewidth}{0.4pt}
\renewcommand{\footrulewidth}{0.4pt}

\fancypagestyle{myfancy}{
    \fancyhf{} % Clear all headers and footers
    \fancyhead[LE]{\nouppercase{\leftmark}}
    \fancyhead[RO]{Optimización energética para vivienda}
    \fancyfoot[LE]{\thepage}
    \fancyfoot[RE]{Escuela Técnica Superior de Ingenieros Industriales (UPM)}
    \fancyfoot[LO]{Luis D. Aranda Sánchez}
    \fancyfoot[RO]{\thepage}
    \renewcommand{\headrulewidth}{0.4pt}
    \renewcommand{\footrulewidth}{0.4pt}
}

\fancypagestyle{simple}{
    \fancyhf{} % Clear all headers and footers
    \renewcommand{\headrulewidth}{0pt}
    \renewcommand{\footrulewidth}{0pt}
}

% Line spacing
\setstretch{1.2}

% Document starts here
\begin{document}

% Portada
\begin{titlepage}
    \centering
    {\scshape\LARGE Universidad Politécnica de Madrid \par}
    \vspace{1cm}
    {\scshape\Large Escuela Técnica Superior de Ingenieros Industriales\par}
    \vspace{1.5cm}
    {\huge\bfseries Optimización energética de sistema híbrido con bomba de calor, suelo radiante, fotovoltaica y almacenamiento para vivienda \par}
    \vspace{1.5cm}
    {\Large\bfseries Trabajo de Fin de Máster\par}
    \vspace{0.5cm}
    {\large Máster Universitario en Ingeniería de la Energía \par}
    \vspace{2cm}
    {\Large Luis D. Aranda Sánchez\par}
    \vfill
    Director: Javier Rodríguez Martín
    \vfill
    {\large Septiembre 6, 2024\par}
\end{titlepage}

% Resumen (máximo de 5 páginas, incluyendo al final Palabras clave)
\clearpage
\pagestyle{simple}
% \newpage
\chapter*{Resumen}
\addcontentsline{toc}{chapter}{Resumen}
\input{capitulos/resumen/main.tex}

% Índice (paginado)
\clearpage
\pagestyle{simple}
% \newpage
\tableofcontents

% Introducción (donde se incluya los antecedentes y justificación)
\clearpage
\pagestyle{myfancy}
\newpage
\chapter{Introducción}
\input{capitulos/introduccion/main.tex}

% Objetivos
\chapter{Objetivos}
\input{capitulos/objetivos/main.tex}

% Metodología
\chapter{Metodología}
\input{capitulos/metodologia/main.tex}

% Resultados y discusión (incluyendo la valoración de impactos y de aspectos de responsabilidad legal, ética y profesional relacionados con el trabajo)
\chapter{Resultados y Discusión}
\input{capitulos/resultados_discusion/main.tex}

% Conclusiones
\chapter{Conclusiones}
\input{capitulos/conclusiones/main.tex}

% Planificación temporal y presupuesto
\chapter{Planificación Temporal y Presupuesto}
\input{capitulos/planificacion_presupuesto/main.tex}

% Bibliografía
\newpage
\addcontentsline{toc}{chapter}{Bibliografía}
\printbibliography

\end{document}


% Conclusiones
\chapter{Conclusiones}
\documentclass[a4paper,11pt,twoside]{report}
\usepackage[left=25mm,right=25mm,top=25mm,bottom=25mm,includehead,includefoot,headsep=15mm,footskip=15mm]{geometry}
\usepackage{graphicx}
\usepackage{fancyhdr}
\usepackage{titlesec}
\usepackage[spanish]{babel}
\usepackage[utf8]{inputenc}
\usepackage{amsmath}
\usepackage{setspace}
\usepackage{svg}
\usepackage{hyperref}
\usepackage[backend=biber,style=numeric]{biblatex}
\addbibresource{references.bib}
\hypersetup{
    colorlinks=true,
    linkcolor=blue,      % color of internal links (sections, etc.)
    urlcolor=blue,       % color of external links
    pdftitle={Optimización energética de sistema híbrido con bomba de calor, suelo radiante, fotovoltaica y almacenamiento para vivienda},    % title
    pdfauthor={Luis D. Aranda Sánchez},     % author
    pdfkeywords={palabra1, palabra2, código1, etc.} % list of keywords
}

% Font change to Arial
\usepackage{helvet}
\renewcommand{\familydefault}{\sfdefault}

% Chapter titles in uppercase and larger font
\titleformat{\chapter}[hang]{\large\bfseries}{\thechapter.}{1em}{\MakeUppercase}
\titleformat{\section}[hang]{\bfseries}{\thesection.}{1em}{}
\titleformat{\subsection}[hang]{\bfseries}{\thesubsection.}{1em}{}

% Fancyhdr setup
\setlength{\headheight}{14.30174pt} % Adjust to recommended value, slightly larger for safety
\fancyhf{} % Clear all headers and footers
\fancyhead[LE]{\nouppercase{\leftmark}}
\fancyhead[RO]{Optimización energética para vivienda}
\fancyfoot[LE]{\thepage}
\fancyfoot[RE]{Escuela Técnica Superior de Ingenieros Industriales (UPM)}
\fancyfoot[LO]{Luis D. Aranda Sánchez}
\fancyfoot[RO]{\thepage}
\renewcommand{\headrulewidth}{0.4pt}
\renewcommand{\footrulewidth}{0.4pt}

\fancypagestyle{myfancy}{
    \fancyhf{} % Clear all headers and footers
    \fancyhead[LE]{\nouppercase{\leftmark}}
    \fancyhead[RO]{Optimización energética para vivienda}
    \fancyfoot[LE]{\thepage}
    \fancyfoot[RE]{Escuela Técnica Superior de Ingenieros Industriales (UPM)}
    \fancyfoot[LO]{Luis D. Aranda Sánchez}
    \fancyfoot[RO]{\thepage}
    \renewcommand{\headrulewidth}{0.4pt}
    \renewcommand{\footrulewidth}{0.4pt}
}

\fancypagestyle{simple}{
    \fancyhf{} % Clear all headers and footers
    \renewcommand{\headrulewidth}{0pt}
    \renewcommand{\footrulewidth}{0pt}
}

% Line spacing
\setstretch{1.2}

% Document starts here
\begin{document}

% Portada
\begin{titlepage}
    \centering
    {\scshape\LARGE Universidad Politécnica de Madrid \par}
    \vspace{1cm}
    {\scshape\Large Escuela Técnica Superior de Ingenieros Industriales\par}
    \vspace{1.5cm}
    {\huge\bfseries Optimización energética de sistema híbrido con bomba de calor, suelo radiante, fotovoltaica y almacenamiento para vivienda \par}
    \vspace{1.5cm}
    {\Large\bfseries Trabajo de Fin de Máster\par}
    \vspace{0.5cm}
    {\large Máster Universitario en Ingeniería de la Energía \par}
    \vspace{2cm}
    {\Large Luis D. Aranda Sánchez\par}
    \vfill
    Director: Javier Rodríguez Martín
    \vfill
    {\large Septiembre 6, 2024\par}
\end{titlepage}

% Resumen (máximo de 5 páginas, incluyendo al final Palabras clave)
\clearpage
\pagestyle{simple}
% \newpage
\chapter*{Resumen}
\addcontentsline{toc}{chapter}{Resumen}
\input{capitulos/resumen/main.tex}

% Índice (paginado)
\clearpage
\pagestyle{simple}
% \newpage
\tableofcontents

% Introducción (donde se incluya los antecedentes y justificación)
\clearpage
\pagestyle{myfancy}
\newpage
\chapter{Introducción}
\input{capitulos/introduccion/main.tex}

% Objetivos
\chapter{Objetivos}
\input{capitulos/objetivos/main.tex}

% Metodología
\chapter{Metodología}
\input{capitulos/metodologia/main.tex}

% Resultados y discusión (incluyendo la valoración de impactos y de aspectos de responsabilidad legal, ética y profesional relacionados con el trabajo)
\chapter{Resultados y Discusión}
\input{capitulos/resultados_discusion/main.tex}

% Conclusiones
\chapter{Conclusiones}
\input{capitulos/conclusiones/main.tex}

% Planificación temporal y presupuesto
\chapter{Planificación Temporal y Presupuesto}
\input{capitulos/planificacion_presupuesto/main.tex}

% Bibliografía
\newpage
\addcontentsline{toc}{chapter}{Bibliografía}
\printbibliography

\end{document}


% Planificación temporal y presupuesto
\chapter{Planificación Temporal y Presupuesto}
\documentclass[a4paper,11pt,twoside]{report}
\usepackage[left=25mm,right=25mm,top=25mm,bottom=25mm,includehead,includefoot,headsep=15mm,footskip=15mm]{geometry}
\usepackage{graphicx}
\usepackage{fancyhdr}
\usepackage{titlesec}
\usepackage[spanish]{babel}
\usepackage[utf8]{inputenc}
\usepackage{amsmath}
\usepackage{setspace}
\usepackage{svg}
\usepackage{hyperref}
\usepackage[backend=biber,style=numeric]{biblatex}
\addbibresource{references.bib}
\hypersetup{
    colorlinks=true,
    linkcolor=blue,      % color of internal links (sections, etc.)
    urlcolor=blue,       % color of external links
    pdftitle={Optimización energética de sistema híbrido con bomba de calor, suelo radiante, fotovoltaica y almacenamiento para vivienda},    % title
    pdfauthor={Luis D. Aranda Sánchez},     % author
    pdfkeywords={palabra1, palabra2, código1, etc.} % list of keywords
}

% Font change to Arial
\usepackage{helvet}
\renewcommand{\familydefault}{\sfdefault}

% Chapter titles in uppercase and larger font
\titleformat{\chapter}[hang]{\large\bfseries}{\thechapter.}{1em}{\MakeUppercase}
\titleformat{\section}[hang]{\bfseries}{\thesection.}{1em}{}
\titleformat{\subsection}[hang]{\bfseries}{\thesubsection.}{1em}{}

% Fancyhdr setup
\setlength{\headheight}{14.30174pt} % Adjust to recommended value, slightly larger for safety
\fancyhf{} % Clear all headers and footers
\fancyhead[LE]{\nouppercase{\leftmark}}
\fancyhead[RO]{Optimización energética para vivienda}
\fancyfoot[LE]{\thepage}
\fancyfoot[RE]{Escuela Técnica Superior de Ingenieros Industriales (UPM)}
\fancyfoot[LO]{Luis D. Aranda Sánchez}
\fancyfoot[RO]{\thepage}
\renewcommand{\headrulewidth}{0.4pt}
\renewcommand{\footrulewidth}{0.4pt}

\fancypagestyle{myfancy}{
    \fancyhf{} % Clear all headers and footers
    \fancyhead[LE]{\nouppercase{\leftmark}}
    \fancyhead[RO]{Optimización energética para vivienda}
    \fancyfoot[LE]{\thepage}
    \fancyfoot[RE]{Escuela Técnica Superior de Ingenieros Industriales (UPM)}
    \fancyfoot[LO]{Luis D. Aranda Sánchez}
    \fancyfoot[RO]{\thepage}
    \renewcommand{\headrulewidth}{0.4pt}
    \renewcommand{\footrulewidth}{0.4pt}
}

\fancypagestyle{simple}{
    \fancyhf{} % Clear all headers and footers
    \renewcommand{\headrulewidth}{0pt}
    \renewcommand{\footrulewidth}{0pt}
}

% Line spacing
\setstretch{1.2}

% Document starts here
\begin{document}

% Portada
\begin{titlepage}
    \centering
    {\scshape\LARGE Universidad Politécnica de Madrid \par}
    \vspace{1cm}
    {\scshape\Large Escuela Técnica Superior de Ingenieros Industriales\par}
    \vspace{1.5cm}
    {\huge\bfseries Optimización energética de sistema híbrido con bomba de calor, suelo radiante, fotovoltaica y almacenamiento para vivienda \par}
    \vspace{1.5cm}
    {\Large\bfseries Trabajo de Fin de Máster\par}
    \vspace{0.5cm}
    {\large Máster Universitario en Ingeniería de la Energía \par}
    \vspace{2cm}
    {\Large Luis D. Aranda Sánchez\par}
    \vfill
    Director: Javier Rodríguez Martín
    \vfill
    {\large Septiembre 6, 2024\par}
\end{titlepage}

% Resumen (máximo de 5 páginas, incluyendo al final Palabras clave)
\clearpage
\pagestyle{simple}
% \newpage
\chapter*{Resumen}
\addcontentsline{toc}{chapter}{Resumen}
\input{capitulos/resumen/main.tex}

% Índice (paginado)
\clearpage
\pagestyle{simple}
% \newpage
\tableofcontents

% Introducción (donde se incluya los antecedentes y justificación)
\clearpage
\pagestyle{myfancy}
\newpage
\chapter{Introducción}
\input{capitulos/introduccion/main.tex}

% Objetivos
\chapter{Objetivos}
\input{capitulos/objetivos/main.tex}

% Metodología
\chapter{Metodología}
\input{capitulos/metodologia/main.tex}

% Resultados y discusión (incluyendo la valoración de impactos y de aspectos de responsabilidad legal, ética y profesional relacionados con el trabajo)
\chapter{Resultados y Discusión}
\input{capitulos/resultados_discusion/main.tex}

% Conclusiones
\chapter{Conclusiones}
\input{capitulos/conclusiones/main.tex}

% Planificación temporal y presupuesto
\chapter{Planificación Temporal y Presupuesto}
\input{capitulos/planificacion_presupuesto/main.tex}

% Bibliografía
\newpage
\addcontentsline{toc}{chapter}{Bibliografía}
\printbibliography

\end{document}


% Bibliografía
\newpage
\addcontentsline{toc}{chapter}{Bibliografía}
\printbibliography

\end{document}


% Índice (paginado)
\clearpage
\pagestyle{simple}
% \newpage
\tableofcontents

% Introducción (donde se incluya los antecedentes y justificación)
\clearpage
\pagestyle{myfancy}
\newpage
\chapter{Introducción}
\documentclass[a4paper,11pt,twoside]{report}
\usepackage[left=25mm,right=25mm,top=25mm,bottom=25mm,includehead,includefoot,headsep=15mm,footskip=15mm]{geometry}
\usepackage{graphicx}
\usepackage{fancyhdr}
\usepackage{titlesec}
\usepackage[spanish]{babel}
\usepackage[utf8]{inputenc}
\usepackage{amsmath}
\usepackage{setspace}
\usepackage{svg}
\usepackage{hyperref}
\usepackage[backend=biber,style=numeric]{biblatex}
\addbibresource{references.bib}
\hypersetup{
    colorlinks=true,
    linkcolor=blue,      % color of internal links (sections, etc.)
    urlcolor=blue,       % color of external links
    pdftitle={Optimización energética de sistema híbrido con bomba de calor, suelo radiante, fotovoltaica y almacenamiento para vivienda},    % title
    pdfauthor={Luis D. Aranda Sánchez},     % author
    pdfkeywords={palabra1, palabra2, código1, etc.} % list of keywords
}

% Font change to Arial
\usepackage{helvet}
\renewcommand{\familydefault}{\sfdefault}

% Chapter titles in uppercase and larger font
\titleformat{\chapter}[hang]{\large\bfseries}{\thechapter.}{1em}{\MakeUppercase}
\titleformat{\section}[hang]{\bfseries}{\thesection.}{1em}{}
\titleformat{\subsection}[hang]{\bfseries}{\thesubsection.}{1em}{}

% Fancyhdr setup
\setlength{\headheight}{14.30174pt} % Adjust to recommended value, slightly larger for safety
\fancyhf{} % Clear all headers and footers
\fancyhead[LE]{\nouppercase{\leftmark}}
\fancyhead[RO]{Optimización energética para vivienda}
\fancyfoot[LE]{\thepage}
\fancyfoot[RE]{Escuela Técnica Superior de Ingenieros Industriales (UPM)}
\fancyfoot[LO]{Luis D. Aranda Sánchez}
\fancyfoot[RO]{\thepage}
\renewcommand{\headrulewidth}{0.4pt}
\renewcommand{\footrulewidth}{0.4pt}

\fancypagestyle{myfancy}{
    \fancyhf{} % Clear all headers and footers
    \fancyhead[LE]{\nouppercase{\leftmark}}
    \fancyhead[RO]{Optimización energética para vivienda}
    \fancyfoot[LE]{\thepage}
    \fancyfoot[RE]{Escuela Técnica Superior de Ingenieros Industriales (UPM)}
    \fancyfoot[LO]{Luis D. Aranda Sánchez}
    \fancyfoot[RO]{\thepage}
    \renewcommand{\headrulewidth}{0.4pt}
    \renewcommand{\footrulewidth}{0.4pt}
}

\fancypagestyle{simple}{
    \fancyhf{} % Clear all headers and footers
    \renewcommand{\headrulewidth}{0pt}
    \renewcommand{\footrulewidth}{0pt}
}

% Line spacing
\setstretch{1.2}

% Document starts here
\begin{document}

% Portada
\begin{titlepage}
    \centering
    {\scshape\LARGE Universidad Politécnica de Madrid \par}
    \vspace{1cm}
    {\scshape\Large Escuela Técnica Superior de Ingenieros Industriales\par}
    \vspace{1.5cm}
    {\huge\bfseries Optimización energética de sistema híbrido con bomba de calor, suelo radiante, fotovoltaica y almacenamiento para vivienda \par}
    \vspace{1.5cm}
    {\Large\bfseries Trabajo de Fin de Máster\par}
    \vspace{0.5cm}
    {\large Máster Universitario en Ingeniería de la Energía \par}
    \vspace{2cm}
    {\Large Luis D. Aranda Sánchez\par}
    \vfill
    Director: Javier Rodríguez Martín
    \vfill
    {\large Septiembre 6, 2024\par}
\end{titlepage}

% Resumen (máximo de 5 páginas, incluyendo al final Palabras clave)
\clearpage
\pagestyle{simple}
% \newpage
\chapter*{Resumen}
\addcontentsline{toc}{chapter}{Resumen}
\documentclass[a4paper,11pt,twoside]{report}
\usepackage[left=25mm,right=25mm,top=25mm,bottom=25mm,includehead,includefoot,headsep=15mm,footskip=15mm]{geometry}
\usepackage{graphicx}
\usepackage{fancyhdr}
\usepackage{titlesec}
\usepackage[spanish]{babel}
\usepackage[utf8]{inputenc}
\usepackage{amsmath}
\usepackage{setspace}
\usepackage{svg}
\usepackage{hyperref}
\usepackage[backend=biber,style=numeric]{biblatex}
\addbibresource{references.bib}
\hypersetup{
    colorlinks=true,
    linkcolor=blue,      % color of internal links (sections, etc.)
    urlcolor=blue,       % color of external links
    pdftitle={Optimización energética de sistema híbrido con bomba de calor, suelo radiante, fotovoltaica y almacenamiento para vivienda},    % title
    pdfauthor={Luis D. Aranda Sánchez},     % author
    pdfkeywords={palabra1, palabra2, código1, etc.} % list of keywords
}

% Font change to Arial
\usepackage{helvet}
\renewcommand{\familydefault}{\sfdefault}

% Chapter titles in uppercase and larger font
\titleformat{\chapter}[hang]{\large\bfseries}{\thechapter.}{1em}{\MakeUppercase}
\titleformat{\section}[hang]{\bfseries}{\thesection.}{1em}{}
\titleformat{\subsection}[hang]{\bfseries}{\thesubsection.}{1em}{}

% Fancyhdr setup
\setlength{\headheight}{14.30174pt} % Adjust to recommended value, slightly larger for safety
\fancyhf{} % Clear all headers and footers
\fancyhead[LE]{\nouppercase{\leftmark}}
\fancyhead[RO]{Optimización energética para vivienda}
\fancyfoot[LE]{\thepage}
\fancyfoot[RE]{Escuela Técnica Superior de Ingenieros Industriales (UPM)}
\fancyfoot[LO]{Luis D. Aranda Sánchez}
\fancyfoot[RO]{\thepage}
\renewcommand{\headrulewidth}{0.4pt}
\renewcommand{\footrulewidth}{0.4pt}

\fancypagestyle{myfancy}{
    \fancyhf{} % Clear all headers and footers
    \fancyhead[LE]{\nouppercase{\leftmark}}
    \fancyhead[RO]{Optimización energética para vivienda}
    \fancyfoot[LE]{\thepage}
    \fancyfoot[RE]{Escuela Técnica Superior de Ingenieros Industriales (UPM)}
    \fancyfoot[LO]{Luis D. Aranda Sánchez}
    \fancyfoot[RO]{\thepage}
    \renewcommand{\headrulewidth}{0.4pt}
    \renewcommand{\footrulewidth}{0.4pt}
}

\fancypagestyle{simple}{
    \fancyhf{} % Clear all headers and footers
    \renewcommand{\headrulewidth}{0pt}
    \renewcommand{\footrulewidth}{0pt}
}

% Line spacing
\setstretch{1.2}

% Document starts here
\begin{document}

% Portada
\begin{titlepage}
    \centering
    {\scshape\LARGE Universidad Politécnica de Madrid \par}
    \vspace{1cm}
    {\scshape\Large Escuela Técnica Superior de Ingenieros Industriales\par}
    \vspace{1.5cm}
    {\huge\bfseries Optimización energética de sistema híbrido con bomba de calor, suelo radiante, fotovoltaica y almacenamiento para vivienda \par}
    \vspace{1.5cm}
    {\Large\bfseries Trabajo de Fin de Máster\par}
    \vspace{0.5cm}
    {\large Máster Universitario en Ingeniería de la Energía \par}
    \vspace{2cm}
    {\Large Luis D. Aranda Sánchez\par}
    \vfill
    Director: Javier Rodríguez Martín
    \vfill
    {\large Septiembre 6, 2024\par}
\end{titlepage}

% Resumen (máximo de 5 páginas, incluyendo al final Palabras clave)
\clearpage
\pagestyle{simple}
% \newpage
\chapter*{Resumen}
\addcontentsline{toc}{chapter}{Resumen}
\input{capitulos/resumen/main.tex}

% Índice (paginado)
\clearpage
\pagestyle{simple}
% \newpage
\tableofcontents

% Introducción (donde se incluya los antecedentes y justificación)
\clearpage
\pagestyle{myfancy}
\newpage
\chapter{Introducción}
\input{capitulos/introduccion/main.tex}

% Objetivos
\chapter{Objetivos}
\input{capitulos/objetivos/main.tex}

% Metodología
\chapter{Metodología}
\input{capitulos/metodologia/main.tex}

% Resultados y discusión (incluyendo la valoración de impactos y de aspectos de responsabilidad legal, ética y profesional relacionados con el trabajo)
\chapter{Resultados y Discusión}
\input{capitulos/resultados_discusion/main.tex}

% Conclusiones
\chapter{Conclusiones}
\input{capitulos/conclusiones/main.tex}

% Planificación temporal y presupuesto
\chapter{Planificación Temporal y Presupuesto}
\input{capitulos/planificacion_presupuesto/main.tex}

% Bibliografía
\newpage
\addcontentsline{toc}{chapter}{Bibliografía}
\printbibliography

\end{document}


% Índice (paginado)
\clearpage
\pagestyle{simple}
% \newpage
\tableofcontents

% Introducción (donde se incluya los antecedentes y justificación)
\clearpage
\pagestyle{myfancy}
\newpage
\chapter{Introducción}
\documentclass[a4paper,11pt,twoside]{report}
\usepackage[left=25mm,right=25mm,top=25mm,bottom=25mm,includehead,includefoot,headsep=15mm,footskip=15mm]{geometry}
\usepackage{graphicx}
\usepackage{fancyhdr}
\usepackage{titlesec}
\usepackage[spanish]{babel}
\usepackage[utf8]{inputenc}
\usepackage{amsmath}
\usepackage{setspace}
\usepackage{svg}
\usepackage{hyperref}
\usepackage[backend=biber,style=numeric]{biblatex}
\addbibresource{references.bib}
\hypersetup{
    colorlinks=true,
    linkcolor=blue,      % color of internal links (sections, etc.)
    urlcolor=blue,       % color of external links
    pdftitle={Optimización energética de sistema híbrido con bomba de calor, suelo radiante, fotovoltaica y almacenamiento para vivienda},    % title
    pdfauthor={Luis D. Aranda Sánchez},     % author
    pdfkeywords={palabra1, palabra2, código1, etc.} % list of keywords
}

% Font change to Arial
\usepackage{helvet}
\renewcommand{\familydefault}{\sfdefault}

% Chapter titles in uppercase and larger font
\titleformat{\chapter}[hang]{\large\bfseries}{\thechapter.}{1em}{\MakeUppercase}
\titleformat{\section}[hang]{\bfseries}{\thesection.}{1em}{}
\titleformat{\subsection}[hang]{\bfseries}{\thesubsection.}{1em}{}

% Fancyhdr setup
\setlength{\headheight}{14.30174pt} % Adjust to recommended value, slightly larger for safety
\fancyhf{} % Clear all headers and footers
\fancyhead[LE]{\nouppercase{\leftmark}}
\fancyhead[RO]{Optimización energética para vivienda}
\fancyfoot[LE]{\thepage}
\fancyfoot[RE]{Escuela Técnica Superior de Ingenieros Industriales (UPM)}
\fancyfoot[LO]{Luis D. Aranda Sánchez}
\fancyfoot[RO]{\thepage}
\renewcommand{\headrulewidth}{0.4pt}
\renewcommand{\footrulewidth}{0.4pt}

\fancypagestyle{myfancy}{
    \fancyhf{} % Clear all headers and footers
    \fancyhead[LE]{\nouppercase{\leftmark}}
    \fancyhead[RO]{Optimización energética para vivienda}
    \fancyfoot[LE]{\thepage}
    \fancyfoot[RE]{Escuela Técnica Superior de Ingenieros Industriales (UPM)}
    \fancyfoot[LO]{Luis D. Aranda Sánchez}
    \fancyfoot[RO]{\thepage}
    \renewcommand{\headrulewidth}{0.4pt}
    \renewcommand{\footrulewidth}{0.4pt}
}

\fancypagestyle{simple}{
    \fancyhf{} % Clear all headers and footers
    \renewcommand{\headrulewidth}{0pt}
    \renewcommand{\footrulewidth}{0pt}
}

% Line spacing
\setstretch{1.2}

% Document starts here
\begin{document}

% Portada
\begin{titlepage}
    \centering
    {\scshape\LARGE Universidad Politécnica de Madrid \par}
    \vspace{1cm}
    {\scshape\Large Escuela Técnica Superior de Ingenieros Industriales\par}
    \vspace{1.5cm}
    {\huge\bfseries Optimización energética de sistema híbrido con bomba de calor, suelo radiante, fotovoltaica y almacenamiento para vivienda \par}
    \vspace{1.5cm}
    {\Large\bfseries Trabajo de Fin de Máster\par}
    \vspace{0.5cm}
    {\large Máster Universitario en Ingeniería de la Energía \par}
    \vspace{2cm}
    {\Large Luis D. Aranda Sánchez\par}
    \vfill
    Director: Javier Rodríguez Martín
    \vfill
    {\large Septiembre 6, 2024\par}
\end{titlepage}

% Resumen (máximo de 5 páginas, incluyendo al final Palabras clave)
\clearpage
\pagestyle{simple}
% \newpage
\chapter*{Resumen}
\addcontentsline{toc}{chapter}{Resumen}
\input{capitulos/resumen/main.tex}

% Índice (paginado)
\clearpage
\pagestyle{simple}
% \newpage
\tableofcontents

% Introducción (donde se incluya los antecedentes y justificación)
\clearpage
\pagestyle{myfancy}
\newpage
\chapter{Introducción}
\input{capitulos/introduccion/main.tex}

% Objetivos
\chapter{Objetivos}
\input{capitulos/objetivos/main.tex}

% Metodología
\chapter{Metodología}
\input{capitulos/metodologia/main.tex}

% Resultados y discusión (incluyendo la valoración de impactos y de aspectos de responsabilidad legal, ética y profesional relacionados con el trabajo)
\chapter{Resultados y Discusión}
\input{capitulos/resultados_discusion/main.tex}

% Conclusiones
\chapter{Conclusiones}
\input{capitulos/conclusiones/main.tex}

% Planificación temporal y presupuesto
\chapter{Planificación Temporal y Presupuesto}
\input{capitulos/planificacion_presupuesto/main.tex}

% Bibliografía
\newpage
\addcontentsline{toc}{chapter}{Bibliografía}
\printbibliography

\end{document}


% Objetivos
\chapter{Objetivos}
\documentclass[a4paper,11pt,twoside]{report}
\usepackage[left=25mm,right=25mm,top=25mm,bottom=25mm,includehead,includefoot,headsep=15mm,footskip=15mm]{geometry}
\usepackage{graphicx}
\usepackage{fancyhdr}
\usepackage{titlesec}
\usepackage[spanish]{babel}
\usepackage[utf8]{inputenc}
\usepackage{amsmath}
\usepackage{setspace}
\usepackage{svg}
\usepackage{hyperref}
\usepackage[backend=biber,style=numeric]{biblatex}
\addbibresource{references.bib}
\hypersetup{
    colorlinks=true,
    linkcolor=blue,      % color of internal links (sections, etc.)
    urlcolor=blue,       % color of external links
    pdftitle={Optimización energética de sistema híbrido con bomba de calor, suelo radiante, fotovoltaica y almacenamiento para vivienda},    % title
    pdfauthor={Luis D. Aranda Sánchez},     % author
    pdfkeywords={palabra1, palabra2, código1, etc.} % list of keywords
}

% Font change to Arial
\usepackage{helvet}
\renewcommand{\familydefault}{\sfdefault}

% Chapter titles in uppercase and larger font
\titleformat{\chapter}[hang]{\large\bfseries}{\thechapter.}{1em}{\MakeUppercase}
\titleformat{\section}[hang]{\bfseries}{\thesection.}{1em}{}
\titleformat{\subsection}[hang]{\bfseries}{\thesubsection.}{1em}{}

% Fancyhdr setup
\setlength{\headheight}{14.30174pt} % Adjust to recommended value, slightly larger for safety
\fancyhf{} % Clear all headers and footers
\fancyhead[LE]{\nouppercase{\leftmark}}
\fancyhead[RO]{Optimización energética para vivienda}
\fancyfoot[LE]{\thepage}
\fancyfoot[RE]{Escuela Técnica Superior de Ingenieros Industriales (UPM)}
\fancyfoot[LO]{Luis D. Aranda Sánchez}
\fancyfoot[RO]{\thepage}
\renewcommand{\headrulewidth}{0.4pt}
\renewcommand{\footrulewidth}{0.4pt}

\fancypagestyle{myfancy}{
    \fancyhf{} % Clear all headers and footers
    \fancyhead[LE]{\nouppercase{\leftmark}}
    \fancyhead[RO]{Optimización energética para vivienda}
    \fancyfoot[LE]{\thepage}
    \fancyfoot[RE]{Escuela Técnica Superior de Ingenieros Industriales (UPM)}
    \fancyfoot[LO]{Luis D. Aranda Sánchez}
    \fancyfoot[RO]{\thepage}
    \renewcommand{\headrulewidth}{0.4pt}
    \renewcommand{\footrulewidth}{0.4pt}
}

\fancypagestyle{simple}{
    \fancyhf{} % Clear all headers and footers
    \renewcommand{\headrulewidth}{0pt}
    \renewcommand{\footrulewidth}{0pt}
}

% Line spacing
\setstretch{1.2}

% Document starts here
\begin{document}

% Portada
\begin{titlepage}
    \centering
    {\scshape\LARGE Universidad Politécnica de Madrid \par}
    \vspace{1cm}
    {\scshape\Large Escuela Técnica Superior de Ingenieros Industriales\par}
    \vspace{1.5cm}
    {\huge\bfseries Optimización energética de sistema híbrido con bomba de calor, suelo radiante, fotovoltaica y almacenamiento para vivienda \par}
    \vspace{1.5cm}
    {\Large\bfseries Trabajo de Fin de Máster\par}
    \vspace{0.5cm}
    {\large Máster Universitario en Ingeniería de la Energía \par}
    \vspace{2cm}
    {\Large Luis D. Aranda Sánchez\par}
    \vfill
    Director: Javier Rodríguez Martín
    \vfill
    {\large Septiembre 6, 2024\par}
\end{titlepage}

% Resumen (máximo de 5 páginas, incluyendo al final Palabras clave)
\clearpage
\pagestyle{simple}
% \newpage
\chapter*{Resumen}
\addcontentsline{toc}{chapter}{Resumen}
\input{capitulos/resumen/main.tex}

% Índice (paginado)
\clearpage
\pagestyle{simple}
% \newpage
\tableofcontents

% Introducción (donde se incluya los antecedentes y justificación)
\clearpage
\pagestyle{myfancy}
\newpage
\chapter{Introducción}
\input{capitulos/introduccion/main.tex}

% Objetivos
\chapter{Objetivos}
\input{capitulos/objetivos/main.tex}

% Metodología
\chapter{Metodología}
\input{capitulos/metodologia/main.tex}

% Resultados y discusión (incluyendo la valoración de impactos y de aspectos de responsabilidad legal, ética y profesional relacionados con el trabajo)
\chapter{Resultados y Discusión}
\input{capitulos/resultados_discusion/main.tex}

% Conclusiones
\chapter{Conclusiones}
\input{capitulos/conclusiones/main.tex}

% Planificación temporal y presupuesto
\chapter{Planificación Temporal y Presupuesto}
\input{capitulos/planificacion_presupuesto/main.tex}

% Bibliografía
\newpage
\addcontentsline{toc}{chapter}{Bibliografía}
\printbibliography

\end{document}


% Metodología
\chapter{Metodología}
\documentclass[a4paper,11pt,twoside]{report}
\usepackage[left=25mm,right=25mm,top=25mm,bottom=25mm,includehead,includefoot,headsep=15mm,footskip=15mm]{geometry}
\usepackage{graphicx}
\usepackage{fancyhdr}
\usepackage{titlesec}
\usepackage[spanish]{babel}
\usepackage[utf8]{inputenc}
\usepackage{amsmath}
\usepackage{setspace}
\usepackage{svg}
\usepackage{hyperref}
\usepackage[backend=biber,style=numeric]{biblatex}
\addbibresource{references.bib}
\hypersetup{
    colorlinks=true,
    linkcolor=blue,      % color of internal links (sections, etc.)
    urlcolor=blue,       % color of external links
    pdftitle={Optimización energética de sistema híbrido con bomba de calor, suelo radiante, fotovoltaica y almacenamiento para vivienda},    % title
    pdfauthor={Luis D. Aranda Sánchez},     % author
    pdfkeywords={palabra1, palabra2, código1, etc.} % list of keywords
}

% Font change to Arial
\usepackage{helvet}
\renewcommand{\familydefault}{\sfdefault}

% Chapter titles in uppercase and larger font
\titleformat{\chapter}[hang]{\large\bfseries}{\thechapter.}{1em}{\MakeUppercase}
\titleformat{\section}[hang]{\bfseries}{\thesection.}{1em}{}
\titleformat{\subsection}[hang]{\bfseries}{\thesubsection.}{1em}{}

% Fancyhdr setup
\setlength{\headheight}{14.30174pt} % Adjust to recommended value, slightly larger for safety
\fancyhf{} % Clear all headers and footers
\fancyhead[LE]{\nouppercase{\leftmark}}
\fancyhead[RO]{Optimización energética para vivienda}
\fancyfoot[LE]{\thepage}
\fancyfoot[RE]{Escuela Técnica Superior de Ingenieros Industriales (UPM)}
\fancyfoot[LO]{Luis D. Aranda Sánchez}
\fancyfoot[RO]{\thepage}
\renewcommand{\headrulewidth}{0.4pt}
\renewcommand{\footrulewidth}{0.4pt}

\fancypagestyle{myfancy}{
    \fancyhf{} % Clear all headers and footers
    \fancyhead[LE]{\nouppercase{\leftmark}}
    \fancyhead[RO]{Optimización energética para vivienda}
    \fancyfoot[LE]{\thepage}
    \fancyfoot[RE]{Escuela Técnica Superior de Ingenieros Industriales (UPM)}
    \fancyfoot[LO]{Luis D. Aranda Sánchez}
    \fancyfoot[RO]{\thepage}
    \renewcommand{\headrulewidth}{0.4pt}
    \renewcommand{\footrulewidth}{0.4pt}
}

\fancypagestyle{simple}{
    \fancyhf{} % Clear all headers and footers
    \renewcommand{\headrulewidth}{0pt}
    \renewcommand{\footrulewidth}{0pt}
}

% Line spacing
\setstretch{1.2}

% Document starts here
\begin{document}

% Portada
\begin{titlepage}
    \centering
    {\scshape\LARGE Universidad Politécnica de Madrid \par}
    \vspace{1cm}
    {\scshape\Large Escuela Técnica Superior de Ingenieros Industriales\par}
    \vspace{1.5cm}
    {\huge\bfseries Optimización energética de sistema híbrido con bomba de calor, suelo radiante, fotovoltaica y almacenamiento para vivienda \par}
    \vspace{1.5cm}
    {\Large\bfseries Trabajo de Fin de Máster\par}
    \vspace{0.5cm}
    {\large Máster Universitario en Ingeniería de la Energía \par}
    \vspace{2cm}
    {\Large Luis D. Aranda Sánchez\par}
    \vfill
    Director: Javier Rodríguez Martín
    \vfill
    {\large Septiembre 6, 2024\par}
\end{titlepage}

% Resumen (máximo de 5 páginas, incluyendo al final Palabras clave)
\clearpage
\pagestyle{simple}
% \newpage
\chapter*{Resumen}
\addcontentsline{toc}{chapter}{Resumen}
\input{capitulos/resumen/main.tex}

% Índice (paginado)
\clearpage
\pagestyle{simple}
% \newpage
\tableofcontents

% Introducción (donde se incluya los antecedentes y justificación)
\clearpage
\pagestyle{myfancy}
\newpage
\chapter{Introducción}
\input{capitulos/introduccion/main.tex}

% Objetivos
\chapter{Objetivos}
\input{capitulos/objetivos/main.tex}

% Metodología
\chapter{Metodología}
\input{capitulos/metodologia/main.tex}

% Resultados y discusión (incluyendo la valoración de impactos y de aspectos de responsabilidad legal, ética y profesional relacionados con el trabajo)
\chapter{Resultados y Discusión}
\input{capitulos/resultados_discusion/main.tex}

% Conclusiones
\chapter{Conclusiones}
\input{capitulos/conclusiones/main.tex}

% Planificación temporal y presupuesto
\chapter{Planificación Temporal y Presupuesto}
\input{capitulos/planificacion_presupuesto/main.tex}

% Bibliografía
\newpage
\addcontentsline{toc}{chapter}{Bibliografía}
\printbibliography

\end{document}


% Resultados y discusión (incluyendo la valoración de impactos y de aspectos de responsabilidad legal, ética y profesional relacionados con el trabajo)
\chapter{Resultados y Discusión}
\documentclass[a4paper,11pt,twoside]{report}
\usepackage[left=25mm,right=25mm,top=25mm,bottom=25mm,includehead,includefoot,headsep=15mm,footskip=15mm]{geometry}
\usepackage{graphicx}
\usepackage{fancyhdr}
\usepackage{titlesec}
\usepackage[spanish]{babel}
\usepackage[utf8]{inputenc}
\usepackage{amsmath}
\usepackage{setspace}
\usepackage{svg}
\usepackage{hyperref}
\usepackage[backend=biber,style=numeric]{biblatex}
\addbibresource{references.bib}
\hypersetup{
    colorlinks=true,
    linkcolor=blue,      % color of internal links (sections, etc.)
    urlcolor=blue,       % color of external links
    pdftitle={Optimización energética de sistema híbrido con bomba de calor, suelo radiante, fotovoltaica y almacenamiento para vivienda},    % title
    pdfauthor={Luis D. Aranda Sánchez},     % author
    pdfkeywords={palabra1, palabra2, código1, etc.} % list of keywords
}

% Font change to Arial
\usepackage{helvet}
\renewcommand{\familydefault}{\sfdefault}

% Chapter titles in uppercase and larger font
\titleformat{\chapter}[hang]{\large\bfseries}{\thechapter.}{1em}{\MakeUppercase}
\titleformat{\section}[hang]{\bfseries}{\thesection.}{1em}{}
\titleformat{\subsection}[hang]{\bfseries}{\thesubsection.}{1em}{}

% Fancyhdr setup
\setlength{\headheight}{14.30174pt} % Adjust to recommended value, slightly larger for safety
\fancyhf{} % Clear all headers and footers
\fancyhead[LE]{\nouppercase{\leftmark}}
\fancyhead[RO]{Optimización energética para vivienda}
\fancyfoot[LE]{\thepage}
\fancyfoot[RE]{Escuela Técnica Superior de Ingenieros Industriales (UPM)}
\fancyfoot[LO]{Luis D. Aranda Sánchez}
\fancyfoot[RO]{\thepage}
\renewcommand{\headrulewidth}{0.4pt}
\renewcommand{\footrulewidth}{0.4pt}

\fancypagestyle{myfancy}{
    \fancyhf{} % Clear all headers and footers
    \fancyhead[LE]{\nouppercase{\leftmark}}
    \fancyhead[RO]{Optimización energética para vivienda}
    \fancyfoot[LE]{\thepage}
    \fancyfoot[RE]{Escuela Técnica Superior de Ingenieros Industriales (UPM)}
    \fancyfoot[LO]{Luis D. Aranda Sánchez}
    \fancyfoot[RO]{\thepage}
    \renewcommand{\headrulewidth}{0.4pt}
    \renewcommand{\footrulewidth}{0.4pt}
}

\fancypagestyle{simple}{
    \fancyhf{} % Clear all headers and footers
    \renewcommand{\headrulewidth}{0pt}
    \renewcommand{\footrulewidth}{0pt}
}

% Line spacing
\setstretch{1.2}

% Document starts here
\begin{document}

% Portada
\begin{titlepage}
    \centering
    {\scshape\LARGE Universidad Politécnica de Madrid \par}
    \vspace{1cm}
    {\scshape\Large Escuela Técnica Superior de Ingenieros Industriales\par}
    \vspace{1.5cm}
    {\huge\bfseries Optimización energética de sistema híbrido con bomba de calor, suelo radiante, fotovoltaica y almacenamiento para vivienda \par}
    \vspace{1.5cm}
    {\Large\bfseries Trabajo de Fin de Máster\par}
    \vspace{0.5cm}
    {\large Máster Universitario en Ingeniería de la Energía \par}
    \vspace{2cm}
    {\Large Luis D. Aranda Sánchez\par}
    \vfill
    Director: Javier Rodríguez Martín
    \vfill
    {\large Septiembre 6, 2024\par}
\end{titlepage}

% Resumen (máximo de 5 páginas, incluyendo al final Palabras clave)
\clearpage
\pagestyle{simple}
% \newpage
\chapter*{Resumen}
\addcontentsline{toc}{chapter}{Resumen}
\input{capitulos/resumen/main.tex}

% Índice (paginado)
\clearpage
\pagestyle{simple}
% \newpage
\tableofcontents

% Introducción (donde se incluya los antecedentes y justificación)
\clearpage
\pagestyle{myfancy}
\newpage
\chapter{Introducción}
\input{capitulos/introduccion/main.tex}

% Objetivos
\chapter{Objetivos}
\input{capitulos/objetivos/main.tex}

% Metodología
\chapter{Metodología}
\input{capitulos/metodologia/main.tex}

% Resultados y discusión (incluyendo la valoración de impactos y de aspectos de responsabilidad legal, ética y profesional relacionados con el trabajo)
\chapter{Resultados y Discusión}
\input{capitulos/resultados_discusion/main.tex}

% Conclusiones
\chapter{Conclusiones}
\input{capitulos/conclusiones/main.tex}

% Planificación temporal y presupuesto
\chapter{Planificación Temporal y Presupuesto}
\input{capitulos/planificacion_presupuesto/main.tex}

% Bibliografía
\newpage
\addcontentsline{toc}{chapter}{Bibliografía}
\printbibliography

\end{document}


% Conclusiones
\chapter{Conclusiones}
\documentclass[a4paper,11pt,twoside]{report}
\usepackage[left=25mm,right=25mm,top=25mm,bottom=25mm,includehead,includefoot,headsep=15mm,footskip=15mm]{geometry}
\usepackage{graphicx}
\usepackage{fancyhdr}
\usepackage{titlesec}
\usepackage[spanish]{babel}
\usepackage[utf8]{inputenc}
\usepackage{amsmath}
\usepackage{setspace}
\usepackage{svg}
\usepackage{hyperref}
\usepackage[backend=biber,style=numeric]{biblatex}
\addbibresource{references.bib}
\hypersetup{
    colorlinks=true,
    linkcolor=blue,      % color of internal links (sections, etc.)
    urlcolor=blue,       % color of external links
    pdftitle={Optimización energética de sistema híbrido con bomba de calor, suelo radiante, fotovoltaica y almacenamiento para vivienda},    % title
    pdfauthor={Luis D. Aranda Sánchez},     % author
    pdfkeywords={palabra1, palabra2, código1, etc.} % list of keywords
}

% Font change to Arial
\usepackage{helvet}
\renewcommand{\familydefault}{\sfdefault}

% Chapter titles in uppercase and larger font
\titleformat{\chapter}[hang]{\large\bfseries}{\thechapter.}{1em}{\MakeUppercase}
\titleformat{\section}[hang]{\bfseries}{\thesection.}{1em}{}
\titleformat{\subsection}[hang]{\bfseries}{\thesubsection.}{1em}{}

% Fancyhdr setup
\setlength{\headheight}{14.30174pt} % Adjust to recommended value, slightly larger for safety
\fancyhf{} % Clear all headers and footers
\fancyhead[LE]{\nouppercase{\leftmark}}
\fancyhead[RO]{Optimización energética para vivienda}
\fancyfoot[LE]{\thepage}
\fancyfoot[RE]{Escuela Técnica Superior de Ingenieros Industriales (UPM)}
\fancyfoot[LO]{Luis D. Aranda Sánchez}
\fancyfoot[RO]{\thepage}
\renewcommand{\headrulewidth}{0.4pt}
\renewcommand{\footrulewidth}{0.4pt}

\fancypagestyle{myfancy}{
    \fancyhf{} % Clear all headers and footers
    \fancyhead[LE]{\nouppercase{\leftmark}}
    \fancyhead[RO]{Optimización energética para vivienda}
    \fancyfoot[LE]{\thepage}
    \fancyfoot[RE]{Escuela Técnica Superior de Ingenieros Industriales (UPM)}
    \fancyfoot[LO]{Luis D. Aranda Sánchez}
    \fancyfoot[RO]{\thepage}
    \renewcommand{\headrulewidth}{0.4pt}
    \renewcommand{\footrulewidth}{0.4pt}
}

\fancypagestyle{simple}{
    \fancyhf{} % Clear all headers and footers
    \renewcommand{\headrulewidth}{0pt}
    \renewcommand{\footrulewidth}{0pt}
}

% Line spacing
\setstretch{1.2}

% Document starts here
\begin{document}

% Portada
\begin{titlepage}
    \centering
    {\scshape\LARGE Universidad Politécnica de Madrid \par}
    \vspace{1cm}
    {\scshape\Large Escuela Técnica Superior de Ingenieros Industriales\par}
    \vspace{1.5cm}
    {\huge\bfseries Optimización energética de sistema híbrido con bomba de calor, suelo radiante, fotovoltaica y almacenamiento para vivienda \par}
    \vspace{1.5cm}
    {\Large\bfseries Trabajo de Fin de Máster\par}
    \vspace{0.5cm}
    {\large Máster Universitario en Ingeniería de la Energía \par}
    \vspace{2cm}
    {\Large Luis D. Aranda Sánchez\par}
    \vfill
    Director: Javier Rodríguez Martín
    \vfill
    {\large Septiembre 6, 2024\par}
\end{titlepage}

% Resumen (máximo de 5 páginas, incluyendo al final Palabras clave)
\clearpage
\pagestyle{simple}
% \newpage
\chapter*{Resumen}
\addcontentsline{toc}{chapter}{Resumen}
\input{capitulos/resumen/main.tex}

% Índice (paginado)
\clearpage
\pagestyle{simple}
% \newpage
\tableofcontents

% Introducción (donde se incluya los antecedentes y justificación)
\clearpage
\pagestyle{myfancy}
\newpage
\chapter{Introducción}
\input{capitulos/introduccion/main.tex}

% Objetivos
\chapter{Objetivos}
\input{capitulos/objetivos/main.tex}

% Metodología
\chapter{Metodología}
\input{capitulos/metodologia/main.tex}

% Resultados y discusión (incluyendo la valoración de impactos y de aspectos de responsabilidad legal, ética y profesional relacionados con el trabajo)
\chapter{Resultados y Discusión}
\input{capitulos/resultados_discusion/main.tex}

% Conclusiones
\chapter{Conclusiones}
\input{capitulos/conclusiones/main.tex}

% Planificación temporal y presupuesto
\chapter{Planificación Temporal y Presupuesto}
\input{capitulos/planificacion_presupuesto/main.tex}

% Bibliografía
\newpage
\addcontentsline{toc}{chapter}{Bibliografía}
\printbibliography

\end{document}


% Planificación temporal y presupuesto
\chapter{Planificación Temporal y Presupuesto}
\documentclass[a4paper,11pt,twoside]{report}
\usepackage[left=25mm,right=25mm,top=25mm,bottom=25mm,includehead,includefoot,headsep=15mm,footskip=15mm]{geometry}
\usepackage{graphicx}
\usepackage{fancyhdr}
\usepackage{titlesec}
\usepackage[spanish]{babel}
\usepackage[utf8]{inputenc}
\usepackage{amsmath}
\usepackage{setspace}
\usepackage{svg}
\usepackage{hyperref}
\usepackage[backend=biber,style=numeric]{biblatex}
\addbibresource{references.bib}
\hypersetup{
    colorlinks=true,
    linkcolor=blue,      % color of internal links (sections, etc.)
    urlcolor=blue,       % color of external links
    pdftitle={Optimización energética de sistema híbrido con bomba de calor, suelo radiante, fotovoltaica y almacenamiento para vivienda},    % title
    pdfauthor={Luis D. Aranda Sánchez},     % author
    pdfkeywords={palabra1, palabra2, código1, etc.} % list of keywords
}

% Font change to Arial
\usepackage{helvet}
\renewcommand{\familydefault}{\sfdefault}

% Chapter titles in uppercase and larger font
\titleformat{\chapter}[hang]{\large\bfseries}{\thechapter.}{1em}{\MakeUppercase}
\titleformat{\section}[hang]{\bfseries}{\thesection.}{1em}{}
\titleformat{\subsection}[hang]{\bfseries}{\thesubsection.}{1em}{}

% Fancyhdr setup
\setlength{\headheight}{14.30174pt} % Adjust to recommended value, slightly larger for safety
\fancyhf{} % Clear all headers and footers
\fancyhead[LE]{\nouppercase{\leftmark}}
\fancyhead[RO]{Optimización energética para vivienda}
\fancyfoot[LE]{\thepage}
\fancyfoot[RE]{Escuela Técnica Superior de Ingenieros Industriales (UPM)}
\fancyfoot[LO]{Luis D. Aranda Sánchez}
\fancyfoot[RO]{\thepage}
\renewcommand{\headrulewidth}{0.4pt}
\renewcommand{\footrulewidth}{0.4pt}

\fancypagestyle{myfancy}{
    \fancyhf{} % Clear all headers and footers
    \fancyhead[LE]{\nouppercase{\leftmark}}
    \fancyhead[RO]{Optimización energética para vivienda}
    \fancyfoot[LE]{\thepage}
    \fancyfoot[RE]{Escuela Técnica Superior de Ingenieros Industriales (UPM)}
    \fancyfoot[LO]{Luis D. Aranda Sánchez}
    \fancyfoot[RO]{\thepage}
    \renewcommand{\headrulewidth}{0.4pt}
    \renewcommand{\footrulewidth}{0.4pt}
}

\fancypagestyle{simple}{
    \fancyhf{} % Clear all headers and footers
    \renewcommand{\headrulewidth}{0pt}
    \renewcommand{\footrulewidth}{0pt}
}

% Line spacing
\setstretch{1.2}

% Document starts here
\begin{document}

% Portada
\begin{titlepage}
    \centering
    {\scshape\LARGE Universidad Politécnica de Madrid \par}
    \vspace{1cm}
    {\scshape\Large Escuela Técnica Superior de Ingenieros Industriales\par}
    \vspace{1.5cm}
    {\huge\bfseries Optimización energética de sistema híbrido con bomba de calor, suelo radiante, fotovoltaica y almacenamiento para vivienda \par}
    \vspace{1.5cm}
    {\Large\bfseries Trabajo de Fin de Máster\par}
    \vspace{0.5cm}
    {\large Máster Universitario en Ingeniería de la Energía \par}
    \vspace{2cm}
    {\Large Luis D. Aranda Sánchez\par}
    \vfill
    Director: Javier Rodríguez Martín
    \vfill
    {\large Septiembre 6, 2024\par}
\end{titlepage}

% Resumen (máximo de 5 páginas, incluyendo al final Palabras clave)
\clearpage
\pagestyle{simple}
% \newpage
\chapter*{Resumen}
\addcontentsline{toc}{chapter}{Resumen}
\input{capitulos/resumen/main.tex}

% Índice (paginado)
\clearpage
\pagestyle{simple}
% \newpage
\tableofcontents

% Introducción (donde se incluya los antecedentes y justificación)
\clearpage
\pagestyle{myfancy}
\newpage
\chapter{Introducción}
\input{capitulos/introduccion/main.tex}

% Objetivos
\chapter{Objetivos}
\input{capitulos/objetivos/main.tex}

% Metodología
\chapter{Metodología}
\input{capitulos/metodologia/main.tex}

% Resultados y discusión (incluyendo la valoración de impactos y de aspectos de responsabilidad legal, ética y profesional relacionados con el trabajo)
\chapter{Resultados y Discusión}
\input{capitulos/resultados_discusion/main.tex}

% Conclusiones
\chapter{Conclusiones}
\input{capitulos/conclusiones/main.tex}

% Planificación temporal y presupuesto
\chapter{Planificación Temporal y Presupuesto}
\input{capitulos/planificacion_presupuesto/main.tex}

% Bibliografía
\newpage
\addcontentsline{toc}{chapter}{Bibliografía}
\printbibliography

\end{document}


% Bibliografía
\newpage
\addcontentsline{toc}{chapter}{Bibliografía}
\printbibliography

\end{document}


% Objetivos
\chapter{Objetivos}
\documentclass[a4paper,11pt,twoside]{report}
\usepackage[left=25mm,right=25mm,top=25mm,bottom=25mm,includehead,includefoot,headsep=15mm,footskip=15mm]{geometry}
\usepackage{graphicx}
\usepackage{fancyhdr}
\usepackage{titlesec}
\usepackage[spanish]{babel}
\usepackage[utf8]{inputenc}
\usepackage{amsmath}
\usepackage{setspace}
\usepackage{svg}
\usepackage{hyperref}
\usepackage[backend=biber,style=numeric]{biblatex}
\addbibresource{references.bib}
\hypersetup{
    colorlinks=true,
    linkcolor=blue,      % color of internal links (sections, etc.)
    urlcolor=blue,       % color of external links
    pdftitle={Optimización energética de sistema híbrido con bomba de calor, suelo radiante, fotovoltaica y almacenamiento para vivienda},    % title
    pdfauthor={Luis D. Aranda Sánchez},     % author
    pdfkeywords={palabra1, palabra2, código1, etc.} % list of keywords
}

% Font change to Arial
\usepackage{helvet}
\renewcommand{\familydefault}{\sfdefault}

% Chapter titles in uppercase and larger font
\titleformat{\chapter}[hang]{\large\bfseries}{\thechapter.}{1em}{\MakeUppercase}
\titleformat{\section}[hang]{\bfseries}{\thesection.}{1em}{}
\titleformat{\subsection}[hang]{\bfseries}{\thesubsection.}{1em}{}

% Fancyhdr setup
\setlength{\headheight}{14.30174pt} % Adjust to recommended value, slightly larger for safety
\fancyhf{} % Clear all headers and footers
\fancyhead[LE]{\nouppercase{\leftmark}}
\fancyhead[RO]{Optimización energética para vivienda}
\fancyfoot[LE]{\thepage}
\fancyfoot[RE]{Escuela Técnica Superior de Ingenieros Industriales (UPM)}
\fancyfoot[LO]{Luis D. Aranda Sánchez}
\fancyfoot[RO]{\thepage}
\renewcommand{\headrulewidth}{0.4pt}
\renewcommand{\footrulewidth}{0.4pt}

\fancypagestyle{myfancy}{
    \fancyhf{} % Clear all headers and footers
    \fancyhead[LE]{\nouppercase{\leftmark}}
    \fancyhead[RO]{Optimización energética para vivienda}
    \fancyfoot[LE]{\thepage}
    \fancyfoot[RE]{Escuela Técnica Superior de Ingenieros Industriales (UPM)}
    \fancyfoot[LO]{Luis D. Aranda Sánchez}
    \fancyfoot[RO]{\thepage}
    \renewcommand{\headrulewidth}{0.4pt}
    \renewcommand{\footrulewidth}{0.4pt}
}

\fancypagestyle{simple}{
    \fancyhf{} % Clear all headers and footers
    \renewcommand{\headrulewidth}{0pt}
    \renewcommand{\footrulewidth}{0pt}
}

% Line spacing
\setstretch{1.2}

% Document starts here
\begin{document}

% Portada
\begin{titlepage}
    \centering
    {\scshape\LARGE Universidad Politécnica de Madrid \par}
    \vspace{1cm}
    {\scshape\Large Escuela Técnica Superior de Ingenieros Industriales\par}
    \vspace{1.5cm}
    {\huge\bfseries Optimización energética de sistema híbrido con bomba de calor, suelo radiante, fotovoltaica y almacenamiento para vivienda \par}
    \vspace{1.5cm}
    {\Large\bfseries Trabajo de Fin de Máster\par}
    \vspace{0.5cm}
    {\large Máster Universitario en Ingeniería de la Energía \par}
    \vspace{2cm}
    {\Large Luis D. Aranda Sánchez\par}
    \vfill
    Director: Javier Rodríguez Martín
    \vfill
    {\large Septiembre 6, 2024\par}
\end{titlepage}

% Resumen (máximo de 5 páginas, incluyendo al final Palabras clave)
\clearpage
\pagestyle{simple}
% \newpage
\chapter*{Resumen}
\addcontentsline{toc}{chapter}{Resumen}
\documentclass[a4paper,11pt,twoside]{report}
\usepackage[left=25mm,right=25mm,top=25mm,bottom=25mm,includehead,includefoot,headsep=15mm,footskip=15mm]{geometry}
\usepackage{graphicx}
\usepackage{fancyhdr}
\usepackage{titlesec}
\usepackage[spanish]{babel}
\usepackage[utf8]{inputenc}
\usepackage{amsmath}
\usepackage{setspace}
\usepackage{svg}
\usepackage{hyperref}
\usepackage[backend=biber,style=numeric]{biblatex}
\addbibresource{references.bib}
\hypersetup{
    colorlinks=true,
    linkcolor=blue,      % color of internal links (sections, etc.)
    urlcolor=blue,       % color of external links
    pdftitle={Optimización energética de sistema híbrido con bomba de calor, suelo radiante, fotovoltaica y almacenamiento para vivienda},    % title
    pdfauthor={Luis D. Aranda Sánchez},     % author
    pdfkeywords={palabra1, palabra2, código1, etc.} % list of keywords
}

% Font change to Arial
\usepackage{helvet}
\renewcommand{\familydefault}{\sfdefault}

% Chapter titles in uppercase and larger font
\titleformat{\chapter}[hang]{\large\bfseries}{\thechapter.}{1em}{\MakeUppercase}
\titleformat{\section}[hang]{\bfseries}{\thesection.}{1em}{}
\titleformat{\subsection}[hang]{\bfseries}{\thesubsection.}{1em}{}

% Fancyhdr setup
\setlength{\headheight}{14.30174pt} % Adjust to recommended value, slightly larger for safety
\fancyhf{} % Clear all headers and footers
\fancyhead[LE]{\nouppercase{\leftmark}}
\fancyhead[RO]{Optimización energética para vivienda}
\fancyfoot[LE]{\thepage}
\fancyfoot[RE]{Escuela Técnica Superior de Ingenieros Industriales (UPM)}
\fancyfoot[LO]{Luis D. Aranda Sánchez}
\fancyfoot[RO]{\thepage}
\renewcommand{\headrulewidth}{0.4pt}
\renewcommand{\footrulewidth}{0.4pt}

\fancypagestyle{myfancy}{
    \fancyhf{} % Clear all headers and footers
    \fancyhead[LE]{\nouppercase{\leftmark}}
    \fancyhead[RO]{Optimización energética para vivienda}
    \fancyfoot[LE]{\thepage}
    \fancyfoot[RE]{Escuela Técnica Superior de Ingenieros Industriales (UPM)}
    \fancyfoot[LO]{Luis D. Aranda Sánchez}
    \fancyfoot[RO]{\thepage}
    \renewcommand{\headrulewidth}{0.4pt}
    \renewcommand{\footrulewidth}{0.4pt}
}

\fancypagestyle{simple}{
    \fancyhf{} % Clear all headers and footers
    \renewcommand{\headrulewidth}{0pt}
    \renewcommand{\footrulewidth}{0pt}
}

% Line spacing
\setstretch{1.2}

% Document starts here
\begin{document}

% Portada
\begin{titlepage}
    \centering
    {\scshape\LARGE Universidad Politécnica de Madrid \par}
    \vspace{1cm}
    {\scshape\Large Escuela Técnica Superior de Ingenieros Industriales\par}
    \vspace{1.5cm}
    {\huge\bfseries Optimización energética de sistema híbrido con bomba de calor, suelo radiante, fotovoltaica y almacenamiento para vivienda \par}
    \vspace{1.5cm}
    {\Large\bfseries Trabajo de Fin de Máster\par}
    \vspace{0.5cm}
    {\large Máster Universitario en Ingeniería de la Energía \par}
    \vspace{2cm}
    {\Large Luis D. Aranda Sánchez\par}
    \vfill
    Director: Javier Rodríguez Martín
    \vfill
    {\large Septiembre 6, 2024\par}
\end{titlepage}

% Resumen (máximo de 5 páginas, incluyendo al final Palabras clave)
\clearpage
\pagestyle{simple}
% \newpage
\chapter*{Resumen}
\addcontentsline{toc}{chapter}{Resumen}
\input{capitulos/resumen/main.tex}

% Índice (paginado)
\clearpage
\pagestyle{simple}
% \newpage
\tableofcontents

% Introducción (donde se incluya los antecedentes y justificación)
\clearpage
\pagestyle{myfancy}
\newpage
\chapter{Introducción}
\input{capitulos/introduccion/main.tex}

% Objetivos
\chapter{Objetivos}
\input{capitulos/objetivos/main.tex}

% Metodología
\chapter{Metodología}
\input{capitulos/metodologia/main.tex}

% Resultados y discusión (incluyendo la valoración de impactos y de aspectos de responsabilidad legal, ética y profesional relacionados con el trabajo)
\chapter{Resultados y Discusión}
\input{capitulos/resultados_discusion/main.tex}

% Conclusiones
\chapter{Conclusiones}
\input{capitulos/conclusiones/main.tex}

% Planificación temporal y presupuesto
\chapter{Planificación Temporal y Presupuesto}
\input{capitulos/planificacion_presupuesto/main.tex}

% Bibliografía
\newpage
\addcontentsline{toc}{chapter}{Bibliografía}
\printbibliography

\end{document}


% Índice (paginado)
\clearpage
\pagestyle{simple}
% \newpage
\tableofcontents

% Introducción (donde se incluya los antecedentes y justificación)
\clearpage
\pagestyle{myfancy}
\newpage
\chapter{Introducción}
\documentclass[a4paper,11pt,twoside]{report}
\usepackage[left=25mm,right=25mm,top=25mm,bottom=25mm,includehead,includefoot,headsep=15mm,footskip=15mm]{geometry}
\usepackage{graphicx}
\usepackage{fancyhdr}
\usepackage{titlesec}
\usepackage[spanish]{babel}
\usepackage[utf8]{inputenc}
\usepackage{amsmath}
\usepackage{setspace}
\usepackage{svg}
\usepackage{hyperref}
\usepackage[backend=biber,style=numeric]{biblatex}
\addbibresource{references.bib}
\hypersetup{
    colorlinks=true,
    linkcolor=blue,      % color of internal links (sections, etc.)
    urlcolor=blue,       % color of external links
    pdftitle={Optimización energética de sistema híbrido con bomba de calor, suelo radiante, fotovoltaica y almacenamiento para vivienda},    % title
    pdfauthor={Luis D. Aranda Sánchez},     % author
    pdfkeywords={palabra1, palabra2, código1, etc.} % list of keywords
}

% Font change to Arial
\usepackage{helvet}
\renewcommand{\familydefault}{\sfdefault}

% Chapter titles in uppercase and larger font
\titleformat{\chapter}[hang]{\large\bfseries}{\thechapter.}{1em}{\MakeUppercase}
\titleformat{\section}[hang]{\bfseries}{\thesection.}{1em}{}
\titleformat{\subsection}[hang]{\bfseries}{\thesubsection.}{1em}{}

% Fancyhdr setup
\setlength{\headheight}{14.30174pt} % Adjust to recommended value, slightly larger for safety
\fancyhf{} % Clear all headers and footers
\fancyhead[LE]{\nouppercase{\leftmark}}
\fancyhead[RO]{Optimización energética para vivienda}
\fancyfoot[LE]{\thepage}
\fancyfoot[RE]{Escuela Técnica Superior de Ingenieros Industriales (UPM)}
\fancyfoot[LO]{Luis D. Aranda Sánchez}
\fancyfoot[RO]{\thepage}
\renewcommand{\headrulewidth}{0.4pt}
\renewcommand{\footrulewidth}{0.4pt}

\fancypagestyle{myfancy}{
    \fancyhf{} % Clear all headers and footers
    \fancyhead[LE]{\nouppercase{\leftmark}}
    \fancyhead[RO]{Optimización energética para vivienda}
    \fancyfoot[LE]{\thepage}
    \fancyfoot[RE]{Escuela Técnica Superior de Ingenieros Industriales (UPM)}
    \fancyfoot[LO]{Luis D. Aranda Sánchez}
    \fancyfoot[RO]{\thepage}
    \renewcommand{\headrulewidth}{0.4pt}
    \renewcommand{\footrulewidth}{0.4pt}
}

\fancypagestyle{simple}{
    \fancyhf{} % Clear all headers and footers
    \renewcommand{\headrulewidth}{0pt}
    \renewcommand{\footrulewidth}{0pt}
}

% Line spacing
\setstretch{1.2}

% Document starts here
\begin{document}

% Portada
\begin{titlepage}
    \centering
    {\scshape\LARGE Universidad Politécnica de Madrid \par}
    \vspace{1cm}
    {\scshape\Large Escuela Técnica Superior de Ingenieros Industriales\par}
    \vspace{1.5cm}
    {\huge\bfseries Optimización energética de sistema híbrido con bomba de calor, suelo radiante, fotovoltaica y almacenamiento para vivienda \par}
    \vspace{1.5cm}
    {\Large\bfseries Trabajo de Fin de Máster\par}
    \vspace{0.5cm}
    {\large Máster Universitario en Ingeniería de la Energía \par}
    \vspace{2cm}
    {\Large Luis D. Aranda Sánchez\par}
    \vfill
    Director: Javier Rodríguez Martín
    \vfill
    {\large Septiembre 6, 2024\par}
\end{titlepage}

% Resumen (máximo de 5 páginas, incluyendo al final Palabras clave)
\clearpage
\pagestyle{simple}
% \newpage
\chapter*{Resumen}
\addcontentsline{toc}{chapter}{Resumen}
\input{capitulos/resumen/main.tex}

% Índice (paginado)
\clearpage
\pagestyle{simple}
% \newpage
\tableofcontents

% Introducción (donde se incluya los antecedentes y justificación)
\clearpage
\pagestyle{myfancy}
\newpage
\chapter{Introducción}
\input{capitulos/introduccion/main.tex}

% Objetivos
\chapter{Objetivos}
\input{capitulos/objetivos/main.tex}

% Metodología
\chapter{Metodología}
\input{capitulos/metodologia/main.tex}

% Resultados y discusión (incluyendo la valoración de impactos y de aspectos de responsabilidad legal, ética y profesional relacionados con el trabajo)
\chapter{Resultados y Discusión}
\input{capitulos/resultados_discusion/main.tex}

% Conclusiones
\chapter{Conclusiones}
\input{capitulos/conclusiones/main.tex}

% Planificación temporal y presupuesto
\chapter{Planificación Temporal y Presupuesto}
\input{capitulos/planificacion_presupuesto/main.tex}

% Bibliografía
\newpage
\addcontentsline{toc}{chapter}{Bibliografía}
\printbibliography

\end{document}


% Objetivos
\chapter{Objetivos}
\documentclass[a4paper,11pt,twoside]{report}
\usepackage[left=25mm,right=25mm,top=25mm,bottom=25mm,includehead,includefoot,headsep=15mm,footskip=15mm]{geometry}
\usepackage{graphicx}
\usepackage{fancyhdr}
\usepackage{titlesec}
\usepackage[spanish]{babel}
\usepackage[utf8]{inputenc}
\usepackage{amsmath}
\usepackage{setspace}
\usepackage{svg}
\usepackage{hyperref}
\usepackage[backend=biber,style=numeric]{biblatex}
\addbibresource{references.bib}
\hypersetup{
    colorlinks=true,
    linkcolor=blue,      % color of internal links (sections, etc.)
    urlcolor=blue,       % color of external links
    pdftitle={Optimización energética de sistema híbrido con bomba de calor, suelo radiante, fotovoltaica y almacenamiento para vivienda},    % title
    pdfauthor={Luis D. Aranda Sánchez},     % author
    pdfkeywords={palabra1, palabra2, código1, etc.} % list of keywords
}

% Font change to Arial
\usepackage{helvet}
\renewcommand{\familydefault}{\sfdefault}

% Chapter titles in uppercase and larger font
\titleformat{\chapter}[hang]{\large\bfseries}{\thechapter.}{1em}{\MakeUppercase}
\titleformat{\section}[hang]{\bfseries}{\thesection.}{1em}{}
\titleformat{\subsection}[hang]{\bfseries}{\thesubsection.}{1em}{}

% Fancyhdr setup
\setlength{\headheight}{14.30174pt} % Adjust to recommended value, slightly larger for safety
\fancyhf{} % Clear all headers and footers
\fancyhead[LE]{\nouppercase{\leftmark}}
\fancyhead[RO]{Optimización energética para vivienda}
\fancyfoot[LE]{\thepage}
\fancyfoot[RE]{Escuela Técnica Superior de Ingenieros Industriales (UPM)}
\fancyfoot[LO]{Luis D. Aranda Sánchez}
\fancyfoot[RO]{\thepage}
\renewcommand{\headrulewidth}{0.4pt}
\renewcommand{\footrulewidth}{0.4pt}

\fancypagestyle{myfancy}{
    \fancyhf{} % Clear all headers and footers
    \fancyhead[LE]{\nouppercase{\leftmark}}
    \fancyhead[RO]{Optimización energética para vivienda}
    \fancyfoot[LE]{\thepage}
    \fancyfoot[RE]{Escuela Técnica Superior de Ingenieros Industriales (UPM)}
    \fancyfoot[LO]{Luis D. Aranda Sánchez}
    \fancyfoot[RO]{\thepage}
    \renewcommand{\headrulewidth}{0.4pt}
    \renewcommand{\footrulewidth}{0.4pt}
}

\fancypagestyle{simple}{
    \fancyhf{} % Clear all headers and footers
    \renewcommand{\headrulewidth}{0pt}
    \renewcommand{\footrulewidth}{0pt}
}

% Line spacing
\setstretch{1.2}

% Document starts here
\begin{document}

% Portada
\begin{titlepage}
    \centering
    {\scshape\LARGE Universidad Politécnica de Madrid \par}
    \vspace{1cm}
    {\scshape\Large Escuela Técnica Superior de Ingenieros Industriales\par}
    \vspace{1.5cm}
    {\huge\bfseries Optimización energética de sistema híbrido con bomba de calor, suelo radiante, fotovoltaica y almacenamiento para vivienda \par}
    \vspace{1.5cm}
    {\Large\bfseries Trabajo de Fin de Máster\par}
    \vspace{0.5cm}
    {\large Máster Universitario en Ingeniería de la Energía \par}
    \vspace{2cm}
    {\Large Luis D. Aranda Sánchez\par}
    \vfill
    Director: Javier Rodríguez Martín
    \vfill
    {\large Septiembre 6, 2024\par}
\end{titlepage}

% Resumen (máximo de 5 páginas, incluyendo al final Palabras clave)
\clearpage
\pagestyle{simple}
% \newpage
\chapter*{Resumen}
\addcontentsline{toc}{chapter}{Resumen}
\input{capitulos/resumen/main.tex}

% Índice (paginado)
\clearpage
\pagestyle{simple}
% \newpage
\tableofcontents

% Introducción (donde se incluya los antecedentes y justificación)
\clearpage
\pagestyle{myfancy}
\newpage
\chapter{Introducción}
\input{capitulos/introduccion/main.tex}

% Objetivos
\chapter{Objetivos}
\input{capitulos/objetivos/main.tex}

% Metodología
\chapter{Metodología}
\input{capitulos/metodologia/main.tex}

% Resultados y discusión (incluyendo la valoración de impactos y de aspectos de responsabilidad legal, ética y profesional relacionados con el trabajo)
\chapter{Resultados y Discusión}
\input{capitulos/resultados_discusion/main.tex}

% Conclusiones
\chapter{Conclusiones}
\input{capitulos/conclusiones/main.tex}

% Planificación temporal y presupuesto
\chapter{Planificación Temporal y Presupuesto}
\input{capitulos/planificacion_presupuesto/main.tex}

% Bibliografía
\newpage
\addcontentsline{toc}{chapter}{Bibliografía}
\printbibliography

\end{document}


% Metodología
\chapter{Metodología}
\documentclass[a4paper,11pt,twoside]{report}
\usepackage[left=25mm,right=25mm,top=25mm,bottom=25mm,includehead,includefoot,headsep=15mm,footskip=15mm]{geometry}
\usepackage{graphicx}
\usepackage{fancyhdr}
\usepackage{titlesec}
\usepackage[spanish]{babel}
\usepackage[utf8]{inputenc}
\usepackage{amsmath}
\usepackage{setspace}
\usepackage{svg}
\usepackage{hyperref}
\usepackage[backend=biber,style=numeric]{biblatex}
\addbibresource{references.bib}
\hypersetup{
    colorlinks=true,
    linkcolor=blue,      % color of internal links (sections, etc.)
    urlcolor=blue,       % color of external links
    pdftitle={Optimización energética de sistema híbrido con bomba de calor, suelo radiante, fotovoltaica y almacenamiento para vivienda},    % title
    pdfauthor={Luis D. Aranda Sánchez},     % author
    pdfkeywords={palabra1, palabra2, código1, etc.} % list of keywords
}

% Font change to Arial
\usepackage{helvet}
\renewcommand{\familydefault}{\sfdefault}

% Chapter titles in uppercase and larger font
\titleformat{\chapter}[hang]{\large\bfseries}{\thechapter.}{1em}{\MakeUppercase}
\titleformat{\section}[hang]{\bfseries}{\thesection.}{1em}{}
\titleformat{\subsection}[hang]{\bfseries}{\thesubsection.}{1em}{}

% Fancyhdr setup
\setlength{\headheight}{14.30174pt} % Adjust to recommended value, slightly larger for safety
\fancyhf{} % Clear all headers and footers
\fancyhead[LE]{\nouppercase{\leftmark}}
\fancyhead[RO]{Optimización energética para vivienda}
\fancyfoot[LE]{\thepage}
\fancyfoot[RE]{Escuela Técnica Superior de Ingenieros Industriales (UPM)}
\fancyfoot[LO]{Luis D. Aranda Sánchez}
\fancyfoot[RO]{\thepage}
\renewcommand{\headrulewidth}{0.4pt}
\renewcommand{\footrulewidth}{0.4pt}

\fancypagestyle{myfancy}{
    \fancyhf{} % Clear all headers and footers
    \fancyhead[LE]{\nouppercase{\leftmark}}
    \fancyhead[RO]{Optimización energética para vivienda}
    \fancyfoot[LE]{\thepage}
    \fancyfoot[RE]{Escuela Técnica Superior de Ingenieros Industriales (UPM)}
    \fancyfoot[LO]{Luis D. Aranda Sánchez}
    \fancyfoot[RO]{\thepage}
    \renewcommand{\headrulewidth}{0.4pt}
    \renewcommand{\footrulewidth}{0.4pt}
}

\fancypagestyle{simple}{
    \fancyhf{} % Clear all headers and footers
    \renewcommand{\headrulewidth}{0pt}
    \renewcommand{\footrulewidth}{0pt}
}

% Line spacing
\setstretch{1.2}

% Document starts here
\begin{document}

% Portada
\begin{titlepage}
    \centering
    {\scshape\LARGE Universidad Politécnica de Madrid \par}
    \vspace{1cm}
    {\scshape\Large Escuela Técnica Superior de Ingenieros Industriales\par}
    \vspace{1.5cm}
    {\huge\bfseries Optimización energética de sistema híbrido con bomba de calor, suelo radiante, fotovoltaica y almacenamiento para vivienda \par}
    \vspace{1.5cm}
    {\Large\bfseries Trabajo de Fin de Máster\par}
    \vspace{0.5cm}
    {\large Máster Universitario en Ingeniería de la Energía \par}
    \vspace{2cm}
    {\Large Luis D. Aranda Sánchez\par}
    \vfill
    Director: Javier Rodríguez Martín
    \vfill
    {\large Septiembre 6, 2024\par}
\end{titlepage}

% Resumen (máximo de 5 páginas, incluyendo al final Palabras clave)
\clearpage
\pagestyle{simple}
% \newpage
\chapter*{Resumen}
\addcontentsline{toc}{chapter}{Resumen}
\input{capitulos/resumen/main.tex}

% Índice (paginado)
\clearpage
\pagestyle{simple}
% \newpage
\tableofcontents

% Introducción (donde se incluya los antecedentes y justificación)
\clearpage
\pagestyle{myfancy}
\newpage
\chapter{Introducción}
\input{capitulos/introduccion/main.tex}

% Objetivos
\chapter{Objetivos}
\input{capitulos/objetivos/main.tex}

% Metodología
\chapter{Metodología}
\input{capitulos/metodologia/main.tex}

% Resultados y discusión (incluyendo la valoración de impactos y de aspectos de responsabilidad legal, ética y profesional relacionados con el trabajo)
\chapter{Resultados y Discusión}
\input{capitulos/resultados_discusion/main.tex}

% Conclusiones
\chapter{Conclusiones}
\input{capitulos/conclusiones/main.tex}

% Planificación temporal y presupuesto
\chapter{Planificación Temporal y Presupuesto}
\input{capitulos/planificacion_presupuesto/main.tex}

% Bibliografía
\newpage
\addcontentsline{toc}{chapter}{Bibliografía}
\printbibliography

\end{document}


% Resultados y discusión (incluyendo la valoración de impactos y de aspectos de responsabilidad legal, ética y profesional relacionados con el trabajo)
\chapter{Resultados y Discusión}
\documentclass[a4paper,11pt,twoside]{report}
\usepackage[left=25mm,right=25mm,top=25mm,bottom=25mm,includehead,includefoot,headsep=15mm,footskip=15mm]{geometry}
\usepackage{graphicx}
\usepackage{fancyhdr}
\usepackage{titlesec}
\usepackage[spanish]{babel}
\usepackage[utf8]{inputenc}
\usepackage{amsmath}
\usepackage{setspace}
\usepackage{svg}
\usepackage{hyperref}
\usepackage[backend=biber,style=numeric]{biblatex}
\addbibresource{references.bib}
\hypersetup{
    colorlinks=true,
    linkcolor=blue,      % color of internal links (sections, etc.)
    urlcolor=blue,       % color of external links
    pdftitle={Optimización energética de sistema híbrido con bomba de calor, suelo radiante, fotovoltaica y almacenamiento para vivienda},    % title
    pdfauthor={Luis D. Aranda Sánchez},     % author
    pdfkeywords={palabra1, palabra2, código1, etc.} % list of keywords
}

% Font change to Arial
\usepackage{helvet}
\renewcommand{\familydefault}{\sfdefault}

% Chapter titles in uppercase and larger font
\titleformat{\chapter}[hang]{\large\bfseries}{\thechapter.}{1em}{\MakeUppercase}
\titleformat{\section}[hang]{\bfseries}{\thesection.}{1em}{}
\titleformat{\subsection}[hang]{\bfseries}{\thesubsection.}{1em}{}

% Fancyhdr setup
\setlength{\headheight}{14.30174pt} % Adjust to recommended value, slightly larger for safety
\fancyhf{} % Clear all headers and footers
\fancyhead[LE]{\nouppercase{\leftmark}}
\fancyhead[RO]{Optimización energética para vivienda}
\fancyfoot[LE]{\thepage}
\fancyfoot[RE]{Escuela Técnica Superior de Ingenieros Industriales (UPM)}
\fancyfoot[LO]{Luis D. Aranda Sánchez}
\fancyfoot[RO]{\thepage}
\renewcommand{\headrulewidth}{0.4pt}
\renewcommand{\footrulewidth}{0.4pt}

\fancypagestyle{myfancy}{
    \fancyhf{} % Clear all headers and footers
    \fancyhead[LE]{\nouppercase{\leftmark}}
    \fancyhead[RO]{Optimización energética para vivienda}
    \fancyfoot[LE]{\thepage}
    \fancyfoot[RE]{Escuela Técnica Superior de Ingenieros Industriales (UPM)}
    \fancyfoot[LO]{Luis D. Aranda Sánchez}
    \fancyfoot[RO]{\thepage}
    \renewcommand{\headrulewidth}{0.4pt}
    \renewcommand{\footrulewidth}{0.4pt}
}

\fancypagestyle{simple}{
    \fancyhf{} % Clear all headers and footers
    \renewcommand{\headrulewidth}{0pt}
    \renewcommand{\footrulewidth}{0pt}
}

% Line spacing
\setstretch{1.2}

% Document starts here
\begin{document}

% Portada
\begin{titlepage}
    \centering
    {\scshape\LARGE Universidad Politécnica de Madrid \par}
    \vspace{1cm}
    {\scshape\Large Escuela Técnica Superior de Ingenieros Industriales\par}
    \vspace{1.5cm}
    {\huge\bfseries Optimización energética de sistema híbrido con bomba de calor, suelo radiante, fotovoltaica y almacenamiento para vivienda \par}
    \vspace{1.5cm}
    {\Large\bfseries Trabajo de Fin de Máster\par}
    \vspace{0.5cm}
    {\large Máster Universitario en Ingeniería de la Energía \par}
    \vspace{2cm}
    {\Large Luis D. Aranda Sánchez\par}
    \vfill
    Director: Javier Rodríguez Martín
    \vfill
    {\large Septiembre 6, 2024\par}
\end{titlepage}

% Resumen (máximo de 5 páginas, incluyendo al final Palabras clave)
\clearpage
\pagestyle{simple}
% \newpage
\chapter*{Resumen}
\addcontentsline{toc}{chapter}{Resumen}
\input{capitulos/resumen/main.tex}

% Índice (paginado)
\clearpage
\pagestyle{simple}
% \newpage
\tableofcontents

% Introducción (donde se incluya los antecedentes y justificación)
\clearpage
\pagestyle{myfancy}
\newpage
\chapter{Introducción}
\input{capitulos/introduccion/main.tex}

% Objetivos
\chapter{Objetivos}
\input{capitulos/objetivos/main.tex}

% Metodología
\chapter{Metodología}
\input{capitulos/metodologia/main.tex}

% Resultados y discusión (incluyendo la valoración de impactos y de aspectos de responsabilidad legal, ética y profesional relacionados con el trabajo)
\chapter{Resultados y Discusión}
\input{capitulos/resultados_discusion/main.tex}

% Conclusiones
\chapter{Conclusiones}
\input{capitulos/conclusiones/main.tex}

% Planificación temporal y presupuesto
\chapter{Planificación Temporal y Presupuesto}
\input{capitulos/planificacion_presupuesto/main.tex}

% Bibliografía
\newpage
\addcontentsline{toc}{chapter}{Bibliografía}
\printbibliography

\end{document}


% Conclusiones
\chapter{Conclusiones}
\documentclass[a4paper,11pt,twoside]{report}
\usepackage[left=25mm,right=25mm,top=25mm,bottom=25mm,includehead,includefoot,headsep=15mm,footskip=15mm]{geometry}
\usepackage{graphicx}
\usepackage{fancyhdr}
\usepackage{titlesec}
\usepackage[spanish]{babel}
\usepackage[utf8]{inputenc}
\usepackage{amsmath}
\usepackage{setspace}
\usepackage{svg}
\usepackage{hyperref}
\usepackage[backend=biber,style=numeric]{biblatex}
\addbibresource{references.bib}
\hypersetup{
    colorlinks=true,
    linkcolor=blue,      % color of internal links (sections, etc.)
    urlcolor=blue,       % color of external links
    pdftitle={Optimización energética de sistema híbrido con bomba de calor, suelo radiante, fotovoltaica y almacenamiento para vivienda},    % title
    pdfauthor={Luis D. Aranda Sánchez},     % author
    pdfkeywords={palabra1, palabra2, código1, etc.} % list of keywords
}

% Font change to Arial
\usepackage{helvet}
\renewcommand{\familydefault}{\sfdefault}

% Chapter titles in uppercase and larger font
\titleformat{\chapter}[hang]{\large\bfseries}{\thechapter.}{1em}{\MakeUppercase}
\titleformat{\section}[hang]{\bfseries}{\thesection.}{1em}{}
\titleformat{\subsection}[hang]{\bfseries}{\thesubsection.}{1em}{}

% Fancyhdr setup
\setlength{\headheight}{14.30174pt} % Adjust to recommended value, slightly larger for safety
\fancyhf{} % Clear all headers and footers
\fancyhead[LE]{\nouppercase{\leftmark}}
\fancyhead[RO]{Optimización energética para vivienda}
\fancyfoot[LE]{\thepage}
\fancyfoot[RE]{Escuela Técnica Superior de Ingenieros Industriales (UPM)}
\fancyfoot[LO]{Luis D. Aranda Sánchez}
\fancyfoot[RO]{\thepage}
\renewcommand{\headrulewidth}{0.4pt}
\renewcommand{\footrulewidth}{0.4pt}

\fancypagestyle{myfancy}{
    \fancyhf{} % Clear all headers and footers
    \fancyhead[LE]{\nouppercase{\leftmark}}
    \fancyhead[RO]{Optimización energética para vivienda}
    \fancyfoot[LE]{\thepage}
    \fancyfoot[RE]{Escuela Técnica Superior de Ingenieros Industriales (UPM)}
    \fancyfoot[LO]{Luis D. Aranda Sánchez}
    \fancyfoot[RO]{\thepage}
    \renewcommand{\headrulewidth}{0.4pt}
    \renewcommand{\footrulewidth}{0.4pt}
}

\fancypagestyle{simple}{
    \fancyhf{} % Clear all headers and footers
    \renewcommand{\headrulewidth}{0pt}
    \renewcommand{\footrulewidth}{0pt}
}

% Line spacing
\setstretch{1.2}

% Document starts here
\begin{document}

% Portada
\begin{titlepage}
    \centering
    {\scshape\LARGE Universidad Politécnica de Madrid \par}
    \vspace{1cm}
    {\scshape\Large Escuela Técnica Superior de Ingenieros Industriales\par}
    \vspace{1.5cm}
    {\huge\bfseries Optimización energética de sistema híbrido con bomba de calor, suelo radiante, fotovoltaica y almacenamiento para vivienda \par}
    \vspace{1.5cm}
    {\Large\bfseries Trabajo de Fin de Máster\par}
    \vspace{0.5cm}
    {\large Máster Universitario en Ingeniería de la Energía \par}
    \vspace{2cm}
    {\Large Luis D. Aranda Sánchez\par}
    \vfill
    Director: Javier Rodríguez Martín
    \vfill
    {\large Septiembre 6, 2024\par}
\end{titlepage}

% Resumen (máximo de 5 páginas, incluyendo al final Palabras clave)
\clearpage
\pagestyle{simple}
% \newpage
\chapter*{Resumen}
\addcontentsline{toc}{chapter}{Resumen}
\input{capitulos/resumen/main.tex}

% Índice (paginado)
\clearpage
\pagestyle{simple}
% \newpage
\tableofcontents

% Introducción (donde se incluya los antecedentes y justificación)
\clearpage
\pagestyle{myfancy}
\newpage
\chapter{Introducción}
\input{capitulos/introduccion/main.tex}

% Objetivos
\chapter{Objetivos}
\input{capitulos/objetivos/main.tex}

% Metodología
\chapter{Metodología}
\input{capitulos/metodologia/main.tex}

% Resultados y discusión (incluyendo la valoración de impactos y de aspectos de responsabilidad legal, ética y profesional relacionados con el trabajo)
\chapter{Resultados y Discusión}
\input{capitulos/resultados_discusion/main.tex}

% Conclusiones
\chapter{Conclusiones}
\input{capitulos/conclusiones/main.tex}

% Planificación temporal y presupuesto
\chapter{Planificación Temporal y Presupuesto}
\input{capitulos/planificacion_presupuesto/main.tex}

% Bibliografía
\newpage
\addcontentsline{toc}{chapter}{Bibliografía}
\printbibliography

\end{document}


% Planificación temporal y presupuesto
\chapter{Planificación Temporal y Presupuesto}
\documentclass[a4paper,11pt,twoside]{report}
\usepackage[left=25mm,right=25mm,top=25mm,bottom=25mm,includehead,includefoot,headsep=15mm,footskip=15mm]{geometry}
\usepackage{graphicx}
\usepackage{fancyhdr}
\usepackage{titlesec}
\usepackage[spanish]{babel}
\usepackage[utf8]{inputenc}
\usepackage{amsmath}
\usepackage{setspace}
\usepackage{svg}
\usepackage{hyperref}
\usepackage[backend=biber,style=numeric]{biblatex}
\addbibresource{references.bib}
\hypersetup{
    colorlinks=true,
    linkcolor=blue,      % color of internal links (sections, etc.)
    urlcolor=blue,       % color of external links
    pdftitle={Optimización energética de sistema híbrido con bomba de calor, suelo radiante, fotovoltaica y almacenamiento para vivienda},    % title
    pdfauthor={Luis D. Aranda Sánchez},     % author
    pdfkeywords={palabra1, palabra2, código1, etc.} % list of keywords
}

% Font change to Arial
\usepackage{helvet}
\renewcommand{\familydefault}{\sfdefault}

% Chapter titles in uppercase and larger font
\titleformat{\chapter}[hang]{\large\bfseries}{\thechapter.}{1em}{\MakeUppercase}
\titleformat{\section}[hang]{\bfseries}{\thesection.}{1em}{}
\titleformat{\subsection}[hang]{\bfseries}{\thesubsection.}{1em}{}

% Fancyhdr setup
\setlength{\headheight}{14.30174pt} % Adjust to recommended value, slightly larger for safety
\fancyhf{} % Clear all headers and footers
\fancyhead[LE]{\nouppercase{\leftmark}}
\fancyhead[RO]{Optimización energética para vivienda}
\fancyfoot[LE]{\thepage}
\fancyfoot[RE]{Escuela Técnica Superior de Ingenieros Industriales (UPM)}
\fancyfoot[LO]{Luis D. Aranda Sánchez}
\fancyfoot[RO]{\thepage}
\renewcommand{\headrulewidth}{0.4pt}
\renewcommand{\footrulewidth}{0.4pt}

\fancypagestyle{myfancy}{
    \fancyhf{} % Clear all headers and footers
    \fancyhead[LE]{\nouppercase{\leftmark}}
    \fancyhead[RO]{Optimización energética para vivienda}
    \fancyfoot[LE]{\thepage}
    \fancyfoot[RE]{Escuela Técnica Superior de Ingenieros Industriales (UPM)}
    \fancyfoot[LO]{Luis D. Aranda Sánchez}
    \fancyfoot[RO]{\thepage}
    \renewcommand{\headrulewidth}{0.4pt}
    \renewcommand{\footrulewidth}{0.4pt}
}

\fancypagestyle{simple}{
    \fancyhf{} % Clear all headers and footers
    \renewcommand{\headrulewidth}{0pt}
    \renewcommand{\footrulewidth}{0pt}
}

% Line spacing
\setstretch{1.2}

% Document starts here
\begin{document}

% Portada
\begin{titlepage}
    \centering
    {\scshape\LARGE Universidad Politécnica de Madrid \par}
    \vspace{1cm}
    {\scshape\Large Escuela Técnica Superior de Ingenieros Industriales\par}
    \vspace{1.5cm}
    {\huge\bfseries Optimización energética de sistema híbrido con bomba de calor, suelo radiante, fotovoltaica y almacenamiento para vivienda \par}
    \vspace{1.5cm}
    {\Large\bfseries Trabajo de Fin de Máster\par}
    \vspace{0.5cm}
    {\large Máster Universitario en Ingeniería de la Energía \par}
    \vspace{2cm}
    {\Large Luis D. Aranda Sánchez\par}
    \vfill
    Director: Javier Rodríguez Martín
    \vfill
    {\large Septiembre 6, 2024\par}
\end{titlepage}

% Resumen (máximo de 5 páginas, incluyendo al final Palabras clave)
\clearpage
\pagestyle{simple}
% \newpage
\chapter*{Resumen}
\addcontentsline{toc}{chapter}{Resumen}
\input{capitulos/resumen/main.tex}

% Índice (paginado)
\clearpage
\pagestyle{simple}
% \newpage
\tableofcontents

% Introducción (donde se incluya los antecedentes y justificación)
\clearpage
\pagestyle{myfancy}
\newpage
\chapter{Introducción}
\input{capitulos/introduccion/main.tex}

% Objetivos
\chapter{Objetivos}
\input{capitulos/objetivos/main.tex}

% Metodología
\chapter{Metodología}
\input{capitulos/metodologia/main.tex}

% Resultados y discusión (incluyendo la valoración de impactos y de aspectos de responsabilidad legal, ética y profesional relacionados con el trabajo)
\chapter{Resultados y Discusión}
\input{capitulos/resultados_discusion/main.tex}

% Conclusiones
\chapter{Conclusiones}
\input{capitulos/conclusiones/main.tex}

% Planificación temporal y presupuesto
\chapter{Planificación Temporal y Presupuesto}
\input{capitulos/planificacion_presupuesto/main.tex}

% Bibliografía
\newpage
\addcontentsline{toc}{chapter}{Bibliografía}
\printbibliography

\end{document}


% Bibliografía
\newpage
\addcontentsline{toc}{chapter}{Bibliografía}
\printbibliography

\end{document}


% Metodología
\chapter{Metodología}
\documentclass[a4paper,11pt,twoside]{report}
\usepackage[left=25mm,right=25mm,top=25mm,bottom=25mm,includehead,includefoot,headsep=15mm,footskip=15mm]{geometry}
\usepackage{graphicx}
\usepackage{fancyhdr}
\usepackage{titlesec}
\usepackage[spanish]{babel}
\usepackage[utf8]{inputenc}
\usepackage{amsmath}
\usepackage{setspace}
\usepackage{svg}
\usepackage{hyperref}
\usepackage[backend=biber,style=numeric]{biblatex}
\addbibresource{references.bib}
\hypersetup{
    colorlinks=true,
    linkcolor=blue,      % color of internal links (sections, etc.)
    urlcolor=blue,       % color of external links
    pdftitle={Optimización energética de sistema híbrido con bomba de calor, suelo radiante, fotovoltaica y almacenamiento para vivienda},    % title
    pdfauthor={Luis D. Aranda Sánchez},     % author
    pdfkeywords={palabra1, palabra2, código1, etc.} % list of keywords
}

% Font change to Arial
\usepackage{helvet}
\renewcommand{\familydefault}{\sfdefault}

% Chapter titles in uppercase and larger font
\titleformat{\chapter}[hang]{\large\bfseries}{\thechapter.}{1em}{\MakeUppercase}
\titleformat{\section}[hang]{\bfseries}{\thesection.}{1em}{}
\titleformat{\subsection}[hang]{\bfseries}{\thesubsection.}{1em}{}

% Fancyhdr setup
\setlength{\headheight}{14.30174pt} % Adjust to recommended value, slightly larger for safety
\fancyhf{} % Clear all headers and footers
\fancyhead[LE]{\nouppercase{\leftmark}}
\fancyhead[RO]{Optimización energética para vivienda}
\fancyfoot[LE]{\thepage}
\fancyfoot[RE]{Escuela Técnica Superior de Ingenieros Industriales (UPM)}
\fancyfoot[LO]{Luis D. Aranda Sánchez}
\fancyfoot[RO]{\thepage}
\renewcommand{\headrulewidth}{0.4pt}
\renewcommand{\footrulewidth}{0.4pt}

\fancypagestyle{myfancy}{
    \fancyhf{} % Clear all headers and footers
    \fancyhead[LE]{\nouppercase{\leftmark}}
    \fancyhead[RO]{Optimización energética para vivienda}
    \fancyfoot[LE]{\thepage}
    \fancyfoot[RE]{Escuela Técnica Superior de Ingenieros Industriales (UPM)}
    \fancyfoot[LO]{Luis D. Aranda Sánchez}
    \fancyfoot[RO]{\thepage}
    \renewcommand{\headrulewidth}{0.4pt}
    \renewcommand{\footrulewidth}{0.4pt}
}

\fancypagestyle{simple}{
    \fancyhf{} % Clear all headers and footers
    \renewcommand{\headrulewidth}{0pt}
    \renewcommand{\footrulewidth}{0pt}
}

% Line spacing
\setstretch{1.2}

% Document starts here
\begin{document}

% Portada
\begin{titlepage}
    \centering
    {\scshape\LARGE Universidad Politécnica de Madrid \par}
    \vspace{1cm}
    {\scshape\Large Escuela Técnica Superior de Ingenieros Industriales\par}
    \vspace{1.5cm}
    {\huge\bfseries Optimización energética de sistema híbrido con bomba de calor, suelo radiante, fotovoltaica y almacenamiento para vivienda \par}
    \vspace{1.5cm}
    {\Large\bfseries Trabajo de Fin de Máster\par}
    \vspace{0.5cm}
    {\large Máster Universitario en Ingeniería de la Energía \par}
    \vspace{2cm}
    {\Large Luis D. Aranda Sánchez\par}
    \vfill
    Director: Javier Rodríguez Martín
    \vfill
    {\large Septiembre 6, 2024\par}
\end{titlepage}

% Resumen (máximo de 5 páginas, incluyendo al final Palabras clave)
\clearpage
\pagestyle{simple}
% \newpage
\chapter*{Resumen}
\addcontentsline{toc}{chapter}{Resumen}
\documentclass[a4paper,11pt,twoside]{report}
\usepackage[left=25mm,right=25mm,top=25mm,bottom=25mm,includehead,includefoot,headsep=15mm,footskip=15mm]{geometry}
\usepackage{graphicx}
\usepackage{fancyhdr}
\usepackage{titlesec}
\usepackage[spanish]{babel}
\usepackage[utf8]{inputenc}
\usepackage{amsmath}
\usepackage{setspace}
\usepackage{svg}
\usepackage{hyperref}
\usepackage[backend=biber,style=numeric]{biblatex}
\addbibresource{references.bib}
\hypersetup{
    colorlinks=true,
    linkcolor=blue,      % color of internal links (sections, etc.)
    urlcolor=blue,       % color of external links
    pdftitle={Optimización energética de sistema híbrido con bomba de calor, suelo radiante, fotovoltaica y almacenamiento para vivienda},    % title
    pdfauthor={Luis D. Aranda Sánchez},     % author
    pdfkeywords={palabra1, palabra2, código1, etc.} % list of keywords
}

% Font change to Arial
\usepackage{helvet}
\renewcommand{\familydefault}{\sfdefault}

% Chapter titles in uppercase and larger font
\titleformat{\chapter}[hang]{\large\bfseries}{\thechapter.}{1em}{\MakeUppercase}
\titleformat{\section}[hang]{\bfseries}{\thesection.}{1em}{}
\titleformat{\subsection}[hang]{\bfseries}{\thesubsection.}{1em}{}

% Fancyhdr setup
\setlength{\headheight}{14.30174pt} % Adjust to recommended value, slightly larger for safety
\fancyhf{} % Clear all headers and footers
\fancyhead[LE]{\nouppercase{\leftmark}}
\fancyhead[RO]{Optimización energética para vivienda}
\fancyfoot[LE]{\thepage}
\fancyfoot[RE]{Escuela Técnica Superior de Ingenieros Industriales (UPM)}
\fancyfoot[LO]{Luis D. Aranda Sánchez}
\fancyfoot[RO]{\thepage}
\renewcommand{\headrulewidth}{0.4pt}
\renewcommand{\footrulewidth}{0.4pt}

\fancypagestyle{myfancy}{
    \fancyhf{} % Clear all headers and footers
    \fancyhead[LE]{\nouppercase{\leftmark}}
    \fancyhead[RO]{Optimización energética para vivienda}
    \fancyfoot[LE]{\thepage}
    \fancyfoot[RE]{Escuela Técnica Superior de Ingenieros Industriales (UPM)}
    \fancyfoot[LO]{Luis D. Aranda Sánchez}
    \fancyfoot[RO]{\thepage}
    \renewcommand{\headrulewidth}{0.4pt}
    \renewcommand{\footrulewidth}{0.4pt}
}

\fancypagestyle{simple}{
    \fancyhf{} % Clear all headers and footers
    \renewcommand{\headrulewidth}{0pt}
    \renewcommand{\footrulewidth}{0pt}
}

% Line spacing
\setstretch{1.2}

% Document starts here
\begin{document}

% Portada
\begin{titlepage}
    \centering
    {\scshape\LARGE Universidad Politécnica de Madrid \par}
    \vspace{1cm}
    {\scshape\Large Escuela Técnica Superior de Ingenieros Industriales\par}
    \vspace{1.5cm}
    {\huge\bfseries Optimización energética de sistema híbrido con bomba de calor, suelo radiante, fotovoltaica y almacenamiento para vivienda \par}
    \vspace{1.5cm}
    {\Large\bfseries Trabajo de Fin de Máster\par}
    \vspace{0.5cm}
    {\large Máster Universitario en Ingeniería de la Energía \par}
    \vspace{2cm}
    {\Large Luis D. Aranda Sánchez\par}
    \vfill
    Director: Javier Rodríguez Martín
    \vfill
    {\large Septiembre 6, 2024\par}
\end{titlepage}

% Resumen (máximo de 5 páginas, incluyendo al final Palabras clave)
\clearpage
\pagestyle{simple}
% \newpage
\chapter*{Resumen}
\addcontentsline{toc}{chapter}{Resumen}
\input{capitulos/resumen/main.tex}

% Índice (paginado)
\clearpage
\pagestyle{simple}
% \newpage
\tableofcontents

% Introducción (donde se incluya los antecedentes y justificación)
\clearpage
\pagestyle{myfancy}
\newpage
\chapter{Introducción}
\input{capitulos/introduccion/main.tex}

% Objetivos
\chapter{Objetivos}
\input{capitulos/objetivos/main.tex}

% Metodología
\chapter{Metodología}
\input{capitulos/metodologia/main.tex}

% Resultados y discusión (incluyendo la valoración de impactos y de aspectos de responsabilidad legal, ética y profesional relacionados con el trabajo)
\chapter{Resultados y Discusión}
\input{capitulos/resultados_discusion/main.tex}

% Conclusiones
\chapter{Conclusiones}
\input{capitulos/conclusiones/main.tex}

% Planificación temporal y presupuesto
\chapter{Planificación Temporal y Presupuesto}
\input{capitulos/planificacion_presupuesto/main.tex}

% Bibliografía
\newpage
\addcontentsline{toc}{chapter}{Bibliografía}
\printbibliography

\end{document}


% Índice (paginado)
\clearpage
\pagestyle{simple}
% \newpage
\tableofcontents

% Introducción (donde se incluya los antecedentes y justificación)
\clearpage
\pagestyle{myfancy}
\newpage
\chapter{Introducción}
\documentclass[a4paper,11pt,twoside]{report}
\usepackage[left=25mm,right=25mm,top=25mm,bottom=25mm,includehead,includefoot,headsep=15mm,footskip=15mm]{geometry}
\usepackage{graphicx}
\usepackage{fancyhdr}
\usepackage{titlesec}
\usepackage[spanish]{babel}
\usepackage[utf8]{inputenc}
\usepackage{amsmath}
\usepackage{setspace}
\usepackage{svg}
\usepackage{hyperref}
\usepackage[backend=biber,style=numeric]{biblatex}
\addbibresource{references.bib}
\hypersetup{
    colorlinks=true,
    linkcolor=blue,      % color of internal links (sections, etc.)
    urlcolor=blue,       % color of external links
    pdftitle={Optimización energética de sistema híbrido con bomba de calor, suelo radiante, fotovoltaica y almacenamiento para vivienda},    % title
    pdfauthor={Luis D. Aranda Sánchez},     % author
    pdfkeywords={palabra1, palabra2, código1, etc.} % list of keywords
}

% Font change to Arial
\usepackage{helvet}
\renewcommand{\familydefault}{\sfdefault}

% Chapter titles in uppercase and larger font
\titleformat{\chapter}[hang]{\large\bfseries}{\thechapter.}{1em}{\MakeUppercase}
\titleformat{\section}[hang]{\bfseries}{\thesection.}{1em}{}
\titleformat{\subsection}[hang]{\bfseries}{\thesubsection.}{1em}{}

% Fancyhdr setup
\setlength{\headheight}{14.30174pt} % Adjust to recommended value, slightly larger for safety
\fancyhf{} % Clear all headers and footers
\fancyhead[LE]{\nouppercase{\leftmark}}
\fancyhead[RO]{Optimización energética para vivienda}
\fancyfoot[LE]{\thepage}
\fancyfoot[RE]{Escuela Técnica Superior de Ingenieros Industriales (UPM)}
\fancyfoot[LO]{Luis D. Aranda Sánchez}
\fancyfoot[RO]{\thepage}
\renewcommand{\headrulewidth}{0.4pt}
\renewcommand{\footrulewidth}{0.4pt}

\fancypagestyle{myfancy}{
    \fancyhf{} % Clear all headers and footers
    \fancyhead[LE]{\nouppercase{\leftmark}}
    \fancyhead[RO]{Optimización energética para vivienda}
    \fancyfoot[LE]{\thepage}
    \fancyfoot[RE]{Escuela Técnica Superior de Ingenieros Industriales (UPM)}
    \fancyfoot[LO]{Luis D. Aranda Sánchez}
    \fancyfoot[RO]{\thepage}
    \renewcommand{\headrulewidth}{0.4pt}
    \renewcommand{\footrulewidth}{0.4pt}
}

\fancypagestyle{simple}{
    \fancyhf{} % Clear all headers and footers
    \renewcommand{\headrulewidth}{0pt}
    \renewcommand{\footrulewidth}{0pt}
}

% Line spacing
\setstretch{1.2}

% Document starts here
\begin{document}

% Portada
\begin{titlepage}
    \centering
    {\scshape\LARGE Universidad Politécnica de Madrid \par}
    \vspace{1cm}
    {\scshape\Large Escuela Técnica Superior de Ingenieros Industriales\par}
    \vspace{1.5cm}
    {\huge\bfseries Optimización energética de sistema híbrido con bomba de calor, suelo radiante, fotovoltaica y almacenamiento para vivienda \par}
    \vspace{1.5cm}
    {\Large\bfseries Trabajo de Fin de Máster\par}
    \vspace{0.5cm}
    {\large Máster Universitario en Ingeniería de la Energía \par}
    \vspace{2cm}
    {\Large Luis D. Aranda Sánchez\par}
    \vfill
    Director: Javier Rodríguez Martín
    \vfill
    {\large Septiembre 6, 2024\par}
\end{titlepage}

% Resumen (máximo de 5 páginas, incluyendo al final Palabras clave)
\clearpage
\pagestyle{simple}
% \newpage
\chapter*{Resumen}
\addcontentsline{toc}{chapter}{Resumen}
\input{capitulos/resumen/main.tex}

% Índice (paginado)
\clearpage
\pagestyle{simple}
% \newpage
\tableofcontents

% Introducción (donde se incluya los antecedentes y justificación)
\clearpage
\pagestyle{myfancy}
\newpage
\chapter{Introducción}
\input{capitulos/introduccion/main.tex}

% Objetivos
\chapter{Objetivos}
\input{capitulos/objetivos/main.tex}

% Metodología
\chapter{Metodología}
\input{capitulos/metodologia/main.tex}

% Resultados y discusión (incluyendo la valoración de impactos y de aspectos de responsabilidad legal, ética y profesional relacionados con el trabajo)
\chapter{Resultados y Discusión}
\input{capitulos/resultados_discusion/main.tex}

% Conclusiones
\chapter{Conclusiones}
\input{capitulos/conclusiones/main.tex}

% Planificación temporal y presupuesto
\chapter{Planificación Temporal y Presupuesto}
\input{capitulos/planificacion_presupuesto/main.tex}

% Bibliografía
\newpage
\addcontentsline{toc}{chapter}{Bibliografía}
\printbibliography

\end{document}


% Objetivos
\chapter{Objetivos}
\documentclass[a4paper,11pt,twoside]{report}
\usepackage[left=25mm,right=25mm,top=25mm,bottom=25mm,includehead,includefoot,headsep=15mm,footskip=15mm]{geometry}
\usepackage{graphicx}
\usepackage{fancyhdr}
\usepackage{titlesec}
\usepackage[spanish]{babel}
\usepackage[utf8]{inputenc}
\usepackage{amsmath}
\usepackage{setspace}
\usepackage{svg}
\usepackage{hyperref}
\usepackage[backend=biber,style=numeric]{biblatex}
\addbibresource{references.bib}
\hypersetup{
    colorlinks=true,
    linkcolor=blue,      % color of internal links (sections, etc.)
    urlcolor=blue,       % color of external links
    pdftitle={Optimización energética de sistema híbrido con bomba de calor, suelo radiante, fotovoltaica y almacenamiento para vivienda},    % title
    pdfauthor={Luis D. Aranda Sánchez},     % author
    pdfkeywords={palabra1, palabra2, código1, etc.} % list of keywords
}

% Font change to Arial
\usepackage{helvet}
\renewcommand{\familydefault}{\sfdefault}

% Chapter titles in uppercase and larger font
\titleformat{\chapter}[hang]{\large\bfseries}{\thechapter.}{1em}{\MakeUppercase}
\titleformat{\section}[hang]{\bfseries}{\thesection.}{1em}{}
\titleformat{\subsection}[hang]{\bfseries}{\thesubsection.}{1em}{}

% Fancyhdr setup
\setlength{\headheight}{14.30174pt} % Adjust to recommended value, slightly larger for safety
\fancyhf{} % Clear all headers and footers
\fancyhead[LE]{\nouppercase{\leftmark}}
\fancyhead[RO]{Optimización energética para vivienda}
\fancyfoot[LE]{\thepage}
\fancyfoot[RE]{Escuela Técnica Superior de Ingenieros Industriales (UPM)}
\fancyfoot[LO]{Luis D. Aranda Sánchez}
\fancyfoot[RO]{\thepage}
\renewcommand{\headrulewidth}{0.4pt}
\renewcommand{\footrulewidth}{0.4pt}

\fancypagestyle{myfancy}{
    \fancyhf{} % Clear all headers and footers
    \fancyhead[LE]{\nouppercase{\leftmark}}
    \fancyhead[RO]{Optimización energética para vivienda}
    \fancyfoot[LE]{\thepage}
    \fancyfoot[RE]{Escuela Técnica Superior de Ingenieros Industriales (UPM)}
    \fancyfoot[LO]{Luis D. Aranda Sánchez}
    \fancyfoot[RO]{\thepage}
    \renewcommand{\headrulewidth}{0.4pt}
    \renewcommand{\footrulewidth}{0.4pt}
}

\fancypagestyle{simple}{
    \fancyhf{} % Clear all headers and footers
    \renewcommand{\headrulewidth}{0pt}
    \renewcommand{\footrulewidth}{0pt}
}

% Line spacing
\setstretch{1.2}

% Document starts here
\begin{document}

% Portada
\begin{titlepage}
    \centering
    {\scshape\LARGE Universidad Politécnica de Madrid \par}
    \vspace{1cm}
    {\scshape\Large Escuela Técnica Superior de Ingenieros Industriales\par}
    \vspace{1.5cm}
    {\huge\bfseries Optimización energética de sistema híbrido con bomba de calor, suelo radiante, fotovoltaica y almacenamiento para vivienda \par}
    \vspace{1.5cm}
    {\Large\bfseries Trabajo de Fin de Máster\par}
    \vspace{0.5cm}
    {\large Máster Universitario en Ingeniería de la Energía \par}
    \vspace{2cm}
    {\Large Luis D. Aranda Sánchez\par}
    \vfill
    Director: Javier Rodríguez Martín
    \vfill
    {\large Septiembre 6, 2024\par}
\end{titlepage}

% Resumen (máximo de 5 páginas, incluyendo al final Palabras clave)
\clearpage
\pagestyle{simple}
% \newpage
\chapter*{Resumen}
\addcontentsline{toc}{chapter}{Resumen}
\input{capitulos/resumen/main.tex}

% Índice (paginado)
\clearpage
\pagestyle{simple}
% \newpage
\tableofcontents

% Introducción (donde se incluya los antecedentes y justificación)
\clearpage
\pagestyle{myfancy}
\newpage
\chapter{Introducción}
\input{capitulos/introduccion/main.tex}

% Objetivos
\chapter{Objetivos}
\input{capitulos/objetivos/main.tex}

% Metodología
\chapter{Metodología}
\input{capitulos/metodologia/main.tex}

% Resultados y discusión (incluyendo la valoración de impactos y de aspectos de responsabilidad legal, ética y profesional relacionados con el trabajo)
\chapter{Resultados y Discusión}
\input{capitulos/resultados_discusion/main.tex}

% Conclusiones
\chapter{Conclusiones}
\input{capitulos/conclusiones/main.tex}

% Planificación temporal y presupuesto
\chapter{Planificación Temporal y Presupuesto}
\input{capitulos/planificacion_presupuesto/main.tex}

% Bibliografía
\newpage
\addcontentsline{toc}{chapter}{Bibliografía}
\printbibliography

\end{document}


% Metodología
\chapter{Metodología}
\documentclass[a4paper,11pt,twoside]{report}
\usepackage[left=25mm,right=25mm,top=25mm,bottom=25mm,includehead,includefoot,headsep=15mm,footskip=15mm]{geometry}
\usepackage{graphicx}
\usepackage{fancyhdr}
\usepackage{titlesec}
\usepackage[spanish]{babel}
\usepackage[utf8]{inputenc}
\usepackage{amsmath}
\usepackage{setspace}
\usepackage{svg}
\usepackage{hyperref}
\usepackage[backend=biber,style=numeric]{biblatex}
\addbibresource{references.bib}
\hypersetup{
    colorlinks=true,
    linkcolor=blue,      % color of internal links (sections, etc.)
    urlcolor=blue,       % color of external links
    pdftitle={Optimización energética de sistema híbrido con bomba de calor, suelo radiante, fotovoltaica y almacenamiento para vivienda},    % title
    pdfauthor={Luis D. Aranda Sánchez},     % author
    pdfkeywords={palabra1, palabra2, código1, etc.} % list of keywords
}

% Font change to Arial
\usepackage{helvet}
\renewcommand{\familydefault}{\sfdefault}

% Chapter titles in uppercase and larger font
\titleformat{\chapter}[hang]{\large\bfseries}{\thechapter.}{1em}{\MakeUppercase}
\titleformat{\section}[hang]{\bfseries}{\thesection.}{1em}{}
\titleformat{\subsection}[hang]{\bfseries}{\thesubsection.}{1em}{}

% Fancyhdr setup
\setlength{\headheight}{14.30174pt} % Adjust to recommended value, slightly larger for safety
\fancyhf{} % Clear all headers and footers
\fancyhead[LE]{\nouppercase{\leftmark}}
\fancyhead[RO]{Optimización energética para vivienda}
\fancyfoot[LE]{\thepage}
\fancyfoot[RE]{Escuela Técnica Superior de Ingenieros Industriales (UPM)}
\fancyfoot[LO]{Luis D. Aranda Sánchez}
\fancyfoot[RO]{\thepage}
\renewcommand{\headrulewidth}{0.4pt}
\renewcommand{\footrulewidth}{0.4pt}

\fancypagestyle{myfancy}{
    \fancyhf{} % Clear all headers and footers
    \fancyhead[LE]{\nouppercase{\leftmark}}
    \fancyhead[RO]{Optimización energética para vivienda}
    \fancyfoot[LE]{\thepage}
    \fancyfoot[RE]{Escuela Técnica Superior de Ingenieros Industriales (UPM)}
    \fancyfoot[LO]{Luis D. Aranda Sánchez}
    \fancyfoot[RO]{\thepage}
    \renewcommand{\headrulewidth}{0.4pt}
    \renewcommand{\footrulewidth}{0.4pt}
}

\fancypagestyle{simple}{
    \fancyhf{} % Clear all headers and footers
    \renewcommand{\headrulewidth}{0pt}
    \renewcommand{\footrulewidth}{0pt}
}

% Line spacing
\setstretch{1.2}

% Document starts here
\begin{document}

% Portada
\begin{titlepage}
    \centering
    {\scshape\LARGE Universidad Politécnica de Madrid \par}
    \vspace{1cm}
    {\scshape\Large Escuela Técnica Superior de Ingenieros Industriales\par}
    \vspace{1.5cm}
    {\huge\bfseries Optimización energética de sistema híbrido con bomba de calor, suelo radiante, fotovoltaica y almacenamiento para vivienda \par}
    \vspace{1.5cm}
    {\Large\bfseries Trabajo de Fin de Máster\par}
    \vspace{0.5cm}
    {\large Máster Universitario en Ingeniería de la Energía \par}
    \vspace{2cm}
    {\Large Luis D. Aranda Sánchez\par}
    \vfill
    Director: Javier Rodríguez Martín
    \vfill
    {\large Septiembre 6, 2024\par}
\end{titlepage}

% Resumen (máximo de 5 páginas, incluyendo al final Palabras clave)
\clearpage
\pagestyle{simple}
% \newpage
\chapter*{Resumen}
\addcontentsline{toc}{chapter}{Resumen}
\input{capitulos/resumen/main.tex}

% Índice (paginado)
\clearpage
\pagestyle{simple}
% \newpage
\tableofcontents

% Introducción (donde se incluya los antecedentes y justificación)
\clearpage
\pagestyle{myfancy}
\newpage
\chapter{Introducción}
\input{capitulos/introduccion/main.tex}

% Objetivos
\chapter{Objetivos}
\input{capitulos/objetivos/main.tex}

% Metodología
\chapter{Metodología}
\input{capitulos/metodologia/main.tex}

% Resultados y discusión (incluyendo la valoración de impactos y de aspectos de responsabilidad legal, ética y profesional relacionados con el trabajo)
\chapter{Resultados y Discusión}
\input{capitulos/resultados_discusion/main.tex}

% Conclusiones
\chapter{Conclusiones}
\input{capitulos/conclusiones/main.tex}

% Planificación temporal y presupuesto
\chapter{Planificación Temporal y Presupuesto}
\input{capitulos/planificacion_presupuesto/main.tex}

% Bibliografía
\newpage
\addcontentsline{toc}{chapter}{Bibliografía}
\printbibliography

\end{document}


% Resultados y discusión (incluyendo la valoración de impactos y de aspectos de responsabilidad legal, ética y profesional relacionados con el trabajo)
\chapter{Resultados y Discusión}
\documentclass[a4paper,11pt,twoside]{report}
\usepackage[left=25mm,right=25mm,top=25mm,bottom=25mm,includehead,includefoot,headsep=15mm,footskip=15mm]{geometry}
\usepackage{graphicx}
\usepackage{fancyhdr}
\usepackage{titlesec}
\usepackage[spanish]{babel}
\usepackage[utf8]{inputenc}
\usepackage{amsmath}
\usepackage{setspace}
\usepackage{svg}
\usepackage{hyperref}
\usepackage[backend=biber,style=numeric]{biblatex}
\addbibresource{references.bib}
\hypersetup{
    colorlinks=true,
    linkcolor=blue,      % color of internal links (sections, etc.)
    urlcolor=blue,       % color of external links
    pdftitle={Optimización energética de sistema híbrido con bomba de calor, suelo radiante, fotovoltaica y almacenamiento para vivienda},    % title
    pdfauthor={Luis D. Aranda Sánchez},     % author
    pdfkeywords={palabra1, palabra2, código1, etc.} % list of keywords
}

% Font change to Arial
\usepackage{helvet}
\renewcommand{\familydefault}{\sfdefault}

% Chapter titles in uppercase and larger font
\titleformat{\chapter}[hang]{\large\bfseries}{\thechapter.}{1em}{\MakeUppercase}
\titleformat{\section}[hang]{\bfseries}{\thesection.}{1em}{}
\titleformat{\subsection}[hang]{\bfseries}{\thesubsection.}{1em}{}

% Fancyhdr setup
\setlength{\headheight}{14.30174pt} % Adjust to recommended value, slightly larger for safety
\fancyhf{} % Clear all headers and footers
\fancyhead[LE]{\nouppercase{\leftmark}}
\fancyhead[RO]{Optimización energética para vivienda}
\fancyfoot[LE]{\thepage}
\fancyfoot[RE]{Escuela Técnica Superior de Ingenieros Industriales (UPM)}
\fancyfoot[LO]{Luis D. Aranda Sánchez}
\fancyfoot[RO]{\thepage}
\renewcommand{\headrulewidth}{0.4pt}
\renewcommand{\footrulewidth}{0.4pt}

\fancypagestyle{myfancy}{
    \fancyhf{} % Clear all headers and footers
    \fancyhead[LE]{\nouppercase{\leftmark}}
    \fancyhead[RO]{Optimización energética para vivienda}
    \fancyfoot[LE]{\thepage}
    \fancyfoot[RE]{Escuela Técnica Superior de Ingenieros Industriales (UPM)}
    \fancyfoot[LO]{Luis D. Aranda Sánchez}
    \fancyfoot[RO]{\thepage}
    \renewcommand{\headrulewidth}{0.4pt}
    \renewcommand{\footrulewidth}{0.4pt}
}

\fancypagestyle{simple}{
    \fancyhf{} % Clear all headers and footers
    \renewcommand{\headrulewidth}{0pt}
    \renewcommand{\footrulewidth}{0pt}
}

% Line spacing
\setstretch{1.2}

% Document starts here
\begin{document}

% Portada
\begin{titlepage}
    \centering
    {\scshape\LARGE Universidad Politécnica de Madrid \par}
    \vspace{1cm}
    {\scshape\Large Escuela Técnica Superior de Ingenieros Industriales\par}
    \vspace{1.5cm}
    {\huge\bfseries Optimización energética de sistema híbrido con bomba de calor, suelo radiante, fotovoltaica y almacenamiento para vivienda \par}
    \vspace{1.5cm}
    {\Large\bfseries Trabajo de Fin de Máster\par}
    \vspace{0.5cm}
    {\large Máster Universitario en Ingeniería de la Energía \par}
    \vspace{2cm}
    {\Large Luis D. Aranda Sánchez\par}
    \vfill
    Director: Javier Rodríguez Martín
    \vfill
    {\large Septiembre 6, 2024\par}
\end{titlepage}

% Resumen (máximo de 5 páginas, incluyendo al final Palabras clave)
\clearpage
\pagestyle{simple}
% \newpage
\chapter*{Resumen}
\addcontentsline{toc}{chapter}{Resumen}
\input{capitulos/resumen/main.tex}

% Índice (paginado)
\clearpage
\pagestyle{simple}
% \newpage
\tableofcontents

% Introducción (donde se incluya los antecedentes y justificación)
\clearpage
\pagestyle{myfancy}
\newpage
\chapter{Introducción}
\input{capitulos/introduccion/main.tex}

% Objetivos
\chapter{Objetivos}
\input{capitulos/objetivos/main.tex}

% Metodología
\chapter{Metodología}
\input{capitulos/metodologia/main.tex}

% Resultados y discusión (incluyendo la valoración de impactos y de aspectos de responsabilidad legal, ética y profesional relacionados con el trabajo)
\chapter{Resultados y Discusión}
\input{capitulos/resultados_discusion/main.tex}

% Conclusiones
\chapter{Conclusiones}
\input{capitulos/conclusiones/main.tex}

% Planificación temporal y presupuesto
\chapter{Planificación Temporal y Presupuesto}
\input{capitulos/planificacion_presupuesto/main.tex}

% Bibliografía
\newpage
\addcontentsline{toc}{chapter}{Bibliografía}
\printbibliography

\end{document}


% Conclusiones
\chapter{Conclusiones}
\documentclass[a4paper,11pt,twoside]{report}
\usepackage[left=25mm,right=25mm,top=25mm,bottom=25mm,includehead,includefoot,headsep=15mm,footskip=15mm]{geometry}
\usepackage{graphicx}
\usepackage{fancyhdr}
\usepackage{titlesec}
\usepackage[spanish]{babel}
\usepackage[utf8]{inputenc}
\usepackage{amsmath}
\usepackage{setspace}
\usepackage{svg}
\usepackage{hyperref}
\usepackage[backend=biber,style=numeric]{biblatex}
\addbibresource{references.bib}
\hypersetup{
    colorlinks=true,
    linkcolor=blue,      % color of internal links (sections, etc.)
    urlcolor=blue,       % color of external links
    pdftitle={Optimización energética de sistema híbrido con bomba de calor, suelo radiante, fotovoltaica y almacenamiento para vivienda},    % title
    pdfauthor={Luis D. Aranda Sánchez},     % author
    pdfkeywords={palabra1, palabra2, código1, etc.} % list of keywords
}

% Font change to Arial
\usepackage{helvet}
\renewcommand{\familydefault}{\sfdefault}

% Chapter titles in uppercase and larger font
\titleformat{\chapter}[hang]{\large\bfseries}{\thechapter.}{1em}{\MakeUppercase}
\titleformat{\section}[hang]{\bfseries}{\thesection.}{1em}{}
\titleformat{\subsection}[hang]{\bfseries}{\thesubsection.}{1em}{}

% Fancyhdr setup
\setlength{\headheight}{14.30174pt} % Adjust to recommended value, slightly larger for safety
\fancyhf{} % Clear all headers and footers
\fancyhead[LE]{\nouppercase{\leftmark}}
\fancyhead[RO]{Optimización energética para vivienda}
\fancyfoot[LE]{\thepage}
\fancyfoot[RE]{Escuela Técnica Superior de Ingenieros Industriales (UPM)}
\fancyfoot[LO]{Luis D. Aranda Sánchez}
\fancyfoot[RO]{\thepage}
\renewcommand{\headrulewidth}{0.4pt}
\renewcommand{\footrulewidth}{0.4pt}

\fancypagestyle{myfancy}{
    \fancyhf{} % Clear all headers and footers
    \fancyhead[LE]{\nouppercase{\leftmark}}
    \fancyhead[RO]{Optimización energética para vivienda}
    \fancyfoot[LE]{\thepage}
    \fancyfoot[RE]{Escuela Técnica Superior de Ingenieros Industriales (UPM)}
    \fancyfoot[LO]{Luis D. Aranda Sánchez}
    \fancyfoot[RO]{\thepage}
    \renewcommand{\headrulewidth}{0.4pt}
    \renewcommand{\footrulewidth}{0.4pt}
}

\fancypagestyle{simple}{
    \fancyhf{} % Clear all headers and footers
    \renewcommand{\headrulewidth}{0pt}
    \renewcommand{\footrulewidth}{0pt}
}

% Line spacing
\setstretch{1.2}

% Document starts here
\begin{document}

% Portada
\begin{titlepage}
    \centering
    {\scshape\LARGE Universidad Politécnica de Madrid \par}
    \vspace{1cm}
    {\scshape\Large Escuela Técnica Superior de Ingenieros Industriales\par}
    \vspace{1.5cm}
    {\huge\bfseries Optimización energética de sistema híbrido con bomba de calor, suelo radiante, fotovoltaica y almacenamiento para vivienda \par}
    \vspace{1.5cm}
    {\Large\bfseries Trabajo de Fin de Máster\par}
    \vspace{0.5cm}
    {\large Máster Universitario en Ingeniería de la Energía \par}
    \vspace{2cm}
    {\Large Luis D. Aranda Sánchez\par}
    \vfill
    Director: Javier Rodríguez Martín
    \vfill
    {\large Septiembre 6, 2024\par}
\end{titlepage}

% Resumen (máximo de 5 páginas, incluyendo al final Palabras clave)
\clearpage
\pagestyle{simple}
% \newpage
\chapter*{Resumen}
\addcontentsline{toc}{chapter}{Resumen}
\input{capitulos/resumen/main.tex}

% Índice (paginado)
\clearpage
\pagestyle{simple}
% \newpage
\tableofcontents

% Introducción (donde se incluya los antecedentes y justificación)
\clearpage
\pagestyle{myfancy}
\newpage
\chapter{Introducción}
\input{capitulos/introduccion/main.tex}

% Objetivos
\chapter{Objetivos}
\input{capitulos/objetivos/main.tex}

% Metodología
\chapter{Metodología}
\input{capitulos/metodologia/main.tex}

% Resultados y discusión (incluyendo la valoración de impactos y de aspectos de responsabilidad legal, ética y profesional relacionados con el trabajo)
\chapter{Resultados y Discusión}
\input{capitulos/resultados_discusion/main.tex}

% Conclusiones
\chapter{Conclusiones}
\input{capitulos/conclusiones/main.tex}

% Planificación temporal y presupuesto
\chapter{Planificación Temporal y Presupuesto}
\input{capitulos/planificacion_presupuesto/main.tex}

% Bibliografía
\newpage
\addcontentsline{toc}{chapter}{Bibliografía}
\printbibliography

\end{document}


% Planificación temporal y presupuesto
\chapter{Planificación Temporal y Presupuesto}
\documentclass[a4paper,11pt,twoside]{report}
\usepackage[left=25mm,right=25mm,top=25mm,bottom=25mm,includehead,includefoot,headsep=15mm,footskip=15mm]{geometry}
\usepackage{graphicx}
\usepackage{fancyhdr}
\usepackage{titlesec}
\usepackage[spanish]{babel}
\usepackage[utf8]{inputenc}
\usepackage{amsmath}
\usepackage{setspace}
\usepackage{svg}
\usepackage{hyperref}
\usepackage[backend=biber,style=numeric]{biblatex}
\addbibresource{references.bib}
\hypersetup{
    colorlinks=true,
    linkcolor=blue,      % color of internal links (sections, etc.)
    urlcolor=blue,       % color of external links
    pdftitle={Optimización energética de sistema híbrido con bomba de calor, suelo radiante, fotovoltaica y almacenamiento para vivienda},    % title
    pdfauthor={Luis D. Aranda Sánchez},     % author
    pdfkeywords={palabra1, palabra2, código1, etc.} % list of keywords
}

% Font change to Arial
\usepackage{helvet}
\renewcommand{\familydefault}{\sfdefault}

% Chapter titles in uppercase and larger font
\titleformat{\chapter}[hang]{\large\bfseries}{\thechapter.}{1em}{\MakeUppercase}
\titleformat{\section}[hang]{\bfseries}{\thesection.}{1em}{}
\titleformat{\subsection}[hang]{\bfseries}{\thesubsection.}{1em}{}

% Fancyhdr setup
\setlength{\headheight}{14.30174pt} % Adjust to recommended value, slightly larger for safety
\fancyhf{} % Clear all headers and footers
\fancyhead[LE]{\nouppercase{\leftmark}}
\fancyhead[RO]{Optimización energética para vivienda}
\fancyfoot[LE]{\thepage}
\fancyfoot[RE]{Escuela Técnica Superior de Ingenieros Industriales (UPM)}
\fancyfoot[LO]{Luis D. Aranda Sánchez}
\fancyfoot[RO]{\thepage}
\renewcommand{\headrulewidth}{0.4pt}
\renewcommand{\footrulewidth}{0.4pt}

\fancypagestyle{myfancy}{
    \fancyhf{} % Clear all headers and footers
    \fancyhead[LE]{\nouppercase{\leftmark}}
    \fancyhead[RO]{Optimización energética para vivienda}
    \fancyfoot[LE]{\thepage}
    \fancyfoot[RE]{Escuela Técnica Superior de Ingenieros Industriales (UPM)}
    \fancyfoot[LO]{Luis D. Aranda Sánchez}
    \fancyfoot[RO]{\thepage}
    \renewcommand{\headrulewidth}{0.4pt}
    \renewcommand{\footrulewidth}{0.4pt}
}

\fancypagestyle{simple}{
    \fancyhf{} % Clear all headers and footers
    \renewcommand{\headrulewidth}{0pt}
    \renewcommand{\footrulewidth}{0pt}
}

% Line spacing
\setstretch{1.2}

% Document starts here
\begin{document}

% Portada
\begin{titlepage}
    \centering
    {\scshape\LARGE Universidad Politécnica de Madrid \par}
    \vspace{1cm}
    {\scshape\Large Escuela Técnica Superior de Ingenieros Industriales\par}
    \vspace{1.5cm}
    {\huge\bfseries Optimización energética de sistema híbrido con bomba de calor, suelo radiante, fotovoltaica y almacenamiento para vivienda \par}
    \vspace{1.5cm}
    {\Large\bfseries Trabajo de Fin de Máster\par}
    \vspace{0.5cm}
    {\large Máster Universitario en Ingeniería de la Energía \par}
    \vspace{2cm}
    {\Large Luis D. Aranda Sánchez\par}
    \vfill
    Director: Javier Rodríguez Martín
    \vfill
    {\large Septiembre 6, 2024\par}
\end{titlepage}

% Resumen (máximo de 5 páginas, incluyendo al final Palabras clave)
\clearpage
\pagestyle{simple}
% \newpage
\chapter*{Resumen}
\addcontentsline{toc}{chapter}{Resumen}
\input{capitulos/resumen/main.tex}

% Índice (paginado)
\clearpage
\pagestyle{simple}
% \newpage
\tableofcontents

% Introducción (donde se incluya los antecedentes y justificación)
\clearpage
\pagestyle{myfancy}
\newpage
\chapter{Introducción}
\input{capitulos/introduccion/main.tex}

% Objetivos
\chapter{Objetivos}
\input{capitulos/objetivos/main.tex}

% Metodología
\chapter{Metodología}
\input{capitulos/metodologia/main.tex}

% Resultados y discusión (incluyendo la valoración de impactos y de aspectos de responsabilidad legal, ética y profesional relacionados con el trabajo)
\chapter{Resultados y Discusión}
\input{capitulos/resultados_discusion/main.tex}

% Conclusiones
\chapter{Conclusiones}
\input{capitulos/conclusiones/main.tex}

% Planificación temporal y presupuesto
\chapter{Planificación Temporal y Presupuesto}
\input{capitulos/planificacion_presupuesto/main.tex}

% Bibliografía
\newpage
\addcontentsline{toc}{chapter}{Bibliografía}
\printbibliography

\end{document}


% Bibliografía
\newpage
\addcontentsline{toc}{chapter}{Bibliografía}
\printbibliography

\end{document}


% Resultados y discusión (incluyendo la valoración de impactos y de aspectos de responsabilidad legal, ética y profesional relacionados con el trabajo)
\chapter{Resultados y Discusión}
\documentclass[a4paper,11pt,twoside]{report}
\usepackage[left=25mm,right=25mm,top=25mm,bottom=25mm,includehead,includefoot,headsep=15mm,footskip=15mm]{geometry}
\usepackage{graphicx}
\usepackage{fancyhdr}
\usepackage{titlesec}
\usepackage[spanish]{babel}
\usepackage[utf8]{inputenc}
\usepackage{amsmath}
\usepackage{setspace}
\usepackage{svg}
\usepackage{hyperref}
\usepackage[backend=biber,style=numeric]{biblatex}
\addbibresource{references.bib}
\hypersetup{
    colorlinks=true,
    linkcolor=blue,      % color of internal links (sections, etc.)
    urlcolor=blue,       % color of external links
    pdftitle={Optimización energética de sistema híbrido con bomba de calor, suelo radiante, fotovoltaica y almacenamiento para vivienda},    % title
    pdfauthor={Luis D. Aranda Sánchez},     % author
    pdfkeywords={palabra1, palabra2, código1, etc.} % list of keywords
}

% Font change to Arial
\usepackage{helvet}
\renewcommand{\familydefault}{\sfdefault}

% Chapter titles in uppercase and larger font
\titleformat{\chapter}[hang]{\large\bfseries}{\thechapter.}{1em}{\MakeUppercase}
\titleformat{\section}[hang]{\bfseries}{\thesection.}{1em}{}
\titleformat{\subsection}[hang]{\bfseries}{\thesubsection.}{1em}{}

% Fancyhdr setup
\setlength{\headheight}{14.30174pt} % Adjust to recommended value, slightly larger for safety
\fancyhf{} % Clear all headers and footers
\fancyhead[LE]{\nouppercase{\leftmark}}
\fancyhead[RO]{Optimización energética para vivienda}
\fancyfoot[LE]{\thepage}
\fancyfoot[RE]{Escuela Técnica Superior de Ingenieros Industriales (UPM)}
\fancyfoot[LO]{Luis D. Aranda Sánchez}
\fancyfoot[RO]{\thepage}
\renewcommand{\headrulewidth}{0.4pt}
\renewcommand{\footrulewidth}{0.4pt}

\fancypagestyle{myfancy}{
    \fancyhf{} % Clear all headers and footers
    \fancyhead[LE]{\nouppercase{\leftmark}}
    \fancyhead[RO]{Optimización energética para vivienda}
    \fancyfoot[LE]{\thepage}
    \fancyfoot[RE]{Escuela Técnica Superior de Ingenieros Industriales (UPM)}
    \fancyfoot[LO]{Luis D. Aranda Sánchez}
    \fancyfoot[RO]{\thepage}
    \renewcommand{\headrulewidth}{0.4pt}
    \renewcommand{\footrulewidth}{0.4pt}
}

\fancypagestyle{simple}{
    \fancyhf{} % Clear all headers and footers
    \renewcommand{\headrulewidth}{0pt}
    \renewcommand{\footrulewidth}{0pt}
}

% Line spacing
\setstretch{1.2}

% Document starts here
\begin{document}

% Portada
\begin{titlepage}
    \centering
    {\scshape\LARGE Universidad Politécnica de Madrid \par}
    \vspace{1cm}
    {\scshape\Large Escuela Técnica Superior de Ingenieros Industriales\par}
    \vspace{1.5cm}
    {\huge\bfseries Optimización energética de sistema híbrido con bomba de calor, suelo radiante, fotovoltaica y almacenamiento para vivienda \par}
    \vspace{1.5cm}
    {\Large\bfseries Trabajo de Fin de Máster\par}
    \vspace{0.5cm}
    {\large Máster Universitario en Ingeniería de la Energía \par}
    \vspace{2cm}
    {\Large Luis D. Aranda Sánchez\par}
    \vfill
    Director: Javier Rodríguez Martín
    \vfill
    {\large Septiembre 6, 2024\par}
\end{titlepage}

% Resumen (máximo de 5 páginas, incluyendo al final Palabras clave)
\clearpage
\pagestyle{simple}
% \newpage
\chapter*{Resumen}
\addcontentsline{toc}{chapter}{Resumen}
\documentclass[a4paper,11pt,twoside]{report}
\usepackage[left=25mm,right=25mm,top=25mm,bottom=25mm,includehead,includefoot,headsep=15mm,footskip=15mm]{geometry}
\usepackage{graphicx}
\usepackage{fancyhdr}
\usepackage{titlesec}
\usepackage[spanish]{babel}
\usepackage[utf8]{inputenc}
\usepackage{amsmath}
\usepackage{setspace}
\usepackage{svg}
\usepackage{hyperref}
\usepackage[backend=biber,style=numeric]{biblatex}
\addbibresource{references.bib}
\hypersetup{
    colorlinks=true,
    linkcolor=blue,      % color of internal links (sections, etc.)
    urlcolor=blue,       % color of external links
    pdftitle={Optimización energética de sistema híbrido con bomba de calor, suelo radiante, fotovoltaica y almacenamiento para vivienda},    % title
    pdfauthor={Luis D. Aranda Sánchez},     % author
    pdfkeywords={palabra1, palabra2, código1, etc.} % list of keywords
}

% Font change to Arial
\usepackage{helvet}
\renewcommand{\familydefault}{\sfdefault}

% Chapter titles in uppercase and larger font
\titleformat{\chapter}[hang]{\large\bfseries}{\thechapter.}{1em}{\MakeUppercase}
\titleformat{\section}[hang]{\bfseries}{\thesection.}{1em}{}
\titleformat{\subsection}[hang]{\bfseries}{\thesubsection.}{1em}{}

% Fancyhdr setup
\setlength{\headheight}{14.30174pt} % Adjust to recommended value, slightly larger for safety
\fancyhf{} % Clear all headers and footers
\fancyhead[LE]{\nouppercase{\leftmark}}
\fancyhead[RO]{Optimización energética para vivienda}
\fancyfoot[LE]{\thepage}
\fancyfoot[RE]{Escuela Técnica Superior de Ingenieros Industriales (UPM)}
\fancyfoot[LO]{Luis D. Aranda Sánchez}
\fancyfoot[RO]{\thepage}
\renewcommand{\headrulewidth}{0.4pt}
\renewcommand{\footrulewidth}{0.4pt}

\fancypagestyle{myfancy}{
    \fancyhf{} % Clear all headers and footers
    \fancyhead[LE]{\nouppercase{\leftmark}}
    \fancyhead[RO]{Optimización energética para vivienda}
    \fancyfoot[LE]{\thepage}
    \fancyfoot[RE]{Escuela Técnica Superior de Ingenieros Industriales (UPM)}
    \fancyfoot[LO]{Luis D. Aranda Sánchez}
    \fancyfoot[RO]{\thepage}
    \renewcommand{\headrulewidth}{0.4pt}
    \renewcommand{\footrulewidth}{0.4pt}
}

\fancypagestyle{simple}{
    \fancyhf{} % Clear all headers and footers
    \renewcommand{\headrulewidth}{0pt}
    \renewcommand{\footrulewidth}{0pt}
}

% Line spacing
\setstretch{1.2}

% Document starts here
\begin{document}

% Portada
\begin{titlepage}
    \centering
    {\scshape\LARGE Universidad Politécnica de Madrid \par}
    \vspace{1cm}
    {\scshape\Large Escuela Técnica Superior de Ingenieros Industriales\par}
    \vspace{1.5cm}
    {\huge\bfseries Optimización energética de sistema híbrido con bomba de calor, suelo radiante, fotovoltaica y almacenamiento para vivienda \par}
    \vspace{1.5cm}
    {\Large\bfseries Trabajo de Fin de Máster\par}
    \vspace{0.5cm}
    {\large Máster Universitario en Ingeniería de la Energía \par}
    \vspace{2cm}
    {\Large Luis D. Aranda Sánchez\par}
    \vfill
    Director: Javier Rodríguez Martín
    \vfill
    {\large Septiembre 6, 2024\par}
\end{titlepage}

% Resumen (máximo de 5 páginas, incluyendo al final Palabras clave)
\clearpage
\pagestyle{simple}
% \newpage
\chapter*{Resumen}
\addcontentsline{toc}{chapter}{Resumen}
\input{capitulos/resumen/main.tex}

% Índice (paginado)
\clearpage
\pagestyle{simple}
% \newpage
\tableofcontents

% Introducción (donde se incluya los antecedentes y justificación)
\clearpage
\pagestyle{myfancy}
\newpage
\chapter{Introducción}
\input{capitulos/introduccion/main.tex}

% Objetivos
\chapter{Objetivos}
\input{capitulos/objetivos/main.tex}

% Metodología
\chapter{Metodología}
\input{capitulos/metodologia/main.tex}

% Resultados y discusión (incluyendo la valoración de impactos y de aspectos de responsabilidad legal, ética y profesional relacionados con el trabajo)
\chapter{Resultados y Discusión}
\input{capitulos/resultados_discusion/main.tex}

% Conclusiones
\chapter{Conclusiones}
\input{capitulos/conclusiones/main.tex}

% Planificación temporal y presupuesto
\chapter{Planificación Temporal y Presupuesto}
\input{capitulos/planificacion_presupuesto/main.tex}

% Bibliografía
\newpage
\addcontentsline{toc}{chapter}{Bibliografía}
\printbibliography

\end{document}


% Índice (paginado)
\clearpage
\pagestyle{simple}
% \newpage
\tableofcontents

% Introducción (donde se incluya los antecedentes y justificación)
\clearpage
\pagestyle{myfancy}
\newpage
\chapter{Introducción}
\documentclass[a4paper,11pt,twoside]{report}
\usepackage[left=25mm,right=25mm,top=25mm,bottom=25mm,includehead,includefoot,headsep=15mm,footskip=15mm]{geometry}
\usepackage{graphicx}
\usepackage{fancyhdr}
\usepackage{titlesec}
\usepackage[spanish]{babel}
\usepackage[utf8]{inputenc}
\usepackage{amsmath}
\usepackage{setspace}
\usepackage{svg}
\usepackage{hyperref}
\usepackage[backend=biber,style=numeric]{biblatex}
\addbibresource{references.bib}
\hypersetup{
    colorlinks=true,
    linkcolor=blue,      % color of internal links (sections, etc.)
    urlcolor=blue,       % color of external links
    pdftitle={Optimización energética de sistema híbrido con bomba de calor, suelo radiante, fotovoltaica y almacenamiento para vivienda},    % title
    pdfauthor={Luis D. Aranda Sánchez},     % author
    pdfkeywords={palabra1, palabra2, código1, etc.} % list of keywords
}

% Font change to Arial
\usepackage{helvet}
\renewcommand{\familydefault}{\sfdefault}

% Chapter titles in uppercase and larger font
\titleformat{\chapter}[hang]{\large\bfseries}{\thechapter.}{1em}{\MakeUppercase}
\titleformat{\section}[hang]{\bfseries}{\thesection.}{1em}{}
\titleformat{\subsection}[hang]{\bfseries}{\thesubsection.}{1em}{}

% Fancyhdr setup
\setlength{\headheight}{14.30174pt} % Adjust to recommended value, slightly larger for safety
\fancyhf{} % Clear all headers and footers
\fancyhead[LE]{\nouppercase{\leftmark}}
\fancyhead[RO]{Optimización energética para vivienda}
\fancyfoot[LE]{\thepage}
\fancyfoot[RE]{Escuela Técnica Superior de Ingenieros Industriales (UPM)}
\fancyfoot[LO]{Luis D. Aranda Sánchez}
\fancyfoot[RO]{\thepage}
\renewcommand{\headrulewidth}{0.4pt}
\renewcommand{\footrulewidth}{0.4pt}

\fancypagestyle{myfancy}{
    \fancyhf{} % Clear all headers and footers
    \fancyhead[LE]{\nouppercase{\leftmark}}
    \fancyhead[RO]{Optimización energética para vivienda}
    \fancyfoot[LE]{\thepage}
    \fancyfoot[RE]{Escuela Técnica Superior de Ingenieros Industriales (UPM)}
    \fancyfoot[LO]{Luis D. Aranda Sánchez}
    \fancyfoot[RO]{\thepage}
    \renewcommand{\headrulewidth}{0.4pt}
    \renewcommand{\footrulewidth}{0.4pt}
}

\fancypagestyle{simple}{
    \fancyhf{} % Clear all headers and footers
    \renewcommand{\headrulewidth}{0pt}
    \renewcommand{\footrulewidth}{0pt}
}

% Line spacing
\setstretch{1.2}

% Document starts here
\begin{document}

% Portada
\begin{titlepage}
    \centering
    {\scshape\LARGE Universidad Politécnica de Madrid \par}
    \vspace{1cm}
    {\scshape\Large Escuela Técnica Superior de Ingenieros Industriales\par}
    \vspace{1.5cm}
    {\huge\bfseries Optimización energética de sistema híbrido con bomba de calor, suelo radiante, fotovoltaica y almacenamiento para vivienda \par}
    \vspace{1.5cm}
    {\Large\bfseries Trabajo de Fin de Máster\par}
    \vspace{0.5cm}
    {\large Máster Universitario en Ingeniería de la Energía \par}
    \vspace{2cm}
    {\Large Luis D. Aranda Sánchez\par}
    \vfill
    Director: Javier Rodríguez Martín
    \vfill
    {\large Septiembre 6, 2024\par}
\end{titlepage}

% Resumen (máximo de 5 páginas, incluyendo al final Palabras clave)
\clearpage
\pagestyle{simple}
% \newpage
\chapter*{Resumen}
\addcontentsline{toc}{chapter}{Resumen}
\input{capitulos/resumen/main.tex}

% Índice (paginado)
\clearpage
\pagestyle{simple}
% \newpage
\tableofcontents

% Introducción (donde se incluya los antecedentes y justificación)
\clearpage
\pagestyle{myfancy}
\newpage
\chapter{Introducción}
\input{capitulos/introduccion/main.tex}

% Objetivos
\chapter{Objetivos}
\input{capitulos/objetivos/main.tex}

% Metodología
\chapter{Metodología}
\input{capitulos/metodologia/main.tex}

% Resultados y discusión (incluyendo la valoración de impactos y de aspectos de responsabilidad legal, ética y profesional relacionados con el trabajo)
\chapter{Resultados y Discusión}
\input{capitulos/resultados_discusion/main.tex}

% Conclusiones
\chapter{Conclusiones}
\input{capitulos/conclusiones/main.tex}

% Planificación temporal y presupuesto
\chapter{Planificación Temporal y Presupuesto}
\input{capitulos/planificacion_presupuesto/main.tex}

% Bibliografía
\newpage
\addcontentsline{toc}{chapter}{Bibliografía}
\printbibliography

\end{document}


% Objetivos
\chapter{Objetivos}
\documentclass[a4paper,11pt,twoside]{report}
\usepackage[left=25mm,right=25mm,top=25mm,bottom=25mm,includehead,includefoot,headsep=15mm,footskip=15mm]{geometry}
\usepackage{graphicx}
\usepackage{fancyhdr}
\usepackage{titlesec}
\usepackage[spanish]{babel}
\usepackage[utf8]{inputenc}
\usepackage{amsmath}
\usepackage{setspace}
\usepackage{svg}
\usepackage{hyperref}
\usepackage[backend=biber,style=numeric]{biblatex}
\addbibresource{references.bib}
\hypersetup{
    colorlinks=true,
    linkcolor=blue,      % color of internal links (sections, etc.)
    urlcolor=blue,       % color of external links
    pdftitle={Optimización energética de sistema híbrido con bomba de calor, suelo radiante, fotovoltaica y almacenamiento para vivienda},    % title
    pdfauthor={Luis D. Aranda Sánchez},     % author
    pdfkeywords={palabra1, palabra2, código1, etc.} % list of keywords
}

% Font change to Arial
\usepackage{helvet}
\renewcommand{\familydefault}{\sfdefault}

% Chapter titles in uppercase and larger font
\titleformat{\chapter}[hang]{\large\bfseries}{\thechapter.}{1em}{\MakeUppercase}
\titleformat{\section}[hang]{\bfseries}{\thesection.}{1em}{}
\titleformat{\subsection}[hang]{\bfseries}{\thesubsection.}{1em}{}

% Fancyhdr setup
\setlength{\headheight}{14.30174pt} % Adjust to recommended value, slightly larger for safety
\fancyhf{} % Clear all headers and footers
\fancyhead[LE]{\nouppercase{\leftmark}}
\fancyhead[RO]{Optimización energética para vivienda}
\fancyfoot[LE]{\thepage}
\fancyfoot[RE]{Escuela Técnica Superior de Ingenieros Industriales (UPM)}
\fancyfoot[LO]{Luis D. Aranda Sánchez}
\fancyfoot[RO]{\thepage}
\renewcommand{\headrulewidth}{0.4pt}
\renewcommand{\footrulewidth}{0.4pt}

\fancypagestyle{myfancy}{
    \fancyhf{} % Clear all headers and footers
    \fancyhead[LE]{\nouppercase{\leftmark}}
    \fancyhead[RO]{Optimización energética para vivienda}
    \fancyfoot[LE]{\thepage}
    \fancyfoot[RE]{Escuela Técnica Superior de Ingenieros Industriales (UPM)}
    \fancyfoot[LO]{Luis D. Aranda Sánchez}
    \fancyfoot[RO]{\thepage}
    \renewcommand{\headrulewidth}{0.4pt}
    \renewcommand{\footrulewidth}{0.4pt}
}

\fancypagestyle{simple}{
    \fancyhf{} % Clear all headers and footers
    \renewcommand{\headrulewidth}{0pt}
    \renewcommand{\footrulewidth}{0pt}
}

% Line spacing
\setstretch{1.2}

% Document starts here
\begin{document}

% Portada
\begin{titlepage}
    \centering
    {\scshape\LARGE Universidad Politécnica de Madrid \par}
    \vspace{1cm}
    {\scshape\Large Escuela Técnica Superior de Ingenieros Industriales\par}
    \vspace{1.5cm}
    {\huge\bfseries Optimización energética de sistema híbrido con bomba de calor, suelo radiante, fotovoltaica y almacenamiento para vivienda \par}
    \vspace{1.5cm}
    {\Large\bfseries Trabajo de Fin de Máster\par}
    \vspace{0.5cm}
    {\large Máster Universitario en Ingeniería de la Energía \par}
    \vspace{2cm}
    {\Large Luis D. Aranda Sánchez\par}
    \vfill
    Director: Javier Rodríguez Martín
    \vfill
    {\large Septiembre 6, 2024\par}
\end{titlepage}

% Resumen (máximo de 5 páginas, incluyendo al final Palabras clave)
\clearpage
\pagestyle{simple}
% \newpage
\chapter*{Resumen}
\addcontentsline{toc}{chapter}{Resumen}
\input{capitulos/resumen/main.tex}

% Índice (paginado)
\clearpage
\pagestyle{simple}
% \newpage
\tableofcontents

% Introducción (donde se incluya los antecedentes y justificación)
\clearpage
\pagestyle{myfancy}
\newpage
\chapter{Introducción}
\input{capitulos/introduccion/main.tex}

% Objetivos
\chapter{Objetivos}
\input{capitulos/objetivos/main.tex}

% Metodología
\chapter{Metodología}
\input{capitulos/metodologia/main.tex}

% Resultados y discusión (incluyendo la valoración de impactos y de aspectos de responsabilidad legal, ética y profesional relacionados con el trabajo)
\chapter{Resultados y Discusión}
\input{capitulos/resultados_discusion/main.tex}

% Conclusiones
\chapter{Conclusiones}
\input{capitulos/conclusiones/main.tex}

% Planificación temporal y presupuesto
\chapter{Planificación Temporal y Presupuesto}
\input{capitulos/planificacion_presupuesto/main.tex}

% Bibliografía
\newpage
\addcontentsline{toc}{chapter}{Bibliografía}
\printbibliography

\end{document}


% Metodología
\chapter{Metodología}
\documentclass[a4paper,11pt,twoside]{report}
\usepackage[left=25mm,right=25mm,top=25mm,bottom=25mm,includehead,includefoot,headsep=15mm,footskip=15mm]{geometry}
\usepackage{graphicx}
\usepackage{fancyhdr}
\usepackage{titlesec}
\usepackage[spanish]{babel}
\usepackage[utf8]{inputenc}
\usepackage{amsmath}
\usepackage{setspace}
\usepackage{svg}
\usepackage{hyperref}
\usepackage[backend=biber,style=numeric]{biblatex}
\addbibresource{references.bib}
\hypersetup{
    colorlinks=true,
    linkcolor=blue,      % color of internal links (sections, etc.)
    urlcolor=blue,       % color of external links
    pdftitle={Optimización energética de sistema híbrido con bomba de calor, suelo radiante, fotovoltaica y almacenamiento para vivienda},    % title
    pdfauthor={Luis D. Aranda Sánchez},     % author
    pdfkeywords={palabra1, palabra2, código1, etc.} % list of keywords
}

% Font change to Arial
\usepackage{helvet}
\renewcommand{\familydefault}{\sfdefault}

% Chapter titles in uppercase and larger font
\titleformat{\chapter}[hang]{\large\bfseries}{\thechapter.}{1em}{\MakeUppercase}
\titleformat{\section}[hang]{\bfseries}{\thesection.}{1em}{}
\titleformat{\subsection}[hang]{\bfseries}{\thesubsection.}{1em}{}

% Fancyhdr setup
\setlength{\headheight}{14.30174pt} % Adjust to recommended value, slightly larger for safety
\fancyhf{} % Clear all headers and footers
\fancyhead[LE]{\nouppercase{\leftmark}}
\fancyhead[RO]{Optimización energética para vivienda}
\fancyfoot[LE]{\thepage}
\fancyfoot[RE]{Escuela Técnica Superior de Ingenieros Industriales (UPM)}
\fancyfoot[LO]{Luis D. Aranda Sánchez}
\fancyfoot[RO]{\thepage}
\renewcommand{\headrulewidth}{0.4pt}
\renewcommand{\footrulewidth}{0.4pt}

\fancypagestyle{myfancy}{
    \fancyhf{} % Clear all headers and footers
    \fancyhead[LE]{\nouppercase{\leftmark}}
    \fancyhead[RO]{Optimización energética para vivienda}
    \fancyfoot[LE]{\thepage}
    \fancyfoot[RE]{Escuela Técnica Superior de Ingenieros Industriales (UPM)}
    \fancyfoot[LO]{Luis D. Aranda Sánchez}
    \fancyfoot[RO]{\thepage}
    \renewcommand{\headrulewidth}{0.4pt}
    \renewcommand{\footrulewidth}{0.4pt}
}

\fancypagestyle{simple}{
    \fancyhf{} % Clear all headers and footers
    \renewcommand{\headrulewidth}{0pt}
    \renewcommand{\footrulewidth}{0pt}
}

% Line spacing
\setstretch{1.2}

% Document starts here
\begin{document}

% Portada
\begin{titlepage}
    \centering
    {\scshape\LARGE Universidad Politécnica de Madrid \par}
    \vspace{1cm}
    {\scshape\Large Escuela Técnica Superior de Ingenieros Industriales\par}
    \vspace{1.5cm}
    {\huge\bfseries Optimización energética de sistema híbrido con bomba de calor, suelo radiante, fotovoltaica y almacenamiento para vivienda \par}
    \vspace{1.5cm}
    {\Large\bfseries Trabajo de Fin de Máster\par}
    \vspace{0.5cm}
    {\large Máster Universitario en Ingeniería de la Energía \par}
    \vspace{2cm}
    {\Large Luis D. Aranda Sánchez\par}
    \vfill
    Director: Javier Rodríguez Martín
    \vfill
    {\large Septiembre 6, 2024\par}
\end{titlepage}

% Resumen (máximo de 5 páginas, incluyendo al final Palabras clave)
\clearpage
\pagestyle{simple}
% \newpage
\chapter*{Resumen}
\addcontentsline{toc}{chapter}{Resumen}
\input{capitulos/resumen/main.tex}

% Índice (paginado)
\clearpage
\pagestyle{simple}
% \newpage
\tableofcontents

% Introducción (donde se incluya los antecedentes y justificación)
\clearpage
\pagestyle{myfancy}
\newpage
\chapter{Introducción}
\input{capitulos/introduccion/main.tex}

% Objetivos
\chapter{Objetivos}
\input{capitulos/objetivos/main.tex}

% Metodología
\chapter{Metodología}
\input{capitulos/metodologia/main.tex}

% Resultados y discusión (incluyendo la valoración de impactos y de aspectos de responsabilidad legal, ética y profesional relacionados con el trabajo)
\chapter{Resultados y Discusión}
\input{capitulos/resultados_discusion/main.tex}

% Conclusiones
\chapter{Conclusiones}
\input{capitulos/conclusiones/main.tex}

% Planificación temporal y presupuesto
\chapter{Planificación Temporal y Presupuesto}
\input{capitulos/planificacion_presupuesto/main.tex}

% Bibliografía
\newpage
\addcontentsline{toc}{chapter}{Bibliografía}
\printbibliography

\end{document}


% Resultados y discusión (incluyendo la valoración de impactos y de aspectos de responsabilidad legal, ética y profesional relacionados con el trabajo)
\chapter{Resultados y Discusión}
\documentclass[a4paper,11pt,twoside]{report}
\usepackage[left=25mm,right=25mm,top=25mm,bottom=25mm,includehead,includefoot,headsep=15mm,footskip=15mm]{geometry}
\usepackage{graphicx}
\usepackage{fancyhdr}
\usepackage{titlesec}
\usepackage[spanish]{babel}
\usepackage[utf8]{inputenc}
\usepackage{amsmath}
\usepackage{setspace}
\usepackage{svg}
\usepackage{hyperref}
\usepackage[backend=biber,style=numeric]{biblatex}
\addbibresource{references.bib}
\hypersetup{
    colorlinks=true,
    linkcolor=blue,      % color of internal links (sections, etc.)
    urlcolor=blue,       % color of external links
    pdftitle={Optimización energética de sistema híbrido con bomba de calor, suelo radiante, fotovoltaica y almacenamiento para vivienda},    % title
    pdfauthor={Luis D. Aranda Sánchez},     % author
    pdfkeywords={palabra1, palabra2, código1, etc.} % list of keywords
}

% Font change to Arial
\usepackage{helvet}
\renewcommand{\familydefault}{\sfdefault}

% Chapter titles in uppercase and larger font
\titleformat{\chapter}[hang]{\large\bfseries}{\thechapter.}{1em}{\MakeUppercase}
\titleformat{\section}[hang]{\bfseries}{\thesection.}{1em}{}
\titleformat{\subsection}[hang]{\bfseries}{\thesubsection.}{1em}{}

% Fancyhdr setup
\setlength{\headheight}{14.30174pt} % Adjust to recommended value, slightly larger for safety
\fancyhf{} % Clear all headers and footers
\fancyhead[LE]{\nouppercase{\leftmark}}
\fancyhead[RO]{Optimización energética para vivienda}
\fancyfoot[LE]{\thepage}
\fancyfoot[RE]{Escuela Técnica Superior de Ingenieros Industriales (UPM)}
\fancyfoot[LO]{Luis D. Aranda Sánchez}
\fancyfoot[RO]{\thepage}
\renewcommand{\headrulewidth}{0.4pt}
\renewcommand{\footrulewidth}{0.4pt}

\fancypagestyle{myfancy}{
    \fancyhf{} % Clear all headers and footers
    \fancyhead[LE]{\nouppercase{\leftmark}}
    \fancyhead[RO]{Optimización energética para vivienda}
    \fancyfoot[LE]{\thepage}
    \fancyfoot[RE]{Escuela Técnica Superior de Ingenieros Industriales (UPM)}
    \fancyfoot[LO]{Luis D. Aranda Sánchez}
    \fancyfoot[RO]{\thepage}
    \renewcommand{\headrulewidth}{0.4pt}
    \renewcommand{\footrulewidth}{0.4pt}
}

\fancypagestyle{simple}{
    \fancyhf{} % Clear all headers and footers
    \renewcommand{\headrulewidth}{0pt}
    \renewcommand{\footrulewidth}{0pt}
}

% Line spacing
\setstretch{1.2}

% Document starts here
\begin{document}

% Portada
\begin{titlepage}
    \centering
    {\scshape\LARGE Universidad Politécnica de Madrid \par}
    \vspace{1cm}
    {\scshape\Large Escuela Técnica Superior de Ingenieros Industriales\par}
    \vspace{1.5cm}
    {\huge\bfseries Optimización energética de sistema híbrido con bomba de calor, suelo radiante, fotovoltaica y almacenamiento para vivienda \par}
    \vspace{1.5cm}
    {\Large\bfseries Trabajo de Fin de Máster\par}
    \vspace{0.5cm}
    {\large Máster Universitario en Ingeniería de la Energía \par}
    \vspace{2cm}
    {\Large Luis D. Aranda Sánchez\par}
    \vfill
    Director: Javier Rodríguez Martín
    \vfill
    {\large Septiembre 6, 2024\par}
\end{titlepage}

% Resumen (máximo de 5 páginas, incluyendo al final Palabras clave)
\clearpage
\pagestyle{simple}
% \newpage
\chapter*{Resumen}
\addcontentsline{toc}{chapter}{Resumen}
\input{capitulos/resumen/main.tex}

% Índice (paginado)
\clearpage
\pagestyle{simple}
% \newpage
\tableofcontents

% Introducción (donde se incluya los antecedentes y justificación)
\clearpage
\pagestyle{myfancy}
\newpage
\chapter{Introducción}
\input{capitulos/introduccion/main.tex}

% Objetivos
\chapter{Objetivos}
\input{capitulos/objetivos/main.tex}

% Metodología
\chapter{Metodología}
\input{capitulos/metodologia/main.tex}

% Resultados y discusión (incluyendo la valoración de impactos y de aspectos de responsabilidad legal, ética y profesional relacionados con el trabajo)
\chapter{Resultados y Discusión}
\input{capitulos/resultados_discusion/main.tex}

% Conclusiones
\chapter{Conclusiones}
\input{capitulos/conclusiones/main.tex}

% Planificación temporal y presupuesto
\chapter{Planificación Temporal y Presupuesto}
\input{capitulos/planificacion_presupuesto/main.tex}

% Bibliografía
\newpage
\addcontentsline{toc}{chapter}{Bibliografía}
\printbibliography

\end{document}


% Conclusiones
\chapter{Conclusiones}
\documentclass[a4paper,11pt,twoside]{report}
\usepackage[left=25mm,right=25mm,top=25mm,bottom=25mm,includehead,includefoot,headsep=15mm,footskip=15mm]{geometry}
\usepackage{graphicx}
\usepackage{fancyhdr}
\usepackage{titlesec}
\usepackage[spanish]{babel}
\usepackage[utf8]{inputenc}
\usepackage{amsmath}
\usepackage{setspace}
\usepackage{svg}
\usepackage{hyperref}
\usepackage[backend=biber,style=numeric]{biblatex}
\addbibresource{references.bib}
\hypersetup{
    colorlinks=true,
    linkcolor=blue,      % color of internal links (sections, etc.)
    urlcolor=blue,       % color of external links
    pdftitle={Optimización energética de sistema híbrido con bomba de calor, suelo radiante, fotovoltaica y almacenamiento para vivienda},    % title
    pdfauthor={Luis D. Aranda Sánchez},     % author
    pdfkeywords={palabra1, palabra2, código1, etc.} % list of keywords
}

% Font change to Arial
\usepackage{helvet}
\renewcommand{\familydefault}{\sfdefault}

% Chapter titles in uppercase and larger font
\titleformat{\chapter}[hang]{\large\bfseries}{\thechapter.}{1em}{\MakeUppercase}
\titleformat{\section}[hang]{\bfseries}{\thesection.}{1em}{}
\titleformat{\subsection}[hang]{\bfseries}{\thesubsection.}{1em}{}

% Fancyhdr setup
\setlength{\headheight}{14.30174pt} % Adjust to recommended value, slightly larger for safety
\fancyhf{} % Clear all headers and footers
\fancyhead[LE]{\nouppercase{\leftmark}}
\fancyhead[RO]{Optimización energética para vivienda}
\fancyfoot[LE]{\thepage}
\fancyfoot[RE]{Escuela Técnica Superior de Ingenieros Industriales (UPM)}
\fancyfoot[LO]{Luis D. Aranda Sánchez}
\fancyfoot[RO]{\thepage}
\renewcommand{\headrulewidth}{0.4pt}
\renewcommand{\footrulewidth}{0.4pt}

\fancypagestyle{myfancy}{
    \fancyhf{} % Clear all headers and footers
    \fancyhead[LE]{\nouppercase{\leftmark}}
    \fancyhead[RO]{Optimización energética para vivienda}
    \fancyfoot[LE]{\thepage}
    \fancyfoot[RE]{Escuela Técnica Superior de Ingenieros Industriales (UPM)}
    \fancyfoot[LO]{Luis D. Aranda Sánchez}
    \fancyfoot[RO]{\thepage}
    \renewcommand{\headrulewidth}{0.4pt}
    \renewcommand{\footrulewidth}{0.4pt}
}

\fancypagestyle{simple}{
    \fancyhf{} % Clear all headers and footers
    \renewcommand{\headrulewidth}{0pt}
    \renewcommand{\footrulewidth}{0pt}
}

% Line spacing
\setstretch{1.2}

% Document starts here
\begin{document}

% Portada
\begin{titlepage}
    \centering
    {\scshape\LARGE Universidad Politécnica de Madrid \par}
    \vspace{1cm}
    {\scshape\Large Escuela Técnica Superior de Ingenieros Industriales\par}
    \vspace{1.5cm}
    {\huge\bfseries Optimización energética de sistema híbrido con bomba de calor, suelo radiante, fotovoltaica y almacenamiento para vivienda \par}
    \vspace{1.5cm}
    {\Large\bfseries Trabajo de Fin de Máster\par}
    \vspace{0.5cm}
    {\large Máster Universitario en Ingeniería de la Energía \par}
    \vspace{2cm}
    {\Large Luis D. Aranda Sánchez\par}
    \vfill
    Director: Javier Rodríguez Martín
    \vfill
    {\large Septiembre 6, 2024\par}
\end{titlepage}

% Resumen (máximo de 5 páginas, incluyendo al final Palabras clave)
\clearpage
\pagestyle{simple}
% \newpage
\chapter*{Resumen}
\addcontentsline{toc}{chapter}{Resumen}
\input{capitulos/resumen/main.tex}

% Índice (paginado)
\clearpage
\pagestyle{simple}
% \newpage
\tableofcontents

% Introducción (donde se incluya los antecedentes y justificación)
\clearpage
\pagestyle{myfancy}
\newpage
\chapter{Introducción}
\input{capitulos/introduccion/main.tex}

% Objetivos
\chapter{Objetivos}
\input{capitulos/objetivos/main.tex}

% Metodología
\chapter{Metodología}
\input{capitulos/metodologia/main.tex}

% Resultados y discusión (incluyendo la valoración de impactos y de aspectos de responsabilidad legal, ética y profesional relacionados con el trabajo)
\chapter{Resultados y Discusión}
\input{capitulos/resultados_discusion/main.tex}

% Conclusiones
\chapter{Conclusiones}
\input{capitulos/conclusiones/main.tex}

% Planificación temporal y presupuesto
\chapter{Planificación Temporal y Presupuesto}
\input{capitulos/planificacion_presupuesto/main.tex}

% Bibliografía
\newpage
\addcontentsline{toc}{chapter}{Bibliografía}
\printbibliography

\end{document}


% Planificación temporal y presupuesto
\chapter{Planificación Temporal y Presupuesto}
\documentclass[a4paper,11pt,twoside]{report}
\usepackage[left=25mm,right=25mm,top=25mm,bottom=25mm,includehead,includefoot,headsep=15mm,footskip=15mm]{geometry}
\usepackage{graphicx}
\usepackage{fancyhdr}
\usepackage{titlesec}
\usepackage[spanish]{babel}
\usepackage[utf8]{inputenc}
\usepackage{amsmath}
\usepackage{setspace}
\usepackage{svg}
\usepackage{hyperref}
\usepackage[backend=biber,style=numeric]{biblatex}
\addbibresource{references.bib}
\hypersetup{
    colorlinks=true,
    linkcolor=blue,      % color of internal links (sections, etc.)
    urlcolor=blue,       % color of external links
    pdftitle={Optimización energética de sistema híbrido con bomba de calor, suelo radiante, fotovoltaica y almacenamiento para vivienda},    % title
    pdfauthor={Luis D. Aranda Sánchez},     % author
    pdfkeywords={palabra1, palabra2, código1, etc.} % list of keywords
}

% Font change to Arial
\usepackage{helvet}
\renewcommand{\familydefault}{\sfdefault}

% Chapter titles in uppercase and larger font
\titleformat{\chapter}[hang]{\large\bfseries}{\thechapter.}{1em}{\MakeUppercase}
\titleformat{\section}[hang]{\bfseries}{\thesection.}{1em}{}
\titleformat{\subsection}[hang]{\bfseries}{\thesubsection.}{1em}{}

% Fancyhdr setup
\setlength{\headheight}{14.30174pt} % Adjust to recommended value, slightly larger for safety
\fancyhf{} % Clear all headers and footers
\fancyhead[LE]{\nouppercase{\leftmark}}
\fancyhead[RO]{Optimización energética para vivienda}
\fancyfoot[LE]{\thepage}
\fancyfoot[RE]{Escuela Técnica Superior de Ingenieros Industriales (UPM)}
\fancyfoot[LO]{Luis D. Aranda Sánchez}
\fancyfoot[RO]{\thepage}
\renewcommand{\headrulewidth}{0.4pt}
\renewcommand{\footrulewidth}{0.4pt}

\fancypagestyle{myfancy}{
    \fancyhf{} % Clear all headers and footers
    \fancyhead[LE]{\nouppercase{\leftmark}}
    \fancyhead[RO]{Optimización energética para vivienda}
    \fancyfoot[LE]{\thepage}
    \fancyfoot[RE]{Escuela Técnica Superior de Ingenieros Industriales (UPM)}
    \fancyfoot[LO]{Luis D. Aranda Sánchez}
    \fancyfoot[RO]{\thepage}
    \renewcommand{\headrulewidth}{0.4pt}
    \renewcommand{\footrulewidth}{0.4pt}
}

\fancypagestyle{simple}{
    \fancyhf{} % Clear all headers and footers
    \renewcommand{\headrulewidth}{0pt}
    \renewcommand{\footrulewidth}{0pt}
}

% Line spacing
\setstretch{1.2}

% Document starts here
\begin{document}

% Portada
\begin{titlepage}
    \centering
    {\scshape\LARGE Universidad Politécnica de Madrid \par}
    \vspace{1cm}
    {\scshape\Large Escuela Técnica Superior de Ingenieros Industriales\par}
    \vspace{1.5cm}
    {\huge\bfseries Optimización energética de sistema híbrido con bomba de calor, suelo radiante, fotovoltaica y almacenamiento para vivienda \par}
    \vspace{1.5cm}
    {\Large\bfseries Trabajo de Fin de Máster\par}
    \vspace{0.5cm}
    {\large Máster Universitario en Ingeniería de la Energía \par}
    \vspace{2cm}
    {\Large Luis D. Aranda Sánchez\par}
    \vfill
    Director: Javier Rodríguez Martín
    \vfill
    {\large Septiembre 6, 2024\par}
\end{titlepage}

% Resumen (máximo de 5 páginas, incluyendo al final Palabras clave)
\clearpage
\pagestyle{simple}
% \newpage
\chapter*{Resumen}
\addcontentsline{toc}{chapter}{Resumen}
\input{capitulos/resumen/main.tex}

% Índice (paginado)
\clearpage
\pagestyle{simple}
% \newpage
\tableofcontents

% Introducción (donde se incluya los antecedentes y justificación)
\clearpage
\pagestyle{myfancy}
\newpage
\chapter{Introducción}
\input{capitulos/introduccion/main.tex}

% Objetivos
\chapter{Objetivos}
\input{capitulos/objetivos/main.tex}

% Metodología
\chapter{Metodología}
\input{capitulos/metodologia/main.tex}

% Resultados y discusión (incluyendo la valoración de impactos y de aspectos de responsabilidad legal, ética y profesional relacionados con el trabajo)
\chapter{Resultados y Discusión}
\input{capitulos/resultados_discusion/main.tex}

% Conclusiones
\chapter{Conclusiones}
\input{capitulos/conclusiones/main.tex}

% Planificación temporal y presupuesto
\chapter{Planificación Temporal y Presupuesto}
\input{capitulos/planificacion_presupuesto/main.tex}

% Bibliografía
\newpage
\addcontentsline{toc}{chapter}{Bibliografía}
\printbibliography

\end{document}


% Bibliografía
\newpage
\addcontentsline{toc}{chapter}{Bibliografía}
\printbibliography

\end{document}


% Conclusiones
\chapter{Conclusiones}
\documentclass[a4paper,11pt,twoside]{report}
\usepackage[left=25mm,right=25mm,top=25mm,bottom=25mm,includehead,includefoot,headsep=15mm,footskip=15mm]{geometry}
\usepackage{graphicx}
\usepackage{fancyhdr}
\usepackage{titlesec}
\usepackage[spanish]{babel}
\usepackage[utf8]{inputenc}
\usepackage{amsmath}
\usepackage{setspace}
\usepackage{svg}
\usepackage{hyperref}
\usepackage[backend=biber,style=numeric]{biblatex}
\addbibresource{references.bib}
\hypersetup{
    colorlinks=true,
    linkcolor=blue,      % color of internal links (sections, etc.)
    urlcolor=blue,       % color of external links
    pdftitle={Optimización energética de sistema híbrido con bomba de calor, suelo radiante, fotovoltaica y almacenamiento para vivienda},    % title
    pdfauthor={Luis D. Aranda Sánchez},     % author
    pdfkeywords={palabra1, palabra2, código1, etc.} % list of keywords
}

% Font change to Arial
\usepackage{helvet}
\renewcommand{\familydefault}{\sfdefault}

% Chapter titles in uppercase and larger font
\titleformat{\chapter}[hang]{\large\bfseries}{\thechapter.}{1em}{\MakeUppercase}
\titleformat{\section}[hang]{\bfseries}{\thesection.}{1em}{}
\titleformat{\subsection}[hang]{\bfseries}{\thesubsection.}{1em}{}

% Fancyhdr setup
\setlength{\headheight}{14.30174pt} % Adjust to recommended value, slightly larger for safety
\fancyhf{} % Clear all headers and footers
\fancyhead[LE]{\nouppercase{\leftmark}}
\fancyhead[RO]{Optimización energética para vivienda}
\fancyfoot[LE]{\thepage}
\fancyfoot[RE]{Escuela Técnica Superior de Ingenieros Industriales (UPM)}
\fancyfoot[LO]{Luis D. Aranda Sánchez}
\fancyfoot[RO]{\thepage}
\renewcommand{\headrulewidth}{0.4pt}
\renewcommand{\footrulewidth}{0.4pt}

\fancypagestyle{myfancy}{
    \fancyhf{} % Clear all headers and footers
    \fancyhead[LE]{\nouppercase{\leftmark}}
    \fancyhead[RO]{Optimización energética para vivienda}
    \fancyfoot[LE]{\thepage}
    \fancyfoot[RE]{Escuela Técnica Superior de Ingenieros Industriales (UPM)}
    \fancyfoot[LO]{Luis D. Aranda Sánchez}
    \fancyfoot[RO]{\thepage}
    \renewcommand{\headrulewidth}{0.4pt}
    \renewcommand{\footrulewidth}{0.4pt}
}

\fancypagestyle{simple}{
    \fancyhf{} % Clear all headers and footers
    \renewcommand{\headrulewidth}{0pt}
    \renewcommand{\footrulewidth}{0pt}
}

% Line spacing
\setstretch{1.2}

% Document starts here
\begin{document}

% Portada
\begin{titlepage}
    \centering
    {\scshape\LARGE Universidad Politécnica de Madrid \par}
    \vspace{1cm}
    {\scshape\Large Escuela Técnica Superior de Ingenieros Industriales\par}
    \vspace{1.5cm}
    {\huge\bfseries Optimización energética de sistema híbrido con bomba de calor, suelo radiante, fotovoltaica y almacenamiento para vivienda \par}
    \vspace{1.5cm}
    {\Large\bfseries Trabajo de Fin de Máster\par}
    \vspace{0.5cm}
    {\large Máster Universitario en Ingeniería de la Energía \par}
    \vspace{2cm}
    {\Large Luis D. Aranda Sánchez\par}
    \vfill
    Director: Javier Rodríguez Martín
    \vfill
    {\large Septiembre 6, 2024\par}
\end{titlepage}

% Resumen (máximo de 5 páginas, incluyendo al final Palabras clave)
\clearpage
\pagestyle{simple}
% \newpage
\chapter*{Resumen}
\addcontentsline{toc}{chapter}{Resumen}
\documentclass[a4paper,11pt,twoside]{report}
\usepackage[left=25mm,right=25mm,top=25mm,bottom=25mm,includehead,includefoot,headsep=15mm,footskip=15mm]{geometry}
\usepackage{graphicx}
\usepackage{fancyhdr}
\usepackage{titlesec}
\usepackage[spanish]{babel}
\usepackage[utf8]{inputenc}
\usepackage{amsmath}
\usepackage{setspace}
\usepackage{svg}
\usepackage{hyperref}
\usepackage[backend=biber,style=numeric]{biblatex}
\addbibresource{references.bib}
\hypersetup{
    colorlinks=true,
    linkcolor=blue,      % color of internal links (sections, etc.)
    urlcolor=blue,       % color of external links
    pdftitle={Optimización energética de sistema híbrido con bomba de calor, suelo radiante, fotovoltaica y almacenamiento para vivienda},    % title
    pdfauthor={Luis D. Aranda Sánchez},     % author
    pdfkeywords={palabra1, palabra2, código1, etc.} % list of keywords
}

% Font change to Arial
\usepackage{helvet}
\renewcommand{\familydefault}{\sfdefault}

% Chapter titles in uppercase and larger font
\titleformat{\chapter}[hang]{\large\bfseries}{\thechapter.}{1em}{\MakeUppercase}
\titleformat{\section}[hang]{\bfseries}{\thesection.}{1em}{}
\titleformat{\subsection}[hang]{\bfseries}{\thesubsection.}{1em}{}

% Fancyhdr setup
\setlength{\headheight}{14.30174pt} % Adjust to recommended value, slightly larger for safety
\fancyhf{} % Clear all headers and footers
\fancyhead[LE]{\nouppercase{\leftmark}}
\fancyhead[RO]{Optimización energética para vivienda}
\fancyfoot[LE]{\thepage}
\fancyfoot[RE]{Escuela Técnica Superior de Ingenieros Industriales (UPM)}
\fancyfoot[LO]{Luis D. Aranda Sánchez}
\fancyfoot[RO]{\thepage}
\renewcommand{\headrulewidth}{0.4pt}
\renewcommand{\footrulewidth}{0.4pt}

\fancypagestyle{myfancy}{
    \fancyhf{} % Clear all headers and footers
    \fancyhead[LE]{\nouppercase{\leftmark}}
    \fancyhead[RO]{Optimización energética para vivienda}
    \fancyfoot[LE]{\thepage}
    \fancyfoot[RE]{Escuela Técnica Superior de Ingenieros Industriales (UPM)}
    \fancyfoot[LO]{Luis D. Aranda Sánchez}
    \fancyfoot[RO]{\thepage}
    \renewcommand{\headrulewidth}{0.4pt}
    \renewcommand{\footrulewidth}{0.4pt}
}

\fancypagestyle{simple}{
    \fancyhf{} % Clear all headers and footers
    \renewcommand{\headrulewidth}{0pt}
    \renewcommand{\footrulewidth}{0pt}
}

% Line spacing
\setstretch{1.2}

% Document starts here
\begin{document}

% Portada
\begin{titlepage}
    \centering
    {\scshape\LARGE Universidad Politécnica de Madrid \par}
    \vspace{1cm}
    {\scshape\Large Escuela Técnica Superior de Ingenieros Industriales\par}
    \vspace{1.5cm}
    {\huge\bfseries Optimización energética de sistema híbrido con bomba de calor, suelo radiante, fotovoltaica y almacenamiento para vivienda \par}
    \vspace{1.5cm}
    {\Large\bfseries Trabajo de Fin de Máster\par}
    \vspace{0.5cm}
    {\large Máster Universitario en Ingeniería de la Energía \par}
    \vspace{2cm}
    {\Large Luis D. Aranda Sánchez\par}
    \vfill
    Director: Javier Rodríguez Martín
    \vfill
    {\large Septiembre 6, 2024\par}
\end{titlepage}

% Resumen (máximo de 5 páginas, incluyendo al final Palabras clave)
\clearpage
\pagestyle{simple}
% \newpage
\chapter*{Resumen}
\addcontentsline{toc}{chapter}{Resumen}
\input{capitulos/resumen/main.tex}

% Índice (paginado)
\clearpage
\pagestyle{simple}
% \newpage
\tableofcontents

% Introducción (donde se incluya los antecedentes y justificación)
\clearpage
\pagestyle{myfancy}
\newpage
\chapter{Introducción}
\input{capitulos/introduccion/main.tex}

% Objetivos
\chapter{Objetivos}
\input{capitulos/objetivos/main.tex}

% Metodología
\chapter{Metodología}
\input{capitulos/metodologia/main.tex}

% Resultados y discusión (incluyendo la valoración de impactos y de aspectos de responsabilidad legal, ética y profesional relacionados con el trabajo)
\chapter{Resultados y Discusión}
\input{capitulos/resultados_discusion/main.tex}

% Conclusiones
\chapter{Conclusiones}
\input{capitulos/conclusiones/main.tex}

% Planificación temporal y presupuesto
\chapter{Planificación Temporal y Presupuesto}
\input{capitulos/planificacion_presupuesto/main.tex}

% Bibliografía
\newpage
\addcontentsline{toc}{chapter}{Bibliografía}
\printbibliography

\end{document}


% Índice (paginado)
\clearpage
\pagestyle{simple}
% \newpage
\tableofcontents

% Introducción (donde se incluya los antecedentes y justificación)
\clearpage
\pagestyle{myfancy}
\newpage
\chapter{Introducción}
\documentclass[a4paper,11pt,twoside]{report}
\usepackage[left=25mm,right=25mm,top=25mm,bottom=25mm,includehead,includefoot,headsep=15mm,footskip=15mm]{geometry}
\usepackage{graphicx}
\usepackage{fancyhdr}
\usepackage{titlesec}
\usepackage[spanish]{babel}
\usepackage[utf8]{inputenc}
\usepackage{amsmath}
\usepackage{setspace}
\usepackage{svg}
\usepackage{hyperref}
\usepackage[backend=biber,style=numeric]{biblatex}
\addbibresource{references.bib}
\hypersetup{
    colorlinks=true,
    linkcolor=blue,      % color of internal links (sections, etc.)
    urlcolor=blue,       % color of external links
    pdftitle={Optimización energética de sistema híbrido con bomba de calor, suelo radiante, fotovoltaica y almacenamiento para vivienda},    % title
    pdfauthor={Luis D. Aranda Sánchez},     % author
    pdfkeywords={palabra1, palabra2, código1, etc.} % list of keywords
}

% Font change to Arial
\usepackage{helvet}
\renewcommand{\familydefault}{\sfdefault}

% Chapter titles in uppercase and larger font
\titleformat{\chapter}[hang]{\large\bfseries}{\thechapter.}{1em}{\MakeUppercase}
\titleformat{\section}[hang]{\bfseries}{\thesection.}{1em}{}
\titleformat{\subsection}[hang]{\bfseries}{\thesubsection.}{1em}{}

% Fancyhdr setup
\setlength{\headheight}{14.30174pt} % Adjust to recommended value, slightly larger for safety
\fancyhf{} % Clear all headers and footers
\fancyhead[LE]{\nouppercase{\leftmark}}
\fancyhead[RO]{Optimización energética para vivienda}
\fancyfoot[LE]{\thepage}
\fancyfoot[RE]{Escuela Técnica Superior de Ingenieros Industriales (UPM)}
\fancyfoot[LO]{Luis D. Aranda Sánchez}
\fancyfoot[RO]{\thepage}
\renewcommand{\headrulewidth}{0.4pt}
\renewcommand{\footrulewidth}{0.4pt}

\fancypagestyle{myfancy}{
    \fancyhf{} % Clear all headers and footers
    \fancyhead[LE]{\nouppercase{\leftmark}}
    \fancyhead[RO]{Optimización energética para vivienda}
    \fancyfoot[LE]{\thepage}
    \fancyfoot[RE]{Escuela Técnica Superior de Ingenieros Industriales (UPM)}
    \fancyfoot[LO]{Luis D. Aranda Sánchez}
    \fancyfoot[RO]{\thepage}
    \renewcommand{\headrulewidth}{0.4pt}
    \renewcommand{\footrulewidth}{0.4pt}
}

\fancypagestyle{simple}{
    \fancyhf{} % Clear all headers and footers
    \renewcommand{\headrulewidth}{0pt}
    \renewcommand{\footrulewidth}{0pt}
}

% Line spacing
\setstretch{1.2}

% Document starts here
\begin{document}

% Portada
\begin{titlepage}
    \centering
    {\scshape\LARGE Universidad Politécnica de Madrid \par}
    \vspace{1cm}
    {\scshape\Large Escuela Técnica Superior de Ingenieros Industriales\par}
    \vspace{1.5cm}
    {\huge\bfseries Optimización energética de sistema híbrido con bomba de calor, suelo radiante, fotovoltaica y almacenamiento para vivienda \par}
    \vspace{1.5cm}
    {\Large\bfseries Trabajo de Fin de Máster\par}
    \vspace{0.5cm}
    {\large Máster Universitario en Ingeniería de la Energía \par}
    \vspace{2cm}
    {\Large Luis D. Aranda Sánchez\par}
    \vfill
    Director: Javier Rodríguez Martín
    \vfill
    {\large Septiembre 6, 2024\par}
\end{titlepage}

% Resumen (máximo de 5 páginas, incluyendo al final Palabras clave)
\clearpage
\pagestyle{simple}
% \newpage
\chapter*{Resumen}
\addcontentsline{toc}{chapter}{Resumen}
\input{capitulos/resumen/main.tex}

% Índice (paginado)
\clearpage
\pagestyle{simple}
% \newpage
\tableofcontents

% Introducción (donde se incluya los antecedentes y justificación)
\clearpage
\pagestyle{myfancy}
\newpage
\chapter{Introducción}
\input{capitulos/introduccion/main.tex}

% Objetivos
\chapter{Objetivos}
\input{capitulos/objetivos/main.tex}

% Metodología
\chapter{Metodología}
\input{capitulos/metodologia/main.tex}

% Resultados y discusión (incluyendo la valoración de impactos y de aspectos de responsabilidad legal, ética y profesional relacionados con el trabajo)
\chapter{Resultados y Discusión}
\input{capitulos/resultados_discusion/main.tex}

% Conclusiones
\chapter{Conclusiones}
\input{capitulos/conclusiones/main.tex}

% Planificación temporal y presupuesto
\chapter{Planificación Temporal y Presupuesto}
\input{capitulos/planificacion_presupuesto/main.tex}

% Bibliografía
\newpage
\addcontentsline{toc}{chapter}{Bibliografía}
\printbibliography

\end{document}


% Objetivos
\chapter{Objetivos}
\documentclass[a4paper,11pt,twoside]{report}
\usepackage[left=25mm,right=25mm,top=25mm,bottom=25mm,includehead,includefoot,headsep=15mm,footskip=15mm]{geometry}
\usepackage{graphicx}
\usepackage{fancyhdr}
\usepackage{titlesec}
\usepackage[spanish]{babel}
\usepackage[utf8]{inputenc}
\usepackage{amsmath}
\usepackage{setspace}
\usepackage{svg}
\usepackage{hyperref}
\usepackage[backend=biber,style=numeric]{biblatex}
\addbibresource{references.bib}
\hypersetup{
    colorlinks=true,
    linkcolor=blue,      % color of internal links (sections, etc.)
    urlcolor=blue,       % color of external links
    pdftitle={Optimización energética de sistema híbrido con bomba de calor, suelo radiante, fotovoltaica y almacenamiento para vivienda},    % title
    pdfauthor={Luis D. Aranda Sánchez},     % author
    pdfkeywords={palabra1, palabra2, código1, etc.} % list of keywords
}

% Font change to Arial
\usepackage{helvet}
\renewcommand{\familydefault}{\sfdefault}

% Chapter titles in uppercase and larger font
\titleformat{\chapter}[hang]{\large\bfseries}{\thechapter.}{1em}{\MakeUppercase}
\titleformat{\section}[hang]{\bfseries}{\thesection.}{1em}{}
\titleformat{\subsection}[hang]{\bfseries}{\thesubsection.}{1em}{}

% Fancyhdr setup
\setlength{\headheight}{14.30174pt} % Adjust to recommended value, slightly larger for safety
\fancyhf{} % Clear all headers and footers
\fancyhead[LE]{\nouppercase{\leftmark}}
\fancyhead[RO]{Optimización energética para vivienda}
\fancyfoot[LE]{\thepage}
\fancyfoot[RE]{Escuela Técnica Superior de Ingenieros Industriales (UPM)}
\fancyfoot[LO]{Luis D. Aranda Sánchez}
\fancyfoot[RO]{\thepage}
\renewcommand{\headrulewidth}{0.4pt}
\renewcommand{\footrulewidth}{0.4pt}

\fancypagestyle{myfancy}{
    \fancyhf{} % Clear all headers and footers
    \fancyhead[LE]{\nouppercase{\leftmark}}
    \fancyhead[RO]{Optimización energética para vivienda}
    \fancyfoot[LE]{\thepage}
    \fancyfoot[RE]{Escuela Técnica Superior de Ingenieros Industriales (UPM)}
    \fancyfoot[LO]{Luis D. Aranda Sánchez}
    \fancyfoot[RO]{\thepage}
    \renewcommand{\headrulewidth}{0.4pt}
    \renewcommand{\footrulewidth}{0.4pt}
}

\fancypagestyle{simple}{
    \fancyhf{} % Clear all headers and footers
    \renewcommand{\headrulewidth}{0pt}
    \renewcommand{\footrulewidth}{0pt}
}

% Line spacing
\setstretch{1.2}

% Document starts here
\begin{document}

% Portada
\begin{titlepage}
    \centering
    {\scshape\LARGE Universidad Politécnica de Madrid \par}
    \vspace{1cm}
    {\scshape\Large Escuela Técnica Superior de Ingenieros Industriales\par}
    \vspace{1.5cm}
    {\huge\bfseries Optimización energética de sistema híbrido con bomba de calor, suelo radiante, fotovoltaica y almacenamiento para vivienda \par}
    \vspace{1.5cm}
    {\Large\bfseries Trabajo de Fin de Máster\par}
    \vspace{0.5cm}
    {\large Máster Universitario en Ingeniería de la Energía \par}
    \vspace{2cm}
    {\Large Luis D. Aranda Sánchez\par}
    \vfill
    Director: Javier Rodríguez Martín
    \vfill
    {\large Septiembre 6, 2024\par}
\end{titlepage}

% Resumen (máximo de 5 páginas, incluyendo al final Palabras clave)
\clearpage
\pagestyle{simple}
% \newpage
\chapter*{Resumen}
\addcontentsline{toc}{chapter}{Resumen}
\input{capitulos/resumen/main.tex}

% Índice (paginado)
\clearpage
\pagestyle{simple}
% \newpage
\tableofcontents

% Introducción (donde se incluya los antecedentes y justificación)
\clearpage
\pagestyle{myfancy}
\newpage
\chapter{Introducción}
\input{capitulos/introduccion/main.tex}

% Objetivos
\chapter{Objetivos}
\input{capitulos/objetivos/main.tex}

% Metodología
\chapter{Metodología}
\input{capitulos/metodologia/main.tex}

% Resultados y discusión (incluyendo la valoración de impactos y de aspectos de responsabilidad legal, ética y profesional relacionados con el trabajo)
\chapter{Resultados y Discusión}
\input{capitulos/resultados_discusion/main.tex}

% Conclusiones
\chapter{Conclusiones}
\input{capitulos/conclusiones/main.tex}

% Planificación temporal y presupuesto
\chapter{Planificación Temporal y Presupuesto}
\input{capitulos/planificacion_presupuesto/main.tex}

% Bibliografía
\newpage
\addcontentsline{toc}{chapter}{Bibliografía}
\printbibliography

\end{document}


% Metodología
\chapter{Metodología}
\documentclass[a4paper,11pt,twoside]{report}
\usepackage[left=25mm,right=25mm,top=25mm,bottom=25mm,includehead,includefoot,headsep=15mm,footskip=15mm]{geometry}
\usepackage{graphicx}
\usepackage{fancyhdr}
\usepackage{titlesec}
\usepackage[spanish]{babel}
\usepackage[utf8]{inputenc}
\usepackage{amsmath}
\usepackage{setspace}
\usepackage{svg}
\usepackage{hyperref}
\usepackage[backend=biber,style=numeric]{biblatex}
\addbibresource{references.bib}
\hypersetup{
    colorlinks=true,
    linkcolor=blue,      % color of internal links (sections, etc.)
    urlcolor=blue,       % color of external links
    pdftitle={Optimización energética de sistema híbrido con bomba de calor, suelo radiante, fotovoltaica y almacenamiento para vivienda},    % title
    pdfauthor={Luis D. Aranda Sánchez},     % author
    pdfkeywords={palabra1, palabra2, código1, etc.} % list of keywords
}

% Font change to Arial
\usepackage{helvet}
\renewcommand{\familydefault}{\sfdefault}

% Chapter titles in uppercase and larger font
\titleformat{\chapter}[hang]{\large\bfseries}{\thechapter.}{1em}{\MakeUppercase}
\titleformat{\section}[hang]{\bfseries}{\thesection.}{1em}{}
\titleformat{\subsection}[hang]{\bfseries}{\thesubsection.}{1em}{}

% Fancyhdr setup
\setlength{\headheight}{14.30174pt} % Adjust to recommended value, slightly larger for safety
\fancyhf{} % Clear all headers and footers
\fancyhead[LE]{\nouppercase{\leftmark}}
\fancyhead[RO]{Optimización energética para vivienda}
\fancyfoot[LE]{\thepage}
\fancyfoot[RE]{Escuela Técnica Superior de Ingenieros Industriales (UPM)}
\fancyfoot[LO]{Luis D. Aranda Sánchez}
\fancyfoot[RO]{\thepage}
\renewcommand{\headrulewidth}{0.4pt}
\renewcommand{\footrulewidth}{0.4pt}

\fancypagestyle{myfancy}{
    \fancyhf{} % Clear all headers and footers
    \fancyhead[LE]{\nouppercase{\leftmark}}
    \fancyhead[RO]{Optimización energética para vivienda}
    \fancyfoot[LE]{\thepage}
    \fancyfoot[RE]{Escuela Técnica Superior de Ingenieros Industriales (UPM)}
    \fancyfoot[LO]{Luis D. Aranda Sánchez}
    \fancyfoot[RO]{\thepage}
    \renewcommand{\headrulewidth}{0.4pt}
    \renewcommand{\footrulewidth}{0.4pt}
}

\fancypagestyle{simple}{
    \fancyhf{} % Clear all headers and footers
    \renewcommand{\headrulewidth}{0pt}
    \renewcommand{\footrulewidth}{0pt}
}

% Line spacing
\setstretch{1.2}

% Document starts here
\begin{document}

% Portada
\begin{titlepage}
    \centering
    {\scshape\LARGE Universidad Politécnica de Madrid \par}
    \vspace{1cm}
    {\scshape\Large Escuela Técnica Superior de Ingenieros Industriales\par}
    \vspace{1.5cm}
    {\huge\bfseries Optimización energética de sistema híbrido con bomba de calor, suelo radiante, fotovoltaica y almacenamiento para vivienda \par}
    \vspace{1.5cm}
    {\Large\bfseries Trabajo de Fin de Máster\par}
    \vspace{0.5cm}
    {\large Máster Universitario en Ingeniería de la Energía \par}
    \vspace{2cm}
    {\Large Luis D. Aranda Sánchez\par}
    \vfill
    Director: Javier Rodríguez Martín
    \vfill
    {\large Septiembre 6, 2024\par}
\end{titlepage}

% Resumen (máximo de 5 páginas, incluyendo al final Palabras clave)
\clearpage
\pagestyle{simple}
% \newpage
\chapter*{Resumen}
\addcontentsline{toc}{chapter}{Resumen}
\input{capitulos/resumen/main.tex}

% Índice (paginado)
\clearpage
\pagestyle{simple}
% \newpage
\tableofcontents

% Introducción (donde se incluya los antecedentes y justificación)
\clearpage
\pagestyle{myfancy}
\newpage
\chapter{Introducción}
\input{capitulos/introduccion/main.tex}

% Objetivos
\chapter{Objetivos}
\input{capitulos/objetivos/main.tex}

% Metodología
\chapter{Metodología}
\input{capitulos/metodologia/main.tex}

% Resultados y discusión (incluyendo la valoración de impactos y de aspectos de responsabilidad legal, ética y profesional relacionados con el trabajo)
\chapter{Resultados y Discusión}
\input{capitulos/resultados_discusion/main.tex}

% Conclusiones
\chapter{Conclusiones}
\input{capitulos/conclusiones/main.tex}

% Planificación temporal y presupuesto
\chapter{Planificación Temporal y Presupuesto}
\input{capitulos/planificacion_presupuesto/main.tex}

% Bibliografía
\newpage
\addcontentsline{toc}{chapter}{Bibliografía}
\printbibliography

\end{document}


% Resultados y discusión (incluyendo la valoración de impactos y de aspectos de responsabilidad legal, ética y profesional relacionados con el trabajo)
\chapter{Resultados y Discusión}
\documentclass[a4paper,11pt,twoside]{report}
\usepackage[left=25mm,right=25mm,top=25mm,bottom=25mm,includehead,includefoot,headsep=15mm,footskip=15mm]{geometry}
\usepackage{graphicx}
\usepackage{fancyhdr}
\usepackage{titlesec}
\usepackage[spanish]{babel}
\usepackage[utf8]{inputenc}
\usepackage{amsmath}
\usepackage{setspace}
\usepackage{svg}
\usepackage{hyperref}
\usepackage[backend=biber,style=numeric]{biblatex}
\addbibresource{references.bib}
\hypersetup{
    colorlinks=true,
    linkcolor=blue,      % color of internal links (sections, etc.)
    urlcolor=blue,       % color of external links
    pdftitle={Optimización energética de sistema híbrido con bomba de calor, suelo radiante, fotovoltaica y almacenamiento para vivienda},    % title
    pdfauthor={Luis D. Aranda Sánchez},     % author
    pdfkeywords={palabra1, palabra2, código1, etc.} % list of keywords
}

% Font change to Arial
\usepackage{helvet}
\renewcommand{\familydefault}{\sfdefault}

% Chapter titles in uppercase and larger font
\titleformat{\chapter}[hang]{\large\bfseries}{\thechapter.}{1em}{\MakeUppercase}
\titleformat{\section}[hang]{\bfseries}{\thesection.}{1em}{}
\titleformat{\subsection}[hang]{\bfseries}{\thesubsection.}{1em}{}

% Fancyhdr setup
\setlength{\headheight}{14.30174pt} % Adjust to recommended value, slightly larger for safety
\fancyhf{} % Clear all headers and footers
\fancyhead[LE]{\nouppercase{\leftmark}}
\fancyhead[RO]{Optimización energética para vivienda}
\fancyfoot[LE]{\thepage}
\fancyfoot[RE]{Escuela Técnica Superior de Ingenieros Industriales (UPM)}
\fancyfoot[LO]{Luis D. Aranda Sánchez}
\fancyfoot[RO]{\thepage}
\renewcommand{\headrulewidth}{0.4pt}
\renewcommand{\footrulewidth}{0.4pt}

\fancypagestyle{myfancy}{
    \fancyhf{} % Clear all headers and footers
    \fancyhead[LE]{\nouppercase{\leftmark}}
    \fancyhead[RO]{Optimización energética para vivienda}
    \fancyfoot[LE]{\thepage}
    \fancyfoot[RE]{Escuela Técnica Superior de Ingenieros Industriales (UPM)}
    \fancyfoot[LO]{Luis D. Aranda Sánchez}
    \fancyfoot[RO]{\thepage}
    \renewcommand{\headrulewidth}{0.4pt}
    \renewcommand{\footrulewidth}{0.4pt}
}

\fancypagestyle{simple}{
    \fancyhf{} % Clear all headers and footers
    \renewcommand{\headrulewidth}{0pt}
    \renewcommand{\footrulewidth}{0pt}
}

% Line spacing
\setstretch{1.2}

% Document starts here
\begin{document}

% Portada
\begin{titlepage}
    \centering
    {\scshape\LARGE Universidad Politécnica de Madrid \par}
    \vspace{1cm}
    {\scshape\Large Escuela Técnica Superior de Ingenieros Industriales\par}
    \vspace{1.5cm}
    {\huge\bfseries Optimización energética de sistema híbrido con bomba de calor, suelo radiante, fotovoltaica y almacenamiento para vivienda \par}
    \vspace{1.5cm}
    {\Large\bfseries Trabajo de Fin de Máster\par}
    \vspace{0.5cm}
    {\large Máster Universitario en Ingeniería de la Energía \par}
    \vspace{2cm}
    {\Large Luis D. Aranda Sánchez\par}
    \vfill
    Director: Javier Rodríguez Martín
    \vfill
    {\large Septiembre 6, 2024\par}
\end{titlepage}

% Resumen (máximo de 5 páginas, incluyendo al final Palabras clave)
\clearpage
\pagestyle{simple}
% \newpage
\chapter*{Resumen}
\addcontentsline{toc}{chapter}{Resumen}
\input{capitulos/resumen/main.tex}

% Índice (paginado)
\clearpage
\pagestyle{simple}
% \newpage
\tableofcontents

% Introducción (donde se incluya los antecedentes y justificación)
\clearpage
\pagestyle{myfancy}
\newpage
\chapter{Introducción}
\input{capitulos/introduccion/main.tex}

% Objetivos
\chapter{Objetivos}
\input{capitulos/objetivos/main.tex}

% Metodología
\chapter{Metodología}
\input{capitulos/metodologia/main.tex}

% Resultados y discusión (incluyendo la valoración de impactos y de aspectos de responsabilidad legal, ética y profesional relacionados con el trabajo)
\chapter{Resultados y Discusión}
\input{capitulos/resultados_discusion/main.tex}

% Conclusiones
\chapter{Conclusiones}
\input{capitulos/conclusiones/main.tex}

% Planificación temporal y presupuesto
\chapter{Planificación Temporal y Presupuesto}
\input{capitulos/planificacion_presupuesto/main.tex}

% Bibliografía
\newpage
\addcontentsline{toc}{chapter}{Bibliografía}
\printbibliography

\end{document}


% Conclusiones
\chapter{Conclusiones}
\documentclass[a4paper,11pt,twoside]{report}
\usepackage[left=25mm,right=25mm,top=25mm,bottom=25mm,includehead,includefoot,headsep=15mm,footskip=15mm]{geometry}
\usepackage{graphicx}
\usepackage{fancyhdr}
\usepackage{titlesec}
\usepackage[spanish]{babel}
\usepackage[utf8]{inputenc}
\usepackage{amsmath}
\usepackage{setspace}
\usepackage{svg}
\usepackage{hyperref}
\usepackage[backend=biber,style=numeric]{biblatex}
\addbibresource{references.bib}
\hypersetup{
    colorlinks=true,
    linkcolor=blue,      % color of internal links (sections, etc.)
    urlcolor=blue,       % color of external links
    pdftitle={Optimización energética de sistema híbrido con bomba de calor, suelo radiante, fotovoltaica y almacenamiento para vivienda},    % title
    pdfauthor={Luis D. Aranda Sánchez},     % author
    pdfkeywords={palabra1, palabra2, código1, etc.} % list of keywords
}

% Font change to Arial
\usepackage{helvet}
\renewcommand{\familydefault}{\sfdefault}

% Chapter titles in uppercase and larger font
\titleformat{\chapter}[hang]{\large\bfseries}{\thechapter.}{1em}{\MakeUppercase}
\titleformat{\section}[hang]{\bfseries}{\thesection.}{1em}{}
\titleformat{\subsection}[hang]{\bfseries}{\thesubsection.}{1em}{}

% Fancyhdr setup
\setlength{\headheight}{14.30174pt} % Adjust to recommended value, slightly larger for safety
\fancyhf{} % Clear all headers and footers
\fancyhead[LE]{\nouppercase{\leftmark}}
\fancyhead[RO]{Optimización energética para vivienda}
\fancyfoot[LE]{\thepage}
\fancyfoot[RE]{Escuela Técnica Superior de Ingenieros Industriales (UPM)}
\fancyfoot[LO]{Luis D. Aranda Sánchez}
\fancyfoot[RO]{\thepage}
\renewcommand{\headrulewidth}{0.4pt}
\renewcommand{\footrulewidth}{0.4pt}

\fancypagestyle{myfancy}{
    \fancyhf{} % Clear all headers and footers
    \fancyhead[LE]{\nouppercase{\leftmark}}
    \fancyhead[RO]{Optimización energética para vivienda}
    \fancyfoot[LE]{\thepage}
    \fancyfoot[RE]{Escuela Técnica Superior de Ingenieros Industriales (UPM)}
    \fancyfoot[LO]{Luis D. Aranda Sánchez}
    \fancyfoot[RO]{\thepage}
    \renewcommand{\headrulewidth}{0.4pt}
    \renewcommand{\footrulewidth}{0.4pt}
}

\fancypagestyle{simple}{
    \fancyhf{} % Clear all headers and footers
    \renewcommand{\headrulewidth}{0pt}
    \renewcommand{\footrulewidth}{0pt}
}

% Line spacing
\setstretch{1.2}

% Document starts here
\begin{document}

% Portada
\begin{titlepage}
    \centering
    {\scshape\LARGE Universidad Politécnica de Madrid \par}
    \vspace{1cm}
    {\scshape\Large Escuela Técnica Superior de Ingenieros Industriales\par}
    \vspace{1.5cm}
    {\huge\bfseries Optimización energética de sistema híbrido con bomba de calor, suelo radiante, fotovoltaica y almacenamiento para vivienda \par}
    \vspace{1.5cm}
    {\Large\bfseries Trabajo de Fin de Máster\par}
    \vspace{0.5cm}
    {\large Máster Universitario en Ingeniería de la Energía \par}
    \vspace{2cm}
    {\Large Luis D. Aranda Sánchez\par}
    \vfill
    Director: Javier Rodríguez Martín
    \vfill
    {\large Septiembre 6, 2024\par}
\end{titlepage}

% Resumen (máximo de 5 páginas, incluyendo al final Palabras clave)
\clearpage
\pagestyle{simple}
% \newpage
\chapter*{Resumen}
\addcontentsline{toc}{chapter}{Resumen}
\input{capitulos/resumen/main.tex}

% Índice (paginado)
\clearpage
\pagestyle{simple}
% \newpage
\tableofcontents

% Introducción (donde se incluya los antecedentes y justificación)
\clearpage
\pagestyle{myfancy}
\newpage
\chapter{Introducción}
\input{capitulos/introduccion/main.tex}

% Objetivos
\chapter{Objetivos}
\input{capitulos/objetivos/main.tex}

% Metodología
\chapter{Metodología}
\input{capitulos/metodologia/main.tex}

% Resultados y discusión (incluyendo la valoración de impactos y de aspectos de responsabilidad legal, ética y profesional relacionados con el trabajo)
\chapter{Resultados y Discusión}
\input{capitulos/resultados_discusion/main.tex}

% Conclusiones
\chapter{Conclusiones}
\input{capitulos/conclusiones/main.tex}

% Planificación temporal y presupuesto
\chapter{Planificación Temporal y Presupuesto}
\input{capitulos/planificacion_presupuesto/main.tex}

% Bibliografía
\newpage
\addcontentsline{toc}{chapter}{Bibliografía}
\printbibliography

\end{document}


% Planificación temporal y presupuesto
\chapter{Planificación Temporal y Presupuesto}
\documentclass[a4paper,11pt,twoside]{report}
\usepackage[left=25mm,right=25mm,top=25mm,bottom=25mm,includehead,includefoot,headsep=15mm,footskip=15mm]{geometry}
\usepackage{graphicx}
\usepackage{fancyhdr}
\usepackage{titlesec}
\usepackage[spanish]{babel}
\usepackage[utf8]{inputenc}
\usepackage{amsmath}
\usepackage{setspace}
\usepackage{svg}
\usepackage{hyperref}
\usepackage[backend=biber,style=numeric]{biblatex}
\addbibresource{references.bib}
\hypersetup{
    colorlinks=true,
    linkcolor=blue,      % color of internal links (sections, etc.)
    urlcolor=blue,       % color of external links
    pdftitle={Optimización energética de sistema híbrido con bomba de calor, suelo radiante, fotovoltaica y almacenamiento para vivienda},    % title
    pdfauthor={Luis D. Aranda Sánchez},     % author
    pdfkeywords={palabra1, palabra2, código1, etc.} % list of keywords
}

% Font change to Arial
\usepackage{helvet}
\renewcommand{\familydefault}{\sfdefault}

% Chapter titles in uppercase and larger font
\titleformat{\chapter}[hang]{\large\bfseries}{\thechapter.}{1em}{\MakeUppercase}
\titleformat{\section}[hang]{\bfseries}{\thesection.}{1em}{}
\titleformat{\subsection}[hang]{\bfseries}{\thesubsection.}{1em}{}

% Fancyhdr setup
\setlength{\headheight}{14.30174pt} % Adjust to recommended value, slightly larger for safety
\fancyhf{} % Clear all headers and footers
\fancyhead[LE]{\nouppercase{\leftmark}}
\fancyhead[RO]{Optimización energética para vivienda}
\fancyfoot[LE]{\thepage}
\fancyfoot[RE]{Escuela Técnica Superior de Ingenieros Industriales (UPM)}
\fancyfoot[LO]{Luis D. Aranda Sánchez}
\fancyfoot[RO]{\thepage}
\renewcommand{\headrulewidth}{0.4pt}
\renewcommand{\footrulewidth}{0.4pt}

\fancypagestyle{myfancy}{
    \fancyhf{} % Clear all headers and footers
    \fancyhead[LE]{\nouppercase{\leftmark}}
    \fancyhead[RO]{Optimización energética para vivienda}
    \fancyfoot[LE]{\thepage}
    \fancyfoot[RE]{Escuela Técnica Superior de Ingenieros Industriales (UPM)}
    \fancyfoot[LO]{Luis D. Aranda Sánchez}
    \fancyfoot[RO]{\thepage}
    \renewcommand{\headrulewidth}{0.4pt}
    \renewcommand{\footrulewidth}{0.4pt}
}

\fancypagestyle{simple}{
    \fancyhf{} % Clear all headers and footers
    \renewcommand{\headrulewidth}{0pt}
    \renewcommand{\footrulewidth}{0pt}
}

% Line spacing
\setstretch{1.2}

% Document starts here
\begin{document}

% Portada
\begin{titlepage}
    \centering
    {\scshape\LARGE Universidad Politécnica de Madrid \par}
    \vspace{1cm}
    {\scshape\Large Escuela Técnica Superior de Ingenieros Industriales\par}
    \vspace{1.5cm}
    {\huge\bfseries Optimización energética de sistema híbrido con bomba de calor, suelo radiante, fotovoltaica y almacenamiento para vivienda \par}
    \vspace{1.5cm}
    {\Large\bfseries Trabajo de Fin de Máster\par}
    \vspace{0.5cm}
    {\large Máster Universitario en Ingeniería de la Energía \par}
    \vspace{2cm}
    {\Large Luis D. Aranda Sánchez\par}
    \vfill
    Director: Javier Rodríguez Martín
    \vfill
    {\large Septiembre 6, 2024\par}
\end{titlepage}

% Resumen (máximo de 5 páginas, incluyendo al final Palabras clave)
\clearpage
\pagestyle{simple}
% \newpage
\chapter*{Resumen}
\addcontentsline{toc}{chapter}{Resumen}
\input{capitulos/resumen/main.tex}

% Índice (paginado)
\clearpage
\pagestyle{simple}
% \newpage
\tableofcontents

% Introducción (donde se incluya los antecedentes y justificación)
\clearpage
\pagestyle{myfancy}
\newpage
\chapter{Introducción}
\input{capitulos/introduccion/main.tex}

% Objetivos
\chapter{Objetivos}
\input{capitulos/objetivos/main.tex}

% Metodología
\chapter{Metodología}
\input{capitulos/metodologia/main.tex}

% Resultados y discusión (incluyendo la valoración de impactos y de aspectos de responsabilidad legal, ética y profesional relacionados con el trabajo)
\chapter{Resultados y Discusión}
\input{capitulos/resultados_discusion/main.tex}

% Conclusiones
\chapter{Conclusiones}
\input{capitulos/conclusiones/main.tex}

% Planificación temporal y presupuesto
\chapter{Planificación Temporal y Presupuesto}
\input{capitulos/planificacion_presupuesto/main.tex}

% Bibliografía
\newpage
\addcontentsline{toc}{chapter}{Bibliografía}
\printbibliography

\end{document}


% Bibliografía
\newpage
\addcontentsline{toc}{chapter}{Bibliografía}
\printbibliography

\end{document}


% Planificación temporal y presupuesto
\chapter{Planificación Temporal y Presupuesto}
\documentclass[a4paper,11pt,twoside]{report}
\usepackage[left=25mm,right=25mm,top=25mm,bottom=25mm,includehead,includefoot,headsep=15mm,footskip=15mm]{geometry}
\usepackage{graphicx}
\usepackage{fancyhdr}
\usepackage{titlesec}
\usepackage[spanish]{babel}
\usepackage[utf8]{inputenc}
\usepackage{amsmath}
\usepackage{setspace}
\usepackage{svg}
\usepackage{hyperref}
\usepackage[backend=biber,style=numeric]{biblatex}
\addbibresource{references.bib}
\hypersetup{
    colorlinks=true,
    linkcolor=blue,      % color of internal links (sections, etc.)
    urlcolor=blue,       % color of external links
    pdftitle={Optimización energética de sistema híbrido con bomba de calor, suelo radiante, fotovoltaica y almacenamiento para vivienda},    % title
    pdfauthor={Luis D. Aranda Sánchez},     % author
    pdfkeywords={palabra1, palabra2, código1, etc.} % list of keywords
}

% Font change to Arial
\usepackage{helvet}
\renewcommand{\familydefault}{\sfdefault}

% Chapter titles in uppercase and larger font
\titleformat{\chapter}[hang]{\large\bfseries}{\thechapter.}{1em}{\MakeUppercase}
\titleformat{\section}[hang]{\bfseries}{\thesection.}{1em}{}
\titleformat{\subsection}[hang]{\bfseries}{\thesubsection.}{1em}{}

% Fancyhdr setup
\setlength{\headheight}{14.30174pt} % Adjust to recommended value, slightly larger for safety
\fancyhf{} % Clear all headers and footers
\fancyhead[LE]{\nouppercase{\leftmark}}
\fancyhead[RO]{Optimización energética para vivienda}
\fancyfoot[LE]{\thepage}
\fancyfoot[RE]{Escuela Técnica Superior de Ingenieros Industriales (UPM)}
\fancyfoot[LO]{Luis D. Aranda Sánchez}
\fancyfoot[RO]{\thepage}
\renewcommand{\headrulewidth}{0.4pt}
\renewcommand{\footrulewidth}{0.4pt}

\fancypagestyle{myfancy}{
    \fancyhf{} % Clear all headers and footers
    \fancyhead[LE]{\nouppercase{\leftmark}}
    \fancyhead[RO]{Optimización energética para vivienda}
    \fancyfoot[LE]{\thepage}
    \fancyfoot[RE]{Escuela Técnica Superior de Ingenieros Industriales (UPM)}
    \fancyfoot[LO]{Luis D. Aranda Sánchez}
    \fancyfoot[RO]{\thepage}
    \renewcommand{\headrulewidth}{0.4pt}
    \renewcommand{\footrulewidth}{0.4pt}
}

\fancypagestyle{simple}{
    \fancyhf{} % Clear all headers and footers
    \renewcommand{\headrulewidth}{0pt}
    \renewcommand{\footrulewidth}{0pt}
}

% Line spacing
\setstretch{1.2}

% Document starts here
\begin{document}

% Portada
\begin{titlepage}
    \centering
    {\scshape\LARGE Universidad Politécnica de Madrid \par}
    \vspace{1cm}
    {\scshape\Large Escuela Técnica Superior de Ingenieros Industriales\par}
    \vspace{1.5cm}
    {\huge\bfseries Optimización energética de sistema híbrido con bomba de calor, suelo radiante, fotovoltaica y almacenamiento para vivienda \par}
    \vspace{1.5cm}
    {\Large\bfseries Trabajo de Fin de Máster\par}
    \vspace{0.5cm}
    {\large Máster Universitario en Ingeniería de la Energía \par}
    \vspace{2cm}
    {\Large Luis D. Aranda Sánchez\par}
    \vfill
    Director: Javier Rodríguez Martín
    \vfill
    {\large Septiembre 6, 2024\par}
\end{titlepage}

% Resumen (máximo de 5 páginas, incluyendo al final Palabras clave)
\clearpage
\pagestyle{simple}
% \newpage
\chapter*{Resumen}
\addcontentsline{toc}{chapter}{Resumen}
\documentclass[a4paper,11pt,twoside]{report}
\usepackage[left=25mm,right=25mm,top=25mm,bottom=25mm,includehead,includefoot,headsep=15mm,footskip=15mm]{geometry}
\usepackage{graphicx}
\usepackage{fancyhdr}
\usepackage{titlesec}
\usepackage[spanish]{babel}
\usepackage[utf8]{inputenc}
\usepackage{amsmath}
\usepackage{setspace}
\usepackage{svg}
\usepackage{hyperref}
\usepackage[backend=biber,style=numeric]{biblatex}
\addbibresource{references.bib}
\hypersetup{
    colorlinks=true,
    linkcolor=blue,      % color of internal links (sections, etc.)
    urlcolor=blue,       % color of external links
    pdftitle={Optimización energética de sistema híbrido con bomba de calor, suelo radiante, fotovoltaica y almacenamiento para vivienda},    % title
    pdfauthor={Luis D. Aranda Sánchez},     % author
    pdfkeywords={palabra1, palabra2, código1, etc.} % list of keywords
}

% Font change to Arial
\usepackage{helvet}
\renewcommand{\familydefault}{\sfdefault}

% Chapter titles in uppercase and larger font
\titleformat{\chapter}[hang]{\large\bfseries}{\thechapter.}{1em}{\MakeUppercase}
\titleformat{\section}[hang]{\bfseries}{\thesection.}{1em}{}
\titleformat{\subsection}[hang]{\bfseries}{\thesubsection.}{1em}{}

% Fancyhdr setup
\setlength{\headheight}{14.30174pt} % Adjust to recommended value, slightly larger for safety
\fancyhf{} % Clear all headers and footers
\fancyhead[LE]{\nouppercase{\leftmark}}
\fancyhead[RO]{Optimización energética para vivienda}
\fancyfoot[LE]{\thepage}
\fancyfoot[RE]{Escuela Técnica Superior de Ingenieros Industriales (UPM)}
\fancyfoot[LO]{Luis D. Aranda Sánchez}
\fancyfoot[RO]{\thepage}
\renewcommand{\headrulewidth}{0.4pt}
\renewcommand{\footrulewidth}{0.4pt}

\fancypagestyle{myfancy}{
    \fancyhf{} % Clear all headers and footers
    \fancyhead[LE]{\nouppercase{\leftmark}}
    \fancyhead[RO]{Optimización energética para vivienda}
    \fancyfoot[LE]{\thepage}
    \fancyfoot[RE]{Escuela Técnica Superior de Ingenieros Industriales (UPM)}
    \fancyfoot[LO]{Luis D. Aranda Sánchez}
    \fancyfoot[RO]{\thepage}
    \renewcommand{\headrulewidth}{0.4pt}
    \renewcommand{\footrulewidth}{0.4pt}
}

\fancypagestyle{simple}{
    \fancyhf{} % Clear all headers and footers
    \renewcommand{\headrulewidth}{0pt}
    \renewcommand{\footrulewidth}{0pt}
}

% Line spacing
\setstretch{1.2}

% Document starts here
\begin{document}

% Portada
\begin{titlepage}
    \centering
    {\scshape\LARGE Universidad Politécnica de Madrid \par}
    \vspace{1cm}
    {\scshape\Large Escuela Técnica Superior de Ingenieros Industriales\par}
    \vspace{1.5cm}
    {\huge\bfseries Optimización energética de sistema híbrido con bomba de calor, suelo radiante, fotovoltaica y almacenamiento para vivienda \par}
    \vspace{1.5cm}
    {\Large\bfseries Trabajo de Fin de Máster\par}
    \vspace{0.5cm}
    {\large Máster Universitario en Ingeniería de la Energía \par}
    \vspace{2cm}
    {\Large Luis D. Aranda Sánchez\par}
    \vfill
    Director: Javier Rodríguez Martín
    \vfill
    {\large Septiembre 6, 2024\par}
\end{titlepage}

% Resumen (máximo de 5 páginas, incluyendo al final Palabras clave)
\clearpage
\pagestyle{simple}
% \newpage
\chapter*{Resumen}
\addcontentsline{toc}{chapter}{Resumen}
\input{capitulos/resumen/main.tex}

% Índice (paginado)
\clearpage
\pagestyle{simple}
% \newpage
\tableofcontents

% Introducción (donde se incluya los antecedentes y justificación)
\clearpage
\pagestyle{myfancy}
\newpage
\chapter{Introducción}
\input{capitulos/introduccion/main.tex}

% Objetivos
\chapter{Objetivos}
\input{capitulos/objetivos/main.tex}

% Metodología
\chapter{Metodología}
\input{capitulos/metodologia/main.tex}

% Resultados y discusión (incluyendo la valoración de impactos y de aspectos de responsabilidad legal, ética y profesional relacionados con el trabajo)
\chapter{Resultados y Discusión}
\input{capitulos/resultados_discusion/main.tex}

% Conclusiones
\chapter{Conclusiones}
\input{capitulos/conclusiones/main.tex}

% Planificación temporal y presupuesto
\chapter{Planificación Temporal y Presupuesto}
\input{capitulos/planificacion_presupuesto/main.tex}

% Bibliografía
\newpage
\addcontentsline{toc}{chapter}{Bibliografía}
\printbibliography

\end{document}


% Índice (paginado)
\clearpage
\pagestyle{simple}
% \newpage
\tableofcontents

% Introducción (donde se incluya los antecedentes y justificación)
\clearpage
\pagestyle{myfancy}
\newpage
\chapter{Introducción}
\documentclass[a4paper,11pt,twoside]{report}
\usepackage[left=25mm,right=25mm,top=25mm,bottom=25mm,includehead,includefoot,headsep=15mm,footskip=15mm]{geometry}
\usepackage{graphicx}
\usepackage{fancyhdr}
\usepackage{titlesec}
\usepackage[spanish]{babel}
\usepackage[utf8]{inputenc}
\usepackage{amsmath}
\usepackage{setspace}
\usepackage{svg}
\usepackage{hyperref}
\usepackage[backend=biber,style=numeric]{biblatex}
\addbibresource{references.bib}
\hypersetup{
    colorlinks=true,
    linkcolor=blue,      % color of internal links (sections, etc.)
    urlcolor=blue,       % color of external links
    pdftitle={Optimización energética de sistema híbrido con bomba de calor, suelo radiante, fotovoltaica y almacenamiento para vivienda},    % title
    pdfauthor={Luis D. Aranda Sánchez},     % author
    pdfkeywords={palabra1, palabra2, código1, etc.} % list of keywords
}

% Font change to Arial
\usepackage{helvet}
\renewcommand{\familydefault}{\sfdefault}

% Chapter titles in uppercase and larger font
\titleformat{\chapter}[hang]{\large\bfseries}{\thechapter.}{1em}{\MakeUppercase}
\titleformat{\section}[hang]{\bfseries}{\thesection.}{1em}{}
\titleformat{\subsection}[hang]{\bfseries}{\thesubsection.}{1em}{}

% Fancyhdr setup
\setlength{\headheight}{14.30174pt} % Adjust to recommended value, slightly larger for safety
\fancyhf{} % Clear all headers and footers
\fancyhead[LE]{\nouppercase{\leftmark}}
\fancyhead[RO]{Optimización energética para vivienda}
\fancyfoot[LE]{\thepage}
\fancyfoot[RE]{Escuela Técnica Superior de Ingenieros Industriales (UPM)}
\fancyfoot[LO]{Luis D. Aranda Sánchez}
\fancyfoot[RO]{\thepage}
\renewcommand{\headrulewidth}{0.4pt}
\renewcommand{\footrulewidth}{0.4pt}

\fancypagestyle{myfancy}{
    \fancyhf{} % Clear all headers and footers
    \fancyhead[LE]{\nouppercase{\leftmark}}
    \fancyhead[RO]{Optimización energética para vivienda}
    \fancyfoot[LE]{\thepage}
    \fancyfoot[RE]{Escuela Técnica Superior de Ingenieros Industriales (UPM)}
    \fancyfoot[LO]{Luis D. Aranda Sánchez}
    \fancyfoot[RO]{\thepage}
    \renewcommand{\headrulewidth}{0.4pt}
    \renewcommand{\footrulewidth}{0.4pt}
}

\fancypagestyle{simple}{
    \fancyhf{} % Clear all headers and footers
    \renewcommand{\headrulewidth}{0pt}
    \renewcommand{\footrulewidth}{0pt}
}

% Line spacing
\setstretch{1.2}

% Document starts here
\begin{document}

% Portada
\begin{titlepage}
    \centering
    {\scshape\LARGE Universidad Politécnica de Madrid \par}
    \vspace{1cm}
    {\scshape\Large Escuela Técnica Superior de Ingenieros Industriales\par}
    \vspace{1.5cm}
    {\huge\bfseries Optimización energética de sistema híbrido con bomba de calor, suelo radiante, fotovoltaica y almacenamiento para vivienda \par}
    \vspace{1.5cm}
    {\Large\bfseries Trabajo de Fin de Máster\par}
    \vspace{0.5cm}
    {\large Máster Universitario en Ingeniería de la Energía \par}
    \vspace{2cm}
    {\Large Luis D. Aranda Sánchez\par}
    \vfill
    Director: Javier Rodríguez Martín
    \vfill
    {\large Septiembre 6, 2024\par}
\end{titlepage}

% Resumen (máximo de 5 páginas, incluyendo al final Palabras clave)
\clearpage
\pagestyle{simple}
% \newpage
\chapter*{Resumen}
\addcontentsline{toc}{chapter}{Resumen}
\input{capitulos/resumen/main.tex}

% Índice (paginado)
\clearpage
\pagestyle{simple}
% \newpage
\tableofcontents

% Introducción (donde se incluya los antecedentes y justificación)
\clearpage
\pagestyle{myfancy}
\newpage
\chapter{Introducción}
\input{capitulos/introduccion/main.tex}

% Objetivos
\chapter{Objetivos}
\input{capitulos/objetivos/main.tex}

% Metodología
\chapter{Metodología}
\input{capitulos/metodologia/main.tex}

% Resultados y discusión (incluyendo la valoración de impactos y de aspectos de responsabilidad legal, ética y profesional relacionados con el trabajo)
\chapter{Resultados y Discusión}
\input{capitulos/resultados_discusion/main.tex}

% Conclusiones
\chapter{Conclusiones}
\input{capitulos/conclusiones/main.tex}

% Planificación temporal y presupuesto
\chapter{Planificación Temporal y Presupuesto}
\input{capitulos/planificacion_presupuesto/main.tex}

% Bibliografía
\newpage
\addcontentsline{toc}{chapter}{Bibliografía}
\printbibliography

\end{document}


% Objetivos
\chapter{Objetivos}
\documentclass[a4paper,11pt,twoside]{report}
\usepackage[left=25mm,right=25mm,top=25mm,bottom=25mm,includehead,includefoot,headsep=15mm,footskip=15mm]{geometry}
\usepackage{graphicx}
\usepackage{fancyhdr}
\usepackage{titlesec}
\usepackage[spanish]{babel}
\usepackage[utf8]{inputenc}
\usepackage{amsmath}
\usepackage{setspace}
\usepackage{svg}
\usepackage{hyperref}
\usepackage[backend=biber,style=numeric]{biblatex}
\addbibresource{references.bib}
\hypersetup{
    colorlinks=true,
    linkcolor=blue,      % color of internal links (sections, etc.)
    urlcolor=blue,       % color of external links
    pdftitle={Optimización energética de sistema híbrido con bomba de calor, suelo radiante, fotovoltaica y almacenamiento para vivienda},    % title
    pdfauthor={Luis D. Aranda Sánchez},     % author
    pdfkeywords={palabra1, palabra2, código1, etc.} % list of keywords
}

% Font change to Arial
\usepackage{helvet}
\renewcommand{\familydefault}{\sfdefault}

% Chapter titles in uppercase and larger font
\titleformat{\chapter}[hang]{\large\bfseries}{\thechapter.}{1em}{\MakeUppercase}
\titleformat{\section}[hang]{\bfseries}{\thesection.}{1em}{}
\titleformat{\subsection}[hang]{\bfseries}{\thesubsection.}{1em}{}

% Fancyhdr setup
\setlength{\headheight}{14.30174pt} % Adjust to recommended value, slightly larger for safety
\fancyhf{} % Clear all headers and footers
\fancyhead[LE]{\nouppercase{\leftmark}}
\fancyhead[RO]{Optimización energética para vivienda}
\fancyfoot[LE]{\thepage}
\fancyfoot[RE]{Escuela Técnica Superior de Ingenieros Industriales (UPM)}
\fancyfoot[LO]{Luis D. Aranda Sánchez}
\fancyfoot[RO]{\thepage}
\renewcommand{\headrulewidth}{0.4pt}
\renewcommand{\footrulewidth}{0.4pt}

\fancypagestyle{myfancy}{
    \fancyhf{} % Clear all headers and footers
    \fancyhead[LE]{\nouppercase{\leftmark}}
    \fancyhead[RO]{Optimización energética para vivienda}
    \fancyfoot[LE]{\thepage}
    \fancyfoot[RE]{Escuela Técnica Superior de Ingenieros Industriales (UPM)}
    \fancyfoot[LO]{Luis D. Aranda Sánchez}
    \fancyfoot[RO]{\thepage}
    \renewcommand{\headrulewidth}{0.4pt}
    \renewcommand{\footrulewidth}{0.4pt}
}

\fancypagestyle{simple}{
    \fancyhf{} % Clear all headers and footers
    \renewcommand{\headrulewidth}{0pt}
    \renewcommand{\footrulewidth}{0pt}
}

% Line spacing
\setstretch{1.2}

% Document starts here
\begin{document}

% Portada
\begin{titlepage}
    \centering
    {\scshape\LARGE Universidad Politécnica de Madrid \par}
    \vspace{1cm}
    {\scshape\Large Escuela Técnica Superior de Ingenieros Industriales\par}
    \vspace{1.5cm}
    {\huge\bfseries Optimización energética de sistema híbrido con bomba de calor, suelo radiante, fotovoltaica y almacenamiento para vivienda \par}
    \vspace{1.5cm}
    {\Large\bfseries Trabajo de Fin de Máster\par}
    \vspace{0.5cm}
    {\large Máster Universitario en Ingeniería de la Energía \par}
    \vspace{2cm}
    {\Large Luis D. Aranda Sánchez\par}
    \vfill
    Director: Javier Rodríguez Martín
    \vfill
    {\large Septiembre 6, 2024\par}
\end{titlepage}

% Resumen (máximo de 5 páginas, incluyendo al final Palabras clave)
\clearpage
\pagestyle{simple}
% \newpage
\chapter*{Resumen}
\addcontentsline{toc}{chapter}{Resumen}
\input{capitulos/resumen/main.tex}

% Índice (paginado)
\clearpage
\pagestyle{simple}
% \newpage
\tableofcontents

% Introducción (donde se incluya los antecedentes y justificación)
\clearpage
\pagestyle{myfancy}
\newpage
\chapter{Introducción}
\input{capitulos/introduccion/main.tex}

% Objetivos
\chapter{Objetivos}
\input{capitulos/objetivos/main.tex}

% Metodología
\chapter{Metodología}
\input{capitulos/metodologia/main.tex}

% Resultados y discusión (incluyendo la valoración de impactos y de aspectos de responsabilidad legal, ética y profesional relacionados con el trabajo)
\chapter{Resultados y Discusión}
\input{capitulos/resultados_discusion/main.tex}

% Conclusiones
\chapter{Conclusiones}
\input{capitulos/conclusiones/main.tex}

% Planificación temporal y presupuesto
\chapter{Planificación Temporal y Presupuesto}
\input{capitulos/planificacion_presupuesto/main.tex}

% Bibliografía
\newpage
\addcontentsline{toc}{chapter}{Bibliografía}
\printbibliography

\end{document}


% Metodología
\chapter{Metodología}
\documentclass[a4paper,11pt,twoside]{report}
\usepackage[left=25mm,right=25mm,top=25mm,bottom=25mm,includehead,includefoot,headsep=15mm,footskip=15mm]{geometry}
\usepackage{graphicx}
\usepackage{fancyhdr}
\usepackage{titlesec}
\usepackage[spanish]{babel}
\usepackage[utf8]{inputenc}
\usepackage{amsmath}
\usepackage{setspace}
\usepackage{svg}
\usepackage{hyperref}
\usepackage[backend=biber,style=numeric]{biblatex}
\addbibresource{references.bib}
\hypersetup{
    colorlinks=true,
    linkcolor=blue,      % color of internal links (sections, etc.)
    urlcolor=blue,       % color of external links
    pdftitle={Optimización energética de sistema híbrido con bomba de calor, suelo radiante, fotovoltaica y almacenamiento para vivienda},    % title
    pdfauthor={Luis D. Aranda Sánchez},     % author
    pdfkeywords={palabra1, palabra2, código1, etc.} % list of keywords
}

% Font change to Arial
\usepackage{helvet}
\renewcommand{\familydefault}{\sfdefault}

% Chapter titles in uppercase and larger font
\titleformat{\chapter}[hang]{\large\bfseries}{\thechapter.}{1em}{\MakeUppercase}
\titleformat{\section}[hang]{\bfseries}{\thesection.}{1em}{}
\titleformat{\subsection}[hang]{\bfseries}{\thesubsection.}{1em}{}

% Fancyhdr setup
\setlength{\headheight}{14.30174pt} % Adjust to recommended value, slightly larger for safety
\fancyhf{} % Clear all headers and footers
\fancyhead[LE]{\nouppercase{\leftmark}}
\fancyhead[RO]{Optimización energética para vivienda}
\fancyfoot[LE]{\thepage}
\fancyfoot[RE]{Escuela Técnica Superior de Ingenieros Industriales (UPM)}
\fancyfoot[LO]{Luis D. Aranda Sánchez}
\fancyfoot[RO]{\thepage}
\renewcommand{\headrulewidth}{0.4pt}
\renewcommand{\footrulewidth}{0.4pt}

\fancypagestyle{myfancy}{
    \fancyhf{} % Clear all headers and footers
    \fancyhead[LE]{\nouppercase{\leftmark}}
    \fancyhead[RO]{Optimización energética para vivienda}
    \fancyfoot[LE]{\thepage}
    \fancyfoot[RE]{Escuela Técnica Superior de Ingenieros Industriales (UPM)}
    \fancyfoot[LO]{Luis D. Aranda Sánchez}
    \fancyfoot[RO]{\thepage}
    \renewcommand{\headrulewidth}{0.4pt}
    \renewcommand{\footrulewidth}{0.4pt}
}

\fancypagestyle{simple}{
    \fancyhf{} % Clear all headers and footers
    \renewcommand{\headrulewidth}{0pt}
    \renewcommand{\footrulewidth}{0pt}
}

% Line spacing
\setstretch{1.2}

% Document starts here
\begin{document}

% Portada
\begin{titlepage}
    \centering
    {\scshape\LARGE Universidad Politécnica de Madrid \par}
    \vspace{1cm}
    {\scshape\Large Escuela Técnica Superior de Ingenieros Industriales\par}
    \vspace{1.5cm}
    {\huge\bfseries Optimización energética de sistema híbrido con bomba de calor, suelo radiante, fotovoltaica y almacenamiento para vivienda \par}
    \vspace{1.5cm}
    {\Large\bfseries Trabajo de Fin de Máster\par}
    \vspace{0.5cm}
    {\large Máster Universitario en Ingeniería de la Energía \par}
    \vspace{2cm}
    {\Large Luis D. Aranda Sánchez\par}
    \vfill
    Director: Javier Rodríguez Martín
    \vfill
    {\large Septiembre 6, 2024\par}
\end{titlepage}

% Resumen (máximo de 5 páginas, incluyendo al final Palabras clave)
\clearpage
\pagestyle{simple}
% \newpage
\chapter*{Resumen}
\addcontentsline{toc}{chapter}{Resumen}
\input{capitulos/resumen/main.tex}

% Índice (paginado)
\clearpage
\pagestyle{simple}
% \newpage
\tableofcontents

% Introducción (donde se incluya los antecedentes y justificación)
\clearpage
\pagestyle{myfancy}
\newpage
\chapter{Introducción}
\input{capitulos/introduccion/main.tex}

% Objetivos
\chapter{Objetivos}
\input{capitulos/objetivos/main.tex}

% Metodología
\chapter{Metodología}
\input{capitulos/metodologia/main.tex}

% Resultados y discusión (incluyendo la valoración de impactos y de aspectos de responsabilidad legal, ética y profesional relacionados con el trabajo)
\chapter{Resultados y Discusión}
\input{capitulos/resultados_discusion/main.tex}

% Conclusiones
\chapter{Conclusiones}
\input{capitulos/conclusiones/main.tex}

% Planificación temporal y presupuesto
\chapter{Planificación Temporal y Presupuesto}
\input{capitulos/planificacion_presupuesto/main.tex}

% Bibliografía
\newpage
\addcontentsline{toc}{chapter}{Bibliografía}
\printbibliography

\end{document}


% Resultados y discusión (incluyendo la valoración de impactos y de aspectos de responsabilidad legal, ética y profesional relacionados con el trabajo)
\chapter{Resultados y Discusión}
\documentclass[a4paper,11pt,twoside]{report}
\usepackage[left=25mm,right=25mm,top=25mm,bottom=25mm,includehead,includefoot,headsep=15mm,footskip=15mm]{geometry}
\usepackage{graphicx}
\usepackage{fancyhdr}
\usepackage{titlesec}
\usepackage[spanish]{babel}
\usepackage[utf8]{inputenc}
\usepackage{amsmath}
\usepackage{setspace}
\usepackage{svg}
\usepackage{hyperref}
\usepackage[backend=biber,style=numeric]{biblatex}
\addbibresource{references.bib}
\hypersetup{
    colorlinks=true,
    linkcolor=blue,      % color of internal links (sections, etc.)
    urlcolor=blue,       % color of external links
    pdftitle={Optimización energética de sistema híbrido con bomba de calor, suelo radiante, fotovoltaica y almacenamiento para vivienda},    % title
    pdfauthor={Luis D. Aranda Sánchez},     % author
    pdfkeywords={palabra1, palabra2, código1, etc.} % list of keywords
}

% Font change to Arial
\usepackage{helvet}
\renewcommand{\familydefault}{\sfdefault}

% Chapter titles in uppercase and larger font
\titleformat{\chapter}[hang]{\large\bfseries}{\thechapter.}{1em}{\MakeUppercase}
\titleformat{\section}[hang]{\bfseries}{\thesection.}{1em}{}
\titleformat{\subsection}[hang]{\bfseries}{\thesubsection.}{1em}{}

% Fancyhdr setup
\setlength{\headheight}{14.30174pt} % Adjust to recommended value, slightly larger for safety
\fancyhf{} % Clear all headers and footers
\fancyhead[LE]{\nouppercase{\leftmark}}
\fancyhead[RO]{Optimización energética para vivienda}
\fancyfoot[LE]{\thepage}
\fancyfoot[RE]{Escuela Técnica Superior de Ingenieros Industriales (UPM)}
\fancyfoot[LO]{Luis D. Aranda Sánchez}
\fancyfoot[RO]{\thepage}
\renewcommand{\headrulewidth}{0.4pt}
\renewcommand{\footrulewidth}{0.4pt}

\fancypagestyle{myfancy}{
    \fancyhf{} % Clear all headers and footers
    \fancyhead[LE]{\nouppercase{\leftmark}}
    \fancyhead[RO]{Optimización energética para vivienda}
    \fancyfoot[LE]{\thepage}
    \fancyfoot[RE]{Escuela Técnica Superior de Ingenieros Industriales (UPM)}
    \fancyfoot[LO]{Luis D. Aranda Sánchez}
    \fancyfoot[RO]{\thepage}
    \renewcommand{\headrulewidth}{0.4pt}
    \renewcommand{\footrulewidth}{0.4pt}
}

\fancypagestyle{simple}{
    \fancyhf{} % Clear all headers and footers
    \renewcommand{\headrulewidth}{0pt}
    \renewcommand{\footrulewidth}{0pt}
}

% Line spacing
\setstretch{1.2}

% Document starts here
\begin{document}

% Portada
\begin{titlepage}
    \centering
    {\scshape\LARGE Universidad Politécnica de Madrid \par}
    \vspace{1cm}
    {\scshape\Large Escuela Técnica Superior de Ingenieros Industriales\par}
    \vspace{1.5cm}
    {\huge\bfseries Optimización energética de sistema híbrido con bomba de calor, suelo radiante, fotovoltaica y almacenamiento para vivienda \par}
    \vspace{1.5cm}
    {\Large\bfseries Trabajo de Fin de Máster\par}
    \vspace{0.5cm}
    {\large Máster Universitario en Ingeniería de la Energía \par}
    \vspace{2cm}
    {\Large Luis D. Aranda Sánchez\par}
    \vfill
    Director: Javier Rodríguez Martín
    \vfill
    {\large Septiembre 6, 2024\par}
\end{titlepage}

% Resumen (máximo de 5 páginas, incluyendo al final Palabras clave)
\clearpage
\pagestyle{simple}
% \newpage
\chapter*{Resumen}
\addcontentsline{toc}{chapter}{Resumen}
\input{capitulos/resumen/main.tex}

% Índice (paginado)
\clearpage
\pagestyle{simple}
% \newpage
\tableofcontents

% Introducción (donde se incluya los antecedentes y justificación)
\clearpage
\pagestyle{myfancy}
\newpage
\chapter{Introducción}
\input{capitulos/introduccion/main.tex}

% Objetivos
\chapter{Objetivos}
\input{capitulos/objetivos/main.tex}

% Metodología
\chapter{Metodología}
\input{capitulos/metodologia/main.tex}

% Resultados y discusión (incluyendo la valoración de impactos y de aspectos de responsabilidad legal, ética y profesional relacionados con el trabajo)
\chapter{Resultados y Discusión}
\input{capitulos/resultados_discusion/main.tex}

% Conclusiones
\chapter{Conclusiones}
\input{capitulos/conclusiones/main.tex}

% Planificación temporal y presupuesto
\chapter{Planificación Temporal y Presupuesto}
\input{capitulos/planificacion_presupuesto/main.tex}

% Bibliografía
\newpage
\addcontentsline{toc}{chapter}{Bibliografía}
\printbibliography

\end{document}


% Conclusiones
\chapter{Conclusiones}
\documentclass[a4paper,11pt,twoside]{report}
\usepackage[left=25mm,right=25mm,top=25mm,bottom=25mm,includehead,includefoot,headsep=15mm,footskip=15mm]{geometry}
\usepackage{graphicx}
\usepackage{fancyhdr}
\usepackage{titlesec}
\usepackage[spanish]{babel}
\usepackage[utf8]{inputenc}
\usepackage{amsmath}
\usepackage{setspace}
\usepackage{svg}
\usepackage{hyperref}
\usepackage[backend=biber,style=numeric]{biblatex}
\addbibresource{references.bib}
\hypersetup{
    colorlinks=true,
    linkcolor=blue,      % color of internal links (sections, etc.)
    urlcolor=blue,       % color of external links
    pdftitle={Optimización energética de sistema híbrido con bomba de calor, suelo radiante, fotovoltaica y almacenamiento para vivienda},    % title
    pdfauthor={Luis D. Aranda Sánchez},     % author
    pdfkeywords={palabra1, palabra2, código1, etc.} % list of keywords
}

% Font change to Arial
\usepackage{helvet}
\renewcommand{\familydefault}{\sfdefault}

% Chapter titles in uppercase and larger font
\titleformat{\chapter}[hang]{\large\bfseries}{\thechapter.}{1em}{\MakeUppercase}
\titleformat{\section}[hang]{\bfseries}{\thesection.}{1em}{}
\titleformat{\subsection}[hang]{\bfseries}{\thesubsection.}{1em}{}

% Fancyhdr setup
\setlength{\headheight}{14.30174pt} % Adjust to recommended value, slightly larger for safety
\fancyhf{} % Clear all headers and footers
\fancyhead[LE]{\nouppercase{\leftmark}}
\fancyhead[RO]{Optimización energética para vivienda}
\fancyfoot[LE]{\thepage}
\fancyfoot[RE]{Escuela Técnica Superior de Ingenieros Industriales (UPM)}
\fancyfoot[LO]{Luis D. Aranda Sánchez}
\fancyfoot[RO]{\thepage}
\renewcommand{\headrulewidth}{0.4pt}
\renewcommand{\footrulewidth}{0.4pt}

\fancypagestyle{myfancy}{
    \fancyhf{} % Clear all headers and footers
    \fancyhead[LE]{\nouppercase{\leftmark}}
    \fancyhead[RO]{Optimización energética para vivienda}
    \fancyfoot[LE]{\thepage}
    \fancyfoot[RE]{Escuela Técnica Superior de Ingenieros Industriales (UPM)}
    \fancyfoot[LO]{Luis D. Aranda Sánchez}
    \fancyfoot[RO]{\thepage}
    \renewcommand{\headrulewidth}{0.4pt}
    \renewcommand{\footrulewidth}{0.4pt}
}

\fancypagestyle{simple}{
    \fancyhf{} % Clear all headers and footers
    \renewcommand{\headrulewidth}{0pt}
    \renewcommand{\footrulewidth}{0pt}
}

% Line spacing
\setstretch{1.2}

% Document starts here
\begin{document}

% Portada
\begin{titlepage}
    \centering
    {\scshape\LARGE Universidad Politécnica de Madrid \par}
    \vspace{1cm}
    {\scshape\Large Escuela Técnica Superior de Ingenieros Industriales\par}
    \vspace{1.5cm}
    {\huge\bfseries Optimización energética de sistema híbrido con bomba de calor, suelo radiante, fotovoltaica y almacenamiento para vivienda \par}
    \vspace{1.5cm}
    {\Large\bfseries Trabajo de Fin de Máster\par}
    \vspace{0.5cm}
    {\large Máster Universitario en Ingeniería de la Energía \par}
    \vspace{2cm}
    {\Large Luis D. Aranda Sánchez\par}
    \vfill
    Director: Javier Rodríguez Martín
    \vfill
    {\large Septiembre 6, 2024\par}
\end{titlepage}

% Resumen (máximo de 5 páginas, incluyendo al final Palabras clave)
\clearpage
\pagestyle{simple}
% \newpage
\chapter*{Resumen}
\addcontentsline{toc}{chapter}{Resumen}
\input{capitulos/resumen/main.tex}

% Índice (paginado)
\clearpage
\pagestyle{simple}
% \newpage
\tableofcontents

% Introducción (donde se incluya los antecedentes y justificación)
\clearpage
\pagestyle{myfancy}
\newpage
\chapter{Introducción}
\input{capitulos/introduccion/main.tex}

% Objetivos
\chapter{Objetivos}
\input{capitulos/objetivos/main.tex}

% Metodología
\chapter{Metodología}
\input{capitulos/metodologia/main.tex}

% Resultados y discusión (incluyendo la valoración de impactos y de aspectos de responsabilidad legal, ética y profesional relacionados con el trabajo)
\chapter{Resultados y Discusión}
\input{capitulos/resultados_discusion/main.tex}

% Conclusiones
\chapter{Conclusiones}
\input{capitulos/conclusiones/main.tex}

% Planificación temporal y presupuesto
\chapter{Planificación Temporal y Presupuesto}
\input{capitulos/planificacion_presupuesto/main.tex}

% Bibliografía
\newpage
\addcontentsline{toc}{chapter}{Bibliografía}
\printbibliography

\end{document}


% Planificación temporal y presupuesto
\chapter{Planificación Temporal y Presupuesto}
\documentclass[a4paper,11pt,twoside]{report}
\usepackage[left=25mm,right=25mm,top=25mm,bottom=25mm,includehead,includefoot,headsep=15mm,footskip=15mm]{geometry}
\usepackage{graphicx}
\usepackage{fancyhdr}
\usepackage{titlesec}
\usepackage[spanish]{babel}
\usepackage[utf8]{inputenc}
\usepackage{amsmath}
\usepackage{setspace}
\usepackage{svg}
\usepackage{hyperref}
\usepackage[backend=biber,style=numeric]{biblatex}
\addbibresource{references.bib}
\hypersetup{
    colorlinks=true,
    linkcolor=blue,      % color of internal links (sections, etc.)
    urlcolor=blue,       % color of external links
    pdftitle={Optimización energética de sistema híbrido con bomba de calor, suelo radiante, fotovoltaica y almacenamiento para vivienda},    % title
    pdfauthor={Luis D. Aranda Sánchez},     % author
    pdfkeywords={palabra1, palabra2, código1, etc.} % list of keywords
}

% Font change to Arial
\usepackage{helvet}
\renewcommand{\familydefault}{\sfdefault}

% Chapter titles in uppercase and larger font
\titleformat{\chapter}[hang]{\large\bfseries}{\thechapter.}{1em}{\MakeUppercase}
\titleformat{\section}[hang]{\bfseries}{\thesection.}{1em}{}
\titleformat{\subsection}[hang]{\bfseries}{\thesubsection.}{1em}{}

% Fancyhdr setup
\setlength{\headheight}{14.30174pt} % Adjust to recommended value, slightly larger for safety
\fancyhf{} % Clear all headers and footers
\fancyhead[LE]{\nouppercase{\leftmark}}
\fancyhead[RO]{Optimización energética para vivienda}
\fancyfoot[LE]{\thepage}
\fancyfoot[RE]{Escuela Técnica Superior de Ingenieros Industriales (UPM)}
\fancyfoot[LO]{Luis D. Aranda Sánchez}
\fancyfoot[RO]{\thepage}
\renewcommand{\headrulewidth}{0.4pt}
\renewcommand{\footrulewidth}{0.4pt}

\fancypagestyle{myfancy}{
    \fancyhf{} % Clear all headers and footers
    \fancyhead[LE]{\nouppercase{\leftmark}}
    \fancyhead[RO]{Optimización energética para vivienda}
    \fancyfoot[LE]{\thepage}
    \fancyfoot[RE]{Escuela Técnica Superior de Ingenieros Industriales (UPM)}
    \fancyfoot[LO]{Luis D. Aranda Sánchez}
    \fancyfoot[RO]{\thepage}
    \renewcommand{\headrulewidth}{0.4pt}
    \renewcommand{\footrulewidth}{0.4pt}
}

\fancypagestyle{simple}{
    \fancyhf{} % Clear all headers and footers
    \renewcommand{\headrulewidth}{0pt}
    \renewcommand{\footrulewidth}{0pt}
}

% Line spacing
\setstretch{1.2}

% Document starts here
\begin{document}

% Portada
\begin{titlepage}
    \centering
    {\scshape\LARGE Universidad Politécnica de Madrid \par}
    \vspace{1cm}
    {\scshape\Large Escuela Técnica Superior de Ingenieros Industriales\par}
    \vspace{1.5cm}
    {\huge\bfseries Optimización energética de sistema híbrido con bomba de calor, suelo radiante, fotovoltaica y almacenamiento para vivienda \par}
    \vspace{1.5cm}
    {\Large\bfseries Trabajo de Fin de Máster\par}
    \vspace{0.5cm}
    {\large Máster Universitario en Ingeniería de la Energía \par}
    \vspace{2cm}
    {\Large Luis D. Aranda Sánchez\par}
    \vfill
    Director: Javier Rodríguez Martín
    \vfill
    {\large Septiembre 6, 2024\par}
\end{titlepage}

% Resumen (máximo de 5 páginas, incluyendo al final Palabras clave)
\clearpage
\pagestyle{simple}
% \newpage
\chapter*{Resumen}
\addcontentsline{toc}{chapter}{Resumen}
\input{capitulos/resumen/main.tex}

% Índice (paginado)
\clearpage
\pagestyle{simple}
% \newpage
\tableofcontents

% Introducción (donde se incluya los antecedentes y justificación)
\clearpage
\pagestyle{myfancy}
\newpage
\chapter{Introducción}
\input{capitulos/introduccion/main.tex}

% Objetivos
\chapter{Objetivos}
\input{capitulos/objetivos/main.tex}

% Metodología
\chapter{Metodología}
\input{capitulos/metodologia/main.tex}

% Resultados y discusión (incluyendo la valoración de impactos y de aspectos de responsabilidad legal, ética y profesional relacionados con el trabajo)
\chapter{Resultados y Discusión}
\input{capitulos/resultados_discusion/main.tex}

% Conclusiones
\chapter{Conclusiones}
\input{capitulos/conclusiones/main.tex}

% Planificación temporal y presupuesto
\chapter{Planificación Temporal y Presupuesto}
\input{capitulos/planificacion_presupuesto/main.tex}

% Bibliografía
\newpage
\addcontentsline{toc}{chapter}{Bibliografía}
\printbibliography

\end{document}


% Bibliografía
\newpage
\addcontentsline{toc}{chapter}{Bibliografía}
\printbibliography

\end{document}


% Bibliografía
\newpage
\addcontentsline{toc}{chapter}{Bibliografía}
\printbibliography

\end{document}


% Bibliografía
\cleardoublepage
\addcontentsline{toc}{chapter}{Bibliografía}
\printbibliography

\end{document}
