\documentclass[a4paper,11pt,twoside]{report}
\usepackage[left=25mm,right=25mm,top=25mm,bottom=25mm,includehead,includefoot,headsep=15mm,footskip=15mm]{geometry}
\usepackage{graphicx}
\usepackage{fancyhdr}
\usepackage{titlesec}
\usepackage[spanish]{babel}
\usepackage[utf8]{inputenc}
\usepackage{amsmath}
\usepackage{setspace}
\usepackage{url}
\usepackage{svg}
\usepackage{booktabs}
\usepackage{hyperref}
\usepackage[backend=biber,style=numeric]{biblatex}
\addbibresource{references.bib}
\hypersetup{
    colorlinks=true,
    linkcolor=blue,      % color of internal links (sections, etc.)
    urlcolor=blue,       % color of external links
    pdftitle={Optimización energética de sistema híbrido con bomba de calor, suelo radiante, fotovoltaica y almacenamiento para vivienda},    % title
    pdfauthor={Luis D. Aranda Sánchez},     % author
    pdfkeywords={palabra1, palabra2, código1, etc.} % list of keywords
}

% Code colorscheme
\usepackage{minted}
\setminted{
    frame=lines,
    linenos,
    fontsize=\small,
    numbersep=5pt,
    tabsize=4,
    breaklines,
    breakanywhere,
    style=friendly  % Cambia este valor por el esquema de colores que prefieras
}

% Font change to Arial
\usepackage{helvet}
\renewcommand{\familydefault}{\sfdefault}

% Chapter titles in uppercase and larger font
\titleformat{\chapter}[hang]{\large\bfseries}{\thechapter.}{1em}{\MakeUppercase}
\titleformat{\section}[hang]{\bfseries}{\thesection.}{1em}{}
\titleformat{\subsection}[hang]{\bfseries}{\thesubsection.}{1em}{}

% Fancyhdr setup
\setlength{\headheight}{14.30174pt} % Adjust to recommended value, slightly larger for safety
\fancyhf{} % Clear all headers and footers
\fancyhead[LE]{\nouppercase{\leftmark}}
\fancyhead[RO]{Optimización energética para vivienda}
\fancyfoot[LE]{\thepage}
\fancyfoot[RE]{Escuela Técnica Superior de Ingenieros Industriales (UPM)}
\fancyfoot[LO]{Luis D. Aranda Sánchez}
\fancyfoot[RO]{\thepage}
\renewcommand{\headrulewidth}{0.4pt}
\renewcommand{\footrulewidth}{0.4pt}

\fancypagestyle{myfancy}{
    \fancyhf{} % Clear all headers and footers
    \fancyhead[LE]{\nouppercase{\leftmark}}
    \fancyhead[RO]{Optimización energética para vivienda}
    \fancyfoot[LE]{\thepage}
    \fancyfoot[RE]{Escuela Técnica Superior de Ingenieros Industriales (UPM)}
    \fancyfoot[LO]{Luis D. Aranda Sánchez}
    \fancyfoot[RO]{\thepage}
    \renewcommand{\headrulewidth}{0.4pt}
    \renewcommand{\footrulewidth}{0.4pt}
}

% Redefine the plain page style (start of chapters)
\fancypagestyle{plain}{
    \fancyhf{}
    \fancyfoot[LE]{\thepage}
    \fancyfoot[RE]{Escuela Técnica Superior de Ingenieros Industriales (UPM)}
    \fancyfoot[LO]{Luis D. Aranda Sánchez}
    \fancyfoot[RO]{\thepage}
    \renewcommand{\headrulewidth}{0pt}
    \renewcommand{\footrulewidth}{0.4pt}
}
\title{A short example}

% Line spacing
\setstretch{1.2}

% Ajusta el espacio superior de las notas al pie
\setlength{\skip\footins}{1.5em}

% Document starts here
\begin{document}

% Portada
\begin{titlepage}
	\centering
	{\scshape\LARGE Universidad Politécnica de Madrid \par}
	\vspace{1cm}
	{\scshape\Large Escuela Técnica Superior de Ingenieros Industriales\par}
	\vspace{1.5cm}
	{\huge\bfseries Optimización energética de sistema híbrido con bomba de calor, suelo radiante, fotovoltaica y almacenamiento para vivienda \par}
	\vspace{1.5cm}
	{\Large\bfseries Trabajo de Fin de Máster\par}
	\vspace{0.5cm}
	{\large Máster Universitario en Ingeniería de la Energía \par}
	\vspace{2cm}
	{\Large Luis D. Aranda Sánchez\par}
	\vfill
	Director: Javier Rodríguez Martín
	\vfill
	{\large Septiembre 6, 2024\par}
\end{titlepage}

% Resumen (máximo de 5 páginas, incluyendo al final Palabras clave)
\cleardoublepage
\chapter*{Resumen}
\addcontentsline{toc}{chapter}{Resumen}
\section{Simulacion del sistema termico}

\section{Optimizacion control sistema termico con derivadas adjuntas}

\section{Optimizacion control sistema termico con SAND}

aqui vemos que SAND es superior a MDF, en caso de ser viable su uso,
y por tanto a partir de ahora estudiamos solo el uso de SAND

\section{Ecuaciones sistema electrico}

\section{Optimizacion SAND del control del sistema completo}
\section{Optimizacion SAND del control y dimensionamiento del sistema completo}
\subsection{Autoconsumo con compensación simplificada}
- 10 dias
- 30 dias
- 365 dias

- dimensionamiento con media ponderada
no es realista suponer que tenemos conocimiento perfecto de precios futuros.
realmente tendriamos que optimizar nuestro sistema con nuestra estimacion
de precios, como con una red neural, aunque en este trabajo no se cubre este caso.
\subsection{Sistema off-grid}
\subsection{Libre venta de electricidad a precio de mercado}

\section{Valoracion de impactos}


% Índice (paginado)
\cleardoublepage
\pagestyle{myfancy}
\tableofcontents

% Introducción
\cleardoublepage
\chapter{Introducción}
\section{Simulacion del sistema termico}

\section{Optimizacion control sistema termico con derivadas adjuntas}

\section{Optimizacion control sistema termico con SAND}

aqui vemos que SAND es superior a MDF, en caso de ser viable su uso,
y por tanto a partir de ahora estudiamos solo el uso de SAND

\section{Ecuaciones sistema electrico}

\section{Optimizacion SAND del control del sistema completo}
\section{Optimizacion SAND del control y dimensionamiento del sistema completo}
\subsection{Autoconsumo con compensación simplificada}
- 10 dias
- 30 dias
- 365 dias

- dimensionamiento con media ponderada
no es realista suponer que tenemos conocimiento perfecto de precios futuros.
realmente tendriamos que optimizar nuestro sistema con nuestra estimacion
de precios, como con una red neural, aunque en este trabajo no se cubre este caso.
\subsection{Sistema off-grid}
\subsection{Libre venta de electricidad a precio de mercado}

\section{Valoracion de impactos}


% Objetivos
\cleardoublepage
\chapter{Objetivos}
\section{Simulacion del sistema termico}

\section{Optimizacion control sistema termico con derivadas adjuntas}

\section{Optimizacion control sistema termico con SAND}

aqui vemos que SAND es superior a MDF, en caso de ser viable su uso,
y por tanto a partir de ahora estudiamos solo el uso de SAND

\section{Ecuaciones sistema electrico}

\section{Optimizacion SAND del control del sistema completo}
\section{Optimizacion SAND del control y dimensionamiento del sistema completo}
\subsection{Autoconsumo con compensación simplificada}
- 10 dias
- 30 dias
- 365 dias

- dimensionamiento con media ponderada
no es realista suponer que tenemos conocimiento perfecto de precios futuros.
realmente tendriamos que optimizar nuestro sistema con nuestra estimacion
de precios, como con una red neural, aunque en este trabajo no se cubre este caso.
\subsection{Sistema off-grid}
\subsection{Libre venta de electricidad a precio de mercado}

\section{Valoracion de impactos}


% Metodología
\cleardoublepage
\chapter{Metodología}
\section{Simulacion del sistema termico}

\section{Optimizacion control sistema termico con derivadas adjuntas}

\section{Optimizacion control sistema termico con SAND}

aqui vemos que SAND es superior a MDF, en caso de ser viable su uso,
y por tanto a partir de ahora estudiamos solo el uso de SAND

\section{Ecuaciones sistema electrico}

\section{Optimizacion SAND del control del sistema completo}
\section{Optimizacion SAND del control y dimensionamiento del sistema completo}
\subsection{Autoconsumo con compensación simplificada}
- 10 dias
- 30 dias
- 365 dias

- dimensionamiento con media ponderada
no es realista suponer que tenemos conocimiento perfecto de precios futuros.
realmente tendriamos que optimizar nuestro sistema con nuestra estimacion
de precios, como con una red neural, aunque en este trabajo no se cubre este caso.
\subsection{Sistema off-grid}
\subsection{Libre venta de electricidad a precio de mercado}

\section{Valoracion de impactos}


% Adquisición de datos
\cleardoublepage
\chapter{Adquisición de datos}
\section{Simulacion del sistema termico}

\section{Optimizacion control sistema termico con derivadas adjuntas}

\section{Optimizacion control sistema termico con SAND}

aqui vemos que SAND es superior a MDF, en caso de ser viable su uso,
y por tanto a partir de ahora estudiamos solo el uso de SAND

\section{Ecuaciones sistema electrico}

\section{Optimizacion SAND del control del sistema completo}
\section{Optimizacion SAND del control y dimensionamiento del sistema completo}
\subsection{Autoconsumo con compensación simplificada}
- 10 dias
- 30 dias
- 365 dias

- dimensionamiento con media ponderada
no es realista suponer que tenemos conocimiento perfecto de precios futuros.
realmente tendriamos que optimizar nuestro sistema con nuestra estimacion
de precios, como con una red neural, aunque en este trabajo no se cubre este caso.
\subsection{Sistema off-grid}
\subsection{Libre venta de electricidad a precio de mercado}

\section{Valoracion de impactos}


% Resultados y discusión (incluyendo la valoración de impactos y de aspectos de responsabilidad legal, ética y profesional relacionados con el trabajo)
\cleardoublepage
\chapter{Resultados y Discusión}
\section{Simulacion del sistema termico}

\section{Optimizacion control sistema termico con derivadas adjuntas}

\section{Optimizacion control sistema termico con SAND}

aqui vemos que SAND es superior a MDF, en caso de ser viable su uso,
y por tanto a partir de ahora estudiamos solo el uso de SAND

\section{Ecuaciones sistema electrico}

\section{Optimizacion SAND del control del sistema completo}
\section{Optimizacion SAND del control y dimensionamiento del sistema completo}
\subsection{Autoconsumo con compensación simplificada}
- 10 dias
- 30 dias
- 365 dias

- dimensionamiento con media ponderada
no es realista suponer que tenemos conocimiento perfecto de precios futuros.
realmente tendriamos que optimizar nuestro sistema con nuestra estimacion
de precios, como con una red neural, aunque en este trabajo no se cubre este caso.
\subsection{Sistema off-grid}
\subsection{Libre venta de electricidad a precio de mercado}

\section{Valoracion de impactos}


% Conclusiones
\cleardoublepage
\chapter{Conclusiones}
\section{Simulacion del sistema termico}

\section{Optimizacion control sistema termico con derivadas adjuntas}

\section{Optimizacion control sistema termico con SAND}

aqui vemos que SAND es superior a MDF, en caso de ser viable su uso,
y por tanto a partir de ahora estudiamos solo el uso de SAND

\section{Ecuaciones sistema electrico}

\section{Optimizacion SAND del control del sistema completo}
\section{Optimizacion SAND del control y dimensionamiento del sistema completo}
\subsection{Autoconsumo con compensación simplificada}
- 10 dias
- 30 dias
- 365 dias

- dimensionamiento con media ponderada
no es realista suponer que tenemos conocimiento perfecto de precios futuros.
realmente tendriamos que optimizar nuestro sistema con nuestra estimacion
de precios, como con una red neural, aunque en este trabajo no se cubre este caso.
\subsection{Sistema off-grid}
\subsection{Libre venta de electricidad a precio de mercado}

\section{Valoracion de impactos}


% Planificación temporal y presupuesto
\cleardoublepage
\chapter{Planificación Temporal y Presupuesto}
\section{Simulacion del sistema termico}

\section{Optimizacion control sistema termico con derivadas adjuntas}

\section{Optimizacion control sistema termico con SAND}

aqui vemos que SAND es superior a MDF, en caso de ser viable su uso,
y por tanto a partir de ahora estudiamos solo el uso de SAND

\section{Ecuaciones sistema electrico}

\section{Optimizacion SAND del control del sistema completo}
\section{Optimizacion SAND del control y dimensionamiento del sistema completo}
\subsection{Autoconsumo con compensación simplificada}
- 10 dias
- 30 dias
- 365 dias

- dimensionamiento con media ponderada
no es realista suponer que tenemos conocimiento perfecto de precios futuros.
realmente tendriamos que optimizar nuestro sistema con nuestra estimacion
de precios, como con una red neural, aunque en este trabajo no se cubre este caso.
\subsection{Sistema off-grid}
\subsection{Libre venta de electricidad a precio de mercado}

\section{Valoracion de impactos}


% Bibliografía
\cleardoublepage
\addcontentsline{toc}{chapter}{Bibliografía}
\printbibliography

\end{document}
